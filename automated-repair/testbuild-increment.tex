% Options for packages loaded elsewhere
\PassOptionsToPackage{unicode}{hyperref}
\PassOptionsToPackage{hyphens}{url}
%
\documentclass[
]{article}
\usepackage{lmodern}
\usepackage{amssymb,amsmath}
\usepackage{ifxetex,ifluatex}
\ifnum 0\ifxetex 1\fi\ifluatex 1\fi=0 % if pdftex
  \usepackage[T1]{fontenc}
  \usepackage[utf8]{inputenc}
  \usepackage{textcomp} % provide euro and other symbols
\else % if luatex or xetex
  \usepackage{unicode-math}
  \defaultfontfeatures{Scale=MatchLowercase}
  \defaultfontfeatures[\rmfamily]{Ligatures=TeX,Scale=1}
\fi
% Use upquote if available, for straight quotes in verbatim environments
\IfFileExists{upquote.sty}{\usepackage{upquote}}{}
\IfFileExists{microtype.sty}{% use microtype if available
  \usepackage[]{microtype}
  \UseMicrotypeSet[protrusion]{basicmath} % disable protrusion for tt fonts
}{}
\makeatletter
\@ifundefined{KOMAClassName}{% if non-KOMA class
  \IfFileExists{parskip.sty}{%
    \usepackage{parskip}
  }{% else
    \setlength{\parindent}{0pt}
    \setlength{\parskip}{6pt plus 2pt minus 1pt}}
}{% if KOMA class
  \KOMAoptions{parskip=half}}
\makeatother
\usepackage{xcolor}
\IfFileExists{xurl.sty}{\usepackage{xurl}}{} % add URL line breaks if available
\IfFileExists{bookmark.sty}{\usepackage{bookmark}}{\usepackage{hyperref}}
\hypersetup{
  hidelinks,
  pdfcreator={LaTeX via pandoc}}
\urlstyle{same} % disable monospaced font for URLs
\setlength{\emergencystretch}{3em} % prevent overfull lines
\providecommand{\tightlist}{%
  \setlength{\itemsep}{0pt}\setlength{\parskip}{0pt}}
\setcounter{secnumdepth}{-\maxdimen} % remove section numbering

\date{}

\begin{document}

14th Amendment Course

Table of Contents

\hypertarget{procedural-due-process}{%
\section{Procedural Due Process}\label{procedural-due-process}}

\hypertarget{goldberg-v.-kelly}{%
\subsubsection{Goldberg v. Kelly}\label{goldberg-v.-kelly}}

397 U.S. 254 (1970)

\textbf{MR. JUSTICE BRENNAN delivered the opinion of the Court.} The
question for decision is whether a State that terminates public
assistance payments to a particular recipient without affording him the
opportunity for an evidentiary hearing prior to termination denies the
recipient procedural due process in violation of the Due Process Clause
of the Fourteenth Amendment.

This action was brought in the District Court for the Southern District
of New York by residents of New York City receiving financial aid under
the federally assisted program of Aid to Families with Dependent
Children (AFDC) or under New York State's general Home Relief program
Their complaint alleged that the New York State and New York City
officials administering these programs terminated, or were about to
terminate, such aid without prior notice and hearing, thereby denying
them due process of law At the time the suits were filed there was no
requirement of prior notice or hearing of any kind before termination of
financial aid. However, the State and city adopted procedures for notice
and hearing after the suits were brought, and the plaintiffs, appellees
here, then challenged the constitutional adequacy of those procedures.

The State Commissioner of Social Services amended the State Department
of Social Services' Official Regulations to require that local social
services officials proposing to discontinue or suspend a recipient's
financial aid do so according to a procedure that conforms to either
subdivision (a) or subdivision (b) of § 351 of the regulations as
amended The City of New York elected to promulgate a local procedure
according to subdivision (b). That subdivision, so far as here
pertinent, provides that the local procedure must include the giving of
notice to the recipient of the reasons for a proposed discontinuance or
suspension at least seven days prior to its effective date, with notice
also that upon request the recipient may have the proposal reviewed by a
local welfare official holding a position superior to that of the
supervisor who approved the proposed discontinuance or suspension, and,
further, that the recipient may submit, for purposes of the review, a
written statement to demonstrate why his grant should not be
discontinued or suspended. The decision by the reviewing official
whether to discontinue or suspend aid must be made expeditiously, with
written notice of the decision to the recipient. The section further
expressly provides that ``{[}a{]}ssistance shall not be discontinued or
suspended prior to the date such notice of decision is sent to the
recipient and his representative, if any, or prior to the proposed
effective date of discontinuance or suspension, whichever occurs
later.''

Pursuant to subdivision (b), the New York City Department of Social
Services promulgated Procedure No.~68-18. A caseworker who has doubts
about the recipient's continued eligibility must first discuss them with
the recipient. If the caseworker concludes that the recipient is no
longer eligible, he recommends termination of aid to a unit supervisor.
If the latter concurs, he sends the recipient a letter stating the
reasons for proposing to terminate aid and notifying him that within
seven days he may request that a higher official review the record, and
may support the request with a written statement prepared personally or
with the aid of an attorney or other person. If the reviewing official
affirms the determination of ineligibility, aid is stopped immediately
and the recipient is informed by letter of the reasons for the action.
Appellees' challenge to this procedure emphasizes the absence of any
provisions for the personal appearance of the recipient before the
reviewing official, for oral presentation of evidence, and for
confrontation and cross-examination of adverse witnesses However, the
letter does inform the recipient that he may request a post-termination
``fair hearing.''{[}5{]} This is a proceeding before an independent
state hearing officer at which the recipient may appear personally,
offer oral evidence, confront and cross-examine the witnesses against
him, and have a record made of the hearing. If the recipient prevails at
the ``fair hearing'' he is paid all funds erroneously withheld. A
recipient whose aid is not restored by a ``fair hearing'' decision may
have judicial review.

The constitutional issue to be decided, therefore, is the narrow one
whether the Due Process Clause requires that the recipient be afforded
an evidentiary hearing before the termination of benefits The District
Court held that only a pre-termination evidentiary hearing would satisfy
the constitutional command, and rejected the argument of the state and
city officials that the combination of the post-termination ``fair
hearing'' with the informal pre-termination review disposed of all due
process claims. The court said: ``While post-termination review is
relevant, there is one overpowering fact which controls here. By
hypothesis, a welfare recipient is destitute, without funds or assets. .
Suffice it to say that to cut off a welfare recipient in the face of .
.'brutal need' without a prior hearing of some sort is unconscionable,
unless overwhelming considerations justify it.'' Kelly v. Wyman, 294 F.
Supp. 893, 899, 900 (1968). The court rejected the argument that the
need to protect the public's tax revenues supplied the requisite
``overwhelming consideration.'' ``Against the justified desire to
protect public funds must be weighed the individual's over-powering need
in this unique situation not to be wrongfully deprived of assistance . .
While the problem of additional expense must be kept in mind, it does
not justify denying a hearing meeting the ordinary standards of due
process. Under all the circumstances, we hold that due process requires
an adequate hearing before termination of welfare benefits, and the fact
that there is a later constitutionally fair proceeding does not alter
the result.'' Although state officials were party defendants in the
action, only the Commissioner of Social Services of the City of New York
appealed. We noted probable jurisdiction, to decide important issues
that have been the subject of disagreement in principle between the
three-judge court in the present case and that convened in Wheeler v.
Montgomery, No.~14, post, p.~280, also decided today. We affirm.

Appellant does not contend that procedural due process is not applicable
to the termination of welfare benefits. Such benefits are a matter of
statutory entitlement for persons qualified to receive them Their
termination involves state action that adjudicates important rights. The
constitutional challenge cannot be answered by an argument that public
assistance benefits are ``a'privilege' and not a'right.''' Shapiro v.
Thompsonn. 6 (1969). Relevant constitutional restraints apply as much to
the withdrawal of public assistance benefits as to disqualification for
unemployment compensation, Sherbert v. Verner; or to denial of a tax
exemption, Speiser v. Randall; or to discharge from public employment,
Slochower v. Board of Higher Education The extent to which procedural
due process must be afforded the recipient is influenced by the extent
to which he may be ``condemned to suffer grievous loss,'' Joint
Anti-Fascist Refugee Committee v. McGrath (Frankfurter, J., concurring),
and depends upon whether the recipient's interest in avoiding that loss
outweighs the governmental interest in summary adjudication.
Accordingly, as we said in Cafeteria \& Restaurant Workers Union v.
McElroy, ``consideration of what procedures due process may require
under any given set of circumstances must begin with a determination of
the precise nature of the government function involved as well as of the
private interest that has been affected by governmental action.''

It is true, of course, that some governmental benefits may be
administratively terminated without affording the recipient a
pre-termination evidentiary hearing But we agree with the District Court
that when welfare is discontinued, only a pre-termination evidentiary
hearing provides the recipient with procedural due process. Cf. Sniadach
v. Family Finance Corp.. For qualified recipients, welfare provides the
means to obtain essential food, clothing, housing, and medical care Cf.
Nash v. Florida Industrial Commission. Thus the crucial factor in this
context---a factor not present in the case of the blacklisted government
contractor, the discharged government employee, the taxpayer denied a
tax exemption, or virtually anyone else whose governmental entitlements
are ended---is that termination of aid pending resolution of a
controversy over eligibility may deprive an eligible recipient of the
very means by which to live while he waits. Since he lacks independent
resources, his situation becomes immediately desperate. His need to
concentrate upon finding the means for daily subsistence, in turn,
adversely affects his ability to seek redress from the welfare
bureaucracy.

Moreover, important governmental interests are promoted by affording
recipients a pre-termination evidentiary hearing. From its founding the
Nation's basic commitment has been to foster the dignity and well-being
of all persons within its borders. We have come to recognize that forces
not within the control of the poor contribute to their poverty This
perception, against the background of our traditions, has significantly
influenced the development of the contemporary public assistance system.
Welfare, by meeting the basic demands of subsistence, can help bring
within the reach of the poor the same opportunities that are available
to others to participate meaningfully in the life of the community. At
the same time, welfare guards against the societal malaise that may flow
from a widespread sense of unjustified frustration and insecurity.
Public assistance, then, is not mere charity, but a means to ``promote
the general Welfare, and secure the Blessings of Liberty to ourselves
and our Posterity.'' The same governmental interests that counsel the
provision of welfare, counsel as well its uninterrupted provision to
those eligible to receive it; pre-termination evidentiary hearings are
indispensable to that end.

Appellant does not challenge the force of these considerations but
argues that they are outweighed by countervailing governmental interests
in conserving fiscal and administrative resources. These interests, the
argument goes, justify the delay of any evidentiary hearing until after
discontinuance of the grants. Summary adjudication protects the public
fisc by stopping payments promptly upon discovery of reason to believe
that a recipient is no longer eligible. Since most terminations are
accepted without challenge, summary adjudication also conserves both the
fisc and administrative time and energy by reducing the number of
evidentiary hearings actually held.

We agree with the District Court, however, that these governmental
interests are not overriding in the welfare context. The requirement of
a prior hearing doubtless involves some greater expense, and the
benefits paid to ineligible recipients pending decision at the hearing
probably cannot be recouped, since these recipients are likely to be
judgment-proof. But the State is not without weapons to minimize these
increased costs. Much of the drain on fiscal and administrative
resources can be reduced by developing procedures for prompt
pre-termination hearings and by skillful use of personnel and
facilities. Indeed, the very provision for a post-termination
evidentiary hearing in New York's Home Relief program is itself cogent
evidence that the State recognizes the primacy of the public interest in
correct eligibility determinations and therefore in the provision of
procedural safeguards. Thus, the interest of the eligible recipient in
uninterrupted receipt of public assistance, coupled with the State's
interest that his payments not be erroneously terminated, clearly
outweighs the State's competing concern to prevent any increase in its
fiscal and administrative burdens. As the District Court correctly
concluded, ``{[}t{]}he stakes are simply too high for the welfare
recipient, and the possibility for honest error or irritable misjudgment
too great, to allow termination of aid without giving the recipient a
chance, if he so desires, to be fully informed of the case against him
so that he may contest its basis and produce evidence in rebuttal.'' 294
F. Supp..

We also agree with the District Court, however, that the pre-termination
hearing need not take the form of a judicial or quasi-judicial trial. We
bear in mind that the statutory ``fair hearing'' will provide the
recipient with a full administrative review Accordingly, the
pre-termination hearing has one function only: to produce an initial
determination of the validity of the welfare department's grounds for
discontinuance of payments in order to protect a recipient against an
erroneous termination of his benefits. Cf. Sniadach v. Family Finance
Corp.~(HARLAN, J., concurring). Thus, a complete record and a
comprehensive opinion, which would serve primarily to facilitate
judicial review and to guide future decisions, need not be provided at
the pre-termination stage. We recognize, too, that both welfare
authorities and recipients have an interest in relatively speedy
resolution of questions of eligibility, that they are used to dealing
with one another informally, and that some welfare departments have very
burdensome caseloads. These considerations justify the limitation of the
pre-termination hearing to minimum procedural safeguards, adapted to the
particular characteristics of welfare recipients, and to the limited
nature of the controversies to be resolved. We wish to add that we, no
less than the dissenters, recognize the importance of not imposing upon
the States or the Federal Government in this developing field of law any
procedural requirements beyond those demanded by rudimentary due
process.

``The fundamental requisite of due process of law is the opportunity to
be heard.'' Grannis v. Ordean. The hearing must be ``at a meaningful
time and in a meaningful manner.'' Armstrong v. Manzo. In the present
context these principles require that a recipient have timely and
adequate notice detailing the reasons for a proposed termination, and an
effective opportunity to defend by confronting any adverse witnesses and
by presenting his own arguments and evidence orally. These rights are
important in cases such as those before us, where recipients have
challenged proposed terminations as resting on incorrect or misleading
factual premises or on misapplication of rules or policies to the facts
of particular cases.

We are not prepared to say that the seven-day notice currently provided
by New York City is constitutionally insufficient per se, although there
may be cases where fairness would require that a longer time be given.
Nor do we see any constitutional deficiency in the content or form of
the notice. New York employs both a letter and a personal conference
with a caseworker to inform a recipient of the precise questions raised
about his continued eligibility. Evidently the recipient is told the
legal and factual bases for the Department's doubts. This combination is
probably the most effective method of communicating with recipients.

The city's procedures presently do not permit recipients to appear
personally with or without counsel before the official who finally
determines continued eligibility. Thus a recipient is not permitted to
present evidence to that official orally, or to confront or
cross-examine adverse witnesses. These omissions are fatal to the
constitutional adequacy of the procedures.

The opportunity to be heard must be tailored to the capacities and
circumstances of those who are to be heard It is not enough that a
welfare recipient may present his position to the decision maker in
writing or secondhand through his caseworker. Written submissions are an
unrealistic option for most recipients, who lack the educational
attainment necessary to write effectively and who cannot obtain
professional assistance. Moreover, written submissions do not afford the
flexibility of oral presentations; they do not permit the recipient to
mold his argument to the issues the decision maker appears to regard as
important. Particularly where credibility and veracity are at issue, as
they must be in many termination proceedings, written submissions are a
wholly unsatisfactory basis for decision. The secondhand presentation to
the decisionmaker by the caseworker has its own deficiencies; since the
caseworker usually gathers the facts upon which the charge of
ineligibility rests, the presentation of the recipient's side of the
controversy cannot safely be left to him. Therefore a recipient must be
allowed to state his position orally. Informal procedures will suffice;
in this context due process does not require a particular order of proof
or mode of offering evidence.

In almost every setting where important decisions turn on questions of
fact, due process requires an opportunity to confront and cross-examine
adverse witnesses. E. g., ICC v. Louisville \& N. R. Co.~(1913); Willner
v. Committee on Character \& Fitness (1963). What we said in Greene v.
McElroy (1959), is particularly pertinent here:

``Certain principles have remained relatively immutable in our
jurisprudence. One of these is that where governmental action seriously
injures an individual, and the reasonableness of the action depends on
fact findings, the evidence used to prove the Government's case must be
disclosed to the individual so that he has an opportunity to show that
it is untrue. While this is important in the case of documentary
evidence, it is even more important where the evidence consists of the
testimony of individuals whose memory might be faulty or who, in fact,
might be perjurers or persons motivated by malice, vindictiveness,
intolerance, prejudice, or jealousy. We have formalized these
protections in the requirements of confrontation and cross-examination.
They have ancient roots. They find expression in the Sixth Amendment . .
This Court has been zealous to protect these rights from erosion. It has
spoken out not only in criminal cases, . but also in all types of cases
where administrative. . actions were under scrutiny.''

Welfare recipients must therefore be given an opportunity to confront
and cross-examine the witnesses relied on by the department.

``The right to be heard would be, in many cases, of little avail if it
did not comprehend the right to be heard by counsel.'' Powell v. Alabama
(1932). We do not say that counsel must be provided at the
pre-termination hearing, but only that the recipient must be allowed to
retain an attorney if he so desires. Counsel can help delineate the
issues, present the factual contentions in an orderly manner, conduct
cross-examination, and generally safeguard the interests of the
recipient. We do not anticipate that this assistance will unduly prolong
or otherwise encumber the hearing. Evidently HEW has reached the same
conclusion. See 45 CFR § 205 (1969).

Finally, the decisionmaker's conclusion as to a recipient's eligibility
must rest solely on the legal rules and evidence adduced at the hearing.
Ohio Bell Tel. Co.~v. PUC; United States v. Abilene \& S. R. Co.~(1924).
To demonstrate compliance with this elementary requirement, the decision
maker should state the reasons for his determination and indicate the
evidence he relied on, cf.~Wichita R. \& Light Co.~v. PUC (1922), though
his statement need not amount to a full opinion or even formal findings
of fact and conclusions of law. And, of course, an impartial decision
maker is essential. Cf. In re Murchison; Wong Yang Sung v. McGrath
(1950). We agree with the District Court that prior involvement in some
aspects of a case will not necessarily bar a welfare official from
acting as a decision maker. He should not, however, have participated in
making the determination under review.

Affirmed.

\emph{From the footnotes to majority opinion:}

It may be realistic today to regard welfare entitlements as more like
``property'' than a ``gratuity.'' Much of the existing wealth in this
country takes the form of rights that do not fall within traditional
common-law concepts of property. It has been aptly noted that

\begin{quote}
{[}s{]}ociety today is built around entitlement. The automobile dealer
has his franchise, the doctor and lawyer their professional licenses,
the worker his union membership, contract, and pension rights, the
executive his contract and stock options; all are devices to aid
security and independence. Many of the most important of these
entitlements now flow from government: subsidies to farmers and
businessmen, routes for airlines and channels for television stations;
long term contracts for defense, space, and education; social security
pensions for individuals. Such sources of security, whether private or
public, are no longer regarded as luxuries or gratuities; to the
recipients they are essentials, fully deserved, and in no sense a form
of charity. It is only the poor whose entitlements, although recognized
by public policy, have not been effectively enforced." Reich, Individual
Rights and Social Welfare: The Emerging Legal Issues, 74 Yale L. J.
1245, 1255 (1965). See also Reich, The New Property, 73 Yale L. J. 733
(1964). See also Goldsmith v. United States Board of Tax Appeals (right
of a certified public accountant to practice before the Board of Tax
Appeals); Hornsby v. Allen, C. A. 5th Cir. 1964) (right to obtain a
retail liquor store license); Dixon v. Alabama State Board of Education,
C. A. 5th Cir.), cert. denied (right to attend a public college).
\end{quote}

\textbf{MR. JUSTICE BLACK, dissenting.}

In the last half century the United States, along with many, perhaps
most, other nations of the world, has moved far toward becoming a
welfare state, that is, a nation that for one reason or another taxes
its most affluent people to help support, feed, clothe, and shelter its
less fortunate citizens. The result is that today more than nine million
men, women, and children in the United States receive some kind of state
or federally financed public assistance in the form of allowances or
gratuities, generally paid them periodically, usually by the week,
month, or quarter Since these gratuities are paid on the basis of need,
the list of recipients is not static, and some people go off the lists
and others are added from time to time. These ever-changing lists put a
constant administrative burden on government and it certainly could not
have reasonably anticipated that this burden would include the
additional procedural expense imposed by the Court today.

The dilemma of the ever-increasing poor in the midst of constantly
growing affluence presses upon us and must inevitably be met within the
framework of our democratic constitutional government, if our system is
to survive as such. It was largely to escape just such pressing economic
problems and attendant government repression that people from Europe,
Asia, and other areas settled this country and formed our Nation. Many
of those settlers had personally suffered from persecutions of various
kinds and wanted to get away from governments that had unrestrained
powers to make life miserable for their citizens. It was for this
reason, or so I believe, that on reaching these new lands the early
settlers undertook to curb their governments by confining their powers
within written boundaries, which eventually became written constitutions
They wrote their basic charters as nearly as men's collective wisdom
could do so as to proclaim to their people and their officials an
emphatic command that: ``Thus far and no farther shall you go; and where
we neither delegate powers to you, nor prohibit your exercise of them,
we the people are left free.''

Representatives of the people of the Thirteen Original Colonies spent
long, hot months in the summer of 1787 in Philadelphia, Pennsylvania,
creating a government of limited powers. They divided it into three
departments ---Legislative, Judicial, and Executive. The Judicial
Department was to have no part whatever in making any laws. In fact
proposals looking to vesting some power in the Judiciary to take part in
the legislative process and veto laws were offered, considered, and
rejected by the Constitutional Convention In my judgment there is not
one word, phrase, or sentence from the beginning to the end of the
Constitution from which it can be inferred that judges were granted any
such legislative power. True, Marbury v. Madison, 1 Cranch 137 (1803),
held, and properly, I think, that courts must be the final interpreters
of the Constitution, and I recognize that the holding can provide an
opportunity to slide imperceptibly into constitutional amendment and law
making. But when federal judges use this judicial power for legislative
purposes, I think they wander out of their field of vested powers and
transgress into the area constitutionally assigned to the Congress and
the people. That is precisely what I believe the Court is doing in this
case. Hence my dissent.

The more than a million names on the relief rolls in New York and the
more than nine million names on the rolls of all the 50 States were not
put there at random. The names are there because state welfare officials
believed that those people were eligible for assistance. Probably in the
officials' haste to make out the lists many names were put there
erroneously in order to alleviate immediate suffering, and undoubtedly
some people are drawing relief who are not entitled under the law to do
so. Doubtless some draw relief checks from time to time who know they
are not eligible, either because they are not actually in need or for
some other reason. Many of those who thus draw undeserved gratuities are
without sufficient property to enable the government to collect back
from them any money they wrongfully receive. But the Court today holds
that it would violate the Due Process Clause of the Fourteenth Amendment
to stop paying those people weekly or monthly allowances unless the
government first affords them a full ``evidentiary hearing'' even though
welfare officials are persuaded that the recipients are not rightfully
entitled to receive a penny under the law. In other words, although some
recipients might be on the lists for payment wholly because of
deliberate fraud on their part, the Court holds that the government is
helpless and must continue, until after an evidentiary hearing, to pay
money that it does not owe, never has owed, and never could owe. I do
not believe there is any provision in our Constitution that should thus
paralyze the government's efforts to protect itself against making
payments to people who are not entitled to them.

Particularly do I not think that the Fourteenth Amendment should be
given such an unnecessarily broad construction. That Amendment came into
being primarily to protect Negroes from discrimination, and while some
of its language can and does protect others, all know that the chief
purpose behind it was to protect ex-slaves. Cf. Adamson v. California,
and n.~5 (1947) (dissenting opinion). The Court, however, relies upon
the Fourteenth Amendment and in effect says that failure of the
government to pay a promised charitable installment to an individual
deprives that individual of his own property, in violation of the Due
Process Clause of the Fourteenth Amendment. It somewhat strains
credulity to say that the government's promise of charity to an
individual is property belonging to that individual when the government
denies that the individual is honestly entitled to receive such a
payment.

I would have little, if any, objection to the majority's decision in
this case if it were written as the report of the House Committee on
Education and Labor, but as an opinion ostensibly resting on the
language of the Constitution I find it woefully deficient. Once the
verbiage is pared away it is obvious that this Court today adopts the
views of the District Court ``that to cut off a welfare recipient in the
face of . .'brutal need' without a prior hearing of some sort is
unconscionable,'' and therefore, says the Court, unconstitutional. The
majority reaches this result by a process of weighing ``the recipient's
interest in avoiding'' the termination of welfare benefits against ``the
governmental interest in summary adjudication.'' Today's balancing act
requires a ``pre-termination evidentiary hearing,'' yet there is nothing
that indicates what tomorrow's balance will be. Although the majority
attempts to bolster its decision with limited quotations from prior
cases, it is obvious that today's result does not depend on the language
of the Constitution itself or the principles of other decisions, but
solely on the collective judgment of the majority as to what would be a
fair and humane procedure in this case.

This decision is thus only another variant of the view often expressed
by some members of this Court that the Due Process Clause forbids any
conduct that a majority of the Court believes ``unfair,'' ``indecent,''
or ``shocking to their consciences.'' See, e. g., Rochin v. California.
Neither these words nor any like them appear anywhere in the Due Process
Clause. If they did, they would leave the majority of Justices free to
hold any conduct unconstitutional that they should conclude on their own
to be unfair or shocking to them Had the drafters of the Due Process
Clause meant to leave judges such ambulatory power to declare laws
unconstitutional, the chief value of a written constitution, as the
Founders saw it, would have been lost. In fact, if that view of due
process is correct, the Due Process Clause could easily swallow up all
other parts of the Constitution. And truly the Constitution would always
be ``what the judges say it is'' at a given moment, not what the
Founders wrote into the document A written constitution, designed to
guarantee protection against governmental abuses, including those of
judges, must have written standards that mean something definite and
have an explicit content. I regret very much to be compelled to say that
the Court today makes a drastic and dangerous departure from a
Constitution written to control and limit the government and the judges
and moves toward a constitution designed to be no more and no less than
what the judges of a particular social and economic philosophy declare
on the one hand to be fair or on the other hand to be shocking and
unconscionable.

The procedure required today as a matter of constitutional law finds no
precedent in our legal system. Reduced to its simplest terms, the
problem in this case is similar to that frequently encountered when two
parties have an ongoing legal relationship that requires one party to
make periodic payments to the other. Often the situation arises where
the party ``owing'' the money stops paying it and justifies his conduct
by arguing that the recipient is not legally entitled to payment. The
recipient can, of course, disagree and go to court to compel payment.
But I know of no situation in our legal system in which the person
alleged to owe money to another is required by law to continue making
payments to a judgment-proof claimant without the benefit of any
security or bond to insure that these payments can be recovered if he
wins his legal argument. Yet today's decision in no way obligates the
welfare recipient to pay back any benefits wrongfully received during
the pre-termination evidentiary hearings or post any bond, and in all
``fairness'' it could not do so. These recipients are by definition too
poor to post a bond or to repay the benefits that, as the majority
assumes, must be spent as received to insure survival.

The Court apparently feels that this decision will benefit the poor and
needy. In my judgment the eventual result will be just the opposite.
While today's decision requires only an administrative, evidentiary
hearing, the inevitable logic of the approach taken will lead to
constitutionally imposed, time-consuming delays of a full adversary
process of administrative and judicial review. In the next case the
welfare recipients are bound to argue that cutting off benefits before
judicial review of the agency's decision is also a denial of due
process. Since, by hypothesis, termination of aid at that point may
still ``deprive an eligible recipient of the very means by which to live
while he waits,'' I would be surprised if the weighing process did not
compel the conclusion that termination without full judicial review
would be unconscionable. After all, at each step, as the majority seems
to feel, the issue is only one of weighing the government's pocketbook
against the actual survival of the recipient, and surely that balance
must always tip in favor of the individual. Similarly today's decision
requires only the opportunity to have the benefit of counsel at the
administrative hearing, but it is difficult to believe that the same
reasoning process would not require the appointment of counsel, for
otherwise the right to counsel is a meaningless one since these people
are too poor to hire their own advocates. Cf. Gideon v. Wainwright. Thus
the end result of today's decision may well be that the government, once
it decides to give welfare benefits, cannot reverse that decision until
the recipient has had the benefits of full administrative and judicial
review, including, of course, the opportunity to present his case to
this Court. Since this process will usually entail a delay of several
years, the inevitable result of such a constitutionally imposed burden
will be that the government will not put a claimant on the rolls
initially until it has made an exhaustive investigation to determine his
eligibility. While this Court will perhaps have insured that no needy
person will be taken off the rolls without a full ``due process''
proceeding, it will also have insured that many will never get on the
rolls, or at least that they will remain destitute during the lengthy
proceedings followed to determine initial eligibility.

For the foregoing reasons I dissent from the Court's holding. The
operation of a welfare state is a new experiment for our Nation. For
this reason, among others, I feel that new experiments in carrying out a
welfare program should not be frozen into our constitutional structure.
They should be left, as are other legislative determinations, to the
Congress and the legislatures that the people elect to make our laws.

I am aware that some feel that the process employed in reaching today's
decision is not dependent on the individual views of the Justices
involved, but is a mere objective search for the ``collective conscience
of mankind,'' but in my view that description is only a euphemism for an
individual's judgment. Judges are as human as anyone and as likely as
others to see the world through their own eyes and find the ``collective
conscience'' remarkably similar to their own. Cf. Griswold v.
Connecticut (1965) (BLACK, J., dissenting); Sniadach v. Family Finance
Corp.~(1969) (BLACK, J., dissenting).

To realize how uncertain a standard of ``fundamental fairness'' would
be, one has only to reflect for a moment on the possible disagreement if
the ``fairness'' of the procedure in this case were propounded to the
head of the National Welfare Rights Organization, the president of the
national Chamber of Commerce, and the chairman of the John Birch
Society.

\hypertarget{mathews-v.-eldridge}{%
\subsubsection{Mathews v. Eldridge}\label{mathews-v.-eldridge}}

424 U.S. 319 (1976)

\textbf{MR. JUSTICE POWELL delivered the opinion of the Court.} The
issue in this case is whether the Due Process Clause of the Fifth
Amendment requires that prior to the termination of Social Security
disability benefit payments the recipient be afforded an opportunity for
an evidentiary hearing.

Cash benefits are provided to workers during periods in which they are
completely disabled under the disability insurance benefits program
created by the 1956 amendments to the Social Security Act. 70 Stat. 815,
42 U. S. C. § 423 Respondent Eldridge was first awarded benefits in June
1968. In March 1972, he received a questionnaire from the state agency
charged with monitoring his medical condition. Eldridge completed the
questionnaire, indicating that his condition had not improved and
identifying the medical sources, including physicians, from whom he had
received treatment recently. The state agency then obtained reports from
his physician and a psychiatric consultant. After considering these
reports and other information in his file the agency informed Eldridge
by letter that it had made a tentative determination that his disability
had ceased in May 1972. The letter included a statement of reasons for
the proposed termination of benefits, and advised Eldridge that he might
request reasonable time in which to obtain and submit additional
information pertaining to his condition.

In his written response, Eldridge disputed one characterization of his
medical condition and indicated that the agency already had enough
evidence to establish his disability The state agency then made its
final determination that he had ceased to be disabled in May 1972. This
determination was accepted by the Social Security Administration (SSA),
which notified Eldridge in July that his benefits would terminate after
that month. The notification also advised him of his right to seek
reconsideration by the state agency of this initial determination within
six months.

Instead of requesting reconsideration Eldridge commenced this action
challenging the constitutional validity of the administrative procedures
established by the Secretary of Health, Education, and Welfare for
assessing whether there exists a continuing disability. He sought an
immediate reinstatement of benefits pending a hearing on the issue of
his disability 361 F. Supp. 520 (WD Va. 1973). The Secretary moved to
dismiss on the grounds that Eldridge's benefits had been terminated in
accordance with valid administrative regulations and procedures and that
he had failed to exhaust available remedies. In support of his
contention that due process requires a pretermination hearing, Eldridge
relied exclusively upon this Court's decision in Goldberg v. Kelly,
which established a right to an ``evidentiary hearing'' prior to
termination of welfare benefits The Secretary contended that Goldberg
was not controlling since eligibility for disability benefits, unlike
eligibility for welfare benefits, is not based on financial need and
since issues of credibility and veracity do not play a significant role
in the disability entitlement decision, which turns primarily on medical
evidence.

The District Court concluded that the administrative procedures pursuant
to which the Secretary had terminated Eldridge's benefits abridged his
right to procedural due process. The court viewed the interest of the
disability recipient in uninterrupted benefits as indistinguishable from
that of the welfare recipient in Goldberg. It further noted that
decisions subsequent to Goldberg demonstrated that the due process
requirement of pretermination hearings is not limited to situations
involving the deprivation of vital necessities. See Fuentes v. Shevin
(1972); Bell v. Burson. Reasoning that disability determinations may
involve subjective judgments based on conflicting medical and nonmedical
evidence, the District Court held that prior to termination of benefits
Eldridge had to be afforded an evidentiary hearing of the type required
for welfare beneficiaries under Title IV of the Social Security Act. 361
F. Supp. Relying entirely upon the District Court's opinion, the Court
of Appeals for the Fourth Circuit affirmed the injunction barring
termination of Eldridge's benefits prior to an evidentiary hearing. We
reverse.

Procedural due process imposes constraints on governmental decisions
which deprive individuals of ``liberty'' or ``property'' interests
within the meaning of the Due Process Clause of the Fifth or Fourteenth
Amendment. The Secretary does not contend that procedural due process is
inapplicable to terminations of Social Security disability benefits. He
recognizes, as has been implicit in our prior decisions, e. g.,
Richardson v. Belcher (1971); Richardson v. Perales (1971); Flemming v.
Nestor, that the interest of an individual in continued receipt of these
benefits is a statutorily created ``property'' interest protected by the
Fifth Amendment. Cf. Arnett v. Kennedy POWELL, J., concurring in part)
(1974); Board of Regents v. Roth (1972); Bell v. Burson; Goldberg v.
Kelly. Rather, the Secretary contends that the existing administrative
procedures, detailed below, provide all the process that is
constitutionally due before a recipient can be deprived of that
interest.

This Court consistently has held that some form of hearing is required
before an individual is finally deprived of a property interest. Wolff
v. McDonnell (1974). See, e. g., Phillips v. Commissioner (1931). See
also Dent v. West Virginia (1889). The ``right to be heard before being
condemned to suffer grievous loss of any kind, even though it may not
involve the stigma and hardships of a criminal conviction, is a
principle basic to our society.'' Joint Anti-Fascist Comm. v. McGrath
(Frankfurter, J., concurring). The fundamental requirement of due
process is the opportunity to be heard ``at a meaningful time and in a
meaningful manner.'' Armstrong v. Manzo. See Grannis v. Ordean. Eldridge
agrees that the review procedures available to a claimant before the
initial determination of ineligibility becomes final would be adequate
if disability benefits were not terminated until after the evidentiary
hearing stage of the administrative process. The dispute centers upon
what process is due prior to the initial termination of benefits,
pending review.

In recent years this Court increasingly has had occasion to consider the
extent to which due process requires an evidentiary hearing prior to the
deprivation of some type of property interest even if such a hearing is
provided thereafter. In only one case, Goldberg v. Kelly, has the Court
held that a hearing closely approximating a judicial trial is necessary.
In other cases requiring some type of pretermination hearing as a matter
of constitutional right the Court has spoken sparingly about the
requisite procedures. Sniadach v. Family Finance Corp., involving
garnishment of wages, was entirely silent on the matter. In Fuentes v.
Shevin, the Court said only that in a replevin suit between two private
parties the initial determination required something more than an ex
parte proceeding before a court clerk. Similarly, Bell v. Burson, held,
in the context of the revocation of a state-granted driver's license,
that due process required only that the prerevocation hearing involve a
probable-cause determination as to the fault of the licensee, noting
that the hearing ``need not take the form of a full adjudication of the
question of liability.'' See also North Georgia Finishing, Inc.~v.
Di-Chem, Inc.. More recently, in Arnett v. Kennedywe sustained the
validity of procedures by which a federal employee could be dismissed
for cause. They included notice of the action sought, a copy of the
charge, reasonable time for filing a written response, and an
opportunity for an oral appearance. Following dismissal, an evidentiary
hearing was provided.

These decisions underscore the truism that ``\,`{[}d{]}ue process,'
unlike some legal rules, is not a technical conception with a fixed
content unrelated to time, place and circumstances.'' Cafeteria Workers
v. McElroy. ``{[}D{]}ue process is flexible and calls for such
procedural protections as the particular situation demands.'' Morrissey
v. Brewer. Accordingly, resolution of the issue whether the
administrative procedures provided here are constitutionally sufficient
requires analysis of the governmental and private interests that are
affected. Arnett v. Kennedy (POWELL, J., concurring in part); Goldberg
v. Kelly; Cafeteria Workers v. McElroy. More precisely, our prior
decisions indicate that identification of the specific dictates of due
process generally requires consideration of three distinct factors:
First, the private interest that will be affected by the official
action; second, the risk of an erroneous deprivation of such interest
through the procedures used, and the probable value, if any, of
additional or substitute procedural safeguards; and finally, the
Government's interest, including the function involved and the fiscal
and administrative burdens that the additional or substitute procedural
requirement would entail. See, e. g., Goldberg v. Kelly.

We turn first to a description of the procedures for the termination of
Social Security disability benefits, and thereafter consider the factors
bearing upon the constitutional adequacy of these procedures.

The disability insurance program is administered jointly by state and
federal agencies. State agencies make the initial determination whether
a disability exists, when it began, and when it ceased. 42 U. S. C. §
421 (a) The standards applied and the procedures followed are prescribed
by the Secretary, see § 421 (b), who has delegated his responsibilities
and powers under the Act to the SSA. See 40 Fed. Reg. 4473 (1975).

In order to establish initial and continued entitlement to disability
benefits a worker must demonstrate that he is unable

``to engage in any substantial gainful activity by reason of any
medically determinable physical or mental impairment which can be
expected to result in death or which has lasted or can be expected to
last for a continuous period of not less than 12 months . .'' 42 U. S.
C. § 423 (d) (1) (A).

To satisfy this test the worker bears a continuing burden of showing, by
means of ``medically acceptable clinical and laboratory diagnostic
techniques,'' § 423 (d) (3), that he has a physical or mental impairment
of such severity that

``he is not only unable to do his previous work but cannot, considering
his age, education, and work experience, engage in any other kind of
substantial gainful work which exists in the national economy,
regardless of whether such work exists in the immediate area in which he
lives, or whether a specific job vacancy exists for him, or whether he
would be hired if he applied for work.'' § 423 (d) (2) (A).

The principal reasons for benefits terminations are that the worker is
no longer disabled or has returned to work. As Eldridge's benefits were
terminated because he was determined to be no longer disabled, we
consider only the sufficiency of the procedures involved in such cases.

The continuing-eligibility investigation is made by a state agency
acting through a ``team'' consisting of a physician and a nonmedical
person trained in disability evaluation. The agency periodically
communicates with the disabled worker, usually by mail---in which case
he is sent a detailed questionnaire---or by telephone, and requests
information concerning his present condition, including current medical
restrictions and sources of treatment, and any additional information
that he considers relevant to his continued entitlement to benefits. CM
§ 6705 ; Disability Insurance State Manual (DISM) § 353 (TL No.~137,
Mar.~5, 1975).

Information regarding the recipient's current condition is also obtained
from his sources of medical treatment. DISM § 353 If there is a conflict
between the information provided by the beneficiary and that obtained
from medical sources such as his physician, or between two sources of
treatment, the agency may arrange for an examination by an independent
consulting physician Whenever the agency's tentative assessment of the
beneficiary's condition differs from his own assessment, the beneficiary
is informed that benefits may be terminated, provided a summary of the
evidence upon which the proposed determination to terminate is based,
and afforded an opportunity to review the medical reports and other
evidence in his case file He also may respond in writing and submit
additional evidence. § 353

The state agency then makes its final determination, which is reviewed
by an examiner in the SSA Bureau of Disability Insurance. 42 U. S. C. §
421 (c); CM §§ 6701 (b), (c) If, as is usually the case, the SSA accepts
the agency determination it notifies the recipient in writing, informing
him of the reasons for the decision, and of his right to seek de novo
reconsideration by the state agency. 20 CFR §§ 404 404 (1975) Upon
acceptance by the SSA, benefits are terminated effective two months
after the month in which medical recovery is found to have occurred. 42
U. S. C. § 423 (a) (1970 ed., Supp. III).

If the recipient seeks reconsideration by the state agency and the
determination is adverse, the SSA reviews the reconsideration
determination and notifies the recipient of the decision. He then has a
right to an evidentiary hearing before an SSA administrative law judge.
20 CFR §§ 404 404 (1975). The hearing is nonadversary, and the SSA is
not represented by counsel. As at all prior and subsequent stages of the
administrative process, however, the claimant may be represented by
counsel or other spokesmen. § 404 If this hearing results in an adverse
decision, the claimant is entitled to request discretionary review by
the SSA Appeals Council, § 404 and finally may obtain judicial review.
42 U. S. C. § 405 (g); 20 CFR § 404 (1975).

Should it be determined at any point after termination of benefits, that
the claimant's disability extended beyond the date of cessation
initially established, the worker is entitled to retroactive payments.
42 U. S. C. § 404. Cf. § 423 (b); 20 CFR §§ 404 404 404 (1975). If, on
the other hand, a beneficiary receives any payments to which he is later
determined not to be entitled, the statute authorizes the Secretary to
attempt to recoup these funds in specified circumstances. 42 U. S. C. §
404.

Despite the elaborate character of the administrative procedures
provided by the Secretary, the courts below held them to be
constitutionally inadequate, concluding that due process requires an
evidentiary hearing prior to termination. In light of the private and
governmental interests at stake here and the nature of the existing
procedures, we think this was error.

Since a recipient whose benefits are terminated is awarded full
retroactive relief if he ultimately prevails, his sole interest is in
the uninterrupted receipt of this source of income pending final
administrative decision on his claim. His potential injury is thus
similar in nature to that of the welfare recipient in Goldberg, see 397
U. S., the nonprobationary federal employee in Arnett, see 416 U. S.,
and the wage earner in Sniadach. See 395 U. S..

Only in Goldberg has the Court held that due process requires an
evidentiary hearing prior to a temporary deprivation. It was emphasized
there that welfare assistance is given to persons on the very margin of
subsistence:

``The crucial factor in this context---a factor not present in the case
of . virtually anyone else whose governmental entitlements are
ended---is that termination of aid pending resolution of a controversy
over eligibility may deprive an eligible recipient of the very means by
which to live while he waits.'' 397 U. S. (emphasis in original).

Eligibility for disability benefits, in contrast, is not based upon
financial need Indeed, it is wholly unrelated to the worker's income or
support from many other sources, such as earnings of other family
members, workmen's compensation awards tort claims awards, savings,
private insurance, public or private pensions, veterans' benefits, food
stamps, public assistance, or the ``many other important programs, both
public and private, which contain provisions for disability payments
affecting a substantial portion of the work force . .'' Richardson v.
Belcher (Douglas, J., dissenting). See Staff of the House Committee on
Ways and Means, Report on the Disability Insurance Program, 93d Cong.,
2d Sess., 9 (1974) (hereinafter Staff Report).

As Goldberg illustrates, the degree of potential deprivation that may be
created by a particular decision is a factor to be considered in
assessing the validity of any administrative decisionmaking process. Cf.
Morrissey v. Brewer. The potential deprivation here is generally likely
to be less than in Goldberg, although the degree of difference can be
overstated. As the District Court emphasized, to remain eligible for
benefits a recipient must be ``unable to engage in substantial gainful
activity.'' 42 U. S. C. § 423; 361 F. Supp.. Thus, in contrast to the
discharged federal employee in Arnett, there is little possibility that
the terminated recipient will be able to find even temporary employment
to ameliorate the interim loss.

As we recognized last Term in Fusari v. Steinberg, ``the possible length
of wrongful deprivation of . benefits {[}also{]} is an important factor
in assessing the impact of official action on the private interests.''
The Secretary concedes that the delay between a request for a hearing
before an administrative law judge and a decision on the claim is
currently between 10 and 11 months. Since a terminated recipient must
first obtain a reconsideration decision as a prerequisite to invoking
his right to an evidentiary hearing, the delay between the actual cutoff
of benefits and final decision after a hearing exceeds one year.

In view of the torpidity of this administrative review process, cf.~and
the typically modest resources of the family unit of the physically
disabled worker the hardship imposed upon the erroneously terminated
disability recipient may be significant. Still, the disabled worker's
need is likely to be less than that of a welfare recipient. In addition
to the possibility of access to private resources, other forms of
government assistance will become available where the termination of
disability benefits places a worker or his family below the subsistence
level See Arnett v. Kennedy, 416 U. S. (POWELL, J., concurring in part);
(WHITE, J., concurring in part and dissenting in part). In view of these
potential sources of temporary income, there is less reason here than in
Goldberg to depart from the ordinary principle, established by our
decisions, that something less than an evidentiary hearing is sufficient
prior to adverse administrative action.

An additional factor to be considered here is the fairness and
reliability of the existing pretermination procedures, and the probable
value, if any, of additional procedural safeguards. Central to the
evaluation of any administrative process is the nature of the relevant
inquiry. See Mitchell v. W. T. Grant Co.; Friendly, Some Kind of
Hearing, 123 U. Pa. L. Rev.~1267, 1281 (1975). In order to remain
eligible for benefits the disabled worker must demonstrate by means of
``medically acceptable clinical and laboratory diagnostic techniques,''
42 U. S. C. § 423 (d) (3), that he is unable ``to engage in any
substantial gainful activity by reason of any medically determinable
physical or mental impairment . .'' § 423 (d) (1) (A) (emphasis
supplied). In short, a medical assessment of the worker's physical or
mental condition is required. This is a more sharply focused and easily
documented decision than the typical determination of welfare
entitlement. In the latter case, a wide variety of information may be
deemed relevant, and issues of witness credibility and veracity often
are critical to the decisionmaking process. Goldberg noted that in such
circumstances ``written submissions are a wholly unsatisfactory basis
for decision.'' 397 U. S..

By contrast, the decision whether to discontinue disability benefits
will turn, in most cases, upon ``routine, standard, and unbiased medical
reports by physician specialists,'' Richardson v. Perales, concerning a
subject whom they have personally examined In Richardson the Court
recognized the ``reliability and probative worth of written medical
reports,'' emphasizing that while there may be ``professional
disagreement with the medical conclusions'' the ``specter of
questionable credibility and veracity is not present.'' 407. To be sure,
credibility and veracity may be a factor in the ultimate disability
assessment in some cases. But procedural due process rules are shaped by
the risk of error inherent in the truthfinding process as applied to the
generality of cases, not the rare exceptions. The potential value of an
evidentiary hearing, or even oral presentation to the decisionmaker, is
substantially less in this context than in Goldberg.

The decision in Goldberg also was based on the Court's conclusion that
written submissions were an inadequate substitute for oral presentation
because they did not provide an effective means for the recipient to
communicate his case to the decisionmaker. Written submissions were
viewed as an unrealistic option, for most recipients lacked the
``educational attainment necessary to write effectively'' and could not
afford professional assistance. In addition, such submissions would not
provide the ``flexibility of oral presentations'' or ``permit the
recipient to mold his argument to the issues the decision maker appears
to regard as important.'' 397 U. S.. In the context of the
disability-benefits-entitlement assessment the administrative procedures
under review here fully answer these objections.

The detailed questionnaire which the state agency periodically sends the
recipient identifies with particularity the information relevant to the
entitlement decision, and the recipient is invited to obtain assistance
from the local SSA office in completing the questionnaire. More
important, the information critical to the entitlement decision usually
is derived from medical sources, such as the treating physician. Such
sources are likely to be able to communicate more effectively through
written documents than are welfare recipients or the lay witnesses
supporting their cause. The conclusions of physicians often are
supported by X-rays and the results of clinical or laboratory tests,
information typically more amenable to written than to oral
presentation.

A further safeguard against mistake is the policy of allowing the
disability recipient's representative full access to all information
relied upon by the state agency. In addition, prior to the cutoff of
benefits the agency informs the recipient of its tentative assessment,
the reasons therefor, and provides a summary of the evidence that it
considers most relevant. Opportunity is then afforded the recipient to
submit additional evidence or arguments, enabling him to challenge
directly the accuracy of information in his file as well as the
correctness of the agency's tentative conclusions. These procedures,
again as contrasted with those before the Court in Goldberg, enable the
recipient to ``mold'' his argument to respond to the precise issues
which the decisionmaker regards as crucial.

Despite these carefully structured procedures, amici point to the
significant reversal rate for appealed cases as clear evidence that the
current process is inadequate. Depending upon the base selected and the
line of analysis followed, the relevant reversal rates urged by the
contending parties vary from a high of 58 \% for appealed
reconsideration decisions to an overall reversal rate of only 3 \% Bare
statistics rarely provide a satisfactory measure of the fairness of a
decisionmaking process. Their adequacy is especially suspect here since
the administrative review system is operated on an openfile basis. A
recipient may always submit new evidence, and such submissions may
result in additional medical examinations. Such fresh examinations were
held in approximately 30\% to 40\% of the appealed cases in fiscal 1973,
either at the reconsideration or evidentiary hearing stage of the
administrative process. Staff Report 238. In this context, the value of
reversal rate statistics as one means of evaluating the adequacy of the
pretermination process is diminished. Thus, although we view such
information as relevant, it is certainly not controlling in this case.

In striking the appropriate due process balance the final factor to be
assessed is the public interest. This includes the administrative burden
and other societal costs that would be associated with requiring, as a
matter of constitutional right, an evidentiary hearing upon demand in
all cases prior to the termination of disability benefits. The most
visible burden would be the incremental cost resulting from the
increased number of hearings and the expense of providing benefits to
ineligible recipients pending decision. No one can predict the extent of
the increase, but the fact that full benefits would continue until after
such hearings would assure the exhaustion in most cases of this
attractive option. Nor would the theoretical right of the Secretary to
recover undeserved benefits result, as a practical matter, in any
substantial offset to the added outlay of public funds. The parties
submit widely varying estimates of the probable additional financial
cost. We only need say that experience with the constitutionalizing of
government procedures suggests that the ultimate additional cost in
terms of money and administrative burden would not be insubstantial.

Financial cost alone is not a controlling weight in determining whether
due process requires a particular procedural safeguard prior to some
administrative decision. But the Government's interest, and hence that
of the public, in conserving scarce fiscal and administrative resources
is a factor that must be weighed. At some point the benefit of an
additional safeguard to the individual affected by the administrative
action and to society in terms of increased assurance that the action is
just, may be outweighed by the cost. Significantly, the cost of
protecting those whom the preliminary administrative process has
identified as likely to be found undeserving may in the end come out of
the pockets of the deserving since resources available for any
particular program of social welfare are not unlimited. See Friendly U.
Pa. L. Rev., 1303.

But more is implicated in cases of this type than ad hoc weighing of
fiscal and administrative burdens against the interests of a particular
category of claimants. The ultimate balance involves a determination as
to when, under our constitutional system, judicial-type procedures must
be imposed upon administrative action to assure fairness. We reiterate
the wise admonishment of Mr.~Justice Frankfurter that differences in the
origin and function of administrative agencies ``preclude wholesale
transplantation of the rules of procedure, trial, and review which have
evolved from the history and experience of courts.'' FCC v. Pottsville
Broadcasting Co.. The judicial model of an evidentiary hearing is
neither a required, nor even the most effective, method of
decisionmaking in all circumstances. The essence of due process is the
requirement that ``a person in jeopardy of serious loss {[}be given{]}
notice of the case against him and opportunity to meet it.'' Joint
Anti-Fascist Comm. v. McGrath (Frankfurter, J., concurring). All that is
necessary is that the procedures be tailored, in light of the decision
to be made, to ``the capacities and circumstances of those who are to be
heard,'' Goldberg v. Kelly (footnote omitted), to insure that they are
given a meaningful opportunity to present their case. In assessing what
process is due in this case, substantial weight must be given to the
good-faith judgments of the individuals charged by Congress with the
administration of social welfare programs that the procedures they have
provided assure fair consideration of the entitlement claims of
individuals. See Arnett v. Kennedy (WHITE, J., concurring in part and
dissenting in part). This is especially so where, as here, the
prescribed procedures not only provide the claimant with an effective
process for asserting his claim prior to any administrative action, but
also assure a right to an evidentiary hearing, as well as to subsequent
judicial review, before the denial of his claim becomes final. Cf.
Boddie v. Connecticut.

We conclude that an evidentiary hearing is not required prior to the
termination of disability benefits and that the present administrative
procedures fully comport with due process.

The judgment of the Court of Appeals is

Reversed.

\textbf{MR. JUSTICE BRENNAN, with whom MR. JUSTICE MARSHALL concurs,
dissenting.} For the reasons stated in my dissenting opinion in
Richardson v. Wright, I agree with the District Court and the Court of
Appeals that, prior to termination of benefits, Eldridge must be
afforded an evidentiary hearing of the type required for welfare
beneficiaries under Title IV of the Social Security Act, 42 U. S. C. §
601 et seq. See Goldberg v. Kelly. I would add that the Court's
consideration that a discontinuance of disability benefits may cause the
recipient to suffer only a limited deprivation is no argument. It is
speculative. Moreover, the very legislative determination to provide
disability benefits, without any prerequisite determination of need in
fact, presumes a need by the recipient which is not this Court's
function to denigrate. Indeed, in the present case, it is indicated that
because disability benefits were terminated there was a foreclosure upon
the Eldridge home and the family's furniture was repossessed, forcing
Eldridge, his wife, and their children to sleep in one bed. Tr. of Oral
Arg. 39, 47-48. Finally, it is also no argument that a worker, who has
been placed in the untenable position of having been denied disability
benefits, may still seek other forms of public assistance.

\hypertarget{londoner-v.-city-and-county-of-denver}{%
\subsubsection{Londoner v. City and County of
Denver}\label{londoner-v.-city-and-county-of-denver}}

210 U.S. 373 (1908)

\textbf{MR. JUSTICE MOODY delivered the opinion of the court.} The
plaintiffs in error began this proceeding in a state court of Colorado
to relieve lands owned by them from an assessment of a tax for the cost
of paving a street upon which the lands abutted. The relief sought was
granted by the trial court, but its action was reversed by the Supreme
Court of the State, which ordered judgment for the defendants. 33
Colorado, 104. The case is here on writ of error. The Supreme Court held
that the tax was assessed in conformity with the constitution and laws
of the State, and its decision on that question is conclusive.

The assignments of error relied upon are as follows:

"Fifth. The Supreme Court of Colorado more particularly erred in holding
and deciding that the city authorities, in following the procedure in
this Eighth Avenue Paving District, No.~1, of the city of Denver,
Colorado, in the manner in which the record, evidence and decree of the
trial court affirmatively shows that they did, constituted due process
of law as to these several appellees (now plaintiffs in error) as
guaranteed by the Fourteenth Amendment of the Constitution of the United
States.

These assignments will be passed upon in the order in which they seem to
arise in the consideration of the whole case.

The tax complained of was assessed under the provisions of the charter
of the city of Denver, which confers upon the city the power to make
local improvements and to assess the cost upon property specially
benefited. It does not seem necessary to set forth fully the elaborate
provisions of the charter regulating the exercise of this power, except
where they call for special examination. The board of public works, upon
the petition of a majority of the owners of the frontage to be assessed,
may order the paving of a street. The board must, however, first adopt
specifications, mark out a district of assessment, cause a map to be
made and an estimate of the cost, with the approximate amount to be
assessed upon each lot of land. Before action notice by publication and
an opportunity to be heard to any person interested must be given by the
board.

The board may then order the improvement, but must recommend to the city
council a form of ordinance authorizing it, and establishing an
assessment district, which is not amendable by the council. The council
may then, in its discretion, pass or refuse to pass the ordinance. If
the ordinance is passed, the contract for the work is made by the mayor.
The charter provides that ``the finding of the city council, by
ordinance, that any improvements provided for in this article were duly
ordered after notice duly given, or that a petition or remonstrance was
or was not filed as above provided, or was or was not subscribed by the
required number of owners aforesaid shall be conclusive in every court
or other tribunal.'' The charter then provides for the assessment of the
cost in the following sections:

"SEC. 29. Upon completion of any local improvement, or, in the case of
sewers, upon completion from time to time of any part or parts thereof,
affording complete drainage for any part or parts of the district and
acceptance thereof by the board of public works, or whenever the total
cost of any such improvement, or of any such part or parts of any sewer,
can be definitely ascertained, the board of public works shall prepare a
statement therein showing the whole cost of the improvement, or such
parts thereof, including six per cent additional for costs of collection
and other incidentals, and interest to the next succeeding date upon
which general taxes, or the first installment thereof, are by the laws
of this State made payable; and apportioning the same upon each lot or
tract of land to be assessed for the same, as hereinabove provided; and
shall cause the same to be certified by the president and filed in the
office of the city clerk.

"SEC. 30. The city clerk shall thereupon, by advertisement for ten days
in some newspaper of general circulation, published in the city of
Denver, notify the owners of the real estate to be assessed that said
improvements have been, or are about to be, completed and accepted,
therein specifying the whole cost of the improvements and the share so
apportioned to each lot or tract of land; and that any complaints or
objections that may be made in writing, by the owners, to the city
council and filed with the city clerk within thirty days from the first
publication of such notice, will be heard and determined by the city
council before the passage of any ordinance assessing the cost of said
improvements.

``SEC. 31. After the period specified in said notice the city council,
sitting as a board of equalization, shall hear and determine all such
complaints and objections, and may recommend to the board of public
works any modification of the apportionments made by said board; the
board may thereupon make such modifications and changes as to them may
seem equitable and just, or may confirm the first apportionment, and
shall notify the city council of their final decision; and the city
council shall thereupon by ordinance assess the cost of said
improvements against all the real estate in said district respectively
in the proportions above mentioned.''

It appears from the charter that, in the execution of the power to make
local improvements and assess the cost upon the property specially
benefited, the main steps to be taken by the city authorities are
plainly marked and separated: 1. The board of public works must transmit
to the city council a resolution ordering the work to be done and the
form of an ordinance authorizing it and creating an assessment district.
This it can do only upon certain conditions, one of which is that there
shall first be filed a petition asking the improvement, signed by the
owners of the majority of the frontage to be assessed. 2. The passage of
that ordinance by the city council, which is given authority to
determine conclusively whether the action of the board was duly taken.
3. The assessment of the cost upon the landowners after due notice and
opportunity for hearing.

The fifth assignment, though general, vague and obscure, fairly raises,
we think, the question whether the assessment was made without notice
and opportunity for hearing to those affected by it, thereby denying to
them due process of law. The trial court found as a fact that no
opportunity for hearing was afforded, and the Supreme Court did not
disturb this finding. The record discloses what was actually done, and
there seems to be no dispute about it. After the improvement was
completed the board of public works, in compliance with § 29 of the
charter, certified to the city clerk a statement of the cost, and an
apportionment of it to the lots of land to be assessed. Thereupon the
city clerk, in compliance with § 30, published a notice stating, inter
alia, that the written complaints or objections of the owners, if filed
within thirty days, would be ``heard and determined by the city council
before the passage of any ordinance assessing the cost.'' Those
interested, therefore, were informed that if they reduced their
complaints and objections to writing, and filed them within thirty days,
those complaints and objections would be heard, and would be heard
before any assessment was made. The notice given in this case, although
following the words of the statute, did not fix the time for hearing,
and apparently there were no stated sittings of the council acting as a
board of equalization. But the notice purported only to fix the time for
filing the complaints and objections, and to inform those who should
file them that they would be heard before action. The statute expressly
required no other notice, but it was sustained in the court below on the
authority of Paulsen v. Portland, because there was an implied power in
the city council to give notice of the time for hearing. We think that
the court rightly conceived the meaning of that case and that the
statute could be sustained only upon the theory drawn from it. Resting
upon the assurance that they would be heard, the plaintiffs in error
filed within the thirty days the following paper:

"Denver, Colorado, January 13, 1900.

"To the Honorable Board of Public Works and the Honorable Mayor and City
Council of the City of Denver:

"The undersigned, by Joshua Grozier, their attorney, do hereby most
earnestly and strenuously protest and object to the passage of the
contemplated or any assessing ordinance against the property in Eighth
Avenue Paving District No.~1, so called, for each of the following
reasons, to wit:

"1st. That said assessment and all and each of the proceedings leading
up to the same were and are illegal, voidable and void, and the
attempted assessment if made will be void and uncollectible.

"2nd. That said assessment and the cost of said pretended improvement
should be collected, if at all, as a general tax against the city at
large and not as a special assessment.

"3d. That property in said city not assessed is benefited by the said
pretended improvement and certain property assessed is not benefited by
said pretended improvement and other property assessed is not benefited
by said pretended improvement to the extent of the assessment; that the
individual pieces of property in said district are not benefited to the
extent assessed against them and each of them respectively; that the
assessment is abitrary and property assessed in an equal amount is not
benefited equally; that the boundaries of said pretended district were
arbitrarily created without regard to the benefits or any other method
of assessment known to law; that said assessment is outrageously large.

"4th. That each of the laws and each section thereof under which the
proceedings in said pretended district were attempted to be had do not
confer the authority for such proceedings; that the 1893 city charter
was not properly passed and is not a law of the State of Colorado by
reason of not properly or at all passing the legislature; that each of
the provisions of said charter under which said proceedings were
attempted are unconstitutional and violative of fundamental principles
of law, the Constitution of the United States and the state
constitution, or some one or more of the provisions of one or more of
the same.

"5th. Because the pretended notice of assessment is invalid and was not
published in accordance with the law, and is in fact no notice at all;
because there was and is no valid ordinance creating said district;
because each notice required by the 1893 city charter to be given, where
it was attempted to give such notice, was insufficient, and was not
properly given or properly published.

"6th. Because of non-compliance by the contractor with his contract and
failure to complete the work in accordance with the contract; because
the contract for said work was let without right or authority; because
said pretended district is incomplete and the work under said contract
has not been completed in accordance with said contract; because items
too numerous to mention, which were not a proper charge in the said
assessment, are included therein.

"7th. Because the work was done under pretended grants of authority
contained in pretended laws, which laws were violative of the
constitution and fundamental laws of the State and Union.

"8th. Because the city had no jurisdiction in the premises. No petition
subscribed by the owners of a majority of the frontage in the district
to be assessed for said improvements was ever obtained or presented.

"9th. Because of delay by the board of public works in attempting to let
the contract and because the said pretended improvement was never
properly nor sufficiently petitioned for; because the contracts were not
let nor the work done in accordance with the petitions, if any, for the
work, and because the city had no jurisdiction in the premises.

"10th. Because before ordering the pretended improvement full details
and specifications for the same, permitting and encouraging competition
and determining the number of installments and time within which the
costs shall be payable, the rate of interest on unpaid installments, and
the district of lands to be assessed, together with a map showing the
approximate amounts to be assessed, were not adopted by the board of
public works before the letting of the contract for the work and
furnishing of material; because advertisement for 20 days in two daily
newspapers of general circulation, giving notice to the owners of real
estate in the district of the kind of improvements proposed, the number
of installments and time in which payable, rate of interest and extent
of the district, probable cost and time when a resolution ordering the
improvement would be considered, was not made either properly or at all,
and if ever attempted to be made was not made according to law or as
required by the law or charter.

"11th. Because the attempted advertisement for bids on the contract
attempted to be let were not properly published and were published and
let, and the proceedings had, if at all, in such a way as to be
prejudicial to the competition of bidders and to deter bidders; and the
completion of the contracts after being attempted to be let was
permitted to lag in such a manner as not to comply with the contract,
charter or laws, and the power to let the contract attempted to be let
was not within the power of the parties attempting to let the same;
because the city council is or was by some of the proceedings deprived
of legislative discretion, and the board of public works and other
pretended bodies given such discretion, which discretion they delegated
to others having no right or power to exercise the same; and executive
functions were conferred on bodies having no right, power or authority
to exercise the same and taken away from others to whom such power was
attempted to be granted or given or who should properly exercise the
same; that judicial power was attempted to be conferred on the board of
public works, so called, and the city council, and other bodies or
pretended bodies not judicial or quasi-judicial in character, having no
right, power or authority to exercise the same, and the courts attempted
to be deprived thereof.

``Wherefore, because of the foregoing and numerous other good and
sufficient reasons, the undersigned object and protest against the
passage of the said proposed assessing ordinance.''

This certainly was a complaint against and objection to the proposed
assessment. Instead of affording the plaintiffs in error an opportunity
to be heard upon its allegations, the city council, without notice to
them, met as a board of equalization, not in a stated but in a specially
called session, and, without any hearing, adopted the following
resolution:

"Whereas, complaints have been filed by the various persons and firms as
the owners of real estate included within the Eighth Avenue Paving
District No.~1, of the city of Denver against the proposed assessments
on said property for the cost of said paving, the names and description
of the real estate respectively owned by such persons being more
particularly described in the various complaints filed with the city
clerk; and

"Whereas, no complaint or objection has been filed or made against the
apportionment of said assessment made by the board of public works of
the city of Denver, but the complaints and objections filed deny wholly
the right of the city to assess any district or portion of the
assessable property of the city of Denver; therefore, be it.

``Resolved, by the city council of the city of Denver, sitting as a
board of equalization, that the apportionments of said assessment made
by said board of public works be, and the same are hereby, confirmed and
approved.''

Subsequently, without further notice or hearing, the city council
enacted the ordinance of assessment whose validity is to be determined
in this case. The facts out of which the question on this assignment
arises may be compressed into small compass. The first step in the
assessment proceedings was by the certificate of the board of public
works of the cost of the improvement and a preliminary apportionment of
it. The last step was the enactment of the assessment ordinance. From
beginning to end of the proceedings the landowners, although allowed to
formulate and file complaints and objections, were not afforded an
opportunity to be heard upon them. Upon these facts was there a denial
by the State of the due process of law guaranteed by the Fourteenth
Amendment to the Constitution of the United States?

In the assessment, apportionment and collection of taxes upon property
within their jurisdiction the Constitution of the United States imposes
few restrictions upon the States. In the enforcement of such
restrictions as the Constitution does impose this court has regarded
substance and not form. But where the legislature of a State, instead of
fixing the tax itself, commits to some subordinate body the duty of
determining whether, in what amount, and upon whom it shall be levied,
and of making its assessment and apportionment, due process of law
requires that at some stage of the proceedings before the tax becomes
irrevocably fixed, the taxpayer shall have an opportunity to be heard,
of which he must have notice, either personal, by publication, or by a
law fixing the time and place of the hearing. Hager v. Reclamation
District; Kentucky Railroad Tax Cases; Winona \& St.~Peter Land Co.~v.
Minnesota; Lent v. Tillson; Glidden v. Harrington; Hibben v. Smith;
Security Trust Co.~v. Lexington; Central of Georgia v. Wright. It must
be remembered that the law of Colorado denies the landowner the right to
object in the courts to the assessment, upon the ground that the
objections are cognizable only by the board of equalization.

If it is enough that, under such circumstances, an opportunity is given
to submit in writing all objections to and complaints of the tax to the
board, then there was a hearing afforded in the case at bar. But we
think that something more than that, even in proceedings for taxation,
is required by due process of law. Many requirements essential in
strictly judicial proceedings may be dispensed with in proceedings of
this nature. But even here a hearing in its very essence demands that he
who is entitled to it shall have the right to support his allegations by
argument however brief, and, if need be, by proof, however informal.
Pittsburg \&c.; Railway Co.~v. Backus; Fallbrook Irrigation District v.
Bradley, et seq. It is apparent that such a hearing was denied to the
plaintiffs in error. The denial was by the city council, which, while
acting as a board of equalization, represents the State. Raymond v.
Chicago Traction Co.. The assessment was therefore void, and the
plaintiffs in error were entitled to a decree discharging their lands
from a lien on account of it. It is not now necessary to consider the
tenth assignment of error.

Judgment reversed.

THE CHIEF JUSTICE and MR. JUSTICE HOLMES dissent.

\hypertarget{bi-metallic-investment-co.-v.-state-board-of-equalization-of-colorado}{%
\subsubsection{Bi-Metallic Investment Co.~v. State Board of Equalization
of
Colorado}\label{bi-metallic-investment-co.-v.-state-board-of-equalization-of-colorado}}

239 U.S. 441 (1915)

\textbf{MR. JUSTICE HOLMES delivered the opinion of the court.} This is
a suit to enjoin the State Board of Equalization and the Colorado Tax
Commission from putting in force, and the defendant Pitcher as assessor
of Denver from obeying, an order of the boards increasing the valuation
of all taxable property in Denver forty per cent. The order was
sustained and the suit directed to be dismissed by the Supreme Court of
the State. 56 Colorado, 512. See 56 Colorado, 343. The plaintiff is the
owner of real estate in Denver and brings the case here on the ground
that it was given no opportunity to be heard and that therefore its
property will be taken without due process of law, contrary to the
Fourteenth Amendment of the Constitution of the United States. That is
the only question with which we have to deal. There are suggestions on
the one side that the construction of the state constitution and laws
was an unwarranted surprise and on the other that the decision might
have been placed, although it was not, on the ground that there was an
adequate remedy at law. With these suggestions we have nothing to do.
They are matters purely of state law. The answer to the former needs no
amplification; that to the latter is that the allowance of equitable
relief is a question of state policy and that as the Supreme Court of
the State treated the merits as legitimately before it, we are not to
speculate whether it might or might not have thrown out the suit upon
the preliminary ground.

For the purposes of decision we assume that the constitutional question
is presented in the baldest way --- that neither the plaintiff nor the
assessor of Denver, who presents a brief on the plaintiff's side, nor
any representative of the city and county, was given an opportunity to
be heard, other than such as they may have had by reason of the fact
that the time of meeting of the boards is fixed by law. On this
assumption it is obvious that injustice may be suffered if some property
in the county already has been valued at its full worth. But if certain
property has been valued at a rate different from that generally
prevailing in the county the owner has had his opportunity to protest
and appeal as usual in our system of taxation, Hagar v. Reclamation
District, 710, so that it must be assumed that the property owners in
the county all stand alike. The question then is whether all individuals
have a constitutional right to be heard before a matter can be decided
in which all are equally concerned --- here, for instance, before a
superior board decides that the local taxing officers have adopted a
system of undervaluation throughout a county, as notoriously often has
been the case. The answer of this court in the State Railroad Tax Cases,
at least as to any further notice, was that it was hard to believe that
the proposition was seriously made.

Where a rule of conduct applies to more than a few people it is
impracticable that every one should have a direct voice in its adoption.
The Constitution does not require all public acts to be done in town
meeting or an assembly of the whole. General statutes within the state
power are passed that affect the person or property of individuals,
sometimes to the point of ruin, without giving them a chance to be
heard. Their rights are protected in the only way that they can be in a
complex society, by their power, immediate or remote, over those who
make the rule. If the result in this case had been reached as it might
have been by the State's doubling the rate of taxation, no one would
suggest that the Fourteenth Amendment was violated unless every person
affected had been allowed an opportunity to raise his voice against it
before the body entrusted by the state constitution with the power. In
considering this case in this court we must assume that the proper state
machinery has been used, and the question is whether, if the state
constitution had declared that Denver had been undervalued as compared
with the rest of the State and had decreed that for the current year the
valuation should be forty per cent. higher, the objection now urged
could prevail. It appears to us that to put the question is to answer
it. There must be a limit to individual argument in such matters if
government is to go on. In Londoner v. Denver, 385, a local board had to
determine'whether, in what amount, and upon whom' a tax for paving a
street should be levied for special benefits. A relatively small number
of persons was concerned, who were exceptionally affected, in each case
upon individual grounds, and it was held that they had a right to a
hearing. But that decision is far from reaching a general determination
dealing only with the principle upon which all the assessments in a
county had been laid.

Judgment affirmed.

\hypertarget{hamdi-v.-rumsfeld}{%
\subsubsection{Hamdi v. Rumsfeld}\label{hamdi-v.-rumsfeld}}

542 U.S. 507 (2004)

\textbf{JUSTICE O'CONNOR announced the judgment of the Court and
delivered an opinion, in which THE CHIEF JUSTICE, JUSTICE KENNEDY, and
JUSTICE BREYER join.}

At this difficult time in our Nation's history, we are called upon to
consider the legality of the Government's detention of a United States
citizen on United States soil as an ``enemy combatant'' and to address
the process that is constitutionally owed to one who seeks to challenge
his classification as such. The United States Court of Appeals for the
Fourth Circuit held that petitioner Yaser Hamdi's detention was legally
authorized and that he was entitled to no further opportunity to
challenge his enemy-combatant label. We now vacate and remand. We hold
that although Congress authorized the detention of combatants in the
narrow circumstances alleged here, due process demands that a citizen
held in the United States as an enemy combatant be given a meaningful
opportunity to contest the factual basis for that detention before a
neutral decisionmaker.

On September 11, 2001, the al Qaeda terrorist network used hijacked
commercial airliners to attack prominent targets in the United States.
Approximately 3,000 people were killed in those attacks. One week later,
in response to these ``acts of treacherous violence,'' Congress passed a
resolution authorizing the President to ``use all necessary and
appropriate force against those nations, organizations, or persons he
determines planned, authorized, committed, or aided the terrorist
attacks'' or ``harbored such organizations or persons, in order to
prevent any future acts of international terrorism against the United
States by such nations, organizations or persons.'' Authorization for
Use of Military Force (AUMF), 115 Stat. 224. Soon thereafter, the
President ordered United States Armed Forces to Afghanistan, with a
mission to subdue al Qaeda and quell the Taliban regime that was known
to support it.

This case arises out of the detention of a man whom the Government
alleges took up arms with the Taliban during this conflict. His name is
Yaser Esam Hamdi. Born in Louisiana in 1980, Hamdi moved with his family
to Saudi Arabia as a child. By 2001, the parties agree, he resided in
Afghanistan. At some point that year, he was seized by members of the
Northern Alliance, a coalition of military groups opposed to the Taliban
government, and eventually was turned over to the United States
military. The Government asserts that it initially detained and
interrogated Hamdi in Afghanistan before transferring him to the United
States Naval Base in Guantanamo Bay in January 2002. In April 2002, upon
learning that Hamdi is an American citizen, authorities transferred him
to a naval brig in Norfolk, Virginia, where he remained until a recent
transfer to a brig in Charleston, South Carolina. The Government
contends that Hamdi is an ``enemy combatant,'' and that this status
justifies holding him in the United States indefinitely --- without
formal charges or proceedings --- unless and until it makes the
determination that access to counsel or further process is warranted.

In June 2002, Hamdi's father, Esam Fouad Hamdi, filed the present
petition for a writ of habeas corpus under 28 U. S. C. § 2241 in the
Eastern District of Virginia, naming as petitioners his son and himself
as next friend. The elder Hamdi alleges in the petition that he has had
no contact with his son since the Government took custody of him in
2001, and that the Government has held his son ``without access to legal
counsel or notice of any charges pending against him.'' App. 103, 104.
The petition contends that Hamdi's detention was not legally authorized.
It argues that, ``{[}a{]}s an American citizen, . Hamdi enjoys the full
protections of the Constitution,'' and that Hamdi's detention in the
United States without charges, access to an impartial tribunal, or
assistance of counsel ``violated and continue{[}s{]} to violate the
Fifth and Fourteenth Amendments to the United States Constitution.'' The
habeas petition asks that the court, among other things, (1) appoint
counsel for Hamdi; (2) order respondents to cease interrogating him; (3)
declare that he is being held in violation of the Fifth and Fourteenth
Amendments; (4) ``{[}t{]}o the extent Respondents contest any material
factual allegations in this Petition, schedule an evidentiary hearing,
at which Petitioners may adduce proof in support of their allegations'';
and (5) order that Hamdi be released from his ``unlawful custody.''
-109. Although his habeas petition provides no details with regard to
the factual circumstances surrounding his son's capture and detention,
Hamdi's father has asserted in documents found elsewhere in the record
that his son went to Afghanistan to do ``relief work,'' and that he had
been in that country less than two months before September 11, 2001, and
could not have received military training. -189. The 20-year-old was
traveling on his own for the first time, his father says, and
``{[}b{]}ecause of his lack of experience, he was trapped in Afghanistan
once the military campaign began.''

The District Court found that Hamdi's father was a proper next friend,
appointed the federal public defender as counsel for the petitioners,
and ordered that counsel be given access to Hamdi. -116. The United
States Court of Appeals for the Fourth Circuit reversed that order,
holding that the District Court had failed to extend appropriate
deference to the Government's security and intelligence interests. 283
(2002). It directed the District Court to consider ``the most cautious
procedures first,'' and to conduct a deferential inquiry into Hamdi's
status, It opined that ``if Hamdi is indeed an'enemy combatant' who was
captured during hostilities in Afghanistan, the government's present
detention of him is a lawful one.''

On remand, the Government filed a response and a motion to dismiss the
petition. It attached to its response a declaration from one Michael
Mobbs (hereinafter Mobbs Declaration), who identified himself as Special
Advisor to the Under Secretary of Defense for Policy. Mobbs indicated
that in this position, he has been ``substantially involved with matters
related to the detention of enemy combatants in the current war against
the al Qaeda terrorists and those who support and harbor them (including
the Taliban).'' App. 148. He expressed his ``familiar{[}ity{]}'' with
Department of Defense and United States military policies and procedures
applicable to the detention, control, and transfer of al Qaeda and
Taliban personnel, and declared that ``{[}b{]}ased upon my review of
relevant records and reports, I am also familiar with the facts and
circumstances related to the capture of \ldots{} Hamdi and his detention
by U. S. military forces.''

Mobbs then set forth what remains the sole evidentiary support that the
Government has provided to the courts for Hamdi's detention. The
declaration states that Hamdi ``traveled to Afghanistan'' in July or
August 2001, and that he thereafter ``affiliated with a Taliban military
unit and received weapons training.'' It asserts that Hamdi ``remained
with his Taliban unit following the attacks of September 11'' and that,
during the time when Northern Alliance forces were ``engaged in battle
with the Taliban,'' ``Hamdi's Taliban unit surrendered'' to those
forces, after which he ``surrender{[}ed{]} his Kalishnikov assault
rifle'' to them. -149. The Mobbs Declaration also states that, because
al Qaeda and the Taliban ``were and are hostile forces engaged in armed
conflict with the armed forces of the United States,'' ``individuals
associated with'' those groups ``were and continue to be enemy
combatants.'' Mobbs states that Hamdi was labeled an enemy combatant
``{[}b{]}ased upon his interviews and in light of his association with
the Taliban.'' According to the declaration, a series of ``U. S.
military screening team{[}s{]}'' determined that Hamdi met ``the
criteria for enemy combatants,'' and ``{[}a{]} subsequent interview of
Hamdi has confirmed the fact that he surrendered and gave his firearm to
Northern Alliance forces, which supports his classification as an enemy
combatant.'' -150.

After the Government submitted this declaration, the Fourth Circuit
directed the District Court to proceed in accordance with its earlier
ruling and, specifically, to ```consider the sufficiency of the Mobbs
declaration as an independent matter before proceeding further.''' The
District Court found that the Mobbs Declaration fell ``far short'' of
supporting Hamdi's detention. App. 292. It criticized the generic and
hearsay nature of the affidavit, calling it ``little more than the
government's'say-so.''' It ordered the Government to turn over numerous
materials for in camera review, including copies of all of Hamdi's
statements and the notes taken from interviews with him that related to
his reasons for going to Afghanistan and his activities therein; a list
of all interrogators who had questioned Hamdi and their names and
addresses; statements by members of the Northern Alliance regarding
Hamdi's surrender and capture; a list of the dates and locations of his
capture and subsequent detentions; and the names and titles of the
United States Government officials who made the determinations that
Hamdi was an enemy combatant and that he should be moved to a naval
brig. -186. The court indicated that all of these materials were
necessary for ``meaningful judicial review'' of whether Hamdi's
detention was legally authorized and whether Hamdi had received
sufficient process to satisfy the Due Process Clause of the Constitution
and relevant treaties or military regulations. -292.

The Government sought to appeal the production order, and the District
Court certified the question of whether the Mobbs Declaration,
```standing alone, is sufficient as a matter of law to allow a
meaningful judicial review of {[}Hamdi's{]} classification as an enemy
combatant.''' 316 F. 3d. The Fourth Circuit reversed, but did not
squarely answer the certified question. It instead stressed that,
because it was ``undisputed that Hamdi was captured in a zone of active
combat in a foreign theater of conflict,'' no factual inquiry or
evidentiary hearing allowing Hamdi to be heard or to rebut the
Government's assertions was necessary or proper. Concluding that the
factual averments in the Mobbs Declaration, ``if accurate,'' provided a
sufficient basis upon which to conclude that the President had
constitutionally detained Hamdi pursuant to the President's war powers,
it ordered the habeas petition dismissed. The Fourth Circuit emphasized
that the ``vital purposes'' of the detention of uncharged enemy
combatants --- preventing those combatants from rejoining the enemy
while relieving the military of the burden of litigating the
circumstances of wartime captures halfway around the globe --- were
interests ``directly derived from the war powers of Articles I and II.''
-466. In that court's view, because "Article

contains nothing analogous to the specific powers of war so carefully
enumerated in Articles I and II," separation of powers principles
prohibited a federal court from ``delv{[}ing{]} further into Hamdi's
status and capture,'' Accordingly, the District Court's more vigorous
inquiry ``went far beyond the acceptable scope of review.''

On the more global question of whether legal authorization exists for
the detention of citizen enemy combatants at all, the Fourth Circuit
rejected Hamdi's arguments that 18 U. S. C. § 4001(a) and Article 5 of
the Geneva Convention rendered any such detentions unlawful. The court
expressed doubt as to Hamdi's argument that § 4001(a), which provides
that ``{[}n{]}o citizen shall be imprisoned or otherwise detained by the
United States except pursuant to an Act of Congress,'' required express
congressional authorization of detentions of this sort. But it held
that, in any event, such authorization was found in the post-September
11 AUMF. 316 F. 3d. Because ``capturing and detaining enemy combatants
is an inherent part of warfare,'' the court held, ``the'necessary and
appropriate force' referenced in the congressional resolution
necessarily includes the capture and detention of any and all hostile
forces arrayed against our troops.'' ; see also -468 (noting that
Congress, in 10 U. S. C. § 956(5), had specifically authorized the
expenditure of funds for keeping prisoners of war and persons whose
status was determined ``to be similar to prisoners of war,'' and
concluding that this appropriation measure also demonstrated that
Congress had ``authoriz{[}ed these individuals'{]} detention in the
first instance''). The court likewise rejected Hamdi's Geneva Convention
claim, concluding that the convention is not self-executing and that,
even if it were, it would not preclude the Executive from detaining
Hamdi until the cessation of hostilities. 316 F. 3d.

Finally, the Fourth Circuit rejected Hamdi's contention that its legal
analyses with regard to the authorization for the detention scheme and
the process to which he was constitutionally entitled should be altered
by the fact that he is an American citizen detained on American soil.
Relying on Ex parte Quirin, 317 U. S. 1 (1942), the court emphasized
that ``{[}o{]}ne who takes up arms against the United States in a
foreign theater of war, regardless of his citizenship, may properly be
designated an enemy combatant and treated as such.'' 316 F. 3d. ``The
privilege of citizenship,'' the court held, ``entitles Hamdi to a
limited judicial inquiry into his detention, but only to determine its
legality under the war powers of the political branches. At least where
it is undisputed that he was present in a zone of active combat
operations, we are satisfied that the Constitution does not entitle him
to a searching review of the factual determinations underlying his
seizure there.''

The Fourth Circuit denied rehearing en banc, and we granted certiorari.
540 U. S. 1099 (2004). We now vacate the judgment below and remand.

The threshold question before us is whether the Executive has the
authority to detain citizens who qualify as ``enemy combatants.'' There
is some debate as to the proper scope of this term, and the Government
has never provided any court with the full criteria that it uses in
classifying individuals as such. It has made clear, however, that, for
purposes of this case, the ``enemy combatant'' that it is seeking to
detain is an individual who, it alleges, was
``\texttt{part\ of\ or\ supporting\ forces\ hostile\ to\ the\ United\ States\ or\ coalition\ partners\textquotesingle{}"\ in\ Afghanistan\ and\ who\ "}engaged
in an armed conflict against the United States''' there. Brief for
Respondents 3. We therefore answer only the narrow question before us:
whether the detention of citizens falling within that definition is
authorized.

The Government maintains that no explicit congressional authorization is
required, because the Executive possesses plenary authority to detain
pursuant to Article I of the Constitution.

We do not reach the question whether Article I provides such authority,
however, because we agree with the Government's alternative position,
that Congress has in fact authorized Hamdi's detention, through the
AUMF.

Our analysis on that point, set forth below, substantially overlaps with
our analysis of Hamdi's principal argument for the illegality of his
detention. He posits that his detention is forbidden by 18 U. S. C. §
4001(a). Section 4001(a) states that ``{[}n{]}o citizen shall be
imprisoned or otherwise detained by the United States except pursuant to
an Act of Congress.'' Congress passed § 4001(a) in 1971 as part of a
bill to repeal the Emergency Detention Act of 1950, 50 U. S. C. § 811 et
seq., which provided procedures for executive detention, during times of
emergency, of individuals deemed likely to engage in espionage or
sabotage. Congress was particularly concerned about the possibility that
the Act could be used to reprise the Japanese-American internment camps
of World War II. H. R. Rep.~No.~92-116 (1971); (``The concentration camp
implications of the legislation render it abhorrent''). The Government
again presses two alternative positions. First, it argues that §
4001(a), in light of its legislative history and its location in Title
18, applies only to ``the control of civilian prisons and related
detentions,'' not to military detentions. Brief for Respondents 21.
Second, it maintains that § 4001(a) is satisfied, because Hamdi is being
detained ``pursuant to an Act of Congress''---the AUMF. -22. Again,
because we conclude that the Government's second assertion is correct,
we do not address the first. In other words, for the reasons that
follow, we conclude that the AUMF is explicit congressional
authorization for the detention of individuals in the narrow category we
describe (assuming, without deciding, that such authorization is
required), and that the AUMF satisfied § 4001(a)'s requirement that a
detention be ``pursuant to an Act of Congress'' (assuming, without
deciding, that § 4001(a) applies to military detentions).

The AUMF authorizes the President to use ``all necessary and appropriate
force'' against ``nations, organizations, or persons'' associated with
the September 11, 2001, terrorist attacks. 115 Stat. 224. There can be
no doubt that individuals who fought against the United States in
Afghanistan as part of the Taliban, an organization known to have
supported the al Qaeda terrorist network responsible for those attacks,
are individuals Congress sought to target in passing the AUMF. We
conclude that detention of individuals falling into the limited category
we are considering, for the duration of the particular conflict in which
they were captured, is so fundamental and accepted an incident to war as
to be an exercise of the ``necessary and appropriate force'' Congress
has authorized the President to use.

The capture and detention of lawful combatants and the capture,
detention, and trial of unlawful combatants, by ``universal agreement
and practice,'' are ``important incident{[}s{]} of war.'' Ex parte
Quirin, 30. The purpose of detention is to prevent captured individuals
from returning to the field of battle and taking up arms once again.
Naqvi, Doubtful Prisoner-of-War Status, 84 Int'l Rev.~Red Cross 571, 572
(2002) (``{[}C{]}aptivity in war is'neither revenge, nor punishment, but
solely protective custody, the only purpose of which is to prevent the
prisoners of war from further participation in the war''' (quoting
decision of Nuremberg Military Tribunal, reprinted in 41 Am. J. Int'l L.
172, 229 (1947)); W. Winthrop, Military Law and Precedents 788 (rev. 2d
ed.~1920) (``The time has long passed when'no quarter' was the rule on
the battlefield. . It is now recognized that'Captivity is neither a
punishment nor an act of vengeance,' but'merely a temporary detention
which is devoid of all penal character.' . .'A prisoner of war is no
convict; his imprisonment is a simple war measure''' (citations
omitted)); cf.~In re Territo, CA9 1946) (``The object of capture is to
prevent the captured individual from serving the enemy. He is disarmed
and from then on must be removed as completely as practicable from the
front, treated humanely and in time exchanged, repatriated or otherwise
released'' (footnotes omitted)).

There is no bar to this Nation's holding one of its own citizens as an
enemy combatant. In Quirin, one of the detainees, Haupt, alleged that he
was a naturalized United States citizen. 317 U. S.. We held that
``{[}c{]}itizens who associate themselves with the military arm of the
enemy government, and with its aid, guidance and direction enter this
country bent on hostile acts, are enemy belligerents within the meaning
of \ldots{} the law of war.'' -38. While Haupt was tried for violations
of the law of war, nothing in Quirin suggests that his citizenship would
have precluded his mere detention for the duration of the relevant
hostilities. See -31. See also Lieber Code ¶ 153, Instructions for the
Government of Armies of the United States in the Field, Gen.~Order
No.~100 (1863), reprinted in 2 F. Lieber, Miscellaneous Writings,
p.~273, ¶ 153 (1880) (contemplating, in code binding the Union Army
during the Civil War, that ``captured rebels'' would be treated ``as
prisoners of war''). Nor can we see any reason for drawing such a line
here. A citizen, no less than an alien, can be ``part of or supporting
forces hostile to the United States or coalition partners'' and
``engaged in an armed conflict against the United States,'' Brief for
Respondents 3; such a citizen, if released, would pose the same threat
of returning to the front during the ongoing conflict.

In light of these principles, it is of no moment that the AUMF does not
use specific language of detention. Because detention to prevent a
combatant's return to the battlefield is a fundamental incident of
waging war, in permitting the use of ``necessary and appropriate
force,'' Congress has clearly and unmistakably authorized detention in
the narrow circumstances considered here.

Hamdi objects, nevertheless, that Congress has not authorized the
indefinite detention to which he is now subject. The Government responds
that ``the detention of enemy combatants during World War II was just
as'indefinite' while that war was being fought.'' We take Hamdi's
objection to be not to the lack of certainty regarding the date on which
the conflict will end, but to the substantial prospect of perpetual
detention. We recognize that the national security underpinnings of the
``war on terror,'' although crucially important, are broad and
malleable. As the Government concedes, ``given its unconventional
nature, the current conflict is unlikely to end with a formal cease-fire
agreement.'' The prospect Hamdi raises is therefore not farfetched. If
the Government does not consider this unconventional war won for two
generations, and if it maintains during that time that Hamdi might, if
released, rejoin forces fighting against the United States, then the
position it has taken throughout the litigation of this case suggests
that Hamdi's detention could last for the rest of his life.

It is a clearly established principle of the law of war that detention
may last no longer than active hostilities. See Article 118 of the
Geneva Convention (III) Relative to the Treatment of Prisoners of War,
Aug.~12, 1949, 6 U. S. T. 3316, 3406, T. I. A. S. No.~3364 (``Prisoners
of war shall be released and repatriated without delay after the
cessation of active hostilities''). See also Article 20 of the Hague
Convention (II) on Laws and Customs of War on Land, July 29, 1899, 32
Stat. 1817 (as soon as possible after ``conclusion of peace''); Hague
Convention (IV)Oct.~18, 1907, 36 Stat. 2301 (``conclusion of peace''
(Art. 20)); Geneva ConventionJuly 27, 1929, 47 Stat. 2055 (repatriation
should be accomplished with the least possible delay after conclusion of
peace (Art. 75)); Paust, Judicial Power to Determine the Status and
Rights of Persons Detained without Trial, 44 Harv. Int'l L. J. 503,
510-511 (2003) (prisoners of war ``can be detained during an armed
conflict, but the detaining country must release and repatriate
them'without delay after the cessation of active hostilities,' unless
they are being lawfully prosecuted or have been lawfully convicted of
crimes and are serving sentences'' (citing Arts. 118, 85, 99, 119, 129,
Geneva Convention (III), 6 U. S. T., 3392, 3406, 3418)).

Hamdi contends that the AUMF does not authorize indefinite or perpetual
detention. Certainly, we agree that indefinite detention for the purpose
of interrogation is not authorized. Further, we understand Congress'
grant of authority for the use of ``necessary and appropriate force'' to
include the authority to detain for the duration of the relevant
conflict, and our understanding is based on longstanding law-of-war
principles. If the practical circumstances of a given conflict are
entirely unlike those of the conflicts that informed the development of
the law of war, that understanding may unravel. But that is not the
situation we face as of this date. Active combat operations against
Taliban fighters apparently are ongoing in Afghanistan. See, e. g.,
Constable, U. S. Launches New Operation in Afghanistan, Washington Post,
Mar.~14, 2004, p.~A22 (reporting that 13,500 United States troops remain
in Afghanistan, including several thousand new arrivals); Dept. of
Defense, News Transcript, Gen.~J. Abizaid Central Command Operations
Update Briefing, Apr.~30, 2004,

http://www.defenselink.mil/transcripts/2004/tr20040430-1402.html (as
visited June 8, 2004, and available in Clerk of Court's case file)
(media briefing describing ongoing operations in Afghanistan involving
20,000 United States troops). The United States may detain, for the
duration of these hostilities, individuals legitimately determined to be
Taliban combatants who ``engaged in an armed conflict against the United
States.'' If the record establishes that United States troops are still
involved in active combat in Afghanistan, those detentions are part of
the exercise of ``necessary and appropriate force,'' and therefore are
authorized by the AUMF.

Ex parte Milligan, 4 Wall. 2, 125 (1866), does not undermine our holding
about the Government's authority to seize enemy combatants, as we define
that term today. In that case, the Court made repeated reference to the
fact that its inquiry into whether the military tribunal had
jurisdiction to try and punish Milligan turned in large part on the fact
that Milligan was not a prisoner of war, but a resident of Indiana
arrested while at home there. 131. That fact was central to its
conclusion. Had Milligan been captured while he was assisting
Confederate soldiers by carrying a rifle against Union troops on a
Confederate battlefield, the holding of the Court might well have been
different. The Court's repeated explanations that Milligan was not a
prisoner of war suggest that had these different circumstances been
present he could have been detained under military authority for the
duration of the conflict, whether or not he was a citizen.

Moreover, as JUSTICE SCALIA acknowledges, the Court in Ex parte Quirin,
317 U. S. 1 (1942), dismissed the language of Milligan that the
petitioners had suggested prevented them from being subject to military
process. Post (dissenting opinion). Clear in this rejection was a
disavowal of the New York State cases cited in Milligan, 4 Wall., on
which JUSTICE SCALIA relies. See Both Smith v. Shaw and M'Connell v.
Hampton were civil suits for false imprisonment. Even accepting that
these cases once could have been viewed as standing for the sweeping
proposition for which JUSTICE SCALIA cites them---that the military does
not have authority to try an American citizen accused of spying against
his country during wartime---Quirin makes undeniably clear that this is
not the law today.

Haupt, like the citizens in Smith and M'Connell, was accused of being a
spy. The Court in Quirin found him ``subject to trial and punishment by
{[}a{]} military tribuna{[}l{]}'' for those acts, and held that his
citizenship did not change this result. 317 U. S., 37-38.

Quirin was a unanimous opinion. It both postdates and clarifies
Milligan, providing us with the most apposite precedent that we have on
the question of whether citizens may be detained in such circumstances.
Brushing aside such precedent---particularly when doing so gives rise to
a host of new questions never dealt with by this Court---is unjustified
and unwise.

To the extent that JUSTICE SCALIA accepts the precedential value of
Quirin, he argues that it cannot guide our inquiry here because
``{[}i{]}n Quirin it was uncontested that the petitioners were members
of enemy forces,'' while Hamdi challenges his classification as an enemy
combatant. Post. But it is unclear why, in the paradigm outlined by
JUSTICE SCALIA, such a concession should have any relevance. JUSTICE
SCALIA envisions a system in which the only options are congressional
suspension of the writ of habeas corpus or prosecution for treason or
some other crime. Post. He does not explain how his historical analysis
supports the addition of a third option---detention under some other
process after concession of enemy-combatant status---or why a concession
should carry any different effect than proof of enemy-combatant status
in a proceeding that comports with due process. To be clear, our opinion
only finds legislative authority to detain under the AUMF once it is
sufficiently clear that the individual is, in fact, an enemy combatant;
whether that is established by concession or by some other process that
verifies this fact with sufficient certainty seems beside the point.

Further, JUSTICE SCALIA largely ignores the context of this case: a
United States citizen captured in a foreign combat zone. JUSTICE SCALIA
refers to only one case involving this factual scenario --- a case in
which a United States citizen-prisoner of war (a member of the Italian
army) from World War II was seized on the battlefield in Sicily and then
held in the United States. The court in that case held that the military
detention of that United States citizen was lawful. See In re Territo,
156 F. 2d.

JUSTICE SCALIA'S treatment of that case---in a footnote---suffers from
the same defect as does his treatment of Quirin: Because JUSTICE SCALIA
finds the fact of battlefield capture irrelevant, his distinction based
on the fact that the petitioner ``conceded'' enemy-combatant status is
beside the point. See. JUSTICE SCALIA can point to no case or other
authority for the proposition that those captured on a foreign
battlefield (whether detained there or in U. S. territory) cannot be
detained outside the criminal process.

Moreover, JUSTICE SCALIA presumably would come to a different result if
Hamdi had been kept in Afghanistan or even Guantanamo Bay. See post.
This creates a perverse incentive. Military authorities faced with the
stark choice of submitting to the full-blown criminal process or
releasing a suspected enemy combatant captured on the battlefield will
simply keep citizen-detainees abroad. Indeed, the Government transferred
Hamdi from Guantanamo Bay to the United States naval brig only after it
learned that he might be an American citizen. It is not at all clear why
that should make a determinative constitutional difference.

Even in cases in which the detention of enemy combatants is legally
authorized, there remains the question of what process is
constitutionally due to a citizen who disputes his enemy-combatant
status. Hamdi argues that he is owed a meaningful and timely hearing and
that ``extra-judicial detention {[}that{]} begins and ends with the
submission of an affidavit based on third-hand hearsay'' does not
comport with the Fifth and Fourteenth Amendments. Brief for Petitioners
16. The Government counters that any more process than was provided
below would be both unworkable and ``constitutionally intolerable.''
Brief for Respondents 46. Our resolution of this dispute requires a
careful examination both of the writ of habeas corpus, which Hamdi now
seeks to employ as a mechanism of judicial review, and of the Due
Process Clause, which informs the procedural contours of that mechanism
in this instance.

Though they reach radically different conclusions on the process that
ought to attend the present proceeding, the parties begin on common
ground. All agree that, absent suspension, the writ of habeas corpus
remains available to every individual detained within the United States.
U. S. Const., Art. I, § 9, cl. 2 (``The Privilege of the Writ of Habeas
Corpus shall not be suspended, unless when in Cases of Rebellion or
Invasion the public Safety may require it''). Only in the rarest of
circumstances has Congress seen fit to suspend the writ. See, e. g., Act
of Mar.~3, 1863, ch.~81, § 1, 12 Stat. 755; Act of Apr.~20, 1871,
ch.~22, § 4, 17 Stat. 14. At all other times, it has remained a critical
check on the Executive, ensuring that it does not detain individuals
except in accordance with law. See INS v. St.~Cyr. All agree suspension
of the writ has not occurred here. Thus, it is undisputed that Hamdi was
properly before an Article III court to challenge his detention under 28
U. S. C. § 2241. Brief for Respondents 12. Further, all agree that §
2241 and its companion provisions provide at least a skeletal outline of
the procedures to be afforded a petitioner in federal habeas review.
Most notably, § 2243 provides that ``the person detained may, under
oath, deny any of the facts set forth in the return or allege any other
material facts,'' and § 2246 allows the taking of evidence in habeas
proceedings by deposition, affidavit, or interrogatories.

The simple outline of § 2241 makes clear both that Congress envisioned
that habeas petitioners would have some opportunity to present and rebut
facts and that courts in cases like this retain some ability to vary the
ways in which they do so as mandated by due process. The Government
recognizes the basic procedural protections required by the habeas
statute, but asks us to hold that, given both the flexibility of the
habeas mechanism and the circumstances presented in this case, the
presentation of the Mobbs Declaration to the habeas court completed the
required factual development. It suggests two separate reasons for its
position that no further process is due.

First, the Government urges the adoption of the Fourth Circuit's holding
below---that because it is ``undisputed'' that Hamdi's seizure took
place in a combat zone, the habeas determination can be made purely as a
matter of law, with no further hearing or factfinding necessary. This
argument is easily rejected. As the dissenters from the denial of
rehearing en banc noted, the circumstances surrounding Hamdi's seizure
cannot in any way be characterized as ``undisputed,'' as ``those
circumstances are neither conceded in fact, nor susceptible to
concession in law, because Hamdi has not been permitted to speak for
himself or even through counsel as to those circumstances.'' 337 F. 3d
(opinion of Luttig, J.); see also -372 (opinion of Motz, J.). Further,
the ``facts'' that constitute the alleged concession are insufficient to
support Hamdi's detention. Under the definition of enemy combatant that
we accept today as falling within the scope of Congress' authorization,
Hamdi would need to be ``part of or supporting forces hostile to the
United States or coalition partners'' and ``engaged in an armed conflict
against the United States'' to justify his detention in the United
States for the duration of the relevant conflict. Brief for Respondents
3. The habeas petition states only that ``{[}w{]}hen seized by the
United States Government, Mr.~Hamdi resided in Afghanistan.'' App. 104.
An assertion that one resided in a country in which combat operations
are taking place is not a concession that one was ``captured in a zone
of active combat'' operations in a foreign theater of war, 316 F. 3d
(emphasis added), and certainly is not a concession that one was ``part
of or supporting forces hostile to the United States or coalition
partners'' and ``engaged in an armed conflict against the United
States.'' Accordingly, we reject any argument that Hamdi has made
concessions that eliminate any right to further process.

The Government's second argument requires closer consideration. This is
the argument that further factual exploration is unwarranted and
inappropriate in light of the extraordinary constitutional interests at
stake. Under the Government's most extreme rendition of this argument,
``{[}r{]}espect for separation of powers and the limited institutional
capabilities of courts in matters of military decision-making in
connection with an ongoing conflict'' ought to eliminate entirely any
individual process, restricting the courts to investigating only whether
legal authorization exists for the broader detention scheme. At most,
the Government argues, courts should review its determination that a
citizen is an enemy combatant under a very deferential ``some evidence''
standard. (``Under the some evidence standard, the focus is exclusively
on the factual basis supplied by the Executive to support its own
determination'' (citing Superintendent, Mass. Correctional Institution
at Walpole v. Hill (1985) (explaining that the some evidence standard
``does not require'' a ``weighing of the evidence,'' but rather calls
for assessing ``whether there is any evidence in the record that could
support the conclusion''))). Under this review, a court would assume the
accuracy of the Government's articulated basis for Hamdi's detention, as
set forth in the Mobbs Declaration, and assess only whether that
articulated basis was a legitimate one. Brief for Respondents 36; see
also 316 F. 3d (declining to address whether the ``some evidence''
standard should govern the adjudication of such claims, but noting that
``{[}t{]}he factual averments in the {[}Mobbs{]} affidavit, if accurate,
are sufficient to confirm'' the legality of Hamdi's detention).

In response, Hamdi emphasizes that this Court consistently has
recognized that an individual challenging his detention may not be held
at the will of the Executive without recourse to some proceeding before
a neutral tribunal to determine whether the Executive's asserted
justifications for that detention have basis in fact and warrant in law.
See, e. g., Zadvydas v. Davis; Addington v. Texas (1979). He argues that
the Fourth Circuit inappropriately ``ceded power to the Executive during
wartime to define the conduct for which a citizen may be detained, judge
whether that citizen has engaged in the proscribed conduct, and imprison
that citizen indefinitely,'' Brief for Petitioners 21, and that due
process demands that he receive a hearing in which he may challenge the
Mobbs Declaration and adduce his own counter evidence. The District
Court, agreeing with Hamdi, apparently believed that the appropriate
process would approach the process that accompanies a criminal trial. It
therefore disapproved of the hearsay nature of the Mobbs Declaration and
anticipated quite extensive discovery of various military affairs.
Anything less, it concluded, would not be ``meaningful judicial
review.''

Both of these positions highlight legitimate concerns. And both
emphasize the tension that often exists between the autonomy that the
Government asserts is necessary in order to pursue effectively a
particular goal and the process that a citizen contends he is due before
he is deprived of a constitutional right. The ordinary mechanism that we
use for balancing such serious competing interests, and for determining
the procedures that are necessary to ensure that a citizen is not
``deprived of life, liberty, or property, without due process of law,''
U. S. Const., Amdt. 5, is the test that we articulated in Mathews v.
Eldridge. See, e. g., Heller v. Doe (1993); Zinermon v. Burch (1990);
United States v. Salerno; Schall v. Martin (1984); Addington v. Texas.
Mathews dictates that the process due in any given instance is
determined by weighing ``the private interest that will be affected by
the official action'' against the Government's asserted interest,
``including the function involved'' and the burdens the Government would
face in providing greater process. 424 U. S.. The Mathews calculus then
contemplates a judicious balancing of these concerns, through an
analysis of ``the risk of an erroneous deprivation'' of the private
interest if the process were reduced and the ``probable value, if any,
of additional or substitute procedural safeguards.'' We take each of
these steps in turn.

It is beyond question that substantial interests lie on both sides of
the scale in this case. Hamdi's ``private interest \ldots{} affected by
the official action,'' , is the most elemental of liberty
interests---the interest in being free from physical detention by one's
own government. Foucha v. Louisiana (``Freedom from bodily restraint has
always been at the core of the liberty protected by the Due Process
Clause from arbitrary governmental action''); see also Parham v. J. R.
(noting the ``substantial liberty interest in not being confined
unnecessarily''). ``In our society liberty is the norm,'' and detention
without trial ``is the carefully limited exception.'' Salerno. ``We have
always been careful not to'minimize the importance and fundamental
nature' of the individual's right to liberty,'' Foucha (quoting
Salerno), and we will not do so today.

Nor is the weight on this side of the Mathews scale offset by the
circumstances of war or the accusation of treasonous behavior, for
``{[}i{]}t is clear that commitment for any purpose constitutes a
significant deprivation of liberty that requires due process
protection,'' Jones v. United States (emphasis added; internal quotation
marks omitted), and at this stage in the Mathews calculus, we consider
the interest of the erroneously detained individual. Carey v. Piphus.

(``Procedural due process rules are meant to protect persons not from
the deprivation, but from the mistaken or unjustified deprivation of
life, liberty, or property''); see also {[}cite{]} (noting ``the
importance to organized society that procedural due process be
observed,'' and emphasizing that ``the right to procedural due process
is'absolute' in the sense that it does not depend upon the merits of a
claimant's substantive assertions''). Indeed, as amicus briefs from
media and relief organizations emphasize, the risk of erroneous
deprivation of a citizen's liberty in the absence of sufficient process
here is very real. See Brief for AmeriCares et al.~as Amici Curiae 13-22
(noting ways in which ``{[}t{]}he nature of humanitarian relief work and
journalism present a significant risk of mistaken military
detentions''). Moreover, as critical as the Government's interest may be
in detaining those who actually pose an immediate threat to the national
security of the United States during ongoing international conflict,
history and common sense teach us that an unchecked system of detention
carries the potential to become a means for oppression and abuse of
others who do not present that sort of threat. See Ex parte Milligan, 4
Wall. (``{[}The Founders{]} knew---the history of the world told
them---the nation they were founding, be its existence short or long,
would be involved in war; how often or how long continued, human
foresight could not tell; and that unlimited power, wherever lodged at
such a time, was especially hazardous to freemen''). Because we live in
a society in which ``{[}m{]}ere public intolerance or animosity cannot
constitutionally justify the deprivation of a person's physical
liberty,'' O'Connor v. Donaldson, our starting point for the Mathews v.
Eldridge analysis is unaltered by the allegations surrounding the
particular detainee or the organizations with which he is alleged to
have associated. We reaffirm today the fundamental nature of a citizen's
right to be free from involuntary confinement by his own government
without due process of law, and we weigh the opposing governmental
interests against the curtailment of liberty that such confinement
entails.

On the other side of the scale are the weighty and sensitive
governmental interests in ensuring that those who have in fact fought
with the enemy during a war do not return to battle against the United
States. As discussed above, the law of war and the realities of combat
may render such detentions both necessary and appropriate, and our due
process analysis need not blink at those realities. Without doubt, our
Constitution recognizes that core strategic matters of warmaking belong
in the hands of those who are best positioned and most politically
accountable for making them. Department of Navy v. Egan (noting the
reluctance of the courts ``to intrude upon the authority of the
Executive in military and national security affairs''); Youngstown Sheet
\& Tube Co.~v. Sawyer (acknowledging ``broad powers in military
commanders engaged in day-to-day fighting in a theater of war'').

The Government also argues at some length that its interests in reducing
the process available to alleged enemy combatants are heightened by the
practical difficulties that would accompany a system of trial-like
process. In its view, military officers who are engaged in the serious
work of waging battle would be unnecessarily and dangerously distracted
by litigation half a world away, and discovery into military operations
would both intrude on the sensitive secrets of national defense and
result in a futile search for evidence buried under the rubble of war.
Brief for Respondents 46-49. To the extent that these burdens are
triggered by heightened procedures, they are properly taken into account
in our due process analysis.

Striking the proper constitutional balance here is of great importance
to the Nation during this period of ongoing combat. But it is equally
vital that our calculus not give short shrift to the values that this
country holds dear or to the privilege that is American citizenship. It
is during our most challenging and uncertain moments that our Nation's
commitment to due process is most severely tested; and it is in those
times that we must preserve our commitment at home to the principles for
which we fight abroad. See Kennedy v. Mendoza-Martinez (1963) (``The
imperative necessity for safeguarding these rights to procedural due
process under the gravest of emergencies has existed throughout our
constitutional history, for it is then, under the pressing exigencies of
crisis, that there is the greatest temptation to dispense with
fundamental constitutional guarantees which, it is feared, will inhibit
governmental action''); see also United States v. Robel (``It would
indeed be ironic if, in the name of national defense, we would sanction
the subversion of one of those liberties \ldots{} which makes the
defense of the Nation worthwhile'').

With due recognition of these competing concerns, we believe that
neither the process proposed by the Government nor the process
apparently envisioned by the District Court below strikes the proper
constitutional balance when a United States citizen is detained in the
United States as an enemy combatant. That is, ``the risk of an erroneous
deprivation'' of a detainee's liberty interest is unacceptably high
under the Government's proposed rule, while some of the ``additional or
substitute procedural safeguards'' suggested by the District Court are
unwarranted in light of their limited ``probable value'' and the burdens
they may impose on the military in such cases. Mathews.

We therefore hold that a citizen-detainee seeking to challenge his
classification as an enemy combatant must receive notice of the factual
basis for his classification, and a fair opportunity to rebut the
Government's factual assertions before a neutral decisionmaker. See
Cleveland Bd. of Ed. v. Loudermill (``An essential principle of due
process is that a deprivation of life, liberty, or property'be preceded
by notice and opportunity for hearing appropriate to the nature of the
case''' (quoting Mullane v. Central Hanover Bank \& Trust Co.); Concrete
Pipe \& Products of Cal., Inc.~v. Construction Laborers Pension Trust
for Southern Cal. (``due process requires a'neutral and detached judge
in the first instance''' (quoting Ward v. Monroeville (1972))). ``For
more than a century the central meaning of procedural due process has
been clear:`Parties whose rights are to be affected are entitled to be
heard; and in order that they may enjoy that right they must first be
notified.' It is equally fundamental that the right to notice and an
opportunity to be heard'must be granted at a meaningful time and in a
meaningful manner.''' Fuentes v. Shevin (quoting Baldwin v. Hale, 1
Wall. 223, 233 (1864); Armstrong v. Manzo (other citations omitted)).
These essential constitutional promises may not be eroded.

At the same time, the exigencies of the circumstances may demand that,
aside from these core elements, enemy-combatant proceedings may be
tailored to alleviate their uncommon potential to burden the Executive
at a time of ongoing military conflict. Hearsay, for example, may need
to be accepted as the most reliable available evidence from the
Government in such a proceeding. Likewise, the Constitution would not be
offended by a presumption in favor of the Government's evidence, so long
as that presumption remained a rebuttable one and fair opportunity for
rebuttal were provided. Thus, once the Government puts forth credible
evidence that the habeas petitioner meets the enemy-combatant criteria,
the onus could shift to the petitioner to rebut that evidence with more
persuasive evidence that he falls outside the criteria. A
burden-shifting scheme of this sort would meet the goal of ensuring that
the errant tourist, embedded journalist, or local aid worker has a
chance to prove military error while giving due regard to the Executive
once it has put forth meaningful support for its conclusion that the
detainee is in fact an enemy combatant. In the words of Mathews, process
of this sort would sufficiently address the ``risk of an erroneous
deprivation'' of a detainee's liberty interest while eliminating certain
procedures that have questionable additional value in light of the
burden on the Government. 424 U. S..

We think it unlikely that this basic process will have the dire impact
on the central functions of warmaking that the Government forecasts. The
parties agree that initial captures on the battlefield need not receive
the process we have discussed here; that process is due only when the
determination is made to continue to hold those who have been seized.
The Government has made clear in its briefing that documentation
regarding battlefield detainees already is kept in the ordinary course
of military affairs. Brief for Respondents 3-4. Any factfinding
imposition created by requiring a knowledgeable affiant to summarize
these records to an independent tribunal is a minimal one. Likewise,
arguments that military officers ought not have to wage war under the
threat of litigation lose much of their steam when factual disputes at
enemy-combatant hearings are limited to the alleged combatant's acts.
This focus meddles little, if at all, in the strategy or conduct of war,
inquiring only into the appropriateness of continuing to detain an
individual claimed to have taken up arms against the United States.
While we accord the greatest respect and consideration to the judgments
of military authorities in matters relating to the actual prosecution of
a war, and recognize that the scope of that discretion necessarily is
wide, it does not infringe on the core role of the military for the
courts to exercise their own time-honored and constitutionally mandated
roles of reviewing and resolving claims like those presented here. Cf.
Korematsu v. United States (1944) (Murphy, J., dissenting) (``{[}L{]}ike
other claims conflicting with the asserted constitutional rights of the
individual, the military claim must subject itself to the judicial
process of having its reasonableness determined and its conflicts with
other interests reconciled''); Sterling v. Constantin (``What are the
allowable limits of military discretion, and whether or not they have
been overstepped in a particular case, are judicial questions'').

In sum, while the full protections that accompany challenges to
detentions in other settings may prove unworkable and inappropriate in
the enemy-combatant setting, the threats to military operations posed by
a basic system of independent review are not so weighty as to trump a
citizen's core rights to challenge meaningfully the Government's case
and to be heard by an impartial adjudicator.

In so holding, we necessarily reject the Government's assertion that
separation of powers principles mandate a heavily circumscribed role for
the courts in such circumstances. Indeed, the position that the courts
must forgo any examination of the individual case and focus exclusively
on the legality of the broader detention scheme cannot be mandated by
any reasonable view of separation of powers, as this approach serves
only to condense power into a single branch of government. We have long
since made clear that a state of war is not a blank check for the
President when it comes to the rights of the Nation's citizens.
Youngstown Sheet \& Tube. Whatever power the United States Constitution
envisions for the Executive in its exchanges with other nations or with
enemy organizations in times of conflict, it most assuredly envisions a
role for all three branches when individual liberties are at stake.
Mistretta v. United States (it was ``the central judgment of the Framers
of the Constitution that, within our political scheme, the separation of
governmental powers into three coordinate Branches is essential to the
preservation of liberty''); Home Building \& Loan Assn. v. Blaisdell
(The war power ``is a power to wage war successfully, and thus it
permits the harnessing of the entire energies of the people in a supreme
cooperative effort to preserve the nation. But even the war power does
not remove constitutional limitations safeguarding essential
liberties''). Likewise, we have made clear that, unless Congress acts to
suspend it, the Great Writ of habeas corpus allows the Judicial Branch
to play a necessary role in maintaining this delicate balance of
governance, serving as an important judicial check on the Executive's
discretion in the realm of detentions. See St.~Cyr (``At its historical
core, the writ of habeas corpus has served as a means of reviewing the
legality of Executive detention, and it is in that context that its
protections have been strongest''). Thus, while we do not question that
our due process assessment must pay keen attention to the particular
burdens faced by the Executive in the context of military action, it
would turn our system of checks and balances on its head to suggest that
a citizen could not make his way to court with a challenge to the
factual basis for his detention by his government, simply because the
Executive opposes making available such a challenge. Absent suspension
of the writ by Congress, a citizen detained as an enemy combatant is
entitled to this process.

Because we conclude that due process demands some system for a citizen
detainee to refute his classification, the proposed ``some evidence''
standard is inadequate. Any process in which the Executive's factual
assertions go wholly unchallenged or are simply presumed correct without
any opportunity for the alleged combatant to demonstrate otherwise falls
constitutionally short. As the Government itself has recognized, we have
utilized the ``some evidence'' standard in the past as a standard of
review, not as a standard of proof. Brief for Respondents 35. That is,
it primarily has been employed by courts in examining an administrative
record developed after an adversarial proceeding---one with process at
least of the sort that we today hold is constitutionally mandated in the
citizen enemy-combatant setting. See, e. g., St.~Cyr; Hill. This
standard therefore is ill suited to the situation in which a habeas
petitioner has received no prior proceedings before any tribunal and had
no prior opportunity to rebut the Executive's factual assertions before
a neutral decisionmaker.

Today we are faced only with such a case. Aside from unspecified
``screening'' processes, Brief for Respondents 3-4, and military
interrogations in which the Government suggests Hamdi could have
contested his classification, Tr. of Oral Arg. 40, 42, Hamdi has
received no process. An interrogation by one's captor, however effective
an intelligence-gathering tool, hardly constitutes a constitutionally
adequate factfinding before a neutral decisionmaker. Compare Brief for
Respondents 42-43 (discussing the ``secure interrogation environment,''
and noting that military interrogations require a controlled
``interrogation dynamic'' and ``a relationship of trust and dependency''
and are ``a critical source'' of ``timely and effective intelligence'')
with Concrete Pipe (``{[}O{]}ne is entitled as a matter of due process
of law to an adjudicator who is not in a situation which would offer a
possible temptation to the average man as a judge \ldots{} which might
lead him not to hold the balance nice, clear and true'' (internal
quotation marks omitted)). That even purportedly fair adjudicators ``are
disqualified by their interest in the controversy to be decided is, of
course, the general rule.'' Tumey v. Ohio. Plainly, the ``process''
Hamdi has received is not that to which he is entitled under the Due
Process Clause.

There remains the possibility that the standards we have articulated
could be met by an appropriately authorized and properly constituted
military tribunal. Indeed, it is notable that military regulations
already provide for such process in related instances, dictating that
tribunals be made available to determine the status of enemy detainees
who assert prisoner-of-war status under the Geneva Convention. See
Headquarters Depts. of Army, Navy, Air Force, and Marine Corps, Enemy
Prisoners of War, Retained Personnel, Civilian Internees and Other
Detainees, Army Regulation 190-8, ch.~1, § 1-6 (1997). In the absence of
such process, however, a court that receives a petition for a writ of
habeas corpus from an alleged enemy combatant must itself ensure that
the minimum requirements of due process are achieved. Both courts below
recognized as much, focusing their energies on the question of whether
Hamdi was due an opportunity to rebut the Government's case against him.
The Government, too, proceeded on this assumption, presenting its
affidavit and then seeking that it be evaluated under a deferential
standard of review based on burdens that it alleged would accompany any
greater process. As we have discussed, a habeas court in a case such as
this may accept affidavit evidence like that contained in the Mobbs
Declaration, so long as it also permits the alleged combatant to present
his own factual case to rebut the Government's return. We anticipate
that a District Court would proceed with the caution that we have
indicated is necessary in this setting, engaging in a factfinding
process that is both prudent and incremental. We have no reason to doubt
that courts faced with these sensitive matters will pay proper heed both
to the matters of national security that might arise in an individual
case and to the constitutional limitations safeguarding essential
liberties that remain vibrant even in times of security concerns.

Hamdi asks us to hold that the Fourth Circuit also erred by denying him
immediate access to counsel upon his detention and by disposing of the
case without permitting him to meet with an attorney. Brief for
Petitioners 19. Since our grant of certiorari in this case, Hamdi has
been appointed counsel, with whom he has met for consultation purposes
on several occasions, and with whom he is now being granted unmonitored
meetings. He unquestionably has the right to access to counsel in
connection with the proceedings on remand. No further consideration of
this issue is necessary at this stage of the case.

The judgment of the United States Court of Appeals for the Fourth
Circuit is vacated, and the case is remanded for further proceedings.

It is so ordered.

\textbf{JUSTICE SCALIA, with whom JUSTICE STEVENS joins, dissenting.}

Petitioner Yaser Hamdi, a presumed American citizen, has been imprisoned
without charge or hearing in the Norfolk and Charleston Naval Brigs for
more than two years, on the allegation that he is an enemy combatant who
bore arms against his country for the Taliban. His father claims to the
contrary, that he is an inexperienced aid worker caught in the wrong
place at the wrong time. This case brings into conflict the competing
demands of national security and our citizens' constitutional right to
personal liberty. Although I share the plurality's evident unease as it
seeks to reconcile the two, I do not agree with its resolution.

Where the Government accuses a citizen of waging war against it, our
constitutional tradition has been to prosecute him in federal court for
treason or some other crime. Where the exigencies of war prevent that,
the Constitution's Suspension Clause, Art. I, § 9, cl. 2, allows
Congress to relax the usual protections temporarily. Absent suspension,
however, the Executive's assertion of military exigency has not been
thought sufficient to permit detention without charge. No one contends
that the congressional Authorization for Use of Military Force, on which
the Government relies to justify its actions here, is an implementation
of the Suspension Clause. Accordingly, I would reverse the judgment
below.

The very core of liberty secured by our Anglo-Saxon system of separated
powers has been freedom from indefinite imprisonment at the will of the
Executive. Blackstone stated this principle clearly:

"Of great importance to the public is the preservation of this personal
liberty: for if once it were left in the power of any, the highest,
magistrate to imprison arbitrarily whomever he or his officers thought
proper . there would soon be an end of all other rights and immunities.
. To bereave a man of life, or by violence to confiscate his estate,
without accusation or trial, would be so gross and notorious an act of
despotism, as must at once convey the alarm of tyranny throughout the
whole kingdom. But confinement of the person, by secretly hurrying him
to gaol, where his sufferings are unknown or forgotten; is a less
public, a less striking, and therefore a more dangerous engine of
arbitrary government. .

``To make imprisonment lawful, it must either be, by process from the
courts of judicature, or by warrant from some legal officer, having
authority to commit to prison; which warrant must be in writing, under
the hand and seal of the magistrate, and express the causes of the
commitment, in order to be examined into (if necessary) upon a habeas
corpus. If there be no cause expressed, the gaoler is not bound to
detain the prisoner. For the law judges in this respect, . that it is
unreasonable to send a prisoner, and not to signify withal the crimes
alleged against him.'' 1 W. Blackstone, Commentaries on the Laws of
England 131-133 (1765) (hereinafter Blackstone).

These words were well known to the Founders. Hamilton quoted from this
very passage in The Federalist No.~84, p.~444 (G. Carey \& J. McClellan
eds.~2001). The two ideas central to Blackstone's understanding --- due
process as the right secured, and habeas corpus as the instrument by
which due process could be insisted upon by a citizen illegally
imprisoned ---found expression in the Constitution's Due Process and
Suspension Clauses. See Amdt. 5; Art. I, § 9, cl. 2.

The gist of the Due Process Clause, as understood at the founding and
since, was to force the Government to follow those common-law procedures
traditionally deemed necessary before depriving a person of life,
liberty, or property. When a citizen was deprived of liberty because of
alleged criminal conduct, those procedures typically required committal
by a magistrate followed by indictment and trial. See, e. g., 2 \& 3
Philip \& Mary, ch.~10 (1555); 3 J. Story, Commentaries on the
Constitution of the United States § 1783, p.~661 (1833) (hereinafter
Story) (equating ``due process of law'' with ``due presentment or
indictment, and being brought in to answer thereto by due process of the
common law''). The Due Process Clause ``in effect affirms the right of
trial according to the process and proceedings of the common law.'' See
also T. Cooley, General Principles of Constitutional Law 224 (1880)
(``When life and liberty are in question, there must in every instance
be judicial proceedings; and that requirement implies an accusation, a
hearing before an impartial tribunal, with proper jurisdiction, and a
conviction and judgment before the punishment can be inflicted''
(internal quotation marks omitted)).

To be sure, certain types of permissible noncriminal detention --- that
is, those not dependent upon the contention that the citizen had
committed a criminal act --- did not require the protections of criminal
procedure. However, these fell into a limited number of well-recognized
exceptions --- civil commitment of the mentally ill, for example, and
temporary detention in quarantine of the infectious. See Opinion on the
Writ of Habeas Corpus, Wilm. 77, 88Eng. Rep.~29, 36-37 (H. L. 1758)
(Wilmot, J.). It is unthinkable that the Executive could render
otherwise criminal grounds for detention noncriminal merely by
disclaiming an intent to prosecute, or by asserting that it was
incapacitating dangerous offenders rather than punishing wrongdoing.

Cf. Kansas v. Hendricks (``A finding of dangerousness, standing alone,
is ordinarily not a sufficient ground upon which to justify indefinite
involuntary commitment'').

These due process rights have historically been vindicated by the writ
of habeas corpus. In England before the founding, the writ developed
into a tool for challenging executive confinement. It was not always
effective. For example, in Darnel's Case, 3 How. St.~Tr. 1 (K. B. 1627),
King Charles I detained without charge several individuals for failing
to assist England's war against France and Spain. The prisoners sought
writs of habeas corpus, arguing that without specific charges,
``imprisonment shall not continue on for a time, but for ever; and the
subjects of this kingdom may be restrained of their liberties
perpetually.'' The Attorney General replied that the Crown's interest in
protecting the realm justified imprisonment in ``a matter of state . not
ripe nor timely'' for the ordinary process of accusation and trial. The
court denied relief, producing widespread outrage, and Parliament
responded with the Petition of Right, accepted by the King in 1628,
which expressly prohibited imprisonment without formal charges, see 3
Car. 1, ch.~1, §§ 5, 10.

The struggle between subject and Crown continued, and culminated in the
Habeas Corpus Act of 1679, 31 Car. 2, ch.~2, described by Blackstone as
a ``second magna carta, and stable bulwark of our liberties.'' 1
Blackstone 133. The Act governed all persons ``committed or detained .
for any crime.'' § 3. In cases other than felony or treason plainly
expressed in the warrant of commitment, the Act required release upon
appropriate sureties (unless the commitment was for a nonbailable
offense). Where the commitment was for felony or high treason, the Act
did not require immediate release, but instead required the Crown to
commence criminal proceedings within a specified time. § 7. If the
prisoner was not ``indicted some Time in the next Term,'' the judge was
``required . to set at Liberty the Prisoner upon Bail'' unless the King
was unable to produce his witnesses. Able or no, if the prisoner was not
brought to trial by the next succeeding term, the Act provided that ``he
shall be discharged from his Imprisonment.'' English courts sat four
terms per year, see 3 Blackstone 275-277, so the practical effect of
this provision was that imprisonment without indictment or trial for
felony or high treason under § 7 would not exceed approximately three to
six months.

The writ of habeas corpus was preserved in the Constitution --- the only
common-law writ to be explicitly mentioned. See Art. I, § 9, cl. 2.
Hamilton lauded ``the establishment of the writ of habeas corpus'' in
his Federalist defense as a means to protect against ``the practice of
arbitrary imprisonments . in all ages, {[}one of{]} the favourite and
most formidable instruments of tyranny.'' The Federalist No.~84. Indeed,
availability of the writ under the new Constitution (along with the
requirement of trial by jury in criminal cases, see Art. III, § 2, cl.
3) was his basis for arguing that additional, explicit procedural
protections were unnecessary. See The Federalist No.~83.

The allegations here, of course, are no ordinary accusations of criminal
activity. Yaser Esam Hamdi has been imprisoned because the Government
believes he participated in the waging of war against the United States.
The relevant question, then, is whether there is a different, special
procedure for imprisonment of a citizen accused of wrongdoing by aiding
the enemy in wartime. A

JUSTICE O'CONNOR, writing for a plurality of this Court, asserts that
captured enemy combatants (other than those suspected of war crimes)
have traditionally been detained until the cessation of hostilities and
then released. That is probably an accurate description of wartime
practice with respect to enemy aliens. The tradition with respect to
American citizens, however, has been quite different. Citizens aiding
the enemy have been treated as traitors subject to the criminal process.

As early as 1350, England's Statute of Treasons made it a crime to
``levy War against our Lord the King in his Realm, or be adherent to the
King's Enemies in his Realm, giving to them Aid and Comfort, in the
Realm, or elsewhere.'' 25 Edw. 3, Stat. 5, c.~2. In his 1762 Discourse
on High Treason, Sir Michael Foster explained:

"With regard to Natural-born Subjects there can be no Doubt. They owe
Allegiance to the Crown at all Times and in all Places.

. .

"The joining with Rebels in an Act of Rebellion, or with Enemies in Acts
of Hostility, will make a Man a Traitor: in the one Case within the
Clause of Levying War, in the other within that of Adhering to the
King's enemies.

. .

``States in Actual Hostility with Us, though no War be solemnly
Declared, are Enemies within the meaning of the Act. And therefore in an
Indictment on the Clause of Adhering to the King's Enemies, it is
sufficient to Aver that the Prince or State Adhered to is an Enemy,
without shewing any War Proclaimed. . And if the Subject of a Foreign
Prince in Amity with Us, invadeth the Kingdom without Commission from
his Sovereign, He is an Enemy. And a Subject of England adhering to Him
is a Traitor within this Clause of the Act.'' A Report of Some
Proceedings on the Commission . for the Trial of the Rebels in the Year
1746 in the County of Surry, and of Other Crown Cases, Introduction, §
1, p.~183; Ch. 2, § 8, p.~216; § 12, p.~219.

Subjects accused of levying war against the King were routinely
prosecuted for treason. E.g., Harding's Case, 2 Ventris 315, 86 Eng.
Rep.~461 (K. B. 1690); Trial of Parkyns, 13 How. St.~Tr. 63 (K. B.
1696); Trial of Vaughan, 13 How. St.~Tr. 485 (K. B. 1696); Trial of
Downie, 24 How. St.~Tr. 1 (1794). The Founders inherited the
understanding that a citizen's levying war against the Government was to
be punished criminally. The Constitution provides: ``Treason against the
United States, shall consist only in levying War against them, or in
adhering to their Enemies, giving them Aid and Comfort''; and
establishes a heightened proof requirement (two witnesses) in order to
``convic{[}t{]}'' of that offense. Art. III, § 3, cl. 1.

In more recent times, too, citizens have been charged and tried in
Article III courts for acts of war against the United States, even when
their noncitizen co-conspirators were not. For example, two American
citizens alleged to have participated during World War II in a spying
conspiracy on behalf of Germany were tried in federal court. See United
States v. Fricke, 259 F. 673 (SDNY 1919); United States v. Robinson, 259
F. 685 (SDNY 1919). A German member of the same conspiracy was subjected
to military process. See United States ex rel. Wessels v. McDonald, 265
F. 754 (EDNY 1920). During World War III, the famous German saboteurs of
Ex parte Quirin, 317 U. S. 1 (1942), received military process, but the
citizens who associated with them (with the exception of one
citizen-saboteur, discussed below) were punished under the criminal
process. See Haupt v. United States; L. Fisher, Nazi Saboteurs on Trial
80-84 (2003); see also Cramer v. United States, 325 U. S. 1 (1945).

The modern treason statute is 18 U. S. C. § 2381; it basically tracks
the language of the constitutional provision. Other provisions of Title
18 criminalize various acts of warmaking and adherence to the enemy.
See, e.g., § 32 (destruction of aircraft or aircraft facilities), §
2332a (use of weapons of mass destruction), § 2332b (acts of terrorism
transcending national boundaries), § 2339A (providing material support
to terrorists), § 2339B (providing material support to certain terrorist
organizations), § 2382 (misprision of treason), § 2383 (rebellion or
insurrection), § 2384 (seditious conspiracy), § 2390 (enlistment to
serve in armed hostility against the United States). See also 31 CFR §
595 (2003) (prohibiting the ``making or receiving of any contribution of
funds, goods, or services'' to terrorists); 50 U. S. C. § 1705(b)
(criminalizing violations of 31 CFR § 595 ). The only citizen other than
Hamdi known to be imprisoned in connection with military hostilities in
Afghanistan against the United States was subjected to criminal process
and convicted upon a guilty plea. See United States v. Lindh, 212 F.
Supp. 2d 541 (ED Va. 2002) (denying motions for dismissal); Seelye, N.
Y. Times, Oct.~5, 2002, p.~A1, col.~5.

There are times when military exigency renders resort to the traditional
criminal process impracticable. English law accommodated such exigencies
by allowing legislative suspension of the writ of habeas corpus for
brief periods. Blackstone explained:

``And yet sometimes, when the state is in real danger, even this {[}i.
e., executive detention{]} may be a necessary measure. But the happiness
of our constitution is, that it is not left to the executive power to
determine when the danger of the state is so great, as to render this
measure expedient. For the parliament only, or legislative power,
whenever it sees proper, can authorize the crown, by suspending the
habeas corpus act for a short and limited time, to imprison suspected
persons without giving any reason for so doing. . In like manner this
experiment ought only to be tried in cases of extreme emergency; and in
these the nation parts with it{[}s{]} liberty for a while, in order to
preserve it for ever.'' 1 Blackstone 132.

Where the Executive has not pursued the usual course of charge,
committal, and conviction, it has historically secured the Legislature's
explicit approval of a suspension. In England, Parliament on numerous
occasions passed temporary suspensions in times of threatened invasion
or rebellion. E. g., 1 W. \& M., c.~7 (1688) (threatened return of James
II); 7 \& 8 Will. 3, c.~11 (1696) (same); 17 Geo. 2, c.~6 (1744)
(threatened French invasion); 19 Geo. 2, c.~1 (1746) (threatened
rebellion in Scotland); 17 Geo. 3, c.~9 (1777) (the American
Revolution). Not long after Massachusetts had adopted a clause in its
constitution explicitly providing for habeas corpus, see Mass. Const.
pt.~2, ch.~6 (1780), reprinted in 3 Federal and State Constitutions,
Colonial Charters and Other Organic Laws 1888, 1910 (F. Thorpe
ed.~1909), it suspended the writ in order to deal with Shay's Rebellion,
see Act for Suspending the Privilege of the Writ of Habeas Corpus,
ch.~10, 1786 Mass. Acts p.~510.

Our Federal Constitution contains a provision explicitly permitting
suspension, but limiting the situations in which it may be invoked:
``The Privilege of the Writ of Habeas Corpus shall not be suspended,
unless when in Cases of Rebellion or Invasion the public Safety may
require it.'' Art. I, § 9, cl. 2. Although this provision does not state
that suspension must be effected by, or authorized by, a legislative
act, it has been so understood, consistent with English practice and the
Clause's placement in Article I. See Ex parte Bollman, 4 Cranch 75, 101
(1807); Ex parte Merryman, 17 F. Cas. 144, 151-152 (CD Md. 1861) (Taney,
C. J., rejecting Lincoln's unauthorized suspension); 3 Story § 1336.

The Suspension Clause was by design a safety valve, the Constitution's
only ``express provision for exercise of extraordinary authority because
of a crisis,'' Youngstown Sheet \& Tube Co.~v. Sawyer (Jackson, J.,
concurring). Very early in the Nation's history, President Jefferson
unsuccessfully sought a suspension of habeas corpus to deal with Aaron
Burr's conspiracy to overthrow the Government. See 16 Annals of Congress
402-425 (1807). During the Civil War, Congress passed its first Act
authorizing Executive suspension of the writ of habeas corpus, see Act
of Mar.~3, 1863, 12 Stat. 755, to the relief of those many who thought
President Lincoln's unauthorized proclamations of suspension (e. g.,
Proclamation No.~1, 13 Stat. 730 (1862)) unconstitutional. Later
Presidential proclamations of suspension relied upon the congressional
authorization, e. g., Proclamation No.~7, 13 Stat. 734 (1863). During
Reconstruction, Congress passed the Ku Klux Klan Act, which included a
provision authorizing suspension of the writ, invoked by President Grant
in quelling a rebellion in nine South Carolina counties. See Act of
Apr.~20, 1871, ch.~22, § 4, 17 Stat. 14; A Proclamation {[}of Oct.~17,
1871{]}, 7 Compilation of the Messages and Papers of the Presidents
136-138 (J. Richardson ed.~1899) (hereinafter Messages and Papers);
-139.

Two later Acts of Congress provided broad suspension authority to
governors of U. S. possessions. The Philippine Civil Government Act of
1902 provided that the Governor of the Philippines could suspend the
writ in case of rebellion, insurrection, or invasion. Act of July 1,
1902, ch.~1369, § 5, 32 Stat. 692. In 1905 the writ was suspended for
nine months by proclamation of the Governor. See Fisher v. Baker (1906).
The Hawaiian Organic Act of 1900 likewise provided that the Governor of
Hawaii could suspend the writ in case of rebellion or invasion (or
threat thereof). Ch. 339, § 67, 31 Stat. 153.

Of course the extensive historical evidence of criminal convictions and
habeas suspensions does not necessarily refute the Government's position
in this case. When the writ is suspended, the Government is entirely
free from judicial oversight. It does not claim such total liberation
here, but argues that it need only produce what it calls ``some
evidence'' to satisfy a habeas court that a detained individual is an
enemy combatant. See Brief for Respondents 34. Even if suspension of the
writ on the one hand, and committal for criminal charges on the other
hand, have been the only traditional means of dealing with citizens who
levied war against their own country, it is theoretically possible that
the Constitution does not require a choice between these alternatives.

I believe, however, that substantial evidence does refute that
possibility. First, the text of the 1679 Habeas Corpus Act makes clear
that indefinite imprisonment on reasonable suspicion is not an available
option of treatment for those accused of aiding the enemy, absent a
suspension of the writ. In the United States, this Act was read as
``enforc{[}ing{]} the common law,'' Ex parte Watkins, 3 Pet. 193, 202
(1830), and shaped the early understanding of the scope of the writ. As
noted above, see, § 7 of the Act specifically addressed those committed
for high treason, and provided a remedy if they were not indicted and
tried by the second succeeding court term. That remedy was not a
bobtailed judicial inquiry into whether there were reasonable grounds to
believe the prisoner had taken up arms against the King. Rather, if the
prisoner was not indicted and tried within the prescribed time, ``he
shall be discharged from his Imprisonment.'' 31 Car. 2, c.~2, § 7. The
Act does not contain any exception for wartime. That omission is
conspicuous, since § 7 explicitly addresses the offense of ``High
Treason,'' which often involved offenses of a military nature. See cases
cited.

Writings from the founding generation also suggest that, without
exception, the only constitutional alternatives are to charge the crime
or suspend the writ. In 1788, Thomas Jefferson wrote to James Madison
questioning the need for a Suspension Clause in cases of rebellion in
the proposed Constitution. His letter illustrates the constraints under
which the Founders understood themselves to operate:

``Why suspend the Hab. corp. in insurrections and rebellions? The
parties who may be arrested may be charged instantly with a well defined
crime. Of course the judge will remand them. If the publick safety
requires that the government should have a man imprisoned on less
probable testimony in those than in other emergencies; let him be taken
and tried, retaken and retried, while the necessity continues, only
giving him redress against the government for damages.'' 13 Papers of
Thomas Jefferson 442 (July 31, 1788) (J. Boyd ed.~1956).

A similar view was reflected in the 1807 House debates over suspension
during the armed uprising that came to be known as Burr's conspiracy:

``With regard to those persons who may be implicated in the conspiracy,
if the writ of habeas corpus be not suspended, what will be the
consequence? When apprehended, they will be brought before a court of
justice, who will decide whether there is any evidence that will justify
their commitment for farther prosecution. From the communication of the
Executive, it appeared there was sufficient evidence to authorize their
commitment. Several months would elapse before their final trial, which
would give time to collect evidence, and if this shall be sufficient,
they will not fail to receive the punishment merited by their crimes,
and inflicted by the laws of their country.'' 16 Annals of Congress
(remarks of Rep.~Burwell).

The absence of military authority to imprison citizens indefinitely in
wartime --- whether or not a probability of treason had been established
by means less than jury trial --- was confirmed by three cases decided
during and immediately after the War of 1812. In the first, In re Stacy,
10 Johns.

(N. Y. 1813), a citizen was taken into military custody on suspicion
that he was ``carrying provisions and giving information to the enemy.''
(emphasis deleted). Stacy petitioned for a writ of habeas corpus, and,
after the defendant custodian attempted to avoid complying, Chief
Justice Kent ordered attachment against him. Kent noted that the
military was ``without any color of authority in any military tribunal
to try a citizen for that crime'' and that it was ``holding him in the
closest confinement, and contemning the civil authority of the state.''
-.

Two other cases, later cited with approval by this Court in Ex parte
Milligan, 4 Wall. 2, 128-129 (1866), upheld verdicts for false
imprisonment against military officers. In Smith v. Shaw, 12 Johns. (N.
Y. 1815), the court affirmed an award of damages for detention of a
citizen on suspicion that he was, among other things, ``an enemy's spy
in time of war.'' The court held that ``{[}n{]}one of the offences
charged against Shaw were cognizable by a court-martial, except that
which related to his being a spy; and if he was an American citizen, he
could not be charged with such an offence. He might be amenable to the
civil authority for treason; but could not be punished, under martial
law, as a spy.'' ``If the defendant was justifiable in doing what he
did, every citizen of the United States would, in time of war, be
equally exposed to a like exercise of military power and authority.''
Finally, in M'Connell v. Hampton, 12 Johns. (N. Y. 1815), a jury awarded
\$9,000 for false imprisonment after a military officer confined a
citizen on charges of treason; the judges on appeal did not question the
verdict but found the damages excessive, in part because ``it does not
appear that {[}the defendant{]} . knew {[}the plaintiff{]} was a
citizen.'' (Spencer, J.). See generally Wuerth, The President's Power to
Detain ``Enemy Combatants'': Modern Lessons from Mr.~Madison's Forgotten
War, 98 Nw. U. L. Rev.~1567 (2004).

President Lincoln, when he purported to suspend habeas corpus without
congressional authorization during the Civil War, apparently did not
doubt that suspension was required if the prisoner was to be held
without criminal trial. In his famous message to Congress on July 4,
1861, he argued only that he could suspend the writ, not that even
without suspension, his imprisonment of citizens without criminal trial
was permitted. See Special Session Message, 6 Messages and Papers 20-31.

Further evidence comes from this Court's decision in Ex parte Milligan.
There, the Court issued the writ to an American citizen who had been
tried by military commission for offenses that included conspiring to
overthrow the Government, seize munitions, and liberate prisoners of
war. -7. The Court rejected in no uncertain terms the Government's
assertion that military jurisdiction was proper ``under the'laws and
usages of war,''' :

``It can serve no useful purpose to inquire what those laws and usages
are, whence they originated, where found, and on whom they operate; they
can never be applied to citizens in states which have upheld the
authority of the government, and where the courts are open and their
process unobstructed,''

Milligan is not exactly this case, of course, since the petitioner was
threatened with death, not merely imprisonment. But the reasoning and
conclusion of Milligan logically cover the present case. The Government
justifies imprisonment of Hamdi on principles of the law of war and
admits that, absent the war, it would have no such authority. But if the
law of war cannot be applied to citizens where courts are open, then
Hamdi's imprisonment without criminal trial is no less unlawful than
Milligan's trial by military tribunal.

Milligan responded to the argument, repeated by the Government in this
case, that it is dangerous to leave suspected traitors at large in time
of war:

``If it was dangerous, in the distracted condition of affairs, to leave
Milligan unrestrained of his liberty, because he'conspired against the
government, afforded aid and comfort to rebels, and incited the people
to insurrection,' the law said arrest him, confine him closely, render
him powerless to do further mischief; and then present his case to the
grand jury of the district, with proofs of his guilt, and, if indicted,
try him according to the course of the common law. If this had been
done, the Constitution would have been vindicated, the law of 1863
enforced, and the securities for personal liberty preserved and
defended.''

Thus, criminal process was viewed as the primary means --- and the only
means absent congressional action suspending the writ --- not only to
punish traitors, but to incapacitate them.

The proposition that the Executive lacks indefinite wartime detention
authority over citizens is consistent with the Founders' general
mistrust of military power permanently at the Executive's disposal. In
the Founders' view, the ``blessings of liberty'' were threatened by
``those military establishments which must gradually poison its very
fountain.'' The Federalist No.~45, p.~238 (J. Madison). No fewer than 10
issues of the Federalist were devoted in whole or part to allaying fears
of oppression from the proposed Constitution's authorization of standing
armies in peacetime. Many safeguards in the Constitution reflect these
concerns. Congress's authority ``{[}t{]}o raise and support Armies'' was
hedged with the proviso that ``no Appropriation of Money to that Use
shall be for a longer Term than two Years.'' U.S. Const., Art. I, § 8,
cl. 12. Except for the actual command of military forces, all
authorization for their maintenance and all explicit authorization for
their use is placed in the control of Congress under Article I, rather
than the President under Article II. As Hamilton explained, the
President's military authority would be ``much inferior'' to that of the
British King:

``It would amount to nothing more than the supreme command and direction
of the military and naval forces, as first general and admiral of the
confederacy: while that of the British king extends to the declaring of
war, and to the raising and regulating of fleets and armies; all which,
by the constitution under consideration, would appertain to the
legislature.'' The Federalist No.~69, p.~357.

A view of the Constitution that gives the Executive authority to use
military force rather than the force of law against citizens on American
soil flies in the face of the mistrust that engendered these provisions.

The Government argues that our more recent jurisprudence ratifies its
indefinite imprisonment of a citizen within the territorial jurisdiction
of federal courts. It places primary reliance upon Ex parte Quirin, 317
U. S. 1 (1942), a World War II case upholding the trial by military
commission of eight German saboteurs, one of whom, Herbert Haupt, was a
U. S. citizen. The case was not this Court's finest hour. The Court
upheld the commission and denied relief in a brief per curiam issued the
day after oral argument concluded, see -19, unnumbered note; a week
later the Government carried out the commission's death sentence upon
six saboteurs, including Haupt. The Court eventually explained its
reasoning in a written opinion issued several months later.

Only three paragraphs of the Court's lengthy opinion dealt with the
particular circumstances of Haupt's case. See The Government argued that
Haupt, like the other petitioners, could be tried by military commission
under the laws of war. In agreeing with that contention, Quirin
purported to interpret the language of Milligan quoted above (the law of
war ``can never be applied to citizens in states which have upheld the
authority of the government, and where the courts are open and their
process unobstructed'') in the following manner:

``Elsewhere in its opinion . the Court was at pains to point out that
Milligan, a citizen twenty years resident in Indiana, who had never been
a resident of any of the states in rebellion, was not an enemy
belligerent either entitled to the status of a prisoner of war or
subject to the penalties imposed upon unlawful belligerents. We construe
the Court's statement as to the inapplicability of the law of war to
Milligan's case as having particular reference to the facts before it.
From them the Court concluded that Milligan, not being a part of or
associated with the armed forces of the enemy, was a non-belligerent,
not subject to the law of war . .'' 317 U. S..

In my view this seeks to revise Milligan rather than describe it.
Milligan had involved (among other issues) two separate questions: (1)
whether the military trial of Milligan was justified by the laws of war,
and if not (2) whether the President's suspension of the writ, pursuant
to congressional authorization, prevented the issuance of habeas corpus.
The Court's categorical language about the law of war's inapplicability
to citizens where the courts are open (with no exception mentioned for
citizens who were prisoners of war) was contained in its discussion of
the first point. See 4 Wall.. The factors pertaining to whether Milligan
could reasonably be considered a belligerent and prisoner of war, while
mentioned earlier in the opinion, see were made relevant and brought to
bear in the Court's later discussion, see of whether Milligan came
within the statutory provision that effectively made an exception to
Congress's authorized suspension of the writ for (as the Court described
it) ``all parties, not prisoners of war, resident in their respective
jurisdictions, . who were citizens of states in which the administration
of the laws in the Federal tribunals was unimpaired,'' Milligan thus
understood was in accord with the traditional law of habeas corpus I
have described: Though treason often occurred in wartime, there was,
absent provision for special treatment in a congressional suspension of
the writ, no exception to the right to trial by jury for citizens who
could be called ``belligerents'' or ``prisoners of war.''

But even if Quirin gave a correct description of Milligan, or made an
irrevocable revision of it, Quirin would still not justify denial of the
writ here. In Quirin it was uncontested that the petitioners were
members of enemy forces. They were ``admitted enemy invaders,'' 317 U.
S. (emphasis added), and it was ``undisputed'' that they had landed in
the United States in service of German forces, The specific holding of
the Court was only that, ``upon the conceded facts,'' the petitioners
were ``plainly within {[}the{]} boundaries'' of military jurisdiction,
(emphasis

added) But where those jurisdictional

facts are not conceded---where the petitioner insists that he is not a
belligerent ---Quirin left the pre-existing law in place: Absent
suspension of the writ, a citizen held where the courts are open is
entitled either to criminal trial or to a judicial decree requiring his
release It follows from what I have said that Hamdi is entitled to a
habeas decree requiring his release unless (1) criminal proceedings are
promptly brought, or (2) Congress has suspended the writ of habeas
corpus. A suspension of the writ could, of course, lay down conditions
for continued detention, similar to those that today's opinion
prescribes under the Due Process Clause. Cf. Act of Mar.~3, 1863, 12
Stat. 755. But there is a world of difference between the people's
representatives' determining the need for that suspension (and
prescribing the conditions for it), and this Court's doing so.

The plurality finds justification for Hamdi's imprisonment in the
Authorization for Use of Military Force, 115 Stat. 224, which provides:

``That the President is authorized to use all necessary and appropriate
force against those nations, organizations, or persons he determines
planned, authorized, committed, or aided the terrorist attacks that
occurred on September 11, 2001, or harbored such organizations or
persons, in order to prevent any future acts of international terrorism
against the United States by such nations, organizations or persons.'' §
2(a).

This is not remotely a congressional suspension of the writ, and no one
claims that it is. Contrary to the plurality's view, I do not think this
statute even authorizes detention of a citizen with the clarity
necessary to satisfy the interpretive canon that statutes should be
construed so as to avoid grave constitutional concerns, see Edward J.
DeBartolo Corp.~v. Florida Gulf Coast Building \& Constr. Trades
Council; with the clarity necessary to comport with cases such as Ex
parte Endo, and Duncan v. Kahanamoku; or with the clarity necessary to
overcome the statutory prescription that ``{[}n{]}o citizen shall be
imprisoned or otherwise detained by the United States except pursuant to
an Act of Congress.'' 18 U. S. C. § 4001(a) But even if it did, I would
not permit it to overcome Hamdi's entitlement to habeas corpus relief.
The Suspension Clause of the Constitution, which carefully circumscribes
the conditions under which the writ can be withheld, would be a sham if
it could be evaded by congressional prescription of requirements other
than the common-law requirement of committal for criminal prosecution
that render the writ, though available, unavailing. If the Suspension
Clause does not guarantee the citizen that he will either be tried or
released, unless the conditions for suspending the writ exist and the
grave action of suspending the writ has been taken; if it merely
guarantees the citizen that he will not be detained unless Congress by
ordinary legislation says he can be detained; it guarantees him very
little indeed.

It should not be thought, however, that the plurality's evisceration of
the Suspension Clause augments, principally, the power of Congress. As
usual, the major effect of its constitutional improvisation is to
increase the power of the Court. Having found a congressional
authorization for detention of citizens where none clearly exists; and
having discarded the categorical procedural protection of the Suspension
Clause; the plurality then proceeds, under the guise of the Due Process
Clause, to prescribe what procedural protections it thinks appropriate.
It ``weigh{[}s{]} the private interest . against the Government's
asserted interest,'' (internal quotation marks omitted), and---just as
though writing a new Constitution---comes up with an unheard-of system
in which the citizen rather than the Government bears the burden of
proof, testimony is by hearsay rather than live witnesses, and the
presiding officer may well be a ``neutral'' military officer rather than
judge and jury. See It claims authority to engage in this sort of
``judicious balancing'' from Mathews v. Eldridge, a case involving
\ldots{} the withdrawal of disability benefits! Whatever the merits of
this technique when newly recognized property rights are at issue (and
even there they are questionable), it has no place where the
Constitution and the common law already supply an answer.

Having distorted the Suspension Clause, the plurality finishes up by
transmogrifying the Great Writ---disposing of the present habeas
petition by remanding for the District Court to ``engag{[}e{]} in a
factfinding process that is both prudent and incremental,'' ``In the
absence of {[}the Executive's prior provision of procedures that satisfy
due process{]}, \ldots{} a court that receives a petition for a writ of
habeas corpus from an alleged enemy combatant must itself ensure that
the minimum requirements of due process are achieved.'' This judicial
remediation of executive default is unheard of. The role of habeas
corpus is to determine the legality of executive detention, not to
supply the omitted process necessary to make it legal. See Preiser v.
Rodriguez (``{[}T{]}he essence of habeas corpus is an attack by a person
in custody upon the legality of that custody, and \ldots{} the
traditional function of the writ is to secure release from illegal
custody''); 1 Blackstone 132-133. It is not the habeas court's function
to make illegal detention legal by supplying a process that the
Government could have provided, but chose not to. If Hamdi is being
imprisoned in violation of the Constitution (because without due process
of law), then his habeas petition should be granted; the Executive may
then hand him over to the criminal authorities, whose detention for the
purpose of prosecution will be lawful, or else must release him.

There is a certain harmony of approach in the plurality's making up for
Congress's failure to invoke the Suspension Clause and its making up for
the Executive's failure to apply what it says are needed procedures---an
approach that reflects what might be called a Mr.~Fix-it Mentality. The
plurality seems to view it as its mission to Make Everything Come Out
Right, rather than merely to decree the consequences, as far as
individual rights are concerned, of the other two branches' actions and
omissions. Has the Legislature failed to suspend the writ in the current
dire emergency? Well, we will remedy that failure by prescribing the
reasonable conditions that a suspension should have included. And has
the Executive failed to live up to those reasonable conditions? Well, we
will ourselves make that failure good, so that this dangerous fellow (if
he is dangerous) need not be set free. The problem with this approach is
not only that it steps out of the courts' modest and limited role in a
democratic society; but that by repeatedly doing what it thinks the
political branches ought to do it encourages their lassitude and saps
the vitality of government by the people.

Several limitations give my views in this matter a relatively narrow
compass. They apply only to citizens, accused of being enemy combatants,
who are detained within the territorial jurisdiction of a federal court.
This is not likely to be a numerous group; currently we know of only
two, Hamdi and Jose Padilla. Where the citizen is captured outside and
held outside the United States, the constitutional requirements may be
different. Cf. Johnson v. Eisentrager (1950); Reid v. Covert (1957)
(Harlan, J., concurring in result); Rasul v. Bush, (SCALIA, J.,
dissenting). Moreover, even within the United States, the accused
citizen-enemy combatant may lawfully be detained once prosecution is in
progress or in contemplation. See, e.g., County of Riverside v.
McLaughlin (brief detention pending judicial determination after
warrantless arrest); United States v. Salerno (pretrial detention under
the Bail Reform Act). The Government has been notably successful in
securing conviction, and hence long-term custody or execution, of those
who have waged war against the state.

I frankly do not know whether these tools are sufficient to meet the
Government's security needs, including the need to obtain intelligence
through interrogation. It is far beyond my competence, or the Court's
competence, to determine that. But it is not beyond Congress's. If the
situation demands it, the Executive can ask Congress to authorize
suspension of the writ---which can be made subject to whatever
conditions Congress deems appropriate, including even the procedural
novelties invented by the plurality today. To be sure, suspension is
limited by the Constitution to cases of rebellion or invasion. But
whether the attacks of September 11, 2001, constitute an ``invasion,''
and whether those attacks still justify suspension several years later,
are questions for Congress rather than this Court. See 3 Story § 1336 If
civil rights

are to be curtailed during wartime, it must be done openly and
democratically, as the Constitution requires, rather than by silent
erosion through an opinion of this Court.

The Founders well understood the difficult tradeoff between safety and
freedom. ``Safety from external danger,'' Hamilton declared,

``is the most powerful director of national conduct. Even the ardent
love of liberty will, after a time, give way to its dictates. The
violent destruction of life and property incident to war; the continual
effort and alarm attendant on a state of continual danger, will compel
nations the most attached to liberty, to resort for repose and security
to institutions which have a tendency to destroy their civil and
political rights. To be more safe, they, at length, become willing to
run the risk of being less free.'' The Federalist No.~8, p.~33 (A.
Hamilton).

The Founders warned us about the risk, and equipped us with a
Constitution designed to deal with it.

Many think it not only inevitable but entirely proper that liberty give
way to security in times of national crisis---that, at the extremes of
military exigency, inter arma silent leges. Whatever the general merits
of the view that war silences law or modulates its voice, that view has
no place in the interpretation and application of a Constitution
designed precisely to confront war and, in a manner that accords with
democratic principles, to accommodate it. Because the Court has
proceeded to meet the current emergency in a manner the Constitution
does not envision, I respectfully dissent.

\begin{center}\rule{0.5\linewidth}{\linethickness}\end{center}

Notes:

\begin{enumerate}
\def\labelenumi{\arabic{enumi}.}
\item
  As I shall discuss presently, see infra, the Court purported to limit
  this language in Ex parte Quirin. Whatever Quirin's effect on
  Milligan's precedential value, however, it cannot undermine its value
  as an indicator of original meaning. Cf. Reid v. Covert (plurality
  opinion) (Milligan remains ``one of the great landmarks in this
  Court's history'').
\item
  Without bothering to respond to this analysis, the plurality states
  that Milligan ``turned in large part'' upon the defendant's lack of
  prisoner-of-war status, and that the Milligan Court explicitly and
  repeatedly said so. Neither is true. To the extent, however, that
  prisoner-of-war status was relevant in Milligan, it was only because
  prisoners of war received different statutory treatment under the
  conditional suspension then in effect.
\item
  The only two Court of Appeals cases from World War II cited by the
  Government in which citizens were detained without trial likewise
  involved petitioners who were conceded to have been members of enemy
  forces. See In re Territo, -145 (CA9 1946); Colepaugh v. Looney, CA10
  1956). The plurality complains that Territo is the only case I have
  identified in which ``a United States citizen {[}was{]} captured in a
  foreign combat zone,'' Indeed it is; such cases must surely be rare.
  But given the constitutional tradition I have described, the burden is
  not upon me to find cases in which the writ was granted to citizens in
  this country who had been captured on foreign battlefields; it is upon
  those who would carve out an exception for such citizens (as the
  plurality's complaint suggests it would) to find a single case (other
  than one where enemy status was admitted) in which habeas was denied.
\item
  The plurality's assertion that Quirin somehow ``clarifies'' Milligan,
  is simply false. As I discuss and this page, the Quirin Court
  propounded a mistaken understanding of Milligan; but nonetheless its
  holding was limited to ``the case presented by the present record,''
  and to ``the conceded facts,'' and thus avoided conflict with the
  earlier case. See 317 U. S. (emphasis added). The plurality, ignoring
  this expressed limitation, thinks it ``beside the point'' whether
  belligerency is conceded or found ``by some other process'' (not
  necessarily a jury trial) ``that verifies this fact with sufficient
  certainty.'' But the whole point of the procedural guarantees in the
  Bill of Rights is to limit the methods by which the Government can
  determine facts that the citizen disputes and on which the citizen's
  liberty depends. The plurality's claim that Quirin's one-paragraph
  discussion of Milligan provides a ``{[}c{]}lear \ldots{} disavowal''
  of two false imprisonment cases from the War of 1812, thus defies
  logic; unlike the plaintiffs in those cases, Haupt was concededly a
  member of an enemy force.
\end{enumerate}

The Government also cites Moyer v. Peabody, a suit for damages against
the Governor of Colorado, for violation of due process in detaining the
alleged ringleader of a rebellion quelled by the state militia after the
Governor's declaration of a state of insurrection and (he contended)
suspension of the writ ``as incident thereto.'' Ex parte Moyer, 35 Colo.
154, 157, 91 P. 738, 740 (1905). But the holding of Moyer v. Peabody
(even assuming it is transferable from state-militia detention after
state suspension to federal standing-army detention without suspension)
is simply that ``{[}s{]}o long as such arrests {[}were{]} made in good
faith and in the honest belief that they {[}were{]} needed in order to
head the insurrection off,'' 212 U. S., an action in damages could not
lie. This ``good-faith'' analysis is a forebear of our modern doctrine
of qualified immunity. Cf. Scheuer v. Rhodes (1974) (understanding Moyer
in this way). Moreover, the detention at issue in Moyer lasted about 2½
months, see 212 U. S., roughly the length of time permissible under the
1679 Habeas Corpus Act, see.

In addition to Moyer v. Peabody, JUSTICE THOMAS relies upon Luther v.
Borden, 7 How. 1 (1849), a case in which the state legislature had
imposed martial law---a step even more drastic than suspension of the
writ. See post (dissenting opinion). But martial law has not been
imposed here, and in any case is limited to ``the theatre of active
military operations, where war really prevails,'' and where therefore
the courts are closed. Ex parte Milligan, 4 Wall. 2, 127 (1866); see
also -130 (distinguishing

Luther).

\begin{enumerate}
\def\labelenumi{\arabic{enumi}.}
\setcounter{enumi}{4}
\item
  The plurality rejects any need for ``specific language of detention''
  on the ground that detention of alleged combatants is a ``fundamental
  incident of waging war.'' Its authorities do not support that holding
  in the context of the present case. Some are irrelevant because they
  do not address the detention of American citizens. E.g., Naqvi,
  Doubtful Prisoner-of-War Status, 84 Int'l Rev.~Red Cross 571, 572
  (2002). The plurality's assertion that detentions of citizen and alien
  combatants are equally authorized has no basis in law or common sense.
  Citizens and noncitizens, even if equally dangerous, are not similarly
  situated. See, e.g., Milligan; Johnson v. Eisentrager; Rev.~Stat.
  4067, 50 U.S.C. § 21 (Alien Enemy Act). That captivity may be
  consistent with the principles of international law does not prove
  that it also complies with the restrictions that the Constitution
  places on the American Government's treatment of its own citizens. Of
  the authorities cited by the plurality that do deal with detention of
  citizens, Quirinand Territohave already been discussed and rejected.
  See, and n.~3. The remaining authorities pertain to U.S. detention of
  citizens during the Civil War, and are irrelevant for two reasons: (1)
  the Lieber Code was issued following a congressional authorization of
  suspension of the writ, see Instructions for the Government of Armies
  of the United States in the Field, Gen.~Order No.~100 (1863),
  reprinted in 2 F. Lieber, Miscellaneous Writings, p.~246; Act of
  Mar.~3, 1863, 12 Stat. 755, §§ 1, 2; and (2) citizens of the
  Confederacy, while citizens of the United States, were also regarded
  as citizens of a hostile power.
\item
  JUSTICE THOMAS worries that the constitutional conditions for
  suspension of the writ will not exist ``during many . emergencies
  during which \ldots{} detention authority might be necessary,'' post.
  It is difficult to imagine situations in which security is so
  seriously threatened as to justify indefinite imprisonment without
  trial, and yet the constitutional conditions of rebellion or invasion
  are not met.
\end{enumerate}

\textbf{JUSTICE THOMAS, dissenting.}

The Executive Branch, acting pursuant to the powers vested in the
President by the Constitution and with explicit congressional approval,
has determined that Yaser Hamdi is an enemy combatant and should be
detained. This detention falls squarely within the Federal Government's
war powers, and we lack the expertise and capacity to second-guess that
decision. As such, petitioners' habeas challenge should fail, and there
is no reason to remand the case. The plurality reaches a contrary
conclusion by failing adequately to consider basic principles of the
constitutional structure as it relates to national security and foreign
affairs and by using the balancing scheme of Mathews v. Eldridge. I do
not think that the Federal Government's war powers can be balanced away
by this Court. Arguably, Congress could provide for additional
procedural protections, but until it does, we have no right to insist
upon them. But even if I were to agree with the general approach the
plurality takes, I could not accept the particulars. The plurality
utterly fails to account for the Government's compelling interests and
for our own institutional inability to weigh competing concerns
correctly. I respectfully dissent.

``It is'obvious and unarguable' that no governmental interest is more
compelling than the security of the Nation.'' Haig v. Agee (quoting
Aptheker v. Secretary of State). The national security, after all, is
the primary responsibility and purpose of the Federal Government. See,
e.g., Youngstown Sheet \& Tube Co.~v. Sawyer (Clark, J., concurring in
judgment); The Federalist No.~23, pp.~146-147 (J. Cooke ed.~1961) (A.
Hamilton) (``The principle purposes to be answered by Union are these
--- The common defence of the members---the preservation of the public
peace as well against internal convulsions as external attacks''). But
because the Founders understood that they could not foresee the myriad
potential threats to national security that might later arise, they
chose to create a Federal Government that necessarily possesses
sufficient power to handle any threat to the security of the Nation. The
power to protect the Nation

``ought to exist without limitation . {[}b{]}ecause it is impossible to
foresee or define the extent and variety of national exigencies, or the
correspondent extent \& variety of the means which may be necessary to
satisfy them. The circumstances that endanger the safety of nations are
infinite; and for this reason no constitutional shackles can wisely be
imposed on the power to which the care of it is committed.''

See also Nos. 34 and 41.

The Founders intended that the President have primary
responsibility---along with the necessary power---to protect the
national security and to conduct the Nation's foreign relations. They
did so principally because the structural advantages of a unitary
Executive are essential in these domains. ``Energy in the executive is a
leading character in the definition of good government. It is essential
to the protection of the community against foreign attacks.'' No.~70 (A.
Hamilton). The principle ``ingredien{[}t{]}'' for ``energy in the
executive'' is ``unity.'' This is because ``{[}d{]}ecision, activity,
secrecy, and dispatch will generally characterise the proceedings of one
man, in a much more eminent degree, than the proceedings of any greater
number.''

These structural advantages are most important in the national-security
and foreign-affairs contexts. ``Of all the cares or concerns of
government, the direction of war most peculiarly demands those qualities
which distinguish the exercise of power by a single hand.'' No.~74 (A.
Hamilton). Also for these reasons, John Marshall explained that
``{[}t{]}he President is the sole organ of the nation in its external
relations, and its sole representative with foreign nations.'' 10 Annals
of Cong. 613 (1800); see -614. To this end, the Constitution vests in
the President ``{[}t{]}he executive Power,'' Art. II, § 1, provides that
he ``shall be Commander in Chief of the'' Armed Forces, § 2, and places
in him the power to recognize foreign governments, § 3.

This Court has long recognized these features and has accordingly held
that the President has constitutional authority to protect the national
security and that this authority carries with it broad discretion.

``If a war be made by invasion of a foreign nation, the President is not
only authorized but bound to resist force by force. He does not initiate
the war, but is bound to accept the challenge without waiting for any
special legislative authority\ldots. Whether the President in fulfilling
his duties, as Commander in-chief, in suppressing an insurrection, has
met with such armed hostile resistance \ldots{} is a question to be
decided by him.'' Prize Cases, 2 Black 635, 668, 670 (1863).

The Court has acknowledged that the President has the authority to
``employ {[}the Nation's Armed Forces{]} in the manner he may deem most
effectual to harass and conquer and subdue the enemy.'' Fleming v. Page,
9 How. 603, 615 (1850). With respect to foreign affairs as well, the
Court has recognized the President's independent authority and need to
be free from interference. See, e. g., United States v. Curtiss-Wright
Export Corp.~(explaining that the President ``has his confidential
sources of information. He has his agents in the form of diplomatic,
consular and other officials. Secrecy in respect of information gathered
by them may be highly necessary, and the premature disclosure of it
productive of harmful results''); Chicago \& Southern Air Lines, Inc.~v.
Waterman S. S. Corp..

Congress, to be sure, has a substantial and essential role in both
foreign affairs and national security. But it is crucial to recognize
that judicial interference in these domains destroys the purpose of
vesting primary responsibility in a unitary Executive. I cannot improve
on Justice Jackson's words, speaking for the Court:

``The President, both as Commander-in-Chief and as the Nation's organ
for foreign affairs, has available intelligence services whose reports
are not and ought not to be published to the world. It would be
intolerable that courts, without the relevant information, should review
and perhaps nullify actions of the Executive taken on information
properly held secret. Nor can courts sit in camera in order to be taken
into executive confidences. But even if courts could require full
disclosure, the very nature of executive decisions as to foreign policy
is political, not judicial. Such decisions are wholly confided by our
Constitution to the political departments of the government, Executive
and Legislative. They are delicate, complex, and involve large elements
of prophecy. They are and should be undertaken only by those directly
responsible to the people whose welfare they advance or imperil. They
are decisions of a kind for which the Judiciary has neither aptitude,
facilities nor responsibility and which has long been held to belong in
the domain of political power not subject to judicial intrusion or
inquiry.''

Several points, made forcefully by Justice Jackson, are worth
emphasizing. First, with respect to certain decisions relating to
national security and foreign affairs, the courts simply lack the
relevant information and expertise to second-guess determinations made
by the President based on information properly withheld. Second, even if
the courts could compel the Executive to produce the necessary
information, such decisions are simply not amenable to judicial
determination because ``{[}t{]}hey are delicate, complex, and involve
large elements of prophecy.'' Third, the Court in Chicago \& Southern
Air Lines and elsewhere has correctly recognized the primacy of the
political branches in the foreign-affairs and national-security
contexts.

For these institutional reasons and because ``Congress cannot anticipate
and legislate with regard to every possible action the President may
find it necessary to take or every possible situation in which he might
act,'' it should come as no surprise that ``{[}s{]}uch failure of
Congress \ldots{} does not,`especially \ldots{} in the areas of foreign
policy and national security,' imply'congressional disapproval' of
action taken by the Executive.'' Dames \& Moore v. Regan (quoting Agee).
Rather, in these domains, the fact that Congress has provided the
President with broad authorities does not imply---and the Judicial
Branch should not infer---that Congress intended to deprive him of
particular powers not specifically enumerated. See Dames \& Moore. As
far as the courts are concerned, ``the enactment of legislation closely
related to the question of the President's authority in a particular
case which evinces legislative intent to accord the President broad
discretion may be considered to'invite''measures on independent
presidential responsibility.'" (quoting Youngstown (Jackson, J.,
concurring)).

Finally, and again for the same reasons, where ``the President acts
pursuant to an express or implied authorization from Congress, he
exercises not only his powers but also those delegated by Congress{[},
and i{]}n such a case the executive action'would be supported by the
strongest of presumptions and the widest latitude of judicial
interpretation, and the burden of persuasion would rest heavily upon any
who might attack it.''' Dames \& Moore (quoting Youngstown (Jackson, J.,
concurring)). That is why the Court has explained, in a case analogous
to this one, that ``the detention{[},{]} ordered by the President in the
declared exercise of his powers as Commander in Chief of the Army in
time of war and of grave public danger{[}, is{]} not to be set aside by
the courts without the clear conviction that {[}it is{]} in conflict
with the Constitution or laws of Congress constitutionally enacted.'' Ex
parte Quirin. See also Ex parte Milligan, 4 Wall. 2, 133 (1866) (Chase,
C. J., concurring in judgment) (stating that a sentence imposed by a
military commission ``must not be set aside except upon the clearest
conviction that it cannot be reconciled with the Constitution and the
constitutional legislation of Congress''). This deference extends to the
President's determination of all the factual predicates necessary to
conclude that a given action is appropriate. See Quirin (``We are not
here concerned with any question of the guilt or innocence of
petitioners''). See also Hirabayashi v. United States; Prize Cases, 2
Black; Martin v. Mott, 12 Wheat. 19, 29-30 (1827).

To be sure, the Court has at times held, in specific circumstances, that
the military acted beyond its warmaking authority. But these cases are
distinguishable in important ways. In Ex parte Endo, the Court held
unlawful the detention of an admittedly law-abiding and loyal American
of Japanese ancestry. It did so because the Government's asserted reason
for the detention had nothing to do with the congressional and executive
authorities upon which the Government relied. Those authorities
permitted detention for the purpose of preventing espionage and sabotage
and thus could not be pressed into service for detaining a loyal
citizen. See -302. Further, the Court ``stress{[}ed{]} the silence
\ldots{} of the {[}relevant{]} Act and the Executive Orders.'' (emphasis
added); see also -304. The Court sensibly held that the Government could
not detain a loyal citizen pursuant to executive and congressional
authorities that could not conceivably be implicated given the
Government's factual allegations. And in Youngstown, Justice Jackson
emphasized that ``Congress ha{[}d{]} not left seizure of private
property an open field but ha{[}d{]} covered it by three statutory
policies inconsistent with th{[}e{]} seizure.'' 343 U.S. (concurring
opinion). See also Milligan (Chase, C. J., concurring in judgment)
(noting that the Government failed to comply with statute directly on
point).

I acknowledge that the question whether Hamdi's executive detention is
lawful is a question properly resolved by the Judicial Branch, though
the question comes to the Court with the strongest presumptions in favor
of the Government. The plurality agrees that Hamdi's detention is lawful
if he is an enemy combatant. But the question whether Hamdi is actually
an enemy combatant is ``of a kind for which the Judiciary has neither
aptitude, facilities nor responsibility and which has long been held to
belong in the domain of political power not subject to judicial
intrusion or inquiry.'' Chicago \& Southern Air Lines. That is, although
it is appropriate for the Court to determine the judicial question
whether the President has the asserted authority, see, e.g., Ex parte
Endowe lack the information and expertise to question whether Hamdi is
actually an enemy combatant, a question the resolution of which is
committed to other branches In the

words of then-Judge Scalia:

``In Old Testament days, when judges ruled the people of Israel and led
them into battle, a court professing the belief that it could order a
halt to a military operation in foreign lands might not have been a
startling phenomenon. But in modern times, and in a country where such
governmental functions have been committed to elected delegates of the
people, such an assertion of jurisdiction is extraordinary. The
{[}C{]}ourt's decision today reflects a willingness to extend judicial
power into areas where we do not know, and have no way of finding out,
what serious harm we may be doing.'' Ramirez de Arellano v. Weinberger,
-1551 (CADC 1984) (en banc) (dissenting opinion) (footnote omitted).

See also n.~1 (noting that ``{[}e{]}ven the ancient Israelites
eventually realized the shortcomings of judicial commanders-in-chief'').
The decision whether someone is an enemy combatant is, no doubt,
``delicate, complex, and involv{[}es{]} large elements of prophecy,''
Chicago \& Southern Air Lines, which, incidentally might in part explain
why ``the Government has never provided any court with the full criteria
that it uses in classifying individuals as such,'' See also infra
(discussing other military decisions).

```The war power of the national government is''the power to wage war
successfully.``''' Lichter v. United States, n.~9 (1948) (quoting
Hughes, War Powers Under the Constitution, 42 A. B. A. Rep.~232, 238
(1917)). It follows that this power ``is not limited to victories in the
field, but carries with it the inherent power to guard against the
immediate renewal of the conflict,'' In re Yamashita; see also Stewart
v. Kahn, 11 Wall. 493, 507 (1871), and quite obviously includes the
ability to detain those (even United States citizens) who fight against
our troops or those of our allies, see, e.g., Quirin, 30-31; -39; Duncan
v. Kahanamoku (1946); W. Winthrop, Military Law and Precedents 788 (2d
ed.~rev. 1920); W. Whiting, War Powers Under the Constitution of the
United States 167 (43d ed.~1871); -46 (noting that Civil War ``rebels''
may be treated as foreign belligerents); see also Although the President
very well may have inherent authority to detain those arrayed against
our troops, I agree with the plurality that we need not decide that
question because Congress has authorized the President to do so. See The
Authorization for Use of Military Force (AUMF), 115 Stat. 224,
authorizes the President to ``use all necessary and appropriate force
against those nations, organizations, or persons he determines planned,
authorized, committed, or aided the terrorist attacks'' of September 11,
2001. Indeed, the Court has previously concluded that language
materially identical to the AUMF authorizes the Executive to ``make the
ordinary use of the soldiers \ldots; that he may kill persons who resist
and, of course, that he may use the milder measure of seizing {[}and
detaining{]} the bodies of those whom he considers to stand in the way
of restoring peace.'' Moyer v. Peabody.

The plurality, however, qualifies its recognition of the President's
authority to detain enemy combatants in the war on terrorism in ways
that are at odds with our precedent. Thus, the plurality relies
primarily on Article 118 of the Geneva Convention (III) Relative to the
Treatment of Prisoners of War, Aug.~12, 1949, 6 U. S. T. 3406, T. I. A.
S. No.~3364, for the proposition that ``{[}i{]}t is a clearly
established principle of the law of war that detention may last no
longer than active hostilities.'' It then appears to limit the
President's authority to detain by requiring that ``the record
establis{[}h{]} that United States troops are still involved in active
combat in Afghanistan'' because, in that case, detention would be ``part
of the exercise of'necessary and appropriate force.''' But I do not
believe that we may diminish the Federal Government's war powers by
reference to a treaty and certainly not to a treaty that does not apply.
See n.~6, infra. Further, we are bound by the political branches'
determination that the United States is at war. See, e.g., Ludecke v.
Watkins (1948); Prize Cases, 2 Black; Mott, 12 Wheat.. And, in any case,
the power to detain does not end with the cessation of formal
hostilities. See, e.g., Madsen v. Kinsella; Johnson v. Eisentrager;
cf.~Moyer.

Accordingly, the President's action here is ``supported by the strongest
of presumptions and the widest latitude of judicial interpretation.''
Dames \& Moore (internal quotation marks omitted) The question becomes

whether the Federal Government (rather than the President acting alone)
has power to detain Hamdi as an enemy combatant. More precisely, we must
determine whether the Government may detain Hamdi given the procedures
that were used.

I agree with the plurality that the Federal Government has power to
detain those that the Executive Branch determines to be enemy
combatants. See But I do not think that the plurality has adequately
explained the breadth of the President's authority to detain enemy
combatants, an authority that includes making virtually conclusive
factual findings. In my view, the structural considerations discussed
above, as recognized in our precedent, demonstrate that we lack the
capacity and responsibility to second-guess this determination.

This makes complete sense once the process that is due Hamdi is made
clear. As an initial matter, it is possible that the Due Process Clause
requires only ``that our Government must proceed according to the'law of
the land'---that is, according to written constitutional and statutory
provisions.'' In re Winship (Black, J., dissenting). I need not go this
far today because the Court has already explained the nature of due
process in this context.

In a case strikingly similar to this one, the Court addressed a
Governor's authority to detain for an extended period a person the
executive believed to be responsible, in part, for a local insurrection.
Justice Holmes wrote for a unanimous Court:

``When it comes to a decision by the head of the State upon a matter
involving its life, the ordinary rights of individuals must yield to
what he deems the necessities of the moment. Public danger warrants the
substitution of executive process for judicial process. This was
admitted with regard to killing men in the actual clash of arms, and we
think it obvious, although it was disputed, that the same is true of
temporary detention to prevent apprehended harm.'' Moyer (citation
omitted; emphasis added).

The Court answered Moyer's claim that he had been denied due process by
emphasizing that

``it is familiar that what is due process of law depends on
circumstances. It varies with the subject-matter and the necessities of
the situation. Thus summary proceedings suffice for taxes, and executive
decisions for exclusion from the country\ldots. Such arrests are not
necessarily for punishment, but are by way of precaution to prevent the
exercise of hostile power.'' -85 (citations omitted).

In this context, due process requires nothing more than a good-faith
executive determination To be clear: The Court

has held that an executive, acting pursuant to statutory and
constitutional authority, may, consistent with the Due Process Clause,
unilaterally decide to detain an individual if the executive deems this
necessary for the public safety even if he is mistaken.

Moyer is not an exceptional case. In Luther v. Borden, 7 How. 1 (1849),
the Court discussed the President's constitutional and statutory
authority, in response to a request from a state legislature or
executive, ```to call forth such number of the militia of any other
State or States, as may be applied for, as he may judge sufficient to
suppress {[}an{]} insurrection.''' (quoting Act of Feb.~28, 1795). The
Court explained that courts could not review the President's decision to
recognize one of the competing legislatures or executives. See 7 How..
If a court could second-guess this determination, ``it would become the
duty of the court (provided it came to the conclusion that the President
had decided incorrectly) to discharge those who were arrested or
detained by the troops in the service of the United States.'' ``If the
judicial power extends so far,'' the Court concluded, ``the guarantee
contained in the Constitution of the United States {[}referring to Art.
IV, § 4{]} is a guarantee of anarchy, and not of order.'' The Court
clearly contemplated that the President had authority to detain as he
deemed necessary, and such detentions evidently comported with the Due
Process Clause as long as the President correctly decided to call forth
the militia, a question the Court said it could not review.

The Court also addressed the natural concern that placing ``this power
in the President is dangerous to liberty, and may be abused.'' The Court
noted that ``{[}a{]}ll power may be abused if placed in unworthy
hands,'' and explained that ``it would be difficult . to point out any
other hands in which this power would be more safe, and at the same time
equally effectual.'' Putting that aside, the Court emphasized that this
power ``is conferred upon him by the Constitution and laws of the United
States, and must therefore be respected and enforced in its judicial
tribunals.'' Finally, the Court explained that if the President abused
this power ``it would be in the power of Congress to apply the proper
remedy. But the courts must administer the law as they find it.''

Almost 140 years later, in United States v. Salerno, the Court explained
that the Due Process Clause ``lays down {[}no{]} categorical
imperative.'' The Court continued:

``We have repeatedly held that the Government's regulatory interest in
community safety can, in appropriate circumstances, outweigh an
individual's liberty interest. For example, in times of war or
insurrection, when society's interest is at its peak, the Government may
detain individuals whom the Government believes to be dangerous.''

The Court cited Ludecke v. Watkins, for this latter proposition even
though Ludecke actually involved detention of enemy aliens. See also
Selective Draft Law Cases; Jacobson v.

Massachusetts (1905) (upholding legislated mass vaccinations and
approving of forced quarantines of Americans even if they show no signs
of illness); cf.~Kansas v. Hendricks; Juragua Iron Co.~v. United States.

The Government's asserted authority to detain an individual that the
President has determined to be an enemy combatant, at least while
hostilities continue, comports with the Due Process Clause. As these
cases also show, the Executive's decision that a detention is necessary
to protect the public need not and should not be subjected to judicial
second-guessing. Indeed, at least in the context of enemy-combatant
determinations, this would defeat the unity, secrecy, and dispatch that
the Founders believed to be so important to the warmaking function. See
Part I.

I therefore cannot agree with JUSTICE SCALIA'S conclusion that the
Government must choose between using standard criminal processes and
suspending the writ. See (dissenting opinion). JUSTICE SCALIA relies
heavily upon Ex parte Milligan, 4 Wall. 2 (1866), see -572, and three
cases decided by New York state courts in the wake of the War of 1812,
see I admit that Milligan supports his position. But because the
Executive Branch there, unlike here, did not follow a specific statutory
mechanism provided by Congress, the Court did not need to reach the
broader question of Congress' power, and its discussion on this point
was arguably dicta, see 4 Wall., as four Justices believed, see 134-136
(Chase, C. J., joined by Wayne, Swayne, and Miller, JJ., concurring in
judgment).

More importantly, the Court referred frequently and pervasively to the
criminal nature of the proceedings instituted against Milligan. In fact,
this feature serves to distinguish the state cases as well. See In re
Stacy, 10 Johns. (N. Y. 1813) (``A military commander is here assuming
criminal jurisdiction over a private citizen'' (emphasis added)); Smith
v. Shaw, 12 Johns. (N. Y. 1815) (Shaw ``might be amenable to the civil
authority for treason; but could not be punished, under martial law, as
a spy'' (emphasis added)); M'Connell v. Hampton, 12 Johns. (N. Y. 1815)
(same for treason).

Although I do acknowledge that the reasoning of these cases might apply
beyond criminal punishment, the punishment-nonpunishment distinction
harmonizes all of the precedent. And, subsequent cases have at least
implicitly distinguished Milligan in just this way. See, e. g., Moyer
(``Such arrests are not necessarily for punishment, but are by way of
precaution''). Finally, Quirin overruled Milligan to the extent that
those cases are inconsistent. See Quirin (limiting Milligan to its
facts). Because the Government does not detain Hamdi in order to punish
him, as the plurality acknowledges, see Milligan and the New York cases
do not control.

JUSTICE SCALIA also finds support in a letter Thomas Jefferson wrote to
James Madison. See I agree that this provides some evidence for his
position. But I think this plainly insufficient to rebut the authorities
upon which I have relied. In any event, I do not believe that JUSTICE
SCALIA'S evidence leads to the necessary ``clear conviction that {[}the
detention is{]} in conflict with the Constitution or laws of Congress
constitutionally enacted,'' Quirin, to justify nullifying the
President's wartime action.

Finally, JUSTICE SCALIA'S position raises an additional concern. JUSTICE
SCALIA apparently does not disagree that the Federal Government has all
power necessary to protect the Nation. If criminal processes do not
suffice, however, JUSTICE SCALIA would require Congress to suspend the
writ. See But the fact that the writ may not be suspended ``unless when
in Cases of Rebellion or Invasion the public Safety may require it,''
Art. I, § 9, cl. 2, poses two related problems. First, this condition
might not obtain here or during many other emergencies during which this
detention authority might be necessary. Congress would then have to
choose between acting unconstitutionally4 and

depriving the President of the tools he needs to protect the Nation.
Second, I do not see how suspension would make constitutional otherwise
unconstitutional detentions ordered by the President. It simply removes
a remedy. JUSTICE SCALIA'S position might therefore require one or both
of the political branches to act unconstitutionally in order to protect
the Nation. But the power to protect the Nation must be the power to do
so lawfully.

Accordingly, I conclude that the Government's detention of Hamdi as an
enemy combatant does not violate the Constitution. By detaining Hamdi,
the President, in the prosecution of a war and authorized by Congress,
has acted well within his authority. Hamdi thereby received all the
process to which he was due under the circumstances. I therefore believe
that this is no occasion to balance the competing interests, as the
plurality unconvincingly attempts to do.

Although I do not agree with the plurality that the balancing approach
of Mathews v. Eldridge, is the appropriate analytical tool with which to
analyze this case,5 I cannot help but explain that the

plurality misapplies its chosen framework, one that if applied correctly
would probably lead to the result I have reached. The plurality devotes
two paragraphs to its discussion of the Government's interest, though
much of those two paragraphs explain why the Government's concerns are
misplaced. See But: ``It is'obvious and unarguable' that no governmental
interest is more compelling than the security of the Nation.'' Agee
(quoting Aptheker). In Moyer, the Court recognized the paramount
importance of the Governor's interest in the tranquility of a Colorado
town. At issue here is the far more significant interest of the security
of the Nation. The Government seeks to further that interest by
detaining an enemy soldier not only to prevent him from rejoining the
ongoing fight. Rather, as the Government explains, detention can serve
to gather critical intelligence regarding the intentions and
capabilities of our adversaries, a function that the Government avers
has become all the more important in the war on terrorism. See Brief for
Respondents 15; App. 347-351.

Additional process, the Government explains, will destroy the
intelligence gathering function. Brief for Respondents 43-45. It also
does seem quite likely that, under the process envisioned by the
plurality, various military officials will have to take time to litigate
this matter. And though the plurality does not say so, a meaningful
ability to challenge the Government's factual allegations will probably
require the Government to divulge highly classified information to the
purported enemy combatant, who might then upon release return to the
fight armed with our most closely held secrets.

The plurality manages to avoid these problems by discounting or entirely
ignoring them. After spending a few sentences putatively describing the
Government's interests, the plurality simply assures the Government that
the alleged burdens ``are properly taken into account in our due process
analysis.'' The plurality also announces that ``the risk of an erroneous
deprivation of a detainee's liberty interest is unacceptably high under
the Government's proposed rule.'' (internal quotation marks omitted).
But there is no particular reason to believe that the federal courts
have the relevant information and expertise to make this judgment. And
for the reasons discussed in Part Ithere is every reason to think that
courts cannot and should not make these decisions.

The plurality next opines that ``{[}w{]}e think it unlikely that this
basic process will have the dire impact on the central functions of
warmaking that the Government forecasts.'' Apparently by limiting
hearings ``to the alleged combatant's acts,'' such hearings
``meddl{[}e{]} little, if at all, in the strategy or conduct of war.''
Of course, the meaning of the combatant's acts may become clear only
after quite invasive and extensive inquiry. And again, the federal
courts are simply not situated to make these judgments.

Ultimately, the plurality's dismissive treatment of the Government's
asserted interests arises from its apparent belief that enemy-combatant
determinations are not part of ``the actual prosecution of a war,'' , or
one of the ``central functions of warmaking,'' This seems wrong: Taking
and holding enemy combatants is a quintessential aspect of the
prosecution of war. See, e. g., ; Quirin. Moreover, this highlights
serious difficulties in applying the plurality's balancing approach
here. First, in the war context, we know neither the strength of the
Government's interests nor the costs of imposing additional process.

Second, it is at least difficult to explain why the result should be
different for other military operations that the plurality would
ostensibly recognize as ``central functions of warmaking.'' As the
plurality recounts:

``Parties whose rights are to be affected are entitled to be heard; and
in order that they may enjoy that right they must first be notified. It
is equally fundamental that the right to notice and an opportunity to be
heard must be granted at a meaningful time and in a meaningful manner.''
(internal quotation marks omitted).

See also (``notice'' of the Government's factual assertions and ``a fair
opportunity to rebut {[}those{]} assertions before a neutral
decisionmaker'' are essential elements of due process). Because a
decision to bomb a particular target might extinguish life interests,
the plurality's analysis seems to require notice to potential targets.
To take one more example, in November 2002, a Central Intelligence
Agency (CIA) Predator drone fired a Hellfire missile at a vehicle in
Yemen carrying an al Qaeda leader, a citizen of the United States, and
four others. See Priest, CIA Killed U. S. Citizen In Yemen Missile
Strike, Washington Post, Nov.~8, 2002, p.~A1. It is not clear whether
the CIA knew that an American was in the vehicle. But the plurality's
due process would seem to require notice and opportunity to respond here
as well. Cf. Tennessee v. Garner, 471 U. S. 1 (1985). I offer these
examples not because I think the plurality would demand additional
process in these situations but because it clearly would not. The result
here should be the same.

I realize that many military operations are, in some sense, necessary.
But many, if not most, are merely expedient, and I see no principled
distinction between the military operation the plurality condemns today
(the holding of an enemy combatant based on the process given Hamdi)
from a variety of other military operations. In truth, I doubt that
there is any sensible, bright-line distinction. It could be argued that
bombings and missile strikes are an inherent part of war, and as long as
our forces do not violate the laws of war, it is of no constitutional
moment that civilians might be killed. But this does not serve to
distinguish this case because it is also consistent with the laws of war
to detain enemy combatants exactly as the Government has detained Hamdi.

This, in fact, bolsters my argument in Part

to the extent that the laws of war show that the power to detain is part
of a sovereign's war powers.

Undeniably, Hamdi has been deprived of a serious interest, one actually
protected by the Due Process Clause. Against this, however, is the
Government's overriding interest in protecting the Nation. If a
deprivation of liberty can be justified by the need to protect a town,
the protection of the Nation, a fortiori, justifies it.

I acknowledge that under the plurality's approach, it might, at times,
be appropriate to give detainees access to counsel and notice of the
factual basis for the Government's determination. See But properly
accounting for the Government's interests also requires concluding that
access to counsel and to the factual basis would not always be
warranted. Though common sense suffices, the Government thoroughly
explains that counsel would often destroy the intelligence gathering
function. See Brief for Respondents 42-43. See also App. 347-351
(affidavit of Col. D. Woolfolk). Equally obvious is the Government's
interest in not fighting the war in its own courts, see, e. g., Johnson
v. Eisentrager, and protecting classified information, see, e. g.,
Department of Navy v. Egan (President's ``authority to classify and
control access to information bearing on national security and to
determine'' who gets access ``flows primarily from {[}the Commander in
Chief Clause{]} and exists quite apart from any explicit congressional
grant''); Agee (upholding revocation of former CIA employee's passport
in large part by reference to the Government's need ``to protect the
secrecy of {[}its{]} foreign intelligence operations'').

For these reasons, I would affirm the judgment of the Court of Appeals.
---------------

Notes:

\begin{enumerate}
\def\labelenumi{\arabic{enumi}.}
\item
  Although I have emphasized national-security concerns, the President's
  foreign-affairs responsibilities are also squarely implicated by this
  case. The Government avers that Northern Alliance forces captured
  Hamdi, and the District Court demanded that the Government turn over
  information relating to statements made by members of the Northern
  Alliance. See CA4 2003).
\item
  It could be argued that the habeas statutes are evidence of
  congressional intent that enemy combatants are entitled to challenge
  the factual basis for the Government's determination. See, e.g., 28
  U.S.C. §§ 2243, 2246. But factual development is needed only to the
  extent necessary to resolve the legal challenge to the detention. See,
  e.g., Walker v. Johnston.
\item
  Indeed, it is not even clear that the Court required good faith. See
  Moyer (``It is not alleged that {[}the Governor's{]} judgment was not
  honest, if that be material, or that {[}Moyer{]} was detained after
  fears of the insurrection were at an end'').
\item
  I agree with JUSTICE SCALIA that this Court could not review Congress'
  decision to suspend the writ. See
\item
  Evidently, neither do the parties, who do not cite Mathews even once.
\item
  Hamdi's detention comports with the laws of war, including the Geneva
  Convention (III) Relative to the Treatment of Prisoners of War,
  Aug.~12, 1949, 6 U. S. T. 3406, T. I. A. S. No.~3364. See Brief for
  Respondents 22-24. 7. These observations cast still more doubt on the
  appropriateness and usefulness of Mathews v. Eldridge, in this
  context. It is, for example, difficult to see how the plurality can
  insist that Hamdi unquestionably has the right to access to counsel in
  connection with the proceedings on remand, when new information could
  become available to the Government showing that such access would pose
  a grave risk to national security. In that event, would the Government
  need to hold a hearing before depriving Hamdi of his newly acquired
  right to counsel even if that hearing would itself pose a grave
  threat?
\end{enumerate}

\begin{center}\rule{0.5\linewidth}{\linethickness}\end{center}

\hypertarget{substantive-due-process}{%
\section{Substantive Due Process}\label{substantive-due-process}}

\hypertarget{incorporation}{%
\subsection{Incorporation}\label{incorporation}}

\hypertarget{slaughter-house-cases}{%
\subsubsection{Slaughter-House Cases}\label{slaughter-house-cases}}

83 U.S. 36 (1872)

(Butchers' Benevolent Ass'n v. Crescent City Live-Stock Landing \&
Slaughter-House Co.)

\textbf{Mr.~Justice MILLER, now, April 14th, 1873, delivered the opinion
of the court.} These cases are brought here by writs of error to the
Supreme Court of the State of Louisiana. They arise out of the efforts
of the butchers of New Orleans to resist the Crescent City Live-Stock
Landing and Slaughter-House Company in the exercise of certain powers
conferred by the charter which created it, and which was granted by the
legislature of that State.

The cases named on a preceding page,11 with others which have been
brought here and dismissed by agreement, were all decided by the Supreme
Court of Louisiana in favor of the Slaughter-House Company, as we shall
hereafter call it for the sake of brevity, and these writs are brought
to reverse those decisions.

The records were filed in this court in 1870, and were argued before it
as length on a motion made by plaintiffs in error for an order in the
nature of an injunction or supersedeas, pending the action of the court
on the merits. The opinion on that motion is reported in 10 Wallace,
273.

On account of the importance of the questions involved in these cases
they were, by permission of the court, taken up out of their order on
the docket and argued in January, 1872. At that hearing one of the
justices was absent, and it was found, on consultation, that there was a
diversity of views among those who were present. Impressed with the
gravity of the questions raised in the argument, the court under these
circumstances ordered that the cases be placed on the calendar and
reargued before a full bench. This argument was had early in February
last.

Preliminary to the consideration of those questions is a motion by the
defendant to dismiss the cases, on the ground that the contest between
the parties has been adjusted by an agreement made since the records
came into this court, and that part of that agreement is that these
writs should be dismissed. This motion was heard with the argument on
the merits, and was much pressed by counsel. It is supported by
affidavits and by copies of the written agreement relied on. It is
sufficient to say of these that we do not find in them satisfactory
evidence that the agreement is binding upon all the parties to the
record who are named as plaintiffs in the several writs of error, and
that there are parties now before the court, in each of the three cases,
the names of which appear on a preceding page,12 who have not consented
to their dismissal, and who are not bound by the action of those who
have so consented. They have a right to be heard, and the motion to
dismiss cannot prevail.

The records show that the plaintiffs in error relied upon, and asserted
throughout the entire course of the litigation in the State courts, that
the grant of privileges in the charter of defendant, which they were
contesting, was a violation of the most important provisions of the
thirteenth and fourteenth articles of amendment of the Constitution of
the United States. The jurisdiction and the duty of this court to review
the judgment of the State court on those questions is clear and is
imperative.

The statute thus assailed as unconstitutional was passed March 8th,
1869, and is entitled `An act to protect the health of the city of New
Orleans, to locate the stock-landings and slaughter-houses, and to
incorporate the Crescent City Live-Stock Landing and Slaughter-House
Company.'

The first section forbids the landing or slaughtering of animals whose
flesh is intended for food, within the city of New Orleans and other
parishes and boundaries named and defined, or the keeping or
establishing any slaughter-houses or abattoirs within those limits
except by the corporation thereby created, which is also limited to
certain places afterwards mentioned. Suitable penalties are enacted for
violations of this prohibition.

The second section designates the corporators, gives the name to the
corporation, and confers on it the usual corporate powers.

The third and fourth sections authorize the company to establish and
erect within certain territorial limits, therein defined, one or more
stock-yards, stock-landings, and slaughter-houses, and imposes upon it
the duty of erecting, on or before the first day of June, 1869, one
grand slaughter-house of sufficient capacity for slaughtering five
hundred animals per day.

It declares that the company, after it shall have prepared all the
necessary buildings, yards, and other conveniences for that purpose,
shall have the sole and exclusive privilege of conducting and carrying
on the live-stock landing and slaughter-house business within the limits
and privilege granted by the act, and that all such animals shall be
landed at the stock-landings and slaughtered at the slaughter-houses of
the company, and nowhere else. Penalties are enacted for infractions of
this provision, and prices fixed for the maximum charges of the company
for each steamboat and for each animal landed.

Section five orders the closing up of all other stock-landings and
slaughter-houses after the first day of June, in the parishes of
Orleans, Jefferson, and St.~Bernard, and makes it the duty of the
company to permit any person to slaughter animals in their
slaughter-houses under a heavy penalty for each refusal. Another section
fixes a limit to the charges to be made by the company for each animal
so slaughtered in their building, and another provides for an inspection
of all animals intended to be so slaughtered, by an officer appointed by
the governor of the State for that purpose.

These are the principal features of the statute, and are all that have
any bearing upon the questions to be decided by us.

This statute is denounced not only as creating a monopoly and conferring
odious and exclusive privileges upon a small number of persons at the
expense of the great body of the community of New Orleans, but it is
asserted that it deprives a large and meritorious class of
citizens---the whole of the butchers of the city---of the right to
exercise their trade, the business to which they have been trained and
on which they depend for the support of themselves and their families,
and that the unrestricted exercise of the business of butchering is
necessary to the daily subsistence of the population of the city.

It is, however, the slaughter-house privilege, which is mainly relied on
to justify the charges of gross injustice to the public, and invasion of
private right.

The wisdom of the monopoly granted by the legislature may be open to
question, but it is difficult to see a justification for the assertion
that the butchers are deprived of the right to labor in their
occupation, or the people of their daily service in preparing food, or
how this statute, with the duties and guards imposed upon the company,
can be said to destroy the business of the butcher, or seriously
interfere with its pursuit.

The power here exercised by the legislature of Louisiana is, in its
essential nature, one which has been, up to the present period in the
constitutional history of this country, always conceded to belong to the
States, however it may now be questioned in some of its details.

The plaintiffs in error accepting this issue, allege that the statute is
a violation of the Constitution of the United States in these several
particulars:

That it creates an involuntary servitude forbidden by the thirteenth
article of amendment; That it abridges the privileges and immunities of
citizens of the United States; That it denies to the plaintiffs the
equal protection of the laws; and, That it deprives them of their
property without due process of law; contrary to the provisions of the
first section of the fourteenth article of amendment.

This court is thus called upon for the first time to give construction
to these articles.

The most cursory glance at these articles discloses a unity of purpose,
when taken in connection with the history of the times, which cannot
fail to have an important bearing on any question of doubt concerning
their true meaning. Nor can such doubts, when any reasonably exist, be
safely and rationally solved without a reference to that history; for in
it is found the occasion and the necessity for recurring again to the
great source of power in this country, the people of the States, for
additional guarantees of human rights; additional powers to the Federal
government; additional restraints upon those of the States. Fortunately
that history is fresh within the memory of us all, and its leading
features, as they bear upon the matter before us, free from doubt.

The institution of African slavery, as it existed in about half the
States of the Union, and the contests pervading the public mind for many
years, between those who desired its curtailment and ultimate extinction
and those who desired additional safeguards for its security and
perpetuation, culminated in the effort, on the part of most of the
States in which slavery existed, to separate from the Federal
government, and to resist its authority. This constituted the war of the
rebellion, and whatever auxiliary causes may have contributed to bring
about this war, undoubtedly the overshadowing and efficient cause was
African slavery.

In that struggle slavery, as a legalized social relation, perished. It
perished as a necessity of the bitterness and force of the conflict.
When the armies of freedom found themselves upon the soil of slavery
they could do nothing less than free the poor victims whose enforced
servitude was the foundation of the quarrel. And when hard pressed in
the contest these men (for they proved themselves men in that terrible
crisis) offered their services and were accepted by thousands to aid in
suppressing the unlawful rebellion, slavery was at an end wherever the
Federal government succeeded in that purpose. The proclamation of
President Lincoln expressed an accomplished fact as to a large portion
of the insurrectionary districts, when he declared slavery abolished in
them all. But the war being over, those who had succeeded in
re-establishing the authority of the Federal government were not content
to permit this great act of emancipation to rest on the actual results
of the contest or the proclamation of the Executive, both of which might
have been questioned in after times, and they determined to place this
main and most valuable result in the Constitution of the restored Union
as one of its fundamental articles. Hence the thirteenth article of
amendment of that instrument.

Among the first acts of legislation adopted by several of the States in
the legislative bodies which claimed to be in their normal relations
with the Federal government, were laws which imposed upon the colored
race onerous disabilities and burdens, and curtailed their rights in the
pursuit of life, liberty, and property to such an extent that their
freedom was of little value, while they had lost the protection which
they had received from their former owners from motives both of interest
and humanity.

They were in some States forbidden to appear in the towns in any other
character than menial servants. They were required to reside on and
cultivate the soil without the right to purchase or own it. They were
excluded from many occupations of gain, and were not permitted to give
testimony in the courts in any case where a white man was a party. It
was said that their lives were at the mercy of bad men, either because
the laws for their protection were insufficient or were not enforced.

These circumstances, whatever of falsehood or misconception may have
been mingled with their presentation, forced upon the statesmen who had
conducted the Federal government in safety through the crisis of the
rebellion, and who supposed that by the thirteenth article of amendment
they had secured the result of their labors, the conviction that
something more was necessary in the way of constitutional protection to
the unfortunate race who had suffered so much. They accordingly passed
through Congress the proposition for the fourteenth amendment, and they
declined to treat as restored to their full participation in the
government of the Union the States which had been in insurrection, until
they ratified that article by a formal vote of their legislative bodies.

We repeat, then, in the light of this recapitulation of events, almost
too recent to be called history, but which are familiar to us all; and
on the most casual examination of the language of these amendments, no
one can fail to be impressed with the one pervading purpose found in
them all, lying at the foundation of each, and without which none of
them would have been even suggested; we mean the freedom of the slave
race, the security and firm establishment of that freedom, and the
protection of the newly-made freeman and citizen from the oppressions of
those who had formerly exercised unlimited dominion over him. It is true
that only the fifteenth amendment, in terms, mentions the negro by
speaking of his color and his slavery. But it is just as true that each
of the other articles was addressed to the grievances of that race, and
designed to remedy them as the fifteenth.

We do not say that no one else but the negro can share in this
protection. Both the language and spirit of these articles are to have
their fair and just weight in any question of construction. Undoubtedly
while negro slavery alone was in the mind of the Congress which proposed
the thirteenth article, it forbids any other kind of slavery, now or
hereafter. If Mexican peonage or the Chinese coolie labor system shall
develop slavery of the Mexican or Chinese race within our territory,
this amendment may safely be trusted to make it void. And so if other
rights are assailed by the States which properly and necessarily fall
within the protection of these articles, that protection will apply,
though the party interested may not be of African descent. But what we
do say, and what we wish to be understood is, that in any fair and just
construction of any section or phrase of these amendments, it is
necessary to look to the purpose which we have said was the pervading
spirit of them all, the evil which they were designed to remedy, and the
process of continued addition to the Constitution, until that purpose
was supposed to be accomplished, as far as constitutional law can
accomplish it.

It is quite clear, then, that there is a citizenship of the United
States, and a citizenship of a State, which are distinct from each
other, and which depend upon different characteristics or circumstances
in the individual.

We think this distinction and its explicit recognition in this amendment
of great weight in this argument, because the next paragraph of this
same section, which is the one mainly relied on by the plaintiffs in
error, speaks only of privileges and immunities of citizens of the
United States, and does not speak of those of citizens of the several
States. The argument, however, in favor of the plaintiffs rests wholly
on the assumption that the citizenship is the same, and the privileges
and immunities guaranteed by the clause are the same.

The language is, `No State shall make or enforce any law which shall
abridge the privileges or immunities of citizens of the United States.'
It is a little remarkable, if this clause was intended as a protection
to the citizen of a State against the legislative power of his own
State, that the word citizen of the State should be left out when it is
so carefully used, and used in contradistinction to citizens of the
United States, in the very sentence which precedes it. It is too clear
for argument that the change in phraseology was adopted understandingly
and with a purpose.

Of the privileges and immunities of the citizen of the United States,
and of the privileges and immunities of the citizen of the State, and
what they respectively are, we will presently consider; but we wish to
state here that it is only the former which are placed by this clause
under the protection of the Federal Constitution, and that the latter,
whatever they may be, are not intended to have any additional protection
by this paragraph of the amendment.

If, then, there is a difference between the privileges and immunities
belonging to a citizen of the United States as such, and those belonging
to the citizen of the State as such the latter must rest for their
security and protection where they have heretofore rested; for they are
not embraced by this paragraph of the amendment.

The first occurrence of the words `privileges and immunities' in our
constitutional history, is to be found in the fourth of the articles of
the old Confederation.

It declares `that the better to secure and perpetuate mutual friendship
and intercourse among the people of the different States in this Union,
the free inhabitants of each of these States, paupers, vagabonds, and
fugitives from justice excepted, shall be entitled to all the privileges
and immunities of free citizens in the several States; and the people of
each State shall have free ingress and regress to and from any other
State, and shall enjoy therein all the privileges of trade and commerce,
subject to the same duties, impositions, and restrictions as the
inhabitants thereof respectively.'

In the Constitution of the United States, which superseded the Articles
of Confederation, the corresponding provision is found in section two of
the fourth article, in the following words: `The citizens of each State
shall be entitled to all the privileges and immunities of citizens of
the several States.'

There can be but little question that the purpose of both these
provisions is the same, and that the privileges and immunities intended
are the same in each. In the article of the Confederation we have some
of these specifically mentioned, and enough perhaps to give some general
idea of the class of civil rights meant by the phrase.

Fortunately we are not without judicial construction of this clause of
the Constitution. The first and the leading case on the subject is that
of Corfield v. Coryell, decided by Mr.~Justice Washington in the Circuit
Court for the District of Pennsylvania in 1823 `The inquiry,' he says,
`is, what are the privileges and immunities of citizens of the several
States? We feel no hesitation in confining these expressions to those
privileges and immunities which are fundamental; which belong of right
to the citizens of all free governments, and which have at all times
been enjoyed by citizens of the several States which compose this Union,
from the time of their becoming free, independent, and sovereign. What
these fundamental principles are, it would be more tedious than
difficult to enumerate. They may all, however, be comprehended under the
following general heads: protection by the government, with the right to
acquire and possess property of every kind, and to pursue and obtain
happiness and safety, subject, nevertheless, to such restraints as the
government may prescribe for the general good of the whole.'

This definition of the privileges and immunities of citizens of the
States is adopted in the main by this court in the recent case of Ward
v. The State of Maryland,23 while it declines to undertake an
authoritative definition beyond what was necessary to that decision. The
description, when taken to include others not named, but which are of
the same general character, embraces nearly every civil right for the
establishment and protection of which organized government is
instituted. They are, in the language of Judge Washington, those rights
which the fundamental. Throughout his opinion, they are spoken of as
rights belonging to the individual as a citizen of a State. They are so
spoken of in the constitutional provision which he was construing. And
they have always been held to be the class of rights which the State
governments were created to establish and secure.

In the case of Paul v. Virginia,24 the court, in expounding this clause
of the Constitution, says that `the privileges and immunities secured to
citizens of each State in the several States, by the provision in
question, are those privileges and immunities which are common to the
citizens in the latter States under their constitution and laws by
virtue of their being citizens.'

The constitutional provision there alluded to did not create those
rights, which it called privileges and immunities of citizens of the
States. It threw around them in that clause no security for the citizen
of the State in which they were claimed or exercised. Nor did it profess
to control the power of the State governments over the rights of its own
citizens.

Its sole purpose was to declare to the several States, that whatever
those rights, as you grant or establish them to your own citizens, or as
you limit or qualify, or impose restrictions on their exercise, the
same, neither more nor less, shall be the measure of the rights of
citizens of other States within your jurisdiction.

It would be the vainest show of learning to attempt to prove by
citations of authority, that up to the adoption of the recent
amendments, no claim or pretence was set up that those rights depended
on the Federal government for their existence or protection, beyond the
very few express limitations which the Federal Constitution imposed upon
the States---such, for instance, as the prohibition against ex post
facto laws, bills of attainder, and laws impairing the obligation of
contracts. But with the exception of these and a few other restrictions,
the entire domain of the privileges and immunities of citizens of the
States, as above defined, lay within the constitutional and legislative
power of the States, and without that of the Federal government. Was it
the purpose of the fourteenth amendment, by the simple declaration that
no State should make or enforce any law which shall abridge the
privileges and immunities of citizens of the United States, to transfer
the security and protection of all the civil rights which we have
mentioned, from the States to the Federal government? And where it is
declared that Congress shall have the power to enforce that article, was
it intended to bring within the power of Congress the entire domain of
civil rights heretofore belonging exclusively to the States?

All this and more must follow, if the proposition of the plaintiffs in
error be sound. For not only are these rights subject to the control of
Congress whenever in its discretion any of them are supposed to be
abridged by State legislation, but that body may also pass laws in
advance, limiting and restricting the exercise of legislative power by
the States, in their most ordinary and usual functions, as in its
judgment it may think proper on all such subjects. And still further,
such a construction followed by the reversal of the judgments of the
Supreme Court of Louisiana in these cases, would constitute this court a
perpetual censor upon all legislation of the States, on the civil rights
of their own citizens, with authority to nullify such as it did not
approve as consistent with those rights, as they existed at the time of
the adoption of this amendment. The argument we admit is not always the
most conclusive which is drawn from the consequences urged against the
adoption of a particular construction of an instrument. But when, as in
the case before us, these consequences are so serious, so far-reaching
and pervading, so great a departure from the structure and spirit of our
institutions; when the effect is to fetter and degrade the State
governments by subjecting them to the control of Congress, in the
exercise of powers heretofore universally conceded to them of the most
ordinary and fundamental character; when in fact it radically changes
the whole theory of the relations of the State and Federal governments
to each other and of both these governments to the people; the argument
has a force that is irresistible, in the absence of language which
expresses such a purpose too clearly to admit of doubt.

We are convinced that no such results were intended by the Congress
which proposed these amendments, nor by the legislatures of the States
which ratified them.

Having shown that the privileges and immunities relied on in the
argument are those which belong to citizens of the States as such, and
that they are left to the State governments for security and protection,
and not by this article placed under the special care of the Federal
government, we may hold ourselves excused from defining the privileges
and immunities of citizens of the United States which no State can
abridge, until some case involving those privileges may make it
necessary to do so.

But lest it should be said that no such privileges and immunities are to
be found if those we have been considering are excluded, we venture to
suggest some which own their existence to the Federal government, its
National character, its Constitution, or its laws.

One of these is well described in the case of Crandall v. Nevada It is
said to be the right of the citizen of this great country, protected by
implied guarantees of its Constitution, `to come to the seat of
government to assert any claim he may have upon that government, to
transact any business he may have with it, to seek its protection, to
share its offices, to engage in administering its functions. He has the
right of free access to its seaports, through which all operations of
foreign commerce are conducted, to the subtreasuries, land offices, and
courts of justice in the several States.' And quoting from the language
of Chief Justice Taney in another case, it is said `that for all the
great purposes for which the Federal government was established, we are
one people, with one common country, we are all citizens of the United
States;' and it is, as such citizens, that their rights are supported in
this court in Crandall v. Nevada.

Another privilege of a citizen of the United States is to demand the
care and protection of the Federal government over his life, liberty,
and property when on the high seas or within the jurisdiction of a
foreign government. Of this there can be no doubt, nor that the right
depends upon his character as a citizen of the United States. The right
to peaceably assemble and petition for redress of grievances, the
privilege of the writ of habeas corpus, are rights of the citizen
guaranteed by the Federal Constitution. The right to use the navigable
waters of the United States, however they may penetrate the territory of
the several States, all rights secured to our citizens by treaties with
foreign nations, are dependent upon citizenship of the United States,
and not citizenship of a State. One of these privileges is conferred by
the very article under consideration. It is that a citizen of the United
States can, of his own volition, become a citizen of any State of the
Union by a bon a fide residence therein, with the same rights as other
citizens of that State. To these may be added the rights secured by the
thirteenth and fifteenth articles of amendment, and by the other clause
of the fourteenth, next to be considered.

In the early history of the organization of the government, its
statesmen seem to have divided on the line which should separate the
powers of the National government from those of the State governments,
and though this line has never been very well defined in public opinion,
such a division has continued from that day to this.

The adoption of the first eleven amendments to the Constitution so soon
after the original instrument was accepted, shows a prevailing sense of
danger at that time from the Federal power. And it cannot be denied that
such a jealousy continued to exist with many patriotic men until the
breaking out of the late civil war. It was then discovered that the true
danger to the perpetuity of the Union was in the capacity of the State
organizations to combine and concentrate all the powers of the State,
and of contiguous States, for a determined resistance to the General
Government.

Unquestionably this has given great force to the argument, and added
largely to the number of those who believe in the necessity of a strong
National government.

But, however pervading this sentiment, and however it may have
contributed to the adoption of the amendments we have been considering,
we do not see in those amendments any purpose to destroy the main
features of the general system. Under the pressure of all the excited
feeling growing out of the war, our statesmen have still believed that
the existence of the State with powers for domestic and local
government, including the regulation of civil rights---the rights of
person and of property---was essential to the perfect working of our
complex form of government, though they have thought proper to impose
additional limitations on the States, and to confer additional power on
that of the Nation.

But whatever fluctuations may be seen in the history of public opinion
on this subject during the period of our national existence, we think it
will be found that this court, so far as its functions required, has
always held with a steady and an even hand the balance between State and
Federal power, and we trust that such may continue to be the history of
its relation to that subject so long as it shall have duties to perform
which demand of it a construction of the Constitution, or of any of its
parts.

\hypertarget{adamson-v.-california}{%
\subsubsection{Adamson v. California}\label{adamson-v.-california}}

332 U.S. 46 (1947)

\textbf{Mr.~Justice Reed delivered the opinion of the Court.}

The appellant, Adamson, a citizen of the United States, was convicted,
without recommendation for mercy, by a jury in a Superior Court of the
State of California of murder in the first degree After considering the
same objections to the conviction that are pressed here, the sentence of
death was affirmed by the Supreme Court of the state. 27 Cal. 2d 478,
165 P. 2d 3. Review of that judgment by this Court was sought and
allowed under Judicial Code § 237; 28 U. S. C. § 344 The provisions of
California law which were challenged in the state proceedings as invalid
under the Fourteenth Amendment to the Federal Constitution are those of
the state constitution and penal code in the margin. They permit the
failure of a defendant to explain or to deny evidence against him to be
commented upon by court and by counsel and to be considered by court and
jury The defendant did not testify. As the trial court gave its
instructions and the District Attorney argued the case in accordance
with the constitutional and statutory provisions just referred to, we
have for decision the question of their constitutionality in these
circumstances under the limitations of § 1 of the Fourteenth Amendment.

The appellant was charged in the information with former convictions for
burglary, larceny and robbery and pursuant to § 1025, California Penal
Code, answered that he had suffered the previous convictions. This
answer barred allusion to these charges of convictions on the trial
Under California's interpretation of § 1025 of the Penal Code and § 2051
of the Code of Civil Procedure, however, if the defendant, after
answering affirmatively charges alleging prior convictions, takes the
witness stand to deny or explain away other evidence that has been
introduced ``the commission of these crimes could have been revealed to
the jury on cross-examination to impeach his testimony.'' People v.
Adamson, 27 Cal. 2d 478, 494, 165 P. 2d 3, 11; People v. Braun, 14 Cal.
2d 1, 6, 92 P. 2d 402, 405. This forces an accused who is a repeated
offender to choose between the risk of having his prior offenses
disclosed to the jury or of having it draw harmful inferences from
uncontradicted evidence that can only be denied or explained by the
defendant.

In the first place, appellant urges that the provision of the Fifth
Amendment that no person ``shall be compelled in any criminal case to be
a witness against himself'' is a fundamental national privilege or
immunity protected against state abridgment by the Fourteenth Amendment
or a privilege or immunity secured, through the Fourteenth Amendment,
against deprivation by state action because it is a personal right,
enumerated in the federal Bill of Rights.

Secondly, appellant relies upon the due process of law clause of the
Fourteenth Amendment to invalidate the provisions of the California law,
set out in note 3and as applied (a) because comment on failure to
testify is permitted, (b) because appellant was forced to forego
testimony in person because of danger of disclosure of his past
convictions through cross-examination, and (c) because the presumption
of innocence was infringed by the shifting of the burden of proof to
appellant in permitting comment on his failure to testify.

We shall assume, but without any intention thereby of ruling upon the
issue,6 that permission by law to the court, counsel and jury to comment
upon and consider the failure of defendant ``to explain or to deny by
his testimony any evidence or facts in the case against him'' would
infringe defendant's privilege against self-incrimination under the
Fifth Amendment if this were a trial in a court of the United States
under a similar law. Such an assumption does not determine appellant's
rights under the Fourteenth Amendment. It is settled law that the clause
of the Fifth Amendment, protecting a person against being compelled to
be a witness against himself, is not made effective by the Fourteenth
Amendment as a protection against state action on the ground that
freedom from testimonial compulsion is a right of national citizenship,
or because it is a personal privilege or immunity secured by the Federal
Constitution as one of the rights of man that are listed in the Bill of
Rights.

The reasoning that leads to those conclusions starts with the
unquestioned premise that the Bill of Rights, when adopted, was for the
protection of the individual against the federal government and its
provisions were inapplicable to similar actions done by the states.
Barron v. Baltimore, 7 Pet. 243; Feldman v. United States. With the
adoption of the Fourteenth Amendment, it was suggested that the dual
citizenship recognized by its first sentence7 secured for citizens
federal protection for their elemental privileges and immunities of
state citizenship. The Slaughter-House Cases8 decided, contrary to the
suggestion, that these rights, as privileges and immunities of state
citizenship, remained under the sole protection of the state
governments. This Court, without the expression of a contrary view upon
that phase of the issues before the Court, has approved this
determination. Maxwell v. Bugbee; Hamilton v. Regents. The power to free
defendants in state trials.from self-incrimination was specifically
determined to be beyond the scope of the privileges and immunities
clause of the Fourteenth Amendment in Twining v. New Jersey. ``The
privilege against self-incrimination may be withdrawn and the accused
put upon the stand as a witness for the state.'' 9 The Twining case
likewise disposed of the contention that freedom from testimonial
compulsion, being specifically granted by the Bill of Rights, is a
federal privilege or immunity that is protected by the Fourteenth
Amendment against state invasion. This Court held that the inclusion in
the Bill of Rights of this protection against the power of the national
government did not make the privilege a federal privilege or immunity
secured to citizens by the Constitution against state action. Twining v.
New Jersey; Palko v. Connecticut. After declaring that state and
national citizenship coexist in the same person, the Fourteenth
Amendment forbids a state from abridging the privileges and immunities
of citizens of the United States. As a matter of words, this leaves a
state free to abridge, within the limits of the due process clause, the
privileges and immunities flowing from state citizenship. This reading
of the Federal Constitution has heretofore found favor with the majority
of this Court as a natural and logical interpretation. It accords with
the constitutional doctrine of federalism by leaving to the states the
responsibility of dealing with the privileges and immunities of their
citizens except those inherent in national citizenship It is the
construction placed upon the amendment by justices whose own experience
had given them contemporaneous knowledge of the purposes that led to the
adoption of the Fourteenth Amendment. This construction has become
embedded in our federal system as a functioning element in preserving
the balance between national and state power. We reaffirm the conclusion
of the Twining and Palko cases that protection against
self-incrimination is not a privilege or immunity of national
citizenship.

Appellant secondly contends that if the privilege against
self-incrimination is not a right protected by the privileges and
immunities clause of the Fourteenth Amendment against state action, this
privilege, to its full scope under the Fifth Amendment, inheres in the
right to a fair trial. A right to a fair trial is a right admittedly
protected by the due process clause of the Fourteenth Amendment
Therefore, appellant argues, the due process clause of the Fourteenth
Amendment protects his privilege against self-incrimination. The due
process clause of the Fourteenth Amendment, however, does not draw all
the rights of the federal Bill of Rights under its protection. That
contention was made and rejected in Palko v. Connecticut. It was
rejected with citation of the cases excluding several of the rights,
protected by the Bill of Rights, against infringement by the National
Government. Nothing has been called to our attention that either the
framers of the Fourteenth Amendment or the states that adopted intended
its due process clause to draw within its scope the earlier amendments
to the Constitution. Palko held that such provisions of the Bill of
Rights as were ``implicit in the concept of ordered liberty,'' p.~325,
became secure from state interference by the clause. But it held nothing
more.

Specifically, the due process clause does not protect, by virtue of its
mere existence, the accused's freedom from giving testimony by
compulsion in state trials that is secured to him against federal
interference by the Fifth Amendment. Twining v. New Jersey; Palko v.
Connecticutp. 323. For a state to require testimony from an accused is
not necessarily a breach of a state's obligation to give a fair trial.
Therefore, we must examine the effect of the California law applied in
this trial to see whether the comment on failure to testify violates the
protection against state action that the due process clause does grant
to an accused. The due process clause forbids compulsion to testify by
fear of hurt, torture or exhaustion It forbids any other type of
coercion that falls within the scope of due process California follows
Anglo-American legal tradition in excusing defendants in criminal
prosecutions from compulsory testimony. Cf. VIH Wigmore on Evidence (3d
ed.) § 2252. That is a matter of legal policy and not because of the
requirements of due process under the Fourteenth Amendment So our
inquiry is directed, not at the broad question of the constitutionality
of compulsory testimony from the accused under the due process clause,
but to the constitutionality of the provision of the California law that
permits comment upon his failure to testify. It is, of course, logically
possible that while an accused might be required, under appropriate
penalties, to submit himself as a witness without a violation of due
process, comment by judge or jury on inferences to be drawn from his
failure to testify, in jurisdictions where an accused's privilege
against self-incrimination is protected, might deny due process. For
example, a statute might declare that a permitted refusal to testify
would compel an acceptance of the truth of the prosecution's evidence.

Generally, comment on the failure of an accused to testify is forbidden
in American jurisdictions This arises from state constitutional or
statutory provisions similar in character to the federal provisions.
Fifth Amendment and 28 U. S. C. § 632. California, however, is one of a
few states that permit limited comment upon a defendant's failure to
testify That permission is narrow. The California law is set out in note
3 and authorizes comment by court and counsel upon the ``failure of the
defendant to explain or to deny by his testimony any evidence or facts
in the case against him.'' This does not involve any presumption,
rebuttable or irrebuttable, either of guilt or of the truth of any fact,
that is offered in evidence. Compare Tot v. United States. It allows
inferences to be drawn from proven facts. Beeaúse of this clause, the
court can direct the jury's attention to whatever evidence there may be
that a defendant could deny and the prosecution can argue as to
inferences that may be drawn from the accused's failure to testify.
Compare Caminetti v. United States; Raffel v. United States. There is
here no lack of power in the trial court to adjudge and no denial of a
hearing. California has prescribed a method for advising the jury in the
search for truth-. However sound may be the legislative conclusion that
an accused should not be compelled in any criminal case to be a witness
against himself, we see no reason why comment should not be made upon
his silence. It seems quite natural that when a defendant has
opportunity to deny or explain facts and determines not to do so, the
prosecution should bring out the strength of the evidence by commenting
upon defendant's failure to explain or deny it. The prosecution evidence
may be of facts that may be beyond the knowledge of the accused. If so,
his failure to testify would have little if any weight. But the facts
may be such as are necessarily in the knowledge of the accused. In that
case a failure to explain would point to an inability to explain.

Appellant sets out the circumstances of this case, however, to show
coercion and unfairness in permitting comment. The guilty person was not
seen at the place and time of the crime. There was evidence, however,
that entrance to the place or room where the crime was committed might
have been obtained through a small door. It was freshly broken. Evidence
showed that six fingerprints on the door were petitioner's. Certain
diamond rings were missing from the deceased's possession. There was
evidence that appellant, sometime after the crime, asked an unidentified
person whether the latter would be interested in purchasing a diamond
ring. As has been stated, the information charged other crimes to
appellant and he admitted them. His argument here is that he could not
take the stand to deny the evidence against him because he would be
subjected to a cross-examination as to former crimes to impeach his
veracity and the evidence so produced might well bring about his
conviction. Such cross-examination is allowable in California. People v.
Adamson, 27 Cal. 2d 478, 494, 165 P. 2d 3, 11. Therefore, appellant
contends the California statute permitting comment denies him due
process.

It is true that if comment were forbidden, an accused in this situation
could remain silent and avoid evidence of former crimes and comment upon
his failure to testify. We are of the view, however, that a state may
control such a situation in accordance with its own ideas of the most
efficient administration of criminal justice. The purpose of due process
is not to protect an accused against a proper conviction but against an
unfair conviction. When evidence is before a jury that threatens
conviction, it does not seem unfair to require him to choose between
leaving the adverse evidence unexplained and subjecting himself to
impeachment through disclosure of former crimes. Indeed, this is a
dilemma with which any defendant may be faced. If facts, adverse to the
defendant, are proven by the prosecution, there may be no way to explain
them favorably to the accused except by a witness who may be vulnerable
to impeachment on cross-examination. The defendant must then decide
whether or not to use such a witness. The fact that the witness may also
be the defendant makes the choice more difficult but a denial of due
process does not emerge from the circumstances.

There is no basis in the California law for appellant's objection on due
process or other grounds that the statutory authorization to comment on
the failure to explain or deny adverse testimony shifts the burden of
proof or the duty to go forward with the evidence. Failure of the
accused to testify is not an admission of the truth of the adverse
evidence. Instructions told the jury that the burden of proof remained
upon the state and the presumption of innocence with the accused.
Comment on failure to deny proven facts does not in California tend to
supply any missing element of proof of guilt. People v. Adamson, 27 Cal.
2d 478, 489P. 2d 3, 9-12. It only directs attention to the strength of
the evidence for the prosecution or to the weakness of that for
the-defense. The Supreme Court of California called attention to the
fact that the prosecutor's argument approached the borderline in a
statement that might have been construed as asserting ``that the jury
should infer guilt solely from defendant's silence.'' That court felt
that it was improbable the jury was misled into such an understanding of
their power. We shall not interfere with such a conclusion. People v.
Adamson, 27 Cal. 2d 478, 494P. 2d 3, 12.

We find no other error that gives ground for our intervention in
California's administration of criminal justice.

Affirmed.

\textbf{Mr.~Justice Frankfurter, concurring.}

For historical reasons a limited immunity from the common duty to
testify was written into the Federal Bill of Rights, and I am prepared
to agree that, as part of that immunity, comment on the failure of an
accused to take the witness stand is forbidden in federal prosecutions.
It is so, of course, by explicit act of Congress. 20 Stat. 30; see Bruno
v. United States. But to suggest that such a limitation can be drawn out
of ``due process'' in its protection of ultimate decency in a civilized
society is to suggest that the Due Process Clause fastened fetters of
unreason upon the States. (This opinion is concerned solely with a
discussion of the Due Process Clause of the Fourteenth Amendment. I put
to one side the Privileges or Immunities Clause of that Amendment. For
the mischievous uses to which that clause would lend itself if its scope
were not confined to that given it by all but one of the decisions
beginning with the Slaughter-House Cases, 16 Wall. 36, see the deviation
in Colgate v. Harvey, overruled by Madden v. Kentucky.)

Between the incorporation of the Fourteenth Amendment into the
Constitution and the beginning of the present membership of the Court
--- a period of seventy years --- the scope of that Amendment was passed
upon by forty-three judges. Of all these judges, only one, who may
respectfully be called an eccentric exception, ever indicated the belief
that the Fourteenth Amendment was a shorthand summary of the first eight
Amendments theretofore limiting only the Federal Government, and that
due process incorporated those eight Amendments as restrictions upon the
powers of the States. Among these judges were not only those who would
have to be included among the greatest in the history of the Court, but
--- it is especially relevant to note --- -they included those whose
services in the cause of human rights and the spirit of freedom are the
most conspicuous in our history. It is not invidious to single out
Miller, Davis, Bradley, Waite, Matthews, Gray, Fuller, Holmes, Brandéis,
Stone and Cardozo (to speak only of the dead) as judges who were alert
in safeguarding and promoting the interests of liberty and human dignity
through law. But they were also judges mindful of the relation of our
federal system to a progressively democratic society and therefore duly
regardful of the scope of authority that was left to the States even
after the Civil War. And so they did not find that the Fourteenth
Amendment, concerned as it was with matters fundamental to the pursuit
of justice, fastened upon the States procedural arrangements which, in
the language of Mr.~Justice Cardozo, only those who are ``narrow or
provincial'' would deem essential to ``a fair and enlightened system of
justice.'' Palko v. Connecticut. To suggest that it is inconsistent with
a truly free society to begin prosecutions without an indictment, to try
petty civil cases without the paraphernalia of a common law jury, to
take into consideration that one who has full opportunity to make a
defense remains silent is, in de Tocqueville's phrase, to confound the
familiar with the necessary.

The short answer to the suggestion that the provision of the Fourteenth
Amendment, which ordains ``nor shall any State deprive any person of
life, liberty, or property, without due process of law,'' was a way of
saying that every State must thereafter initiate prosecutions through
indictment by a grand jury, must have a trial by a jury of twelve in
criminal cases, and must have trial by such a jury in common law suits
where the amount in controversy exceeds twenty dollars, is that it is a
strange way of saying it. It would be extraordinarily strange for a
Constitution to convey such specific commands in such a roundabout and
inexplicit way. After all, an amendment to the Constitution should be
read in a `` `sense most obvious to the common understanding at the time
of its adoption.' . For it was for public adoption that it was
proposed.'' See Mr.~Justice Holmes in Eisner v. Macomber. Those reading
the English language with the meaning which it ordinarily conveys, those
conversant with the political and legal history of the concept of due
process, those sensitive to the relations of the States to the central
government as well as the relation of some of the provisions of the Bill
of Rights to the process of justice, would hardly recognize the
Fourteenth Amendment as a cover for the various explicit provisions of
the first eight Amendments. Some of these are enduring reflections of
experience with human nature, while some express the restricted views of
Eighteenth-Century England regarding the best methods for the
ascertainment of facts. The notion that the Fourteenth Amendment was a
covert way of imposing upon the States all the rules which it seemed
important to Eighteenth Century statesmen to write into the Federal
Amendments, was rejected by judges who were themselves witnesses of the
process by which the Fourteenth Amendment became part of the
Constitution. Arguments that may now be adduced to prove that the first
eight Amendments were concealed within the historic phrasing* of the
Fourteenth Amendment were not unknown at the time of its adoption. A
surer estimate of their bearing was possible for judges at the time than
distorting distance is likely to vouchsafe. Any evidence of design or
purpose not contemporaneously known could hardly have influenced those
who ratified the Amendment. Remarks of a particular proponent of the
Amendment, no matter how influential, are not to be deemed part of the
Amendment. What was submitted for ratification was his proposal, not his
speech. Thus, at the time of the ratification of the Fourteenth
Amendment the constitutions of nearly half of the ratifying States did
not have the rigorous requirements of the Fifth Amendment for
instituting criminal proceedings through a grand jury. It could hardly
have occurred to these States that by ratifying the Amendment they
uprooted their established methods for prosecuting crime and fastened
upon themselves a new prosecutorial system.

\textbf{Mr.~Justice Black, dissenting.}

The first ten amendments were proposed and adopted largely because of
fear that Government might unduly interfere with prized individual
liberties. The people wanted and demanded a Bill of Rights written into
their Constitution. The amendments embodying the Bill of Rights were
intended to curb all branches of the Federal Government in the fields
touched by the amendments--- Legislative, Executive, and Judicial. The
Fifth, Sixth, and Eighth Amendments were pointedly aimed at confining
exercise of power by courts and judges within precise boundaries,
particularly in the procedure used for the trial of criminal cases Past
history provided strong reasons for the apprehensions which brought
these procedural amendments into being and attest the wisdom of their
adoption. For the fears of arbitrary court action sprang largely from
the past use of courts in the imposition of criminal punishments to
suppress speech, press, and religion. Hence the constitutional
limitations of courts' powers were, in the view of the Founders,
essential supplements to the First Amendment, which was itself designed
to protect the widest scope for all people to believe and to express the
most divergent political, religious, and other views.

But these limitations were not expressly imposed upon state court
action. In 1833, Barron v. Baltimorewas decided by this Court. It
specifically held inapplicable to the states that provision of the Fifth
Amendment which declares: ``nor shall private property be taken for
public use, without just compensation.'' In deciding the particular
point raised, the Court there said that it could not hold that the first
eight amendments applied to the states. This was the controlling
constitutional rule when the Fourteenth Amendment was proposed in 1866.

My study of the historical events that culminated in the Fourteenth
Amendment, and the expressions of those who sponsored and favored, as
well as those who opposed its submission and passage, persuades me that
one of the chief objects that the provisions of the Amendment's first
section, separately, and as a whole, were intended to accomplish was to
make the Bill of Rights, applicable to the states With full knowledge of
the import of the Barron decision, the framers and backers of the
Fourteenth Amendment proclaimed its purpose to be to overturn the
constitutional rule that case had announced. This historical purpose has
never received full consideration or exposition in any opinion of this
Court interpreting the Amendment.

For this reason, I am attaching to this dissent an appendix which
contains a résumé, by no means complete, of the Amendment's history. In
my judgment that history conclusively demonstrates that the language of
the first section of the Fourteenth Amendment, taken as a whole, was
thought by those responsible for its submission to the people, and by
those who opposed its submission, sufficiently explicit to guarantee
that thereafter no state could deprive its citizens of the privileges
and protections of the Bill of Rights. Whether this Court ever will, or
whether it now should, in the light of past decisions, give full effect
to what the Amendment was intended to accomplish is not necessarily
essential to a decision here. However that may be, our prior decisions,
including Twining, do not prevent our carrying out that purpose, at
least to the extent of making applicable to the states, not a mere part,
as the Court has, .but the full protection of the Fifth Amendment's
provision against compelling evidence from an accused to convict him of
crime. And I further contend that the ``natural law'' formula which the
Court uses to reach its conclusion in this case should be abandoned as
an incongruous excrescence on our Constitution. I believe that formula
to be itself a violation of our Constitution, in that it subtly conveys
to courts, at the expense of legislatures, ultimate power over public
policies in fields where no specific provision of the Constitution
limits legislative power. And my belief seems to be in accord with the
views expressed by this Court, at least for the first two decades after
the Fourteenth Amendment was adopted.

\hypertarget{mcdonald-v.-city-of-chicago}{%
\subsubsection{McDonald v. City of
Chicago}\label{mcdonald-v.-city-of-chicago}}

561 U.S. 742 (2010)

\textbf{Justice ALITO announced the judgment of the Court and delivered
the opinion of the Court with respect to Parts I, II-A, II-B, II-D,
III-A, and III-B, in which THE CHIEF JUSTICE, Justice SCALIA, Justice
KENNEDY, and Justice THOMAS join, and an opinion with respect to Parts
II-C, IV, and V, in which THE CHIEF JUSTICE, Justice SCALIA, and Justice
KENNEDY join.}

Two years ago, in District of Columbia v. Heller, 554 U.S. \_\_\_, we
held that the Second Amendment protects the right to keep and bear arms
for the purpose of self-defense, and we struck down a District of
Columbia law that banned the possession of handguns in the home. The
city of Chicago (City) and the village of Oak Park, a Chicago suburb,
have laws that are similar to the District of Columbia's, but Chicago
and Oak Park argue that their laws are constitutional because the Second
Amendment has no application to the States. We have previously held that
most of the provisions of the Bill of Rights apply with full force to
both the Federal Government and the States. Applying the standard that
is well established in our case law, we hold that the Second Amendment
right is fully applicable to the States.

Petitioners argue that the Chicago and Oak Park laws violate the right
to keep and bear arms for two reasons. Petitioners' primary submission
is that this right is among the ``privileges or immunities of citizens
of the United States'' and that the narrow interpretation of the
Privileges or Immunities Clause adopted in the Slaughter-House
Casesshould now be rejected. As a secondary argument, petitioners
contend that the Fourteenth Amendment's Due Process Clause
``incorporates'' the Second Amendment right.

Chicago and Oak Park (municipal respondents) maintain that a right set
out in the Bill of Rights applies to the States only if that right is an
indispensable attribute of any ```civilized''' legal system. Brief for
Municipal Respondents 9. If it is possible to imagine a civilized
country that does not recognize the right, the municipal respondents
tell us, then that right is not protected by due process. And since
there are civilized countries that ban or strictly regulate the private
possession of handguns, the municipal respondents maintain that due
process does not preclude such measures. -23. In light of the parties'
far-reaching arguments, we begin by recounting this Court's analysis
over the years of the relationship between the provisions of the Bill of
Rights and the States.

The Bill of Rights, including the Second Amendment, originally applied
only to the Federal Government. In Barron ex rel. Tiernan v. Mayor of
Baltimore, 7 Pet. 243, the Court, in an opinion by Chief Justice
Marshall, explained that this question was ``of great importance'' but
``not of much difficulty.'' In less than four pages, the Court firmly
rejected the proposition that the first eight Amendments operate as
limitations on the States, holding that they apply only to the Federal
Government. See also Lessee of Livingston v. Moore, 7 Pet. 469, 551-552,
(``{[}I{]}t is now settled that those amendments {[}in the Bill of
Rights{]} do not extend to the states'').

The constitutional Amendments adopted in the aftermath of the Civil War
fundamentally altered our country's federal system. The provision at
issue in this case, § 1 of the Fourteenth Amendment, provides, among
other things, that a State may not abridge ``the privileges or
immunities of citizens of the United States'' or deprive ``any person of
life, liberty, or property, without due process of law.''

Four years after the adoption of the Fourteenth Amendment, this Court
was asked to interpret the Amendment's reference to ``the privileges or
immunities of citizens of the United States.'' The Slaughter-House
Casesinvolved challenges to a Louisiana law permitting the creation of a
state-sanctioned monopoly on the butchering of animals within the city
of New Orleans. Justice Samuel Miller's opinion for the Court concluded
that the Privileges or Immunities Clause protects only those rights
``which owe their existence to the Federal government, its National
character, its Constitution, or its laws.'' The Court held that other
fundamental rights---rights that predated the creation of the Federal
Government and that ``the State governments were created to establish
and secure''--- were not protected by the Clause.

In drawing a sharp distinction between the rights of federal and state
citizenship, the Court relied on two principal arguments. First, the
Court emphasized that the Fourteenth Amendment's Privileges or
Immunities Clause spoke of ``the privileges or immunities of citizens of
the United States,'' and the Court contrasted this phrasing with the
wording in the first sentence of the Fourteenth Amendment and in the
Privileges and Immunities Clause of Article IV, both of which refer to
state citizenship (Emphasis added.) Second, the Court stated that a
contrary reading would ``radically chang{[}e{]} the whole theory of the
relations of the State and Federal governments to each other and of both
these governments to the people,'' and the Court refused to conclude
that such a change had been made ``in the absence of language which
expresses such a purpose too clearly to admit of doubt.'' Finding the
phrase ``privileges or immunities of citizens of the United States''
lacking by this high standard, the Court reasoned that the phrase must
mean something more limited.

Under the Court's narrow reading, the Privileges or Immunities Clause
protects such things as the right

``to come to the seat of government to assert any claim {[}a citizen{]}
may have upon that government, to transact any business he may have with
it, to seek its protection, to share its offices, to engage in
administering its functions \ldots{} {[}and to{]} become a citizen of
any State of the Union by a bonafide residence therein, with the same
rights as other citizens of that State.'' -80 (internal quotation marks
omitted).

Finding no constitutional protection against state intrusion of the kind
envisioned by the Louisiana statute, the Court upheld the statute. Four
Justices dissented. Justice Field, joined by Chief Justice Chase and
Justices Swayne and Bradley, criticized the majority for reducing the
Fourteenth Amendment's Privileges or Immunities Clause to ``a vain and
idle enactment, which accomplished nothing, and most unnecessarily
excited Congress and the people on its passage.'' ; see also Justice
Field opined that the Privileges or Immunities Clause protects rights
that are ``in their nature \ldots{} fundamental,'' including the right
of every man to pursue his profession without the imposition of unequal
or discriminatory restrictions. -97. Justice Bradley's dissent observed
that ``we are not bound to resort to implication \ldots{} to find an
authoritative declaration of some of the most important privileges and
immunities of citizens of the United States. It is in the Constitution
itself.'' Justice Bradley would have construed the Privileges or
Immunities Clause to include those rights enumerated in the Constitution
as well as some unenumerated rights. Justice Swayne described the
majority's narrow reading of the Privileges or Immunities Clause as
``turn{[}ing{]} \ldots{} what was meant for bread into a stone.''
(dissenting opinion).

Today, many legal scholars dispute the correctness of the narrow
Slaughter-House interpretation. See, e.g., Saenz v. Roe, n.~1, 527,
(THOMAS, J., dissenting) (scholars of the Fourteenth Amendment agree
``that the Clause does not mean what the Court said it meant in 1873'');
Amar, Substance and Method in the Year 2000, 28 Pepperdine L.Rev. 601,
631, n.~178 (2001) (``Virtually no serious modern scholar---left, right,
and center--- thinks that this {[}interpretation{]} is a plausible
reading of the Amendment''); Brief for Constitutional Law Professors as
Amici Curiae 33 (claiming an ``overwhelming consensus among leading
constitutional scholars'' that the opinion is ``egregiously wrong''); C.
Black, A New Birth of Freedom 74-75 (1997).

Three years after the decision in the Slaughter-House Cases, the Court
decided Cruikshank, the first of the three 19th-century cases on which
the Seventh Circuit relied. 92 U.S. 542, In that case, the Court
reviewed convictions stemming from the infamous Colfax Massacre in
Louisiana on Easter Sunday 1873. Dozens of blacks, many unarmed, were
slaughtered by a rival band of armed white men Cruikshank himself
allegedly marched unarmed African-American prisoners through the streets
and then had them summarily executed Ninety-seven men were indicted for
participating in the massacre, but only nine went to trial. Six of the
nine were acquitted of all charges; the remaining three were acquitted
of murder but convicted under the Enforcement Act of 1870, 16 Stat. 140,
for banding and conspiring together to deprive their victims of various
constitutional rights, including the right to bear arms.

The Court reversed all of the convictions, including those relating to
the deprivation of the victims' right to bear arms. Cruikshank, 559. The
Court wrote that the right of bearing arms for a lawful purpose ``is not
a right granted by the Constitution'' and is not ``in any manner
dependent upon that instrument for its existence.'' ``The second
amendment,'' the Court continued, ``declares that it shall not be
infringed; but this \ldots{} means no more than that it shall not be
infringed by Congress.'' ``Our later decisions in Presser v.
Illinois{[}, {]} (1886), and Miller v. Texas{[}, {]} (1894), reaffirmed
that the Second Amendment applies only to the Federal Government.''
Heller, n.~23, 128 S.Ct. n.~23.

As previously noted, the Seventh Circuit concluded that Cruikshank,
Presser, and Miller doomed petitioners' claims at the Court of Appeals
level. Petitioners argue, however, that we should overrule those
decisions and hold that the right to keep and bear arms is one of the
``privileges or immunities of citizens of the United States.'' In
petitioners' view, the Privileges or Immunities Clause protects all of
the rights set out in the Bill of Rights, as well as some others, see
Brief for Petitioners 10, 14, 15-21, but petitioners are unable to
identify the Clause's full scope. Nor is there any consensus on that
question among the scholars who agree that the Slaughter-House Cases'
interpretation is flawed. See Saenz, n.~1, (THOMAS, J., dissenting).

We see no need to reconsider that interpretation here. For many decades,
the question of the rights protected by the Fourteenth Amendment against
state infringement has been analyzed under the Due Process Clause of
that Amendment and not under the Privileges or Immunities Clause. We
therefore decline to disturb the Slaughter-House holding.

At the same time, however, this Court's decisions in Cruikshank,
Presser, and Miller do not preclude us from considering whether the Due
Process Clause of the Fourteenth Amendment makes the Second Amendment
right binding on the States. See Heller, n.~23, 128 S.Ct. n.~23. None of
those cases ``engage{[}d{]} in the sort of Fourteenth Amendment inquiry
required by our later cases.'' As explained more fully below,
Cruikshank, Presser, and Miller all preceded the era in which the Court
began the process of ``selective incorporation'' under the Due Process
Clause, and we have never previously addressed the question whether the
right to keep and bear arms applies to the States under that theory.

Indeed, Cruikshank has not prevented us from holding that other rights
that were at issue in that case are binding on the States through the
Due Process Clause. In Cruikshank, the Court held that the general
``right of the people peaceably to assemble for lawful purposes,'' which
is protected by the First Amendment, applied only against the Federal
Government and not against the States. See 92 U.S.. Nonetheless, over 60
years later the Court held that the right of peaceful assembly was a
``fundamental righ{[}t{]} \ldots{} safeguarded by the due process clause
of the Fourteenth Amendment.'' De Jonge v. Oregon, We follow the same
path here and thus consider whether the right to keep and bear arms
applies to the States under the Due Process Clause.

In the late 19th century, the Court began to consider whether the Due
Process Clause prohibits the States from infringing rights set out in
the Bill of Rights. See Hurtado v. California (due process does not
require grand jury indictment); Chicago, B. \& Q.R. Co.~v. Chicago (due
process prohibits States from taking of private property for public use
without just compensation). Five features of the approach taken during
the ensuing era should be noted.

First, the Court viewed the due process question as entirely separate
from the question whether a right was a privilege or immunity of
national citizenship. See Twining v. New Jersey, Second, the Court
explained that the only rights protected against state infringement by
the Due Process Clause were those rights ``of such a nature that they
are included in the conception of due process of law.'' See also, e.g.,
Adamson v. California; Betts v. Brady; Palko v. Connecticut; Grosjean v.
American Press Co.; Powell v. AlabamaWhile it was ``possible that some
of the personal rights safeguarded by the first eight Amendments against
National action {[}might{]} also be safeguarded against state action,''
the Court stated, this was ``not because those rights are enumerated in
the first eight Amendments.'' Twining.

The Court used different formulations in describing the boundaries of
due process. For example, in Twining, the Court referred to ``immutable
principles of justice which inhere in the very idea of free government
which no member of the Union may disregard.'' 211 U.S., (internal
quotation marks omitted). In Snyder v. Massachusetts, the Court spoke of
rights that are ``so rooted in the traditions and conscience of our
people as to be ranked as fundamental.'' And in Palko, the Court
famously said that due process protects those rights that are ``the very
essence of a scheme of ordered liberty'' and essential to ``a fair and
enlightened system of justice.'' 302 U.S., Third, in some cases decided
during this era the Court ``can be seen as having asked, when inquiring
into whether some particular procedural safeguard was required of a
State, if a civilized system could be imagined that would not accord the
particular protection.'' Duncan v. Louisiana, n.~14, Thus, in holding
that due process prohibits a State from taking private property without
just compensation, the Court described the right as ``a principle of
natural equity, recognized by all temperate and civilized governments,
from a deep and universal sense of its justice.'' Chicago, B. \& Q.R.
Co., Similarly, the Court found that due process did not provide a right
against compelled incrimination in part because this right ``has no
place in the jurisprudence of civilized and free countries outside the
domain of the common law.''

Fourth, the Court during this era was not hesitant to hold that a right
set out in the Bill of Rights failed to meet the test for inclusion
within the protection of the Due Process Clause. The Court found that
some such rights qualified. See, e.g., Gitlow v. New York, (freedom of
speech and press); Near v. Minnesota ex rel. Olson (same);
Powell(assistance of counsel in capital cases); De Jonge(freedom of
assembly); Cantwell v. Connecticut (free exercise of religion). But
others did not. See, e.g., Hurtado(grand jury indictment requirement);
Twining(privilege against self-incrimination).

Finally, even when a right set out in the Bill of Rights was held to
fall within the conception of due process, the protection or remedies
afforded against state infringement sometimes differed from the
protection or remedies provided against abridgment by the Federal
Government. To give one example, in Betts the Court held that, although
the Sixth Amendment required the appointment of counsel in all federal
criminal cases in which the defendant was unable to retain an attorney,
the Due Process Clause required appointment of counsel in state criminal
proceedings only where ``want of counsel in {[}the{]} particular case
\ldots{} result{[}ed{]} in a conviction lacking in \ldots{} fundamental
fairness.'' 316 U.S., Similarly, in Wolf v. Coloradothe Court held that
the ``core of the Fourth Amendment'' was implicit in the concept of
ordered liberty and thus ``enforceable against the States through the
Due Process Clause'' but that the exclusionary rule, which applied in
federal cases, did not apply to the States.

An alternative theory regarding the relationship between the Bill of
Rights and § 1 of the Fourteenth Amendment was championed by Justice
Black. This theory held that § 1 of the Fourteenth Amendment totally
incorporated all of the provisions of the Bill of Rights. See, e.g.,
Adamson, (Black, J., dissenting); Duncan, (Black, J., concurring). As
Justice Black noted, the chief congressional proponents of the
Fourteenth Amendment espoused the view that the Amendment made the Bill
of Rights applicable to the States and, in so doing, overruled this
Court's decision in Barron Adamson, (dissenting opinion) Nonetheless,
the Court never has embraced Justice Black's ``total incorporation''
theory.

While Justice Black's theory was never adopted, the Court eventually
moved in that direction by initiating what has been called a process of
``selective incorporation,'' i.e., the Court began to hold that the Due
Process Clause fully incorporates particular rights contained in the
first eight Amendments. See, e.g., Gideon v. Wainwright, ; Malloy v.
Hogan, ; Pointer v. Texas, ; Washington v. Texas, ; Duncan, ; Benton v.
Maryland.

The decisions during this time abandoned three of the previously noted
characteristics of the earlier period The Court made it clear that the
governing standard is not whether any ``civilized system {[}can{]} be
imagined that would not accord the particular protection.'' Duncan,
n.~14, Instead, the Court inquired whether a particular Bill of Rights
guarantee is fundamental to our scheme of ordered liberty and system of
justice. and n.~14, ; see also (referring to those ``fundamental
principles of liberty and justice which lie at the base of all our civil
and political institutions'' (emphasis added; internal quotation marks
omitted)).

The Court also shed any reluctance to hold that rights guaranteed by the
Bill of Rights met the requirements for protection under the Due Process
Clause. The Court eventually incorporated almost all of the provisions
of the Bill of Rights Only a handful of the Bill of Rights protections
remain unincorporated.

Finally, the Court abandoned ``the notion that the Fourteenth Amendment
applies to the States only a watered-down, subjective version of the
individual guarantees of the Bill of Rights,'' stating that it would be
``incongruous'' to apply different standards ``depending on whether the
claim was asserted in a state or federal court.'' Malloy, (internal
quotation marks omitted). Instead, the Court decisively held that
incorporated Bill of Rights protections ``are all to be enforced against
the States under the Fourteenth Amendment according to the same
standards that protect those personal rights against federal
encroachment.'' ; see also Mapp v. Ohio, ; Ker v. California, ; Aguilar
v. Texas, ; Pointer, ; Duncan, 157S.Ct. 1444; Benton, ; Wallace v.
Jaffree,

Employing this approach, the Court overruled earlier decisions in which
it had held that particular Bill of Rights guarantees or remedies did
not apply to the States. S

With this framework in mind, we now turn directly to the question
whether the Second Amendment right to keep and bear arms is incorporated
in the concept of due process. In answering that question, as just
explained, we must decide whether the right to keep and bear arms is
fundamental to our scheme of ordered liberty, Duncan, or as we have said
in a related context, whether this right is ``deeply rooted in this
Nation's history and tradition,'' Washington v. Glucksberg, (internal
quotation marks omitted).

Our decision in Heller points unmistakably to the answer. Self-defense
is a basic right, recognized by many legal systems from ancient times to
the present day,15 and in Heller, we held that individual self-defense
is ``the central component'' of the Second Amendment right. 128 S.Ct.;
see also \textbf{\emph{, 128 S.Ct. (stating that the ``inherent right of
self-defense has been central to the Second Amendment right'').
Explaining that ``the need for defense of self, family, and property is
most acute'' in the home, , we found that this right applies to handguns
because they are ``the most preferred firearm in the nation to'keep' and
use for protection of one's home and family,'' }}, 128 S.Ct. (some
internal quotation marks omitted); see also \textbf{\emph{, 128 S.Ct.
(noting that handguns are ``overwhelmingly chosen by American society
for {[}the{]} lawful purpose'' of self-defense); }}, 128 S.Ct.
(``{[}T{]}he American people have considered the handgun to be the
quintessential self-defense weapon''). Thus, we concluded, citizens must
be permitted ``to use {[}handguns{]} for the core lawful purpose of
self-defense.''

Heller makes it clear that this right is ``deeply rooted in this
Nation's history and tradition.'' Glucksberg, (internal quotation marks
omitted). Heller explored the right's origins, noting that the 1689
English Bill of Rights explicitly protected a right to keep arms for
self-defense, - \_\_\_, and that by 1765, Blackstone was able to assert
that the right to keep and bear arms was ``one of the fundamental rights
of Englishmen.''

Blackstone's assessment was shared by the American colonists. As we
noted in Heller, King George III's attempt to disarm the colonists in
the 1760's and 1770's ``provoked polemical reactions by Americans
invoking their rights as Englishmen to keep arms.''16 \_\_\_, 128 S.Ct.;
see also L. Levy, Origins of the Bill of Rights 137-143 (1999)
(hereinafter Levy).

The right to keep and bear arms was considered no less fundamental by
those who drafted and ratified the Bill of Rights. ``During the 1788
ratification debates, the fear that the federal government would disarm
the people in order to impose rule through a standing army or select
militia was pervasive in Antifederalist rhetoric.'' Heller\_\_\_, 128
S.Ct. (citing Letters from the Federal Farmer

(Oct.~10, 1787), in 2 The Complete Anti-Federalist 234, 242 (H. Storing
ed )); see also Federal Farmer: An Additional Number of Letters to the
Republican, Letter XVIII (Jan.~25, 1788), in 17 Documentary History of
the Ratification of the Constitution 360, 362-363 (J. Kaminski \& G.
Saladino eds ); S. Halbrook, The Founders' Second Amendment 171-278
(2008). Federalists responded, not by arguing that the right was
insufficiently important to warrant protection but by contending that
the right was adequately protected by the Constitution's assignment of
only limited powers to the Federal Government. Heller\_\_\_, 128 S.Ct.;
cf.~The Federalist No.~46, p.~296 (C. Rossiter ed.~1961) (J. Madison).
Thus, Antifederalists and Federalists alike agreed that the right to
bear arms was fundamental to the newly formed system of government. See
Levy 143-149; J. Malcolm, To Keep and Bear Arms: The Origins of an
Anglo-American Right 155-164 (1994). But those who were fearful that the
new Federal Government would infringe traditional rights such as the
right to keep and bear arms insisted on the adoption of the Bill of
Rights as a condition for ratification of the Constitution. See 1 J.
Elliot, The Debates in the Several State Conventions on the Adoption of
the Federal Constitution 327-331 (2d ed.~1854); 3 -661; 4 ; see also
Levy 26-34; A. Kelly \& W. Harbison, The American Constitution: Its
Origins and Development 110, 118 (7th ed ). This is surely powerful
evidence that the right was regarded as fundamental in the sense
relevant here.

This understanding persisted in the years immediately following the
ratification of the Bill of Rights. In addition to the four States that
had adopted Second Amendment analogues before ratification, nine more
States adopted state constitutional provisions protecting an individual
right to keep and bear arms between 1789 and 1820. Heller\_\_\_,
Founding-era legal commentators confirmed the importance of the right to
early Americans. St.~George Tucker, for example, described the right to
keep and bear arms as ``the true palladium of liberty'' and explained
that prohibitions on the right would place liberty ``on the brink of
destruction.'' 1 Blackstone's Commentaries, Editor's App. 300 (S. Tucker
ed.~1803); see also W. Rawle, A View of the Constitution of the United
States of America, 125-126 (2d ed.~1829) (reprint 2009); 3 J. Story,
Commentaries on the Constitution of the United States § 1890, p.~746
(1833) (``The right of the citizens to keep and bear arms has justly
been considered, as the palladium of the liberties of a republic; since
it offers a strong moral check against the usurpation and arbitrary
power of rulers; and will generally, even if these are successful in the
first instance, enable the people to resist and triumph over them'').

By the 1850's, the perceived threat that had prompted the inclusion of
the Second Amendment in the Bill of Rights---the fear that the National
Government would disarm the universal militia---had largely faded as a
popular concern, but the right to keep and bear arms was highly valued
for purposes of self-defense. See M. Doubler, Civilian in Peace, Soldier
in War 87-90 (2003); Amar, Bill of Rights 258-259. Abolitionist authors
wrote in support of the right. See L. Spooner, The Unconstitutionality
of Slavery 66 (1860) (reprint 1965); J. Tiffany, A Treatise on the
Unconstitutionality of American Slavery 117-118 (1849) (reprint 1969).
And when attempts were made to disarm ``Free-Soilers'' in ``Bloody
Kansas,'' Senator Charles Sumner, who later played a leading role in the
adoption of the Fourteenth Amendment, proclaimed that ``{[}n{]}ever was
{[}the rifle{]} more needed in just self-defense than now in Kansas.''
The Crime Against Kansas: The Apologies for the Crime: The True Remedy,
Speech of Hon.~Charles Sumner in the Senate of the United States 64-65
(1856). Indeed, the 1856 Republican Party Platform protested that in
Kansas the constitutional rights of the people had been ``fraudulently
and violently taken from them'' and the ``right of the people to keep
and bear arms'' had been ``infringed.'' National Party Platforms
1840-1972, p.~27 (5th ed ).

After the Civil War, many of the over 180,000 African Americans who
served in the Union Army returned to the States of the old Confederacy,
where systematic efforts were made to disarm them and other blacks. See
Heller, 128 S.Ct.; E. Foner, Reconstruction: America's Unfinished
Revolution 1863-1877, p.~8 (1988) (hereinafter Foner). The laws of some
States formally prohibited African Americans from possessing firearms.
For example, a Mississippi law provided that ``no freedman, free negro
or mulatto, not in the military service of the United States government,
and not licensed so to do by the board of police of his or her county,
shall keep or carry fire-arms of any kind, or any ammunition, dirk or
bowie knife.'' Certain Offenses of Freedmen, 1865 Miss. Laws p.~165, §
1, in 1 Documentary History of Reconstruction 289 (W. Fleming ed ); see
also Regulations for Freedmen in Louisiana, in -280; H.R. Exec. Doc.
No.~70, 39th Cong., 1st Sess., 233, 236 (1866) (describing a Kentucky
law); E. McPherson, The Political History of the United States of
America During the Period of Reconstruction 40 (1871) (describing a
Florida law); (describing an Alabama law).

Throughout the South, armed parties, often consisting of ex-Confederate
soldiers serving in the state militias, forcibly took firearms from
newly freed slaves. In the first session of the 39th Congress, Senator
Wilson told his colleagues: ``In Mississippi rebel State forces, men who
were in the rebel armies, are traversing the State, visiting the
freedmen, disarming them, perpetrating murders and outrages upon them;
and the same things are done in other sections of the country.'' 39th
Cong. Globe 40 (1865). The Report of the Joint Committee on
Reconstruction --- which was widely reprinted in the press and
distributed by Members of the 39th Congress to their constituents
shortly after Congress approved the Fourteenth Amendment19--- contained
numerous examples of such abuses. See, e.g., Joint Committee on
Reconstruction, H.R.Rep. No.~30, 39th Cong., 1st Sess., pt.~2, pp.~219,
229, 272, pt.~3, pp.~46, 140, pt.~4, pp.~49-50 (1866); see also S. Exec.
Doc. No.~2, 39th Cong., 1st Sess., 23, 36 (1865). In one town, the
``marshal {[}took{]} all arms from returned colored soldiers, and
{[}was{]} very prompt in shooting the blacks whenever an opportunity
occur{[}red{]}.'' H.R. Exec. Doc. No.~70 (internal quotation marks
omitted). As Senator Wilson put it during the debate on a failed
proposal to disband Southern militias: ``There is one unbroken chain of
testimony from all people that are loyal to this country, that the
greatest outrages are perpetrated by armed men who go up and down the
country searching houses, disarming people, committing outrages of every
kind and description.'' 39th Cong. Globe 915 (1866).

Union Army commanders took steps to secure the right of all citizens to
keep and bear arms,21 but the 39th Congress concluded that legislative
action was necessary. Its efforts to safeguard the right to keep and
bear arms demonstrate that the right was still recognized to be
fundamental.

The most explicit evidence of Congress' aim appears in § 14 of the
Freedmen's Bureau Act of 1866, which provided that ``the right \ldots{}
to have full and equal benefit of all laws and proceedings concerning
personal liberty, personal security, and the acquisition, enjoyment, and
disposition of estate, real and personal, including the constitutional
right to bear arms, shall be secured to and enjoyed by all the
citizens\ldots{} without respect to race or color, or previous condition
of slavery.'' 14 Stat. 176-177 (emphasis added) Section 14 thus
explicitly guaranteed that ``all the citizens,'' black and white, would
have ``the constitutional right to bear arms.''

The Civil Rights Act of 1866, 14 Stat. 27, which was considered at the
same time as the Freedmen's Bureau Act, similarly sought to protect the
right of all citizens to keep and bear arms Section 1 of the Civil
Rights Act guaranteed the ``full and equal benefit of all laws and
proceedings for the security of person and property, as is enjoyed by
white citizens.'' This language was virtually identical to language in §
14 of the Freedmen's Bureau Act, 14 Stat. 176-177 (``the right \ldots{}
to have full and equal benefit of all laws and proceedings concerning
personal liberty, personal security, and the acquisition, enjoyment, and
disposition of estate, real and personal''). And as noted, the latter
provision went on to explain that one of the ``laws and proceedings
concerning personal liberty, personal security, and the acquisition,
enjoyment, and disposition of estate, real and personal'' was ``the
constitutional right to bear arms.'' Representative Bingham believed
that the Civil Rights Act protected the same rights as enumerated in the
Freedmen's Bureau bill, which of course explicitly mentioned the right
to keep and bear arms. 39th Cong. Globe 1292. The unavoidable conclusion
is that the Civil Rights Act, like the Freedmen's Bureau Act, aimed to
protect ``the constitutional right to bear arms'' and not simply to
prohibit discrimination. See also Amar, Bill of Rights 264-265 (noting
that one of the ``core purposes of the Civil Rights Act of 1866 and of
the Fourteenth Amendment was to redress the grievances'' of freedmen who
had been stripped of their arms and to ``affirm the full and equal right
of every citizen to self-defense'').

Congress, however, ultimately deemed these legislative remedies
insufficient. Southern resistance, Presidential vetoes, and this Court's
pre-Civil-War precedent persuaded Congress that a constitutional
amendment was necessary to provide full protection for the rights of
blacks Today, it is generally accepted that the Fourteenth Amendment was
understood to provide a constitutional basis for protecting the rights
set out in the Civil Rights Act of 1866. See General Building
Contractors Assn., Inc.~v. Pennsylvania, ; see also Amar, Bill of Rights
187; Calabresi, Two Cheers for Professor Balkin's Originalism, 103 Nw.
U.L.Rev. 663, 669-670 (2009).

In debating the Fourteenth Amendment, the 39th Congress referred to the
right to keep and bear arms as a fundamental right deserving of
protection. Senator Samuel Pomeroy described three ``indispensable''
``safeguards of liberty under our form of Government.'' 39th Cong. Globe
1182. One of these, he said, was the right to keep and bear arms:

``Every man . should have the right to bear arms for the defense of
himself and family and his homestead. And if the cabin door of the
freedman is broken open and the intruder enters for purposes as vile as
were known to slavery, then should a well-loaded musket be in the hand
of the occupant to send the polluted wretch to another world, where his
wretchedness will forever remain complete.''

Even those who thought the Fourteenth Amendment unnecessary believed
that blacks, as citizens, ``have equal right to protection, and to keep
and bear arms for self-defense.'' (Sen.~James Nye); see also Foner
258-259.

Evidence from the period immediately following the ratification of the
Fourteenth Amendment only confirms that the right to keep and bear arms
was considered fundamental. In an 1868 speech addressing the disarmament
of freedmen, Representative Stevens emphasized the necessity of the
right: ``Disarm a community and you rob them of the means of defending
life. Take away their weapons of defense and you take away the
inalienable right of defending liberty.'' ``The fourteenth amendment,
now so happily adopted, settles the whole question.'' Cong. Globe, 40th
Cong., 2d Sess., 1967. And in debating the Civil Rights Act of 1871,
Congress routinely referred to the right to keep and bear arms and
decried the continued disarmament of blacks in the South. See Halbrook,
Freedmen 120-131. Finally, legal commentators from the period emphasized
the fundamental nature of the right. See, e.g., T. Farrar, Manual of the
Constitution of the United States of America § 118, p.~145 (1867)
(reprint 1993); J. Pomeroy, An Introduction to the Constitutional Law of
the United States § 239, pp.~152-153 (3d ed.~1875).

The right to keep and bear arms was also widely protected by state
constitutions at the time when the Fourteenth Amendment was ratified. In
1868, 22 of the 37 States in the Union had state constitutional
provisions explicitly protecting the right to keep and bear arms. See
Calabresi \& Agudo, Individual Rights Under State Constitutions when the
Fourteenth Amendment was Ratified in 1868: What Rights Are Deeply Rooted
in American History and Tradition? 87 Texas L.Rev. 7, 50 (2008) Quite a
few of these state constitutional guarantees, moreover, explicitly
protected the right to keep and bear arms as an individual right to
self-defense. What is more, state constitutions adopted during the
Reconstruction era by former Confederate States included a right to keep
and bear arms. A clear majority of the States in 1868, therefore,
recognized the right to keep and bear arms as being among the
foundational rights necessary to our system of Government.

In sum, it is clear that the Framers and ratifiers of the Fourteenth
Amendment counted the right to keep and bear arms among those
fundamental rights necessary to our system of ordered liberty.

Despite all this evidence, municipal respondents contend that Congress,
in the years immediately following the Civil War, merely sought to
outlaw ``discriminatory measures taken against freedmen, which it
addressed by adopting a non-discrimination principle'' and that even an
outright ban on the possession of firearms was regarded as acceptable,
``so long as it was not done in a discriminatory manner.'' Brief for
Municipal Respondents 7. They argue that Members of Congress
overwhelmingly viewed § 1 of the Fourteenth Amendment ``as an
antidiscrimination rule,'' and they cite statements to the effect that
the section would outlaw discriminatory measures. This argument is
implausible.

First, while § 1 of the Fourteenth Amendment contains ``an
antidiscrimination rule,'' namely, the Equal Protection Clause,
municipal respondents can hardly mean that § 1 does no more than
prohibit discrimination. If that were so, then the First Amendment, as
applied to the States, would not prohibit nondiscriminatory abridgments
of the rights to freedom of speech or freedom of religion; the Fourth
Amendment, as applied to the States, would not prohibit all unreasonable
searches and seizures but only discriminatory searches and
seizures---and so on. We assume that this is not municipal respondents'
view, so what they must mean is that the Second Amendment should be
singled out for special---and specially unfavorable---treatment. We
reject that suggestion.

Second, municipal respondents' argument ignores the clear terms of the
Freedmen's Bureau Act of 1866, which acknowledged the existence of the
right to bear arms. If that law had used language such as ``the equal
benefit of laws concerning the bearing of arms,'' it would be possible
to interpret it as simply a prohibition of racial discrimination. But §
14 speaks of and protects ``the constitutional right to bear arms,'' an
unmistakable reference to the right protected by the Second Amendment.
And it protects the ``full and equal benefit'' of this right in the
States. 14 Stat. 176-177. It would have been nonsensical for Congress to
guarantee the full and equal benefit of a constitutional right that does
not exist.

Third, if the 39th Congress had outlawed only those laws that
discriminate on the basis of race or previous condition of servitude,
African Americans in the South would likely have remained vulnerable to
attack by many of their worst abusers: the state militia and state peace
officers. In the years immediately following the Civil War, a law
banning the possession of guns by all private citizens would have been
nondiscriminatory only in the formal sense. Any such law---like the
Chicago and Oak Park ordinances challenged here---presumably would have
permitted the possession of guns by those acting under the authority of
the State and would thus have left firearms in the hands of the militia
and local peace officers. And as the Report of the Joint Committee on
Reconstruction revealed, see, those groups were widely involved in
harassing blacks in the South.

Fourth, municipal respondents' purely antidiscrimination theory of the
Fourteenth Amendment disregards the plight of whites in the South who
opposed the Black Codes. If the 39th Congress and the ratifying public
had simply prohibited racial discrimination with respect to the bearing
of arms, opponents of the Black Codes would have been left without the
means of self-defense---as had abolitionists in Kansas in the 1850's.

Fifth, the 39th Congress' response to proposals to disband and disarm
the Southern militias is instructive. Despite recognizing and deploring
the abuses of these militias, the 39th Congress balked at a proposal to
disarm them. See 39th Cong. Globe 914; Halbrook, Freedmen. Disarmament,
it was argued, would violate the members' right to bear arms, and it was
ultimately decided to disband the militias but not to disarm their
members. See Act of Mar.~2, 1867, § 6, 14 Stat. 485, 487; Halbrook,
Freedmen 68-69; Cramer 858-861. It cannot be doubted that the right to
bear arms was regarded as a substantive guarantee, not a prohibition
that could be ignored so long as the States legislated in an evenhanded
manner.

Municipal respondents' remaining arguments are at war with our central
holding in Heller: that the Second Amendment protects a personal right
to keep and bear arms for lawful purposes, most notably for self-defense
within the home. Municipal respondents, in effect, ask us to treat the
right recognized in Heller as a second-class right, subject to an
entirely different body of rules than the other Bill of Rights
guarantees that we have held to be incorporated into the Due Process
Clause.

Municipal respondents' main argument is nothing less than a plea to
disregard 50 years of incorporation precedent and return (presumably for
this case only) to a bygone era. Municipal respondents submit that the
Due Process Clause protects only those rights ```recognized by all
temperate and civilized governments, from a deep and universal sense of
{[}their{]} justice.''' Brief for Municipal Respondents 9 (quoting
Chicago, B. \& Q.R. Co., ). According to municipal respondents, if it is
possible to imagine any civilized legal system that does not recognize a
particular right, then the Due Process Clause does not make that right
binding on the States. Brief for Municipal Respondents 9. Therefore, the
municipal respondents continue, because such countries as England,
Canada, Australia, Japan, Denmark, Finland, Luxembourg, and New Zealand
either ban or severely limit handgun ownership, it must follow that no
right to possess such weapons is protected by the Fourteenth Amendment.
-23.

This line of argument is, of course, inconsistent with the
long-established standard we apply in incorporation cases. See Duncan,
and n.~14, And the present-day implications of municipal respondents'
argument are stunning. For example, many of the rights that our Bill of
Rights provides for persons accused of criminal offenses are virtually
unique to this country If our understanding of the right to a jury
trial, the right against self-incrimination, and the right to counsel
were necessary attributes of any civilized country, it would follow that
the United States is the only civilized Nation in the world.

Municipal respondents attempt to salvage their position by suggesting
that their argument applies only to substantive as opposed to procedural
rights. Brief for Municipal Respondents 10, n.~3. But even in this
trimmed form, municipal respondents' argument flies in the face of more
than a half-century of precedent. For example, in Everson v. Board of
Ed. of Ewing, the Court held that the Fourteenth Amendment incorporates
the Establishment Clause of the First Amendment. Yet several of the
countries that municipal respondents recognize as civilized have
established state churches If we were to adopt municipal respondents'
theory, all of this Court's Establishment Clause precedents involving
actions taken by state and local governments would go by the boards.

We turn, finally, to the two dissenting opinions. Justice STEVENS'
eloquent opinion covers ground already addressed, and therefore little
need be added in response. Justice STEVENS would ```ground the
prohibitions against state action squarely on due process, without
intermediate reliance on any of the first eight Amendments.''' Post
(quoting Malloy, (Harlan, J., dissenting)). The question presented in
this case, in his view, ``is whether the particular right asserted by
petitioners applies to the States because of the Fourteenth Amendment
itself, standing on its own bottom.'' Post. He would hold that
``{[}t{]}he rights protected against state infringement by the
Fourteenth Amendment's Due Process Clause need not be identical in shape
or scope to the rights protected against Federal Government infringement
by the various provisions of the Bill of Rights.'' Post.

As we have explained, the Court, for the past half-century, has moved
away from the two-track approach. If we were now to accept Justice
STEVENS' theory across the board, decades of decisions would be
undermined. We assume that this is not what is proposed. What is urged
instead, it appears, is that this theory be revived solely for the
individual right that Heller recognized, over vigorous dissents.

The relationship between the Bill of Rights' guarantees and the States
must be governed by a single, neutral principle. It is far too late to
exhume what Justice Brennan, writing for the Court 46 years ago, derided
as ``the notion that the Fourteenth Amendment applies to the States only
a watered-down, subjective version of the individual guarantees of the
Bill of Rights.'' Malloy, (internal quotation marks omitted).

Justice BREYER's conclusion that the Fourteenth Amendment does not
incorporate the right to keep and bear arms appears to rest primarily on
four factors: First, ``there is no popular consensus'' that the right is
fundamental, post; second, the right does not protect minorities or
persons neglected by those holding political power, post; third,
incorporation of the Second Amendment right would ``amount to a
significant incursion on a traditional and important area of state
concern, altering the constitutional relationship between the States and
the Federal Government'' and preventing local variations, post; and
fourth, determining the scope of the Second Amendment right in cases
involving state and local laws will force judges to answer difficult
empirical questions regarding matters that are outside their area of
expertise, post. Even if we believed that these factors were relevant to
the incorporation inquiry, none of these factors undermines the case for
incorporation of the right to keep and bear arms for self-defense.

First, we have never held that a provision of the Bill of Rights applies
to the States only if there is a ``popular consensus'' that the right is
fundamental, and we see no basis for such a rule. But in this case, as
it turns out, there is evidence of such a consensus. An amicus brief
submitted by 58 Members of the Senate and 251 Members of the House of
Representatives urges us to hold that the right to keep and bear arms is
fundamental. See Brief for Senator Kay Bailey Hutchison et al.~as Amici
Curiae 4. Another brief submitted by 38 States takes the same position.
Brief for State of Texas et al.~as Amici Curiae 6.

Second, petitioners and many others who live in high-crime areas dispute
the proposition that the Second Amendment right does not protect
minorities and those lacking political clout. The plight of Chicagoans
living in high-crime areas was recently highlighted when two Illinois
legislators representing Chicago districts called on the Governor to
deploy the Illinois National Guard to patrol the City's streets The
legislators noted that the number of Chicago homicide victims during the
current year equaled the number of American soldiers killed during that
same period in Afghanistan and Iraq and that 80\% of the Chicago victims
were black Amici supporting incorporation of the right to keep and bear
arms contend that the right is especially important for women and
members of other groups that may be especially vulnerable to violent
crime If, as petitioners believe, their safety and the safety of other
law-abiding members of the community would be enhanced by the possession
of handguns in the home for self-defense, then the Second Amendment
right protects the rights of minorities and other residents of
high-crime areas whose needs are not being met by elected public
officials.

Third, Justice BREYER is correct that incorporation of the Second
Amendment right will to some extent limit the legislative freedom of the
States, but this is always true when a Bill of Rights provision is
incorporated. Incorporation always restricts experimentation and local
variations, but that has not stopped the Court from incorporating
virtually every other provision of the Bill of Rights. ``{[}T{]}he
enshrinement of constitutional rights necessarily takes certain policy
choices off the table.'' Heller, This conclusion is no more remarkable
with respect to the Second Amendment than it is with respect to all the
other limitations on state power found in the Constitution.

Finally, Justice BREYER is incorrect that incorporation will require
judges to assess the costs and benefits of firearms restrictions and
thus to make difficult empirical judgments in an area in which they lack
expertise. As we have noted, while his opinion in Heller recommended an
interest-balancing test, the Court specifically rejected that
suggestion. See. ``The very enumeration of the right takes out of the
hands of government---even the Third Branch of Government---the power to
decide on a case-by-case basis whether the right is really worth
insisting upon.'' Heller\_\_\_.

In Heller, we held that the Second Amendment protects the right to
possess a handgun in the home for the purpose of self-defense. Unless
considerations of stare decisis counsel otherwise, a provision of the
Bill of Rights that protects a right that is fundamental from an
American perspective applies equally to the Federal Government and the
States. See Duncan, and n.~14, We therefore hold that the Due Process
Clause of the Fourteenth Amendment incorporates the Second Amendment
right recognized in Heller. The judgment of the Court of Appeals is
reversed, and the case is remanded for further proceedings.

\textbf{Justice SCALIA, concurring.}

I join the Court's opinion. Despite my misgivings about Substantive Due
Process as an original matter, I have acquiesced in the Court's
incorporation of certain guarantees in the Bill of Rights ``because it
is both long established and narrowly limited.'' Albright v. Oliver,
(SCALIA, J., concurring). This case does not require me to reconsider
that view, since straightforward application of settled doctrine
suffices to decide it.

I write separately only to respond to some aspects of Justice STEVENS'
dissent. Not that aspect which disagrees with the majority's application
of our precedents to this case, which is fully covered by the Court's
opinion. But much of what Justice STEVENS writes is a broad condemnation
of the theory of interpretation which underlies the Court's opinion, a
theory that makes the traditions of our people paramount. He proposes a
different theory, which he claims is more ``cautiou{[}s{]}'' and
respectful of proper limits on the judicial role. Post. It is that claim
I wish to address.

After stressing the substantive dimension of what he has renamed the
``liberty clause,'' post,1 Justice STEVENS proceeds to urge readoption
of the theory of incorporation articulated in Palko v. Connecticut, see
post. But in fact he does not favor application of that theory at all.
For whether Palko requires only that ``a fair and enlightened system of
justice would be impossible without'' the right sought to be
incorporated, or requires in addition that the right be rooted in the
``traditions and conscience of our people,'' (internal quotation marks
omitted), many of the rights Justice STEVENS thinks are incorporated
could not past muster under either test: abortion, post (citing Planned
Parenthood of Southeastern Pa. v. Casey, ); homosexual sodomy, post
(citing Lawrence v. Texas, ); the right to have excluded from criminal
trials evidence obtained in violation of the Fourth Amendment, post
(citing Mapp v. Ohio, 655-657, ); and the right to teach one's children
foreign languages, post (citing Meyer v. Nebraska, ), among others.

That Justice STEVENS is not applying any version of Palko is clear from
comparing, on the one hand, the rights he believes are covered, with, on
the other hand, his conclusion that the right to keep and bear arms is
not covered. Rights that pass his test include not just those ``relating
to marriage, procreation, contraception, family relationships, and child
rearing and education,'' but also rights against ``{[}g{]}overnment
action that shocks the conscience, pointlessly infringes settled
expectations, trespasses into sensitive private realms or life choices
without adequate justification, {[}or{]} perpetrates gross injustice.''
Post (internal quotation marks omitted). Not all such rights are in,
however, since only ``some fundamental aspects of personhood, dignity,
and the like'' are protected, post, at \_\_\_ (emphasis added). Exactly
what is covered is not clear. But whatever else is in, he knows that the
right to keep and bear arms is out, despite its being as ``deeply rooted
in this Nation's history and tradition,'' Washington v. Glucksberg,
(internal quotation marks omitted), as a right can be, see District of
Columbia v. Heller, 554 U.S. \textbf{\emph{, }} - \textbf{\emph{, }} -
\textbf{\emph{, }} - \_\_\_, 2801, I can find no other explanation for
such certitude except that Justice STEVENS, despite his forswearing of
``personal and private notions,'' post (internal quotation marks
omitted), deeply believes it should be out.

The subjective nature of Justice STEVENS' standard is also apparent from
his claim that it is the courts' prerogative--- indeed their duty---to
update the Due Process Clause so that it encompasses new freedoms the
Framers were too narrow-minded to imagine, post, and n.~21. Courts, he
proclaims, must ``do justice to {[}the Clause's{]} urgent call and its
open texture'' by exercising the ``interpretive discretion the latter
embodies.'' Post, (Why the people are not up to the task of deciding
what new rights to protect, even though it is they who are authorized to
make changes, see U.S. Const., Art. V, is never explained ) And it would
be ``judicial abdication'' for a judge to ``tur{[}n{]} his back'' on his
task of determining what the Fourteenth Amendment covers by
``outsourc{[}ing{]}'' the job to ``historical sentiment,'' post---that
is, by being guided by what the American people throughout our history
have thought. It is only we judges, exercising our ``own reasoned
judgment,'' post, who can be entrusted with deciding the Due Process
Clause's scope---which rights serve the Amendment's ``central values,''
post---which basically means picking the rights we want to protect and
discarding those we do not.

Justice STEVENS resists this description, insisting that his approach
provides plenty of ``guideposts'' and ``constraints'' to keep courts
from ``injecting excessive subjectivity'' into the process Post. Plenty
indeed---and that alone is a problem. The ability of omnidirectional
guideposts to constrain is inversely proportional to their number. But
even individually, each lodestar or limitation he lists either is
incapable of restraining judicial whimsy or cannot be squared with the
precedents he seeks to preserve.

He begins with a brief nod to history, post, but as he has just made
clear, he thinks historical inquiry unavailing, post. Moreover, trusting
the meaning of the Due Process Clause to what has historically been
protected is circular, see post, since that would mean no new rights
could get in.

Justice STEVENS moves on to the ``most basic'' constraint on
subjectivity his theory offers: that he would ``esche{[}w{]} attempts to
provide any all-purpose, top-down, totalizing theory of'liberty.'''
Post. The notion that the absence of a coherent theory of the Due
Process Clause will somehow curtail judicial caprice is at war with
reason. Indeterminacy means opportunity for courts to impose whatever
rule they like; it is the problem, not the solution. The idea that
interpretive pluralism would reduce courts' ability to impose their will
on the ignorant masses is not merely naive, but absurd. If there are no
right answers, there are no wrong answers either.

Justice STEVENS also argues that requiring courts to show ``respect for
the democratic process'' should serve as a constraint. Post. That is
true, but Justice STEVENS would have them show respect in an
extraordinary manner. In his view, if a right ``is already being given
careful consideration in, and subjected to ongoing calibration by, the
States, judicial enforcement may not be appropriate.'' In other words, a
right, such as the right to keep and bear arms, that has long been
recognized but on which the States are considering restrictions,
apparently deserves less protection, while a privilege the political
branches (instruments of the democratic process) have withheld entirely
and continue to withhold, deserves more. That topsy-turvy approach
conveniently accomplishes the objective of ensuring that the rights this
Court held protected in Casey, Lawrence, and other such cases fit the
theory---but at the cost of insulting rather than respecting the
democratic process.

The next constraint Justice STEVENS suggests is harder to evaluate. He
describes as ``an important tool for guiding judicial discretion''
``sensitivity to the interaction between the intrinsic aspects of
liberty and the practical realities of contemporary society.'' Post. I
cannot say whether that sensitivity will really guide judges because I
have no idea what it is. Is it some sixth sense instilled in judges when
they ascend to the bench? Or does it mean judges are more constrained
when they agonize about the cosmic conflict between liberty and its
potentially harmful consequences? Attempting to give the concept more
precision, Justice STEVENS explains that ``sensitivity is an aspect of a
deeper principle: the need to approach our work with humility and
caution.'' Both traits are undeniably admirable, though what relation
they bear to sensitivity is a mystery. But it makes no difference, for
the first case Justice STEVENS cites in support, see , Casey, dispels
any illusion that he has a meaningful form of judicial modesty in mind.

Justice STEVENS offers no examples to illustrate the next constraint:
stare decisis, post. But his view of it is surely not very confining,
since he holds out as a ``canonical'' exemplar of the proper approach,
see post, 3118, Lawrence, which overruled a case decided a mere 17 years
earlier, Bowers v. Hardwick. 539 U.S., (it ``was not correct when it was
decided, and it is not correct today''). Moreover, Justice STEVENS would
apply that constraint unevenly: He apparently approves those Warren
Court cases that adopted jot-for-jot incorporation of procedural
protections for criminal defendants, post, but would abandon those
Warren Court rulings that undercut his approach to substantive rights,
on the basis that we have ``cut back'' on cases from that era before,
post.

Justice STEVENS also relies on the requirement of a ``careful
description of the asserted fundamental liberty interest'' to limit
judicial discretion. Post (internal quotation marks omitted). I
certainly agree with that requirement, see Reno v. Flores, though some
cases Justice STEVENS approves have not applied it seriously, see, e.g.,
Lawrence, (``The instant case involves liberty of the person both in its
spatial and in its more transcendent dimensions''). But if the ``careful
description'' requirement is used in the manner we have hitherto
employed, then the enterprise of determining the Due Process Clause's
``conceptual core,'' post, is a waste of time. In the cases he cites we
sought a careful, specific description of the right at issue in order to
determine whether that right, thus narrowly defined, was fundamental.
See, e.g., Glucksberg, ; Reno, ; Collins v. Harker Heights, ; Cruzan v.
Director, Mo. Dept. of Health, ; see also Vacco v. Quill, The threshold
step of defining the asserted right with precision is entirely
unnecessary, however, if (as Justice STEVENS maintains) the ``conceptual
core'' of the ``liberty clause,'' post, includes a number of capacious,
hazily defined categories. There is no need to define the right with
much precision in order to conclude that it pertains to the plaintiff's
``ability independently to define {[}his{]} identity,'' his ``right to
make certain unusually important decisions that will affect his own, or
his family's, destiny,'' or some aspect of his
``{[}s{]}elf-determination, bodily integrity, freedom of conscience,
intimate relationships, political equality, dignity {[}or{]} respect.''
(internal quotation marks omitted). Justice STEVENS must therefore have
in mind some other use for the careful-description requirement---perhaps
just as a means of ensuring that courts ``procee{[}d{]} slowly and
incrementally,'' post. But that could be achieved just as well by having
them draft their opinions in longhand.

If Justice STEVENS' account of the constraints of his approach did not
demonstrate that they do not exist, his application of that approach to
the case before us leaves no doubt. He offers several reasons for
concluding that the Second Amendment right to keep and bear arms is not
fundamental enough to be applied against the States None is persuasive,
but more pertinent to my purpose, each is either intrinsically
indeterminate, would preclude incorporation of rights we have already
held incorporated, or both. His approach therefore does nothing to stop
a judge from arriving at any conclusion he sets out to reach.

Justice STEVENS begins with the odd assertion that ``firearms have a
fundamentally ambivalent relationship to liberty,'' since sometimes they
are used to cause (or sometimes accidentally produce) injury to others.
Post. The source of the rule that only nonambivalent liberties deserve
Due Process protection is never explained ---proof that judges applying
Justice STEVENS' approach can add new elements to the test as they see
fit. The criterion, moreover, is inherently manipulable. Surely Justice
STEVENS does not mean that the Clause covers only rights that have zero
harmful effect on anyone. Otherwise even the First Amendment is out.
Maybe what he means is that the right to keep and bear arms imposes too
great a risk to others' physical well-being. But as the plurality
explains, other rights we have already held incorporated pose similarly
substantial risks to public safety. In all events, Justice STEVENS
supplies neither a standard for how severe the impairment on others'
liberty must be for a right to be disqualified, nor (of course) any
method of measuring the severity.

Justice STEVENS next suggests that the Second Amendment right is not
fundamental because it is ``different in kind'' from other rights we
have recognized. Post. In one respect, of course, the right to keep and
bear arms is different from some other rights we have held the Clause
protects and he would recognize: It is deeply grounded in our nation's
history and tradition. But Justice STEVENS has a different distinction
in mind: Even though he does ``not doubt for a moment that many
Americans . see {[}firearms{]} as critical to their way of life as well
as to their security,'' he pronounces that owning a handgun is not
``critical to leading a life of autonomy, dignity, or political
equality.'' Post. Who says? Deciding what is essential to an
enlightened, liberty-filled life is an inherently political, moral
judgment---the antithesis of an objective approach that reaches
conclusions by applying neutral rules to verifiable evidence.

No determination of what rights the Constitution of the United States
covers would be complete, of course, without a survey of what other
countries do. Post. When it comes to guns, Justice STEVENS explains, our
Nation is already an outlier among ``advanced democracies''; not even
our ``oldest allies'' protect as robust a right as we do, and we should
not widen the gap. Never mind that he explains neither which countries
qualify as ``advanced democracies'' nor why others are irrelevant. For
there is an even clearer indication that this criterion lets judges pick
which rights States must respect and those they can ignore: As the
plurality shows, and nn.~28-29, this follow-the-foreign-crowd
requirement would foreclose rights that we have held (and Justice
STEVENS accepts) are incorporated, but that other ``advanced'' nations
do not recognize---from the exclusionary rule to the Establishment
Clause. A judge applying Justice STEVENS' approach must either throw all
of those rights overboard or, as cases Justice STEVENS approves have
done in considering unenumerated rights, simply ignore foreign law when
it undermines the desired conclusion, see, e.g., Casey(making no mention
of foreign law).

Justice STEVENS also argues that since the right to keep and bear arms
was codified for the purpose of ``prevent{[}ing{]} elimination of the
militia,'' it should be viewed as ```a federalism provision''' logically
incapable of incorporation. Post (quoting Elk Grove Unified School Dist.
v. Newdow, (THOMAS, J., concurring in judgment); some internal quotation
marks omitted). This criterion, too, evidently applies only when judges
want it to. The opinion Justice STEVENS quotes for the ``federalism
provision'' principle, Justice THOMAS's concurrence in Newdow, argued
that incorporation of the Establishment Clause ``makes little sense''
because that Clause was originally understood as a limit on
congressional interference with state establishments of religion. -51,
Justice STEVENS, of course, has no problem with applying the
Establishment Clause to the States. See, e.g., n.~4, (opinion for the
Court by STEVENS, J.) (acknowledging that the Establishment Clause
``appl{[}ies{]} to the States by incorporation into the Fourteenth
Amendment''). While he insists that Clause is not a ``federalism
provision,'' post, n.~40, he does not explain why it is not, but the
right to keep and bear arms is (even though only the latter refers to a
``right of the people''). The ``federalism'' argument prevents the
incorporation of only certain rights.

Justice STEVENS next argues that even if the right to keep and bear arms
is ``deeply rooted in some important senses,'' the roots of States'
efforts to regulate guns run just as deep. Post (internal quotation
marks omitted). But this too is true of other rights we have held
incorporated. No fundamental right---not even the First Amendment---is
absolute. The traditional restrictions go to show the scope of the
right, not its lack of fundamental character. At least that is what they
show (Justice STEVENS would agree) for other rights. Once again,
principles are applied selectively.

Justice STEVENS' final reason for rejecting incorporation of the Second
Amendment reveals, more clearly than any of the others, the game that is
afoot. Assuming that there is a ``plausible constitutional basis'' for
holding that the right to keep and bear arms is incorporated, he asserts
that we ought not to do so for prudential reasons. Post. Even if we had
the authority to withhold rights that are within the Constitution's
command (and we assuredly do not), two of the reasons Justice STEVENS
gives for abstention show just how much power he would hand to judges.
The States'``right to experiment'' with solutions to the problem of gun
violence, he says, is at its apex here because ``the best solution is
far from clear.'' Post (internal quotation marks omitted). That is true
of most serious social problems---whether, for example, ``the best
solution'' for rampant crime is to admit confessions unless they are
affirmatively shown to have been coerced, but see Miranda v. Arizona, or
to permit jurors to impose the death penalty without a requirement that
they be free to consider ``any relevant mitigating factor,'' see Eddings
v. Oklahoma, which in turn leads to the conclusion that defense counsel
has provided inadequate defense if he has not conducted a ``reasonable
investigation'' into potentially mitigating factors, see, e.g., Wiggins
v. Smith, inquiry into which question tends to destroy any prospect of
prompt justice, see, e.g., Wong v. Belmontes, 558 U.S. \_\_\_, (per
curiam) (reversing grant of habeas relief for sentencing on a crime
committed in 1981). The obviousness of the optimal answer is in the eye
of the beholder. The implication of Justice STEVENS' call for abstention
is that if We The Court conclude that They The People's answers to a
problem are silly, we are free to ``interven{[}e{]},'' post, but if we
too are uncertain of the right answer, or merely think the States may be
on to something, we can loosen the leash.

A second reason Justice STEVENS says we should abstain is that the
States have shown they are ``capable'' of protecting the right at issue,
and if anything have protected it too much. Post. That reflects an
assumption that judges can distinguish between a proper democratic
decision to leave things alone (which we should honor), and a case of
democratic market failure (which we should step in to correct). I would
not---and no judge should--- presume to have that sort of omniscience,
which seems to me far more ``arrogant,'' post, than confining courts'
focus to our own national heritage.

Justice STEVENS' response to this concurrence, post, makes the usual
rejoinder of ``living Constitution'' advocates to the criticism that it
empowers judges to eliminate or expand what the people have prescribed:
The traditional, historically focused method, he says, reposes
discretion in judges as well Historical analysis can be difficult; it
sometimes requires resolving threshold questions, and making nuanced
judgments about which evidence to consult and how to interpret it.

I will stipulate to that But the question to be decided is not whether
the historically focused method is a perfect means of restraining
aristocratic judicial Constitution-writing; but whether it is the best
means available in an imperfect world. Or indeed, even more narrowly
than that: whether it is demonstrably much better than what Justice
STEVENS proposes. I think it beyond all serious dispute that it is much
less subjective, and intrudes much less upon the democratic process. It
is less subjective because it depends upon a body of evidence
susceptible of reasoned analysis rather than a variety of vague
ethico-political First Principles whose combined conclusion can be found
to point in any direction the judges favor. In the most controversial
matters brought before this Court---for example, the constitutionality
of prohibiting abortion, assisted suicide, or homosexual sodomy, or the
constitutionality of the death penalty---any historical methodology,
under any plausible standard of proof, would lead to the same conclusion
Moreover, the methodological differences that divide historians, and the
varying interpretive assumptions they bring to their work, post, are
nothing compared to the differences among the American people (though
perhaps not among graduates of prestigious law schools) with regard to
the moral judgments Justice STEVENS would have courts pronounce. And
whether or not special expertise is needed to answer historical
questions, judges most certainly have no ``comparative . advantage,''
post (internal quotation marks omitted), in resolving moral disputes.
What is more, his approach would not eliminate, but multiply, the hard
questions courts must confront, since he would not replace history with
moral philosophy, but would have courts consider both.

And the Court's approach intrudes less upon the democratic process
because the rights it acknowledges are those established by a
constitutional history formed by democratic decisions; and the rights it
fails to acknowledge are left to be democratically adopted or rejected
by the people, with the assurance that their decision is not subject to
judicial revision. Justice STEVENS' approach, on the other hand,
deprives the people of that power, since whatever the Constitution and
laws may say, the list of protected rights will be whatever courts wish
it to be. After all, he notes, the people have been wrong before, post,
and courts may conclude they are wrong in the future. Justice STEVENS
abhors a system in which ``majorities or powerful interest groups always
get their way,'' post, but replaces it with a system in which unelected
and life-tenured judges always get their way. That such usurpation is
effected unabashedly, see post---with ``the judge's cards . laid on the
table,'' ---makes it even worse. In a vibrant democracy, usurpation
should have to be accomplished in the dark. It is Justice STEVENS'
approach, not the Court's, that puts democracy in peril.

I do not entirely understand Justice STEVENS' renaming of the Due
Process Clause. What we call it, of course, does not change what the
Clause says, but shorthand should not obscure what it says. Accepting
for argument's sake the shift in emphasis---from avoiding certain
deprivations without that ``process'' which is ``due,'' to avoiding the
deprivations themselves---the Clause applies not just to deprivations of
``liberty,'' but also to deprivations of ``life'' and even ``property.''

Justice STEVENS insists that he would not make courts the sole
interpreters of the ``liberty clause''; he graciously invites
``{[}a{]}ll Americans'' to ponder what the Clause means to them today.
Post, n.~22. The problem is that in his approach the people's ponderings
do not matter, since whatever the people decide, courts have the last
word.

Justice BREYER is not worried by that prospect. His interpretive
approach applied to incorporation of the Second Amendment includes
consideration of such factors as ``the extent to which incorporation
will further other, perhaps more basic, constitutional aims; and the
extent to which incorporation will advance or hinder the Constitution's
structural aims''; whether recognizing a particular right will ``further
the Constitution's effort to ensure that the government treats each
individual with equal respect'' or will ``help maintain the democratic
form of government''; whether it is ``inconsistent . with the
Constitution's efforts to create governmental institutions well suited
to the carrying out of its constitutional promises''; whether it fits
with ``the Framers' basic reason for believing the Court ought to have
the power of judicial review''; courts' comparative advantage in
answering empirical questions that may be involved in applying the
right; and whether there is a ``strong offsetting justification'' for
removing a decision from the democratic process. Post, 3125-3129
(dissenting opinion).

After defending the careful-description criterion, Justice STEVENS
quickly retreats and cautions courts not to apply it too stringently.
Post. Describing a right too specifically risks robbing it of its
``universal valence and a moral force it might otherwise have,'' , and
``loads the dice against its recognition,'' post, n.~25 (internal
quotation marks omitted). That must be avoided, since it endangers
rights Justice STEVENS does like. See (discussing Lawrence v. Texas). To
make sure those rights get in, we must leave leeway in our description,
so that a right that has not itself been recognized as fundamental can
ride the coattails of one that has been.

Justice STEVENS claims that I mischaracterize his argument by referring
to the Second Amendment right to keep and bear arms, instead of ``the
interest in keeping a firearm of one's choosing in the home,'' the right
he says petitioners assert. Post, n.~36. But it is precisely the
``Second Amendment right to keep and bear arms'' that petitioners argue
is incorporated by the Due Process Clause. See, e.g., Pet. for Cert. i.
Under Justice STEVENS' own approach, that should end the matter. See
post (``{[}W{]}e must pay close attention to the precise liberty
interest the litigants have asked us to vindicate''). In any event, the
demise of watered-down incorporation, see means that we no longer
subdivide Bill of Rights guarantees into their theoretical components,
only some of which apply to the States. The First Amendment freedom of
speech is incorporated---not the freedom to speak on Fridays, or to
speak about philosophy.

Justice STEVENS goes a step farther still, suggesting that the right to
keep and bear arms is not protected by the ``liberty clause'' because it
is not really a liberty at all, but a ``property right.'' Post. Never
mind that the right to bear arms sounds mighty like a liberty; and never
mind that the ``liberty clause'' is really a Due Process Clause which
explicitly protects ``property,'' see United States v. Carlton, (SCALIA,
J., concurring in judgment). Justice STEVENS' theory cannot explain why
the Takings Clause, which unquestionably protects property, has been
incorporated, see Chicago, B. \& Q.R. Co.~v. Chicago, in a decision he
appears to accept, post, n.~14.

As Justice STEVENS notes, see post, I accept as a matter of stare
decisis the requirement that to be fundamental for purposes of the Due
Process Clause, a right must be ``implicit in the concept of ordered
liberty,'' Lawrence, n.~3, (SCALIA, J., dissenting) (internal quotation
marks omitted). But that inquiry provides infinitely less scope for
judicial invention when conducted under the Court's approach, since the
field of candidates is immensely narrowed by the prior requirement that
a right be rooted in this country's traditions. Justice STEVENS, on the
other hand, is free to scan the universe for rights that he thinks
``implicit in the concept, etc.'' The point Justice STEVENS makes here
is merely one example of his demand that an historical approach to the
Constitution prove itself, not merely much better than his in
restraining judicial invention, but utterly perfect in doing so. See
Part III, infra.

Justice STEVENS also asserts that his approach is ``more faithful to
this Nation's constitutional history'' and to ``the values and
commitments of the American people, as they stand today,'' post. But
what he asserts to be the proof of this is that his approach aligns (no
surprise) with those cases he approves (and dubs ``canonical,'' ). Cases
he disfavors are discarded as ``hardly bind{[}ing{]}'' ``excesses,''
post, or less ``enduring,'' post, n.~16. Not proven. Moreover, whatever
relevance Justice STEVENS ascribes to current ``values and commitments
of the American people'' (and that is unclear, see post, n.~47), it is
hard to see how it shows fidelity to them that he disapproves a
different subset of old cases than the Court does.

That is not to say that every historical question on which there is room
for debate is indeterminate, or that every question on which historians
disagree is equally balanced. Cf. post. For example, the historical
analysis of the principal dissent in Heller is as valid as the Court's
only in a two-dimensional world that conflates length and depth.

By the way, Justice STEVENS greatly magnifies the difficulty of an
historical approach by suggesting that it was my burden in Lawrence to
show the ``ancient roots of proscriptions against sodomy,'' post
(internal quotation marks omitted). Au contraire, it was his burden (in
the opinion he joined) to show the ancient roots of the right of sodomy.

Justice THOMAS, concurring in part and concurring in the judgment. I
agree with the Court that the Fourteenth Amendment makes the right to
keep and bear arms set forth in the Second Amendment ``fully applicable
to the States.'' I write separately because I believe there is a more
straightforward path to this conclusion, one that is more faithful to
the Fourteenth Amendment's text and history.

Applying what is now a well-settled test, the plurality opinion
concludes that the right to keep and bear arms applies to the States
through the Fourteenth Amendment's Due Process Clause because it is
``fundamental'' to the American ``scheme of ordered liberty,'' (citing
Duncan v. Louisiana, ), and ```deeply rooted in this Nation's history
and tradition,''' (quoting Washington v. Glucksberg, ). I agree with
that description of the right. But I cannot agree that it is enforceable
against the States through a clause that speaks only to ``process.''
Instead, the right to keep and bear arms is a privilege of American
citizenship that applies to the States through the Fourteenth
Amendment's Privileges or Immunities Clause.

The provision at issue here, § 1 of the Fourteenth Amendment,
significantly altered our system of government. The first sentence of
that section provides that ``{[}a{]}ll persons born or naturalized in
the United States and subject to the jurisdiction thereof, are citizens
of the United States and of the State wherein they reside.'' This
unambiguously overruled this Court's contrary holding in Dred Scott v.
Sandford, 19 How. 393, that the Constitution did not recognize black
Americans as citizens of the United States or their own State. -406.

The meaning of § 1's next sentence has divided this Court for many
years. That sentence begins with the command that ``{[}n{]}o State shall
make or enforce any law which shall abridge the privileges or immunities
of citizens of the United States.'' On its face, this appears to grant
the persons just made United States citizens a certain collection of
rights---i.e., privileges or immunities---attributable to that status.

This Court's precedents accept that point, but define the relevant
collection of rights quite narrowly. In the Slaughter-House Cases, 16
Wall. 36, decided just five years after the Fourteenth Amendment's
adoption, the Court interpreted this text, now known as the Privileges
or Immunities Clause, for the first time. In a closely divided decision,
the Court drew a sharp distinction between the privileges and immunities
of state citizenship and those of federal citizenship, and held that the
Privileges or Immunities Clause protected only the latter category of
rights from state abridgment. The Court defined that category to include
only those rights ``which owe their existence to the Federal government,
its National character, its Constitution, or its laws.'' This arguably
left open the possibility that certain individual rights enumerated in
the Constitution could be considered privileges or immunities of federal
citizenship. See (listing ``{[}t{]}he right to peaceably assemble'' and
``the privilege of the writ of habeas corpus'' as rights potentially
protected by the Privileges or Immunities Clause). But the Court soon
rejected that proposition, interpreting the Privileges or Immunities
Clause even more narrowly in its later cases.

Chief among those cases is United States v. CruikshankL.Ed. 588 (1876).
There, the Court held that members of a white militia who had brutally
murdered as many as 165 black Louisianians congregating outside a
courthouse had not deprived the victims of their privileges as American
citizens to peaceably assemble or to keep and bear arms. ; see L. Keith,
The Colfax Massacre 109 (2008). According to the Court, the right to
peaceably assemble codified in the First Amendment was not a privilege
of United States citizenship because ``{[}t{]}he right . existed long
before the adoption of the Constitution.'' 92 U.S. (emphasis added).
Similarly, the Court held that the right to keep and bear arms was not a
privilege of United States citizenship because it was not ``in any
manner dependent upon that instrument for its existence.'' In other
words, the reason the Framers codified the right to bear arms in the
Second Amendment---its nature as an inalienable right that pre-existed
the Constitution's adoption---was the very reason citizens could not
enforce it against States through the Fourteenth.

That circular reasoning effectively has been the Court's last word on
the Privileges or Immunities Clause In the intervening years, the Court
has held that the Clause prevents state abridgment of only a handful of
rights, such as the right to travel, see Saenz v. Roe, that are not
readily described as essential to liberty.

As a consequence of this Court's marginalization of the Clause,
litigants seeking federal protection of fundamental rights turned to the
remainder of § 1 in search of an alternative fount of such rights. They
found one in a most curious place---that section's command that every
State guarantee ``due process'' to any person before depriving him of
``life, liberty, or property.'' At first, litigants argued that this Due
Process Clause ``incorporated'' certain procedural rights codified in
the Bill of Rights against the States. The Court generally rejected
those claims, however, on the theory that the rights in question were
not sufficiently ``fundamental'' to warrant such treatment. See, e.g.,
Hurtado v. California (grand jury indictment requirement); Maxwell v.
Dow (12-person jury requirement); Twining v. New Jersey (privilege
against self-incrimination).

That changed with time. The Court came to conclude that certain Bill of
Rights guarantees were sufficiently fundamental to fall within § 1's
guarantee of ``due process.'' These included not only procedural
protections listed in the first eight Amendments, see, e.g., Benton v.
Maryland (protection against double jeopardy), but substantive rights as
well, see, e.g., Gitlow v. New York, (right to free speech); Near v.
Minnesota ex rel. Olson, (same). In the process of incorporating these
rights against the States, the Court often applied them differently
against the States than against the Federal Government on the theory
that only those ``fundamental'' aspects of the right required Due
Process Clause protection. See, e.g., Betts v. Brady, (holding that the
Sixth Amendment required the appointment of counsel in all federal
criminal cases in which the defendant was unable to retain an attorney,
but that the Due Process Clause required appointment of counsel in state
criminal cases only where ``want of counsel . result{[}ed{]} in a
conviction lacking in . fundamental fairness''). In more recent years,
this Court has ``abandoned the notion'' that the guarantees in the Bill
of Rights apply differently when incorporated against the States than
they do when applied to the Federal Government. (opinion of the Court)
(internal quotation marks omitted). But our cases continue to adhere to
the view that a right is incorporated through the Due Process Clause
only if it is sufficiently ``fundamental,'' -3050 (plurality
opinion)---a term the Court has long struggled to define.

While this Court has at times concluded that a right gains
``fundamental'' status only if it is essential to the American ``scheme
of ordered liberty'' or ```deeply rooted in this Nation's history and
tradition,''' (plurality opinion) (quoting Glucksberg, ), the Court has
just as often held that a right warrants Due Process Clause protection
if it satisfies a far less measurable range of criteria, see Lawrence v.
Texas, (concluding that the Due Process Clause protects ``liberty of the
person both in its spatial and in its more transcendent dimensions'').
Using the latter approach, the Court has determined that the Due Process
Clause applies rights against the States that are not mentioned in the
Constitution at all, even without seriously arguing that the Clause was
originally understood to protect such rights. See, e.g., Lochner v. New
York; Roe v. Wade; Lawrence.

All of this is a legal fiction. The notion that a constitutional
provision that guarantees only ``process'' before a person is deprived
of life, liberty, or property could define the substance of those rights
strains credulity for even the most casual user of words. Moreover, this
fiction is a particularly dangerous one. The one theme that links the
Court's substantive due process precedents together is their lack of a
guiding principle to distinguish ``fundamental'' rights that warrant
protection from nonfundamental rights that do not. Today's decision
illustrates the point. Replaying a debate that has endured from the
inception of the Court's substantive due process jurisprudence, the
dissents laud the ``flexibility'' in this Court's substantive due
process doctrine, post (STEVENS, J., dissenting); see post (BREYER, J.,
dissenting), while the plurality makes yet another effort to impose
principled restraints on its exercise, see But neither side argues that
the meaning they attribute to the Due Process Clause was consistent with
public understanding at the time of its ratification.

To be sure, the plurality's effort to cabin the exercise of judicial
discretion under the Due Process Clause by focusing its inquiry on those
rights deeply rooted in American history and tradition invites less
opportunity for abuse than the alternatives. See post (BREYER, J.,
dissenting) (arguing that rights should be incorporated against the
States through the Due Process Clause if they are ``well-suited to the
carrying out of . constitutional promises''); post (STEVENS, J.,
dissenting) (warning that there is no ``all-purpose, top-down,
totalizing theory of'liberty''' protected by the Due Process Clause).
But any serious argument over the scope of the Due Process Clause must
acknowledge that neither its text nor its history suggests that it
protects the many substantive rights this Court's cases now claim it
does.

I cannot accept a theory of constitutional interpretation that rests on
such tenuous footing. This Court's substantive due process framework
fails to account for both the text of the Fourteenth Amendment and the
history that led to its adoption, filling that gap with a jurisprudence
devoid of a guiding principle. I believe the original meaning of the
Fourteenth Amendment offers a superior alternative, and that a return to
that meaning would allow this Court to enforce the rights the Fourteenth
Amendment is designed to protect with greater clarity and predictability
than the substantive due process framework has so far managed.

I acknowledge the volume of precedents that have been built upon the
substantive due process framework, and I further acknowledge the
importance of stare decisis to the stability of our Nation's legal
system. But stare decisis is only an ``adjunct'' of our duty as judges
to decide by our best lights what the Constitution means. Planned
Parenthood of Southeastern Pa. v. Casey, (Rehnquist, C. J., concurring
in judgment in part and dissenting in part). It is not ``an inexorable
command.'' Lawrence, Moreover, as judges, we interpret the Constitution
one case or controversy at a time. The question presented in this case
is not whether our entire Fourteenth Amendment jurisprudence must be
preserved or revised, but only whether, and to what extent, a particular
clause in the Constitution protects the particular right at issue here.
With the inquiry appropriately narrowed, I believe this case presents an
opportunity to reexamine, and begin the process of restoring, the
meaning of the Fourteenth Amendment agreed upon by those who ratified
it.

``It cannot be presumed that any clause in the constitution is intended
to be without effect.'' Marbury v. Madison, 1 Cranch 137, 174,
(Marshall, C. J.). Because the Court's Privileges or Immunities Clause
precedents have presumed just that, I set them aside for the moment and
begin with the text.

The Privileges or Immunities Clause of the Fourteenth Amendment declares
that ``{[}n{]}o State . shall abridge the privileges or immunities of
citizens of the United States.'' In interpreting this language, it is
important to recall that constitutional provisions are ```written to be
understood by the voters.''' Heller, 128 S.Ct. (quoting United States v.
Sprague, ). Thus, the objective of this inquiry is to discern what
``ordinary citizens'' at the time of ratification would have understood
the Privileges or Immunities Clause to mean.

The text examined so far demonstrates three points about the meaning of
the Privileges or Immunities Clause in § 1. First, ``privileges'' and
``immunities'' were synonyms for ``rights.'' Second, both the States and
the Federal Government had long recognized the inalienable rights of
their citizens. Third, Article IV, § 2 of the Constitution protected
traveling citizens against state discrimination with respect to the
fundamental rights of state citizenship.

Section 1 overruled Dred Scott's holding that blacks were not citizens
of either the United States or their own State and, thus, did not enjoy
``the privileges and immunities of citizens'' embodied in the
Constitution. 19 How.. The Court in Dred Scott did not distinguish
between privileges and immunities of citizens of the United States and
citizens in the several States, instead referring to the rights of
citizens generally. It did, however, give examples of what the rights of
citizens were---the constitutionally enumerated rights of ``the full
liberty of speech'' and the right ``to keep and carry arms.''

This evidence plainly shows that the ratifying public understood the
Privileges or Immunities Clause to protect constitutionally enumerated
rights, including the right to keep and bear arms. As the Court
demonstrates, there can be no doubt that § 1 was understood to enforce
the Second Amendment against the States. See In my view, this is because
the right to keep and bear arms was understood to be a privilege of
American citizenship guaranteed by the Privileges or Immunities Clause.

\textbf{Justice STEVENS, dissenting.}

I further agree with the plurality that there are weighty arguments
supporting petitioners' second submission, insofar as it concerns the
possession of firearms for lawful self-defense in the home. But these
arguments are less compelling than the plurality suggests; they are much
less compelling when applied outside the home; and their validity does
not depend on the Court's holding in Heller. For that holding sheds no
light on the meaning of the Due Process Clause of the Fourteenth
Amendment. Our decisions construing that Clause to render various
procedural guarantees in the Bill of Rights enforceable against the
States likewise tell us little about the meaning of the word ``liberty''
in the Clause or about the scope of its protection of nonprocedural
rights.

This is a substantive due process case. Section 1 of the Fourteenth
Amendment decrees that no State shall ``deprive any person of life,
liberty, or property, without due process of law.'' The Court has filled
thousands of pages expounding that spare text. As I read the vast corpus
of substantive due process opinions, they confirm several important
principles that ought to guide our resolution of this case. The
principal opinion's lengthy summary of our ``incorporation'' doctrine,
see -3036 (majority opinion), 3030-3031 (plurality opinion), and its
implicit (and untenable) effort to wall off that doctrine from the rest
of our substantive due process jurisprudence, invite a fresh survey of
this old terrain.

The first, and most basic, principle established by our cases is that
the rights protected by the Due Process Clause are not merely procedural
in nature. At first glance, this proposition might seem surprising,
given that the Clause refers to ``process.'' But substance and procedure
are often deeply entwined. Upon closer inspection, the text can be read
to ``impos{[}e{]} nothing less than an obligation to give substantive
content to the words'liberty' and'due process of law,''' Washington v.
Glucksberg, (Souter, J., concurring in judgment), lest superficially
fair procedures be permitted to ``destroy the enjoyment'' of life,
liberty, and property, Poe v. Ullman, (Harlan, J., dissenting), and the
Clause's prepositional modifier be permitted to swallow its primary
command. Procedural guarantees are hollow unless linked to substantive
interests; and no amount of process can legitimize some deprivations.

I have yet to see a persuasive argument that the Framers of the
Fourteenth Amendment thought otherwise. To the contrary, the historical
evidence suggests that, at least by the time of the Civil War if not
much earlier, the phrase ``due process of law'' had acquired substantive
content as a term of art within the legal community This understanding
is consonant with the venerable ``notion that governmental authority has
implied limits which preserve private autonomy,'' a notion which
predates the founding and which finds reinforcement in the
Constitution's Ninth Amendment, see Griswold v. Connecticut, (Goldberg,
J., concurring) The Due Process Clause cannot claim to be the source of
our basic freedoms---no legal document ever could, see Meachum v. Fano,
(STEVENS, J., dissenting)--- but it stands as one of their foundational
guarantors in our law.

If text and history are inconclusive on this point, our precedent leaves
no doubt: It has been ``settled'' for well over a century that the Due
Process Clause ``applies to matters of substantive law as well as to
matters of procedure.'' Whitney v. California, (Brandeis, J.,
concurring). Time and again, we have recognized that in the Fourteenth
Amendment as well as the Fifth, the ``Due Process Clause guarantees more
than fair process, and the'liberty' it protects includes more than the
absence of physical restraint.'' Glucksberg, ``The Clause also includes
a substantive component that'provides heightened protection against
government interference with certain fundamental rights and liberty
interests.''' Troxel v. Granville, (opinion of O'Connor, J., joined by
Rehnquist, C.J., and GINSBURG and BREYER, JJ.) (quoting Glucksberg, ).
Some of our most enduring precedents, accepted today by virtually
everyone, were substantive due process decisions. See, e.g., Loving v.
Virginia, (recognizing due-process as well as equal-protection-based
right to marry person of another race); Bolling v. Sharpe, (outlawing
racial segregation in District of Columbia public schools); Pierce v.
Society of Sisters, (vindicating right of parents to direct upbringing
and education of their children); Meyer v. Nebraska, (striking down
prohibition on teaching of foreign languages).

The second principle woven through our cases is that substantive due
process is fundamentally a matter of personal liberty. For it is the
liberty clause of the Fourteenth Amendment that grounds our most
important holdings in this field. It is the liberty clause that enacts
the Constitution's ``promise'' that a measure of dignity and self-rule
will be afforded to all persons. Planned Parenthood of Southeastern Pa.
v. Casey, It is the liberty clause that reflects and renews ``the
origins of the American heritage of freedom {[}and{]} the abiding
interest in individual liberty that makes certain state intrusions on
the citizen's right to decide how he will live his own life
intolerable.'' Fitzgerald v. Porter Memorial Hospital, C.A 1975)
(Stevens, J.). Our substantive due process cases have episodically
invoked values such as privacy and equality as well, values that in
certain contexts may intersect with or complement a subject's liberty
interests in profound ways. But as I have observed on numerous
occasions, ``most of the significant {[}20th-century{]} cases raising
Bill of Rights issues have, in the final analysis, actually interpreted
the word'liberty' in the Fourteenth Amendment.''

It follows that the term ``incorporation,'' like the term ``unenumerated
rights,'' is something of a misnomer. Whether an asserted substantive
due process interest is explicitly named in one of the first eight
Amendments to the Constitution or is not mentioned, the underlying
inquiry is the same: We must ask whether the interest is ``comprised
within the term liberty.'' Whitney, (Brandeis, J., concurring). As the
second Justice Harlan has shown, ever since the Court began considering
the applicability of the Bill of Rights to the States, ``the Court's
usual approach has been to ground the prohibitions against state action
squarely on due process, without intermediate reliance on any of the
first eight Amendments.'' Malloy v. Hogan, (dissenting opinion); see
also Frankfurter, Memorandum on ``Incorporation'' of the Bill of Rights
into the Due Process Clause of the Fourteenth Amendment, 78 Harv. L.Rev.
746, 747-750 (1965). In the pathmarking case of Gitlow v. New York, for
example, both the majority and dissent evaluated petitioner's free
speech claim not under the First Amendment but as an aspect of ``the
fundamental personal rights and'liberties' protected by the due process
clause of the Fourteenth Amendment from impairment by the States.''

In his own classic opinion in Griswold, (concurring in judgment),
Justice Harlan memorably distilled these precedents' lesson: ``While the
relevant inquiry may be aided by resort to one or more of the provisions
of the Bill of Rights, it is not dependent on them or any of their
radiations. The Due Process Clause of the Fourteenth Amendment stands
\ldots{} on its own bottom.''10 Inclusion in the Bill of Rights is
neither necessary nor sufficient for an interest to be judicially
enforceable under the Fourteenth Amendment. This Court's ```selective
incorporation''' doctrine, is not simply ``related'' to substantive due
process, ; it is a subset thereof.

The third precept to emerge from our case law flows from the second: The
rights protected against state infringement by the Fourteenth
Amendment's Due Process Clause need not be identical in shape or scope
to the rights protected against Federal Government infringement by the
various provisions of the Bill of Rights. As drafted, the Bill of Rights
directly constrained only the Federal Government. See Barron ex rel.
Tiernan v. Mayor of Baltimore, 7 Pet. 243, Although the enactment of the
Fourteenth Amendment profoundly altered our legal order, it ``did not
unstitch the basic federalist pattern woven into our constitutional
fabric.'' Williams v. Florida, (Harlan, J., concurring in result). Nor,
for that matter, did it expressly alter the Bill of Rights. The
Constitution still envisions a system of divided sovereignty, still
``establishes a federal republic where local differences are to be
cherished as elements of liberty'' in the vast run of cases, National
Rifle Assn. of Am. Inc.~v. Chicago, C.A 2009) (Easterbrook, C. J.),
still allocates a general ``police power\ldots{} to the States and the
States alone,'' United States v. Comstock, 560 U.S. \textbf{\emph{, }},
(KENNEDY, J., concurring in judgment). Elementary considerations of
constitutional text and structure suggest there may be legitimate
reasons to hold state governments to different standards than the
Federal Government in certain areas.

It is true, as the Court emphasizes, that we have made numerous
provisions of the Bill of Rights fully applicable to the States. It is
settled, for instance, that the Governor of Alabama has no more power
than the President of the United States to authorize unreasonable
searches and seizures. Ker v. CaliforniaBut we have never accepted a
``total incorporation'' theory of the Fourteenth Amendment, whereby the
Amendment is deemed to subsume the provisions of the Bill of Rights en
masse. See And we have declined to apply several provisions to the
States in any measure. See, e.g., Minneapolis \& St.~Louis R. Co.~v.
Bombolis (Seventh Amendment); Hurtado v. California (Grand Jury Clause).
We have, moreover, resisted a uniform approach to the Sixth Amendment's
criminal jury guarantee, demanding 12-member panels and unanimous
verdicts in federal trials, yet not in state trials. See Apodaca v.
Oregon (plurality opinion); WilliamsIn recent years, the Court has
repeatedly declined to grant certiorari to review that disparity While
those denials have no precedential significance, they confirm the
proposition that the ``incorporation'' of a provision of the Bill of
Rights into the Fourteenth Amendment does not, in itself, mean the
provision must have precisely the same meaning in both contexts.

It is true, as well, that during the 1960's the Court decided a number
of cases involving procedural rights in which it treated the Due Process
Clause as if it transplanted language from the Bill of Rights into the
Fourteenth Amendment. See, e.g., Benton v. Maryland, (Double Jeopardy
Clause); Pointer v. Texas, (Confrontation Clause). ``Jot-for-jot''
incorporation was the norm in this expansionary era. Yet at least one
subsequent opinion suggests that these precedents require perfect
state/federal congruence only on matters ```at the core''' of the
relevant constitutional guarantee. Crist v. Bretz, ; see also S.Ct. 2156
(Powell, J., dissenting). In my judgment, this line of cases is best
understood as having concluded that, to ensure a criminal trial
satisfies essential standards of fairness, some procedures should be the
same in state and federal courts: The need for certainty and uniformity
is more pressing, and the margin for error slimmer, when criminal
justice is at issue. That principle has little relevance to the question
whether a non procedural rule set forth in the Bill of Rights qualifies
as an aspect of the liberty protected by the Fourteenth Amendment.

Notwithstanding some overheated dicta in Malloy, it is therefore an
overstatement to say that the Court has ``abandoned,'' (majority
opinion), 3047 (plurality opinion), a ``two-track approach to
incorporation,'' (plurality opinion). The Court moved away from that
approach in the area of criminal procedure. But the Second Amendment
differs in fundamental respects from its neighboring provisions in the
Bill of Rights, as I shall explain in Part V, infra; and if some 1960's
opinions purported to establish a general method of incorporation, that
hardly binds us in this case. The Court has not hesitated to cut back on
perceived Warren Court excesses in more areas than I can count.

I do not mean to deny that there can be significant practical, as well
as esthetic, benefits from treating rights symmetrically with regard to
the State and Federal Governments. Jot-for-jot incorporation of a
provision may entail greater protection of the right at issue and
therefore greater freedom for those who hold it; jot-for-jot
incorporation may also yield greater clarity about the contours of the
legal rule. See Johnson v. Louisiana, (Douglas, J., dissenting);
Pointer, (Goldberg, J., concurring). In a federalist system such as
ours, however, this approach can carry substantial costs. When a federal
court insists that state and local authorities follow its dictates on a
matter not critical to personal liberty or procedural justice, the
latter may be prevented from engaging in the kind of beneficent
``experimentation in things social and economic'' that ultimately
redounds to the benefit of all Americans. New State Ice Co.~v. Liebmann,
(Brandeis, J., dissenting). The costs of federal courts' imposing a
uniform national standard may be especially high when the relevant
regulatory interests vary significantly across localities, and when the
ruling implicates the States' core police powers.

Furthermore, there is a real risk that, by demanding the provisions of
the Bill of Rights apply identically to the States, federal courts will
cause those provisions to ``be watered down in the needless pursuit of
uniformity.'' Duncan v. Louisiana, n.~21, (Harlan, J., dissenting). When
one legal standard must prevail across dozens of jurisdictions with
disparate needs and customs, courts will often settle on a relaxed
standard. This watering-down risk is particularly acute when we move
beyond the narrow realm of criminal procedure and into the relatively
vast domain of substantive rights. So long as the requirements of
fundamental fairness are always and everywhere respected, it is not
clear that greater liberty results from the jot-for-jot application of a
provision of the Bill of Rights to the States. Indeed, it is far from
clear that proponents of an individual right to keep and bear arms ought
to celebrate today's decision.

So far, I have explained that substantive due process analysis generally
requires us to consider the term ``liberty'' in the Fourteenth
Amendment, and that this inquiry may be informed by but does not depend
upon the content of the Bill of Rights. How should a court go about the
analysis, then? Our precedents have established, not an exact
methodology, but rather a framework for decisionmaking. In this respect,
too, the Court's narrative fails to capture the continuity and
flexibility in our doctrine.

The basic inquiry was described by Justice Cardozo more than 70 years
ago. When confronted with a substantive due process claim, we must ask
whether the allegedly unlawful practice violates values ``implicit in
the concept of ordered liberty.'' Palko v. Connecticut, If the practice
in question lacks any ``oppressive and arbitrary'' character, if
judicial enforcement of the asserted right would not materially
contribute to ``a fair and enlightened system of justice,'' then the
claim is unsuitable for substantive due process protection. 325,
Implicit in Justice Cardozo's test is a recognition that the postulates
of liberty have a universal character. Liberty claims that are
inseparable from the customs that prevail in a certain region, the
idiosyncratic expectations of a certain group, or the personal
preferences of their champions, may be valid claims in some sense; but
they are not of constitutional stature. Whether conceptualized as a
``rational continuum'' of legal precepts, Poe, (Harlan, J., dissenting),
or a seamless web of moral commitments, the rights embraced by the
liberty clause transcend the local and the particular.

Justice Cardozo's test undeniably requires judges to apply their own
reasoned judgment, but that does not mean it involves an exercise in
abstract philosophy. In addition to other constraints I will soon
discuss, see Part III, infra, historical and empirical data of various
kinds ground the analysis. Textual commitments laid down elsewhere in
the Constitution, judicial precedents, English common law, legislative
and social facts, scientific and professional developments, practices of
other civilized societies,15 and, above all else, the ```traditions and
conscience of our people,''' Palko, (quoting Snyder v. Massachusetts, ),
are critical variables. They can provide evidence about which rights
really are vital to ordered liberty, as well as a spur to judicial
action.

The Court errs both in its interpretation of Palko and in its suggestion
that later cases rendered Palko's methodology defunct. Echoing Duncan,
the Court advises that Justice Cardozo's test will not be satisfied
```if a civilized system could be imagined that would not accord the
particular protection.''' (quoting 391 U.S., n.~14, ). Palko does
contain some language that could be read to set an inordinate bar to
substantive due process recognition, reserving it for practices without
which ``neither liberty nor justice would exist.'' 302 U.S., But in view
of Justice Cardozo's broader analysis, as well as the numerous cases
that have upheld liberty claims under the Palko standard, such readings
are plainly overreadings. We have never applied Palko in such a
draconian manner.

Nor, as the Court intimates, see did Duncan mark an irreparable break
from Palko, swapping out liberty for history. Duncan limited its
discussion to ``particular procedural safeguard{[}s{]}'' in the Bill of
Rights relating to ``criminal processes,'' 391 U.S., n.~14, ; it did not
purport to set a standard for other types of liberty interests. Even
with regard to procedural safeguards, Duncan did not jettison the Palko
test so much as refine it: The judge is still tasked with evaluating
whether a practice ``is fundamental\ldots{} to ordered liberty,'' within
the context of the ``Anglo-American'' system. Duncan, n.~14, Several of
our most important recent decisions confirm the proposition that
substantive due process analysis--- from which, once again,
``incorporation'' analysis derives---must not be wholly backward
looking. See, e.g., Lawrence v. Texas, (``{[}H{]}istory and tradition
are the starting point but not in all cases the ending point of the
substantive due process inquiry'' (internal quotation marks omitted));
Michael H. v. Gerald D., n.~6, (garnering only two votes for
history-driven methodology that ``consult{[}s{]} the most specific
tradition available''); see also post (BREYER, J., dissenting)
(explaining that post-Duncan ``incorporation'' cases continued to rely
on more than history).

The Court's flight from Palko leaves its analysis, careful and scholarly
though it is, much too narrow to provide a satisfying answer to this
case. The Court hinges its entire decision on one mode of intellectual
history, culling selected pronouncements and enactments from the 18th
and 19th centuries to ascertain what Americans thought about firearms.
Relying on Duncan and Glucksberg, the plurality suggests that only
interests that have proved ``fundamental from an American perspective,''
or ```deeply rooted in this Nation's history and tradition,''' (quoting
Glucksberg, ), to the Court's satisfaction, may qualify for
incorporation into the Fourteenth Amendment. To the extent the Court's
opinion could be read to imply that the historical pedigree of a right
is the exclusive or dispositive determinant of its status under the Due
Process Clause, the opinion is seriously mistaken.

A rigid historical test is inappropriate in this case, most basically,
because our substantive due process doctrine has never evaluated
substantive rights in purely, or even predominantly, historical terms.
When the Court applied many of the procedural guarantees in the Bill of
Rights to the States in the 1960's, it often asked whether the guarantee
in question was ``fundamental in the context of the criminal processes
maintained by the American States.''17 Duncan, n.~14, That inquiry could
extend back through time, but it was focused not so much on historical
conceptions of the guarantee as on its functional significance within
the States' regimes. This contextualized approach made sense, as the
choice to employ any given trial-type procedure means little in the
abstract. It is only by inquiring into how that procedure intermeshes
with other procedures and practices in a criminal justice system that
its relationship to ``liberty'' and ``due process'' can be determined.

Yet when the Court has used the Due Process Clause to recognize rights
distinct from the trial context---rights relating to the primary conduct
of free individuals--- Justice Cardozo's test has been our guide. The
right to free speech, for instance, has been safeguarded from state
infringement not because the States have always honored it, but because
it is ``essential to free government'' and ``to the maintenance of
democratic institutions''---that is, because the right to free speech is
implicit in the concept of ordered liberty. Thornhill v. Alabama, 96, ;
see also, e.g., Loving, (discussing right to marry person of another
race); Mapp v. Ohio, 655-657, (discussing right to be free from
arbitrary intrusion by police); Schneider v. State (Town of Irvington),
(discussing right to distribute printed matter) While the verbal formula
has varied, the Court has largely been consistent in its liberty-based
approach to substantive interests outside of the adjudicatory system. As
the question before us indisputably concerns such an interest, the
answer cannot be found in a granular inspection of state constitutions
or congressional debates.

More fundamentally, a rigid historical methodology is unfaithful to the
Constitution's command. For if it were really the case that the
Fourteenth Amendment's guarantee of liberty embraces only those rights
``so rooted in our history, tradition, and practice as to require
special protection,'' Glucksberg, n.~17, then the guarantee would serve
little function, save to ratify those rights that state actors have
already been according the most extensive protection Cf. Duncan,
(Harlan, J., dissenting) (critiquing ``circular{[}ity{]}'' of
historicized test for incorporation). That approach is unfaithful to the
expansive principle Americans laid down when they ratified the
Fourteenth Amendment and to the level of generality they chose when they
crafted its language; it promises an objectivity it cannot deliver and
masks the value judgments that pervade any analysis of what customs,
defined in what manner, are sufficiently ```rooted'''; it countenances
the most revolting injustices in the name of continuity,20 for we must
never forget that not only slavery but also the subjugation of women and
other rank forms of discrimination are part of our history; and it
effaces this Court's distinctive role in saying what the law is, leaving
the development and safekeeping of liberty to majoritarian political
processes. It is judicial abdication in the guise of judicial modesty.

Yet while ``the'liberty' specially protected by the Fourteenth
Amendment'' is ``perhaps not capable of being fully clarified,''
Glucksberg, it is capable of being refined and delimited. We have
insisted that only certain types of especially significant personal
interests may qualify for especially heightened protection. Ever since
``the deviant economic due process cases {[}were{]} repudiated,''
(Souter, J., concurring in judgment), our doctrine has steered away from
``laws that touch economic problems, business affairs, or social
conditions,'' Griswold, and has instead centered on ``matters relating
to marriage, procreation, contraception, family relationships, and child
rearing and education,'' Paul v. Davis, These categories are not
exclusive. Government action that shocks the conscience, pointlessly
infringes settled expectations, trespasses into sensitive private realms
or life choices without adequate justification, perpetrates gross
injustice, or simply lacks a rational basis will always be vulnerable to
judicial invalidation. Nor does the fact that an asserted right falls
within one of these categories end the inquiry. More fundamental rights
may receive more robust judicial protection, but the strength of the
individual's liberty interests and the State's regulatory interests must
always be assessed and compared. No right is absolute.

Rather than seek a categorical understanding of the liberty clause, our
precedents have thus elucidated a conceptual core. The clause
safeguards, most basically, ``the ability independently to define one's
identity,'' Roberts v. United States Jaycees, ``the individual's right
to make certain unusually important decisions that will affect his own,
or his family's, destiny,'' Fitzgerald, 523 F d, and the right to be
respected as a human being. Self-determination, bodily integrity,
freedom of conscience, intimate relationships, political equality,
dignity and respect---these are the central values we have found
implicit in the concept of ordered liberty.

Another key constraint on substantive due process analysis is respect
for the democratic process. If a particular liberty interest is already
being given careful consideration in, and subjected to ongoing
calibration by, the States, judicial enforcement may not be appropriate.
When the Court declined to establish a general right to
physician-assisted suicide, for example, it did so in part because ``the
States {[}were{]} currently engaged in serious, thoughtful examinations
of physician-assisted suicide and other similar issues,'' rendering
judicial intervention both less necessary and potentially more
disruptive. Glucksberg, 735, Conversely, we have long appreciated that
more ``searching'' judicial review may be justified when the rights of
``discrete and insular minorities''---groups that may face systematic
barriers in the political system---are at stake. United States v.
Carolene Products Co., n.~4, Courts have a ``comparative \ldots{}
advantage'' over the elected branches on a limited, but significant,
range of legal matters. Post.

Recognizing a new liberty right is a momentous step. It takes that
right, to a considerable extent, ``outside the arena of public debate
and legislative action.'' Glucksberg, Sometimes that momentous step must
be taken; some fundamental aspects of personhood, dignity, and the like
do not vary from State to State, and demand a baseline level of
protection. But sensitivity to the interaction between the intrinsic
aspects of liberty and the practical realities of contemporary society
provides an important tool for guiding judicial discretion.

This sensitivity is an aspect of a deeper principle: the need to
approach our work with humility and caution. Because the relevant
constitutional language is so ``spacious,'' Duncan, I have emphasized
that ``{[}t{]}he doctrine of judicial self-restraint requires us to
exercise the utmost care whenever we are asked to break new ground in
this field.'' Collins, Many of my colleagues and predecessors have
stressed the same point, some with great eloquence. See, e.g., Casey, ;
Moore v. East Cleveland, (plurality opinion); Poe, (Harlan, J.,
dissenting); Adamson v. California, (Frankfurter, J., concurring).
Historical study may discipline as well as enrich the analysis. But the
inescapable reality is that no serious theory of Section 1 of the
Fourteenth Amendment yields clear answers in every case, and ``{[}n{]}o
formula could serve as a substitute, in this area, for judgment and
restraint.'' Poe, (Harlan, J., dissenting).

Several rules of the judicial process help enforce such restraint. In
the substantive due process field as in others, the Court has applied
both the doctrine of stare decisis ---adhering to precedents, respecting
reliance interests, prizing stability and order in the law---and the
common-law method---taking cases and controversies as they present
themselves, proceeding slowly and incrementally, building on what came
before. This restrained methodology was evident even in the heyday of
``incorporation'' during the 1960's. Although it would have been much
easier for the Court simply to declare certain Amendments in the Bill of
Rights applicable to the States in toto, the Court took care to parse
each Amendment into its component guarantees, evaluating them one by
one. This piecemeal approach allowed the Court to scrutinize more
closely the right at issue in any given dispute, reducing both the risk
and the cost of error.

Relatedly, rather than evaluate liberty claims on an abstract plane, the
Court has ``required in substantive-due-process cases a'careful
description' of the asserted fundamental liberty interest.'' Glucksberg,
(quoting Reno v. Flores, ; Collins, ; Cruzan v. Director, Mo. Dept. of
Health, ). And just as we have required such careful description from
the litigants, we have required of ourselves that we ``focus on the
allegations in the complaint to determine how petitioner describes the
constitutional right at stake.'' Collins, ; see also Stevens, Judicial
Restraint, 22 San Diego L.Rev. 437, 446-448 (1985). This does not mean
that we must define the asserted right at the most specific level,
thereby sapping it of a universal valence and a moral force it might
otherwise have It means, simply, that we must pay close attention to the
precise liberty interest the litigants have asked us to vindicate.

Our holdings should be similarly tailored. Even if the most expansive
formulation of a claim does not qualify for substantive due process
recognition, particular components of the claim might. Just because
there may not be a categorical right to physician-assisted suicide, for
example, does not
``\texttt{foreclose\ the\ possibility\ that\ an\ individual\ plaintiff\ seeking\ to\ hasten\ her\ death,\ or\ a\ doctor\ whose\ assistance\ was\ sought,\ could\ prevail\ in\ a\ more\ particularized\ challenge.\textquotesingle{}"\ Glucksberg,\ n.\ 24,\ \ (quoting\ \ (STEVENS,\ J.,\ concurring\ in\ judgments));\ see\ also\ Vacco\ v.\ Quill,\ n.\ 13,\ \ (leaving\ open\ "}the
possibility that some applications of the {[}New York prohibition on
assisted suicide{]} may impose an intolerable intrusion on the patient's
freedom'''). Even if a State's interest in regulating a certain matter
must be permitted, in the general course, to trump the individual's
countervailing liberty interest, there may still be situations in which
the latter ``is entitled to constitutional protection.'' Glucksberg,
(STEVENS, J., concurring in judgments).

As this discussion reflects, to acknowledge that the task of construing
the liberty clause requires judgment is not to say that it is a license
for unbridled judicial lawmaking. To the contrary, only an honest
reckoning with our discretion allows for honest argumentation and
meaningful accountability.

The question in this case, then, is not whether the Second Amendment
right to keep and bear arms (whatever that right's precise contours)
applies to the States because the Amendment has been incorporated into
the Fourteenth Amendment. It has not been. The question, rather, is
whether the particular right asserted by petitioners applies to the
States because of the Fourteenth Amendment itself, standing on its own
bottom. And to answer that question, we need to determine, first, the
nature of the right that has been asserted and, second, whether that
right is an aspect of Fourteenth Amendment ``liberty.'' Even accepting
the Court's holding in Heller, it remains entirely possible that the
right to keep and bear arms identified in that opinion is not judicially
enforceable against the States, or that only part of the right is so
enforceable It is likewise possible for the Court to find in this case
that some part of the Heller right applies to the States, and then to
find in later cases that other parts of the right also apply, or apply
on different terms.

As noted at the outset, the liberty interest petitioners have asserted
is the ``right to possess a functional, personal firearm, including a
handgun, within the home.'' Complaint ¶ 34, App. 23. The city of Chicago
allows residents to keep functional firearms, so long as they are
registered, but it generally prohibits the possession of handguns,
sawed-off shotguns, machine guns, and short-barreled rifles. See
Chicago, Ill., Municipal Code § 8-20-050 (2009) Petitioners' complaint
centered on their desire to keep a handgun at their domicile---it
references the ``home'' in nearly every paragraph. Petitioners now frame
the question that confronts us as ``{[}w{]}hether the Second Amendment
right to keep and bear arms is incorporated as against the States by the
Fourteenth Amendment's Privileges or Immunities or Due Process
Clauses.'' Brief for Petitioners, p.~i. But it is our duty ``to focus on
the allegations in the complaint to determine how petitioner describes
the constitutional right at stake,'' Collins, and the gravamen of this
complaint is plainly an appeal to keep a handgun or other firearm of
one's choosing in the home.

Petitioners' framing of their complaint tracks the Court's ruling in
Heller. The majority opinion contained some dicta suggesting the
possibility of a more expansive arms-bearing right, one that would
travel with the individual to an extent into public places, as ``in case
of confrontation.'' But the Heller plaintiff sought only dispensation to
keep an operable firearm in his home for lawful self-defense, see
\textbf{\emph{, and n.~2), and the Court's opinion was bookended by
reminders that its holding was limited to that one issue, }}, \_\_\_,
2821-2822; accord, (plurality opinion). The distinction between the
liberty right these petitioners have asserted and the Second Amendment
right identified in Heller is therefore evanescent. Both are rooted to
the home. Moreover, even if both rights have the logical potential to
extend further, upon ``future evaluation,'' Heller, it is incumbent upon
us, as federal judges contemplating a novel rule that would bind all 50
States, to proceed cautiously and to decide only what must be decided.

While I agree with the Court that our substantive due process cases
offer a principled basis for holding that petitioners have a
constitutional right to possess a usable firearm in the home, I am
ultimately persuaded that a better reading of our case law supports the
city of Chicago. I would not foreclose the possibility that a particular
plaintiff---say, an elderly widow who lives in a dangerous neighborhood
and does not have the strength to operate a long gun---may have a
cognizable liberty interest in possessing a handgun. But I cannot accept
petitioners' broader submission. A number of factors, taken together,
lead me to this conclusion.

First, firearms have a fundamentally ambivalent relationship to liberty.
Just as they can help homeowners defend their families and property from
intruders, they can help thugs and insurrectionists murder innocent
victims. The threat that firearms will be misused is far from
hypothetical, for gun crime has devastated many of our communities.
Amici calculate that approximately one million Americans have been
wounded or killed by gunfire in the last decade Urban areas such as
Chicago suffer disproportionately from this epidemic of violence.
Handguns contribute disproportionately to it. Just as some homeowners
may prefer handguns because of their small size, light weight, and ease
of operation, some criminals will value them for the same reasons. See
Heller, 128 S.Ct. (BREYER, J., dissenting). In recent years, handguns
were reportedly used in more than four-fifths of firearm murders and
more than half of all murders nationwide.

Hence, in evaluating an asserted right to be free from particular
gun-control regulations, liberty is on both sides of the equation. Guns
may be useful for self-defense, as well as for hunting and sport, but
they also have a unique potential to facilitate death and destruction
and thereby to destabilize ordered liberty. Your interest in keeping and
bearing a certain firearm may diminish my interest in being and feeling
safe from armed violence. And while granting you the right to own a
handgun might make you safer on any given day--- assuming the handgun's
marginal contribution to self-defense outweighs its marginal
contribution to the risk of accident, suicide, and criminal
mischief---it may make you and the community you live in less safe
overall, owing to the increased number of handguns in circulation. It is
at least reasonable for a democratically elected legislature to take
such concerns into account in considering what sorts of regulations
would best serve the public welfare.

The practical impact of various gun-control measures may be highly
controversial, but this basic insight should not be. The idea that
deadly weapons pose a distinctive threat to the social order---and that
reasonable restrictions on their usage therefore impose an acceptable
burden on one's personal liberty---is as old as the Republic. As THE
CHIEF JUSTICE observed just the other day, it is a foundational premise
of modern government that the State holds a monopoly on legitimate
violence: ``A basic step in organizing a civilized society is to take
{[}the{]} sword out of private hands and turn it over to an organized
government, acting on behalf of all the people.'' Robertson v. United
States ex rel. Watson (dissenting opinion). The same holds true for the
handgun. The power a man has in the state of nature ``of doing
whatsoever he thought fit for the preservation of himself and the rest
of mankind, he gives up,'' to a significant extent, ``to be regulated by
laws made by the society.'' J. Locke, Second Treatise of Civil
Government § 129, p.~64 (J. Gough ed ).

Limiting the federal constitutional right to keep and bear arms to the
home complicates the analysis but does not dislodge this conclusion.
Even though the Court has long afforded special solicitude for the
privacy of the home, we have never understood that principle to
``infring{[}e{]} upon'' the authority of the States to proscribe certain
inherently dangerous items, for ``{[}i{]}n such cases, compelling
reasons may exist for overriding the right of the individual to possess
those materials.'' Stanley, n.~11, And, of course, guns that start out
in the home may not stay in the home. Even if the government has a
weaker basis for restricting domestic possession of firearms as compared
to public carriage---and even if a blanket, statewide prohibition on
domestic possession might therefore be unconstitutional---the line
between the two is a porous one. A state or local legislature may
determine that a prophylactic ban on an especially portable weapon is
necessary to police that line.

Second, the right to possess a firearm of one's choosing is different in
kind from the liberty interests we have recognized under the Due Process
Clause. Despite the plethora of substantive due process cases that have
been decided in the post-Lochner century, I have found none that holds,
states, or even suggests that the term ``liberty'' encompasses either
the common-law right of self-defense or a right to keep and bear arms. I
do not doubt for a moment that many Americans feel deeply passionate
about firearms, and see them as critical to their way of life as well as
to their security. Nevertheless, it does not appear to be the case that
the ability to own a handgun, or any particular type of firearm, is
critical to leading a life of autonomy, dignity, or political equality:
The marketplace offers many tools for self-defense, even if they are
imperfect substitutes, and neither petitioners nor their amici make such
a contention. Petitioners' claim is not the kind of substantive
interest, accordingly, on which a uniform, judicially enforced national
standard is presumptively appropriate.

Indeed, in some respects the substantive right at issue may be better
viewed as a property right. Petitioners wish to acquire certain types of
firearms, or to keep certain firearms they have previously acquired.
Interests in the possession of chattels have traditionally been viewed
as property interests subject to definition and regulation by the
States. Cf. Stop the Beach Renourishment, Inc.~v. Florida Dept. of
Environmental Protection, 560 U.S. \textbf{\emph{, }}, \_\_\_ S.Ct.
\textbf{\emph{, }} L.Ed d \_\_\_ (2010) (opinion of SCALIA, J.)
(``Generally speaking, state law defines property interests''). Under
that tradition, Chicago's ordinance is unexceptional.

The liberty interest asserted by petitioners is also dissimilar from
those we have recognized in its capacity to undermine the security of
others. To be sure, some of the Bill of Rights' procedural guarantees
may place ``restrictions on law enforcement'' that have ``controversial
public safety implications.'' (plurality opinion); see also (opinion of
SCALIA, J.). But those implications are generally quite attenuated. A
defendant's invocation of his right to remain silent, to confront a
witness, or to exclude certain evidence cannot directly cause any
threat. The defendant's liberty interest is constrained by (and is
itself a constraint on) the adjudicatory process. The link between
handgun ownership and public safety is much tighter. The handgun is
itself a tool for crime; the handgun's bullets are the violence.

Similarly, it is undeniable that some may take profound offense at a
remark made by the soapbox speaker, the practices of another religion,
or a gay couple's choice to have intimate relations. But that offense is
moral, psychological, or theological in nature; the actions taken by the
rights-bearers do not actually threaten the physical safety of any other
person Firearms may be used to kill another person. If a legislature's
response to dangerous weapons ends up impinging upon the liberty of any
individuals in pursuit of the greater good, it invariably does so on the
basis of more than the majority's ```own moral code,''' Lawrence,
(quoting Casey, ). While specific policies may of course be misguided,
gun control is an area in which it ``is quite wrong \ldots{} to assume
that regulation and liberty occupy mutually exclusive zones---that as
one expands, the other must contract.'' Stevens, 41 U. Miami L.Rev..

Third, the experience of other advanced democracies, including those
that share our British heritage, undercuts the notion that an expansive
right to keep and bear arms is intrinsic to ordered liberty. Many of
these countries place restrictions on the possession, use, and carriage
of firearms far more onerous than the restrictions found in this Nation.
See Municipal Respondents' Brief 21-23 (discussing laws of England,
Canada, Australia, Japan, Denmark, Finland, Luxembourg, and New
Zealand). That the United States is an international outlier in the
permissiveness of its approach to guns does not suggest that our laws
are bad laws. It does suggest that this Court may not need to assume
responsibility for making our laws still more permissive.

Admittedly, these other countries differ from ours in many relevant
respects, including their problems with violent crime and the
traditional role that firearms have played in their societies. But they
are not so different from the United States that we ought to dismiss
their experience entirely. Cf. (plurality opinion); (opinion of SCALIA,
J.). The fact that our oldest allies have almost uniformly found it
appropriate to regulate firearms extensively tends to weaken
petitioners' submission that the right to possess a gun of one's
choosing is fundamental to a life of liberty. While the ``American
perspective'' must always be our focus, (plurality opinion), it is
silly---indeed, arrogant---to think we have nothing to learn about
liberty from the billions of people beyond our borders.

Fourth, the Second Amendment differs in kind from the Amendments that
surround it, with the consequence that its inclusion in the Bill of
Rights is not merely unhelpful but positively harmful to petitioners'
claim. Generally, the inclusion of a liberty interest in the Bill of
Rights points toward the conclusion that it is of fundamental
significance and ought to be enforceable against the States. But the
Second Amendment plays a peculiar role within the Bill, as announced by
its peculiar opening clause Even accepting the Heller Court's view that
the Amendment protects an individual right to keep and bear arms
disconnected from militia service, it remains undeniable that ``the
purpose for which the right was codified'' was ``to prevent elimination
of the militia.'' Heller, 128 S.Ct.; see also United States v. Miller,
(Second Amendment was enacted ``{[}w{]}ith obvious purpose to assure the
continuation and render possible the effectiveness of {[}militia{]}
forces''). It was the States, not private persons, on whose immediate
behalf the Second Amendment was adopted. Notwithstanding the Heller
Court's efforts to write the Second Amendment's preamble out of the
Constitution, the Amendment still serves the structural function of
protecting the States from encroachment by an overreaching Federal
Government.

Fifth, although it may be true that Americans' interest in firearm
possession and state-law recognition of that interest are ``deeply
rooted'' in some important senses, (internal quotation marks omitted),
it is equally true that the States have a long and unbroken history of
regulating firearms. The idea that States may place substantial
restrictions on the right to keep and bear arms short of complete
disarmament is, in fact, far more entrenched than the notion that the
Federal Constitution protects any such right. Federalism is a far
``older and more deeply rooted tradition than is a right to carry,'' or
to own, ``any particular kind of weapon.'' C.A 2009) (Easterbrook, C.
J.).

The preceding sections have already addressed many of the points made by
Justice SCALIA in his concurrence. But in light of that opinion's
fixation on this one, it is appropriate to say a few words about Justice
SCALIA's broader claim: that his preferred method of substantive due
process analysis, a method ``that makes the traditions of our people
paramount,'' is both more restrained and more facilitative of democracy
than the method I have outlined. Colorful as it is, Justice SCALIA's
critique does not have nearly as much force as does his rhetoric. His
theory of substantive due process, moreover, comes with its own profound
difficulties.

Although Justice SCALIA aspires to an ``objective,'' ``neutral'' method
of substantive due process analysis, his actual method is nothing of the
sort. Under the ``historically focused'' approach he advocates, numerous
threshold questions arise before one ever gets to the history. At what
level of generality should one frame the liberty interest in question?
See n.~25. What does it mean for a right to be ```deeply rooted in this
Nation's history and tradition,''' (quoting Glucksberg, )? By what
standard will that proposition be tested? Which types of sources will
count, and how will those sources be weighed and aggregated? There is no
objective, neutral answer to these questions. There is not even a
theory---at least, Justice SCALIA provides none---of how to go about
answering them.

Nor is there any escaping Palko, it seems. To qualify for substantive
due process protection, Justice SCALIA has stated, an asserted liberty
right must be not only deeply rooted in American tradition, ``but it
must also be implicit in the concept of ordered liberty.'' Lawrence,
n.~3, (dissenting opinion) (internal quotation marks omitted). Applying
the latter, Palko-derived half of that test requires precisely the sort
of reasoned judgment---the same multifaceted evaluation of the right's
contours and consequences---that Justice SCALIA mocks in his concurrence
today.

So does applying the first half. It is hardly a novel insight that
history is not an objective science, and that its use can therefore
``point in any direction the judges favor,'' (opinion of SCALIA, J.).
Yet 21 years after the point was brought to his attention by Justice
Brennan, Justice SCALIA remains ``oblivious to the fact that {[}the
concept of'tradition'{]} can be as malleable and elusive as'liberty'
itself.'' Michael H., (dissenting opinion). Even when historical
analysis is focused on a discrete proposition, such as the original
public meaning of the Second Amendment, the evidence often points in
different directions. The historian must choose which pieces to credit
and which to discount, and then must try to assemble them into a
coherent whole. In Heller, Justice SCALIA preferred to rely on sources
created much earlier and later in time than the Second Amendment itself,
see, e.g., 128 S.Ct. (consulting late 19th-century treatises to
ascertain how Americans would have read the Amendment's preamble in
1791); I focused more closely on sources contemporaneous with the
Amendment's drafting and ratification No mechanical yardstick can
measure which of us was correct, either with respect to the materials we
chose to privilege or the insights we gleaned from them.

The malleability and elusiveness of history increase exponentially when
we move from a pure question of original meaning, as in Heller, to
Justice SCALIA's theory of substantive due process. At least with the
former sort of question, the judge can focus on a single legal
provision; the temporal scope of the inquiry is (or should be)
relatively bounded; and there is substantial agreement on what sorts of
authorities merit consideration. With Justice SCALIA's approach to
substantive due process, these guideposts all fall away. The judge must
canvas the entire landscape of American law as it has evolved through
time, and perhaps older laws as well, see, e.g., Lawrence, (SCALIA, J.,
dissenting) (discussing ```ancient roots''' of proscriptions against
sodomy (quoting Bowers v. Hardwick, ), pursuant to a standard (deeply
rootedness) that has never been defined. In conducting this rudderless,
panoramic tour of American legal history, the judge has more than ample
opportunity to ``look over the heads of the crowd and pick out {[}his{]}
friends,'' Roper v. Simmons, 543 U.S. 551, 617, (SCALIA, J.,
dissenting).

My point is not to criticize judges' use of history in general or to
suggest that it always generates indeterminate answers; I have already
emphasized that historical study can discipline as well as enrich
substantive due process analysis. My point is simply that Justice
SCALIA's defense of his method, which holds out objectivity and
restraint as its cardinal---and, it seems, only---virtues, is
unsatisfying on its own terms. For a limitless number of subjective
judgments may be smuggled into his historical analysis. Worse, they may
be buried in the analysis. At least with my approach, the judge's cards
are laid on the table for all to see, and to critique. The judge must
exercise judgment, to be sure. When answering a constitutional question
to which the text provides no clear answer, there is always some amount
of discretion; our constitutional system has always depended on judges'
filling in the document's vast open spaces But there is also
transparency.

Justice SCALIA's approach is even less restrained in another sense: It
would effect a major break from our case law outside of the
``incorporation'' area. Justice SCALIA does not seem troubled by the
fact that his method is largely inconsistent with the Court's canonical
substantive due process decisions, ranging from Meyerand Piercein the
1920's, to Griswoldin the 1960's, to Lawrencein the 2000's. To the
contrary, he seems to embrace this dissonance. My method seeks to
synthesize dozens of cases on which the American people have relied for
decades. Justice SCALIA's method seeks to vaporize them. So I am left to
wonder, which of us is more faithful to this Nation's constitutional
history? And which of us is more faithful to the values and commitments
of the American people, as they stand today? In 1967, when the Court
held in Lovingthat adults have a liberty-based as well as equality-based
right to wed persons of another race, interracial marriage was hardly
``deeply rooted'' in American tradition. Racial segregation and
subordination were deeply rooted. The Court's substantive due process
holding was nonetheless correct---and we should be wary of any
interpretive theory that implies, emphatically, that it was not.

Which leads me to the final set of points I wish to make: Justice
SCALIA's method invites not only bad history, but also bad
constitutional law. As I have already explained, in evaluating a claimed
liberty interest (or any constitutional claim for that matter), it makes
perfect sense to give history significant weight: Justice SCALIA's
position is closer to my own than he apparently feels comfortable
acknowledging. But it makes little sense to give history dispositive
weight in every case. And it makes especially little sense to answer
questions like whether the right to bear arms is ``fundamental'' by
focusing only on the past, given that both the practical significance
and the public understandings of such a right often change as society
changes. What if the evidence had shown that, whereas at one time
firearm possession contributed substantially to personal liberty and
safety, nowadays it contributes nothing, or even tends to undermine
them? Would it still have been reasonable to constitutionalize the
right?

The concern runs still deeper. Not only can historical views be less
than completely clear or informative, but they can also be wrong. Some
notions that many Americans deeply believed to be true, at one time,
turned out not to be true. Some practices that many Americans believed
to be consistent with the Constitution's guarantees of liberty and
equality, at one time, turned out to be inconsistent with them. The fact
that we have a written Constitution does not consign this Nation to a
static legal existence. Although we should always ``pa{[}y{]} a decent
regard to the opinions of former times,'' it ``is not the glory of the
people of America'' to have ``suffered a blind veneration for
antiquity.'' The Federalist No.~14, p.~99, 104 (C. Rossiter ed.~1961)
(J. Madison). It is not the role of federal judges to be amateur
historians. And it is not fidelity to the Constitution to ignore its use
of deliberately capacious language, in an effort to transform
foundational legal commitments into narrow rules of decision.

As for ``the democratic process,'' a method that looks exclusively to
history can easily do more harm than good. Just consider this case. The
net result of Justice SCALIA's supposedly objective analysis is to vest
federal judges--- ultimately a majority of the judges on this
Court---with unprecedented lawmaking powers in an area in which they
have no special qualifications, and in which the give-and-take of the
political process has functioned effectively for decades. Why this
``intrudes much less upon the democratic process,'' than an approach
that would defer to the democratic process on the regulation of firearms
is, to say the least, not self-evident. I cannot even tell what, under
Justice SCALIA's view, constitutes an ``intrusion.''

It is worth pondering, furthermore, the vision of democracy that
underlies Justice SCALIA's critique. Because very few of us would
welcome a system in which majorities or powerful interest groups always
get their way. Under our constitutional scheme, I would have thought
that a judicial approach to liberty claims such as the one I have
outlined---an approach that investigates both the intrinsic nature of
the claimed interest and the practical significance of its judicial
enforcement, that is transparent in its reasoning and sincere in its
effort to incorporate constraints, that is guided by history but not
beholden to it, and that is willing to protect some rights even if they
have not already received uniform protection from the elected
branches---has the capacity to improve, rather than ``{[}im{]}peril,''
our democracy. It all depends on judges' exercising careful, reasoned
judgment. As it always has, and as it always will.

\textbf{Justice BREYER, with whom Justice GINSBURG and Justice SOTOMAYOR
join, dissenting.} In my view, Justice STEVENS has demonstrated that the
Fourteenth Amendment's guarantee of ``substantive due process'' does not
include a general right to keep and bear firearms for purposes of
private self-defense. As he argues, the Framers did not write the Second
Amendment with this objective in view. See (dissenting opinion). Unlike
other forms of substantive liberty, the carrying of arms for that
purpose often puts others' lives at risk. See And the use of arms for
private self-defense does not warrant federal constitutional protection
from state regulation. See The Court, however, does not expressly rest
its opinion upon ``substantive due process'' concerns. Rather, it
directs its attention to this Court's ``incorporation'' precedents and
asks whether the Second Amendment right to private self-defense is
``fundamental'' so that it applies to the States through the Fourteenth
Amendment. See I shall therefore separately consider the question of
``incorporation.'' I can find nothing in the Second Amendment's text,
history, or underlying rationale that could warrant characterizing it as
``fundamental'' insofar as it seeks to protect the keeping and bearing
of arms for private self-defense purposes. Nor can I find any
justification for interpreting the Constitution as transferring ultimate
regulatory authority over the private uses of firearms from
democratically elected legislatures to courts or from the States to the
Federal Government. I therefore conclude that the Fourteenth Amendment
does not ``incorporate'' the Second Amendment's right ``to keep and bear
Arms.'' And I consequently dissent.

The Second Amendment says: ``A well regulated Militia, being necessary
to the security of a free State, the right of the people to keep and
bear Arms, shall not be infringed.'' Two years ago, in District of
Columbia v. Heller, 554 U.S. \textbf{\emph{, the Court rejected the
pre-existing judicial consensus that the Second Amendment was primarily
concerned with the need to maintain a ``well regulated Militia.''
Although the Court acknowledged that ``the threat that the new Federal
Government would destroy the citizens' militia by taking away their arms
was the reason that right \ldots{} was codified in a written
Constitution,'' the Court asserted that ``individual self defense
\ldots{} was the central component of the right itself.'' The Court went
on to hold that the Second Amendment restricted Congress' power to
regulate handguns used for self-defense, and the Court found
unconstitutional the District of Columbia's ban on the possession of
handguns in the home. }},

The Court based its conclusions almost exclusively upon its reading of
history. But the relevant history in Heller was far from clear: Four
dissenting Justices disagreed with the majority's historical analysis.
And subsequent scholarly writing reveals why disputed history provides
treacherous ground on which to build decisions written by judges who are
not expert at history.

In my view, taking Heller as a given, the Fourteenth Amendment does not
incorporate the Second Amendment right to keep and bear arms for
purposes of private self-defense. Under this Court's precedents, to
incorporate the private self-defense right the majority must show that
the right is, e.g., ``fundamental to the American scheme of justice,''
Duncan v. Louisiana. And this it fails to do.

The majority here, like that in Heller, relies almost exclusively upon
history to make the necessary showing. But to do so for incorporation
purposes is both wrong and dangerous. As Justice STEVENS points out, our
society has historically made mistakes--- for example, when considering
certain 18th- and 19th-century property rights to be fundamental. And in
the incorporation context, as elsewhere, history often is unclear about
the answers.

Accordingly, this Court, in considering an incorporation question, has
never stated that the historical status of a right is the only relevant
consideration. Rather, the Court has either explicitly or implicitly
made clear in its opinions that the right in question has remained
fundamental over time. See, e.g., Apodaca v. Oregon, (plurality opinion)
(stating that the incorporation ``inquiry must focus upon the function
served'' by the right in question in ``contemporary society'' (emphasis
added)); Duncan v. Louisiana, (noting that the right in question
``continues to receive strong support''); Klopfer v. North Carolina,
(same). And, indeed, neither of the parties before us in this case has
asked us to employ the majority's history-constrained approach. See
Brief for Petitioners 67-69 (arguing for incorporation based on trends
in contemporary support for the right); Brief for Respondents City of
Chicago et al.~23-31 (hereinafter Municipal Respondents) (looking to
current state practices with respect to the right).

I thus think it proper, above all where history provides no clear
answer, to look to other factors in considering whether a right is
sufficiently ``fundamental'' to remove it from the political process in
every State. I would include among those factors the nature of the
right; any contemporary disagreement about whether the right is
fundamental; the extent to which incorporation will further other,
perhaps more basic, constitutional aims; and the extent to which
incorporation will advance or hinder the Constitution's structural aims,
including its division of powers among different governmental
institutions (and the people as well). Is incorporation needed, for
example, to further the Constitution's effort to ensure that the
government treats each individual with equal respect? Will it help
maintain the democratic form of government that the Constitution
foresees? In a word, will incorporation prove consistent, or
inconsistent, with the Constitution's efforts to create governmental
institutions well suited to the carrying out of its constitutional
promises?

Finally, I would take account of the Framers' basic reason for believing
the Court ought to have the power of judicial review. Alexander Hamilton
feared granting that power to Congress alone, for he feared that
Congress, acting as judges, would not overturn as unconstitutional a
popular statute that it had recently enacted, as legislators. The
Federalist No.~78, p.~405 (G. Carey \& J. McClellan eds.~2001)
(A.Hamilton) (``This independence of the judges is equally requisite to
guard the constitution and the rights of individuals from the effects of
those ill humours, which'' can, at times, lead to ``serious oppressions
of the minor part in the community''). Judges, he thought, may find it
easier to resist popular pressure to suppress the basic rights of an
unpopular minority. See United States v. Carolene Products Co., n.~4,
That being so, it makes sense to ask whether that particular comparative
judicial advantage is relevant to the case at hand. See, e.g., J. Ely,
Democracy and Distrust (1980).

How do these considerations apply here? For one thing, I would apply
them only to the private self-defense right directly at issue. After
all, the Amendment's militia-related purpose is primarily to protect
States from federal regulation, not to protect individuals from
militia-related regulation. Heller, 128 S.Ct.; see also Miller,
Moreover, the Civil War Amendments, the electoral process, the courts,
and numerous other institutions today help to safeguard the States and
the people from any serious threat of federal tyranny. How are state
militias additionally necessary? It is difficult to see how a right
that, as the majority concedes, has ``largely faded as a popular
concern'' could possibly be so fundamental that it would warrant
incorporation through the Fourteenth Amendment. Hence, the incorporation
of the Second Amendment cannot be based on the militia-related aspect of
what Heller found to be more extensive Second Amendment rights.

For another thing, as Heller concedes, the private self-defense right
that the Court would incorporate has nothing to do with ``the reason''
the Framers ``codified'' the right to keep and bear arms ``in a written
Constitution.'' 128 S.Ct. (emphasis added). Heller immediately adds that
the self-defense right was nonetheless ``the central component of the
right.'' In my view, this is the historical equivalent of a claim that
water runs uphill. See Part I. But, taking it as valid, the Framers'
basic reasons for including language in the Constitution would
nonetheless seem more pertinent (in deciding about the contemporary
importance of a right) than the particular scope 17th- or 18th-century
listeners would have then assigned to the words they used. And
examination of the Framers' motivation tells us they did not think the
private armed self-defense right was of paramount importance. See Amar,
The Bill of Rights as a Constitution, 100 Yale L.J. 1131, 1164 (1991)
(``{[}T{]}o see the {[}Second{]} Amendment as primarily concerned with
an individual right to hunt, or protect one's home,'' would be ``like
viewing the heart of the speech and assembly clauses as the right of
persons to meet to play bridge''); see also, e.g., Rakove, The Second
Amendment: The Highest Stage of Originalism, 76 Chi.-Kent L.Rev. 103,
127-128 (2000); Brief for Historians on Early American Legal,
Constitutional, and Pennsylvania History as Amici Curiae 22-33.

Further, there is no popular consensus that the private self-defense
right described in Heller is fundamental. The plurality suggests that
two amici briefs filed in the case show such a consensus, see but, of
course, numerous amici briefs have been filed opposing incorporation as
well. Moreover, every State regulates firearms extensively, and public
opinion is sharply divided on the appropriate level of regulation. Much
of this disagreement rests upon empirical considerations. One side
believes the right essential to protect the lives of those attacked in
the home; the other side believes it essential to regulate the right in
order to protect the lives of others attacked with guns. It seems
unlikely that definitive evidence will develop one way or the other. And
the appropriate level of firearm regulation has thus long been, and
continues to be, a hotly contested matter of political debate. See,
e.g., Siegel, Dead or Alive: Originalism as Popular Constitutionalism in
Heller, 122 Harv. L.Rev. 191, 201-246 (2008). (Numerous sources
supporting arguments and data in Part II-B can be found in the Appendix,
infra.)

Moreover, there is no reason here to believe that incorporation of the
private self-defense right will further any other or broader
constitutional objective. We are aware of no argument that gun-control
regulations target or are passed with the purpose of targeting
``discrete and insular minorities.'' Carolene Products Co., n.~4. Nor
will incorporation help to assure equal respect for individuals. Unlike
the First Amendment's rights of free speech, free press, assembly, and
petition, the private self-defense right does not comprise a necessary
part of the democratic process that the Constitution seeks to establish.
See, e.g., Whitney v. California, (Brandeis, J., concurring). Unlike the
First Amendment's religious protections, the Fourth Amendment's
protection against unreasonable searches and seizures, the Fifth and
Sixth Amendments' insistence upon fair criminal procedure, and the
Eighth Amendment's protection against cruel and unusual punishments, the
private self-defense right does not significantly seek to protect
individuals who might otherwise suffer unfair or inhumane treatment at
the hands of a majority. Unlike the protections offered by many of these
same Amendments, it does not involve matters as to which judges possess
a comparative expertise, by virtue of their close familiarity with the
justice system and its operation. And, unlike the Fifth Amendment's
insistence on just compensation, it does not involve a matter where a
majority might unfairly seize for itself property belonging to a
minority.

Finally, incorporation of the right will work a significant disruption
in the constitutional allocation of decisionmaking authority, thereby
interfering with the Constitution's ability to further its objectives.

First, on any reasonable accounting, the incorporation of the right
recognized in Heller would amount to a significant incursion on a
traditional and important area of state concern, altering the
constitutional relationship between the States and the Federal
Government. Private gun regulation is the quintessential exercise of a
State's ``police power''---i.e., the power to ``protec{[}t{]} \ldots{}
the lives, limbs, health, comfort, and quiet of all persons, and the
protection of all property within the State,'' by enacting ``all kinds
of restraints and burdens'' on both ``persons and property.''
Slaughter-House Cases, 16 Wall. 36, 62, (internal quotation marks
omitted). The Court has long recognized that the Constitution grants the
States special authority to enact laws pursuant to this power. See,
e.g., Medtronic, Inc.~v. Lohr, (noting that States have ``great
latitude'' to use their police powers (internal quotation marks
omitted)); Metropolitan Life Ins. Co.~v. Massachusetts, A decade ago, we
wrote that there is ``no better example of the police power'' than ``the
suppression of violent crime.'' United States v. Morrison, And examples
in which the Court has deferred to state legislative judgments in
respect to the exercise of the police power are legion. See, e.g.,
Gonzales v. Oregon, (assisted suicide); Washington v. Glucksberg,
(same); Berman v. Parker, (``We deal, in other words, with what
traditionally has been known as the police power. An attempt to define
its reach or trace its outer limits is fruitless\ldots{}'').

Second, determining the constitutionality of a particular state gun law
requires finding answers to complex empirically based questions of a
kind that legislatures are better able than courts to make. See, e.g.,
Los Angeles v. Alameda Books, Inc., (plurality opinion); Turner
Broadcasting System, Inc.~v. FCC, And it may require this kind of
analysis in virtually every case.

Government regulation of the right to bear arms normally embodies a
judgment that the regulation will help save lives. The determination
whether a gun regulation is constitutional would thus almost always
require the weighing of the constitutional right to bear arms against
the ``primary concern of every government---a concern for the safety and
indeed the lives of its citizens.'' United States v. Salerno, With
respect to other incorporated rights, this sort of inquiry is sometimes
present. See, e.g., Brandenburg v. Ohio, (per curiam) (free speech);
Sherbert v. Verner, (religion); Brigham City v. Stuart, (Fourth
Amendment); New York v. Quarles, (Fifth Amendment); Salerno, (bail). But
here, this inquiry---calling for the fine tuning of protective
rules---is likely to be part of a daily judicial diet.

Given the competing interests, courts will have to try to answer
empirical questions of a particularly difficult kind. Suppose, for
example, that after a gun regulation's adoption the murder rate went up.
Without the gun regulation would the murder rate have risen even faster?
How is this conclusion affected by the local recession which has left
numerous people unemployed? What about budget cuts that led to a
downsizing of the police force? How effective was that police force to
begin with? And did the regulation simply take guns from those who use
them for lawful purposes without affecting their possession by
criminals?

Consider too that countless gun regulations of many shapes and sizes are
in place in every State and in many local communities. Does the right to
possess weapons for self-defense extend outside the home? To the car? To
work? What sort of guns are necessary for self-defense? Handguns?
Rifles? Semiautomatic weapons? When is a gun semi-automatic? Where are
different kinds of weapons likely needed? Does time-of-day matter? Does
the presence of a child in the house matter? Does the presence of a
convicted felon in the house matter? Do police need special rules
permitting patdowns designed to find guns? When do registration
requirements become severe to the point that they amount to an
unconstitutional ban? Who can possess guns and of what kind? Aliens?
Prior drug offenders? Prior alcohol abusers? How would the right
interact with a state or local government's ability to take special
measures during, say, national security emergencies? As the questions
suggest, state and local gun regulation can become highly complex, and
these ``are only a few uncertainties that quickly come to mind.''

The difficulty of finding answers to these questions is exceeded only by
the importance of doing so. Firearms cause well over 60,000 deaths and
injuries in the United States each year. Those who live in urban areas,
police officers, women, and children, all may be particularly at risk.
And gun regulation may save their lives. Some experts have calculated,
for example, that Chicago's handgun ban has saved several hundred lives,
perhaps close to 1,000, since it was enacted in 1983. Other experts
argue that stringent gun regulations ``can help protect police officers
operating on the front lines against gun violence,'' have reduced
homicide rates in Washington, D. C., and Baltimore, and have helped to
lower New York's crime and homicide rates.

At the same time, the opponents of regulation cast doubt on these
studies. And who is right? Finding out may require interpreting studies
that are only indirectly related to a particular regulatory statute, say
one banning handguns in the home. Suppose studies find more accidents
and suicides where there is a handgun in the home than where there is a
long gun in the home or no gun at all? To what extent do such studies
justify a ban? What if opponents of the ban put forth counter studies?

In answering such questions judges cannot simply refer to judicial
homilies, such as Blackstone's 18th-century perception that a man's home
is his castle. See 4 Blackstone 223. Nor can the plurality so simply
reject, by mere assertion, the fact that ``incorporation will require
judges to assess the costs and benefits of firearms restrictions.'' How
can the Court assess the strength of the government's regulatory
interests without addressing issues of empirical fact? How can the Court
determine if a regulation is appropriately tailored without considering
its impact? And how can the Court determine if there are less
restrictive alternatives without considering what will happen if those
alternatives are implemented?

Perhaps the Court could lessen the difficulty of the mission it has
created for itself by adopting a jurisprudential approach similar to the
many state courts that administer a state constitutional right to bear
arms. See infra (describing state approaches). But the Court has not yet
done so. Cf. Heller, 128 S.Ct. (rejecting an ```interest-balancing'
approach'' similar to that employed by the States); (plurality opinion).
Rather, the Court has haphazardly created a few simple rules, such as
that it will not touch ``prohibitions on the possession of firearms by
felons and the mentally ill,'' ``laws forbidding the carrying of
firearms in sensitive places such as schools and government buildings,''
or ``laws imposing conditions and qualifications on the commercial sale
of arms.'' Heller, 128 S.Ct.; (plurality opinion). But why these rules
and not others? Does the Court know that these regulations are justified
by some special gun-related risk of death? In fact, the Court does not
know. It has simply invented rules that sound sensible without being
able to explain why or how Chicago's handgun ban is different.

The fact is that judges do not know the answers to the kinds of
empirically based questions that will often determine the need for
particular forms of gun regulation. Nor do they have readily available
``tools'' for finding and evaluating the technical material submitted by
others. District Attorney's Office for Third Judicial Dist. v. Osborne,
557 U.S. \textbf{\emph{, }}, ; see also Turner Broadcasting, Judges
cannot easily make empirically based predictions; they have no way to
gather and evaluate the data required to see if such predictions are
accurate; and the nature of litigation and concerns about stare decisis
further make it difficult for judges to change course if predictions
prove inaccurate. Nor can judges rely upon local community views and
values when reaching judgments in circumstances where prediction is
difficult because the basic facts are unclear or unknown.

At the same time, there is no institutional need to send judges off on
this ``mission-almost-impossible.'' Legislators are able to ``amass the
stuff of actual experience and cull conclusions from it.'' United States
v. Gainey, They are far better suited than judges to uncover facts and
to understand their relevance. And legislators, unlike Article

judges, can be held democratically responsible for their empirically
based and value-laden conclusions. We have thus repeatedly affirmed our
preference for ``legislative not judicial solutions'' to this kind of
problem, see, e.g., Patsy v. Board of Regents of Fla., just as we have
repeatedly affirmed the Constitution's preference for democratic
solutions legislated by those whom the people elect.

In New State Ice Co.~v. Liebmann, Justice Brandeis stated in dissent:

``Some people assert that our present plight is due, in part, to the
limitations set by courts upon experimentation in the fields of social
and economic science; and to the discouragement to which proposals for
betterment there have been subjected otherwise. There must be power in
the States and the Nation to remould, through experimentation, our
economic practices and institutions to meet changing social and economic
needs. I cannot believe that the framers of the Fourteenth Amendment, or
the States which ratified it, intended to deprive us of the power to
correct {[}the social problems we face{]}.''

There are 50 state legislatures. The fact that this Court may already
have refused to take this wise advice with respect to Congress in Heller
is no reason to make matters worse here.

Third, the ability of States to reflect local preferences and
conditions---both key virtues of federalism---here has particular
importance. The incidence of gun ownership varies substantially as
between crowded cities and uncongested rural communities, as well as
among the different geographic regions of the country. Thus,
approximately 60\% of adults who live in the relatively sparsely
populated Western States of Alaska, Montana, and Wyoming report that
their household keeps a gun, while fewer than 15\% of adults in the
densely populated Eastern States of Rhode Island, New Jersey, and
Massachusetts say the same.

The nature of gun violence also varies as between rural communities and
cities. Urban centers face significantly greater levels of firearm crime
and homicide, while rural communities have proportionately greater
problems with nonhomicide gun deaths, such as suicides and accidents.
And idiosyncratic local factors can lead to two cities finding
themselves in dramatically different circumstances: For example, in
2008, the murder rate was 40 times higher in New Orleans than it was in
Lincoln, Nebraska.

It is thus unsurprising that States and local communities have
historically differed about the need for gun regulation as well as about
its proper level. Nor is it surprising that ``primarily, and
historically,'' the law has treated the exercise of police powers,
including gun control, as ``matter{[}s{]} of local concern.'' Medtronic,
(internal quotation marks omitted).

Fourth, although incorporation of any right removes decisions from the
democratic process, the incorporation of this particular right does so
without strong offsetting justification---as the example of Oak Park's
handgun ban helps to show. See Oak Park, Ill., Municipal Code, § 27-2-1
(1995). Oak Park decided to ban handguns in 1983, after a local attorney
was shot to death with a handgun that his assailant had smuggled into a
courtroom in a blanket. Brief for Oak Park Citizens Committee for
Handgun Control as Amicus Curiae 1, 21 (hereinafter Oak Park Brief). A
citizens committee spent months gathering information about handguns. It
secured 6,000 signatures from community residents in support of a ban.
-22. And the village board enacted a ban into law.

Subsequently, at the urging of ban opponents the Board held a community
referendum on the matter. The citizens committee argued strongly in
favor of the ban. -23. It pointed out that most guns owned in Oak Park
were handguns and that handguns were misused more often than citizens
used them in self-defense. The ban opponents argued just as strongly to
the contrary. The public decided to keep the ban by a vote of 8,031 to
6,368. And since that time, Oak Park now tells us, crime has decreased
and the community has seen no accidental handgun deaths.

Given the empirical and local value-laden nature of the questions that
lie at the heart of the issue, why, in a Nation whose Constitution
foresees democratic decisionmaking, is it so fundamental a matter as to
require taking that power from the people? What is it here that the
people did not know? What is it that a judge knows better?

In sum, the police power, the superiority of legislative decisionmaking,
the need for local decisionmaking, the comparative desirability of
democratic decisionmaking, the lack of a manageable judicial standard,
and the life-threatening harm that may flow from striking down
regulations all argue against incorporation. Where the incorporation of
other rights has been at issue, some of these problems have arisen. But
in this instance all these problems are present, all at the same time,
and all are likely to be present in most, perhaps nearly all, of the
cases in which the constitutionality of a gun regulation is at issue. At
the same time, the important factors that favor incorporation in other
instances--- e.g., the protection of broader constitutional
objectives---are not present here. The upshot is that all factors
militate against incorporation---with the possible exception of
historical factors.

\hypertarget{fundamental-rights}{%
\subsection{Fundamental Rights}\label{fundamental-rights}}

\hypertarget{lochner-v.-new-york}{%
\subsubsection{Lochner v. New York}\label{lochner-v.-new-york}}

198 U.S. 45 (1905)

This is a writ of error to the county court of Oneida county, in the
state of New York (to which court the record had been remitted), to
review the judgment of the court of appeals of that state, affirming the
judgment of the supreme court, which itself affirmed the judgment of the
county court, convicting the defendant of a misdemeanor on an indictment
under a statute of that state, known, by its short title, as the labor
law. The section of the statute under which the indictment was found is
§ 110, and is reproduced in the margin1 (together with the other
sections of the labor law upon the subject of bakeries, being §§ 111 to
115, both inclusive).

The indictment averred that the defendant `wrongfully and unlawfully
required and permitted an employee working for him in his biscuit,
bread, and cake bakery and confectionery establishment, at the city of
Utica, in this county, to work more than sixty hours in one week,' after
having been theretofore convicted of a violation of the name act; and
therefore, as averred, he committed the crime of misdemeanor, second
offense. The plaintiff in error demurred to the indictment on several
grounds, one of which was that the facts stated did not constitute a
crime. The demurrer was overruled, and, the plaintiff in error having
refused to plead further, a plea of not guilty was entered by order of
the court and the trial commenced, and he was convicted of misdemeanor,
second offense, as indicted, and sentenced to pay a fine of \$50, and to
stand committed until paid, not to exceed fifty days in the Oneida
county jail. A certificate of reasonable doubt was granted by the county
judge of Oneida county, whereon an appeal was taken to the appellate
division of the supreme court, fourth department, where the judgment of
conviction was affirmed. A further appeal was then taken to the court of
appeals, where the judgment of conviction was again affirmed.

'§ 110, Hours of labor in bakeries and confectionery
establishments.---No employee shall be required or permitted to work in
a biscuit, bread, or cake bakery or confectionery establishment more
than sixty hours in any one week, or more than ten hours in any one day,
unless for the purpose of making a shorter work day on the last day of
the week; nor more hours in any one week than will make an average of
ten hours per day for the number of days during such week in which such
employee shall work.

'§ 111. Drainage and plumbing of buildings and rooms occupied by
bakeries.---All buildings or rooms occupied as biscuit, bread, pie, or
cake bakeries, shall be drained and plumbed in a manner conducive to the
proper and healthful sanitary condition thereof, and shall be
constructed with air shafts, windows, or ventilating pipes, sufficient
to insure ventilation. The factory inspector may direct the proper
drainage, plumbing, and ventilation of such rooms or buildings. No
cellar or basement, not now used for a bakery, shall hereafter be so
occupied or used, unless the proprietor shall comply with the sanitary
provisions of this article.

'§ 112. Requirements as to rooms, furniture, utensils, and manufactured
products.---Every room used for the manufacture of flour or meal food
products shall be at least 8 feet in height and shall have, if deemed
necessary by the factory inspector, an impermeable floor constructed of
cement, or of tiles laid in cement, or an additional flooring of wood
properly saturated with linseed oil. The side walls of such rooms shall
be plastered or wainscoted. The factory inspector may require the side
walls and ceiling to be whitewashed at least once in three months. He
may also require the wood work of such walls to be painted. The
furniture and utensils shall be so arranged as to be readily cleansed
and not prevent the proper cleaning of any part of the room. The
manufactured flour or meal food products shall be kept in dry and airy
rooms, so arranged that the floors, shelves, and all other facilities
for storing the same can be properly cleaned. No domestic animals,
except cats, shall be allowed to remain in a room used as a biscuit,
bread, pie, or cake bakery, or any room in such bakery where flour or
meal products are stored.

'§ 113. Wash rooms and closets; sleeping places.---Every such bakery
shall be provided with a proper wash room and water-closet, or
water-closets, apart from the bake room, or rooms where the manufacture
of such food product is conducted, and no water-closet, earth closet,
privy, or ashpit shall be within, or connected directly with, the bake
room of any bakery, hotel, or public restaurant.

'No person shall sleep in a room occupied as a bake room. Sleeping
places for the persons employed in the bakery shall be separate from the
rooms where flour or meal food products are manufactured or stored. If
the sleeping places are on the same floor where such products are
manufactured, stored, or sold, the factory inspector may inspect and
order them put in a proper sanitary condition.

'§ 114. Inspection of bakeries.---The factory inspector shall cause all
bakeries to be inspected. If it be found upon such inspection that the
bakeries so inspected are constructed and conducted in compliance with
the provisions of this chapter, the factory inspector shall issue a
certificate to the person owning or conducting such bakeries.

`§ 115. Notice requiring alterations.---If, in the opinion of the
factory inspector, alterations are required in or upon premises occupied
and used as bakeries, in order to comply with the provisions of this
article, a written notice shall be served by him upon the owner, agent,
or lessee of such premises, either personally or by mail, requiring such
alterations to be made within sixty days after such service, and such
alterations shall be made accordingly.' {[}N. Y. Laws 1897, chap 415.{]}

\textbf{Mr.~Justice Peckham, after making the foregoing statement of the
facts, delivered the opinion of the court:}

The indictment, it will be seen, charges that the plaintiff in error
violated the 110th section of article 8, chapter 415, of the Laws of
1897, known as the labor law of the state of New York, in that he
wrongfully and unlawfully required and permitted an employee working for
him to work more than sixty hours in one week. There is nothing in any
of the opinions delivered in this case, either in the supreme court or
the court of appeals of the state, which construes the section, in using
the word `required,' as referring to any physical force being used to
obtain the labor of an employee. It is assumed that the word means
nothing more than the requirement arising from voluntary contract for
such labor in excess of the number of hours specified in the statute.
There is no pretense in any of the opinions that the statute was
intended to meet a case of involuntary labor in any form. All the
opinions assume that there is no real distinction, so far as this
question is concerned, between the words `required' and `permitted.' The
mandate of the statute, that `no employee shall be required or permitted
to work,' is the substantial equivalent of an enactment that `no
employee shall contract or agree to work,' more than ten hours per day;
and, as there is no provision for special emergencies, the statute is
mandatory in all cases. It is not an act merely fixing the number of
hours which shall constitute a legal day's work, but an absolute
prohibition upon the employer permitting, under any circumstances, more
than ten hours' work to be done in his establishment. The employee may
desire to earn the extra money which would arise from his working more
than the prescribed time, but this statute forbids the employer from
permitting the employee to earn it.

The statute necessarily interferes with the right of contract between
the employer and employees, concerning the number of hours in which the
latter may labor in the bakery of the employer. The general right to
make a contract in relation to his business is part of the liberty of
the individual protected by the 14th Amendment of the Federal
Constitution. Allgeyer v. Louisiana 832, 17 Under that provision no
state can deprive any person of life, liberty, or property without due
process of law. The right to purchase or to sell labor is part of the
liberty protected by this amendment, unless there are circumstances
which exclude the right. There are, however, certain powers, existing in
the sovereignty of each state in the Union, somewhat vaguely termed
police powers, the exact description and limitation of which have not
been attempted by the courts. Those powers, broadly stated, and without,
at present, any attempt at a more specific limitation, relate to the
safety, health, morals, and general welfare of the public. Both property
and liberty are held on such reasonable conditions as may be imposed by
the governing power of the state in the exercise of those powers, and
with such conditions the 14th Amendment was not designed to interfere.
Mugler v. Kansas 205, 8 ; Re Kemmler 519, 10 ; Crowley v. Christensen
620, 11 ; Re Converse 796, 11

The state, therefore, has power to prevent the individual from making
certain kinds of contracts, and in regard to them the Federal
Constitution offers no protection. If the contract be one which the
state, in the legitimate exercise of its police power, has the right to
prohibit, it is not prevented from prohibiting it by the 14th Amendment.
Contracts in violation of a statute, either of the Federal or state
government, or a contract to let one's property for immoral purposes, or
to do any other unlawful act, could obtain no protection from the
Federal Constitution, as coming under the liberty of person or of free
contract. Therefore, when the state, by its legislature, in the assumed
exercise of its police powers, has passed an act which seriously limits
the right to labor or the right of contract in regard to their means of
livelihood between persons who are sui juris (both employer and
employee), it becomes of great importance to determine which shall
prevail,---the right of the individual to labor for such time as he may
choose, or the right of the state to prevent the individual from
laboring, or from entering into any contract to labor, beyond a certain
time prescribed by the state.

This court has recognized the existence and upheld the exercise of the
police powers of the states in many cases which might fairly be
considered as border ones, and it has, in the course of its
determination of questions regarding the asserted invalidity of such
statutes, on the ground of their violation of the rights secured by the
Federal Constitution, been guided by rules of a very liberal nature, the
application of which has resulted, in numerous instances, in upholding
the validity of state statutes thus assailed. Among the later cases
where the state law has been upheld by this court is that of Holden v.
Hardy 780, 18 A provision in the act of the legislature of Utah was
there under consideration, the act limiting the employment of workmen in
all underground mines or workings, to eight hours per day, `except in
cases of emergency, where life or property is in imminent danger.' It
also limited the hours of labor in smelting and other institutions for
the reduction or refining of ores or metals to eight hours per day,
except in like cases of emergency. The act was held to be a valid
exercise of the police powers of the state. A review of many of the
cases on the subject, decided by this and other courts, is given in the
opinion. It was held that the kind of employment, mining, smelting,
etc., and the character of the employees in such kinds of labor, were
such as to make it reasonable and proper for the state to interfere to
prevent the employees from being constrained by the rules laid down by
the proprietors in regard to labor. The following citation from the
observations of the supreme court of Utah in that case was made by the
judge writing the opinion of this court, and approved: `The law in
question is confined to the protection of that class of people engaged
in labor in underground mines, and in smelters and other works wherein
ores are reduced and refined. This law applies only to the classes
subjected by their employment to the peculiar conditions and effects
attending underground mining and work in smelters, and other works for
the reduction and refining of ores. Therefore it is not necessary to
discuss or decide whether the legislature can fix the hours of labor in
other employments.'

It will be observed that, even with regard to that class of labor, the
Utah statute provided for cases of emergency wherein the provisions of
the statute would not apply. The statute now before this court has no
emergency clause in it, and, if the statute is valid, there are no
circumstances and no emergencies under which the slightest violation of
the provisions of the act would be innocent. There is nothing in Holden
v. Hardy which covers the case now before us. Nor does Atkin v. Kansas
148, 24 touch the case at bar. The Atkin Case was decided upon the right
of the state to control its municipal corporations, and to prescribe the
conditions upon which it will permit work of a public character to be
done for a municipality. Knoxville Iron Co.~v. Harbison 55, 22 is
equally far from an authority for this legislation. The employees in
that case were held to be at a disadvantage with the employer in matters
of wages, they being miners and coal workers, and the act simply
provided for the cashing of coal orders when presented by the miner to
the employer.

The latest case decided by this court, involving the police power, is
that of Jacobson v. Massachusetts, decided at this term and reported in
197 U. S. 11, 25 L. ed.------. It related to compulsory vaccination, and
the law was held valid as a proper exercise of the police powers with
reference to the public health. It was stated in the opinion that it was
a case `of an adult who, for aught that appears, was himself in perfect
health and a fit subject of vaccination, and yet, while remaining in the
community, refused to obey the statute and the regulation, adopted in
execution of its provisions, for the protection of the public health and
the public safety, confessedly endangered by the presence of a dangerous
disease.' That case is also far from covering the one now before the
court.

Petit v. Minnesota 716, 20 was upheld as a proper exercise of the police
power relating to the observance of Sunday, and the case held that the
legislature had the right to declare that, as matter of law, keeping
barber shops open on Sunday was not a work of necessity or charity.

It must, of course, be conceded that there is a limit to the valid
exercise of the police power by the state. There is no dispute
concerning this general proposition. Otherwise the 14th Amendment would
have no efficacy and the legislatures of the states would have unbounded
power, and it would be enough to say that any piece of legislation was
enacted to conserve the morals, the health, or the safety of the people;
such legislation would be valid, no matter how absolutely without
foundation the claim might be. The claim of the police power would be a
mere pretext,---become another and delusive name for the supreme
sovereignty of the state to be exercised free from constitutional
restraint. This is not contended for. In every case that comes before
this court, therefore, where legislation of this character is concerned,
and where the protection of the Federal Constitution is sought, the
question necessarily arises: Is this a fair, reasonable, and appropriate
exercise of the police power of the state, or is it an unreasonable,
unnecessary, and arbitrary interference with the right of the individual
to his personal liberty, or to enter into those contracts in relation to
labor which may seem to him appropriate or necessary for the support of
himself and his family? Of course the liberty of contract relating to
labor includes both parties to it. The one has as much right to purchase
as the other to sell labor.

This is not a question of substituting the judgment of the court for
that of the legislature. If the act be within the power of the state it
is valid, although the judgment of the court might be totally opposed to
the enactment of such a law. But the question would still remain: Is it
within the police power of the state? and that question must be answered
by the court.

The question whether this act is valid as a labor law, pure and simple,
may be dismissed in a few words. There is no reasonable ground for
interfering with the liberty of person or the right of free contract, by
determining the hours of labor, in the occupation of a baker. There is
no contention that bakers as a class are not equal in intelligence and
capacity to men in other trades or manual occupations, or that they are
not able to assert their rights and care for themselves without the
protecting arm of the state, interfering with their independence of
judgment and of action. They are in no sense wards of the state. Viewed
in the light of a purely labor law, with no reference whatever to the
question of health, we think that a law like the one before us involves
neither the safety, the morals, nor the welfare, of the public, and that
the interest of the public is not in the slightest degree affected by
such an act. The law must be upheld, if at all, as a law pertaining to
the health of the individual engaged in the occupation of a baker. It
does not affect any other portion of the public than those who are
engaged in that occupation. Clean and wholesome bread does not depend
upon whether the baker works but ten hours per day or only sixty hours a
week. The limitation of the hours of labor does not come within the
police power on that ground.

It is a question of which of two powers or rights shall prevail,---the
power of the state to legislate or the right of the individual to
liberty of person and freedom of contract. The mere assertion that the
subject relates, though but in a remote degree, to the public health,
does not necessarily render the enactment valid. The act must have a
more direct relation, as a means to an end, and the end itself must be
appropriate and legitimate, before an act can be held to be valid which
interferes with the general right of an individual to be free in his
person and in his power to contract in relation to his own labor.

This case has caused much diversity of opinion in the state courts. In
the supreme court two of the five judges composing the court dissented
from the judgment affirming the validity of the act. In the court of
appeals three of the seven judges also dissented from the judgment
upholding the statute. Although found in what is called a labor law of
the state, the court of appeals has upheld the act as one relating to
the public health,---in other words, as a health law. One of the judges
of the court of appeals, in upholding the law, stated that, in his
opinion, the regulation in question could not be sustained unless they
were able to say, from common knowledge, that working in a bakery and
candy factory was an unhealthy employment. The judge held that, while
the evidence was not uniform, it still led him to the conclusion that
the occupation of a baker or confectioner was unhealthy and tended to
result in diseases of the respiratory organs. Three of the judges
dissented from that view, and they thought the occupation of a baker was
not to such an extent unhealthy as to warrant the interference of the
legislature with the liberty of the individual.

We think the limit of the police power has been reached and passed in
this case. There is, in our judgment, no reasonable foundation for
holding this to be necessary or appropriate as a health law to safeguard
the public health, or the health of the individuals who are following
the trade of a baker. If this statute be valid, and if, therefore, a
proper case is made out in which to deny the right of an individual, sui
juris, as employer or employee, to make contracts for the labor of the
latter under the protection of the provisions of the Federal
Constitution, there would seem to be no length to which legislation of
this nature might not go. The case differs widely, as we have already
stated, from the expressions of this court in regard to laws of this
nature, as stated in Holden v. Hardy 780, 18 and Jacobson v.
Massachusetts L. ed.------.

We think that there can be no fair doubt that the trade of a baker, in
and of itself, is not an unhealthy one to that degree which would
authorize the legislature to interfere with the right to labor, and with
the right of free contract on the part of the individual, either as
employer or employee In looking through statistics regarding all trades
and occupations, it may be true that the trade of a baker does not
appear to be as healthy as some other trades, and is also vastly more
healthy than still others. To the common understanding the trade of a
baker has never been regarded as an unhealthy one. Very likely
physicians would not recommend the exercise of that or of any other
trade as a remedy for ill health. Some occupations are more healthy than
others, but we think there are none which might not come under the power
of the legislature to supervise and control the hours of working
therein, if the mere fact that the occupation is not absolutely and
perfectly healthy is to confer that right upon the legislative
department of the government. It might be safely affirmed that almost
all occupations more or less affect the health. There must be more than
the mere fact of the possible existence of some small amount of
unhealthiness to warrant legislative interference with liberty. It is
unfortunately true that labor, even in any department, may possibly
carry with it the seeds of unhealthiness. But are we all, on that
account, at the mercy of legislative majorities? A printer, a tinsmith,
a locksmith, a carpenter, a cabinetmaker, a dry goods clerk, a bank's, a
lawyer's, or a physician's clerk, or a clerk in almost any kind of
business, would all come under the power of the legislature, on this
assumption. No trade, no occupation, no mode of earning one's living,
could escape this all-pervading power, and the acts of the legislature
in limiting the hours of labor in all employments would be valid,
although such limitation might seriously cripple the ability of the
laborer to support himself and his family. In our large cities there are
many buildings into which the sun penetrates for but a short time in
each day, and these buildings are occupied by people carrying on the
business of bankers, brokers, lawyers, real estate, and many other kinds
of business, aided by many clerks, messengers, and other employees. Upon
the assumption of the validity of this act under review, it is not
possible to say that an act, prohibiting lawyers' or bank clerks, or
others, from contracting to labor for their employers more than eight
hours a day would be invalid. It might be said that it is unhealthy to
work more than that number of hours in an apartment lighted by
artificial light during the working hours of the day; that the
occupation of the bank clerk, the lawyer's clerk, the real estate clerk,
or the broker's clerk, in such offices is therefore unhealthy, and the
legislature, in its paternal wisdom, must, therefore, have the right to
legislate on the subject of, and to limit, the hours for such labor;
and, if it exercises that power, and its validity be questioned, it is
sufficient to say, it has reference to the public health; it has
reference to the health of the employees condemned to labor day after
day in buildings where the sun never shines; it is a health law, and
therefore it is valid, and cannot be questioned by the courts.

It is also urged, pursuing the same line of argument, that it is to the
interest of the state that its population should be strong and robust,
and therefore any legislation which may be said to tend to make people
healthy must be valid as health laws, enacted under the police power. If
this be a valid argument and a justification for this kind of
legislation, it follows that the protection of the Federal Constitution
from undue interference with liberty of person and freedom of contract
is visionary, wherever the law is sought to be justified as a valid
exercise of the police power. Scarcely any law but might find shelter
under such assumptions, and conduct, properly so called, as well as
contract, would come under the restrictive sway of the legislature. Not
only the hours of employees, but the hours of employers, could be
regulated, and doctors, lawyers, scientists, all professional men, as
well as athletes and artisans, could be forbidden to fatigue their
brains and bodies by prolonged hours of exercise, lest the fighting
strength of the state be impaired. We mention these extreme cases
because the contention is extreme. We do not believe in the soundness of
the views which uphold this law. On the contrary, we think that such a
law as this, although passed in the assumed exercise of the police
power, and as relating to the public health, or the health of the
employees named, is not within that power, and is invalid. The act is
not, within any fair meaning of the term, a health law, but is an
illegal interference with the rights of individuals, both employers and
employees, to make contracts regarding labor upon such terms as they may
think best, or which they may agree upon with the other parties to such
contracts. Statutes of the nature of that under review, limiting the
hours in which grown and intelligent men may labor to earn their living,
are mere meddlesome interferences with the rights of the individual, and
they are not saved from condemnation by the claim that they are passed
in the exercise of the police power and upon the subject of the health
of the individual whose rights are interfered with, unless there be some
fair ground, reasonable in and of itself, to say that there is material
danger to the public health, or to the health of the employees, if the
hours of labor are not curtailed. If this be not clearly the case, the
individuals whose rights are thus made the subject of legislative
interference are under the protection of the Federal Constitution
regarding their liberty of contract as well as of person; and the
legislature of the state has no power to limit their right as proposed
in this statute. All that it could properly do has been done by it with
regard to the conduct of bakeries, as provided for in the other sections
of the act, above set forth. These several sections provide for the
inspection of the premises where the bakery is carried on, with regard
to furnishing proper wash-rooms and water-closets, apart from the bake
room, also with regard to providing proper drainage, plumbing, and
painting; the sections, in addition, provide for the height of the
ceiling, the cementing or tiling of floors, where necessary in the
opinion of the factory inspector, and for other things of that nature;
alterations are also provided for, and are to be made where necessary in
the opinion of the inspector, in order to comply with the provisions of
the statute. These various sections may be wise and valid regulations,
and they certainly go to the full extent of providing for the
cleanliness and the healthiness, so far as possible, of the quarters in
which bakeries are to be conducted. Adding to all these requirements a
prohibition to enter into any contract of labor in a bakery for more
than a certain number of hours a week is, in our judgment, so wholly
beside the matter of a proper, reasonable, and fair provision as to run
counter to that liberty of person and of free contract provided for in
the Federal Constitution.

It was further urged on the argument that restricting the hours of labor
in the case of bakers was valid because it tended to cleanliness on the
part of the workers, as a man was more apt to be cleanly when not
overworked, and if cleanly then his `output' was also more likely to be
so. What has already been said applies with equal force to this
contention. We do not admit the reasoning to be sufficient to justify
the claimed right of such interference. The state in that case would
assume the position of a supervisor, or pater familias, over every act
of the individual, and its right of governmental interference with his
hours of labor, his hours of exercise, the character thereof, and the
extent to which it shall be carried would be recognized and upheld. In
our judgment it is not possible in fact to discover the connection
between the number of hours a baker may work in the bakery and the
healthful quality of the bread made by the workman. The connection, if
any exist, is too shadowy and thin to build any argument for the
interference of the legislature. If the man works ten hours a day it is
all right, but if ten and a half or eleven his health is in danger and
his bread may be unhealthy, and, therefore, he shall not be permitted to
do it. This, we think, is unreasonable and entirely arbitrary. When
assertions such as we have adverted to become necessary in order to
give, if possible, a plausible foundation for the contention that the
law is a `health law,' it gives rise to at least a suspicion that there
was some other motive dominating the legislature than the purpose to
subserve the public health or welfare.

This interference on the part of the legislatures of the several states
with the ordinary trades and occupations of the people seems to be on
the increase. In the supreme court of New York, in the case of People v.
Beattie, appellate division, first department, decided in 1904 (96 App.
Div. 383, 89 N. Y. Supp. 193), a statute regulating the trade of
horseshoeing, and requiring the person practising such trade to be
examined, and to obtain a certificate from a board of examiners and file
the same with the clerk of the county wherein the person proposes to
practise such trade, was held invalid, as an arbitrary interference with
personal liberty and private property without due process of law. The
attempt was made, unsuccessfully, to justify it as a health law.

The same kind of a statute was held invalid (Re Aubry) by the supreme
court of Washington in December, 1904. 78 Pac. 900. The court held that
the act deprived citizens of their liberty and property without due
process of law, and denied to them the equal protection of the laws. It
also held that the trade of a horseshoer is not a subject of regulation
under the police power of the state, as a business concerning and
directly affecting the health, welfare, or comfort of its inhabitants;
and that, therefore, a law which provided for the examination and
registration of horseshoers in certain cities was unconstitutional, as
an illegitimate exercise of the police power.

The supreme court of Illinois, in Bessette v. People, 193 Ill. 334, 56
L. R. A. 558, 62 N. E. 215, also held that a law of the same nature,
providing for the regulation and licensing of horseshoers, was
unconstitutional as an illegal interference with the liberty of the
individual in adopting and pursuing such calling as he may choose,
subject only to the restraint necessary to secure the common welfare.
See also Godcharles v. Wigeman, 113 Pa. 431, 437, 6 Atl. 354; Low v.
Rees Printing Co.~41 Neb. 127, 145, 24 L. R. A. 702, 43 Am.
St.~Rep.~670, 59 N. W. 362. In these cases the courts upheld the right
of free contract and the right to purchase and sell labor upon such
terms as the parties may agree to.

It is impossible for us to shut our eyes to the fact that many of the
laws of this character, while passed under what is claimed to be the
police power for the purpose of protecting the public health or welfare,
are, in reality, passed from other motives. We are justified in saying
so when, from the character of the law and the subject upon which it
legislates, it is apparent that the public health or welfare bears but
the most remote relation to the law. The purpose of a statute must be
determined from the natural and legal effect of the language employed;
and whether it is or is not repugnant to the Constitution of the United
States must be determined from the natural effect of such statutes when
put into operation, and not from their proclaimed purpose. Minnesota v.
Barber 455, 3 Inters. Com. Rep.~185, 10 ; Brimmer v. Rebman 862, 3
Inters. Com. Rep.~485, 11 The court looks beyond the mere letter of the
law in such cases. Yick Wo v. Hopkins 220, 6

It is manifest to us that the limitation of the hours of labor as
provided for in this section of the statute under which the indictment
was found, and the plaintiff in error convicted, has no such direct
relation to, and no such substantial effect upon, the health of the
employee, as to justify us in regarding the section as really a health
law. It seems to us that the real object and purpose were simply to
regulate the hours of labor between the master and his employees (all
being men, Sui juris), in a private business, not dangerous in any
degree to morals, or in any real and substantial degree to the health of
the employees. Under such circumstances the freedom of master and
employee to contract with each other in relation to their employment,
and in defining the same, cannot be prohibited or interfered with,
without violating the Federal Constitution.

The judgment of the Court of Appeals of New York, as well as that of the
Supreme Court and of the County Court of Oneida County, must be reversed
and the case remanded to the County Court for further proceedings not
inconsistent with this opinion.

Reversed.

\textbf{Mr.~Justice Holmes dissenting:}

I regret sincerely that I am unable to agree with the judgment in this
case, and that I think it my duty to express my dissent.

This case is decided upon an economic theory which a large part of the
country does not entertain. If it were a question whether I agreed with
that theory, I should desire to study it further and long before making
up my mind. But I do not conceive that to be my duty, because I strongly
believe that my agreement or disagreement has nothing to do with the
right of a majority to embody their opinions in law. It is settled by
various decisions of this court that state constitutions and state laws
may regulate life in many ways which we as legislators might think as
injudicious, or if you like as tyrannical, as this, and which, equally
with this, interfere with the liberty to contract. Sunday laws and usury
laws are ancient examples. A more modern one is the prohibition of
lotteries. The liberty of the citizen to do as he likes so long as he
does not interfere with the liberty of others to do the same, which has
been a shibboleth for some well-known writers, is interfered with by
school laws, by the Postoffice, by every state or municipal institution
which takes his money for purposes thought desirable, whether he likes
it or not. The 14th Amendment does not enact Mr.~Herbert Spencer's
Social Statics. The other day we sustained the Massachusetts vaccination
law. Jacobson v. Massachusetts ---United States and state statutes and
decisions cutting down the liberty to contract by way of combination are
familiar to this court. Northern Securities Co.~v. United States 679, 24
Two years ago we upheld the prohibition of sales of stock on margins, or
for future delivery, in the Constitution of California. Otis v. Parker
323, 23 The decision sustaining an eight-hour law for miners is still
recent. Holden v. Hardy 780, 18 Some of these laws embody convictions or
prejudices which judges are likely to share. Some may not. But a
Constitution is not intended to embody a particular economic theory,
whether of paternalism and the organic relation of the citizen to the
state or of laissez faire.

It is made for people of fundamentally differing views, and the accident
of our finding certain opinions natural and familiar, or novel, and even
shocking, ought not to conclude our judgment upon the question whether
statutes embodying them conflict with the Constitution of the United
States.

General propositions do not decide concrete cases. The decision will
depend on a judgment or intuition more subtle than any articulate major
premise. But I think that the proposition just stated, if it is
accepted, will carry us far toward the end. Every opinion tends to
become a law. I think that the word `liberty,' in the 14th Amendment, is
perverted when it is held to prevent the natural outcome of a dominant
opinion, unless it can be said that a rational and fair man necessarily
would admit that the statute proposed would infringe fundamental
principles as they have been understood by the traditions of our people
and our law. It does not need research to show that no such sweeping
condemnation can be passed upon the statute before us. A reasonable man
might think it a proper measure on the score of health. Men whom I
certainly could not pronounce unreasonable would uphold it as a first
installment of a general regulation of the hours of work. Whether in the
latter aspect it would be open to the charge of inequality I think it
unnecessary to discuss.

\textbf{Mr.~Justice Harlan (with whom Mr.~Justice White and Mr.~Justice
Day concurred) dissenting:} While this court has not attempted to mark
the precise boundaries of what is called the police power of the state,
the existence of the power has been uniformly recognized, equally by the
Federal and State courts.

All the cases agree that this power extends at least to the protection
of the lives, the health, and the safety of the public against the
injurious exercise by any citizen of his own rights.

In Patterson v. Kentucky, after referring to the general principle that
rights given by the Constitution cannot be impaired by state legislation
of any kind, this court said: `It {[}this court{]} has, nevertheless,
with marked distinctness and uniformity, recognized the necessity,
growing out of the fundamental conditions of civil society, of upholding
state police regulations which were enacted in good faith, and had
appropriate and direct connection with that protection to life, health,
and property which each state owes to her citizens.' So in Barbier v.
Connolly: `But neither the {[}14th{]} Amendment, ---broad and
comprehensive as it is,---nor any other amendment, was designed to
interfere with the power of the state, sometimes termed its police
power, to prescribe regulations to promote the health, peace, morals,
education, and good order of the people.'

Speaking generally, the state, in the exercise of its powers, may not
unduly interfere with the right of the citizen to enter into contracts
that may be necessary and essential in the enjoyment of the inherent
rights belonging to everyone, among which rights is the right `to be
free in the enjoyment of all his faculties, to be free to use them in
all lawful ways, to live and work where he will, to earn his livelihood
by any lawful calling, to pursue any livelihood or avocation.' This was
declared in Allgeyer v. Louisiana, 17 But in the same case it was
conceded that the right to contract in relation to persons and property,
or to do business, within a state, may be `regulated, and sometimes
prohibited, when the contracts or business conflict with the policy of
the state as contained in its statutes.'

So, as said in Holden v. Hardy: `This right of contract, however, is
itself subject to certain limitations which the state may lawfully
impose in the exercise of its police powers. While this power is
inherent in all governments, it has doubtless been greatly expanded in
its application during the past century, owing to an enormous increase
in the number of occupations which are dangerous, or so far detrimental,
to the health of employees as to demand special precautions for their
well-being and protection, or the safety of adjacent property. While
this court has held, notably in the cases Davidson v. New Orleans and
Yick Wo. v. Hopkins, that the police power cannot be put forward as an
excuse for oppressive and unjust legislation, it may be lawfully
resorted to for the purpose of preserving the public health, safety, or
morals, or the abatement of public nuisances; and a large discretion 'is
necessarily vested in the legislature to determine, not only what the
interests of the public required, but what measures are necessary for
the protection of such interests.' Lawton v. Steele, Referring to the
limitations placed by the state upon the hours of workmen, the court in
the same case said: `These employments, when too long pursued, the
legislature has judged to be detrimental to the health of the employees,
and, so long as there are reasonable grounds for believing that this is
so, its decision upon this subject cannot be reviewed by the Federal
courts.'

Subsequently, in Gundling v. Chicago, this court said: 'Regulations
respecting the pursuit of a lawful trade or business are of very
frequent occurrence in the various cities of the country, and what such
regulations shall be and to what particular trade, business, or
occupation they shall apply, are questions for the state to determine,
and their determination comes within the proper exercise of the police
power by the state, and, unless the regulations are so utterly
unreasonable and extravagant in their nature and purpose that the
property and personal rights of the citizen are unnecessarily, and in a
manner wholly arbitrary, interfered with or destroyed without due
process of law, they do not extend beyond the power of the state to
pass, and they form no subject for Federal interference. As stated in
Crowley v. Christensen 'the possession and enjoyment of all rights are
subject to such reasonable conditions as may be deemed by the governing
authority of the country essential to the safety, health, peace, good
order, and morals of the community."

In St.~Louis I. M. \& S. R. Co.~v. Paul, and in Knoxville Iron Co.~v.
Harbison it was distinctly adjudged that the right of contract was not
`absolute, but may be subjected to the restraints demanded by the safety
and welfare of the state.' Those cases illustrate the extent to which
the state may restrict or interfere with the exercise of the right of
contracting.

The authorities on the same line are so numerous that further citations
are unnecessary. I take it to be firmly established that what is called
the liberty of contract may, within certain limits, be subjected to
regulations designed and calculated to promote the general welfare, or
to guard the public health, the public morals, or the public safety.
`The liberty secured by the Constitution of the United States to every
person within its jurisdiction does not import.' this court has recently
said, `an absolute right in each person to be at all times and in all
circumstances wholly freed from restraint. There are manifold restraints
to which every person is necessarily subject for the common good.'
Jacobson v. Massachusetts.

Granting, then, that there is a liberty of contract which cannot be
violated even under the sanction of direct legislative enactment, but
assuming, as according to settled law we may assume, that such liberty
of contract is subject to such regulations as the state may reasonably
prescribe for the common good and the well-being of society, what are
the conditions under which the judiciary may declare such regulations to
be in excess of legislative authority and void? Upon this point there is
no room for dispute; for the rule is universal that a legislative
enactment, Federal or state, is never to be disregarded or held invalid
unless it be, beyond question, plainly and palpably in excess of
legislative power. In Jacobson v. Massachusetts, we said that the power
of the courts to review legislative action in respect of a matter
affecting the general welfare exists only `when that which the
legislature has done comes within the rule that, if a statute purporting
to have been enacted to protect the public health, the public morals, or
the public safety has no real or substantial relation to those objects,
or is, beyond all question, a plain, palpable invasion of rights secured
by the fundamental law.' If there be doubt as to the validity of the
statute, that doubt must therefore be resolved in favor of its validity,
and the courts must keep their hands off, leaving the legislature to
meet the responsibility for unwise legislation. If the end which the
legislature seeks to accomplish be one to which its power extends, and
if the means employed to that end, although not the wisest or best, are
yet not plainly and palpably unauthorized by law, then the court cannot
interfere. In other words, when the validity of a statute is questioned,
the burden of proof, so to speak, is upon those who assert it to be
unconstitutional. M'Culloch v. Maryland.

Let these principles be applied to the present case. By the statute in
question it is provided that `no employee shall be required, or
permitted, to work in a biscuit, bread, or cake bakery, or confectionery
establishment, more than sixty hours in any one week, or more than ten
hours in any one day, unless for the purpose of making a shorter work
day on the last day of the week; nor more hours in any one week than
will make an average of ten hours per day for the number of days during
such week in which such employee shall work.'

It is plain that this statute was enacted in order to protect the
physical well-being of those who work in bakery and confectionery
establishments. It may be that the statute had its origin, in part, in
the belief that employers and employees in such establishments were not
upon an equal footing, and that the necessities of the latter often
compelled them to submit to such exactions as unduly taxed their
strength. Be this as it may, the statute must be taken as expressing the
belief of the people of New York that, as a general rule, and in the
case of the average man, labor in excess of sixty hours during a week in
such establishments may endanger the health of those who thus labor.
Whether or not this be wise legislation it is not the province of the
court to inquire. Under our systems of government the courts are not
concerned with the wisdom or policy of legislation. So that, in
determining the question of power to interfere with liberty of contract,
the court may inquire whether the means devised by the state are germane
to an end which may be lawfully accomplished and have a real or
substantial relation to the protection of health, as involved in the
daily work of the persons, male and female, engaged in bakery and
confectionery establishments. But when this inquiry is entered upon I
find it impossible, in view of common experience, to say that there is
here no real or substantial relation between the means employed by the
state and the end sought to be accomplished by its legislation. Mugler
v. Kansas. Nor can I say that the statute has no appropriate or direct
connection with that protection to health which each state owes to her
citizens (Patterson v. Kentucky); or that it is not promotive of the
health of the employees in question (Holden v. Hardy; Lawton v. Steele);
or that the regulation prescribed by the state is utterly unreasonable
and extravagant or wholly arbitrary (Gundling v. Chicago). Still less
can I say that the statute is, beyond question, a plain, palpable
invasion of rights secured by the fundamental law. Jacobson v.
Massachusetts. Therefore I submit that this court will transcend its
functions if it assumes to annul the statute of New York. It must be
remembered that this statute does not apply to all kinds of business. It
applies only to work in bakery and confectionery establishments, in
which, as all know, the air constantly breathed by workmen is not as
pure and healthful as that to be found in some other establishments or
out of doors.

Professor Hirt in his treatise on the `Diseases of the Workers' has
said: `The labor of the bakers is among the hardest and most laborious
imaginable, because it has to be performed under conditions injurious to
the health of those engaged in it. It is hard, very hard, work, not only
because it requires a great deal of physical exertion in an overheated
workshop and during unreasonably long hours, but more so because of the
erratic demands of the public, compelling the baker to perform the
greater part of his work at night, thus depriving him of an opportunity
to enjoy the necessary rest and sleep,---a fact which is highly
injurious to his health.' Another writer says: `The constant inhaling of
flour dust causes inflammation of the lungs and of the bronchial tubes.
The eyes also suffer through this dust, which is responsible for the
many cases of running eyes among the bakers. The long hours of toil to
which all bakers are subjected produce rheumatism, cramps, and swollen
legs. The intense heat in the workshops induces the workers to resort to
cooling drinks, which, together with their habit of exposing the greater
part of their bodies to the change in the atmosphere, is another source
of a number of diseases of various organs. Nearly all bakers are
palefaced and of more delicate health than the workers of other crafts,
which is chiefly due to their hard work and their irregular and
unnatural mode of living, whereby the power of resistance against
disease is greatly diminished. The average age of a baker is below that
of other workmen; they seldom live over their fiftieth year, most of
them dying between the ages of forty and fifty. During periods of
epidemic diseases the bakers are generally the first to succumb to the
disease, and the number swept away during such periods far exceeds the
number of other crafts in comparison to the men employed in the
respective industries. When, in 1720, the plague visited the city of
Marseilles, France, every baker in the city succumbed to the epidemic,
which caused considerable excitement in the neighboring cities and
resulted in measures for the sanitary protection of the bakers.'

In the Eighteenth Annual Report by the New York Bureau of Statistics of
Labor it is stated that among the occupations involving exposure to
conditions that interfere with nutrition is that of a baker. (p.~52.) In
that Report it is also stated that, `from a social point of view,
production will be increased by any change in industrial organization
which diminishes the number of idlers, paupers, and criminals. Shorter
hours of work, by allowing higher standards of comfort and purer family
life, promise to enhance the industrial efficiency of the wage-working
class, improved health, longer life, more content and greater
intelligence and inventiveness.' (p.~82.)

Statistics show that the average daily working time among workingmen in
different countries is, in Australia, eight hours; in Great Britain,
nine; in the United States, nine and three-quarters; in Denmark, nine
and three-quarters; in Norway, ten; Sweden, France, and Switzerland, ten
and one-half; Germany, ten and one-quarter; Belgium, Italy, and Austria,
eleven; and in Russia, twelve hours.

We judicially know that the question of the number of hours during which
a workman should continuously labor has been, for a long period, and is
yet, a subject of serious consideration among civilized peoples, and by
those having special knowledge of the laws of health. Suppose the
statute prohibited labor in bakery and confectionery establishments in
excess of eighteen hours each day. No one, I take it, could dispute the
power of the state to enact such a statute. But the statute before us
does not embrace extreme or exceptional cases. It may be said to occupy
a middle ground in respect of the hours of labor. What is the true
ground for the state to take between legitimate protection, by
legislation, of the public health and liberty of contract is not a
question easily solved, nor one in respect of which there is or can be
absolute certainty. There are very few, if any, questions in political
economy about which entire certainty may be predicated. One writer on
relation of the state to labor has well said: `The manner, occasion, and
degree in which the state may interfere with the industrial freedom of
its citizens is one of the most debatable and difficult questions of
social science.' Jevons, 33.

We also judicially know that the number of hours that should constitute
a day's labor in particular occupations involving the physical strength
and safety of workmen has been the subject of enactments by Congress and
by nearly all of the states. Many, if not most, of those enactments fix
eight hours as the proper basis of a day's labor.

I do not stop to consider whether any particular view of this economic
question presents the sounder theory. What the precise facts are it may
be difficult to say. It is enough for the determination of this case,
and it is enough for this court to know, that the question is one about
which there is room for debate and for an honest difference of opinion.
There are many reasons of a weighty, substantial character, based upon
the experience of mankind, in support of the theory that, all things
considered, more than ten hours' steady work each day, from week to
week, in a bakery or confectionery establishment, may endanger the
health and shorten the lives of the workmen, thereby diminishing their
physical and mental capacity to serve the state and to provide for those
dependent upon them.

If such reasons exist that ought to be the end of this case, for the
state is not amenable to the judiciary, in respect of its legislative
enactments, unless such enactments are plainly, palpably, beyond all
question, inconsistent with the Constitution of the United States. We
are not to presume that the state of New York has acted in bad faith.
Nor can we assume that its legislature acted without due deliberation,
or that it did not determine this question upon the fullest attainable
information and for the common good. We cannot say that the state has
acted without reason, nor ought we to proceed upon the theory that its
action is a mere sham. Our duty, I submit, is to sustain the statute as
not being in conflict with the Federal Constitution, for the
reason---and such is an all-sufficient reason---it is not shown to be
plainly and palpably inconsistent with that instrument. Let the state
alone in the management of its purely domestic affairs, so long as it
does not appear beyond all question that it has violated the Federal
Constitution. This view necessarily results from the principle that the
health and safety of the people of a state are primarily for the state
to guard and protect.

I take leave to say that the New York statute, in the particulars here
involved, cannot be held to be in conflict with the 14th Amendment,
without enlarging the scope of the amendment far beyond its original
purpose, and without bringing under the supervision of this court
matters which have been supposed to belong exclusively to the
legislative departments of the several states when exerting their
conceded power to guard the health and safety of their citizens by such
regulations as they in their wisdom deem best. Health laws of every
description constitute, said Chief Justice Marshall, a part of that mass
of legislation which `embraces everything within the territory of a
state, not surrendered to the general government; all which can be most
advantageously exercised by the states themselves.' Gibbons v. Ogden, 9
Wheat. 1, 203, A decision that the New York statute is void under the
14th Amendment will, in my opinion, involve consequences of a
far-reaching and mischievous character; for such a decision would
seriously cripple the inherent power of the states to care for the
lives, health, and well-being of their citizens. Those are matters which
can be best controlled by the states.

The preservation of the just powers of the states is quite as vital as
the preservation of the powers of the general government.

When this court had before it the question of the constitutionality of a
statute of Kansas making it a criminal offense for a contractor for
public work to permit or require his employees to perform labor upon
such work in excess of eight hours each day, it was contended that the
statute was in derogation of the liberty both of employees and employer.
It was further contended that the Kansas statute was mischievous in its
tendencies. This court, while disposing of the question only as it
affected public work, held that the Kansas statute was not void under
the 14th Amendment. But it took occasion to say what may well be here
repeated: `The responsibility therefor rests upon legislators, not upon
the courts. No evils arising from such legislation could be more far
reaching than those that might come to our system of government if the
judiciary, abandoning the sphere assigned to it by the fundamental law,
should enter the domain of legislation, and upon grounds merely of
justice or reason or wisdom annul statutes that had received the
sanction of the people's representatives. We are reminded by counsel
that it is the solemn duty of the courts in cases before them to guard
the constitutional rights of the citizen against merely arbitrary power.
That is unquestionably true. But it is equally true---indeed, the public
interests imperatively demand---that legislative enactments should be
recognized and enforced by the courts as embodying the will of the
people, unless they are plainly and palpably beyond all question in
violation of the fundamental law of the Constitution.' Atkin v. Kansas,
24

The judgment, in my opinion, should be affirmed.

\hypertarget{united-states-v.-carolene-products-co.}{%
\subsubsection{United States v. Carolene Products
Co.}\label{united-states-v.-carolene-products-co.}}

304 U.S. 144 (1938)

\textbf{Justice STONE delivered the opinion of the Court.}

The question for decision is whether the `Filled Milk Act' of Congress
of March 4, 1923, c.~262, 42 Stat. 1486, 21 U.S.C. §§ 61 63, 21 U.S.C.A.
§ 61---63,1 which prohibits the shipment in interstate commerce of
skimmed milk compounded with any fat or oil other than milk fat, so as
to resemble milk or cream, transcends the power of Congress to regulate
interstate commerce or infringes the Fifth Amendment.

Appellee also complains that the statute denies to it equal protection
of the laws, and in violation of the Fifth Amendment, deprives it of its
property without due process of law, particularly in that the statute
purports to make binding and conclusive upon appellee the legislative
declaration that appellee's product `is an adulterated article of food,
injurious to the public health, and its sale constitutes a fraud on the
public.'

Second. The prohibition of shipment of appellee's product in interstate
commerce does not infringe the Fifth Amendment. Twenty years ago this
Court, in Hebe Co.~v. Shawheld that a state law which forbids the
manufacture and sale of a product assumed to be wholesome and nutritive,
made of condensed skimmed milk, compounded with coconut oil, is not
forbidden by the Fourteenth Amendment. The power of the Legislature to
secure a minimum of particular nutritive elements in a widely used
article of food and to protect the public from fraudulent substitutions,
was not doubted; and the Court thought that there was ample scope for
the legislative judgment that prohibition of the offending article was
an appropriate means of preventing injury to the public.

We see no persuasive reason for departing from that ruling here, where
the Fifth Amendment is concerned; and since none is suggested, we might
rest decision wholly on the presumption of constitutionality. But
affirmative evidence also sustains the statute. In twenty years evidence
has steadily accumulated of the danger to the public health from the
general consumption of foods which have been stripped of elements
essential to the maintenance of health. The Filled Milk Act was adopted
by Congress after committee hearings, in the course of which eminent
scientists and health experts testified. An extensive investigation was
made of the commerce in milk compounds in which vegetable oils have been
substituted for natural milk fat, and of the effect upon the public
health of the use of such compounds as a food substitute for milk. The
conclusions drawn from evidence presented at the hearings were embodied
in reports of the House Committee on Agriculture, H.R. No.~365, 67th
Cong., 1st Sess., and the Senate Committee on Agriculture and Forestry,
Sen.Rep. No.~987, 67th Cong., 4th Sess. Both committees concluded, as
the statute itself declares, that the use of filled milk as a substitute
for pure milk is generally injurious to health and facilitates fraud on
the public.

There is nothing in the Constitution which compels a Legislature, either
national or state, to ignore such evidence, nor need it disregard the
other evidence which amply supports the conclusions of the Congressional
committees that the danger is greatly enhanced where an inferior
product, like appellee's, is indistinguishable from a valuable food of
almost universal use, thus making fraudulent distribution easy and
protection of the consumer difficult.

Appellee raises no valid objection to the present statute by arguing
that its prohibition has not been extended to oleomargarine or other
butter substitutes in which vegetable fats or oils are substituted for
butter fat. The Fifth Amendment has no equal protection clause, and even
that of the Fourteenth, applicable only to the states, does not compel
their Legislatures to prohibit all like evils, or none. A Legislature
may hit at an abuse which it has found, even

Third. We may assume for present purposes that no pronouncement of a
Legislature can forestall attack upon the constitutionality of the
prohibition which it enacts by applying opprobrious epithets to the
prohibited act, and that a statute would deny due process which
precluded the disproof in judicial proceedings of all facts which would
show or tend to show that a statute depriving the suitor of life,
liberty, or property had a rational basis.

But such we think is not the purpose or construction of the statutory
characterization of filled milk as injurious to health and as a fraud
upon the public. There is no need to consider it here as more than a
declaration of the legislative findings deemed to support and justify
the action taken as a constitutional exertion of the legislative power,
aiding informed judicial review, as do the reports of legislative
committees, by revealing the rationale of the legislation. Even in the
absence of such aids, the existence of facts supporting the legislative
judgment is to be presumed, for regulatory legislation affecting
ordinary commercial transactions is not to be pronounced
unconstitutional unless in the light of the facts made known or
generally assumed it is of such a character as to preclude the
assumption that it rests upon some rational basis within the knowledge
and experience of the legislators See Metropolitan Casualty Ins. Co.~v.
Brownell, and cases cited, The present statutory findings affect
appellee no more than the reports of the Congressional committees and
since in the absence of the statutory findings they would be presumed,
their incorporation in the statute is no more prejudicial than
surplusage.

Where the existence of a rational basis for legislation whose
constitutionality is attacked depends upon facts beyond the sphere of
judicial notice, such facts may properly be made the subject of judicial
inquiry, Borden's Farm Products Co.~v. Baldwinand the constitutionality
of a statute predicated upon the existence of a particular state of
facts may be challenged by showing to the court that those facts have
ceased to exist. Chastleton Corporation v. SinclairSimilarly we
recognize that the constitutionality of a statute, valid on its face,
may be assailed by proof of facts tending to show that the statute as
applied to a partic- ular article is without support in reason because
the article, although within the prohibited class, is so different from
others of the class as to be without the reason for the prohibition,
Railroad Retirement Board v. Alton R. Co., 351, 352, 763, see Whitney v.
California, ; cf.~Morf v. Bingaman, though the effect of such proof
depends on the relevant circumstances of each case, as for example the
administrative difficulty of excluding the article from the regulated
class. Carmichael v. Southern Coal \& Coke Co., 512, 109 A.L.R. 1327;
South Carolina State Highway Department v. Barnwell Bros., Inc.decided
February 14, 1938. But by their very nature such inquiries, where the
legislative judgment is drawn in question, must be restricted to the
issue whether any state of facts either known or which could reasonably
be assumed affords support for it. Here the demurrer challenges the
validity of the statute on its face and it is evident from all the
considerations presented to Congress, and those of which we may take
judicial notice, that the question is at least debatable whether
commerce in filled milk should be left unregulated, or in some measure
restricted, or wholly prohibited. As that decision was for Congress,
neither the finding of a court arrived at by weighing the evidence, nor
the verdict of a jury can be substituted for it. Price v. Illinois, ;
Hebe Co.~v. Shaw; Standard Oil Co.~v. Marysville, ; South Carolina v.
Barnwell Bros., Inc.citing Worcester County Trust Co.~v. Riley The
prohibition of shipment in interstate commerce of appellee's product, as
described in the indictment, is a constitutional exercise of the power
to regulate interstate commerce. As the statute is not unconstitutional
on its face, the demurrer should have been overruled and the judgment
will be reversed.

Reversed. 4 There may be narrower scope for operation of the presumption
of constitutionality when legislation appears on its face to be within a
specific prohibition of the Constitution, such as those of the first ten
Amendments, which are deemed equally specific when held to be embraced
within the Fourteenth. See Stromberg v. California, 370, 536, 73 A.L.R.
1484; Lovell v. Griffindecided March 28, 1938. It is unnecessary to
consider now whether legislation which restricts those political
processes which can ordinarily be expected to bring about repeal of
undesirable legislation, is to be subjected to more exacting judicial
scrutiny under the general prohibitions of the Fourteenth Amendment than
are most other types of legislation. On restrictions upon the right to
vote, see Nixon v. Herndon; Nixon v. Condon88 A.L.R. 458; on restraints
upon the dissemination of information, see Near v. Minnesota, 722, 632,
633, ; Grosjean v. American Press Co.; Lovell v. Griffin; on
interferences with political organizations, see Stromberg v. California,
73 A.L.R. 1484; Fiske v. Kansas; Whitney v. California, ; Herndon v.
Lowry; and see Holmes, J., in Gitlow v. New York, ; as to prohibition of
peaceable assembly, see De Jonge v. Oregon, Nor need we enquire whether
similar considerations enter into the review of statutes directed at
particular religious, Pierce v. Society of Sisters39 A.L.R. 468, or
national, Meyer v. Nebraska29 A.L.R. 1446; Bartels v. Iowa; Farrington
v. Tokushigeor racial minorities. Nixon v. Herndon; Nixon v. Condon;
whether prejudice against discrete and insular minorities may be a
special condition, which tends seriously to curtail the operation of
those political processes ordinarily to be relied upon to protect
minorities, and which may call for a correspondingly more searching
judicial inquiry. Compare McCulloch v. Maryland, 4 Wheat. 316, 428, ;
South Carolina State Highway Department v. Barnwell Bros.decided
February 14, 1938, note 2, and cases cited.

\hypertarget{griswold-v.-connecticut}{%
\subsubsection{Griswold v. Connecticut}\label{griswold-v.-connecticut}}

381 U.S. 479 (1965)

\textbf{MR. JUSTICE DOUGLAS delivered the opinion of the Court.}
Appellant Griswold is Executive Director of the Planned Parenthood
League of Connecticut. Appellant Buxton is a licensed physician and a
professor at the Yale Medical School who served as Medical Director for
the League at its Center in New Haven---a center open and operating from
November 1 to November 10, 1961, when appellants were arrested.

They gave information, instruction, and medical advice to married
persons as to the means of preventing conception. They examined the wife
and prescribed the best contraceptive device or material for her use.
Fees were usually charged, although some couples were serviced free.

The statutes whose constitutionality is involved in this appeal are §§
53-32 and 54-196 of the General Statutes of Connecticut (1958 rev.). The
former provides:

``Any person who uses any drug, medicinal article or instrument for the
purpose of preventing conception shall be fined not less than fifty
dollars or imprisoned not less than sixty days nor more than one year or
be both fined and imprisoned.''

Section 54-196 provides:

``Any person who assists, abets, counsels, causes, hires or commands
another to commit any offense may be prosecuted and punished as if he
were the principal offender.''

The appellants were found guilty as accessories and fined \$100 each,
against the claim that the accessory statute as so applied violated the
Fourteenth Amendment. The Appellate Division of the Circuit Court
affirmed. The Supreme Court of Errors affirmed that judgment. 151 Conn.
544, 200 A. 2d 479. We noted probable jurisdiction. 379 U. S. 926.

We think that appellants have standing to raise the constitutional
rights of the married people with whom they had a professional
relationship. Tileston v. Ullman,318 U. S. 44, is different, for there
the plaintiff seeking to represent others asked for a declaratory
judgment. In that situation we thought that the requirements of standing
should be strict, lest the standards of ``case or controversy'' in
Article

of the Constitution become blurred. Here those doubts are removed by
reason of a criminal conviction for serving married couples in violation
of an aiding-and-abetting statute. Certainly the accessory should have
standing to assert that the offense which he is charged with assisting
is not, or cannot constitutionally be, a crime.

This case is more akin to Truax v. Raich, where an employee was
permitted to assert the rights of his employer; to Pierce v. Society of
Sisters, where the owners of private schools were entitled to assert the
rights of potential pupils and their parents; and to Barrows v. Jackson,
where a white defendant, party to a racially restrictive covenant, who
was being sued for damages by the covenantors because she had conveyed
her property to Negroes, was allowed to raise the issue that enforcement
of the covenant violated the rights of prospective Negro purchasers to
equal protection, although no Negro was a party to the suit. And see
Meyer v. Nebraska; Adler v. Board of Education; NAACP v. Alabama; NAACP
v. Button,371 U. S. 415. The rights of husband and wife, pressed here,
are likely to be diluted or adversely affected unless those rights are
considered in a suit involving those who have this kind of confidential
relation to them.

Coming to the merits, we are met with a wide range of questions that
implicate the Due Process Clause of the Fourteenth Amendment. Overtones
of some arguments suggest that Lochner v. New York, should be our guide.
But we decline that invitation as we did in West Coast Hotel Co.~v.
Parrish,300 U. S. 379; Olsen v. Nebraska; Lincoln Union v. Northwestern
Co.; Williamson v. Lee Optical Co.; Giboney v. Empire Storage Co.. We do
not sit as a super-legislature to determine the wisdom, need, and
propriety of laws that touch economic problems, business affairs, or
social conditions. This law, however, operates directly on an intimate
relation of husband and wife and their physician's role in one aspect of
that relation.

The association of people is not mentioned in the Constitution nor in
the Bill of Rights. The right to educate a child in a school of the
parents' choice---whether public or private or parochial---is also not
mentioned. Nor is the right to study any particular subject or any
foreign language. Yet the First Amendment has been construed to include
certain of those rights.

By Pierce v. Society of Sistersthe right to educate one's children as
one chooses is made applicable to the States by the force of the First
and Fourteenth Amendments. By Meyer v. Nebraskathe same dignity is given
the right to study the German language in a private school. In other
words, the State may not, consistently with the spirit of the First
Amendment, contract the spectrum of available knowledge. The right of
freedom of speech and press includes not only the right to utter or to
print, but the right to distribute, the right to receive, the right to
read (Martin v. Struthers and freedom of inquiry, freedom of thought,
and freedom to teach (see Wieman v. Updegraff---indeed the freedom of
the entire university community. Sweezy v. New Hampshire,354 U. S. 234,
249; Barenblatt v. United States; Baggett v. Bullitt. Without those
peripheral rights the specific rights would be less secure. And so we
reaffirm the principle of the Pierce and the Meyer cases.

In NAACP v. Alabama, we protected the ``freedom to associate and privacy
in one's associations,'' noting that freedom of association was a
peripheral First Amendment right. Disclosure of membership lists of a
constitutionally valid association, we held, was invalid ``as entailing
the likelihood of a substantial restraint upon the exercise by
petitioner's members of their right to freedom of association.'' In
other words, the First Amendment has a penumbra where privacy is
protected from governmental intrusion. In like context, we have
protected forms of ``association'' that are not political in the
customary sense but pertain to the social, legal, and economic benefit
of the members. NAACP v. Button. In Schware v. Board of Bar
Examiners,353 U. S. 232, we held it not permissible to bar a lawyer from
practice, because he had once been a member of the Communist Party. The
man's ``association with that Party'' was not shown to be ``anything
more than a political faith in a political party'' (id.) and was not
action of a kind proving bad moral character. -246.

Those cases involved more than the ``right of assembly'' ---a right that
extends to all irrespective of their race or ideology. De Jonge v.
Oregon. The right of ``association,'' like the right of belief (Board of
Education v. Barnette, is more than the right to attend a meeting; it
includes the right to express one's attitudes or philosophies by
membership in a group or by affiliation with it or by other lawful
means. Association in that context is a form of expression of opinion;
and while it is not expressly included in the First Amendment its
existence is necessary in making the express guarantees fully
meaningful.

The foregoing cases suggest that specific guarantees in the Bill of
Rights have penumbras, formed by emanations from those guarantees that
help give them life and substance. See Poe v. Ullman (dissenting
opinion). Various guarantees create zones of privacy. The right of
association contained in the penumbra of the First Amendment is one, as
we have seen. The Third Amendment in its prohibition against the
quartering of soldiers ``in any house'' in time of peace without the
consent of the owner is another facet of that privacy. The Fourth
Amendment explicitly affirms the ``right of the people to be secure in
their persons, houses, papers, and effects, against unreasonable
searches and seizures.'' The Fifth Amendment in its Self-Incrimination
Clause enables the citizen to create a zone of privacy which government
may not force him to surrender to his detriment. The Ninth Amendment
provides: ``The enumeration in the Constitution, of certain rights,
shall not be construed to deny or disparage others retained by the
people.''

The Fourth and Fifth Amendments were described in Boyd v. United States,
as protection against all governmental invasions ``of the sanctity of a
man's home and the privacies of life.''* We recently referred in Mapp v.
Ohio,367 U. S. 643, 656, to the Fourth Amendment as creating a ``right
to privacy, no less important than any other right carefully and
particularly reserved to the people.'' See Beaney, The Constitutional
Right to Privacy, 1962 Sup. Ct. Rev.~212; Griswold, The Right to be Let
Alone, 55 Nw. U. L. Rev.~216 (1960).

We have had many controversies over these penumbral rights of ``privacy
and repose.'' See, e. g., Breard v. Alexandria, 644; Public Utilities
Comm'n v. Pollak; Monroe v. Pape; Lanza v. New York; Frank v. Maryland;
Skinner v. Oklahoma. These cases bear witness that the right of privacy
which presses for recognition here is a legitimate one.

The present case, then, concerns a relationship lying within the zone of
privacy created by several fundamental constitutional guarantees. And it
concerns a law which, in forbidding the use of contraceptives rather
than regulating their manufacture or sale, seeks to achieve its goals by
means having a maximum destructive impact upon that relationship. Such a
law cannot stand in light of the familiar principle, so often applied by
this Court, that a ``governmental purpose to control or prevent
activities constitutionally subject to state regulation may not be
achieved by means which sweep unnecessarily broadly and thereby invade
the area of protected freedoms.'' NAACP v. Alabama. Would we allow the
police to search the sacred precincts of marital bedrooms for telltale
signs of the use of contraceptives? The very idea is repulsive to the
notions of privacy surrounding the marriage relationship.

We deal with a right of privacy older than the Bill of Rights---older
than our political parties, older than our school system. Marriage is a
coming together for better or for worse, hopefully enduring, and
intimate to the degree of being sacred. It is an association that
promotes a way of life, not causes; a harmony in living, not political
faiths; a bilateral loyalty, not commercial or social projects. Yet it
is an association for as noble a purpose as any involved in our prior
decisions.

Reversed.

\textbf{MR. JUSTICE GOLDBERG, whom THE CHIEF JUSTICE and MR. JUSTICE
BRENNAN join, concurring.} I agree with the Court that Connecticut's
birth-control law unconstitutionally intrudes upon the right of marital
privacy, and I join in its opinion and judgment. Although I have not
accepted the view that ``due process'' as used in the Fourteenth
Amendment incorporates all of the first eight Amendments (see my
concurring opinion in Pointer v. Texas, and the dissenting opinion of
MR. JUSTICE BRENNAN in Cohen v. Hurley, I do agree that the concept of
liberty protects those personal rights that are fundamental, and is not
confined to the specific terms of the Bill of Rights. My conclusion that
the concept of liberty is not so restricted and that it embraces the
right of marital privacy though that right is not mentioned explicitly
in the Constitution1 is supported both by numerous decisions of this
Court, referred to in the Court's opinion, and by the language and
history of the Ninth Amendment. In reaching the conclusion that the
right of marital privacy is protected, as being within the protected
penumbra of specific guarantees of the Bill of Rights, the Court refers
to the Ninth Amendment, I add these words to emphasize the relevance of
that Amendment to the Court's holding.

The Court stated many years ago that the Due Process Clause protects
those liberties that are ``so rooted in the traditions and conscience of
our people as to be ranked as fundamental.'' Snyder v. Massachusetts. In
Gitlow v. New York, the Court said:

``For present purposes we may and do assume that freedom of speech and
of the press---which are protected by the First Amendment from
abridgment by Congress---are among the fundamental personal rights
and'liberties' protected by the due process clause of the Fourteenth
Amendment from impairment by the States.'' (Emphasis added.)

And, in Meyer v. Nebraska, the Court, referring to the Fourteenth
Amendment, stated:

``While this Court has not attempted to define with exactness the
liberty thus guaranteed, the term has received much consideration and
some of the included things have been definitely stated. Without doubt,
it denotes not merely freedom from bodily restraint but also {[}for
example,{]} the right . to marry, establish a home and bring up children
. .''

This Court, in a series of decisions, has held that the Fourteenth
Amendment absorbs and applies to the States those specifics of the first
eight amendments which express fundamental personal rights The language
and history of the Ninth Amendment reveal that the Framers of the
Constitution believed that there are additional fundamental rights,
protected from governmental infringement, which exist alongside those
fundamental rights specifically mentioned in the first eight
constitutional amendments.

The Ninth Amendment reads, ``The enumeration in the Constitution, of
certain rights, shall not be construed to deny or disparage others
retained by the people.'' The Amendment is almost entirely the work of
James Madison. It was introduced in Congress by him and passed the House
and Senate with little or no debate and virtually no change in language.
It was proffered to quiet expressed fears that a bill of specifically
enumerated rights3 could not be sufficiently broad to cover all
essential rights and that the specific mention of certain rights would
be interpreted as a denial that others were protected.

In presenting the proposed Amendment, Madison said: ``It has been
objected also against a bill of rights, that, by enumerating particular
exceptions to the grant of power, it would disparage those rights which
were not placed in that enumeration; and it might follow by implication,
that those rights which were not singled out, were intended to be
assigned into the hands of the General Government, and were consequently
insecure. This is one of the most plausible arguments I have ever heard
urged against the admission of a bill of rights into this system; but, I
conceive, that it may be guarded against. I have attempted it, as
gentlemen may see by turning to the last clause of the fourth resolution
{[}the Ninth Amendment{]}.'' I Annals of Congress 439 (Gales and Seaton
ed.~1834).

Mr.~Justice Story wrote of this argument against a bill of rights and
the meaning of the Ninth Amendment:

``In regard to . {[}a{]} suggestion, that the affirmance of certain
rights might disparage others, or might lead to argumentative
implications in favor of other powers, it might be sufficient to say
that such a course of reasoning could never be sustained upon any solid
basis . . But a conclusive answer is, that such an attempt may be
interdicted (as it has been) by a positive declaration in such a bill of
rights that the enumeration of certain rights shall not be construed to
deny or disparage others retained by the people.''

He further stated, referring to the Ninth Amendment: ``This clause was
manifestly introduced to prevent any perverse or ingenious
misapplication of the well-known maxim, that an affirmation in
particular cases implies a negation in all others; and, e converso,that
a negation in particular cases implies an affirmation in all others.''

These statements of Madison and Story make clear that the Framers did
not intend that the first eight amendments be construed to exhaust the
basic and fundamental rights which the Constitution guaranteed to the
people.

While this Court has had little occasion to interpret the Ninth
Amendment,6 ``{[}i{]}t cannot be presumed that any clause in the
constitution is intended to be without effect.'' Marbury v. Madison, 1
Cranch 137, 174. In interpreting the Constitution, ``real effect should
be given to all the words it uses.'' Myers v. United States. The Ninth
Amendment to the Constitution may be regarded by some as a recent
discovery and may be forgotten by others, but since 1791 it has been a
basic part of the Constitution which we are sworn to uphold. To hold
that a right so basic and fundamental and so deep-rooted in our society
as the right of privacy in marriage may be infringed because that right
is not guaranteed in so many words by the first eight amendments to the
Constitution is to ignore the Ninth Amendment and to give it no effect
whatsoever. Moreover, a judicial construction that this fundamental
right is not protected by the Constitution because it is not mentioned
in explicit terms by one of the first eight amendments or elsewhere in
the Constitution would violate the Ninth Amendment, which specifically
states that ``{[}t{]}he enumeration in the Constitution, of certain
rights, shall not be construed to deny or disparage others retained by
the people.'' (Emphasis added.)

A dissenting opinion suggests that my interpretation of the Ninth
Amendment somehow ``broaden{[}s{]} the powers of this Court.'' Post.
With all due respect, I believe that it misses the import of what I am
saying. I do not take the position of my Brother BLACK in his dissent in
Adamson v. California, that the entire Bill of Rights is incorporated in
the Fourteenth Amendment, and I do not mean to imply that the Ninth
Amendment is applied against the States by the Fourteenth. Nor do I mean
to state that the Ninth Amendment constitutes an independent source of
rights protected from infringement by either the States or the Federal
Government. Rather, the Ninth Amendment shows a belief of the
Constitution's authors that fundamental rights exist that are not
expressly enumerated in the first eight amendments and an intent that
the list of rights included there not be deemed exhaustive. As any
student of this Court's opinions knows, this Court has held, often
unanimously, that the Fifth and Fourteenth Amendments protect certain
fundamental personal liberties from abridgment by the Federal Government
or the States. See, e. g., Bolling v. Sharpe; Aptheker v. Secretary of
State; Kent v. Dulles; Cantwell v. Connecticut; NAACP v. Alabama; Gideon
v. Wainwright; New York Times Co.~v. Sullivan. The Ninth Amendment
simply shows the intent of the Constitution's authors that other
fundamental personal rights should not be denied such protection or
disparaged in any other way simply because they are not specifically
listed in the first eight constitutional amendments. I do not see how
this broadens the authority of the Court; rather it serves to support
what this Court has been doing in protecting fundamental rights.

Nor am I turning somersaults with history in arguing that the Ninth
Amendment is relevant in a case dealing with a State's infringement of a
fundamental right. While the Ninth Amendment---and indeed the entire
Bill of Rights---originally concerned restrictions upon federal power,
the subsequently enacted Fourteenth Amendment prohibits the States as
well from abridging fundamental personal liberties. And, the Ninth
Amendment, in indicating that not all such liberties are specifically
mentioned in the first eight amendments, is surely relevant in showing
the existence of other fundamental personal rights, now protected from
state, as well as federal, infringement. In sum, the Ninth Amendment
simply lends strong support to the view that the ``liberty'' protected
by the Fifth and Fourteenth Amendments from infringement by the Federal
Government or the States is not restricted to rights specifically
mentioned in the first eight amendments. Cf. United Public Workers v.
Mitchell.

In determining which rights are fundamental, judges are not left at
large to decide cases in light of their personal and private notions.
Rather, they must look to the ``traditions and {[}collective{]}
conscience of our people'' to determine whether a principle is ``so
rooted {[}there{]} . as to be ranked as fundamental.'' Snyder v.
Massachusetts. The inquiry is whether a right involved ``is of such a
character that it cannot be denied without violating those'fundamental
principles of liberty and justice which lie at the base of all our civil
and political institutions' . .'' Powell v. Alabama. ``Liberty'' also
``gains content from the emanations of . specific {[}constitutional{]}
guarantees'' and ``from experience with the requirements of a free
society.''

I agree fully with the Court that, applying these tests, the right of
privacy is a fundamental personal right, emanating ``from the totality
of the constitutional scheme under which we live.'' Mr.~Justice
Brandeis, dissenting in Olmstead v. United States, comprehensively
summarized the principles underlying the Constitution's guarantees of
privacy:

``The protection guaranteed by the {[}Fourth and Fifth{]} Amendments is
much broader in scope. The makers of our Constitution undertook to
secure conditions favorable to the pursuit of happiness. They recognized
the significance of man's spiritual nature, of his feelings and of his
intellect. They knew that only a part of the pain, pleasure and
satisfactions of life are to be found in material things. They sought to
protect Americans in their beliefs, their thoughts, their emotions and
their sensations. They conferred, as against the Government, the right
to be let alone--- the most comprehensive of rights and the right most
valued by civilized men.''

The Connecticut statutes here involved deal with a particularly
important and sensitive area of privacy---that of the marital relation
and the marital home. This Court recognized in Meyer v. Nebraskathat the
right ``to marry, establish a home and bring up children'' was an
essential part of the liberty guaranteed by the Fourteenth Amendment.
262 U. S.. In Pierce v. Society of Sisters, the Court held
unconstitutional an Oregon Act which forbade parents from sending their
children to private schools because such an act ``unreasonably
interferes with the liberty of parents and guardians to direct the
upbringing and education of children under their control.'' 268 U. S..
As this Court said in Prince v. Massachusetts, the Meyer and Pierce
decisions ``have respected the private realm of family life which the
state cannot enter.''

I agree with MR. JUSTICE HARLAN's statement in his dissenting opinion in
Poe v. Ullman: ``Certainly the safeguarding of the home does not follow
merely from the sanctity of property rights. The home derives its
pre-eminence as the seat of family life. And the integrity of that life
is something so fundamental that it has been found to draw to its
protection the principles of more than one explicitly granted
Constitutional right. . Of this whole'private realm of family life' it
is difficult to imagine what is more private or more intimate than a
husband and wife's marital relations.''

The entire fabric of the Constitution and the purposes that clearly
underlie its specific guarantees demonstrate that the rights to marital
privacy and to marry and raise a family are of similar order and
magnitude as the fundamental rights specifically protected.

Although the Constitution does not speak in so many words of the right
of privacy in marriage, I cannot believe that it offers these
fundamental rights no protection. The fact that no particular provision
of the Constitution explicitly forbids the State from disrupting the
traditional relation of the family---a relation as old and as
fundamental as our entire civilization---surely does not show that the
Government was meant to have the power to do so. Rather, as the Ninth
Amendment expressly recognizes, there are fundamental personal rights
such as this one, which are protected from abridgment by the Government
though not specifically mentioned in the Constitution.

My Brother STEWART, while characterizing the Connecticut birth control
law as ``an uncommonly silly law,'' post, would nevertheless let it
stand on the ground that it is not for the courts to ``\,`substitute
their social and economic beliefs for the judgment of legislative
bodies, who are elected to pass laws.'\,'' Post. Elsewhere, I have
stated that ``{[}w{]}hile I quite agree with Mr.~Justice Brandeis that .
.'a . State may . serve as a laboratory; and try novel social and
economic experiments,' New State Ice Co.~v. Liebmann, 311 (dissenting
opinion), I do not believe that this includes the power to experiment
with the fundamental liberties of citizens . .'' The vice of the
dissenters' views is that it would permit such experimentation by the
States in the area of the fundamental personal rights of its citizens. I
cannot agree that the Constitution grants such power either to the
States or to the Federal Government.

The logic of the dissents would sanction federal or state legislation
that seems to me even more plainly unconstitutional than the statute
before us. Surely the Government, absent a showing of a compelling
subordinating state interest, could not decree that all husbands and
wives must be sterilized after two children have been born to them. Yet
by their reasoning such an invasion of marital privacy would not be
subject to constitutional challenge because, while it might be
``silly,'' no provision of the Constitution specifically prevents the
Government from curtailing the marital right to bear children and raise
a family. While it may shock some of my Brethren that the Court today
holds that the Constitution protects the right of marital privacy, in my
view it is far more shocking to believe that the personal liberty
guaranteed by the Constitution does not include protection against such
totalitarian limitation of family size, which is at complete variance
with our constitutional concepts. Yet, if upon a showing of a slender
basis of rationality, a law outlawing voluntary birth control by married
persons is valid, then, by the same reasoning, a law requiring
compulsory birth control also would seem to be valid. In my view,
however, both types of law would unjustifiably intrude upon rights of
marital privacy which are constitutionally protected.

In a long series of cases this Court has held that where fundamental
personal liberties are involved, they may not be abridged by the States
simply on a showing that a regulatory statute has some rational
relationship to the effectuation of a proper state purpose. ``Where
there is a significant encroachment upon personal liberty, the State may
prevail only upon showing a subordinating interest which is
compelling,'' Bates v. Little Rock. The law must be shown ``necessary,
and not merely rationally related, to the accomplishment of a
permissible state policy.'' McLaughlin v. Florida. See Schneiderv.
Irvington.

Although the Connecticut birth-control law obviously encroaches upon a
fundamental personal liberty, the State does not show that the law
serves any ``subordinating {[}state{]} interest which is compelling'' or
that it is ``necessary . to the accomplishment of a permissible state
policy.'' The State, at most, argues that there is some rational
relation between this statute and what is admittedly a legitimate
subject of state concern---the discouraging of extra-marital relations.
It says that preventing the use of birth-control devices by married
persons helps prevent the indulgence by some in such extramarital
relations. The rationality of this justification is dubious,
particularly in light of the admitted widespread availability to all
persons in the State of Connecticut, unmarried as well as married, of
birth-control devices for the prevention of disease, as distinguished
from the prevention of conception, see Tileston v. Ullman, 129 Conn. 84,
26 A. 2d 582. But, in any event, it is clear that the state interest in
safeguarding marital fidelity can be served by a more discriminately
tailored statute, which does not, like the present one, sweep
unnecessarily broadly, reaching far beyond the evil sought to be dealt
with and intruding upon the privacy of all married couples. See Aptheker
v. Secretary of State; NAACP v. Alabama; McLaughlin v. Florida. Here, as
elsewhere, where, ``{[}p{]}recision of regulation must be the touchstone
in an area so closely touching our most precious freedoms.'' NAACP v.
Button. The State of Connecticut does have statutes, the
constitutionality of which is beyond doubt, which prohibit adultery and
fornication. See Conn. Gen.~Stat. §§ 53 et seq. These statutes
demonstrate that means for achieving the same basic purpose of
protecting marital fidelity are available to Connecticut without the
need to ``invade the area of protected freedoms.'' NAACP v. Alabama. See
McLaughlin v. Florida.

Finally, it should be said of the Court's holding today that it in no
way interferes with a State's proper regulation of sexual promiscuity or
misconduct. As my Brother HARLAN so well stated in his dissenting
opinion in Poe v. Ullman.

``Adultery, homosexuality and the like are sexual intimacies which the
State forbids . but the intimacy of husband and wife is necessarily an
essential and accepted feature of the institution of marriage, an
institution which the State not only must allow, but which always and in
every age it has fostered and protected. It is one thing when the State
exerts its power either to forbid extra-marital sexuality . or to say
who may marry, but it is quite another when, having acknowledged a
marriage and the intimacies inherent in it, it undertakes to regulate by
means of the criminal law the details of that intimacy.''

In sum, I believe that the right of privacy in the marital relation is
fundamental and basic---a personal right ``retained by the people''
within the meaning of the Ninth Amendment. Connecticut cannot
constitutionally abridge this fundamental right, which is protected by
the Fourteenth Amendment from infringement by the States. I agree with
the Court that petitioners' convictions must therefore be reversed.

My Brother STEWART dissents on the ground that he ``can find no .
general right of privacy in the Bill of Rights, in any other part of the
Constitution, or in any case ever before decided by this Court.'' Post.
He would require a more explicit guarantee than the one which the Court
derives from several constitutional amendments. This Court, however, has
never held that the Bill of Rights or the Fourteenth Amendment protects
only those rights that the Constitution specifically mentions by name.
See, e. g., Bolling v. Sharpe; Aptheker v. Secretary of State; Kent v.
Dulles; Carrington v. Rash; Schware v. Board of Bar Examiners; NAACP v.
Alabama; Pierce v. Society of Sisters; Meyer v. Nebraska. To the
contrary, this Court, for example, in Bolling v. Sharpewhile recognizing
that the Fifth Amendment does not contain the ``explicit safeguard'' of
an equal protection clause, nevertheless derived an equal protection
principle from that Amendment's Due Process Clause. And in Schware v.
Board of Bar Examinersthe Court held that the Fourteenth Amendment
protects from arbitrary state action the right to pursue an occupation,
such as the practice of law.

Madison himself had previously pointed out the dangers of inaccuracy
resulting from the fact that ``no language is so copious as to supply
words and phrases for every complex idea.'' The Federalist, No.~37
(Cooke ed.~1961).

Alexander Hamilton was opposed to a bill of rights on the ground that it
was unnecessary because the Federal Government was a government of
delegated powers and it was not granted the power to intrude upon
fundamental personal rights. The Federalist, No.~84 (Cooke ed.~1961). He
also argued, ``I go further, and affirm that bills of rights, in the
sense and in the extent in which they are contended for, are not only
unnecessary in the proposed constitution, but would even be dangerous.
They would contain various exceptions to powers which are not granted;
and on this very account, would afford a colourable pretext to claim
more than were granted. For why declare that things shall not be done
which there is no power to do? Why for instance, should it be said, that
the liberty of the press shall not be restrained, when no power is given
by which restrictions may be imposed? I will not contend that such a
provision would confer a regulating power; but it is evident that it
would furnish, to men disposed to usurp, a plausible pretence for
claiming that power.''

The Ninth Amendment and the Tenth Amendment, which provides, ``The
powers not delegated to the United States by the Constitution, nor
prohibited by it to the States, are reserved to the States respectively,
or to the people,'' were apparently also designed in part to meet the
above-quoted argument of Hamilton.

The Tenth Amendment similarly made clear that the States and the people
retained all those powers not expressly delegated to the Federal
Government.

This Amendment has been referred to as ``The Forgotten Ninth
Amendment,'' in a book with that title by Bennett B. Patterson (1955).
Other commentary on the Ninth Amendment includes Redlich, Are There
``Certain Rights . Retained by the People''? 37 N. Y. U. L. Rev.~787
(1962), and Kelsey, The Ninth Amendment of the Federal Constitution, 11
Ind. L. J. 309 (1936). As far as I am aware, until today this Court has
referred to the Ninth Amendment only in United Public Workers v.
Mitchell; Tennessee Electric Power Co.~v. TVA; and Ashwander v. TVA. See
also Calder v. Bull, 3 Dall. 386, 388; Loan Assn. v. Topeka, 20 Wall.
655, 662-663. In United Public Workers v. Mitchell, the Court stated:
``We accept appellants' contention that the nature of political rights
reserved to the people by the Ninth and Tenth Amendments {[}is{]}
involved. The right claimed as inviolate may be stated as the right of a
citizen to act as a party official or worker to further his own
political views. Thus we have a measure of interference by the Hatch Act
and the Rules with what otherwise would be the freedom of the civil
servant under the First, Ninth and Tenth Amendments. And, if we look
upon due process as a guarantee of freedom in those fields, there is a
corresponding impairment of that right under the Fifth Amendment.''

In light of the tests enunciated in these cases it cannot be said that a
judge's responsibility to determine whether a right is basic and
fundamental in this sense vests him with unrestricted personal
discretion. In fact, a hesitancy to allow too broad a discretion was a
substantial reason leading me to conclude in Pointer v. Texas, that
those rights absorbed by the Fourteenth Amendment and applied to the
States because they are fundamental apply with equal force and to the
same extent against both federal and state governments. In Pointer I
said that the contrary view would require ``this Court to make the
extremely subjective and excessively discretionary determination as to
whether a practice, forbidden the Federal Government by a fundamental
constitutional guarantee, is, as viewed in the factual circumstances
surrounding each individual case, sufficiently repugnant to the notion
of due process as to be forbidden the States.''

\textbf{MR. JUSTICE HARLAN, concurring in the judgment.} I fully agree
with the judgment of reversal, but find myself unable to join the
Court's opinion. The reason is that it seems to me to evince an approach
to this case very much like that taken by my Brothers BLACK and STEWART
in dissent, namely: the Due Process Clause of the Fourteenth Amendment
does not touch this Connecticut statute unless the enactment is found to
violate some right assured by the letter or penumbra of the Bill of
Rights.

In other words, what I find implicit in the Court's opinion is that the
``incorporation'' doctrine may be used to restrict the reach of
Fourteenth Amendment Due Process. For me this is just as unacceptable
constitutional doctrine as is the use of the ``incorporation'' approach
to impose upon the States all the requirements of the Bill of Rights as
found in the provisions of the first eight amendments and in the
decisions of this Court interpreting them. See, e. g., my concurring
opinions in Pointer v. Texas, and Griffin v. California, and my
dissenting opinion in Poe v. Ullman, at pp.~539-545.

In my view, the proper constitutional inquiry in this case is whether
this Connecticut statute infringes the Due Process Clause of the
Fourteenth Amendment because the enactment violates basic values
``implicit in the concept of ordered liberty,'' Palko v. Connecticut.
For reasons stated at length in my dissenting opinion in Poe v. UllmanI
believe that it does. While the relevant inquiry may be aided by resort
to one or more of the provisions of the Bill of Rights, it is not
dependent on them or any of their radiations. The Due Process Clause of
the Fourteenth Amendment stands, in my opinion, on its own bottom.

A further observation seems in order respecting the justification of my
Brothers BLACK and STEWART for their ``incorporation'' approach to this
case. Their approach does not rest on historical reasons, which are of
course wholly lacking (see Fairman, Does the Fourteenth Amendment
Incorporate the Bill of Rights? The Original Understanding, 2 Stan. L.
Rev.~5 (1949)), but on the thesis that by limiting the content of the
Due Process Clause of the Fourteenth Amendment to the protection of
rights which can be found elsewhere in the Constitution, in this
instance in the Bill of Rights, judges will thus be confined to
``interpretation'' of specific constitutional provisions, and will
thereby be restrained from introducing their own notions of
constitutional right and wrong into the ``vague contours of the Due
Process Clause.'' Rochin v. California.

While I could not more heartily agree that judicial ``self restraint''
is an indispensable ingredient of sound constitutional adjudication, I
do submit that the formula suggested for achieving it is more hollow
than real. ``Specific'' provisions of the Constitution, no less than
``due process,'' lend themselves as readily to ``personal''
interpretations by judges whose constitutional outlook is simply to keep
the Constitution in supposed ``tune with the times'' (post, p.~522).
Need one go further than to recall last Term's reapportionment cases,
Wesberry v. Sanders, 376 U. S. 1. and Reynolds v. Sims, where a majority
of the Court ``interpreted'' ``by the People'' (Art. I, § 2) and ``equal
protection'' (Amdt. 14) to command ``one person, one vote,'' an
interpretation that was made in the face of irrefutable and still
unanswered history to the contrary? See my dissenting opinions in those
cases; 377 U. S..

Judicial self-restraint will not, I suggest, be brought about in the
``due process'' area by the historically unfounded incorporation formula
long advanced by my Brother BLACK, and now in part espoused by my
Brother STEWART. It will be achieved in this area, as in other
constitutional areas, only by continual insistence upon respect for the
teachings of history, solid recognition of the basic values that
underlie our society, and wise appreciation of the great roles that the
doctrines of federalism and separation of powers have played in
establishing and preserving American freedoms. See Adamson v.
CaliforniaMr. Justice Frankfurter, concurring). Adherence to these
principles will not, of course, obviate all constitutional differences
of opinion among judges, nor should it. Their continued recognition
will, however, go farther toward keeping most judges from roaming at
large in the constitutional field than will the interpolation into the
Constitution of an artificial and largely illusory restriction on the
content of the Due Process Clause.*

Indeed, my Brother BLACK, in arguing his thesis, is forced to lay aside
a host of cases in which the Court has recognized fundamental rights in
the Fourteenth Amendment without specific reliance upon the Bill of
Rights. Post, p.~512, n.~4. MR. JUSTICE WHITE, concurring in the
judgment. In my view this Connecticut law as applied to married couples
deprives them of ``liberty'' without due process of law, as that concept
is used in the Fourteenth Amendment. I therefore concur in the judgment
of the Court reversing these convictions under Connecticut's aiding and
abetting statute.

It would be unduly repetitious, and belaboring the obvious, to expound
on the impact of this statute on the liberty guaranteed by the
Fourteenth Amendment against arbitrary or capricious denials or on the
nature of this liberty. Suffice it to say that this is not the first
time this Court has had occasion to articulate that the liberty entitled
to protection under the Fourteenth Amendment includes the right ``to
marry, establish a home and bring up children,'' Meyer v. Nebraska, and
``the liberty . to direct the upbringing and education of children,''
Pierce v. Society of Sisters, and that these are among ``the basic civil
rights of man.'' Skinner v. Oklahoma. These decisions affirm that there
is a ``realm of family life which the state cannot enter'' without
substantial justification. Prince v. Massachusetts. Surely the right
invoked in this case, to be free of regulation of the intimacies of the
marriage relationship, ``come{[}s{]} to this Court with a momentum for
respect lacking when appeal is made to liberties which derive merely
from shifting economic arrangements.'' Kovacs v. Cooperopinion of
Frankfurter, J.).

The Connecticut anti-contraceptive statute deals rather substantially
with this relationship. For it forbids all married persons the right to
use birth-control devices, regardless of whether their use is dictated
by considerations of family planning, Trubek v. Ullman, 147 Conn. 633,
165 A. 2d 158, health, or indeed even of life itself. Buxton v. Ullman,
147 Conn. 48, 156 A. 2d 508. The anti-use statute, together with the
general aiding and abetting statute, prohibits doctors from affording
advice to married persons on proper and effective methods of birth
control. Tileston v. Ullman, 129 Conn. 84, 26 A. 2d 582. And the clear
effect of these statutes, as enforced, is to deny disadvantaged citizens
of Connecticut, those without either adequate knowledge or resources to
obtain private counseling, access to medical assistance and up-to-date
information in respect to proper methods of birth control. State v.
Nelson, 126 Conn. 412, 11 A. 2d 856; State v. Griswold, 151 Conn. 544,
200 A. 2d 479. In my view, a statute with these effects bears a
substantial burden of justification when attacked under the Fourteenth
Amendment. Yick Wo v. Hopkins; Skinner v. Oklahoma,316 U. S. 535;
Schware v. Board of Bar Examiners; McLaughlin v. Florida.

An examination of the justification offered, however, cannot be avoided
by saying that the Connecticut anti-use statute invades a protected area
of privacy and association or that it demeans the marriage relationship.
The nature of the right invaded is pertinent, to be sure, for statutes
regulating sensitive areas of liberty do, under the cases of this Court,
require ``strict scrutiny,'' Skinner v. Oklahoma,316 U. S. 535, 541, and
``must be viewed in the light of less drastic means for achieving the
same basic purpose.'' Shelton v. Tucker. ``Where there is a significant
encroachment upon personal liberty, the State may prevail only upon
showing a subordinating interest which is compelling.'' Bates v. Little
Rock. See also McLaughlin v. Florida. But such statutes, if reasonably
necessary for the effectuation of a legitimate and substantial state
interest, and not arbitrary or capricious in application, are not
invalid under the Due Process Clause. Zemel v. Rusk, 381 U. S. 1.*

As I read the opinions of the Connecticut courts and the argument of
Connecticut in this Court, the State claims but one justification for
its anti-use statute. Cf. Allied Stores of Ohio v. Bowers; Martin v.
Walton,368 U. S. 25, 28 (DOUGLAS, J., dissenting). There is no serious
contention that Connecticut thinks the use of artificial or external
methods of contraception immoral or unwise in itself, or that the
anti-use statute is founded upon any policy of promoting population
expansion. Rather, the statute is said to serve the State's policy
against all forms of promiscuous or illicit sexual relationships, be
they premarital or extramarital, concededly a permissible and legitimate
legislative goal.

Without taking issue with the premise that the fear of conception
operates as a deterrent to such relationships in addition to the
criminal proscriptions Connecticut has against such conduct, I wholly
fail to see how the ban on the use of contraceptives by married couples
in any way reinforces the State's ban on illicit sexual relationships.
See Schware v. Board of Bar Examiners. Connecticut does not bar the
importation or possession of contraceptive devices; they are not
considered contraband material under state law, State v. Certain
Contraceptive Materials, 126 Conn. 428, 11 A. 2d 863, and their
availability in that State is not seriously disputed. The only way
Connecticut seeks to limit or control the availability of such devices
is through its general aiding and abetting statute whose operation in
this context has been quite obviously ineffective and whose most serious
use has been against birth-control clinics rendering advice to married,
rather than unmarried, persons. Cf. Yick Wo v. Hopkins. Indeed, after
over 80 years of the State's proscription of use, the legality of the
sale of such devices to prevent disease has never been expressly passed
upon, although it appears that sales have long occurred and have only
infrequently been challenged. This ``undeviating policy . throughout all
the long years . bespeaks more than prosecutorial paralysis.'' Poe v.
Ullman. Moreover, it would appear that the sale of contraceptives to
prevent disease is plainly legal under Connecticut law.

In these circumstances one is rather hard pressed to explain how the ban
on use by married persons in any way prevents use of such devices by
persons engaging in illicit sexual relations and thereby contributes to
the State's policy against such relationships. Neither the state courts
nor the State before the bar of this Court has tendered such an
explanation. It is purely fanciful to believe that the broad
proscription on use facilitates discovery of use by persons engaging in
a prohibited relationship or for some other reason makes such use more
unlikely and thus can be supported by any sort of administrative
consideration. Perhaps the theory is that the flat ban on use prevents
married people from possessing contraceptives and without the ready
availability of such devices for use in the marital relationship, there
will be no or less temptation to use them in extramarital ones. This
reasoning rests on the premise that married people will comply with the
ban in regard to their marital relationship, notwithstanding total
nonenforcement in this context and apparent nonenforcibility, but will
not comply with criminal statutes prohibiting extramarital affairs and
the anti-use statute in respect to illicit sexual relationships, a
premise whose validity has not been demonstrated and whose intrinsic
validity is not very evident. At most the broad ban is of marginal
utility to the declared objective. A statute limiting its prohibition on
use to persons engaging in the prohibited relationship would serve the
end posited by Connecticut in the same way, and with the same
effectiveness, or ineffectiveness, as the broad anti-use statute under
attack in this case. I find nothing in this record justifying the
sweeping scope of this statute, with its telling effect on the freedoms
of married persons, and therefore conclude that it deprives such persons
of liberty without due process of law.

Dissenting opinions assert that the liberty guaranteed by the Due
Process Clause is limited to a guarantee against unduly vague statutes
and against procedural unfairness at trial. Under this view the Court is
without authority to ascertain whether a challenged statute, or its
application, has a permissible purpose and whether the manner of
regulation bears a rational or justifying relationship to this purpose.
A long line of cases makes very clear that this has not been the view of
this Court. Dent v. West Virginia; Jacobson v. Massachusetts; Douglas v.
Noble; Meyer v. Nebraska; Pierce v. Society of Sisters; Schware v. Board
of Bar Examiners; Aptheker v. Secretary of State; Zemel v. Rusk, 381 U.
S. 1. The traditional due process test was well articulated, and
applied, in Schware v. Board of Bar Examinersa case which placed no
reliance on the specific guarantees of the Bill of Rights.

``A State cannot exclude a person from the practice of law or from any
other occupation in a manner or for reasons that contravene the Due
Process or Equal Protection Clause of the Fourteenth Amendment. Dent v.
West Virginia. Cf. Slochower v. Board of Education; Wieman v. Updegraff.
And see Ex parte Secombe, 19 How. 9, 13. A State can require high
standards of qualification, such as good moral character or proficiency
in its law, before it admits an applicant to the bar, but any
qualification must have a rational connection with the applicant's
fitness or capacity to practice law. Douglas v. Noble; Cummings v.
Missouri, 4 Wall. 277, 319-320. Cf. Nebbia v. New York. Obviously an
applicant could not be excluded merely because he was a Republican or a
Negro or a member of a particular church. Even in applying permissible
standards, officers of a State cannot exclude an applicant when there is
no basis for their finding that he fails to meet these standards, or
when their action is invidiously discriminatory.'' 353 U. S.. Cf. Martin
v. WaltonDOUGLAS, J., dissenting).

\textbf{MR. JUSTICE BLACK, with whom MR. JUSTICE STEWART joins,
dissenting.} I agree with my Brother STEWART'S dissenting opinion. And
like him I do not to any extent whatever base my view that this
Connecticut law is constitutional on a belief that the law is wise or
that its policy is a good one. In order that there may be no room at all
to doubt why I vote as I do, I feel constrained to add that the law is
every bit as offensive to me as it is to my Brethren of the majority and
my Brothers HARLAN, WHITE and GOLDBERG who, reciting reasons why it is
offensive to them, hold it unconstitutional. There is no single one of
the graphic and eloquent strictures and criticisms fired at the policy
of this Connecticut law either by the Court's opinion or by those of my
concurring Brethren to which I cannot subscribe---except their
conclusion that the evil qualities they see in the law make it
unconstitutional.

Had the doctor defendant here, or even the nondoctor defendant, been
convicted for doing nothing more than expressing opinions to persons
coming to the clinic that certain contraceptive devices, medicines or
practices would do them good and would be desirable, or for telling
people how devices could be used, I can think of no reasons at this time
why their expressions of views would not be protected by the First and
Fourteenth Amendments, which guarantee freedom of speech. Cf.
Brotherhood of Railroad Trainmen v. Virginia ex rel. Virginia State Bar,
377 U. S. 1; NAACP v. Button. But speech is one thing; conduct and
physical activities are quite another. See, e. g., Cox v. Louisiana; Cox
v. Louisiana; -584 (concurring opinion); Giboney v. Empire Storage \&
Ice Co.; cf.~Reynolds v. United States,98 U. S. 145, 163-164. The two
defendants here were active participants in an organization which gave
physical examinations to women, advised them what kind of contraceptive
devices or medicines would most likely be satisfactory for them, and
then supplied the devices themselves, all for a graduated scale of fees,
based on the family income. Thus these defendants admittedly engaged
with others in a planned course of conduct to help people violate the
Connecticut law. Merely because some speech was used in carrying on that
conduct---just as in ordinary life some speech accompanies most kinds of
conduct---we are not in my view justified in holding that the First
Amendment forbids the State to punish their conduct. Strongly as I
desire to protect all First Amendment freedoms, I am unable to stretch
the Amendment so as to afford protection to the conduct of these
defendants in violating the Connecticut law. What would be the
constitutional fate of the law if hereafter applied to punish nothing
but speech is, as I have said, quite another matter.

The Court talks about a constitutional ``right of privacy'' as though
there is some constitutional provision or provisions forbidding any law
ever to be passed which might abridge the ``privacy'' of individuals.
But there is not. There are, of course, guarantees in certain specific
constitutional provisions which are designed in part to protect privacy
at certain times and places with respect to certain activities. Such,
for example, is the Fourth Amendment's guarantee against ``unreasonable
searches and seizures.'' But I think it belittles that Amendment to talk
about it as though it protects nothing but ``privacy.'' To treat it that
way is to give it a niggardly interpretation, not the kind of liberal
reading I think any Bill of Rights provision should be given. The
average man would very likely not have his feelings soothed any more by
having his property seized openly than by having it seized privately and
by stealth. He simply wants his property left alone. And a person can be
just as much, if not more, irritated, annoyed and injured by an
unceremonious public arrest by a policeman as he is by a seizure in the
privacy of his office or home.

One of the most effective ways of diluting or expanding a
constitutionally guaranteed right is to substitute for the crucial word
or words of a constitutional guarantee another word or words, more or
less flexible and more or less restricted in meaning. This fact is well
illustrated by the use of the term ``right of privacy'' as a
comprehensive substitute for the Fourth Amendment's guarantee against
``unreasonable searches and seizures.'' ``Privacy'' is a broad, abstract
and ambiguous concept which can easily be shrunken in meaning but which
can also, on the other hand, easily be interpreted as a constitutional
ban against many things other than searches and seizures. I have
expressed the view many times that First Amendment freedoms, for
example, have suffered from a failure of the courts to stick to the
simple language of the First Amendment in construing it, instead of
invoking multitudes of words substituted for those the Framers used. See
e. g., New York Times Co.~v. Sullivan concurring opinion); cases
collected in City of El Paso v. Simmons, n.~1 (dissenting opinion);
Black, The Bill of Rights, 35 N. Y. U. L. Rev.~865. For these reasons I
get nowhere in this case by talk about a constitutional ``right of
privacy'' as an emanation from one or more constitutional provisions I
like my privacy as well as the next one, but I am nevertheless compelled
to admit that government has a right to invade it unless prohibited by
some specific constitutional provision. For these reasons I cannot agree
with the Court's judgment and the reasons it gives for holding this
Connecticut law unconstitutional.

This brings me to the arguments made by my Brothers HARLAN, WHITE and
GOLDBERG for invalidating the Connecticut law. Brothers HARLAN2 and
WHITE would invalidate it by reliance on the Due Process Clause of the
Fourteenth Amendment, but Brother GOLDBERG, while agreeing with Brother
HARLAN, relies also on the Ninth Amendment. I have no doubt that the
Connecticut law could be applied in such a way as to abridge freedom of
speech and press and therefore violate the First and Fourteenth
Amendments. My disagreement with the Court's opinion holding that there
is such a violation here is a narrow one, relating to the application of
the First Amendment to the facts and circumstances of this particular
case. But my disagreement with Brothers HARLAN, WHITE and GOLDBERG is
more basic. I think that if properly construed neither the Due Process
Clause nor the Ninth Amendment, nor both together, could under any
circumstances be a proper basis for invalidating the Connecticut law. I
discuss the due process and Ninth Amendment arguments together because
on analysis they turn out to be the same thing---merely using different
words to claim for this Court and the federal judiciary power to
invalidate any legislative act which the judges find irrational,
unreasonable or offensive.

The due process argument which my Brothers HARLAN and WHITE adopt here
is based, as their opinions indicate, on the premise that this Court is
vested with power to invalidate all state laws that it considers to be
arbitrary, capricious, unreasonable, or oppressive, or on this Court's
belief that a particular state law under scrutiny has no ``rational or
justifying'' purpose, or is offensive to a ``sense of fairness and
justice.'' If these formulas based on ``natural justice,'' or others
which mean the same thing,4 are to prevail, they require judges to
determine what is or is not constitutional on the basis of their own
appraisal of what laws are unwise or unnecessary. The power to make such
decisions is of course that of a legislative body. Surely it has to be
admitted that no provision of the Constitution specifically gives such
blanket power to courts to exercise such a supervisory veto over the
wisdom and value of legislative policies and to hold unconstitutional
those laws which they believe unwise or dangerous. I readily admit that
no legislative body, state or national, should pass laws that can justly
be given any of the invidious labels invoked as constitutional excuses
to strike down state laws. But perhaps it is not too much to say that no
legislative body ever does pass laws without believing that they will
accomplish a sane, rational, wise and justifiable purpose. While I
completely subscribe to the holding of Marbury v. Madison, 1 Cranch 137,
and subsequent cases, that our Court has constitutional power to strike
down statutes, state or federal, that violate commands of the Federal
Constitution, I do not believe that we are granted power by the Due
Process Clause or any other constitutional provision or provisions to
measure constitutionality by our belief that legislation is arbitrary,
capricious or unreasonable, or accomplishes no justifiable purpose, or
is offensive to our own notions of ``civilized standards of conduct.''
Such an appraisal of the wisdom of legislation is an attribute of the
power to make laws, not of the power to interpret them. The use by
federal courts of such a formula or doctrine or whatnot to veto federal
or state laws simply takes away from Congress and States the power to
make laws based on their own judgment of fairness and wisdom and
transfers that power to this Court for ultimate determination ---a power
which was specifically denied to federal courts by the convention that
framed the Constitution.

Of the cases on which my Brothers WHITE and GOLDBERG rely so heavily,
undoubtedly the reasoning of two of them supports their result here---as
would that of a number of others which they do not bother to name, e.
g., Lochner v. New York, Coppage v. Kansas, 236 U. S. 1, Jay Burns
Baking Co.~v. Bryan, and Adkins v. Children's Hospital. The two they do
cite and quote from, Meyer v. Nebraska, and Pierce v. Society of
Sisters, were both decided in opinions by Mr.~Justice McReynolds which
elaborated the same natural law due process philosophy found in Lochner
v. New Yorkone of the cases on which he relied in Meyer, along with such
other long-discredited decisions as, e. g., Adams v. Tanner, and Adkins
v. Children's Hospital. Meyer held unconstitutional, as an ``arbitrary''
and unreasonable interference with the right of a teacher to carry on
his occupation and of parents to hire him, a state law forbidding the
teaching of modern foreign languages to young children in the schools
And in Pierce,relying principally on Meyer, Mr.~Justice McReynolds said
that a state law requiring that all children attend public schools
interfered unconstitutionally with the property rights of private school
corporations because it was an ``arbitrary, unreasonable and unlawful
interference'' which threatened ``destruction of their business and
property.'' 268 U. S.. Without expressing an opinion as to whether
either of those cases reached a correct result in light of our later
decisions applying the First Amendment to the State through the
Fourteenth,8 I merely point out that the reasoning stated in Meyer and
Pierce was the same natural law due process philosophy which many later
opinions repudiated, and which I cannot accept. Brothers WHITE and
GOLDBERG also cite other cases, such as NAACP v. Button, Shelton v.
Tucker, and Schneider v. State,308 U. S. 147, which held that States in
regulating conduct could not, consistently with the First Amendment as
applied to them by the Fourteenth, pass unnecessarily broad laws which
might indirectly infringe on First Amendment freedoms See Brotherhood of
Railroad Trainmen v. Virginia ex rel. Virginia State Bar Brothers WHITE
and GOLDBERG now apparently would start from this requirement that laws
be narrowly drafted so as not to curtail free speech and assembly, and
extend it limitlessly to require States to justify any law restricting
``liberty'' as my Brethren define ``liberty.'' This would mean at the
very least, I suppose, that every state criminal statute--- since it
must inevitably curtail ``liberty'' to some extent--- would be suspect,
and would have to be justified to this Court.

My Brother GOLDBERG has adopted the recent discovery12 that the Ninth
Amendment as well as the Due Process Clause can be used by this Court as
authority to strike down all state legislation which this Court thinks
violates ``fundamental principles of liberty and justice,'' or is
contrary to the ``traditions and {[}collective{]} conscience of our
people.'' He also states, without proof satisfactory to me, that in
making decisions on this basis judges will not consider ``their personal
and private notions.'' One may ask how they can avoid considering them.
Our Court certainly has no machinery with which to take a Gallup Poll
And the scientific miracles of this age have not yet produced a gadget
which the Court can use to determine what traditions are rooted in the
``{[}collective{]} conscience of our people.'' Moreover, one would
certainly have to look far beyond the language of the Ninth Amendment14
to find that the Framers vested in this Court any such awesome veto
powers over lawmaking, either by the States or by the Congress. Nor does
anything in the history of the Amendment offer any support for such a
shocking doctrine. The whole history of the adoption of the Constitution
and Bill of Rights points the other way, and the very material quoted by
my Bother GOLDBERG shows that the Ninth Amendment was intended to
protect against the idea that ``by enumerating particular exceptions to
the grant of power'' to the Federal Government, ``those rights which
were not singled out, were intended to be assigned into the hands of the
General Government {[}the United States{]}, and were consequently
insecure.''15 That Amendment was passed, not to broaden the powers of
this Court or any other department of ``the General Government,'' but,
as every student of history knows, to assure the people that the
Constitution in all its provisions was intended to limit the Federal
Government to the powers granted expressly or by necessary implication.
If any broad, unlimited power to hold laws unconstitutional because they
offend what this Court conceives to be the ``{[}collective{]} conscience
of our people'' is vested in this Court by the Ninth Amendment, the
Fourteenth Amendment, or any other provision of the Constitution, it was
not given by the Framers, but rather has been bestowed on the Court by
the Court. This fact is perhaps responsible for the peculiar phenomenon
that for a period of a century and a half no serious suggestion was ever
made that the Ninth Amendment, enacted to protect state powers against
federal invasion, could be used as a weapon of federal power to prevent
state legislatures from passing laws they consider appropriate to govern
local affairs. Use of any such broad, unbounded judicial authority would
make of this Court's members a day-to-day constitutional convention.

I repeat so as not to be misunderstood that this Court does have power,
which it should exercise, to hold laws unconstitutional where they are
forbidden by the Federal Constitution. My point is that there is no
provision of the Constitution which either expressly or impliedly vests
power in this Court to sit as a supervisory agency over acts of duly
constituted legislative bodies and set aside their laws because of the
Court's belief that the legislative policies adopted are unreasonable,
unwise, arbitrary, capricious or irrational. The adoption of such a
loose, flexible, uncontrolled standard for holding laws
unconstitutional, if ever it is finally achieved, will amount to a great
unconstitutional shift of power to the courts which I believe and am
constrained to say will be bad for the courts and worse for the country.
Subjecting federal and state laws to such an unrestrained and
unrestrainable judicial control as to the wisdom of legislative
enactments would, I fear, jeopardize the separation of governmental
powers that the Framers set up and at the same time threaten to take
away much of the power of States to govern themselves which the
Constitution plainly intended them to have.

I realize that many good and able men have eloquently spoken and
written, sometimes in rhapsodical strains, about the duty of this Court
to keep the Constitution in tune with the times. The idea is that the
Constitution must be changed from time to time and that this Court is
charged with a duty to make those changes. For myself, I must with all
deference reject that philosophy. The Constitution makers knew the need
for change and provided for it. Amendments suggested by the people's
elected representatives can be submitted to the people or their selected
agents for ratification. That method of change was good for our Fathers,
and being somewhat old-fashioned I must add it is good enough for me.
And so, I cannot rely on the Due Process Clause or the Ninth Amendment
or any mysterious and uncertain natural law concept as a reason for
striking down this state law. The Due Process Clause with an ``arbitrary
and capricious'' or ``shocking to the conscience'' formula was liberally
used by this Court to strike down economic legislation in the early
decades of this century, threatening, many people thought, the
tranquility and stability of the Nation. See, e. g., Lochner v. New
York,198 U. S. 45. That formula, based on subjective considerations of
``natural justice,'' is no less dangerous when used to enforce this
Court's views about personal rights than those about economic rights. I
had thought that we had laid that formula, as a means for striking down
state legislation, to rest once and for all in cases like West Coast
Hotel Co.~v. Parrish; Olsen v. Nebraska ex rel. Western Reference \&
Bond Assn., and many other opinions See also Lochner v. New YorkHolmes,
J., dissenting).

In Ferguson v. Skrupa, this Court two years ago said in an opinion
joined by all the Justices but one that

``The doctrine that prevailed in Lochner, Coppage, Adkins, Burns, and
like cases---that due process authorizes courts to hold laws
unconstitutional when they believe the legislature has acted
unwisely---has long since been discarded. We have returned to the
original constitutional proposition that courts do not substitute their
social and economic beliefs for the judgment of legislative bodies, who
are elected to pass laws.''

And only six weeks ago, without even bothering to hear argument, this
Court overruled Tyson \& Brother v. Banton, which had held state laws
regulating ticket brokers to be a denial of due process of law Gold v.
DiCarlo. I find April's holding hard to square with what my concurring
Brethren urge today. They would reinstate the Lochner, Coppage, Adkins,
Burns line of cases, cases from which this Court recoiled after the
1930's, and which had been I thought totally discredited until now.
Apparently my Brethren have less quarrel with state economic regulations
than former Justices of their persuasion had. But any limitation upon
their using the natural law due process philosophy to strike down any
state law, dealing with any activity whatever, will obviously be only
self-imposed.

In 1798, when this Court was asked to hold another Connecticut law
unconstitutional, Justice Iredell said:

``{[}I{]}t has been the policy of all the American states, which have,
individually, framed their state constitutions since the revolution, and
of the people of the United States, when they framed the Federal
Constitution, to define with precision the objects of the legislative
power, and to restrain its exercise within marked and settled
boundaries. If any act of Congress, or of the Legislature of a state,
violates those constitutional provisions, it is unquestionably void;
though, I admit, that as the authority to declare it void is of a
delicate and awful nature, the Court will never resort to that
authority, but in a clear and urgent case. If, on the other hand, the
Legislature of the Union, or the Legislature of any member of the Union,
shall pass a law, within the general scope of their constitutional
power, the Court cannot pronounce it to be void, merely because it is,
in their judgment, contrary to the principles of natural justice. The
ideas of natural justice are regulated by no fixed standard: the ablest
and the purest men have differed upon the subject; and all that the
Court could properly say, in such an event, would be, that the
Legislature (possessed of an equal right of opinion) had passed an act
which, in the opinion of the judges, was inconsistent with the abstract
principles of natural justice.'' Calder v. Bull, 3 Dall. 386, 399
(emphasis in original).

I would adhere to that constitutional philosophy in passing on this
Connecticut law today. I am not persuaded to deviate from the view which
I stated in 1947 in Adamson v. California (dissenting opinion):

``Since Marbury v. Madison, 1 Cranch 137, was decided, the practice has
been firmly established, for better or worse, that courts can strike
down legislative enactments which violate the Constitution. This
process, of course, involves interpretation, and since words can have
many meanings, interpretation obviously may result in contraction or
extension of the original purpose of a constitutional provision, thereby
affecting policy. But to pass upon the constitutionality of statutes by
looking to the particular standards enumerated in the Bill of Rights and
other parts of the Constitution is one thing; to invalidate statutes
because of application of'natural law' deemed to be above and undefined
by the Constitution is another.'In the one instance, courts proceeding
within clearly marked constitutional boundaries seek to execute policies
written into the Constitution; in the other, they roam at will in the
limitless area of their own beliefs as to reasonableness and actually
select policies, a responsibility which the Constitution entrusts to the
legislative representatives of the people.' Federal Power Commission v.
Pipeline Co., 601, n.~4.''21(Footnotes omitted.)

The late Judge Learned Hand, after emphasizing his view that judges
should not use the due process formula suggested in the concurring
opinions today or any other formula like it to invalidate legislation
offensive to their ``personal preferences,'' made the statement, with
which I fully agree, that:

``For myself it would be most irksome to be ruled by a bevy of Platonic
Guardians, even if I knew how to choose them, which I assuredly do
not.''

So far as I am concerned, Connecticut's law as applied here is not
forbidden by any provision of the Federal Constitution as that
Constitution was written, and I would therefore affirm.

\emph{From the footnotes:}

A collection of the catchwords and catch phrases invoked by judges who
would strike down under the Fourteenth Amendment laws which offend their
notions of natural justice would fill many pages. Thus it has been said
that this Court can forbid state action which ``shocks the conscience,''
Rochin v. California, sufficiently to ``shock itself into the protective
arms of the Constitution,'' Irvine v. California concurring opinion). It
has been urged that States may not run counter to the ``decencies of
civilized conduct,'' Rochin, or ``some principle of justice so rooted in
the traditions and conscience of our people as to be ranked as
fundamental,'' Snyder v. Massachusetts, or to ``those canons of decency
and fairness which express the notions of justice of English-speaking
peoples,'' Malinski v. New York concurring opinion), or to ``the
community's sense of fair play and decency,'' Rochin. It has been said
that we must decide whether a state law is ``fair, reasonable and
appropriate,'' or is rather ``an unreasonable, unnecessary and arbitrary
interference with the right of the individual to his personal liberty or
to enter into . contracts,'' Lochner v. New York. States, under this
philosophy, cannot act in conflict with ``deeply rooted feelings of the
community,'' Haley v. Ohio separate opinion), or with ``fundamental
notions of fairness and justice,'' See also, e. g., Wolf v. Colorado
(``rights basic to our free society''); Hebert v. Louisiana
(``fundamental principles of liberty and justice''); Adkins v.
Children's Hospital (``arbitrary restraint of . liberties''); Betts v.
Brady (``denial of fundamental fairness, shocking to the universal sense
of justice''); Poe v. Ullman (dissenting opinion) (``intolerable and
unjustifiable''). Perhaps the clearest, frankest and briefest
explanation of how this due process approach works is the statement in
another case handed down today that this Court is to invoke the Due
Process Clause to strike down state procedures or laws which it can
``not tolerate.'' Linkletter v. Walker, post, p.~618.

\textbf{MR. JUSTICE STEWART, whom MR. JUSTICE BLACK joins, dissenting.}
Since 1879 Connecticut has had on its books a law which forbids the use
of contraceptives by anyone. I think this is an uncommonly silly law. As
a practical matter, the law is obviously unenforceable, except in the
oblique context of the present case. As a philosophical matter, I
believe the use of contraceptives in the relationship of marriage should
be left to personal and private choice, based upon each individual's
moral, ethical, and religious beliefs. As a matter of social policy, I
think professional counsel about methods of birth control should be
available to all, so that each individual's choice can be meaningfully
made. But we are not asked in this case to say whether we think this law
is unwise, or even asinine. We are asked to hold that it violates the
United States Constitution. And that I cannot do.

In the course of its opinion the Court refers to no less than six
Amendments to the Constitution: the First, the Third, the Fourth, the
Fifth, the Ninth, and the Fourteenth. But the Court does not say which
of these Amendments, if any, it thinks is infringed by this Connecticut
law.

We are told that the Due Process Clause of the Fourteenth Amendment is
not, as such, the ``guide'' in this case. With that much I agree. There
is no claim that this law, duly enacted by the Connecticut Legislature
is unconstitutionally vague. There is no claim that the appellants were
denied any of the elements of procedural due process at their trial, so
as to make their convictions constitutionally invalid. And, as the Court
says, the day has long passed since the Due Process Clause was regarded
as a proper instrument for determining ``the wisdom, need, and
propriety'' of state laws. Compare Lochner v. New York, with Ferguson v.
Skrupa. My Brothers HARLAN and WHITE to the contrary, ``{[}w{]}e have
returned to the original constitutional proposition that courts do not
substitute their social and economic beliefs for the judgment of
legislative bodies, who are elected to pass laws.'' Ferguson v. Skrupa.

As to the First, Third, Fourth, and Fifth Amendments, I can find nothing
in any of them to invalidate this Connecticut law, even assuming that
all those Amendments are fully applicable against the States It has not
even been argued that this is a law ``respecting an establishment of
religion, or prohibiting the free exercise thereof.'' And surely, unless
the solemn process of constitutional adjudication is to descend to the
level of a play on words, there is not involved here any abridgment of
``the freedom of speech, or of the press; or the right of the people
peaceably to assemble, and to petition the Government for a redress of
grievances.'' No soldier has been quartered in any house There has been
no search, and no seizure Nobody has been compelled to be a witness
against himself.

The Court also quotes the Ninth Amendment, and my Brother GOLDBERG'S
concurring opinion relies heavily upon it. But to say that the Ninth
Amendment has anything to do with this case is to turn somersaults with
history. The Ninth Amendment, like its companion the Tenth, which this
Court held ``states but a truism that all is retained which has not been
surrendered,'' United States v. Darby,312 U. S. 100, 124, was framed by
James Madison and adopted by the States simply to make clear that the
adoption of the Bill of Rights did not alter the plan that the Federal
Government was to be a government of express and limited powers, and
that all rights and powers not delegated to it were retained by the
people and the individual States. Until today no member of this Court
has ever suggested that the Ninth Amendment meant anything else, and the
idea that a federal court could ever use the Ninth Amendment to annual a
law passed by the elected representatives of the people of the State of
Connecticut would have caused James Madison no little wonder.

What provision of the Constitution, then, does make this state law
invalid? The Court says it is the right of privacy ``created by several
fundamental constitutional guarantees.'' With all deference, I can find
no such general right of privacy in the Bill of Rights, in any other
part of the Constitution, or in any case ever before decided by this
Court.

At the oral argument in this case we were told that the Connecticut law
does not ``conform to current community standards.'' But it is not the
function of this Court to decide cases on the basis of community
standards. We are here to decide cases ``agreeably to the Constitution
and laws of the United States.'' It is the essence of judicial duty to
subordinate our own personal views, our own ideas of what legislation is
wise and what is not. If, as I should surely hope, the law before us
does not reflect the standards of the people of Connecticut, the people
of Connecticut can freely exercise their true Ninth and Tenth Amendment
rights to persuade their elected representatives to repeal it. That is
the constitutional way to take this law off the books.

The Amendments in question were, as everyone knows, originally adopted
as limitations upon the power of the newly created Federal Government,
not as limitation upon the powers of the individual States. But the
Court has held that many of the provisions of the first eight amendments
are fully embraced by the Fourteenth Amendment as limitations upon state
action, and some members of the Court have held the view that the
adoption of the Fourteenth Amendment made every provision of the first
eight amendments fully applicable against the States. See Adamson v.
California (dissenting opinion of MR. JUSTICE BLACK).

U. S. Constitution, Amendment I. To be sure, the injunction contained in
the Connecticut statute coincides with the doctrine of certain religious
faiths. But if that were enough to invalidate a law under the provisions
of the First Amendment relating to religion, then most criminal laws
would be invalidated. See, e. g., the Ten Commandments. The Bible,
Exodus 20:2-17 (King James).

If all the appellants had done was to advise people that they thought
the use of contraceptives was desirable, or even to counsel their use,
the appellants would, of course, have a substantial First Amendment
claim. But their activities went far beyond mere advocacy. They
prescribed specific contraceptive devices and furnished patients with
the prescribed contraceptive materials.

Cases like Shelton v. Tuckerand Bates v. Little Rock, relied upon in the
concurring opinions today, dealt with true First Amendment rights of
association and are wholly inapposite here. See also, e. g., NAACP v.
Alabama; Edwards v. South Carolina. Our decision in McLaughlin v.
Florida, is equally far afield. That case held invalid under the Equal
Protection Clause, a state criminal law which discriminated against
Negroes. The Court does not say how far the new constitutional right of
privacy announced today extends. See, e. g., Mueller, Legal Regulation
of Sexual Conduct; Ploscowe, Sex and the Law. I suppose, however, that
even after today a State can constitutionally still punish at least some
offenses which are not committed in public.

See Reynolds v. Sims. The Connecticut House of Representatives recently
passed a bill (House Bill No.~2462) repealing the birth control law. The
State Senate has apparently not yet acted on the measure, and today is
relieved of that responsibility by the Court. New Haven Journal-Courier,
Wed., May 19, 1965, p.~1, col.~4, and p.~13, col.~7. 2

\hypertarget{washington-v.-glucksberg}{%
\subsubsection{Washington v.
Glucksberg}\label{washington-v.-glucksberg}}

521 U.S. 702 (1997)

\textbf{Chief Justice REHNQUIST delivered the opinion of the Court.}

The question presented in this case is whether Washington's prohibition
against ``caus{[}ing{]}'' or ``aid{[}ing{]}'' a suicide offends the
Fourteenth Amendment to the United States Constitution. We hold that it
does not.

It has always been a crime to assist a suicide in the State of
Washington. In 1854, Washington's first Territorial Legislature outlawed
``assisting another in the commission of self-murder.'' Today,
Washington law provides: ``A person is guilty of promoting a suicide
attempt when he knowingly causes or aids another person to attempt
suicide.'' Wash. Rev.Code 9A (1) (1994). ``Promoting a suicide attempt''
is a felony, punishable by up to five years' imprisonment and up to a
\$10,000 fine. ---§§9A (2) and 9A (1)(c). At the same time, Washington's
Natural Death Act, enacted in 1979, states that the ``withholding or
withdrawal of life-sustaining treatment'' at a patient's direction
``shall not, for any purpose, constitute a suicide.'' Wash. Rev.Code §70
(1).

Petitioners in this case are the State of Washington and its Attorney
General. Respondents Harold Glucksberg, M. D., Abigail Halperin, M. D.,
Thomas A. Preston, M. D., and Peter Shalit, M. D., are physicians who
practice in Washington. These doctors occasionally treat terminally ill,
suffering patients, and declare that they would assist these patients in
ending their lives if not for Washington's assisted-suicide ban. 3 In
January 1994, respondents, along with three gravely ill, pseudonymous
plaintiffs who have since died and Compassion in Dying, a nonprofit
organization that counsels people considering physician-assisted
suicide, sued in the United States District Court, seeking a declaration
that Wash Rev.Code 9A (1) (1994) is, on its face, unconstitutional.

The plaintiffs asserted ``the existence of a liberty interest protected
by the Fourteenth Amendment which extends to a personal choice by a
mentally competent, terminally ill adult to commit physician-assisted
suicide.'' Relying primarily on Planned Parenthood v. Caseyand Cruzan v.
Director, Missouri Dept. of Healththe District Court agreed, 850
F.Supp., and concluded that Washington's assisted-suicide ban is
unconstitutional because it ``places an undue burden on the exercise of
{[}that{]} constitutionally protected liberty interest.'' 5 The District
Court also decided that the Washington statute violated the Equal
Protection Clause's requirement that "``all persons similarly situated .
be treated alike.''' (quoting Cleburne v. Cleburne Living Center, Inc.,
).

A panel of the Court of Appeals for the Ninth Circuit reversed,
emphasizing that " {[}i{]}n the two hundred and five years of our
existence no constitutional right to aid in killing oneself has ever
been asserted and upheld by a court of final jurisdiction." Compassion
in Dying v. Washington, The Ninth Circuit reheard the case en banc,
reversed the panel's decision, and affirmed the District Court.
Compassion in Dying v. Washington, Like the District Court, the en banc
Court of Appeals emphasized our Casey and Cruzan decisions. 79 F d.~The
court also discussed what it described as ``historical'' and ``current
societal attitudes'' toward suicide and assisted suicide, -812, and
concluded that ``the Constitution encompasses a due process liberty
interest in controlling the time and manner of one's death-that there
is, in short, a constitutionally-recognized''right to die.``' After''
{[}w{]}eighing and then balancing" this interest against Washington's
various interests, the court held that the State's assisted-suicide ban
was unconstitutional ``as applied to terminally ill competent adults who
wish to hasten their deaths with medication prescribed by their
physicians.'' 837. 6 The court did not reach the District Court's
equal-protection holding. 7 We granted certiorari, 519 U.S. ----, and
now reverse.

We begin, as we do in all due-process cases, by examining our Nation's
history, legal traditions, and practices. See, e.g., Casey, 112 S.Ct.;
Cruzan, 110 S.Ct.; Moore v. East Cleveland, (plurality opinion) (noting
importance of ``careful''respect for the teachings of history``\,`). In
almost every State-indeed, in almost every western democracy-it is a
crime to assist a suicide. 8 The States' assisted-suicide bans are not
innovations. Rather, they are longstanding expressions of the States'
commitment to the protection and preservation of all human life. Cruzan,
110 S.Ct. ('' {[}T{]}he States-indeed, all civilized nations-demonstrate
their commitment to life by treating homicide as a serious crime.
Moreover, the majority of States in this country have laws imposing
criminal penalties on one who assists another to commit suicide``); see
Stanford v. Kentucky, ('' {[}T{]}he primary and most reliable indication
of {[}a national{]} consensus is . the pattern of enacted laws").
Indeed, opposition to and condemnation of suicide-and, therefore, of
assisting suicide-are consistent and enduring themes of our
philosophical, legal, and cultural heritages. See generally, Marzen,
O'Dowd, Crone \& Balch, Suicide: A Constitutional Right?, 24 Duquesne
L.Rev. 1, 17-56 (1985) (hereinafter Marzen); New York State Task Force
on Life and the Law, When Death is Sought: Assisted Suicide and
Euthanasia in the Medical Context 77-82 (May 1994) (hereinafter New York
Task Force).

More specifically, for over 700 years, the Anglo-American common-law
tradition has punished or otherwise disapproved of both suicide and
assisting suicide. In the 13th century, Henry de Bracton, one of the
first legal-treatise writers, observed that " {[}j{]}ust as a man may
commit felony by slaying another so may he do so by slaying himself." 2
Bracton on Laws and Customs of England 423 (f ) (G. Woodbine ed., S.
Thorne transl., 1968). The real and personal property of one who killed
himself to avoid conviction and punishment for a crime were forfeit to
the king; however, thought Bracton, ``if a man slays himself in
weariness of life or because he is unwilling to endure further bodily
pain . {[}only{]} his movable goods {[}were{]} confiscated.'' -424 (f ).
Thus, " {[}t{]}he principle that suicide of a sane person, for whatever
reason, was a punishable felony was . introduced into English common
law.``10 Centuries later, Sir William Blackstone, whose Commentaries on
the Laws of England not only provided a definitive summary of the common
law but was also a primary legal authority for 18th and 19th century
American lawyers, referred to suicide as''self-murder" and ``the
pretended heroism, but real cowardice, of the Stoic philosophers, who
destroyed themselves to avoid those ills which they had not the
fortitude to endure . .'' 4 W. Blackstone, Commentaries Blackstone
emphasized that ``the law has . ranked {[}suicide{]} among the highest
crimes,'' ibid, although, anticipating later developments, he conceded
that the harsh and shameful punishments imposed for suicide
``borde{[}r{]} a little upon severity.''

For the most part, the early American colonies adopted the common-law
approach. For example, the legislators of the Providence Plantations,
which would later become Rhode Island, declared, in 1647, that "
{[}s{]}elf-murder is by all agreed to be the most unnatural, and it is
by this present Assembly declared, to be that, wherein he that doth it,
kills himself out of a premeditated hatred against his own life or other
humor: . his goods and chattels are the king's custom, but not his debts
nor lands; but in case he be an infant, a lunatic, mad or distracted
man, he forfeits nothing." The Earliest Acts and Laws of the Colony of
Rhode Island and Providence Plantations 1647-1719, p.~19 (J. Cushing ed
). Virginia also required ignominious burial for suicides, and their
estates were forfeit to the crown. A. Scott, Criminal Law in Colonial
Virginia 108, and n.~93, 198, and n.~15 (1930).

Over time, however, the American colonies abolished these harsh
common-law penalties. William Penn abandoned the criminal-forfeiture
sanction in Pennsylvania in 1701, and the other colonies (and later, the
other States) eventually followed this example. Cruzan, 110 S.Ct.
(SCALIA, J., concurring). Zephaniah Swift, who would later become Chief
Justice of Connecticut, wrote in 1796 that

" {[}t{]}here can be no act more contemptible, than to attempt to punish
an offender for a crime, by exercising a mean act of revenge upon
lifeless clay, that is insensible of the punishment. There can be no
greater cruelty, than the inflicting {[}of{]} a punishment, as the
forfeiture of goods, which must fall solely on the innocent offspring of
the offender . . -{[}Suicide{]} is so abhorrent to the feelings of
mankind, and that strong love of life which is implanted in the human
heart, that it cannot be so frequently committed, as to become dangerous
to society. There can of course be no necessity of any punishment." 2 Z.
Swift, A System of the Laws of the State of Connecticut 304 (1796). This
statement makes it clear, however, that the movement away from the
common law's harsh sanctions did not represent an acceptance of suicide;
rather, as Chief Justice Swift observed, this change reflected the
growing consensus that it was unfair to punish the suicide's family for
his wrongdoing. Cruzan, 110 S.Ct. (SCALIA, J., concurring). Nonetheless,
although States moved away from Blackstone's treatment of suicide,
courts continued to condemn it as a grave public wrong. See, e.g.,
Bigelow v. Berkshire Life Ins. Co., (suicide is ``an act of criminal
self-destruction''); Von Holden v. Chapman, 87 A.D d 66, 70N.Y.S d 623,
626-627 (1982); Blackwood v. Jones, 111 Fla. 528, 532, 149 So. 600, 601
(1933) (``No sophistry is tolerated . which seek{[}s{]} to justify
self-destruction as commendable or even a matter of personal right'').

That suicide remained a grievous, though nonfelonious, wrong is
confirmed by the fact that colonial and early state legislatures and
courts did not retreat from prohibiting assisting suicide. Swift, in his
early 19th century treatise on the laws of Connecticut, stated that "
{[}i{]}f one counsels another to commit suicide, and the other by reason
of the advice kills himself, the advisor is guilty of murder as
principal." 2 Z. Swift, A Digest of the Laws of the State of Connecticut
270 (1823). This was the well established common-law view, see In re
Joseph G., 34 Cal d 429, 434Cal.Rptr. 163, 166, 667 P d 1176, 1179
(1983); Commonwealth v. Mink, 123 Mass. 422, 428 (1877) ("``Now if the
murder of one's self is felony, the accessory is equally guilty as if he
had aided and abetted in the murder''`) (quoting Chief Justice Parker's
charge to the jury in Commonwealth v. Bowen, 13 Mass. 356 (1816)), as
was the similar principle that the consent of a homicide victim is
``wholly immaterial to the guilt of the person who cause{[}d{]} {[}his
death{]},'' 3 J. Stephen, A History of the Criminal Law of England 16
(1883); see 1 F. Wharton, Criminal Law §§451-452 (9th ed.~1885); Martin
v. Commonwealth, 184 Va. 1009, 1018S.E d 43, 47 (1946) ("``The right to
life and to personal security is not only sacred in the estimation of
the common law, but it is inalienable''\,'). And the prohibitions
against assisting suicide never contained exceptions for those who were
near death. Rather, " {[}t{]}he life of those to whom life ha{[}d{]}
become a burden-of those who {[}were{]} hopelessly diseased or fatally
wounded-nay, even the lives of criminals condemned to death, {[}were{]}
under the protection of law, equally as the lives of those who
{[}were{]} in the full tide of life's enjoyment, and anxious to continue
to live." Blackburn v. State, 23 Ohio St.~146, 163 (1872); see Bowen
(prisoner who persuaded another to commit suicide could be tried for
murder, even though victim was scheduled shortly to be executed).

The earliest American statute explicitly to outlaw assisting suicide was
enacted in New York in 1828, Act of Dec.~10, 1828, ch.~20, §4, 1828 N.Y.
Laws 19 (codified N.Y.Rev.Stat. pt.~4, ch.~1, tit. 2, art. 1, §7, p.~661
(1829)), and many of the new States and Territories followed New York's
example. Marzen 73-74. Between 1857 and 1865, a New York commission led
by Dudley Field drafted a criminal code that prohibited ``aiding'' a
suicide and, specifically, ``furnish{[}ing{]} another person with any
deadly weapon or poisonous drug, knowing that such person intends to use
such weapon or drug in taking his own life.'' -77. By the time the
Fourteenth Amendment was ratified, it was a crime in most States to
assist a suicide. See Cruzan, 110 S.Ct. (SCALIA, J., concurring). The
Field Penal Code was adopted in the Dakota Territory in 1877, in New
York in 1881, and its language served as a model for several other
western States' statutes in the late 19th and early 20th centuries.
Marzen 76, 212-213. California, for example, codified its
assisted-suicide prohibition in 1874, using language similar to the
Field Code's. 11 In this century, the Model Penal Code also prohibited
``aiding'' suicide, prompting many States to enact or revise their
assisted-suicide bans. 12 The Code's drafters observed that ``the
interests in the sanctity of life that are represented by the criminal
homicide laws are threatened by one who expresses a willingness to
participate in taking the life of another, even though the act may be
accomplished with the consent, or at the request, of the suicide
victim.'' American Law Institute, Model Penal Code §210 Comment 5,
p.~100 (Official Draft and Revised Comments 1980).

Though deeply rooted, the States' assisted-suicide bans have in recent
years been reexamined and, generally, reaffirmed. Because of advances in
medicine and technology, Americans today are increasingly likely to die
in institutions, from chronic illnesses. President's Comm'n for the
Study of Ethical Problems in Medicine and Biomedical and Behavioral
Research, Deciding to Forego Life-Sustaining Treatment 16-18 (1983).
Public concern and democratic action are therefore sharply focused on
how best to protect dignity and independence at the end of life, with
the result that there have been many significant changes in state laws
and in the attitudes these laws reflect. Many States, for example, now
permit ``living wills,'' surrogate health-care decisionmaking, and the
withdrawal or refusal of life-sustaining medical treatment. See Vacco v.
Quill, --- U.S. ----, ---------, ---------, --- L.Ed d ----; 79 F d;
People v. Kevorkian, 447 Mich. 436, 478-480, and nn.~53N.W d 714,
731-732, and nn.~53-56 (1994). At the same time, however, voters and
legislators continue for the most part to reaffirm their States'
prohibitions on assisting suicide.

The Washington statute at issue in this case, Wash. Rev.Code §9A (1994),
was enacted in 1975 as part of a revision of that State's criminal code.
Four years later, Washington passed its Natural Death Act, which
specifically stated that the ``withholding or withdrawal of
life-sustaining treatment . shall not, for any purpose, constitute a
suicide'' and that " {[}n{]}othing in this chapter shall be construed to
condone, authorize, or approve mercy killing . . " Natural Death Act,
1979 Wash. Laws, ch.~112, §§8(1), p.~11 (codified at Wash. Rev.Code §§70
(1), 70 (1994)). In 1991, Washington voters rejected a ballot initiative
which, had it passed, would have permitted a form of physician-assisted
suicide. 13 Washington then added a provision to the Natural Death Act
expressly excluding physician-assisted suicide. 1992 Wash. Laws, ch.~98,
§10; Wash. Rev.Code §70 (1994).

California voters rejected an assisted-suicide initiative similar to
Washington's in 1993. On the other hand, in 1994, voters in Oregon
enacted, also through ballot initiative, that State's ``Death With
Dignity Act,'' which legalized physician-assisted suicide for competent,
terminally ill adults Since the Oregon vote, many proposals to legalize
assisted-suicide have been and continue to be introduced in the States'
legislatures, but none has been enacted. 15 And just last year, Iowa and
Rhode Island joined the overwhelming majority of States explicitly
prohibiting assisted suicide. See Iowa Code Ann. §§707A 707A (Supp );
R.I. Gen.~Laws §§11-60-3 (Supp ). Also, on April 30, 1997, President
Clinton signed the Federal Assisted Suicide Funding Restriction Act of
1997, which prohibits the use of federal funds in support of
physician-assisted suicide. Pub.L. 105Stat. 23 (codified U.S.C. §14401
et seq).

Thus, the States are currently engaged in serious, thoughtful
examinations of physician-assisted suicide and other similar issues. For
example, New York State's Task Force on Life and the Law-an ongoing,
blue-ribbon commission composed of doctors, ethicists, lawyers,
religious leaders, and interested laymen-was convened in 1984 and
commissioned with ``a broad mandate to recommend public policy on issues
raised by medical advances.'' New York Task Force vii. Over the past
decade, the Task Force has recommended laws relating to end-of-life
decisions, surrogate pregnancy, and organ donation. -119. After studying
physician-assisted suicide, however, the Task Force unanimously
concluded that " {[}l{]}egalizing assisted suicide and euthanasia would
pose profound risks to many individuals who are ill and vulnerable . .
-{[}T{]}he potential dangers of this dramatic change in public policy
would outweigh any benefit that might be achieved."

Attitudes toward suicide itself have changed since Bracton, but our laws
have consistently condemned, and continue to prohibit, assisting
suicide. Despite changes in medical technology and notwithstanding an
increased emphasis on the importance of end-of-life decisionmaking, we
have not retreated from this prohibition. Against this backdrop of
history, tradition, and practice, we now turn to respondents'
constitutional claim.

The Due Process Clause guarantees more than fair process, and the
``liberty'' it protects includes more than the absence of physical
restraint. Collins v. Harker Heights, (Due Process Clause ``protects
individual liberty against''certain government actions regardless of the
fairness of the procedures used to implement them``') (quoting Daniels
v. Williams, ). The Clause also provides heightened protection against
government interference with certain fundamental rights and liberty
interests. Reno v. Flores, ; Casey, In a long line of cases, we have
held that, in addition to the specific freedoms protected by the Bill of
Rights, the''liberty" specially protected by the Due Process Clause
includes the rights to marry, Loving v. Virginia; to have children,
Skinner v. Oklahoma ex rel. Williamson; to direct the education and
upbringing of one's children, Meyer v. Nebraska; Pierce v. Society of
Sisters; to marital privacy, Griswold v. Connecticut; to use
contraception, ibid; Eisenstadt v. Baird; to bodily integrity, Rochin v.
Californiaand to abortion, Casey. We have also assumed, and strongly
suggested, that the Due Process Clause protects the traditional right to
refuse unwanted lifesaving medical treatment. Cruzan, But we
``ha{[}ve{]} always been reluctant to expand the concept of substantive
due process because guideposts for responsible decisionmaking in this
unchartered area are scarce and open-ended.'' Collins, By extending
constitutional protection to an asserted right or liberty interest, we,
to a great extent, place the matter outside the arena of public debate
and legislative action. We must therefore ``exercise the utmost care
whenever we are asked to break new ground in this field,'' ibid, lest
the liberty protected by the Due Process Clause be subtly transformed
into the policy preferences of the members of this Court, Moore, 97
S.Ct. (plurality opinion).

Our established method of substantive-due-process analysis has two
primary features: First, we have regularly observed that the Due Process
Clause specially protects those fundamental rights and liberties which
are, objectively, ``deeply rooted in this Nation's history and
tradition,'' 97 S.Ct. (plurality opinion); Snyder v. Massachusetts,
(``so rooted in the traditions and conscience of our people as to be
ranked as fundamental''), and ``implicit in the concept of ordered
liberty,'' such that ``neither liberty nor justice would exist if they
were sacrificed,'' Palko v. Connecticut, 326, Second, we have required
in substantive-due-process cases a ``careful description'' of the
asserted fundamental liberty interest. Flores, 113 S.Ct.; Collins, 112
S.Ct.; Cruzan, Our Nation's history, legal traditions, and practices
thus provide the crucial ``guideposts for responsible decisionmaking,''
Collins, that direct and restrain our exposition of the Due Process
Clause. As we stated recently in Flores, the Fourteenth Amendment
``forbids the government to infringe .''fundamental' liberty interests
at all, no matter what process is provided, unless the infringement is
narrowly tailored to serve a compelling state interest." 507 U.S.,

Justice SOUTER, relying on Justice Harlan's dissenting opinion in Poe v.
Ullman, would largely abandon this restrained methodology, and instead
ask ``whether {[}Washington's{]} statute sets up one of those''arbitrary
impositions' or ``purposeless restraints' at odds with the Due Process
Clause of the Fourteenth Amendment,'' post, at \_\_ (quoting Poe,
(Harlan, J., dissenting)) In our view, however, the development of this
Court's substantive-due-process jurisprudence, described briefly
above\_\_, has been a process whereby the outlines of the ``liberty''
specially protected by the Fourteenth Amendment-never fully clarified,
to be sure, and perhaps not capable of being fully clarified-have at
least been carefully refined by concrete examples involving fundamental
rights found to be deeply rooted in our legal tradition. This approach
tends to rein in the subjective elements that are necessarily present in
due-process judicial review. In addition, by establishing a threshold
requirement-that a challenged state action implicate a fundamental
right-before requiring more than a reasonable relation to a legitimate
state interest to justify the action, it avoids the need for complex
balancing of competing interests in every case.

Turning to the claim at issue here, the Court of Appeals stated that "
{[}p{]}roperly analyzed, the first issue to be resolved is whether there
is a liberty interest in determining the time and manner of one's
death," 79 F d, or, in other words, " {[}i{]}s there a right to die?,"
Similarly, respondents assert a ``liberty to choose how to die'' and a
right to ``control of one's final days,'' Brief for Respondents 7, and
describe the asserted liberty as ``the right to choose a humane,
dignified death,'' and ``the liberty to shape death,'' As noted above,
we have a tradition of carefully formulating the interest at stake in
substantive-due-process cases. For example, although Cruzan is often
described as a ``right to die'' case, see 79 F d; --- U.S., at ----, 117
S.Ct. (STEVENS, J., concurring in judgment) (Cruzan recognized ``the
more specific interest in making decisions about how to confront an
imminent death''), we were, in fact, more precise: we assumed that the
Constitution granted competent persons a ``constitutionally protected
right to refuse lifesaving hydration and nutrition.'' Cruzan, 110 S.Ct.;
110 S.Ct. (O'CONNOR, J., concurring) (" {[}A{]} liberty interest in
refusing unwanted medical treatment may be inferred from our prior
decisions``). The Washington statute at issue in this case
prohibits''aid{[}ing{]} another person to attempt suicide," Wash.
Rev.Code §9A (1) (1994), and, thus, the question before us is whether
the ``liberty'' specially protected by the Due Process Clause includes a
right to commit suicide which itself includes a right to assistance in
doing so.

We now inquire whether this asserted right has any place in our Nation's
traditions. Here, as discussed above\_\_-\_\_, we are confronted with a
consistent and almost universal tradition that has long rejected the
asserted right, and continues explicitly to reject it today, even for
terminally ill, mentally competent adults. To hold for respondents, we
would have to reverse centuries of legal doctrine and practice, and
strike down the considered policy choice of almost every State. See
Jackman v. Rosenbaum Co., (``If a thing has been practiced for two
hundred years by common consent, it will need a strong case for the
Fourteenth Amendment to affect it''); Flores, 113 S.Ct. (``The mere
novelty of such a claim is reason enough to doubt that''substantive due
process' sustains it").

Respondents contend, however, that the liberty interest they assert is
consistent with this Court's substantive-due-process line of cases, if
not with this Nation's history and practice. Pointing to Casey and
Cruzan, respondents read our jurisprudence in this area as reflecting a
general tradition of ``self-sovereignty,'' Brief of Respondents 12, and
as teaching that the ``liberty'' protected by the Due Process Clause
includes ``basic and intimate exercises of personal autonomy,'' ; see
Casey, 112 S.Ct. (``It is a promise of the Constitution that there is a
realm of personal liberty which the government may not enter'').
According to respondents, our liberty jurisprudence, and the broad,
individualistic principles it reflects, protects the ``liberty of
competent, terminally ill adults to make end-of-life decisions free of
undue government interference.'' Brief for Respondents 10. The question
presented in this case, however, is whether the protections of the Due
Process Clause include a right to commit suicide with another's
assistance. With this ``careful description'' of respondents' claim in
mind, we turn to Casey and Cruzan.

In Cruzan, we considered whether Nancy Beth Cruzan, who had been
severely injured in an automobile accident and was in a persistive
vegetative state, ``ha{[}d{]} a right under the United States
Constitution which would require the hospital to withdraw
life-sustaining treatment'' at her parents' request. Cruzan, We began
with the observation that " {[}a{]}t common law, even the touching of
one person by another without consent and without legal justification
was a battery." We then discussed the related rule that ``informed
consent is generally required for medical treatment.'' After reviewing a
long line of relevant state cases, we concluded that ``the common-law
doctrine of informed consent is viewed as generally encompassing the
right of a competent individual to refuse medical treatment.'' Next, we
reviewed our own cases on the subject, and stated that " {[}t{]}he
principle that a competent person has a constitutionally protected
liberty interest in refusing unwanted medical treatment may be inferred
from our prior decisions." Therefore, ``for purposes of {[}that{]} case,
we assume{[}d{]} that the United States Constitution would grant a
competent person a constitutionally protected right to refuse lifesaving
hydration and nutrition.'' 110 S.Ct.; see 110 S.Ct. (O'CONNOR, J.,
concurring). We concluded that, notwithstanding this right, the
Constitution permitted Missouri to require clear and convincing evidence
of an incompetent patient's wishes concerning the withdrawal of
life-sustaining treatment.

Respondents contend that in Cruzan we ``acknowledged that competent,
dying persons have the right to direct the removal of life-sustaining
medical treatment and thus hasten death,'' Brief for Respondents 23, and
that ``the constitutional principle behind recognizing the patient's
liberty to direct the withdrawal of artificial life support applies at
least as strongly to the choice to hasten impending death by consuming
lethal medication,'' Similarly, the Court of Appeals concluded that
``Cruzan, by recognizing a liberty interest that includes the refusal of
artificial provision of life-sustaining food and water, necessarily
recognize{[}d{]} a liberty interest in hastening one's own death.''

The right assumed in Cruzan, however, was not simply deduced from
abstract concepts of personal autonomy. Given the common-law rule that
forced medication was a battery, and the long legal tradition protecting
the decision to refuse unwanted medical treatment, our assumption was
entirely consistent with this Nation's history and constitutional
traditions. The decision to commit suicide with the assistance of
another may be just as personal and profound as the decision to refuse
unwanted medical treatment, but it has never enjoyed similar legal
protection. Indeed, the two acts are widely and reasonably regarded as
quite distinct. See Vacco v. Quill, --- U.S., at ---------, In Cruzan
itself, we recognized that most States outlawed assisted suicide-and
even more do today-and we certainly gave no intimation that the right to
refuse unwanted medical treatment could be somehow transmuted into a
right to assistance in committing suicide.

Respondents also rely on Casey. There, the Court's opinion concluded
that ``the essential holding of Roe v. Wade should be retained and once
again reaffirmed.'' Casey, We held, first, that a woman has a right,
before her fetus is viable, to an abortion ``without undue interference
from the State''; second, that States may restrict post-viability
abortions, so long as exceptions are made to protect a woman's life and
health; and third, that the State has legitimate interests throughout a
pregnancy in protecting the health of the woman and the life of the
unborn child. In reaching this conclusion, the opinion discussed in some
detail this Court's substantive-due-process tradition of interpreting
the Due Process Clause to protect certain fundamental rights and
``personal decisions relating to marriage, procreation, contraception,
family relationships, child rearing, and education,'' and noted that
many of those rights and liberties ``involv{[}e{]} the most intimate and
personal choices a person may make in a lifetime.''

The Court of Appeals, like the District Court, found Casey "``highly
instructive''' and "``almost prescriptive''' for determining "``what
liberty interest may inhere in a terminally ill person's choice to
commit suicide''':

``Like the decision of whether or not to have an abortion, the decision
how and when to die is one of''the most intimate and personal choices a
person may make in a lifetime,' a choice ``central to personal dignity
and autonomy.''' 79 F d.~Similarly, respondents emphasize the statement
in Casey that:

``At the heart of liberty is the right to define one's own concept of
existence, of meaning, of the universe, and of the mystery of human
life. Beliefs about these matters could not define the attributes of
personhood were they formed under compulsion of the State.'' Casey,
Brief for Respondents 12. By choosing this language, the Court's opinion
in Casey described, in a general way and in light of our prior cases,
those personal activities and decisions that this Court has identified
as so deeply rooted in our history and traditions, or so fundamental to
our concept of constitutionally ordered liberty, that they are protected
by the Fourteenth Amendment. 19 The opinion moved from the recognition
that liberty necessarily includes freedom of conscience and belief about
ultimate considerations to the observation that ``though the abortion
decision may originate within the zone of conscience and belief, it is
more than a philosophic exercise.'' Casey, 112 S.Ct. (emphasis added).
That many of the rights and liberties protected by the Due Process
Clause sound in personal autonomy does not warrant the sweeping
conclusion that any and all important, intimate, and personal decisions
are so protected, San Antonio Independent School Dist. v. Rodriguez, and
Casey did not suggest otherwise.

The history of the law's treatment of assisted suicide in this country
has been and continues to be one of the rejection of nearly all efforts
to permit it. That being the case, our decisions lead us to conclude
that the asserted ``right'' to assistance in committing suicide is not a
fundamental liberty interest protected by the Due Process Clause. The
Constitution also requires, however, that Washington's assisted-suicide
ban be rationally related to legitimate government interests. See Heller
v. Doe, ; Flores, This requirement is unquestionably met here. As the
court below recognized, 79 F d, 20 Washington's assisted-suicide ban
implicates a number of state interests. 21 See 49 F d; Brief for State
of California et al.~as Amici Curiae 26-29; Brief for United States as
Amicus Curiae 16-27.

First, Washington has an ``unqualified interest in the preservation of
human life.'' Cruzan, The State's prohibition on assisted suicide, like
all homicide laws, both reflects and advances its commitment to this
interest. See 110 S.Ct.; Model Penal Code §210 Comment 5 (" {[}T{]}he
interests in the sanctity of life that are represented by the criminal
homicide laws are threatened by one who expresses a willingness to
participate in taking the life of an other"). 22 This interest is
symbolic and aspirational as well as practical:

``While suicide is no longer prohibited or penalized, the ban against
assisted suicide and euthanasia shores up the notion of limits in human
relationships. It reflects the gravity with which we view the decision
to take one's own life or the life of another, and our reluctance to
encourage or promote these decisions.'' New York Task Force 131-132.

Respondents admit that " {[}t{]}he State has a real interest in
preserving the lives of those who can still contribute to society and
enjoy life." Brief for Respondents 35, n.~23. The Court of Appeals also
recognized Washington's interest in protecting life, but held that the
``weight'' of this interest depends on the ``medical condition and the
wishes of the person whose life is at stake.'' 79 F d.~Washington,
however, has rejected this sliding-scale approach and, through its
assisted-suicide ban, insists that all persons' lives, from beginning to
end, regardless of physical or mental condition, are under the full
protection of the law. See United States v. Rutherford, (``. . Congress
could reasonably have determined to protect the terminally ill, no less
than other patients, from the vast range of self-styled panaceas that
inventive minds can devise''). As we have previously affirmed, the
States ``may properly decline to make judgments about the''quality' of
life that a particular individual may enjoy," Cruzan, This remains true,
as Cruzan makes clear, even for those who are near death.

Relatedly, all admit that suicide is a serious public-health problem,
especially among persons in otherwise vulnerable groups. See Washington
State Dept. of Health, Annual Summary of Vital Statistics 1991,
pp.~29-30 (Oct ) (suicide is a leading cause of death in Washington of
those between the ages of 14 and 54); New York Task Force 10, 23-33
(suicide rate in the general population is about one percent, and
suicide is especially prevalent among the young and the elderly). The
State has an interest in preventing suicide, and in studying,
identifying, and treating its causes. See 79 F d; (Beezer, J.,
dissenting) (``The state recognizes suicide as a manifestation of
medical and psychological anguish''); Marzen 107-146.

Those who attempt suicide-terminally ill or not-often suffer from
depression or other mental disorders. See New York Task Force 13 (more
than 95\% of those who commit suicide had a major psychiatric illness at
the time of death; among the terminally ill, uncontrolled pain is a
``risk factor'' because it contributes to depression);
Physician-Assisted Suicide and Euthanasia in the Netherlands: A Report
of Chairman Charles T. Canady to the Subcommittee on the Constitution of
the House Committee on the Judiciary, 104th Cong., 2d Sess., 10-11
(Comm. Print 1996); cf.~Back, Wallace, Starks, \& Pearlman,
Physician-Assisted Suicide and Euthanasia in Washington State, 275 JAMA
919, 924 (1996) (" {[}I{]}ntolerable physical symptoms are not the
reason most patients request physician-assisted suicide or
euthanasia``). Research indicates, however, that many people who request
physician-assisted suicide withdraw that request if their depression and
pain are treated. H. Hendin, Seduced by Death: Doctors, Patients and the
Dutch Cure 24-25 (1997) (suicidal, terminally ill patients''usually
respond well to treatment for depressive illness and pain medication and
are then grateful to be alive"); New York Task Force 177-178. The New
York Task Force, however, expressed its concern that, because depression
is difficult to diagnose, physicians and medical professionals often
fail to respond adequately to seriously ill patients' needs. Thus, legal
physician-assisted suicide could make it more difficult for the State to
protect depressed or mentally ill persons, or those who are suffering
from untreated pain, from suicidal impulses.

The State also has an interest in protecting the integrity and ethics of
the medical profession. In contrast to the Court of Appeals' conclusion
that ``the integrity of the medical profession would {[}not{]} be
threatened in any way by {[}physician-assisted suicide{]},'' 79 F d, the
American Medical Association, like many other medical and physicians'
groups, has concluded that " {[}p{]}hysician-assisted suicide is
fundamentally incompatible with the physician's role as healer."
American Medical Association, Code of Ethics §2 (1994); see Council on
Ethical and Judicial Affairs, Decisions Near the End of Life, 267 JAMA
2229, 2233 (1992) (" {[}T{]}he societal risks of involving physicians in
medical interventions to cause patients' deaths is too great``); New
York Task Force 103-109 (discussing physicians' views). And
physician-assisted suicide could, it is argued, undermine the trust that
is essential to the doctor-patient relationship by blurring the
time-honored line between healing and harming. Assisted Suicide in the
United States, Hearing before the Subcommittee on the Constitution of
the House Committee on the Judiciary, 104th Cong., 2d Sess., 355-356
(1996) (testimony of Dr.~Leon R. Kass) (''The patient's trust in the
doctor's whole-hearted devotion to his best interests will be hard to
sustain").

Next, the State has an interest in protecting vulnerable
groups-including the poor, the elderly, and disabled persons-from abuse,
neglect, and mistakes. The Court of Appeals dismissed the State's
concern that disadvantaged persons might be pressured into
physician-assisted suicide as ``ludicrous on its face.'' 79 F d.~We have
recognized, however, the real risk of subtle coercion and undue
influence in end-of-life situations. Cruzan, Similarly, the New York
Task Force warned that " {[}l{]}egalizing physician-assisted suicide
would pose profound risks to many individuals who are ill and vulnerable
. . The risk of harm is greatest for the many individuals in our society
whose autonomy and well-being are already compromised by poverty, lack
of access to good medical care, advanced age, or membership in a
stigmatized social group." New York Task Force 120; see Compassion in
Dying, 49 F d (" {[}A{]}n insidious bias against the handicapped-again
coupled with a cost-saving mentality-makes them especially in need of
Washington's statutory protection"). If physician-assisted suicide were
permitted, many might resort to it to spare their families the
substantial financial burden of end-of-life health-care costs.

The State's interest here goes beyond protecting the vulnerable from
coercion; it extends to protecting disabled and terminally ill people
from prejudice, negative and inaccurate stereotypes, and ``societal
indifference.'' 49 F d.~The State's assisted-suicide ban reflects and
reinforces its policy that the lives of terminally ill, disabled, and
elderly people must be no less valued than the lives of the young and
healthy, and that a seriously disabled person's suicidal impulses should
be interpreted and treated the same way as anyone else's. See New York
Task Force 101-102; Physician-Assisted Suicide and Euthanasia in the
Netherlands: A Report of Chairman Charles T. Canady, 20 (discussing
prejudice toward the disabled and the negative messages euthanasia and
assisted suicide send to handicapped patients).

Finally, the State may fear that permitting assisted suicide will start
it down the path to voluntary and perhaps even involuntary euthanasia.
The Court of Appeals struck down Washington's assisted-suicide ban only
``as applied to competent, terminally ill adults who wish to hasten
their deaths by obtaining medication prescribed by their doctors.'' 79 F
d.~Washington insists, however, that the impact of the court's decision
will not and cannot be so limited. Brief for Petitioners 44-47. If
suicide is protected as a matter of constitutional right, it is argued,
``every man and woman in the United States must enjoy it.'' Compassion
in Dying, 49 F d; see Kevorkian, 447 Mich., n.~41, 527 N.W d, n.~41. The
Court of Appeals' decision, and its expansive reasoning, provide ample
support for the State's concerns. The court noted, for example, that the
``decision of a duly appointed surrogate decision maker is for all legal
purposes the decision of the patient himself,'' 79 F d, n.~120; that
``in some instances, the patient may be unable to self-administer the
drugs and . administration by the physician . may be the only way the
patient may be able to receive them,'' ; and that not only physicians,
but also family members and loved ones, will inevitably participate in
assisting suicide. n.~140. Thus, it turns out that what is couched as a
limited right to ``physician-assisted suicide'' is likely, in effect, a
much broader license, which could prove extremely difficult to police
and contain. 23 Washington's ban on assisting suicide prevents such
erosion.

This concern is further supported by evidence about the practice of
euthanasia in the Netherlands. The Dutch government's own study revealed
that in 1990, there were 2,300 cases of voluntary euthanasia (defined as
``the deliberate termination of another's life at his request''), 400
cases of assisted suicide, and more than 1,000 cases of euthanasia
without an explicit request. In addition to these latter 1,000 cases,
the study found an additional 4,941 cases where physicians administered
lethal morphine overdoses without the patients' explicit consent.
Physician-Assisted Suicide and Euthanasia in the Netherlands: A Report
of Chairman Charles T. Canady (citing Dutch study). This study suggests
that, despite the existence of various reporting procedures, euthanasia
in the Netherlands has not been limited to competent, terminally ill
adults who are enduring physical suffering, and that regulation of the
practice may not have prevented abuses in cases involving vulnerable
persons, including severely disabled neonates and elderly persons
suffering from dementia. -21; see generally C. Gomez, Regulating Death:
Euthanasia and the Case of the Netherlands (1991); H. Hendin, Seduced By
Death: Doctors, Patients, and the Dutch Cure (1997). The New York Task
Force, citing the Dutch experience, observed that ``assisted suicide and
euthanasia are closely linked,'' New York Task Force 145, and concluded
that the ``risk of . abuse is neither speculative nor distant,''
Washington, like most other States, reasonably ensures against this risk
by banning, rather than regulating, assisting suicide. See United States
v. 12 200-ft Reels of Super 8MM Film, (``Each step, when taken,
appear{[}s{]} a reasonable step in relation to that which preceded it,
although the aggregate or end result is one that would never have been
seriously considered in the first instance'').

We need not weigh exactingly the relative strengths of these various
interests. They are unquestionably important and legitimate, and
Washington's ban on assisted suicide is at least reasonably related to
their promotion and protection. We therefore hold that Wash. Rev.Code
§9A (1) (1994) does not violate the Fourteenth Amendment, either on its
face or ``as applied to competent, terminally ill adults who wish to
hasten their deaths by obtaining medication prescribed by their
doctors.'' 79 F d.~

Throughout the Nation, Americans are engaged in an earnest and profound
debate about the morality, legality, and practicality of
physician-assisted suicide. Our holding permits this debate to continue,
as it should in a democratic society. The decision of the en banc Court
of Appeals is reversed, and the case is remanded for further proceedings
consistent with this opinion.

It is so ordered.

\textbf{Justice SOUTER, concurring in the judgment.}

Three terminally ill individuals and four physicians who sometimes treat
terminally ill patients brought this challenge to the Washington statute
making it a crime ``knowingly . {[}to{]} ai{[}d{]} another person to
attempt suicide,'' Wash. Rev.Code §9A (1994), claiming on behalf of both
patients and physicians that it would violate substantive due process to
enforce the statute against a doctor who acceded to a dying patient's
request for a drug to be taken by the patient to commit suicide. The
question is whether the statute sets up one of those ``arbitrary
impositions'' or ``purposeless restraints'' at odds with the Due Process
Clause of the Fourteenth Amendment. Poe v. Ullman, (Harlan, J.,
dissenting). I conclude that the statute's application to the doctors
has not been shown to be unconstitutional, but I write separately to
give my reasons for analyzing the substantive due process claims as I
do, and for rejecting this one.

Although the terminally ill original parties have died during the
pendency of this case, the four physicians who remain as respondents
here1 continue to request declaratory and injunctive relief for their
own benefit in discharging their obligations to other dying patients who
request their help. 2 See, e.g., Southern Pacific Terminal Co.~v. ICC,
(question was capable of repetition yet evading review). The case
reaches us on an order granting summary judgment, and we must take as
true the undisputed allegations that each of the patients was mentally
competent and terminally ill, and that each made a knowing and voluntary
choice to ask a doctor to prescribe ``medications . to be
self-administered for the purpose of hastening . death.'' Complaint ¶2
The State does not dispute that each faced a passage to death more
agonizing both mentally and physically, and more protracted over time,
than death by suicide with a physician's help, or that each would have
chosen such a suicide for the sake of personal dignity, apart even from
relief from pain. Each doctor in this case claims to encounter patients
like the original plaintiffs who have died, that is, mentally competent,
terminally ill, and seeking medical help in ``the voluntary
self-termination of life.'' ¶2 -2 While there may be no unanimity on the
physician's professional obligation in such circumstances, I accept here
respondents' representation that providing such patients with
prescriptions for drugs that go beyond pain relief to hasten death
would, in these circumstances, be consistent with standards of medical
practice. Hence, I take it to be true, as respondents say, that the
Washington statute prevents the exercise of a physician's ``best
professional judgment to prescribe medications to {[}such{]} patients in
dosages that would enable them to act to hasten their own deaths.'' ¶2 ;
see also App. 35, 55.

In their brief to this Court, the doctors claim not that they ought to
have a right generally to hasten patients' imminent deaths, but only to
help patients who have made ``personal decisions regarding their own
bodies, medical care, and, fundamentally, the future course of their
lives,'' Brief for Respondents 12, and who have concluded responsibly
and with substantial justification that the brief and anguished
remainders of their lives have lost virtually all value to them.
Respondents fully embrace the notion that the State must be free to
impose reasonable regulations on such physician assistance to ensure
that the patients they assist are indeed among the competent and
terminally ill and that each has made a free and informed choice in
seeking to obtain and use a fatal drug.

In response, the State argues that the interest asserted by the doctors
is beyond constitutional recognition because it has no deep roots in our
history and traditions. Brief for Petitioners 21-25. But even aside from
that, without disputing that the patients here were competent and
terminally ill, the State insists that recognizing the legitimacy of
doctors' assistance of their patients as contemplated here would entail
a number of adverse consequences that the Washington Legislature was
entitled to forestall. The nub of this part of the State's argument is
not that such patients are constitutionally undeserving of relief on
their own account, but that any attempt to confine a right of physician
assistance to the circumstances presented by these doctors is likely to
fail.

First, the State argues that the right could not be confined to the
terminally ill.~Even assuming a fixed definition of that term, the State
observes that it is not always possible to say with certainty how long a
person may live. It asserts that " {[}t{]}here is no principled basis on
which {[}the right{]} can be limited to the prescription of medication
for terminally ill patients to administer to themselves" when the
right's justifying principle is as broad as "``merciful termination of
suffering.''' (citing Y. Kamisar, Are Laws Against Assisted Suicide
Unconstitutional?, Hastings Center Report 32, 36-37 (May-June 1993)).
Second, the State argues that the right could not be confined to the
mentally competent, observing that a person's competence cannot always
be assessed with certainty, Brief for Petitioners 34, and suggesting
further that no principled distinction is possible between a competent
patient acting independently and a patient acting through a duly
appointed and competent surrogate, Next, according to the State, such a
right might entail a right to or at least merge in practice into ``other
forms of life-ending assistance,'' such as euthanasia. -47. Finally, the
State believes that a right to physician assistance could not easily be
distinguished from a right to assistance from others, such as friends,
family, and other health-care workers. The State thus argues that
recognition of the substantive due process right at issue here would
jeopardize the lives of others outside the class defined by the doctors'
claim, creating risks of irresponsible suicides and euthanasia, whose
dangers are concededly within the State's authority to address.

When the physicians claim that the Washington law deprives them of a
right falling within the scope of liberty that the Fourteenth Amendment
guarantees against denial without due process of law, 3 they are not
claiming some sort of procedural defect in the process through which the
statute has been enacted or is administered. Their claim, rather, is
that the State has no substantively adequate justification for barring
the assistance sought by the patient and sought to be offered by the
physician. Thus, we are dealing with a claim to one of those rights
sometimes described as rights of substantive due process and sometimes
as unenumerated rights, in view of the breadth and indeterminacy of the
``due process'' serving as the claim's textual basis. The doctors
accordingly arouse the skepticism of those who find the Due Process
Clause an unduly vague or oxymoronic warrant for judicial review of
substantive state law, just as they also invoke two centuries of
American constitutional practice in recognizing unenumerated,
substantive limits on governmental action. Although this practice has
neither rested on any single textual basis nor expressed a consistent
theory (or, before Poe v. Ullman, a much articulated one), a brief
overview of its history is instructive on two counts. The persistence of
substantive due process in our cases points to the legitimacy of the
modern justification for such judicial review found in Justice Harlan's
dissent in Poe,4 on which I will dwell further on, while the
acknowledged failures of some of these cases point with caution to the
difficulty raised by the present claim.

Before the ratification of the Fourteenth Amendment, substantive
constitutional review resting on a theory of unenumerated rights
occurred largely in the state courts applying state constitutions that
commonly contained either due process clauses like that of the Fifth
Amendment (and later the Fourteenth) or the textual antecedents of such
clauses, repeating Magna Carta's guarantee of ``the law of the land.''
On the basis of such clauses, or of general principles untethered to
specific constitutional language, state courts evaluated the
constitutionality of a wide range of statutes.

Thus, a Connecticut court approved a statute legitimating a class of
previous illegitimate marriages, as falling within the terms of the
``social compact,'' while making clear its power to review
constitutionality in those terms. Goshen v. Stonington, 4 Conn. 209,
225-226 (1822). In the same period, a specialized court of equity,
created under a Tennessee statute solely to hear cases brought by the
state bank against its debtors, found its own authorization
unconstitutional as ``partial'' legislation violating the state
constitution's ``law of the land'' clause. Bank of the State v. Cooper,
10 Tenn. 599, 2 Yerg. 599, 602-608 (1831) (Green, J.); Yer. (Peck, J.);
-623 (Kennedy, J.). And the middle of the 19th century brought the
famous Wynehamer case, invalidating a statute purporting to render
possession of liquor immediately illegal except when kept for narrow,
specified purposes, the state court finding the statute inconsistent
with the state's due process clause. Wynehamer v. People, 13 N.Y. 378,
486-487 (1856). The statute was deemed an excessive threat to the
``fundamental rights of the citizen'' to property. (Comstock, J.).

Even in this early period, however, this Court anticipated the
developments that would presage both the Civil War and the ratification
of the Fourteenth Amendment, by making it clear on several occasions
that it too had no doubt of the judiciary's power to strike down
legislation that conflicted with important but unenumerated principles
of American government. In most such instances, after declaring its
power to invalidate what it might find inconsistent with rights of
liberty and property, the Court nevertheless went on to uphold the
legislative acts under review. See, e.g., Wilkinson v. Leland, 2 Pet.
627, 656-661, ; Calder v. Bull, 3 Dall. 386, 386-395, (opinion of Chase,
J.); see also Corfield v. Coryell, 6 F. Cas. 546, 550-552, No.~3,230
(1823). But in Fletcher v. Peck, 6 Cranch 87, the Court went further. It
struck down an act of the Georgia legislature that purported to rescind
a sale of public land ab initio and reclaim title for the State, and so
deprive subsequent, good-faith purchasers of property conveyed by the
original grantees. The Court rested the invalidation on alternative
sources of authority: the specific prohibitions against bills of
attainder, ex post facto laws, laws impairing contracts in Article I,
§10 of the Constitution; and ``general principles which are common to
our free institutions,'' by which Chief Justice Marshall meant that a
simple deprivation of property by the State could not be an
authentically ``legislative'' act. Fletcher, 6 Cranch,

Fletcher was not, though, the most telling early example of such review.
For its most salient instance in this Court before the adoption of the
Fourteenth Amendment was, of course, the case that the Amendment would
in due course overturn, Dred Scott v. Sandford, 19 How. 393, Unlike
Fletcher, Dred Scott was textually based on a due process clause (in the
Fifth Amendment, applicable to the national government), and it was in
reliance on that clause's protection of property that the Court
invalidated the Missouri Compromise. 19 How.. This substantive
protection of an owner's property in a slave taken to the territories
was traced to the absence of any enumerated power to affect that
property granted to the Congress by Article I of the Constitution, -452,
the implication being that the government had no legitimate interest
that could support the earlier congressional compromise. The ensuing
judgment of history needs no recounting here.

After the ratification of the Fourteenth Amendment, with its guarantee
of due process protection against the States, interpretation of the
words ``liberty'' and ``property'' as used in due process clauses became
a sustained enterprise, with the Court generally describing the due
process criterion in converse terms of reasonableness or arbitrariness.
That standard is fairly traceable to Justice Bradley's dissent in the
Slaughter-House Cases, 16 Wall. 36, in which he said that a person's
right to choose a calling was an element of liberty (as the calling,
once chosen, was an aspect of property) and declared that the liberty
and property protected by due process are not truly recognized if such
rights may be ``arbitrarily assailed,'' Wall.. 6 After that, opinions
comparable to those that preceded Dred Scott expressed willingness to
review legislative action for consistency with the Due Process Clause
even as they upheld the laws in question. See, e.g., Bartemeyer v. Iowa,
18 Wall. 129, 133-135, ; Munn v. Illinois, ; Railroad Comm'n Cases, ;
Mugler v. Kansas, See generally Corwin, Liberty Against Government
(surveying the Court's early Fourteenth Amendment cases and finding
little dissent from the general principle that the Due Process Clause
authorized judicial review of substantive statutes).

The theory became serious, however, beginning with Allgeyer v.
Louisianawhere the Court invalidated a Louisiana statute for excessive
interference with Fourteenth Amendment liberty to contract, -593, and
offered a substantive interpretation of ``liberty,'' that in the
aftermath of the so-called Lochner Era has been scaled back in some
respects, but expanded in others, and never repudiated in principle. The
Court said that Fourteenth Amendment liberty includes ``the right of the
citizen to be free in the enjoyment of all his faculties; to be free to
use them in all lawful ways; to live and work where he will; to earn his
livelihood by any lawful calling; to pursue any livelihood or avocation;
and for that purpose to enter into all contracts which may be proper,
necessary and essential to his carrying out to a successful conclusion
the purposes above mentioned.'' " {[}W{]}e do not intend to hold that in
no such case can the State exercise its police power," the Court added,
but " {[}w{]}hen and how far such power may be legitimately exercised
with regard to these subjects must be left for determination to each
case as it arises." Although this principle was unobjectionable, what
followed for a season was, in the realm of economic legislation, the
echo of Dred Scott. Allgeyer was succeeded within a decade by Lochner v.
New Yorkand the era to which that case gave its name, famous now for
striking down as arbitrary various sorts of economic regulations that
post-New Deal courts have uniformly thought constitutionally sound.
Compare, e.g., 25 S.Ct. (finding New York's maximum-hours law for bakers
``unreasonable and entirely arbitrary'') and Adkins v. Children's
Hospital of D.C., (holding a minimum wage law ``so clearly the product
of a naked, arbitrary exercise of power that it cannot be allowed to
stand under the Constitution of the United States'') with West Coast
Hotel Co.~v. Parrish, (overruling Adkins and approving a minimum-wage
law on the principle that ``regulation which is reasonable in relation
to its subject and is adopted in the interests of the community is due
process''). As the parentheticals here suggest, while the cases in the
Lochner line routinely invoked a correct standard of constitutional
arbitrariness review, they harbored the spirit of Dred Scott in their
absolutist implementation of the standard they espoused.

Even before the deviant economic due process cases had been repudiated,
however, the more durable precursors of modern substantive due process
were reaffirming this Court's obligation to conduct arbitrariness
review, beginning with Meyer v. NebraskaWithout referring to any
specific guarantee of the Bill of Rights, the Court invoked precedents
from the Slaughter-House Cases through Adkins to declare that the
Fourteenth Amendment protected ``the right of the individual to
contract, to engage in any of the common occupations of life, to acquire
useful knowledge, to marry, establish a home and bring up children, to
worship God according to the dictates of his own conscience, and
generally to enjoy those privileges long recognized at common law as
essential to the orderly pursuit of happiness by free men.'' The Court
then held that the same Fourteenth Amendment liberty included a
teacher's right to teach and the rights of parents to direct their
children's education without unreasonable interference by the States,
with the result that Nebraska's prohibition on the teaching of foreign
languages in the lower grades was, ``arbitrary and without reasonable
relation to any end within the competency of the State,'' See also
Pierce v. Society of Sisters, (finding that a statute that all but
outlawed private schools lacked any ``reasonable relation to some
purpose within the competency of the State''); Palko v. Connecticut,
(``even in the field of substantive rights and duties the legislative
judgment, if oppressive and arbitrary, may be overridden by the
courts''; ``Is that {[}injury{]} to which the statute has subjected
{[}the appellant{]} a hardship so acute and shocking that our polity
will not endure it? Does it violate those fundamental principles of
liberty and justice which lie at the base of all our civil and political
institutions?'') (citation and internal quotation marks omitted).

After Meyer and Pierce, two further opinions took the major steps that
lead to the modern law. The first was not even in a due process case but
one about equal protection, Skinner v. Oklahoma ex rel. Williamsonwhere
the Court emphasized the ``fundamental'' nature of individual choice
about procreation and so foreshadowed not only the later prominence of
procreation as a subject of liberty protection, but the corresponding
standard of ``strict scrutiny,'' in this Court's Fourteenth Amendment
law. See Skinner, that is, added decisions regarding procreation to the
list of liberties recognized in Meyer and Pierce and loosely suggested,
as a gloss on their standard of arbitrariness, a judicial obligation to
scrutinize any impingement on such an important interest with heightened
care. In so doing, it suggested a point that Justice Harlan would
develop, that the kind and degree of justification that a sensitive
judge would demand of a State would depend on the importance of the
interest being asserted by the individual. Poe, The second major opinion
leading to the modern doctrine was Justice Harlan's Poe dissent just
cited, the conclusion of which was adopted in Griswold v. Connecticutand
the authority of which was acknowledged in Planned Parenthood of
Southeastern Pa. v. CaseySee also n.~4. The dissent is important for
three things that point to our responsibilities today. The first is
Justice Harlan's respect for the tradition of substantive due process
review itself, and his acknowledgement of the Judiciary's obligation to
carry it on. For two centuries American courts, and for much of that
time this Court, have thought it necessary to provide some degree of
review over the substantive content of legislation under constitutional
standards of textual breadth. The obligation was understood before Dred
Scott and has continued after the repudiation of Lochner's progeny, most
notably on the subjects of segregation in public education, Bolling v.
Sharpe, interracial marriage, Loving v. Virginia, marital privacy and
contraception, Carey v. Population Services Int'l, Griswold v.
Connecticut, abortion, Planned Parenthood of Southeastern Pa. v. Casey,
869, (joint opinion of O'CONNOR, KENNEDY, and SOUTER, JJ.), Roe v. Wade,
personal control of medical treatment, Cruzan v. Director, Mo. Dept. of
Health, (O'CONNOR, J., concurring); 110 S.Ct. (Brennan, J., dissenting);
110 S.Ct. (STEVENS, J., dissenting); see also 110 S.Ct. (majority
opinion), and physical confinement, Foucha v. Louisiana, This enduring
tradition of American constitutional practice is, in Justice Harlan's
view, nothing more than what is required by the judicial authority and
obligation to construe constitutional text and review legislation for
conformity to that text. See Marbury v. Madison, 1 Cranch 137, Like many
judges who preceded him and many who followed, he found it impossible to
construe the text of due process without recognizing substantive, and
not merely procedural, limitations. ``Were due process merely a
procedural safeguard it would fail to reach those situations where the
deprivation of life, liberty or property was accomplished by legislation
which by operating in the future could, given even the fairest possible
procedure in application to individuals, nevertheless destroy the
enjoyment of all three.'' Poe, 7 The text of the Due Process Clause thus
imposes nothing less than an obligation to give substantive content to
the words ``liberty'' and ``due process of law.''

Following the first point of the Poe dissent, on the necessity to engage
in the sort of examination we conduct today, the dissent's second and
third implicitly address those cases, already noted, that are now
condemned with virtual unanimity as disastrous mistakes of substantive
due process review. The second of the dissent's lessons is a reminder
that the business of such review is not the identification of
extratextual absolutes but scrutiny of a legislative resolution (perhaps
unconscious) of clashing principles, each quite possibly worthy in and
of itself, but each to be weighed within the history of our values as a
people. It is a comparison of the relative strengths of opposing claims
that informs the judicial task, not a deduction from some first premise.
Thus informed, judicial review still has no warrant to substitute one
reasonable resolution of the contending positions for another, but
authority to supplant the balance already struck between the contenders
only when it falls outside the realm of the reasonable. Part III, below,
deals with this second point, and also with the dissent's third, which
takes the form of an object lesson in the explicit attention to detail
that is no less essential to the intellectual discipline of substantive
due process review than an understanding of the basic need to account
for the two sides in the controversy and to respect legislation within
the zone of reasonableness.

My understanding of unenumerated rights in the wake of the Poe dissent
and subsequent cases avoids the absolutist failing of many older cases
without embracing the opposite pole of equating reasonableness with past
practice described at a very specific level. See Planned Parenthood of
Southeastern Pa. v. Casey, That understanding begins with a concept of
``ordered liberty,'' Poe, (Harlan, J.); see also Griswold, comprising a
continuum of rights to be free from ``arbitrary impositions and
purposeless restraints,'' Poe, (Harlan, J., dissenting).

``Due Process has not been reduced to any formula; its content cannot be
determined by reference to any code. The best that can be said is that
through the course of this Court's decisions it has represented the
balance which our Nation, built upon postulates of respect for the
liberty of the individual, has struck between that liberty and the
demands of organized society. If the supplying of content to this
Constitutional concept has of necessity been a rational process, it
certainly has not been one where judges have felt free to roam where
unguided speculation might take them. The balance of which I speak is
the balance struck by this country, having regard to what history
teaches are the traditions from which it developed as well as the
traditions from which it broke. That tradition is a living thing. A
decision of this Court which radically departs from it could not long
survive, while a decision which builds on what has survived is likely to
be sound. No formula could serve as a substitute, in this area, for
judgment and restraint.'' See also Moore v. East Cleveland, (plurality
opinion of Powell, J.) (``Appropriate limits on substantive due process
come not from drawing arbitrary lines but rather from careful''respect
for the teachings of history {[}and{]} solid recognition of the basic
values that underlie our society"') (quoting Griswold, 85 S.Ct. (Harlan,
J., concurring)).

After the Poe dissent, as before it, this enforceable concept of liberty
would bar statutory impositions even at relatively trivial levels when
governmental restraints are undeniably irrational as unsupported by any
imaginable rationale. See, e.g., United States v. Carolene Products Co.,
(economic legislation ``not . unconstitutional unless . facts . preclude
the assumption that it rests upon some rational basis''); see also Poe,
548, 1779-1780 (Harlan, J., dissenting) (referring to usual
``presumption of constitutionality'' and ordinary test ``going merely to
the plausibility of {[}a statute's{]} underlying rationale''). Such
instances are suitably rare. The claims of arbitrariness that mark
almost all instances of unenumerated substantive rights are those
resting on ``certain interests requir{[}ing{]} particularly careful
scrutiny of the state needs asserted to justify their abridgment. Cf.
Skinner v. Oklahoma {[}ex rel. Williamson{]}; Bolling v. Sharpe, {[}347
U.S. 497, {]},'' ; that is, interests in liberty sufficiently important
to be judged ``fundamental,'' ; see also (citing Corfield v. Coryell, 4
Wash. C.C. 371, 380 (C.C.E.D.Pa )). In the face of an interest this
powerful a State may not rest on threshold rationality or a presumption
of constitutionality, but may prevail only on the ground of an interest
sufficiently compelling to place within the realm of the reasonable a
refusal to recognize the individual right asserted. Poe, (Harlan, J.,
dissenting) (an ``enactment involv{[}ing{]} . a most fundamental aspect
of''liberty' . {[}is{]} subjec{[}t{]} to ``strict scrutiny''`) (quoting
Skinner v. Oklahoma ex rel. Williamson, 62 S.Ct.); 8Reno v. Flores,
(reaffirming that due process ``forbids the government to infringe
certain''fundamental' liberty interests . unless the infringement is
narrowly tailored to serve a compelling state interest").

This approach calls for a court to assess the relative ``weights'' or
dignities of the contending interests, and to this extent the judicial
method is familiar to the common law. Common law method is subject,
however, to two important constraints in the hands of a court engaged in
substantive due process review. First, such a court is bound to confine
the values that it recognizes to those truly deserving constitutional
stature, either to those expressed in constitutional text, or those
exemplified by ``the traditions from which {[}the Nation{]} developed,''
or revealed by contrast with ``the traditions from which it broke.'' Poe
(Harlan, J., dissenting). "``We may not draw on our merely personal and
private notions and disregard the limits . derived from considerations
that are fused in the whole nature of our judicial process . {[},{]}
considerations deeply rooted in reason and in the compelling traditions
of the legal profession.''' S.Ct. (quoting Rochin v. California, ); see
also Palko v. Connecticut, 58 S.Ct. (looking to "``principle{[}s{]} of
justice so rooted in the traditions and conscience of our people as to
be ranked as fundamental''') (quoting Snyder v. Massachusetts, ).

The second constraint, again, simply reflects the fact that
constitutional review, not judicial lawmaking, is a court's business
here. The weighing or valuing of contending interests in this sphere is
only the first step, forming the basis for determining whether the
statute in question falls inside or outside the zone of what is
reasonable in the way it resolves the conflict between the interests of
state and individual. See, e.g., Poe, (Harlan, J., dissenting);
Youngberg v. Romeo, It is no justification for judicial intervention
merely to identify a reasonable resolution of contending values that
differs from the terms of the legislation under review. It is only when
the legislation's justifying principle, critically valued, is so far
from being commensurate with the individual interest as to be
arbitrarily or pointlessly applied that the statute must give way. Only
if this standard points against the statute can the individual claimant
be said to have a constitutional right. See Cruzan v. Director, Mo.
Dept. of Health, 110 S.Ct. (" {[}D{]}etermining that a person has a
``liberty interest' under the Due Process Clause does not end the
inquiry;''whether {[}the individual's{]} constitutional rights have been
violated must be determined by balancing his liberty interests against
the relevant state interests"') (quoting Youngberg v. Romeo, 102 S.Ct.).

The Poe dissent thus reminds us of the nature of review for
reasonableness or arbitrariness and the limitations entailed by it. But
the opinion cautions against the repetition of past error in another way
as well, more by its example than by any particular statement of
constitutional method: it reminds us that the process of substantive
review by reasoned judgment, Poe, is one of close criticism going to the
details of the opposing interests and to their relationships with the
historically recognized principles that lend them weight or value.

Although the Poe dissent disclaims the possibility of any general
formula for due process analysis (beyond the basic analytic structure
just described), see 544, 1777-1778, Justice Harlan of course assumed
that adjudication under the Due Process Clauses is like any other
instance of judgment dependent on common-law method, being more or less
persuasive according to the usual canons of critical discourse. See also
Casey, 112 S.Ct. (``The inescapable fact is that adjudication of
substantive due process claims may call upon the Court in interpreting
the Constitution to exercise that same capacity which by tradition
courts always have exercised: reasoned judgment''). When identifying and
assessing the competing interests of liberty and authority, for example,
the breadth of expression that a litigant or a judge selects in stating
the competing principles will have much to do with the outcome and may
be dispositive. As in any process of rational argumentation, we
recognize that when a generally accepted principle is challenged, the
broader the attack the less likely it is to succeed. The principle's
defenders will, indeed, often try to characterize any challenge as just
such a broadside, perhaps by couching the defense as if a broadside
attack had occurred. So the Court in Dred Scott treated prohibition of
slavery in the Territories as nothing less than a general assault on the
concept of property. See Dred Scott v. Sandford, 19 How..

Just as results in substantive due process cases are tied to the
selections of statements of the competing interests, the acceptability
of the results is a function of the good reasons for the selections
made. It is here that the value of common-law method becomes apparent,
for the usual thinking of the common law is suspicious of the
all-or-nothing analysis that tends to produce legal petrification
instead of an evolving boundary between the domains of old principles.
Common-law method tends to pay respect instead to detail, seeking to
understand old principles afresh by new examples and new
counterexamples. The ``tradition is a living thing,'' Poe (Harlan, J.,
dissenting), albeit one that moves by moderate steps carefully taken.
``The decision of an apparently novel claim must depend on grounds which
follow closely on well-accepted principles and criteria. The new
decision must take its place in relation to what went before and further
{[}cut{]} a channel for what is to come.'' (Harlan, J., dissenting)
(internal quotation marks omitted). Exact analysis and characterization
of any due process claim is critical to the method and to the result.

So, in Poe, Justice Harlan viewed it as essential to the plaintiffs'
claimed right to use contraceptives that they sought to do so within the
privacy of the marital bedroom. This detail in fact served two crucial
and complementary functions, and provides a lesson for today. It rescued
the individuals' claim from a breadth that would have threatened all
state regulation of contraception or intimate relations; extramarital
intimacy, no matter how privately practiced, was outside the scope of
the right Justice Harlan would have recognized in that case. See -553,
It was, moreover, this same restriction that allowed the interest to be
valued as an aspect of a broader liberty to be free from all
unreasonable intrusions into the privacy of the home and the family life
within it, a liberty exemplified in constitutional provisions such as
the Third and Fourth Amendments, in prior decisions of the Court
involving unreasonable intrusions into the home and family life, and in
the then-prevailing status of marriage as the sole lawful locus of
intimate relations. 551, 1781. 11 The individuals' interest was
therefore at its peak in Poe, because it was supported by a principle
that distinguished of its own force between areas in which government
traditionally had regulated (sexual relations outside of marriage) and
those in which it had not (private marital intimacies), and thus was
broad enough to cover the claim at hand without being so broad as to be
shot-through by exceptions.

On the other side of the balance, the State's interest in Poe was not
fairly characterized simply as preserving sexual morality, or doing so
by regulating contraceptive devices. Just as some of the earlier cases
went astray by speaking without nuance of individual interests in
property or autonomy to contract for labor, so the State's asserted
interest in Poe was not immune to distinctions turning (at least
potentially) on the precise purpose being pursued and the collateral
consequences of the means chosen, see -548, It was assumed that the
State might legitimately enforce limits on the use of contraceptives
through laws regulating divorce and annulment, or even through its tax
policy, , but not necessarily be justified in criminalizing the same
practice in the marital bedroom, which would entail the consequence of
authorizing state enquiry into the intimate relations of a married
couple who chose to close their door, -549, See also Casey, 112 S.Ct.
(strength of State's interest in potential life varies depending on
precise context and character of regulation pursuing that interest).

The same insistence on exactitude lies behind questions, in current
terminology, about the proper level of generality at which to analyze
claims and counter-claims, and the demand for fitness and proper
tailoring of a restrictive statute is just another way of testing the
legitimacy of the generality at which the government sets up its
justification. 12 We may therefore classify Justice Harlan's example of
proper analysis in any of these ways: as applying concepts of normal
critical reasoning, as pointing to the need to attend to the levels of
generality at which countervailing interests are stated, or as examining
the concrete application of principles for fitness with their own
ostensible justifications. But whatever the categories in which we place
the dissent's example, it stands in marked contrast to earlier cases
whose reasoning was marked by comparatively less discrimination, and it
points to the importance of evaluating the claims of the parties now
before us with comparable detail. For here we are faced with an
individual claim not to a right on the part of just anyone to help
anyone else commit suicide under any circumstances, but to the right of
a narrow class to help others also in a narrow class under a set of
limited circumstances. And the claimants are met with the State's
assertion, among others, that rights of such narrow scope cannot be
recognized without jeopardy to individuals whom the State may concededly
protect through its regulations. Respondents claim that a patient facing
imminent death, who anticipates physical suffering and indignity, and is
capable of responsible and voluntary choice, should have a right to a
physician's assistance in providing counsel and drugs to be administered
by the patient to end life promptly. Complaint ¶3 They accordingly claim
that a physician must have the corresponding right to provide such aid,
contrary to the provisions of Wash. Rev.Code §9A (1994). I do not
understand the argument to rest on any assumption that rights either to
suicide or to assistance in committing it are historically based as
such. Respondents, rather, acknowledge the prohibition of each
historically, but rely on the fact that to a substantial extent the
State has repudiated that history. The result of this, respondents say,
is to open the door to claims of such a patient to be accorded one of
the options open to those with different, traditionally cognizable
claims to autonomy in deciding how their bodies and minds should be
treated. They seek the option to obtain the services of a physician to
give them the benefit of advice and medical help, which is said to enjoy
a tradition so strong and so devoid of specifically countervailing state
concern that denial of a physician's help in these circumstances is
arbitrary when physicians are generally free to advise and aid those who
exercise other rights to bodily autonomy. PCS

The dominant western legal codes long condemned suicide and treated
either its attempt or successful accomplishment as a crime, the one
subjecting the individual to penalties, the other penalizing his
survivors by designating the suicide's property as forfeited to the
government. See 4 W. Blackstone, Commentaries - (commenting that English
law considered suicide to be ``ranked . among the highest crimes'' and
deemed persuading another to commit suicide to be murder); see generally
Marzen, O'Dowd, Crone, \& Balch, Suicide: A Constitutional Right?, 24
Duquense L.Rev. 1, 56-63 (1985). While suicide itself has generally not
been considered a punishable crime in the United States, largely because
the common-law punishment of forfeiture was rejected as improperly
penalizing an innocent family, see -99, most States have consistently
punished the act of assisting a suicide as either a common-law or
statutory crime and some continue to view suicide as an unpunishable
crime. See generally 13 Criminal prohibitions on such assistance remain
widespread, as exemplified in the Washington statute in question here.

The principal significance of this history in the State of Washington,
according to respondents, lies in its repudiation of the old tradition
to the extent of eliminating the criminal suicide prohibitions.
Respondents do not argue that the State's decision goes further, to
imply that the State has repudiated any legitimate claim to discourage
suicide or to limit its encouragement. The reasons for the
decriminalization, after all, may have had more to do with difficulties
of law enforcement than with a shift in the value ascribed to life in
various circumstances or in the perceived legitimacy of taking one's
own. See, e.g., Kamisar, Physician-Assisted Suicide: The Last Bridge to
Active Voluntary Euthanasia, in Euthanasia Examined 225, 229 (J. Keown
ed ); CeloCruz, Aid-in-Dying: Should We Decriminalize Physician-Assisted
Suicide and Physician-Committed Euthanasia?, 18 Am. J.L. \& Med. 369,
375 (1992); Marzen, O'Dowd, Crone, \& Balch 24 Duquesne L.Rev.\_\_-\_\_.
Thus it may indeed make sense for the State to take its hands off
suicide as such, while continuing to prohibit the sort of assistance
that would make its commission easier. See, e.g., American Law
Institute, Model Penal Code §210 Comment 5 (1980). Decriminalization
does not, then, imply the existence of a constitutional liberty interest
in suicide as such; it simply opens the door to the assertion of a
cognizable liberty interest in bodily integrity and associated medical
care that would otherwise have been inapposite so long as suicide, as
well as assisting a suicide, was a criminal offense.

This liberty interest in bodily integrity was phrased in a general way
by then-Judge Cardozo when he said, " {[}e{]}very human being of adult
years and sound mind has a right to determine what shall be done with
his own body" in relation to his medical needs. Schloendorff v. Society
of New York Hospital, 211 N.Y. 125, 129, 105 N.E. 92, 93 (1914). The
familiar examples of this right derive from the common law of battery
and include the right to be free from medical invasions into the body,
Cruzan v. Director, Mo. Dept. of Health, as well as a right generally to
resist enforced medication, see Washington v. Harper, 229, 1040-1041,
Thus " {[}i{]}t is settled now . that the Constitution places limits on
a State's right to interfere with a person's most basic decisions about
. bodily integrity." Casey, 112 S.Ct. (citations omitted); see also
Cruzan, 110 S.Ct.; 110 S.Ct. (O'CONNOR, J., concurring); Washington v.
Harper, 110 S.Ct.; Winston v. Lee, ; Rochin v. California,
Constitutional recognition of the right to bodily integrity underlies
the assumed right, good against the State, to require physicians to
terminate artificial life support, Cruzan, 110 S.Ct. (``we assume that
the United States Constitution would grant a competent person a
constitutionally protected right to refuse lifesaving hydration and
nutrition''), and the affirmative right to obtain medical intervention
to cause abortion, see Casey, 896, 2830; cf.~Roe v. Wade.

It is, indeed, in the abortion cases that the most telling recognitions
of the importance of bodily integrity and the concomitant tradition of
medical assistance have occurred. In Roe v. Wade, the plaintiff
contended that the Texas statute making it criminal for any person to
``procure an abortion,'' for a pregnant woman was unconstitutional
insofar as it prevented her from ``terminat{[}ing{]} her pregnancy by an
abortion''performed by a competent, licensed physician, under safe,
clinical conditions,"' and in striking down the statute we stressed the
importance of the relationship between patient and physician, see 156,
728.

The analogies between the abortion cases and this one are several. Even
though the State has a legitimate interest in discouraging abortion, see
Casey, 112 S.Ct. (joint opinion of O'CONNOR, KENNEDY, and SOUTER, JJ.)
Roe, the Court recognized a woman's right to a physician's counsel and
care. Like the decision to commit suicide, the decision to abort
potential life can be made irresponsibly and under the influence of
others, and yet the Court has held in the abortion cases that physicians
are fit assistants. Without physician assistance in abortion, the
woman's right would have too often amounted to nothing more than a right
to self-mutilation, and without a physician to assist in the suicide of
the dying, the patient's right will often be confined to crude methods
of causing death, most shocking and painful to the decedent's survivors.

There is, finally, one more reason for claiming that a physician's
assistance here would fall within the accepted tradition of medical care
in our society, and the abortion cases are only the most obvious
illustration of the further point. While the Court has held that the
performance of abortion procedures can be restricted to physicians, the
Court's opinion in Roe recognized the doctors' role in yet another way.
For, in the course of holding that the decision to perform an abortion
called for a physician's assistance, the Court recognized that the good
physician is not just a mechanic of the human body whose services have
no bearing on a person's moral choices, but one who does more than treat
symptoms, one who ministers to the patient. See 93 S.Ct.; see also
Griswold v. Connecticut, 85 S.Ct. (``This law . operates directly on an
intimate relation of husband and wife and their physician's role in one
aspect of that relation''); see generally R. Cabot, Ether Day Address,
Boston Medical and Surgical J. 287, 288 (1920). This idea of the
physician as serving the whole person is a source of the high value
traditionally placed on the medical relationship. Its value is surely as
apparent here as in the abortion cases, for just as the decision about
abortion is not directed to correcting some pathology, so the decision
in which a dying patient seeks help is not so limited. The patients here
sought not only an end to pain (which they might have had, although
perhaps at the price of stupor) but an end to their short remaining
lives with a dignity that they believed would be denied them by powerful
pain medication, as well as by their consciousness of dependency and
helplessness as they approached death. In that period when the end is
imminent, they said, the decision to end life is closest to decisions
that are generally accepted as proper instances of exercising autonomy
over one's own body, instances recognized under the Constitution and the
State's own law, instances in which the help of physicians is accepted
as falling within the traditional norm.

Respondents argue that the State has in fact already recognized enough
evolving examples of this tradition of patient care to demonstrate the
strength of their claim. Washington, like other States, authorizes
physicians to withdraw life-sustaining medical treatment and
artificially delivered food and water from patients who request it, even
though such actions will hasten death. See Wash. Rev.Code §§70 70
(1994); see generally Notes to Uniform Rights of the Terminally Ill Act,
9B U.L.A. 168-169 (Supp ) (listing state statutes). The State permits
physicians to alleviate anxiety and discomfort when withdrawing
artificial life-supporting devices by administering medication that will
hasten death even further. And it generally permits physicians to
administer medication to patients in terminal conditions when the
primary intent is to alleviate pain, even when the medication is so
powerful as to hasten death and the patient chooses to receive it with
that understanding.

The argument supporting respondents' position thus progresses through
three steps of increasing forcefulness. First, it emphasizes the
decriminalization of suicide. Reliance on this fact is sanctioned under
the standard that looks not only to the tradition retained, but to
society's occasional choices to reject traditions of the legal past. See
Poe v. Ullman, (Harlan, J., dissenting). While the common law prohibited
both suicide and aiding a suicide, with the prohibition on aiding
largely justified by the primary prohibition on self-inflicted death
itself, see, e.g., American Law Institute, Model Penal Code §210 Comment
1, pp.~92-93, and n.~7 (1980), the State's rejection of the traditional
treatment of the one leaves the criminality of the other open to
questioning that previously would not have been appropriate. The second
step in the argument is to emphasize that the State's own act of
decriminalization gives a freedom of choice much like the individual's
option in recognized instances of bodily autonomy. One of these,
abortion, is a legal right to choose in spite of the interest a State
may legitimately invoke in discouraging the practice, just as suicide is
now subject to choice, despite a state interest in discouraging it. The
third step is to emphasize that respondents claim a right to assistance
not on the basis of some broad principle that would be subject to
exceptions if that continuing interest of the State's in discouraging
suicide were to be recognized at all. Respondents base their claim on
the traditional right to medical care and counsel, subject to the
limiting conditions of informed, responsible choice when death is
imminent, conditions that support a strong analogy to rights of care in
other situations in which medical counsel and assistance have been
available as a matter of course. There can be no stronger claim to a
physician's assistance than at the time when death is imminent, a moral
judgment implied by the State's own recognition of the legitimacy of
medical procedures necessarily hastening the moment of impending death.

In my judgment, the importance of the individual interest here, as
within that class of ``certain interests'' demanding careful scrutiny of
the State's contrary claim, see Poe, cannot be gainsaid. Whether that
interest might in some circumstances, or at some time, be seen as
``fundamental'' to the degree entitled to prevail is not, however, a
conclusion that I need draw here, for I am satisfied that the State's
interests described in the following section are sufficiently serious to
defeat the present claim that its law is arbitrary or purposeless.

The State has put forward several interests to justify the Washington
law as applied to physicians treating terminally ill patients, even
those competent to make responsible choices: protecting life generally,
Brief for Petitioners 33, discouraging suicide even if knowing and
voluntary, -38, and protecting terminally ill patients from involuntary
suicide and euthanasia, both voluntary and nonvoluntary, -35.

It is not necessary to discuss the exact strengths of the first two
claims of justification in the present circumstances, for the third is
dispositive for me. That third justification is different from the first
two, for it addresses specific features of respondents' claim, and it
opposes that claim not with a moral judgment contrary to respondents',
but with a recognized state interest in the protection of nonresponsible
individuals and those who do not stand in relation either to death or to
their physicians as do the patients whom respondents describe. The State
claims interests in protecting patients from mistakenly and
involuntarily deciding to end their lives, and in guarding against both
voluntary and involuntary euthanasia. Leaving aside any difficulties in
coming to a clear concept of imminent death, mistaken decisions may
result from inadequate palliative care or a terminal prognosis that
turns out to be error; coercion and abuse may stem from the large
medical bills that family members cannot bear or unreimbursed hospitals
decline to shoulder. Voluntary and involuntary euthanasia may result
once doctors are authorized to prescribe lethal medication in the first
instance, for they might find it pointless to distinguish between
patients who administer their own fatal drugs and those who wish not to,
and their compassion for those who suffer may obscure the distinction
between those who ask for death and those who may be unable to request
it. The argument is that a progression would occur, obscuring the line
between the ill and the dying, and between the responsible and the
unduly influenced, until ultimately doctors and perhaps others would
abuse a limited freedom to aid suicides by yielding to the impulse to
end another's suffering under conditions going beyond the narrow limits
the respondents propose. The State thus argues, essentially, that
respondents' claim is not as narrow as it sounds, simply because no
recognition of the interest they assert could be limited to vindicating
those interests and affecting no others. The State says that the claim,
in practical effect, would entail consequences that the State could,
without doubt, legitimately act to prevent.

The mere assertion that the terminally sick might be pressured into
suicide decisions by close friends and family members would not alone be
very telling. Of course that is possible, not only because the costs of
care might be more than family members could bear but simply because
they might naturally wish to see an end of suffering for someone they
love. But one of the points of restricting any right of assistance to
physicians, would be to condition the right on an exercise of judgment
by someone qualified to assess the patient's responsible capacity and
detect the influence of those outside the medical relationship.

The State, however, goes further, to argue that dependence on the
vigilance of physicians will not be enough. First, the lines proposed
here (particularly the requirement of a knowing and voluntary decision
by the patient) would be more difficult to draw than the lines that have
limited other recently recognized due process rights. Limiting a state
from prosecuting use of artificial contraceptives by married couples
posed no practical threat to the State's capacity to regulate
contraceptives in other ways that were assumed at the time of Poe to be
legitimate; the trimester measurements of Roe and the viability
determination of Casey were easy to make with a real degree of
certainty. But the knowing and responsible mind is harder to assess. 16
Second, this difficulty could become the greater by combining with
another fact within the realm of plausibility, that physicians simply
would not be assiduous to preserve the line. They have compassion, and
those who would be willing to assist in suicide at all might be the most
susceptible to the wishes of a patient, whether the patient were
technically quite responsible or not. Physicians, and their hospitals,
have their own financial incentives, too, in this new age of managed
care. Whether acting from compassion or under some other influence, a
physician who would provide a drug for a patient to administer might
well go the further step of administering the drug himself; so, the
barrier between assisted suicide and euthanasia could become porous, and
the line between voluntary and involuntary euthanasia as well. 17 The
case for the slippery slope is fairly made out here, not because
recognizing one due process right would leave a court with no principled
basis to avoid recognizing another, but because there is a plausible
case that the right claimed would not be readily containable by
reference to facts about the mind that are matters of difficult
judgment, or by gatekeepers who are subject to temptation, noble or not.

Respondents propose an answer to all this, the answer of state
regulation with teeth. Legislation proposed in several States, for
example, would authorize physician-assisted suicide but require two
qualified physicians to confirm the patient's diagnosis, prognosis, and
competence; and would mandate that the patient make repeated requests
witnessed by at least two others over a specified time span; and would
impose reporting requirements and criminal penalties for various acts of
coercion. See App. to Brief for State Legislators as Amici Curiae 1a-2a.

But at least at this moment there are reasons for caution in predicting
the effectiveness of the teeth proposed. Respondents' proposals, as it
turns out, sound much like the guidelines now in place in the
Netherlands, the only place where experience with physician-assisted
suicide and euthanasia has yielded empirical evidence about how such
regulations might affect actual practice. Dutch physicians must engage
in consultation before proceeding, and must decide whether the patient's
decision is voluntary, well considered, and stable, whether the request
to die is enduring and made more than once, and whether the patient's
future will involve unacceptable suffering. See C. Gomez, Regulating
Death 40-43 (1991). There is, however, a substantial dispute today about
what the Dutch experience shows. Some commentators marshall evidence
that the Dutch guidelines have in practice failed to protect patients
from involuntary euthanasia and have been violated with impunity. See,
e.g., H. Hendin, Seduced By Death 75-84 (1997) (noting many cases in
which decisions intended to end the life of a fully competent patient
were made without a request from the patient and without consulting the
patient); Keown, Euthanasia in the Netherlands: Sliding Down the
Slippery Slope?, in Euthanasia Examined 261, 289 (J. Keown ed )
(guidelines have ``proved signally ineffectual; non-voluntary euthanasia
is now widely practised and increasingly condoned in the Netherlands'');
Gomez. This evidence is contested. See, e.g., R. Epstein, Mortal Peril
322 (1997) (``Dutch physicians are not euthanasia enthusiasts and they
are slow to practice it in individual cases''); R. Posner, Aging and Old
Age 242, and n.~23 (1995) (noting fear of ``doctors' rushing patients to
their death'' in the Netherlands ``has not been substantiated and does
not appear realistic''); Van der Wal, Van Eijk, Leenen, \&
Spreeuwenberg, Euthanasia and Assisted Suicide, 2, Do Dutch Family
Doctors Act Prudently?, 9 Family Practice 135 (1992) (finding no serious
abuse in Dutch practice). The day may come when we can say with some
assurance which side is right, but for now it is the substantiality of
the factual disagreement, and the alternatives for resolving it, that
matter. They are, for me, dispositive of the due process claim at this
time.

I take it that the basic concept of judicial review with its possible
displacement of legislative judgment bars any finding that a legislature
has acted arbitrarily when the following conditions are met: there is a
serious factual controversy over the feasibility of recognizing the
claimed right without at the same time making it impossible for the
State to engage in an undoubtedly legitimate exercise of power; facts
necessary to resolve the controversy are not readily ascertainable
through the judicial process; but they are more readily subject to
discovery through legislative factfinding and experimentation. It is
assumed in this case, and must be, that a State's interest in protecting
those unable to make responsible decisions and those who make no
decisions at all entitles the State to bar aid to any but a knowing and
responsible person intending suicide, and to prohibit euthanasia. How,
and how far, a State should act in that interest are judgments for the
State, but the legitimacy of its action to deny a physician the option
to aid any but the knowing and responsible is beyond question.

The capacity of the State to protect the others if respondents were to
prevail is, however, subject to some genuine question, underscored by
the responsible disagreement over the basic facts of the Dutch
experience. This factual controversy is not open to a judicial
resolution with any substantial degree of assurance at this time. It is
not, of course, that any controversy about the factual predicate of a
due process claim disqualifies a court from resolving it. Courts can
recognize captiousness, and most factual issues can be settled in a
trial court. At this point, however, the factual issue at the heart of
this case does not appear to be one of those. The principal enquiry at
the moment is into the Dutch experience, and I question whether an
independent front-line investigation into the facts of a foreign
country's legal administration can be soundly undertaken through
American courtroom litigation. While an extensive literature on any
subject can raise the hopes for judicial understanding, the literature
on this subject is only nascent. Since there is little experience
directly bearing on the issue, the most that can be said is that
whichever way the Court might rule today, events could overtake its
assumptions, as experimentation in some jurisdictions confirmed or
discredited the concerns about progression from assisted suicide to
euthanasia.

Legislatures, on the other hand, have superior opportunities to obtain
the facts necessary for a judgment about the present controversy. Not
only do they have more flexible mechanisms for factfinding than the
Judiciary, but their mechanisms include the power to experiment, moving
forward and pulling back as facts emerge within their own jurisdictions.
There is, indeed, good reason to suppose that in the absence of a
judgment for respondents here, just such experimentation will be
attempted in some of the States. See, e.g., Ore.Rev.Stat. Ann. §§127 et
seq.

---(Supp ); App. to Brief for State Legislators as Amici Curiae 1a
(listing proposed statutes).

I do not decide here what the significance might be of legislative
foot-dragging in ascertaining the facts going to the State's argument
that the right in question could not be confined as claimed. Sometimes a
court may be bound to act regardless of the institutional preferability
of the political branches as forums for addressing constitutional
claims. See, e.g., Bolling v. SharpeNow, it is enough to say that our
examination of legislative reasonableness should consider the fact that
the Legislature of the State of Washington is no more obviously at fault
than this Court is in being uncertain about what would happen if
respondents prevailed today. We therefore have a clear question about
which institution, a legislature or a court, is relatively more
competent to deal with an emerging issue as to which facts currently
unknown could be dispositive. The answer has to be, for the reasons
already stated, that the legislative process is to be preferred. There
is a closely related further reason as well.

One must bear in mind that the nature of the right claimed, if
recognized as one constitutionally required, would differ in no
essential way from other constitutional rights guaranteed by enumeration
or derived from some more definite textual source than ``due process.''
An unenumerated right should not therefore be recognized, with the
effect of displacing the legislative ordering of things, without the
assurance that its recognition would prove as durable as the recognition
of those other rights differently derived. To recognize a right of
lesser promise would simply create a constitutional regime too uncertain
to bring with it the expectation of finality that is one of this Court's
central obligations in making constitutional decisions. See Casey,
Legislatures, however, are not so constrained. The experimentation that
should be out of the question in constitutional adjudication displacing
legislative judgments is entirely proper, as well as highly desirable,
when the legislative power addresses an emerging issue like assisted
suicide. The Court should accordingly stay its hand to allow reasonable
legislative consideration. While I do not decide for all time that
respondents' claim should not be recognized, I acknowledge the
legislative institutional competence as the better one to deal with that
claim at this time.

\hypertarget{roe-v.-wade}{%
\subsubsection{Roe v. Wade}\label{roe-v.-wade}}

410 U.S. 113 (1973)

\textbf{Mr.~Justice BLACKMUN delivered the opinion of the Court.} This
Texas federal appeal and its Georgia companion, Doe v. Boltonpresent
constitutional challenges to state criminal abortion legislation. The
Texas statutes under attack here are typical of those that have been in
effect in many States for approximately a century. The Georgia statutes,
in contrast, have a modern cast and are a legislative product that, to
an extent at least, obviously reflects the influences of recent
attitudinal change, of advancing medical knowledge and techniques, and
of new thinking about an old issue.

We forthwith acknowledge our awareness of the sensitive and emotional
nature of the abortion controversy, of the vigorous opposing views, even
among physicians, and of the deep and seemingly absolute convictions
that the subject inspires. One's philosophy, one's experiences, one's
exposure to the raw edges of human existence, one's religious training,
one's attitudes toward life and family and their values, and the moral
standards one establishes and seeks to observe, are all likely to
influence and to color one's thinking and conclusions about abortion.

In addition, population growth, pollution, poverty, and racial overtones
tend to complicate and not to simplify the problem.

Our task, of course, is to resolve the issue by constitutional
measurement, free of emotion and of predilection. We seek earnestly to
do this, and, because we do, we have inquired into, and in this opinion
place some emphasis upon, medical and medical-legal history and what
that history reveals about man's attitudes toward the abortion procedure
over the centuries. We bear in mind, too, Mr.~Justice Holmes' admonition
in his now-vindicated dissent in Lochner v. New York, :

`(The Constitution) is made for people of fundamentally differing views,
and the accident of our finding certain opinions natural and familiar,
or novel, and even shocking, ought not to conclude our judgment upon the
question whether statutes embodying them conflict with the Constitution
of the United States.'

The Texas statutes that concern us here are Arts. 1191-1194 and 1196 of
the State's Penal Code,1 Vernon's Ann.P.C. These make it a crime to
`procure an abortion,' as therein defined, or to attempt one, except
with respect to `an abortion procured or attempted by medical advice for
the purpose of saving the life of the mother.' Similar statutes are in
existence in a majority of the States Texas first enacted a criminal
abortion statute in 1854. Texas Laws 1854, c.~49, § 1, set forth in 3 H.
Gammel, Laws of Texas 1502 (1898). This was soon modified into language
that has remained substantially unchanged to the present time. See Texas
Penal Code of 1857, c.~7, Arts. 531-536; G. Paschal, Laws of Texas,
Arts. 2192-2197 (1866); Texas Rev.Stat., c.~8, Arts. 536-541 (1879);
Texas Rev.Crim.Stat., Arts. 1071-1076 (1911). The final article in each
of these compilations provided the same exception, as does the present
Article 1196, for an abortion by `medical advice for the purpose of
saving the life of the mother.'

Jane Roe,4 a single woman who was residing in Dallas County, Texas,
instituted this federal action in March 1970 against the District
Attorney of the county. She sought a declaratory judgment that the Texas
criminal abortion statutes were unconstitutional on their face, and an
injunction restraining the defendant from enforcing the statutes.

Roe alleged that she was unmarried and pregnant; that she wished to
terminate her pregnancy by an abortion `performed by a competent,
licensed physician, under safe, clinical conditions'; that she was
unable to get a `legal' abortion in Texas because her life did not
appear to be threatened by the continuation of her pregnancy; and that
she could not afford to travel to another jurisdiction in order to
secure a legal abortion under safe conditions. She claimed that the
Texas statutes were unconstitutionally vague and that they abridged her
right of personal privacy, protected by the First, Fourth, Fifth, Ninth,
and Fourteenth Amendments. By an amendment to her complaint Roe
purported to sue `on behalf of herself and all other women' similarly
situated.

James Hubert Hallford, a licensed physician, sought and was granted
leave to intervene in Roe's action. In his complaint he alleged that he
had been arrested previously for violations of the Texas abortion
statutes and that two such prosecutions were pending against him. He
described conditions of patients who came to him seeking abortions, and
he claimed that for many cases he, as a physician, was unable to
determine whether they fell within or outside the exception recognized
by Article 1196. He alleged that, as a consequence, the statutes were
vague and uncertain, in violation of the Fourteenth Amendment, and that
they violated his own and his patients' rights to privacy in the
doctor-patient relationship and his own right to practice medicine,
rights he claimed were guaranteed by the First, Fourth, Fifth, Ninth,
and Fourteenth Amendments.

John and Mary Doe,5 a married couple, filed a companion complaint to
that of Roe. They also named the District Attorney as defendant, claimed
like constitutional deprivations, and sought declaratory and injunctive
relief. The Does alleged that they were a childless couple; that
Mrs.~Doe was suffering from a `neural-chemical' disorder; that her
physician had `advised her to avoid pregnancy until such time as her
condition has materially improved' (although a pregnancy at the present
time would not present `a serious risk' to her life); that, pursuant to
medical advice, she had discontinued use of birth control pills; and
that if she should become pregnant, she would want to terminate the
pregnancy by an abortion performed by a competent, licensed physician
under safe, clinical conditions. By an amendment to their complaint, the
Does purported to sue `on behalf of themselves and all couples similarly
situated.'

The two actions were consolidated and heard together by a duly convened
three-judge district court. The suits thus presented the situations of
the pregnant single woman, the childless couple, with the wife not
pregnant, and the licensed practicing physician, all joining in the
attack on the Texas criminal abortion statutes. Upon the filing of
affidavits, motions were made for dismissal and for summary judgment.
The court held that Roe and members of her class, and Dr.~Hallford, had
standing to sue and presented justiciable controversies, but that the
Does had failed to allege facts sufficient to state a present
controversy and did not have standing. It concluded that, with respect
to the requests for a declaratory judgment, abstention was not
warranted. On the merits, the District Court held that the `fundamental
right of single women and married persons to choose where to have
children is protected by the Ninth Amendment, through the Fourteenth
Amendment,' and that the Texas criminal abortion statutes were void on
their face because they were both unconstitutionally vague and
constituted an overbroad infringement of the plaintiffs' Ninth Amendment
rights. The court then held that abstention was warranted with respect
to the requests for an injunction. It therefore dismissed the Does'
complaint, declared the abortion statutes void, and dismissed the
application for injunctive relief. 314 F.Supp. 1217, 1225 (N.D.Tex ).

The plaintiffs Roe and Doe and the intervenor Hallford, pursuant to 28
U.S.C. § 1253, have appealed to this Court from that part of the
District Court's judgment denying the injunction. The defendant District
Attorney has purported to cross-appeal, pursuant to the same statute,
from the court's grant of declaratory relief to Roe and Hallford. Both
sides also have taken protective appeals to the United States Court of
Appeals for the Fifth Circuit. That court ordered the appeals held in
abeyance pending decision here. We postponed decision on jurisdiction to
the hearing on the merits. 402 U.S. 941, It might have been preferable
if the defendant, pursuant to our Rule 20, had presented to us a
petition for certiorari before judgment in the Court of Appeals with
respect to the granting of the plaintiffs' prayer for declaratory
relief. Our decisions in Mitchell v. Donovanand Gunn v. University
Committeeare to the effect that § 1253 does not authorize an appeal to
this Court from the grant or denial of declaratory relief alone. We
conclude, nevertheless, that those decisions do not foreclose our review
of both the injunctive and the declaratory aspects of a case of this
kind when it is property here, as this one is, on appeal under § 1253
from specific denial of injunctive relief, and the arguments as to both
aspects are necessarily identical. See Carter v. Jury Comm'n; Florida
Lime and Avocado Growers, Inc.~v. Jacobsen; 80-81, It would be
destructive of time and energy for all concerned were we to rule
otherwise. Cf. Doe v. Bolton.

We are next confronted with issues of justiciability, standing, and
abstention. Have Roe and the Does established that `personal stake in
the outcome of the controversy,' Baker v. Carr, that insures that `the
dispute sought to be adjudicated will be presented in an adversary
context and in a form historically viewed as capable of judicial
resolution,' Flast v. Cohen, and Sierra Club v. Morton, ? And what
effect did the pendency of criminal abortion charges against
Dr.~Hallford in state court have upon the propriety of the federal
court's granting relief to him as a plaintiff-intervenor?

A. Jane Roe. Despite the use of the pseudonym, no suggestion is made
that Roe is a fictitious person. For purposes of her case, we accept as
true, and as established, her existence; her pregnant state, as of the
inception of her suit in March 1970 and as late as May 21 of that year
when she filed an alias affidavit with the District Court; and her
inability to obtain a legal abortion in Texas.

Viewing Roe's case as of the time of its filing and thereafter until as
late as May, there can be little dispute that it then presented a case
or controversy and that, wholly apart from the class aspects, she, as a
pregnant single woman thwarted by the Texas criminal abortion laws, had
standing to challenge those statutes. Abele v. Markle, CA2 1971);
Crossen v. Breckenridge, -839 (CA6 1971); Poe v. Menghini, 339 F.Supp.
986, 990-991 (D.C.Kan. 1972). See Truax v. RaichIndeed, we do not read
the appellee's brief as really asserting anything to the contrary. The
`logical nexus between the status asserted and the claim sought to be
adjudicated,' Flast v. Cohen, and the necessary degree of
contentiousness, Golden v. Zwicklerare both present.

The appellee notes, however, that the record does not disclose that Roe
was pregnant at the time of the District Court hearing on May 22, 1970,6
or on the following June 17 when the court's opinion and judgment were
filed. And he suggests that Roe's case must now be moot because she and
all other members of her class are no longer subject to any 1970
pregnancy.

The usual rule in federal cases is that an actual controversy must exist
at stages of appellate or certiorari review, and not simply at the date
the action is initiated. United States v. Munsingwear, Inc.; Golden v.
Zwickler; SEC v. Medical Committee for Human Rights But when, as here,
pregnancy is a significant fact in the litigation, the normal 266-day
human gestation period is so short that the pregnancy will come to term
before the usual appellate process is complete. If that termination
makes a case moot, pregnancy litigation seldom will survive much beyond
the trial stage, and appellate review will be effectively denied. Our
law should not be that rigid. Pregnancy often comes more than once to
the same woman, and in the general population, if man is to survive, it
will always be with us. Pregnancy provides a classic justification for a
conclusion of nonmootness. It truly could be `capable of repetition, yet
evading review.' Southern Pacific Terminal Co.~v. ICC, See Moore v.
Ogilvie, ; Carroll v. President and Commissioners of Princess Anne, 351,
; United States v. W. T. Grant Co.,

We, therefore, agree with the District Court that Jane Roe had standing
to undertake this litigation, that she presented a justiciable
controversy, and that the termination of her 1970 pregnancy has not
rendered her case moot.

The doctor's position is different. He entered Roe's litigation as a
plaintiff-intervenor, alleging in his complaint that he:

`(I)n the past has been arrested for violating the Texas Abortion Laws
and at the present time stands charged by indictment with violating said
laws in the Criminal District Court of Dallas County, Texas to-wit: (1)
The State of Texas vs.~James H. Hallford, No.~C-69-5307-IH, and (2) The
State of Texas vs.~James H. Hallford, No.~C-69-2524-H. In both cases the
defendant is charged with abortion . .'

In his application for leave to intervene, the doctor made like
representations as to the abortion charges pending in the state court.
These representations were also repeated in the affidavit he executed
and filed in support of his motion for summary judgment.

Dr.~Hallford is, therefore, in the position of seeking, in a federal
court, declaratory and injunctive relief with respect to the same
statutes under which he stands charged in criminal prosecutions
simultaneously pending in state court. Although he stated that he has
been arrested in the past for violating the State's abortion laws, he
makes no allegation of any substantial and immediate threat to any
federally protected right that cannot be asserted in his defense against
the state prosecutions. Neither is there any allegation of harassment or
bad-faith prosecution. In order to escape the rule articulated in the
cases cited in the next paragraph of this opinion that, absent
harassment and bad faith, a defendant in a pending state criminal case
cannot affirmatively challenge in federal court the statutes under which
the State is prosecuting him, Dr.~Hallford seeks to distinguish his
status as a present state defendant from his status as a `potential
future defendant' and to assert only the latter for standing purposes
here.

We see no merit in that distinction. Our decision in Samuels v.
Mackellcompels the conclusion that the District Court erred when it
granted declaratory relief to Dr.~Hallford instead of refraining from so
doing. The court, of course, was correct in refusing to grant injunctive
relief to the doctor. The reasons supportive of that action, however,
are those expressed in Samuels v. Mackelland in Younger v.

Harris; Boyle v. Landry; Perez v. Ledesma; and Byrne v. KaralexisSee
also Dombrowski v. PfisterWe note, in passing, that Younger and its
companion cases were decided after the three-judge District Court
decision in this case.

Dr.~Hallford's complaint in intervention, therefore, is to be dismissed
He is remitted to his defenses in the state criminal proceedings against
him. We reverse the judgment of the District Court insofar as it granted
Dr.~Hallford relief and failed to dismiss his complaint in intervention.

C. The Does. In view of our ruling as to Roe's standing in her case, the
issue of the Does' standing in their case has little significance. The
claims they assert are essentially the same as those of Roe, and they
attack the same statutes. Nevertheless, we briefly note the Does'
posture.

Their pleadings present them as a childless married couple, the woman
not being pregnant, who have no desire to have children at this time
because of their having received medical advice that Mrs.~Doe should
avoid pregnancy, and for `other highly personal reasons.' But they `fear
. they may face the prospect of becoming parents.' And if pregnancy
ensues, they `would want to terminate' it by an abortion. They assert an
inability to obtain an abortion legally in Texas and, consequently, the
prospect of obtaining an illegal abortion there or of going outside
Texas to some place where the procedure could be obtained legally and
competently.

We thus have as plaintiffs a married couple who have, as their asserted
immediate and present injury, only an alleged `detrimental effect upon
(their) marital happiness' because they are forced to `the choice of
refraining from normal sexual relations or of endangering Mary Doe's
health through a possible pregnancy.' Their claim is that sometime in
the future Mrs.~Doe might become pregnant because of possible failure of
contraceptive measures, and at that time in the future she might want an
abortion that might then be illegal under the Texas statutes.

This very phrasing of the Does' position reveals its speculative
character. Their alleged injury rests on possible future contraceptive
failure, possible future pregnancy, possible future unpreparedness for
parenthood, and possible future impairment of health. Any one or more of
these several possibilities may not take place and all may not combine.
In the Does' estimation, these possibilities might have some real or
imagined impact upon their marital happiness. But we are not prepared to
say that the bare allegation of so indirect an injury is sufficient to
present an actual case or controversy. Younger v. Harris, 91 S.Ct.;
Golden v. Zwickler, 89 S.Ct.; Abele v. Markle, 452 F d; Crossen v.
Breckenridge, 446 F d.~The Does' claim falls far short of those resolved
otherwise in the cases that the Does urge upon us, namely, investment
Co.~Institute v. Camp; Association of Data Processing Service
Organizations, Inc.~v. Camp; and Epperson v. ArkansasSee also Truax v.
Raich The Does therefore are not appropriate plaintiffs in this
litigation. Their complaint was properly dismissed by the District
Court, and we affirm that dismissal.

The principal thrust of appellant's attack on the Texas statutes is that
they improperly invade a right, said to be possessed by the pregnant
woman, to choose to terminate her pregnancy. Appellant would discover
this right in the concept of personal `liberty' embodied in the
Fourteenth Amendment's Due Process Clause; or in personal marital,
familial, and sexual privacy said to be protected by the Bill of Rights
or its penumbras, see Griswold v. Connecticut; Eisenstadt v. Baird;
(White, J., concurring in result); or among those rights reserved to the
people by the Ninth Amendment, Griswold v. Connecticut, 85 S.Ct.
(Goldberg, J., concurring). Before addressing this claim, we feel it
desirable briefly to survey, in several aspects, the history of
abortion, for such insight as that history may afford us, and then to
examine the state purposes and interests behind the criminal abortion
laws.

It perhaps is not generally appreciated that the restrictive criminal
abortion laws in effect in a majority of States today are of relatively
recent vintage. Those laws, generally proscribing abortion or its
attempt at any time during pregnancy except when necessary to preserve
the pregnant woman's life, are not of ancient or even of common-law
origin. Instead, they derive from statutory changes effected, for the
most part, in the latter half of the 19th century.

1.. Ancient attitudes. These are not capable of precise determination.
We are told that at the time of the Persian Empire abortifacients were
known and that criminal abortions were severely punished We are also
told, however, that abortion was practiced in Greek times as well as in
the Roman Era,9 and that 'it was resorted to without scruple.'10 The
Ephesian, Soranos, often described as the greatest of the ancient
gynecologists, appears to have been generally opposed to Rome's
prevailing free-abortion practices. He found it necessary to think first
of the life of the mother, and he resorted to abortion when, upon this
standard, he felt the procedure advisable Greek and Roman law afforded
little protection to the unborn. If abortion was prosecuted in some
places, it seems to have been based on a concept of a violation of the
father's right to his offspring. Ancient religion did not bar abortion.

\begin{enumerate}
\def\labelenumi{\arabic{enumi}.}
\setcounter{enumi}{1}
\tightlist
\item
  The Hippocratic Oath. What then of the famous Oath that has stood so
  long as the ethical guide of the medical profession and that bears the
  name of the great Greek (460(?)-377(?) B.C.), who has been described
  as the Father of Medicine, the `wisest and the greatest practitioner
  of his art,' and the `most important and most complete medical
  personality of antiquity,' who dominated the medical schools of his
  time, and who typified the sum of the medical knowledge of the past?13
  The Oath varies somewhat according to the particular translation, but
  in any translation the content is clear: `I will give no deadly
  medicine to anyone if asked, nor suggest any such counsel; and in like
  manner I will not give to a woman a pessary to produce abortion,'14 or
  'I will neither give a deadly drug to anybody if asked for it, nor
  will I make a suggestion to this effect. Similarly, I will not give to
  a woman an abortive remedy.'
\end{enumerate}

Although the Oath is not mentioned in any of the principal briefs in
this case or in Doe v. Boltonit represents the apex of the development
of strict ethical concepts in medicine, and its influence endures to
this day. Why did not the authority of Hippocrates dissuade abortion
practice in his time and that of Rome? The late Dr.~Edelstein provides
us with a theory:16 The Oath was not uncontested even in Hippocrates'
day; only the Pythagorean school of philosophers frowned upon the
related act of suicide. Most Greek thinkers, on the other hand,
commended abortion, at least prior to viability. See Plato, Republic, V,
461; Aristotle, Politics, VII, 1335b 25. For the Pythagoreans, however,
it was a matter of dogma. For them the embryo was animate from the
moment of conception, and abortion meant destruction of a living being.
The abortion clause of the Oath, therefore, `echoes Pythagorean
doctrines,' and `(i)n no other stratum of Greek opinion were such views
held or proposed in the same spirit of uncompromising austerity.'

Dr.~Edelstein then concludes that the Oath originated in a group
representing only a small segment of Greek opinion and that it certainly
was not accepted by all ancient physicians. He points out that medical
writings down to Galen (A.D. 130-200) `give evidence of the violation of
almost every one of its injunctions.'18 But with the end of antiquity a
decided change took place. Resistance against suicide and against
abortion became common. The Oath came to be popular. The emerging
teachings of Christianity were in agreement with the Phthagorean ethic.
The Oath 'became the nucleus of all medical ethics' and `was applauded
as the embodiment of truth.' Thus, suggests Dr.~Edelstein, it is `a
Pythagorean manifesto and not the expression of an absolute standard of
medical conduct.'

This, it seems to us, is a satisfactory and acceptable explanation of
the Hippocratic Oath's apparent rigidity. It enables us to understand,
in historical context, a long-accepted and reversed statement of medical
ethics.

\begin{enumerate}
\def\labelenumi{\arabic{enumi}.}
\setcounter{enumi}{2}
\tightlist
\item
  The common law. It is undisputed that at common law, abortion
  performed before `quickening'-the first recognizable movement of the
  fetus in utero, appearing usually from the 16th to the 18th week of
  pregnancy20-was not an indictable offense The absence of a common-law
  crime for pre-quickening abortion appears to have developed from a
  confluence of earlier philosophical, theological, and civil and canon
  law concepts of when life begins. These disciplines variously
  approached the question in terms of the point at which the embryo or
  fetus became `formed' or recognizably human, or in terms of when a
  `person' came into being, that is, infused with a `soul' or
  `animated.' A loose concensus evolved in early English law that these
  events occurred at some point between conception and live birth This
  was `mediate animation.' Although Christian theology and the canon law
  came to fix the point of animation days for a male and 80 days for a
  female, a view that persisted until the 19th century, there was
  otherwise little agreement about the precise time of formation or
  animation. There was agreement, however, that prior to this point the
  fetus was to be regarded as part of the mother, and its destruction,
  therefore, was not homicide. Due to continued uncertainty about the
  precise time when animation occurred, to the lack of any empirical
  basis for the 40-80-day view, and perhaps to Aquinas' definition of
  movement as one of the two first principles of life, Bracton focused
  upon quickening as the critical point. The significance of quickening
  was echoed by later common-law scholars and found its way into the
  received common law in this country.
\end{enumerate}

Whether abortion of a quick fetus was a felony at common law, or even a
lesser crime, is still disputed. Bracton, writing early in the 13th
century, thought it homicide But the later and predominant view,
following the great common-law scholars, has been that it was, at most,
a lesser offense. In a frequently cited passage, Coke took the position
that abortion of a woman `quick with childe' is `a great misprision, and
no murder.'24 Blackstone followed, saying that while abortion after
quickening had once been considered manslaughter (though not murder),
'modern law' took a less severe view A recent review of the common-law
precedents argues, however, that those precedents contradict Coke and
that even post-quickening abortion was never established as a common-law
crime This is of some importance because while most American courts
ruled, in holding or dictum, that abortion of an unquickened fetus was
not criminal under their received common law,27 others followed Coke in
stating that abortion of a quick fetus was a `misprision,' a term they
translated to mean 'misdemeanor.'28 That their reliance on Coke on this
aspect of the law was uncritical and, apparently in all the reported
cases, dictum (due probably to the paucity of common-law prosecutions
for post-quickening abortion), makes it now appear doubtful that
abortion was ever firmly established as a common-law crime even with
respect to the destruction of a quick fetus.

\begin{enumerate}
\def\labelenumi{\arabic{enumi}.}
\setcounter{enumi}{3}
\tightlist
\item
  The English statutory law. England's first criminal abortion statute,
  Lord Ellenborough's Act, 43 Geo. 3, c.~58, came in 1803. It made
  abortion of a quick fetus, § 1, a capital crime, but in § 2 it
  provided lesser penalties for the felony of abortion before
  quickening, and thus preserved the `quickening' distinction. This
  contrast was continued in the general revision of 1828, 9 Geo. 4,
  c.~31, § 13. It disappeared, however, together with the death penalty,
  in 1837, 7 Will. 4 \& 1 Vict., c.~85, § 6, and did not reappear in the
  Offenses Against the Person Act of 1861, 24 \& 25 Vict., c.~100, § 59,
  that formed the core of English anti-abortion law until the
  liberalizing reforms of 1967. In 1929, the Infant Life (Preservation)
  Act, 19 \& 20 Geo. 5, c.~34, came into being. Its emphasis was upon
  the destruction of `the life of a child capable of being born alive.'
  It made a willful act performed with the necessary intent a felony. It
  contained a proviso that one was not to be found guilty of the offense
  `unless it is proved that the act which caused the death of the child
  was not done in good faith for the purpose only of preserving the life
  of the mother.'
\end{enumerate}

A seemingly notable development in the English law was the case of Rex
v. Bourne, (1939) 1 K.B. 687. This case apparently answered in the
affirmative the question whether an abortion necessary to preserve the
life of the pregnant woman was excepted from the criminal penalties of
the 1861 Act. In his instructions to the jury, Judge MacNaghten referred
to the 1929 Act, and observed that that Act related to `the case where a
child is killed by a willful act at the time when it is being delivered
in the ordinary course of nature.' He concluded that the 1861 Act's use
of the word `unlawfully,' imported the same meaning expressed by the
specific proviso in the 1929 Act, even though there was no mention of
preserving the mother's life in the 1861 Act. He then construed the
phrase `preserving the life of the mother' broadly, that is, `in a
reasonable sense,' to include a serious and permanent threat to the
mother's health, and instructed the jury to acquit Dr.~Bourne if it
found he had acted in a good-faith belief that the abortion was
necessary for this purpose. -694. The jury did acquit.

Recently, Parliament enacted a new abortion law. This is the Abortion
Act of 1967, 15 \& 16 Eliz. 2, c.~87. The Act permits a licensed
physician to perform an abortion where two other licensed physicians
agree (a) `that the continuance of the pregnancy would involve risk to
the life of the pregnant woman, or of injury to the physical or mental
health of the pregnant woman or any existing children of her family,
greater than if the pregnancy were terminated,' or (b) `that there is a
substantial risk that if the child were born it would suffer from such
physical or mental abnormalities as to be seriously handicapped.' The
Act also provides that, in making this determination, `account may be
taken of the pregnant woman's actual or reasonably foreseeable
environment.' It also permits a physician, without the concurrence of
others, to terminate a pregnancy where he is of the good-faith opinion
that the abortion `is immediately necessary to save the life or to
prevent grave permanent injury to the physical or mental health of the
pregnant woman.'

\begin{enumerate}
\def\labelenumi{\arabic{enumi}.}
\setcounter{enumi}{4}
\tightlist
\item
  The American law. In this country, the law in effect in all but a few
  States until mid-19th century was the pre-existing English common law.
  Connecticut, the first State to enact abortion legislation, adopted in
  1821 that part of Lord Ellenborough's Act that related to a woman
  `quick with child.'29 The death penalty was not imposed. Abortion
  before quickening was made a crime in that State only in 1860 In 1828,
  New York enacted legislation31 that, in two respects, was to serve as
  a model for early anti-abortion statutes. First, while barring
  destruction of an unquickened fetus as well as a quick fetus, it made
  the former only a misdemeanor, but the latter second-degree
  manslaughter. Second, it incorporated a concept of therapeutic
  abortion by providing that an abortion was excused if it 'shall have
  been necessary to preserve the life of such mother, or shall have been
  advised by two physicians to be necessary for such purpose.' By 1840,
  when Texas had received the common law,32 only eight American States
  had statutes dealing with abortion It was not until after the War
  Between the States that legislation began generally to replace the
  common law. Most of these initial statutes dealt severely with
  abortion after quickening but were lenient with it before quickening.
  Most punished attempts equally with completed abortions. While many
  statutes included the exception for an abortion thought by one or more
  physicians to be necessary to save the mother's life, that provision
  soon disappeared and the typical law required that the procedure
  actually be necessary for that purpose.
\end{enumerate}

Gradually, in the middle and late 19th century the quickening
distinction disappeared from the statutory law of most States and the
degree of the offense and the penalties were increased. By the end of
the 1950's a large majority of the jurisdictions banned abortion,
however and whenever performed, unless done to save or preserve the life
of the mother The exceptions, Alabama and the District of Columbia,
permitted abortion to preserve the mother's health. 35 Three States
permitted abortions that were not `unlawfully' performed or that were
not `without lawful justification,' leaving interpretation of those
standards to the courts In the past several years, however, a trend
toward liberalization of abortion statutes has resulted in adoption, by
about one-third of the States, of less stringent laws, most of them
patterned after the ALI Model Penal Code, § 230 ,37 set forth as
Appendix B to the opinion in Doe v. Bolton.

It is thus apparent that at common law, at the time of the adoption of
our Constitution, and throughout the major portion of the 19th century,
abortion was viewed with less disfavor than under most American statutes
currently in effect. Phrasing it another way, a woman enjoyed a
substantially broader right to terminate a pregnancy than she does in
most States today. At least with respect to the early stage of
pregnancy, and very possibly without such a limitation, the opportunity
to make this choice was present in this country well into the 19th
century. Even later, the law continued for some time to treat less
punitively an abortion procured in early pregnancy.

\begin{enumerate}
\def\labelenumi{\arabic{enumi}.}
\setcounter{enumi}{5}
\tightlist
\item
  The position of the American Medical Association. The anti-abortion
  mood prevalent in this country in the late 19th century was shared by
  the medical profession. Indeed, the attitude of the profession may
  have played a significant role in the enactment of stringent criminal
  abortion legislation during that period.
\end{enumerate}

An AMA Committee on Criminal Abortion was appointed in May 1857. It
presented its report, 12 Trans. of the Am.Med.Assn. 73-78 (1859), to the
Twelfth Annual Meeting. That report observed that the Committee had been
appointed to investigate criminal abortion `with a view to its general
suppression.' It deplored abortion and its frequency and it listed three
causes of `this general demoralization':

`The first of these causes is a wide-spread popular ignorance of the
true character of the crime-a belief, even among mothers themselves,
that the foetus is not alive till after the period of quickening.'The
second of the agents alluded to is the fact that the profession
themselves are frequently supposed careless of foetal life. . .'The
third reason of the frightful extent of this crime is found in the grave
defects of our laws, both common and statute, as regards the independent
and actual existence of the child before birth, as a living being. These
errors, which are sufficient in most instances to prevent conviction,
are based, and only based, upon mistaken and exploded medical dogmas.
With strange inconsistency, the law fully acknowledges the foetus in
utero and its inherent rights, for civil purposes; while personally and
as criminally affected, it fails to recognize it, and to its life as yet
denies all protection.'

The Committee then offered, and the Association adopted, resolutions
protesting `against such unwarrantable destruction of human life,'
calling upon state legislatures to revise their abortion laws, and
requesting the cooperation of state medical societies `in pressing the
subject.'

In 1871 a long and vivid report was submitted by the Committee on
Criminal Abortion. It ended with the observation, `We had to deal with
human life. In a matter of less importance we could entertain no
compromise. An honest judge on the bench would call things by their
proper names. We could do no less.' 22 Trans. of the Am.Med.Assn. 258
(1871). It proffered resolutions, adopted by the Association, -39,
recommending, among other things, that it `be unlawful and
unprofessional for any physician to induce abortion or premature labor,
without the concurrent opinion of at least one respectable consulting
physician, and then always with a view to the safety of the child-if
that be possible,' and calling `the attention of the clergy of all
denominations to the perverted views of morality entertained by a large
class of females-aye, and men also, on this important question.'

Except for periodic condemnation of the criminal abortionist, no further
formal AMA action took place until 1967. In that year, the Committee on
Human Reproduction urged the adoption of a stated policy of opposition
to induced abortion, except when there is `documented medical evidence'
of a threat to the health or life of the mother, or that the child `may
be born with incapacitating physical deformity or mental deficiency,' or
that a pregnancy `resulting from legally established statutory or
forcible rape or incest may constitute a threat to the mental or
physical health of the patient,' two other physicians `chosen because of
their recognized professional competency have examined the patient and
have concurred in writing,' and the procedure `is performed in a
hospital accredited by the Joint Commission on Accreditation of
Hospitals.' The providing of medical information by physicians to state
legislatures in their consideration of legislation regarding therapeutic
abortion was `to be considered consistent with the principles of ethics
of the American Medical Association.' This recommendation was adopted by
the House of Delegates. Proceedings of the AMA House of Delegates 40-51
(June 1967).

In 1970, after the introduction of a variety of proposed resolutions,
and of a report from its Board of Trustees, a reference committee noted
`polarization of the medical profession on this controversial issue';
division among those who had testified; a difference of opinion among
AMA councils and committees; `the remarkable shift in testimony' in six
months, felt to be influenced `by the rapid changes in state laws and by
the judicial decisions which tend to make abortion more freely
available;' and a feeling `that this trend will continue.' On June 25,
1970, the House of Delegates adopted preambles and most of the
resolutions proposed by the reference committee. The preambles
emphasized `the best interests of the patient,' `sound clinical
judgment,' and `informed patient consent,' in contrast to `mere
acquiescence to the patient's demand.' The resolutions asserted that
abortion is a medical procedure that should be performed by a licensed
physician in an accredited hospital only after consultation with two
other physicians and in conformity with state law, and that no party to
the procedure should be required to violate personally held moral
principles Proceedings of the AMA House of Delegates 220 (June 1970).
The AMA Judicial Council rendered a complementary opinion.

\begin{enumerate}
\def\labelenumi{\arabic{enumi}.}
\setcounter{enumi}{6}
\tightlist
\item
  The position of the American Public Health Association. In October
  1970, the Executive Board of the APHA adopted Standards for Abortion
  Services. These were five in number:
\end{enumerate}

`a. Rapid and simple abortion referral must be readily available through
state and local public health departments, medical societies, or other
non-profit organizations.'b. An important function of counseling should
be to simplify and expedite the provision of abortion services; if
should not delay the obtaining of these services.'c.~Psychiatric
consultation should not be mandatory. As in the case of other
specialized medical services, psychiatric consultation should be sought
for definite indications and not on a routine basis.'d.~A wide range of
individuals from appropriately trained, sympathetic volunteers to highly
skilled physicians may qualify as abortion counselors.'e. Contraception
and/or sterilization should be discussed with each abortion patient.'
Recommended Standards for Abortion Services, 61 Am.J.Pub.Health 396
(1971).

Among factors pertinent to life and health risks associated with
abortion were three that `are recognized as important':

`a. the skill of the physician,'b. the environment in which the abortion
is performed, and above all'c.~The duration of pregnancy, as determined
by uterine size and confirmed by menstrual history.'

It was said that `a well-equipped hospital' offers more protection `to
cope with unforeseen difficulties than an office or clinic without such
resources. . The factor of gestational age is of overriding importance.'
Thus, it was recommended that abortions in the second trimester and
early abortions in the presence of existing medical complications be
performed in hospitals as inpatient procedures. For pregnancies in the
first trimester, abortion in the hospital with or without overnight stay
`is probably the safest practice.' An abortion in an extramural
facility, however, is an acceptable alternative `provided arrangements
exist in advance to admit patients promptly if unforeseen complications
develop.' Standards for an abortion facility were listed. It was said
that at present abortions should be performed by physicians or
osteopaths who are licensed to practice and who have `adequate
training.'

\begin{enumerate}
\def\labelenumi{\arabic{enumi}.}
\setcounter{enumi}{7}
\tightlist
\item
  The position of the American Bar Association. At its meeting in
  February 1972 the ABA House of Delegates approved, with 17 opposing
  votes, the Uniform Abortion Act that had been drafted and approved the
  preceding August by the Conference of Commissioners on Uniform State
  Laws. 58 A.B.A.J. 380 (1972). We set forth the Act in full in the
  margin The Conference has appended an enlightening Prefatory Note.
\end{enumerate}

Three reasons have been advanced to explain historically the enactment
of criminal abortion laws in the 19th century and to justify their
continued existence.

It has been argued occasionally that these laws were the product of a
Victorian social concern to discourage illicit sexual conduct. Texas,
however, does not advance this justification in the present case, and it
appears that no court or commentator has taken the argument seriously
The appellants and amici contend, moreover, that this is not a proper
state purpose at all and suggest that, if it were, the Texas statutes
are overbroad in protecting it since the law fails to distinguish
between married and unwed mothers.

A second reason is concerned with abortion as a medical procedure. When
most criminal abortion laws were first enacted, the procedure was a
hazardous one for the woman This was particularly true prior to the
development of antisepsis. Antiseptic techniques, of course, were based
on discoveries by Lister, Pasteur, and others first announced in 1867,
but were not generally accepted and employed until about the turn of the
century. Abortion mortality was high. Even after 1900, and perhaps until
as late as the development of antibiotics in the 1940's, standard modern
techniques such as dilation and curettage were not nearly so safe as
they are today. Thus, it has been argued that a State's real concern in
enacting a criminal abortion law was to protect the pregnant woman, that
is, to restrain her from submitting to a procedure that placed her life
in serious jeopardy.

Modern medical techniques have altered this situation. Appellants and
various amici refer to medical data indicating that abortion in early
pregnancy, that is, prior to the end of the first trimester, although
not without its risk, is now relatively safe. Mortality rates for women
undergoing early abortions, where the procedure is legal, appear to be
as low as or lower than the rates for normal childbirth Consequently,
any interest of the State in protecting the woman from an inherently
hazardous procedure, except when it would be equally dangerous for her
to forgo it, has largely disappeared. Of course, important state
interests in the areas of health and medical standards do remain.

The State has a legitimate interest in seeing to it that abortion, like
any other medical procedure, is performed under circumstances that
insure maximum safety for the patient. This interest obviously extends
at least to the performing physician and his staff, to the facilities
involved, to the availability of after-care, and to adequate provision
for any complication or emergency that might arise. The prevalence of
high mortality rates at illegal `abortion mills' strengthens, rather
than weakens, the State's interest in regulating the conditions under
which abortions are performed. Moreover, the risk to the woman increases
as her pregnancy continues. Thus, the State retains a definite interest
in protecting the woman's own health and safety when an abortion is
proposed at a late stage of pregnancy,

The third reason is the State's interest-some phrase it in terms of
duty-in protecting prenatal life. Some of the argument for this
justification rests on the theory that a new human life is present from
the moment of conception. 45 The State's interest and general obligation
to protect life then extends, it is argued, to prenatal life. Only when
the life of the pregnant mother herself is at stake, balanced against
the life she carries within her, should the interest of the embryo or
fetus not prevail. Logically, of course, a legitimate state interest in
this area need not stand or fall on acceptance of the belief that life
begins at conception or at some other point prior to life birth. In
assessing the State's interest, recognition may be given to the less
rigid claim that as long as at least potential life is involved, the
State may assert interests beyond the protection of the pregnant woman
alone.

Parties challenging state abortion laws have sharply disputed in some
courts the contention that a purpose of these laws, when enacted, was to
protect prenatal life Pointing to the absence of legislative history to
support the contention, they claim that most state laws were designed
solely to protect the woman. Because medical advances have lessened this
concern, at least with respect to abortion in early pregnancy, they
argue that with respect to such abortions the laws can no longer be
justified by any state interest. There is some scholarly support for
this view of original purpose The few state courts called upon to
interpret their laws in the late 19th and early 20th centuries did focus
on the State's interest in protecting the woman's health rather than in
preserving the embryo and fetus Proponents of this view point out that
in many States, including Texas,49 by statute or judicial
interpretation, the pregnant woman herself could not be prosecuted for
self-abortion or for cooperating in an abortion performed upon her by
another They claim that adoption of the `quickening' distinction through
received common law and state statutes tacitly recognizes the greater
health hazards inherent in late abortion and impliedly repudiates the
theory that life begins at conception.

It is with these interests, and the weight to be attached to them, that
this case is concerned.

The Constitution does not explicitly mention any right of privacy. In a
line of decisions, however, going back perhaps as far as Union Pacific
R. Co.~v. Botsford, the Court has recognized that a right of personal
privacy, or a guarantee of certain areas or zones of privacy, does exist
under the Constitution. In varying contexts, the Court or individual
Justices have, indeed, found at least the roots of that right in the
First Amendment, Stanley v. Georgia, ; in the Fourth and Fifth
Amendments, Terry v. Ohio, Katz v. United States, ; Boyd v. United
Statessee Olmstead v. United States, (Brandeis, J., dissenting); in the
penumbras of the Bill of Rights, Griswold v. Connecticut, 85 S.Ct.; in
the Ninth Amendment, 85 S.Ct. (Goldberg, J., concurring); or in the
concept of liberty guaranteed by the first section of the Fourteenth
Amendment, see Meyer v. Nebraska, These decisions make it clear that
only personal rights that can be deemed `fundamental' or `implicit in
the concept of ordered liberty,' Palko v. Connecticut, are included in
this guarantee of personal privacy. They also make it clear that the
right has some extension to activities relating to marriage, Loving v.
Virginia, ; procreation, Skinner v. Oklahoma, ; contraception,
Eisenstadt v. Baird, 92 S.Ct.; 463-, 1043-1044 (White, J., concurring in
result); family relationships, Prince v. Massachusetts, ; and child
rearing and education, Pierce v. Society of Sisters, Meyer v. Nebraska.

This right of privacy, whether it be founded in the Fourteenth
Amendment's concept of personal liberty and restrictions upon state
action, as we feel it is, or, as the District Court determined, in the
Ninth Amendment's reservation of rights to the people, is broad enough
to encompass a woman's decision whether or not to terminate her
pregnancy. The detriment that the State would impose upon the pregnant
woman by denying this choice altogether is apparent. Specific and direct
harm medically diagnosable even in early pregnancy may be involved.
Maternity, or additional offspring, may force upon the woman a
distressful life and future. Psychological harm may be imminent. Mental
and physical health may be taxed by child care. There is also the
distress, for all concerned, associated with the unwanted child, and
there is the problem of bringing a child into a family already unable,
psychologically and otherwise, to care for it. In other cases, as in
this one, the additional difficulties and continuing stigma of unwed
motherhood may be involved. All these are factors the woman and her
responsible physician necessarily will consider in consultation.

On the basis of elements such as these, appellant and some amici argue
that the woman's right is absolute and that she is entitled to terminate
her pregnancy at whatever time, in whatever way, and for whatever reason
she alone chooses. With this we do not agree. Appellant's arguments that
Texas either has no valid interest at all in regulating the abortion
decision, or no interest strong enough to support any limitation upon
the woman's sole determination, are unpersuasive. The Court's decisions
recognizing a right of privacy also acknowledge that some state
regulation in areas protected by that right is appropriate. As noted
above, a State may properly assert important interests in safeguarding
health, in maintaining medical standards, and in protecting potential
life. At some point in pregnancy, these respective interests become
sufficiently compelling to sustain regulation of the factors that govern
the abortion decision. The privacy right involved, therefore, cannot be
said to be absolute. In fact, it is not clear to us that the claim
asserted by some amici that one has an unlimited right to do with one's
body as one pleases bears a close relationship to the right of privacy
previously articulated in the Court's decisions. The Court has refused
to recognize an unlimited right of this kind in the past. Jacobson v.
Massachusetts (vaccination); Buck v. Bell (sterilization).

We, therefore, conclude that the right of personal privacy includes the
abortion decision, but that this right is not unqualified and must be
considered against important state interests in regulation.

We note that those federal and state courts that have recently
considered abortion law challenges have reached the same conclusion. A
majority, in addition to the District Court in the present case, have
held state laws unconstitutional, at least in part, because of vagueness
or because of overbreadth and abridgment of rights. Others have
sustained state statutes.

Although the results are divided, most of these courts have agreed that
the right of privacy, however based, is broad enough to cover the
abortion decision; that the right, nonetheless, is not absolute and is
subject to some limitations; and that at some point the state interests
as to protection of health, medical standards, and prenatal life, become
dominant. We agree with this approach.

Where certain `fundamental rights' are involved, the Court has held that
regulation limiting these rights may be justified only by a `compelling
state interest,' and that legislative enactments must be narrowly drawn
to express only the legitimate state interests at stake.

In the recent abortion cases, cited above, courts have recognized these
principles. Those striking down state laws have generally scrutinized
the State's interests in protecting health and potential life, and have
concluded that neither interest justified broad limitations on the
reasons for which a physician and his pregnant patient might decide that
she should have an abortion in the early stages of pregnancy. Courts
sustaining state laws have held that the State's determinations to
protect health or prenatal life are dominant and constitutionally
justifiable.

The District Court held that the appellee failed to meet his burden of
demonstrating that the Texas statute's infringement upon Roe's rights
was necessary to support a compelling state interest, and that, although
the appellee presented `several compelling justifications for state
presence in the area of abortions,' the statutes outstripped these
justifications and swept `far beyond any areas of compelling state
interest.' 314 F.Supp.. Appellant and appellee both contest that
holding. Appellant, as has been indicated, claims an absolute right that
bars any state imposition of criminal penalties in the area. Appellee
argues that the State's determination to recognize and protect prenatal
life from and after conception constitutes a compelling state interest.
As noted above, we do not agree fully with either formulation.

A. The appellee and certain amici argue that the fetus is a `person'
within the language and meaning of the Fourteenth Amendment. In support
of this, they outline at length and in detail the well-known facts of
fetal development. If this suggestion of personhood is established, the
appellant's case, of course, collapses, for the fetus' right to life
would then be guaranteed specifically by the Amendment. The appellant
conceded as much on reargument On the other hand, the appellee conceded
on reargument52 that no case could be cited that holds that a fetus is a
person within the meaning of the Fourteenth Amendment.

The Constitution does not define `person' in so many words. Section 1 of
the Fourteenth Amendment contains three references to `person.' The
first, in defining `citizens,' speaks of `persons born or naturalized in
the United States.' The word also appears both in the Due Process Clause
and in the Equal Protection Clause. `Person' is used in other places in
the Constitution: in the listing of qualifications for Representatives
and Senators, Art, I, § 2, cl. 2, and § 3, cl. 3; in the Apportionment
Clause, Art. I, § 2, cl. 3 in the Migration and Importation provision,
Art. I, § 9, cl. 1; in the Emoulument Clause, Art, I, § 9, cl. 8; in the
Electros provisions, Art. II, § 1, cl. 2, and the superseded cl. 3; in
the provision outlining qualifications for the office of President, Art.
II, § 1, cl. 5; in the Extradition provisions, Art. IV, § 2, cl. 2, and
the superseded Fugitive Slave Clause 3; and in the Fifth, Twelfth, and
Twenty-second Amendments, as well as in §§ 2 and 3 of the Fourteenth
Amendment. But in nearly all these instances, the use of the word is
such that it has application only postnatally. None indicates, with any
assurance, that it has any possible prenatal application All this,
together with our observationthat throughout the major portion of the
19th century prevailing legal abortion practices were far freer than
they are today, persuades us that the word `person,' as used in the
Fourteenth Amendment, does not include the unborn This is in accord with
the results reached in those few cases where the issue has been squarely
presented. McGarvey v. Magee-Womens Hospital, 340 F.Supp. 751 (W.D.Pa );
Byrn v. New York City Health \& Hospitals Corp., 31 N.Y d 194, 335 N.Y.S
d 390, 286 N.E d 887 (1972), appeal docketed, No.~72-434; Abele v.
Markle, 351 F.Supp. 224 (D.C.Conn ), appeal docketed, No.~72-730. Cf.
Cheaney v. State, Ind., 285 N.E d; Montana v. Rogers, CA7 1960), aff'd
sub nom. Montana v. Kennedy; Keeler v. Superior Court, 2 Cal d 619, 87
Cal.Rptr. 481, 470 P d 617 (1970); State v. Dickinson, 28 Ohio St d 65,
275 N.E d 599 (1971). Indeed, our decision in United States v.
Vuitchinferentially is to the same effect, for we there would not have
indulged in statutory interpretation favorable to abortion in specified
circumstances if the necessary consequence was the termination of life
entitled to Fourteenth Amendment protection.

This conclusion, however, does not of itself fully answer the
contentions raised by Texas, and we pass on to other considerations.

B. The pregnant woman cannot be isolated in her privacy. She carries an
embryo and, later, a fetus, if one accepts the medical definitions of
the developing young in the human uterus. See Dorland's Illustrated
Medical Dictionary 478(24th ed.~1965). The situation therefore is
inherently different from marital intimacy, or bedroom possession of
obscene material, or marriage, or procreation, or education, with which
Eisenstadt and Griswold, Stanley, Loving, Skinner and Pierce and Meyer
were respectively concerned. As we have intimated above, it is
reasonable and appropriate for a State to decide that at some point in
time another interest, that of health of the mother or that of potential
human life, becomes significantly involved. The woman's privacy is no
longer sole and any right of privacy she possesses must be measured
accordingly.

Texas urges that, apart from the Fourteenth Amendment, life begins at
conception and is present throughout pregnancy, and that, therefore, the
State has a compelling interest in protecting that life from and after
conception. We need not resolve the difficult question of when life
begins. When those trained in the respective disciplines of medicine,
philosophy, and theology are unable to arrive at any consensus, the
judiciary, at this point in the development of man's knowledge, is not
in a position to speculate as to the answer.

It should be sufficient to note briefly the wide divergence of thinking
on this most sensitive and difficult question. There has always been
strong support for the view that life does not begin until live birth.
This was the belief of the Stoics It appears to be the predominant,
though not the unanimous, attitude of the Jewish faith It may be taken
to represent also the position of a large segment of the Protestant
community, insofar as that can be ascertained; organized groups that
have taken a formal position on the abortion issue have generally
regarded abortion as a matter for the conscience of the individual and
her family As we have noted, the common law found greater significance
in quickening. Physicians and their scientific colleagues have regarded
that event with less interest and have tended to focus either upon
conception, upon live birth, or upon the interim point at which the
fetus becomes `viable,' that is, potentially able to live outside the
mother's womb, albeit with artificial aid Viability is usually placed at
about seven months (28 weeks) but may occur earlier, even weeks The
Aristotelian theory of `mediate animation,' that held sway throughout
the Middle Ages and the Renaissance in Europe, continued to be official
Roman Catholic dogma until the 19th century, despite opposition to this
`ensoulment' theory from those in the Church who would recognize the
existence of life from the moment of conception The latter is now, of
course, the official belief of the Catholic Church. As one brief amicus
discloses, this is a view strongly held by many non-Catholics as well,
and by many physicians. Substantial problems for precise definition of
this view are posed, however, by new embryological data that purport to
indicate that conception is a `process' over time, rather than an event,
and by new medical techniques such as menstrual extraction, the
`morning-after' pill, implantation of embryos, artificial insemination,
and even artificial wombs.

In areas other than criminal abortion, the law has been reluctant to
endorse any theory that life, as we recognize it, begins before life
birth or to accord legal rights to the unborn except in narrowly defined
situations and except when the rights are contingent upon life birth.
For example, the traditional rule of tort law denied recovery for
prenatal injuries even though the child was born alive That rule has
been changed in almost every jurisdiction. In most States, recovery is
said to be permitted only if the fetus was viable, or at least quick,
when the injuries were sustained, though few courts have squarely so
held In a recent development, generally opposed by the commentators,
some States permit the parents of a stillborn child to maintain an
action for wrongful death because of prenatal injuries. 65 Such an
action, however, would appear to be one to vindicate the parents'
interest and is thus consistent with the view that the fetus, at most,
represents only the potentiality of life. Similarly, unborn children
have been recognized as acquiring rights or interests by way of
inheritance or other devolution of property, and have been represented
by guardians ad litem Perfection of the interests involved, again, has
generally been contingent upon live birth. In short, the unborn have
never been recognized in the law as persons in the whole sense.

In view of all this, we do not agree that, by adopting one theory of
life, Texas may override the rights of the pregnant woman that are at
stake. We repeat, however, that the State does have an important and
legitimate interest in preserving and protecting the health of the
pregnant woman, whether she be a resident of the State or a non-resident
who seeks medical consultation and treatment there, and that it has
still another important and legitimate interest in protecting the
potentiality of human life. These interests are separate and distinct.
Each grows in substantiality as the woman approaches term and, at a
point during pregnancy, each becomes `compelling.'

With respect to the State's important and legitimate interest in the
health of the mother, the `compelling' point, in the light of present
medical knowledge, is at approximately the end of the first trimester.
This is so because of the now-established medical fact, referred to
above, that until the end of the first trimester mortality in abortion
may be less than mortality in normal childbirth. It follows that, from
and after this point, a State may regulate the abortion procedure to the
extent that the regulation reasonably relates to the preservation and
protection of maternal health. Examples of permissible state regulation
in this area are requirements as to the qualifications of the person who
is to perform the abortion; as to the licensure of that person; as to
the facility in which the procedure is to be performed, that is, whether
it must be a hospital or may be a clinic or some other place of
less-than-hospital status; as to the licensing of the facility; and the
like.

This means, on the other hand, that, for the period of pregnancy prior
to this `compelling' point, the attending physician, in consultation
with his patient, is free to determine, without regulation by the State,
that, in his medical judgment, the patient's pregnancy should be
terminated. If that decision is reached, the judgment may be effectuated
by an abortion free of interference by the State.

With respect to the State's important and legitimate interest in
potential life, the `compelling' point is at viability. This is so
because the fetus then presumably has the capability of meaningful life
outside the mother's womb. State regulation protective of fetal life
after viability thus has both logical and biological justifications. If
the State is interested in protecting fetal life after viability, it may
go so far as to proscribe abortion during that period, except when it is
necessary to preserve the life or health of the mother.

Measured against these standards, Art. 1196 of the Texas Penal Code, in
restricting legal abortions to those `procured or attempted by medical
advice for the purpose of saving the life of the mother,' sweeps too
broadly. The statute makes no distinction between abortions performed
early in pregnancy and those performed later, and it limits to a single
reason, `saving' the mother's life, the legal justification for the
procedure. The statute, therefore, cannot survive the constitutional
attack made upon it here.

This conclusion makes it unnecessary for us to consider the additional
challenge to the Texas statute asserted on grounds of vagueness. See
United States v. Vuitch,

To summarize and to repeat:

\begin{enumerate}
\def\labelenumi{\arabic{enumi}.}
\tightlist
\item
  A state criminal abortion statute of the current Texas type, that
  excepts from criminality only a life-saving procedure on behalf of the
  mother, without regard to pregnancy stage and without recognition of
  the other interests involved, is violative of the Due Process Clause
  of the Fourteenth Amendment.
\end{enumerate}

\begin{enumerate}
\def\labelenumi{(\alph{enumi})}
\tightlist
\item
  For the stage prior to approximately the end of the first trimester,
  the abortion decision and its effectuation must be left to the medical
  judgment of the pregnant woman's attending physician.(b) For the stage
  subsequent to approximately the end of the first trimester, the State,
  in promoting its interest in the health of the mother, may, if it
  chooses, regulate the abortion procedure in ways that are reasonably
  related to maternal health.(c) For the stage subsequent to viability,
  the State in promoting its interest in the potentiality of human life
  may, if it chooses, regulate, and even proscribe, abortion except
  where it is necessary, in appropriate medical judgment, for the
  preservation of the life or health of the mother.
\end{enumerate}

\begin{enumerate}
\def\labelenumi{\arabic{enumi}.}
\setcounter{enumi}{1}
\tightlist
\item
  The State may define the term `physician,' as it has been employed in
  the preceding paragraphs of this Part of this opinion, to mean only a
  physician currently licensed by the State, and may proscribe any
  abortion by a person who is not a physician as so defined.
\end{enumerate}

In Doe v. Boltonprocedural requirements contained in one of the modern
abortion statutes are considered. That opinion and this one, of course,
are to be read together.

This holding, we feel, is consistent with the relative weights of the
respective interests involved, with the lessons and examples of medical
and legal history, with the lenity of the common law, and with the
demands of the profound problems of the present day. The decision leaves
the State free to place increasing restrictions on abortion as the
period of pregnancy lengthens, so long as those restrictions are
tailored to the recognized state interests. The decision vindicates the
right of the physician to administer medical treatment according to his
professional judgment up to the points where important state interests
provide compelling justifications for intervention. Up to those points,
the abortion decision in all its aspects is inherently, and primarily, a
medical decision, and basic responsibility for it must rest with the
physician. If an individual practitioner abuses the privilege of
exercising proper medical judgment, the usual remedies, judicial and
intra-professional, are available.

Our conclusion that Art. 1196 is unconstitutional means, of course, that
the Texas abortion statutes, as a unit, must fall. The exception of Art.
1196 cannot be struck down separately, for then the State would be left
with a statute proscribing all abortion procedures no matter how
medically urgent the case.

Although the District Court granted appellant Roe declaratory relief, it
stopped short of issuing an injunction against enforcement of the Texas
statutes. The Court has recognized that different considerations enter
into a federal court's decision as to declaratory relief, on the one
hand, and injunctive relief, on the other. Zwickler v. Koota, 389 U.S
241, 252-255, ; Dombrowski v. PfisterWe are not dealing with a statute
that, on its face, appears to abridge free expression, an area of
particular concern under Dombrowski and refined in Younger v. Harris, We
find it unnecessary to decide whether the District Court erred in
withholding injunctive relief, for we assume the Texas prosecutorial
authorities will give full credence to this decision that the present
criminal abortion statutes of that State are unconstitutional.

The judgment of the District Court as to intervenor Hallford is
reversed, and Dr.~Hallford's complaint in intervention is dismissed. In
all other respects, the judgment of the District Court is affirmed.
Costs are allowed to the appellee.

It is so ordered.

Affirmed in part and reversed in part.

\hypertarget{planned-parenthood-v.-casey}{%
\subsubsection{Planned Parenthood v.
Casey}\label{planned-parenthood-v.-casey}}

505 U.S. 833 (1992)

\textbf{Justice O'Connor, Justice Kennedy, and Justice Souter announced
the judgment of the Court and delivered the opinion of the Court with
respect to Parts I, II, III, V---A, V---C, and VI, an opinion with
respect to Part V---E, in which Justice Stevens joins, and an opinion
with respect to Parts IV, V---B, and V---D.}

Liberty finds no refuge in a jurisprudence of doubt. Yet 19 years after
our holding that the Constitution protects a woman's right to terminate
her pregnancy in its early stages, Roe v. Wade, that definition of
liberty is still questioned. Joining the respondents as amicus curiae,
the United States, as it has done in five other cases in the last
decade, again asks us to overrule Roe. See Brief for Respondents
104-117; Brief for United States as Amicus Curiae 8.

At issue in these cases are five provisions of the Pennsylvania Abortion
Control Act of 1982, as amended in 1988 and 1989. 18 Pa. Cons. Stat. §§
3203-3220 (1990). Relevant portions of the Act are set forth in the
Appendix. Infra. The Act requires that a woman seeking an abortion give
her informed consent prior to the abortion procedure, and specifies that
she be provided with certain information at least 24 hours before the
abortion is performed. § 3205. For a minor to obtain an abortion, the
Act requires the informed consent of one of her parents, but provides
for a judicial bypass option if the minor does not wish to or cannot
obtain a parent's consent. § 3206. Another provision of the Act requires
that, unless certain exceptions apply, a married woman seeking an
abortion must sign a statement indicating that she has notified her
husband of her intended abortion. § 3209. The Act exempts compliance
with these three requirements in the event of a ``medical emergency,''
which is defined in § 3203 of the Act. See §§ 3203, 3205(a), 3206(a),
3209(c). In addition to the above provisions regulating the performance
of abortions, the Act imposes certain reporting requirements on
facilities that provide abortion services. §§ 3207(b), 3214(a), 3214(f).

Before any of these provisions took effect, the petitioners, who are
five abortion clinics and one physician representing himself as well as
a class of physicians who provide abortion services, brought this suit
seeking declaratory and injunctive relief. Each provision was challenged
as unconstitutional on its face. The District Court entered a
preliminary injunction against the enforcement of the regulations, and,
after a 3-day bench trial, held all the provisions at issue here
unconstitutional, entering a permanent injunction against Pennsylvania's
enforcement of them. 744 F. Supp. 1323 (ED Pa. 1990). The Court of
Appeals for the Third Circuit affirmed in part and reversed in part,
upholding all of the regulations except for the husband notification
requirement. We granted certiorari. 502 U. S. 1056 (1992).

The Court of Appeals found it necessary to follow an elaborate course of
reasoning even to identify the first premise to use to determine whether
the statute enacted by Pennsylvania meets constitutional standards. See
947 F. 2d--- 698. And at oral argument in this Court, the attorney for
the parties challenging the statute took the position that none of the
enactments can be upheld without overruling Roe v. Wade. Tr. of Oral
Arg. 5-6. We disagree with that analysis; but we acknowledge that our
decisions after Roe cast doubt upon the meaning and reach of its
holding. Further, The Chief Justice admits that he would overrule the
central holding ofRoe and adopt the rational relationship test as the
sole criterion of constitutionality. See post, 966. State and federal
courts as well as legislatures throughout the Union must have guidance
as they seek to address this subject in conformance with the
Constitution. Given these premises, we find it imperative to review once
more the principles that define the rights of the woman and the
legitimate authority of the State respecting the termination of
pregnancies by abortion procedures.

After considering the fundamental constitutional questions resolved by
Roe,principles of institutional integrity, and the rule of stare
decisis, we are led to conclude this: the essential holding of Roe v.
Wade should be retained and once again reaffirmed.

It must be stated at the outset and with clarity that Roe's essential
holding, the holding we reaffirm, has three parts. First is a
recognition of the right of the woman to choose to have an abortion
before viability and to obtain it without undue interference from the
State. Before viability, the State's interests are not strong enough to
support a prohibition of abortion or the imposition of a substantial
obstacle to the woman's effective right to elect the procedure. Second
is a confirmation of the State's power to restrict abortions after fetal
viability, if the law contains exceptions for pregnancies which endanger
the woman's life or health. And third is the principle that the State
has legitimate interests from the outset of the pregnancy in protecting
the health of the woman and the life of the fetus that may become a
child. These principles do not contradict one another; and we adhere to
each.

Constitutional protection of the woman's decision to terminate her
pregnancy derives from the Due Process Clause of the Fourteenth
Amendment. It declares that no State shall ``deprive any person of life,
liberty, or property, without due process of law.'' The controlling word
in the cases before us is ``liberty.'' Although a literal reading of the
Clause might suggest that it governs only the procedures by which a
State may deprive persons of liberty, for at least 105 years, since
Mugler v. Kansas (1887), the Clause has been understood to contain a
substantive component as well, one ``barring certain government actions
regardless of the fairness of the procedures used to implement them.''
Daniels v. Williams. As Justice Brandeis (joined by Justice Holmes)
observed, ``{[}d{]}espite arguments to the contrary which had seemed to
me persuasive, it is settled that the due process clause of the
Fourteenth Amendment applies to matters of substantive law as well as to
matters of procedure. Thus all fundamental rights comprised within the
term liberty are protected by the Federal Constitution from invasion by
the States.'' Whitney v. California (concurring opinion). ``{[}T{]}he
guaranties of due process, though having their roots in Magna
Carta's'per legem terrae' and considered as procedural
safeguards'against executive usurpation and tyranny,' have in this
country'become bulwarks also against arbitrary legislation.''' Poe v.
Ullman (Harlan, J., dissenting from dismissal on jurisdictional grounds)
(quoting Hurtado v. California).

The most familiar of the substantive liberties protected by the
Fourteenth Amendment are those recognized by the Bill of Rights. We have
held that the Due Process Clause of the Fourteenth Amendment
incorporates most of the Bill of Rights against the States. See, e. g.,
Duncan v. Louisiana (1968). It is tempting, as a means of curbing the
discretion of federal judges, to suppose that liberty encompasses no
more than those rights already guaranteed to the individual against
federal interference by the express provisions of the first eight
Amendments to the Constitution. See Adamson v. California (1947) (Black,
J., dissenting). But of course this Court has never accepted that view.

It is also tempting, for the same reason, to suppose that the Due
Process Clause protects only those practices, defined at the most
specific level, that were protected against government interference by
other rules of law when the Fourteenth Amendment was ratified. See
Michael H. v. Gerald D., n.~6 (1989) (opinion of Scalia, J.). But such a
view would be inconsistent with our law. It is a promise of the
Constitution that there is a realm of personal liberty which the
government may not enter. We have vindicated this principle before.
Marriage is mentioned nowhere in the Bill of Rights and interracial
marriage was illegal in most States in the 19th century, but the Court
was no doubt correct in finding it to be an aspect of liberty protected
against state interference by the substantive component of the Due
Process Clause in Loving v. Virginia (relying, in an opinion for eight
Justices, on the Due Process Clause). Similar examples may be found in
Turner v. Safley (1987); in Carey v. Population Services International
(1977); in Griswold v. Connecticut (1965), as well as in the separate
opinions of a majority of the Members of the Court in that case, -488
(Goldberg, J., joined by Warren, C. J., and Brennan, J., concurring)
(expressly relying on due process), -502 (Harlan, J., concurring in
judgment) (same), -507 (White, J., concurring in judgment) (same); in
Pierce v. Society of Sisters (1925); and in Meyer v. Nebraska (1923).

Neither the Bill of Rights nor the specific practices of States at the
time of the adoption of the Fourteenth Amendment marks the outer limits
of the substantive sphere of liberty which the Fourteenth Amendment
protects. See U. S. Const., Amdt. 9. As the second Justice Harlan
recognized:

``{[}T{]}he full scope of the liberty guaranteed by the Due Process
Clause cannot be found in or limited by the precise terms of the
specific guarantees elsewhere provided in the Constitution.
This'liberty' is not a series of isolated points pricked out in terms of
the taking of property; the freedom of speech, press, and religion; the
right to keep and bear arms; the freedom from unreasonable searches and
seizures; and so on. It is a rational continuum which, broadly speaking,
includes a freedom from all substantial arbitrary impositions and
purposeless restraints, . and which also recognizes, what a reasonable
and sensitive judgment must, that certain interests require particularly
careful scrutiny of the state needs asserted to justify their
abridgment.'' Poe v. Ullman (opinion dissenting from dismissal on
jurisdictional grounds).

Justice Harlan wrote these words in addressing an issue the full Court
did not reach in Poe v. Ullman, but the Court adopted his position four
Terms later in Griswold v. Connecticut. In Griswold, we held that the
Constitution does not permit a State to forbid a married couple to use
contraceptives. That same freedom was later guaranteed, under the Equal
Protection Clause, for unmarried couples. See Eisenstadt v. Baird.
Constitutional protection was extended to the sale and distribution of
contraceptives in Carey v. Population Services International. It is
settled now, as it was when the Court heard arguments in Roe v. Wade,
that the Constitution places limits on a State's right to interfere with
a person's most basic decisions about family and parenthood, see Carey
v. Population Services International; Moore v. East Cleveland;
Eisenstadt v. Baird; Loving v. Virginia; Griswold v. Connecticut;
Skinner v. Oklahoma ex rel. Williamson; Pierce v. Society of Sisters;
Meyer v. Nebraskaas well as bodily integrity, see, e. g., Washington v.
Harper (1990); Winstonv. Lee; Rochin v. California.

The inescapable fact is that adjudication of substantive due process
claims may call upon the Court in interpreting the Constitution to
exercise that same capacity which by tradition courts always have
exercised: reasoned judgment. Its boundaries are not susceptible of
expression as a simple rule. That does not mean we are free to
invalidate state policy choices with which we disagree; yet neither does
it permit us to shrink from the duties of our office. As Justice Harlan
observed:

"Due process has not been reduced to any formula; its content cannot be
determined by reference to any code.

The best that can be said is that through the course of this Court's
decisions it has represented the balance which our Nation, built upon
postulates of respect for the liberty of the individual, has struck
between that liberty and the demands of organized society. If the
supplying of content to this Constitutional concept has of necessity
been a rational process, it certainly has not been one where judges have
felt free to roam where unguided speculation might take them. The
balance of which I speak is the balance struck by this country, having
regard to what history teaches are the traditions from which it
developed as well as the traditions from which it broke. That tradition
is a living thing. A decision of this Court which radically departs from
it could not long survive, while a decision which builds on what has
survived is likely to be sound. No formula could serve as a substitute,
in this area, for judgment and restraint." Poe v. Ullman (opinion
dissenting from dismissal on jurisdictional grounds).

See also Rochin v. California (Frankfurter, J., writing for the Court)
(``To believe that this judicial exercise of judgment could be avoided
by freezing'due process of law' at some fixed stage of time or thought
is to suggest that the most important aspect of constitutional
adjudication is a function for inanimate machines and not for judges'').

Men and women of good conscience can disagree, and we suppose some
always shall disagree, about the profound moral and spiritual
implications of terminating a pregnancy, even in its earliest stage.
Some of us as individuals find abortion offensive to our most basic
principles of morality, but that cannot control our decision. Our
obligation is to define the liberty of all, not to mandate our own moral
code. The underlying constitutional issue is whether the State can
resolve these philosophic questions in such a definitive way that a
woman lacks all choice in the matter, except perhaps in those rare
circumstances in which the pregnancy is itself a danger to her own life
or health, or is the result of rape or incest.

It is conventional constitutional doctrine that where reasonable people
disagree the government can adopt one position or the other. See, e. g.,
Ferguson v. Skrupa,372 U. S. 726 (1963); Williamson v. Lee Optical of
Okla., Inc.. That theorem, however, assumes a state of affairs in which
the choice does not intrude upon a protected liberty. Thus, while some
people might disagree about whether or not the flag should be saluted,
or disagree about the proposition that it may not be defiled, we have
ruled that a State may not compel or enforce one view or the other. See
West Virginia Bd. of Ed. v. Barnette; Texas v. Johnson.

Our law affords constitutional protection to personal decisions relating
to marriage, procreation, contraception, family relationships, child
rearing, and education. Carey v. Population Services International. Our
cases recognize ``the right of the individual, married or single, to be
free from unwarranted governmental intrusion into matters so
fundamentally affecting a person as the decision whether to bear or
beget a child.'' Eisenstadt v. Baird(emphasis in original). Our
precedents ``have respected the private realm of family life which the
state cannot enter.'' Prince v. Massachusetts. These matters, involving
the most intimate and personal choices a person may make in a lifetime,
choices central to personal dignity and autonomy, are central to the
liberty protected by the Fourteenth Amendment. At the heart of liberty
is the right to define one's own concept of existence, of meaning, of
the universe, and of the mystery of human life. Beliefs about these
matters could not define the attributes of personhood were they formed
under compulsion of the State.

These considerations begin our analysis of the woman's interest in
terminating her pregnancy but cannot end it, for this reason: though the
abortion decision may originate within the zone of conscience and
belief, it is more than a philosophic exercise. Abortion is a unique
act. It is an act fraught with consequences for others: for the woman
who must live with the implications of her decision; for the persons who
perform and assist in the procedure; for the spouse, family, and society
which must confront the knowledge that these procedures exist,
procedures some deem nothing short of an act of violence against
innocent human life; and, depending on one's beliefs, for the life or
potential life that is aborted. Though abortion is conduct, it does not
follow that the State is entitled to proscribe it in all instances. That
is because the liberty of the woman is at stake in a sense unique to the
human condition and so unique to the law. The mother who carries a child
to full term is subject to anxieties, to physical constraints, to pain
that only she must bear. That these sacrifices have from the beginning
of the human race been endured by woman with a pride that ennobles her
in the eyes of others and gives to the infant a bond of love cannot
alone be grounds for the State to insist she make the sacrifice. Her
suffering is too intimate and personal for the State to insist, without
more, upon its own vision of the woman's role, however dominant that
vision has been in the course of our history and our culture. The
destiny of the woman must be shaped to a large extent on her own
conception of her spiritual imperatives and her place in society.

It should be recognized, moreover, that in some critical respects the
abortion decision is of the same character as the decision to use
contraception, to which Griswold v. Connecticut, Eisenstadt v. Baird,
and Carey v. Population Services International afford constitutional
protection. We have no doubt as to the correctness of those decisions.
They support the reasoning in Roe relating to the woman's liberty
because they involve personal decisions concerning not only the meaning
of procreation but also human responsibility and respect for it. As with
abortion, reasonable people will have differences of opinion about these
matters. One view is based on such reverence for the wonder of creation
that any pregnancy ought to be welcomed and carried to full term no
matter how difficult it will be to provide for the child and ensure its
well-being. Another is that the inability to provide for the nurture and
care of the infant is a cruelty to the child and an anguish to the
parent. These are intimate views with infinite variations, and their
deep, personal character underlay our decisions in Griswold, Eisenstadt,
and Carey. The same concerns are present when the woman confronts the
reality that, perhaps despite her attempts to avoid it, she has become
pregnant.

It was this dimension of personal liberty that Roe sought to protect,
and its holding invoked the reasoning and the tradition of the
precedents we have discussed, granting protection to substantive
liberties of the person. Roe was, of course, an extension of those cases
and, as the decision itself indicated, the separate States could act in
some degree to further their own legitimate interests in protecting
prenatal life. The extent to which the legislatures of the States might
act to outweigh the interests of the woman in choosing to terminate her
pregnancy was a subject of debate both in Roe itself and in decisions
following it.

While we appreciate the weight of the arguments made on behalf of the
State in the cases before us, arguments which in their ultimate
formulation conclude that Roe should be overruled, the reservations any
of us may have in reaffirming the central holding of Roe are outweighed
by the explication of individual liberty we have given combined with the
force of stare decisis. We turn now to that doctrine.

The obligation to follow precedent begins with necessity, and a contrary
necessity marks its outer limit. With Cardozo, we recognize that no
judicial system could do society's work if it eyed each issue afresh in
every case that raised it. See B. Cardozo, The Nature of the Judicial
Process 149 (1921). Indeed, the very concept of the rule of law
underlying our own Constitution requires such continuity over time that
a respect for precedent is, by definition, indispensable. See Powell,
Stare Decisis and Judicial Restraint, 1991 Journal of Supreme Court
History 13, 16. At the other extreme, a different necessity would make
itself felt if a prior judicial ruling should come to be seen so clearly
as error that its enforcement was for that very reason doomed.

Even when the decision to overrule a prior case is not, as in the rare,
latter instance, virtually foreordained, it is common wisdom that the
rule of stare decisisis not an ``inexorable command,'' and certainly it
is not such in every constitutional case, see Burnet v. Coronado Oil \&
Gas Co.~(1932) (Brandeis, J., dissenting). See also Payne v. Tennessee
(Souter, J., joined by Kennedy, J., concurring); Arizona v. Rumsey.
Rather, when this Court reexamines a prior holding, its judgment is
customarily informed by a series of prudential and pragmatic
considerations designed to test the consistency of overruling a prior
decision with the ideal of the rule of law, and to gauge the respective
costs of reaffirming and overruling a prior case. Thus, for example, we
may ask whether the rule has proven to be intolerable simply in defying
practical workability, Swift \& Co.~v. Wickham; whether the rule is
subject to a kind of reliance that would lend a special hardship to the
consequences of overruling and add inequity to the cost of repudiation,
e. g., United States v. Title Ins. \& Trust Co.; whether related
principles of law have so far developed as to have left the old rule no
more than a remnant of abandoned doctrine, see Patterson v. McLean
Credit Union (1989); or whether facts have so changed, or come to be
seen so differently, as to have robbed the old rule of significant
application or justification, e. g., Burnet (Brandeis, J., dissenting).

So in this case we may enquire whether Roe's central rule has been found
unworkable; whether the rule's limitation on state power could be
removed without serious inequity to those who have relied upon it or
significant damage to the stability of the society governed by it;
whether the law's growth in the intervening years has left Roe's central
rule a doctrinal anachronism discounted by society; and whether Roe's
premises of fact have so far changed in the ensuing two decades as to
render its central holding somehow irrelevant or unjustifiable in
dealing with the issue it addressed.

Although Roe has engendered opposition, it has in no sense proven
``unworkable,'' see Garcia v. San Antonio Metropolitan Transit
Authority, representing as it does a simple limitation beyond which a
state law is unenforceable. While Roe has, of course, required judicial
assessment of state laws affecting the exercise of the choice guaranteed
against government infringement, and although the need for such review
will remain as a consequence of today's decision, the required
determinations fall within judicial competence.

The inquiry into reliance counts the cost of a rule's repudiation as it
would fall on those who have relied reasonably on the rule's continued
application. Since the classic case for weighing reliance heavily in
favor of following the earlier rule occurs in the commercial context,
see Payne v. Tennes- see, where advance planning of great precision is
most obviously a necessity, it is no cause for surprise that some would
find no reliance worthy of consideration in support of Roe.

While neither respondents nor their amici in so many words deny that the
abortion right invites some reliance prior to its actual exercise, one
can readily imagine an argument stressing the dissimilarity of this case
to one involving property or contract. Abortion is customarily chosen as
an unplanned response to the consequence of unplanned activity or to the
failure of conventional birth control, and except on the assumption that
no intercourse would have occurred but for Roe`s holding, such behavior
may appear to justify no reliance claim. Even if reliance could be
claimed on that unrealistic assumption, the argument might run, any
reliance interest would be de minimis. This argument would be premised
on the hypothesis that reproductive planning could take virtually
immediate account of any sudden restoration of state authority to ban
abortions.

To eliminate the issue of reliance that easily, however, one would need
to limit cognizable reliance to specific instances of sexual activity.
But to do this would be simply to refuse to face the fact that for two
decades of economic and social developments, people have organized
intimate relationships and made choices that define their views of
themselves and their places in society, in reliance on the availability
of abortion in the event that contraception should fail. The ability of
women to participate equally in the economic and social life of the
Nation has been facilitated by their ability to control their
reproductive lives. See, e. g., R. Petchesky, Abortion and Woman's
Choice 109, 133, n.~7 (rev. ed.~1990). The Constitution serves human
values, and while the effect of reliance on Roe cannot be exactly
measured, neither can the certain cost of overruling Roe for people who
have ordered their thinking and living around that case be dismissed.

No evolution of legal principle has left Roe's doctrinal footings weaker
than they were in 1973. No development of constitutional law since the
case was decided has implicitly or explicitly left Roe behind as a mere
survivor of obsolete constitutional thinking.

It will be recognized, of course, that Roe stands at an intersection of
two lines of decisions, but in whichever doctrinal category one reads
the case, the result for present purposes will be the same. The Roe
Court itself placed its holding in the succession of cases most
prominently exemplified by Griswold v. Connecticut. See Roe. When it is
so seen, Roe is clearly in no jeopardy, since subsequent constitutional
developments have neither disturbed, nor do they threaten to diminish,
the scope of recognized protection accorded to the liberty relating to
intimate relationships, the family, and decisions about whether or not
to beget or bear a child. See, e. g., Carey v. Population Services
International; Moore v. East Cleveland.

Roe, however, may be seen not only as an exemplar of Griswold liberty
but as a rule (whether or not mistaken) of personal autonomy and bodily
integrity, with doctrinal affinity to cases recognizing limits on
governmental power to mandate medical treatment or to bar its rejection.
If so, our cases since Roe accord with Roe's view that a State's
interest in the protection of life falls short of justifying any plenary
override of individual liberty claims. Cruzan v. Director, Mo. Dept. of
Health; cf., e. g., Riggins v. Nevada; Washington v. Harper; see also,
e. g., Rochin v. California; Jacobson v. Massachusetts (1905).

Finally, one could classify Roe as sui generis. If the case is so
viewed, then there clearly has been no erosion of its central
determination. The original holding resting on the concurrence of seven
Members of the Court in 1973 was expressly affirmed by a majority of six
in 1983, see Akron v. Akron Center for Reproductive Health, Inc.Akron
I), and by a majority of five in 1986, see Thornburgh v. American
College of Obstetricians and Gynecologists,476 U. S. 747, expressing
adherence to the constitutional ruling despite legislative efforts in
some States to test its limits. More recently, in Webster v.
Reproductive Health Services, although two of the present authors
questioned the trimester framework in a way consistent with our judgment
today, a majority of the Court either decided to reaffirm or declined to
address the constitutional validity of the central holding of Roe.

Nor will courts building upon Roe be likely to hand down erroneous
decisions as a consequence. Even on the assumption that the central
holding of Roe was in error, that error would go only to the strength of
the state interest in fetal protection, not to the recognition afforded
by the Constitution to the woman's liberty. The latter aspect of the
decision fits comfortably within the framework of the Court's prior
decisions, including Skinner v. Oklahoma ex rel. Williamson; Griswold;
Loving v. Virginia, 388 U. S. 1 (1967); and Eisenstadt v. Baird,405 U.
S. 438 (1972), the holdings of which are ``not a series of isolated
points,'' but mark a ``rational continuum.'' Poe v. Ullman (Harlan, J.,
dissenting). As we described in Carey v. Population Services
Internationalthe liberty which encompasses those decisions

``includes'the interest in independence in making certain kinds of
important decisions.' While the outer limits of this aspect of
{[}protected liberty{]} have not been marked by the Court, it is clear
that among the decisions that an individual may make without unjustified
government interference are personal decisions'relating to marriage,
procreation, contraception, family relationships, and child rearing and
education.''' 431 U. S. (citations omitted).

The soundness of this prong of the Roe analysis is apparent from a
consideration of the alternative. If indeed the woman's interest in
deciding whether to bear and beget a child had not been recognized as in
Roe, the State might as readily restrict a woman's right to choose to
carry a pregnancy to term as to terminate it, to further asserted state
interests in population control, or eugenics, for example. Yet Roe has
been sensibly relied upon to counter any such suggestions. E. g., Arnold
v. Board of Education of Escambia County, Ala., CA11 1989)(relying upon
Roe and concluding that government officials violate the Constitution by
coercing a minor to have an abortion); Avery v. County of Burke, CA4
1981) (county agency inducing teenage girl to undergo unwanted
sterilization on the basis of misrepresentation that she had sickle cell
trait); see also In re Quinlan, 70 N. J. 10, 355 A. 2d 647 (relying on
Roe in finding a right to terminate medical treatment), cert. denied sub
nom. Garger v. New Jersey). In any event, because Roe's scope is
confined by the fact of its concern with postconception potential life,
a concern otherwise likely to be implicated only by some forms of
contraception protected independently under Griswold and later cases,
any error in Roe is unlikely to have serious ramifications in future
cases.

We have seen how time has overtaken some of Roe's factual assumptions:
advances in maternal health care allow for abortions safe to the mother
later in pregnancy than was true in 1973, see Akron I, n.~11, and
advances in neonatal care have advanced viability to a point somewhat
earlier. Compare Roe, with Webster (opinion of Rehnquist, C. J.); see
Akron I, and n.~5 (O'Connor, J., dissenting). But these facts go only to
the scheme of time limits on the realization of competing interests, and
the divergences from the factual premises of 1973 have no bearing on the
validity of Roe's central holding, that viability marks the earliest
point at which the State's interest in fetal life is constitutionally
adequate to justify a legislative ban on nontherapeutic abortions. The
soundness or unsoundness of that constitutional judgment in no sense
turns on whether viability occurs at approximately 28 weeks, as was
usual at the time of Roe to 24 weeks, as it sometimes does today, or at
some moment even slightly earlier in pregnancy, as it may if fetal
respiratory capacity can somehow be enhanced in the future. Whenever it
may occur, the attainment of viability may continue to serve as the
critical fact, just as it has done since Roe was decided; which is to
say that no change in Roe's factual underpinning has left its central
holding obsolete, and none supports an argument for overruling it.

The sum of the precedential enquiry to this point shows Roe's
underpinnings unweakened in any way affecting its central holding. While
it has engendered disapproval, it has not been unworkable. An entire
generation has come of age free to assume Roe's concept of liberty in
defining the capacity of women to act in society, and to make
reproductive decisions; no erosion of principle going to liberty or
personal autonomy has left Roe's central holding a doctrinal remnant;
Roe portends no developments at odds with other precedent for the
analysis of personal liberty; and no changes of fact have rendered
viability more or less appropriate as the point at which the balance of
interests tips. Within the bounds of normal stare decisis analysis,
then, and subject to the considerations on which it customarily turns,
the stronger argument is for affirming Roe's central holding, with
whatever degree of personal reluctance any of us may have, not for
overruling it.

In a less significant case, stare decisis analysis could, and would,
stop at the point we have reached. But the sustained and widespread
debate Roe has provoked calls for some comparison between that case and
others of comparable dimension that have responded to national
controversies and taken on the impress of the controversies addressed.
Only two such decisional lines from the past century present themselves
for examination, and in each instance the result reached by the Court
accorded with the principles we apply today.

The first example is that line of cases identified with Lochner v. New
York, which imposed substantive limitations on legislation limiting
economic autonomy in favor of health and welfare regulation, adopting,
in Justice Holmes's view, the theory of laissez-faire. (dissenting
opinion). The Lochner decisions were exemplified by Adkins v. Children's
Hospital of District of Columbia, in which this Court held it to be an
infringement of constitutionally protected liberty of contract to
require the employers of adult women to satisfy minimum wage standards.
Fourteen years later, West Coast Hotel Co.~v. Parrish, signaled the
demise of Lochner by overruling Adkins. In the meantime, the Depression
had come and, with it, the lesson that seemed unmistakable to most
people by 1937, that the interpretation of contractual freedom protected
in Adkins rested on fundamentally false factual assumptions about the
capacity of a relatively unregulated market to satisfy minimal levels of
human welfare. See West Coast Hotel Co.. As Justice Jackson wrote of the
constitutional crisis of 1937 shortly before he came on the bench: ``The
older world of laissez faire was recognized everywhere outside the Court
to be dead.'' The Struggle for Judicial Supremacy 85 (1941). The facts
upon which the earlier case had premised a constitutional resolution of
social controversy had proven to be untrue, and history's demonstration
of their untruth not only justified but required the new choice of
constitutional principle that West Coast Hotel announced. Of course, it
was true that the Court lost something by its misperception, or its lack
of prescience, and the Court-packing crisis only magnified the loss; but
the clear demonstration that the facts of economic life were different
from those previously assumed warranted the repudiation of the old law.

The second comparison that 20th century history invites is with the
cases employing the separate-but-equal rule for applying the Fourteenth
Amendment's equal protection guarantee. They began with Plessy v.
Ferguson, holding that legislatively mandated racial segregation in
public transportation works no denial of equal protection, rejecting the
argument that racial separation enforced by the legal machinery of
American society treats the black race as inferior. The Plessy Court
considered ``the underlying fallacy of the plaintiff's argument to
consist in the assumption that the enforced separation of the two races
stamps the colored race with a badge of inferiority. If this be so, it
is not by reason of anything found in the act, but solely because the
colored race chooses to put that construction upon it.'' Whether, as a
matter of historical fact, the Justices in the Plessy majority believed
this or not, see 562 (Harlan, J., dissenting), this understanding of the
implication of segregation was the stated justification for the Court's
opinion. But this understanding of the facts and the rule it was stated
to justify were repudiated in Brown v. Board of Education (Brown I). As
one commentator observed, the question before the Court in Brown was
``whether discrimination inheres in that segregation which is imposed by
law in the twentieth century in certain specific states in the American
Union. And that question has meaning and can find an answer only on the
ground of history and of common knowledge about the facts of life in the
times and places aforesaid.'' Black, The Lawfulness of the Segregation
Decisions, 69 Yale L. J. 421, 427 (1960).

The Court in Brown addressed these facts of life by observing that
whatever may have been the understanding in Plessy's time of the power
of segregation to stigmatize those who were segregated with a ``badge of
inferiority,'' it was clear by 1954 that legally sanctioned segregation
had just such an effect, to the point that racially separate public
educational facilities were deemed inherently unequal. 347 U. S..
Society's understanding of the facts upon which a constitutional ruling
was sought in 1954 was thus fundamentally different from the basis
claimed for the decision in 1896. While we think Plessy was wrong the
day it was decided, see Plessy (Harlan, J., dissenting), we must also
recognize that the Plessy Court's explanation for its decision was so
clearly at odds with the facts apparent to the Court in 1954 that the
decision to reexamine Plessy was on this ground alone not only justified
but required.

West Coast Hotel and Brown each rested on facts, or an understanding of
facts, changed from those which furnished the claimed justifications for
the earlier constitutional resolutions. Each case was comprehensible as
the Court's response to facts that the country could understand, or had
come to understand already, but which the Court of an earlier day, as
its own declarations disclosed, had not been able to perceive. As the
decisions were thus comprehensible they were also defensible, not merely
as the victories of one doctrinal school over another by dint of numbers
(victories though they were), but as applications of constitutional
principle to facts as they had not been seen by the Court before. In
constitutional adjudication as elsewhere in life, changed circumstances
may impose new obligations, and the thoughtful part of the Nation could
accept each decision to overrule a prior case as a response to the
Court's constitutional duty.

Because the cases before us present no such occasion it could be seen as
no such response. Because neither the factual underpinnings of Roe's
central holding nor our understanding of it has changed (and because no
other indication of weakened precedent has been shown), the Court could
not pretend to be reexamining the prior law with any justification
beyond a present doctrinal disposition to come out differently from the
Court of 1973. To overrule prior law for no other reason than that would
run counter to the view repeated in our cases, that a decision to
overrule should rest on some special reason over and above the belief
that a prior case was wrongly decided. See, e. g., Mitchell v. W. T.
Grant Co.~(Stewart, J., dissenting) (``A basic change in the law upon a
ground no firmer than a change in our membership invites the popular
misconception that this institution is little different from the two
political branches of the Government. No misconception could do more
lasting injury to this Court and to the system of law which it is our
abiding mission to serve''); Mapp v. Ohio (Harlan, J., dissenting).

The examination of the conditions justifying the repudiation of Adkins
by West Coast Hotel and Plessy by Brown is enough to suggest the
terrible price that would have been paid if the Court had not overruled
as it did. In the present cases, however, as our analysis to this point
makes clear, the terrible price would be paid for overruling. Our
analysis would not be complete, however, without explaining why
overruling Roe's central holding would not only reach an unjustifiable
result under principles of stare decisis, but would seriously weaken the
Court's capacity to exercise the judicial power and to function as the
Supreme Court of a Nation dedicated to the rule of law. To understand
why this would be so it is necessary to understand the source of this
Court's authority, the conditions necessary for its preservation, and
its relationship to the country's understanding of itself as a
constitutional Republic.

The root of American governmental power is revealed most clearly in the
instance of the power conferred by the Constitution upon the Judiciary
of the United States and specifically upon this Court. As Americans of
each succeeding generation are rightly told, the Court cannot buy
support for its decisions by spending money and, except to a minor
degree, it cannot independently coerce obedience to its decrees. The
Court's power lies, rather, in its legitimacy, a product of substance
and perception that shows itself in the people's acceptance of the
Judiciary as fit to determine what the Nation's law means and to declare
what it demands.

The underlying substance of this legitimacy is of course the warrant for
the Court's decisions in the Constitution and the lesser sources of
legal principle on which the Court draws. That substance is expressed in
the Court's opinions, and our contemporary understanding is such that a
decision without principled justification would be no judicial act at
all. But even when justification is furnished by apposite legal
principle, something more is required. Because not every conscientious
claim of principled justification will be accepted as such, the
justification claimed must be beyond dispute. The Court must take care
to speak and act in ways that allow people to accept its decisions on
the terms the Court claims for them, as grounded truly in principle, not
as compromises with social and political pressures having, as such, no
bearing on the principled choices that the Court is obliged to make.
Thus, the Court's legitimacy depends on making legally principled
decisions under circumstances in which their principled character is
sufficiently plausible to be accepted by the Nation.

The need for principled action to be perceived as such is implicated to
some degree whenever this, or any other appellate court, overrules a
prior case. This is not to say, of course, that this Court cannot give a
perfectly satisfactory explanation in most cases. People understand that
some of the Constitution's language is hard to fathom and that the
Court's Justices are sometimes able to perceive significant facts or to
understand principles of law that eluded their predecessors and that
justify departures from existing decisions. However upsetting it may be
to those most directly affected when one judicially derived rule
replaces another, the country can accept some correction of error
without necessarily questioning the legitimacy of the Court.

In two circumstances, however, the Court would almost certainly fail to
receive the benefit of the doubt in overruling prior cases. There is,
first, a point beyond which frequent overruling would overtax the
country's belief in the Court's good faith. Despite the variety of
reasons that may inform and justify a decision to overrule, we cannot
forget that such a decision is usually perceived (and perceived
correctly) as, at the least, a statement that a prior decision was
wrong. There is a limit to the amount of error that can plausibly be
imputed to prior Courts. If that limit should be exceeded, disturbance
of prior rulings would be taken as evidence that justifiable
reexamination of principle had given way to drives for particular
results in the short term. The legitimacy of the Court would fade with
the frequency of its vacillation.

That first circumstance can be described as hypothetical; the second is
to the point here and now. Where, in the performance of its judicial
duties, the Court decides a case in such a way as to resolve the sort of
intensely divisive controversy reflected in Roe and those rare,
comparable cases, its decision has a dimension that the resolution of
the normal case does not carry. It is the dimension present whenever the
Court's interpretation of the Constitution calls the contending sides of
a national controversy to end their national division by accepting a
common mandate rooted in the Constitution.

The Court is not asked to do this very often, having thus addressed the
Nation only twice in our lifetime, in the decisions of Brown and Roe.
But when the Court does act in this way, its decision requires an
equally rare precedential force to counter the inevitable efforts to
overturn it and to thwart its implementation. Some of those efforts may
be mere unprincipled emotional reactions; others may proceed from
principles worthy of profound respect. But whatever the premises of
opposition may be, only the most convincing justification under accepted
standards of precedent could suffice to demonstrate that a later
decision overruling the first was anything but a surrender to political
pressure, and an unjustified repudiation of the principle on which the
Court staked its authority in the first instance. So to overrule under
fire in the absence of the most compelling reason to reexamine a
watershed decision would subvert the Court's legitimacy beyond any
serious question. Cf. Brown v. Board of Education (Brown II) (``{[}I{]}t
should go without saying that the vitality of th{[}e{]} constitutional
principles {[}announced in Brown I, {]} cannot be allowed to yield
simply because of disagreement with them'').

The country's loss of confidence in the Judiciary would be underscored
by an equally certain and equally reasonable condemnation for another
failing in overruling unnecessarily and under pressure. Some cost will
be paid by anyone who approves or implements a constitutional decision
where it is unpopular, or who refuses to work to undermine the decision
or to force its reversal. The price may be criticism or ostracism, or it
may be violence. An extra price will be paid by those who themselves
disapprove of the decision's results when viewed outside of
constitutional terms, but who nevertheless struggle to accept it,
because they respect the rule of law. To all those who will be so tested
by following, the Court implicitly undertakes to remain steadfast, lest
in the end a price be paid for nothing. The promise of constancy, once
given, binds its maker for as long as the power to stand by the decision
survives and the understanding of the issue has not changed so
fundamentally as to render the commitment obsolete. From the obligation
of this promise this Court cannot and should not assume any exemption
when duty requires it to decide a case in conformance with the
Constitution. A willing breach of it would be nothing less than a breach
of faith, and no Court that broke its faith with the people could
sensibly expect credit for principle in the decision by which it did
that.

It is true that diminished legitimacy may be restored, but only slowly.
Unlike the political branches, a Court thus weakened could not seek to
regain its position with a new mandate from the voters, and even if the
Court could somehow go to the polls, the loss of its principled
character could not be retrieved by the casting of so many votes. Like
the character of an individual, the legitimacy of the Court must be
earned over time. So, indeed, must be the character of a Nation of
people who aspire to live according to the rule of law. Their belief in
themselves as such a people is not readily separable from their
understanding of the Court invested with the authority to decide their
constitutional cases and speak before all others for their
constitutional ideals. If the Court's legitimacy should be undermined,
then, so would the country be in its very ability to see itself through
its constitutional ideals. The Court's concern with legitimacy is not
for the sake of the Court, but for the sake of the Nation to which it is
responsible.

The Court's duty in the present cases is clear. In 1973, it confronted
the already-divisive issue of governmental power to limit personal
choice to undergo abortion, for which it provided a new resolution based
on the due process guaranteed by the Fourteenth Amendment. Whether or
not a new social consensus is developing on that issue, its divisiveness
is no less today than in 1973, and pressure to overrule the decision,
like pressure to retain it, has grown only more intense. A decision to
overrule Roe's essential holding under the existing circumstances would
address error, if error there was, at the cost of both profound and
unnecessary damage to the Court's legitimacy, and to the Nation's
commitment to the rule of law. It is therefore imperative to adhere to
the essence of Roe's original decision, and we do so today.

From what we have said so far it follows that it is a constitutional
liberty of the woman to have some freedom to terminate her pregnancy. We
conclude that the basic decision in Roe was based on a constitutional
analysis which we cannot now repudiate. The woman's liberty is not so
unlimited, however, that from the outset the State cannot show its
concern for the life of the unborn, and at a later point in fetal
development the State's interest in life has sufficient force so that
the right of the woman to terminate the pregnancy can be restricted.

That brings us, of course, to the point where much criticism has been
directed at Roe, a criticism that always inheres when the Court draws a
specific rule from what in the Constitution is but a general standard.
We conclude, however, that the urgent claims of the woman to retain the
ultimate control over her destiny and her body, claims implicit in the
meaning of liberty, require us to perform that function. Liberty must
not be extinguished for want of a line that is clear. And it falls to us
to give some real substance to the woman's liberty to determine whether
to carry her pregnancy to full term.

We conclude the line should be drawn at viability, so that before that
time the woman has a right to choose to terminate her pregnancy. We
adhere to this principle for two reasons. First, as we have said, is the
doctrine of stare decisis.Any judicial act of line-drawing may seem
somewhat arbitrary, but Roe was a reasoned statement, elaborated with
great care. We have twice reaffirmed it in the face of great opposition.
See Thornburgh v. American College of Obstetricians and Gynecologists;
Akron I Although we must overrule those parts of Thornburgh and Akron I
which, in our view, are inconsistent with Roe's statement that the State
has a legitimate interest in promoting the life or potential life of the
unborn, see infra, the central premise of those cases represents an
unbroken commitment by this Court to the essential holding of Roe. It is
that premise which we reaffirm today.

The second reason is that the concept of viability, as we noted in Roe,
is the time at which there is a realistic possibility of maintaining and
nourishing a life outside the womb, so that the independent existence of
the second life can in reason and all fairness be the object of state
protection that now overrides the rights of the woman. See Roe v. Wade.
Consistent with other constitutional norms, legislatures may draw lines
which appear arbitrary without the necessity of offering a
justification. But courts may not. We must justify the lines we draw.
And there is no line other than viability which is more workable. To be
sure, as we have said, there may be some medical developments that
affect the precise point of viability, see, but this is an imprecision
within tolerable limits given that the medical community and all those
who must apply its discoveries will continue to explore the matter. The
viability line also has, as a practical matter, an element of fairness.
In some broad sense it might be said that a woman who fails to act
before viability has consented to the State's intervention on behalf of
the developing child.

The woman's right to terminate her pregnancy before viability is the
most central principle of Roe v. Wade. It is a rule of law and a
component of liberty we cannot renounce.

On the other side of the equation is the interest of the State in the
protection of potential life. The Roe Court recognized the State's
``important and legitimate interest in protecting the potentiality of
human life.'' Roe. The weight to be given this state interest, not the
strength of the woman's interest, was the difficult question faced in
Roe. We do not need to say whether each of us, had we been Members of
the Court when the valuation of the state interest came before it as an
original matter, would have concluded, as the Roe Court did, that its
weight is insufficient to justify a ban on abortions prior to viability
even when it is subject to certain exceptions. The matter is not before
us in the first instance, and coming as it does after nearly 20 years of
litigation in Roe's wake we are satisfied that the immediate question is
not the soundness of Roe's resolution of the issue, but the precedential
force that must be accorded to its holding. And we have concluded that
the essential holding of Roe should be reaffirmed.

Yet it must be remembered that Roe v. Wade speaks with clarity in
establishing not only the woman's liberty but also the State's
``important and legitimate interest in potential life.'' Roe. That
portion of the decision in Roe has been given too little acknowledgment
and implementation by the Court in its subsequent cases. Those cases
decided that any regulation touching upon the abortion decision must
survive strict scrutiny, to be sustained only if drawn in narrow terms
to further a compelling state interest. See, e. g., Akron I. Not all of
the cases decided under that formulation can be reconciled with the
holding in Roeitself that the State has legitimate interests in the
health of the woman and in protecting the potential life within her. In
resolving this tension, we choose to rely upon Roe, as against the later
cases.

Roe established a trimester framework to govern abortion regulations.
Under this elaborate but rigid construct, almost no regulation at all is
permitted during the first trimester of pregnancy; regulations designed
to protect the woman's health, but not to further the State's interest
in potential life, are permitted during the second trimester; and during
the third trimester, when the fetus is viable, prohibitions are
permitted provided the life or health of the mother is not at stake.
Roe. Most of our cases since Roe have involved the application of rules
derived from the trimester framework. See, e. g., Thornburgh v. American
College of Obstetricians and Gynecologists; Akron I.

The trimester framework no doubt was erected to ensure that the woman's
right to choose not become so subordinate to the State's interest in
promoting fetal life that her choice exists in theory but not in fact.
We do not agree, however, that the trimester approach is necessary to
accomplish this objective. A framework of this rigidity was unnecessary
and in its later interpretation sometimes contradicted the State's
permissible exercise of its powers.

Though the woman has a right to choose to terminate or continue her
pregnancy before viability, it does not at all follow that the State is
prohibited from taking steps to ensure that this choice is thoughtful
and informed. Even in the earliest stages of pregnancy, the State may
enact rules and regulations designed to encourage her to know that there
are philosophic and social arguments of great weight that can be brought
to bear in favor of continuing the pregnancy to full term and that there
are procedures and institutions to allow adoption of unwanted children
as well as a certain degree of state assistance if the mother chooses to
raise the child herself.
"\texttt{{[}T{]}he\ Constitution\ does\ not\ forbid\ a\ State\ or\ city,\ pursuant\ to\ democratic\ processes,\ from\ expressing\ a\ preference\ for\ normal\ childbirth.\textquotesingle{}\ "\ Webster\ v.\ Reproductive\ Health\ Services\ (opinion\ of\ the\ Court)\ (quoting\ Poelker\ v.\ Doe).\ It\ follows\ that\ States\ are\ free\ to\ enact\ laws\ to\ provide\ a\ reasonable\ framework\ for\ a\ woman\ to\ make\ a\ decision\ that\ has\ such\ profound\ and\ lasting\ meaning.\ This,\ too,\ we\ find\ consistent\ with\ Roe}s
central premises, and indeed the inevitable consequence of our holding
that the State has an interest in protecting the life of the unborn.

We reject the trimester framework, which we do not consider to be part
of the essential holding of Roe. See Webster v. Reproductive Health
Services (opinion of Rehnquist, C. J.); (O'Connor, J., concurring in
part and concurring in judgment) (describing the trimester framework as
``problematic''). Measures aimed at ensuring that a woman's choice
contemplates the consequences for the fetus do not necessarily interfere
with the right recognized in Roe, although those measures have been
found to be inconsistent with the rigid trimester framework announced in
that case. A logical reading of the central holding in Roe itself, and a
necessary reconciliation of the liberty of the woman and the interest of
the State in promoting prenatal life, require, in our view, that we
abandon the trimester framework as a rigid prohibition on all
previability regulation aimed at the protection of fetal life. The
trimester framework suffers from these basic flaws: in its formulation
it misconceives the nature of the pregnant woman's interest; and in
practice it undervalues the State's interest in potential life, as
recognized in Roe.

As our jurisprudence relating to all liberties save perhaps abortion has
recognized, not every law which makes a right more difficult to exercise
is, ipso facto, an infringement of that right. An example clarifies the
point. We have held that not every ballot access limitation amounts to
an infringement of the right to vote. Rather, the States are granted
substantial flexibility in establishing the framework within which
voters choose the candidates for whom they wish to vote. Anderson v.
Celebrezze; Norman v. Reed.

The abortion right is similar. Numerous forms of state regulation might
have the incidental effect of increasing the cost or decreasing the
availability of medical care, whether for abortion or any other medical
procedure. The fact that a law which serves a valid purpose, one not
designed to strike at the right itself, has the incidental effect of
making it more difficult or more expensive to procure an abortion cannot
be enough to invalidate it. Only where state regulation imposes an undue
burden on a woman's ability to make this decision does the power of the
State reach into the heart of the liberty protected by the Due Process
Clause.

For the most part, the Court's early abortion cases adhered to this
view. In Maherv. Roe (1977), the Court explained: ``Roe did not declare
an unqualified'constitutional right to an abortion,' as the District
Court seemed to think. Rather, the right protects the woman from unduly
burdensome interference with her freedom to decide whether to terminate
her pregnancy.'' See also Doe v. Bolton (``{[}T{]}he interposition of
the hospital abortion committee is unduly restrictive of the patient's
rights''); Bellotti I (State may not ``impose undue burdens upon a minor
capable of giving an informed consent''); Harris v. McRae (citing
Maher). Cf. Carey v. Population Services International (``{[}T{]}he same
test must be applied to state regulations that burden an individual's
right to decide to prevent conception or terminate pregnancy by
substantially limiting access to the means of effectuating that decision
as is applied to state statutes that prohibit the decision entirely'').

These considerations of the nature of the abortion right illustrate that
it is an overstatement to describe it as a right to decide whether to
have an abortion ``without interference from the State.'' Planned
Parenthood of Central Mo. v. Danforth. All abortion regulations
interfere to some degree with a woman's ability to decide whether to
terminate her pregnancy. It is, as a consequence, not surprising that
despite the protestations contained in the original Roe opinion to the
effect that the Court was not recognizing an absolute right, the Court's
experience applying the trimester framework has led to the striking down
of some abortion regulations which in no real sense deprived women of
the ultimate decision. Those decisions went too far because the right
recognized by Roe is a right ``to be free from unwarranted governmental
intrusion into matters so fundamentally affecting a person as the
decision whether to bear or beget a child.'' Eisenstadt v. Baird. Not
all governmental intrusion is of necessity unwarranted; and that brings
us to the other basic flaw in the trimester framework: even in Roe's
terms, in practice it undervalues the State's interest in the potential
life within the woman.

Roe v. Wade was express in its recognition of the State's ``important
and legitimate interest{[}s{]} in preserving and protecting the health
of the pregnant woman {[}and{]} in protecting the potentiality of human
life.'' 410 U. S.. The trimester framework, however, does not fulfill
Roe's own promise that the State has an interest in protecting fetal
life or potential life. Roe began the contradiction by using the
trimester framework to forbid any regulation of abortion designed to
advance that interest before viability. Before viability, Roe and
subsequent cases treat all governmental attempts to influence a woman's
decision on behalf of the potential life within her as unwarranted. This
treatment is, in our judgment, incompatible with the recognition that
there is a substantial state interest in potential life throughout
pregnancy. Cf. Webster (opinion of Rehnquist, C. J.); Akron I (O'Connor,
J., dissenting).

The very notion that the State has a substantial interest in potential
life leads to the conclusion that not all regulations must be deemed
unwarranted. Not all burdens on the right to decide whether to terminate
a pregnancy will be undue. In our view, the undue burden standard is the
appropriate means of reconciling the State's interest with the woman's
constitutionally protected liberty.

The concept of an undue burden has been utilized by the Court as well as
individual Members of the Court, including two of us, in ways that could
be considered inconsistent. Because we set forth a standard of general
application to which we intend to adhere, it is important to clarify
what is meant by an undue burden.

A finding of an undue burden is a shorthand for the conclusion that a
state regulation has the purpose or effect of placing a substantial
obstacle in the path of a woman seeking an abortion of a nonviable
fetus. A statute with this purpose is invalid because the means chosen
by the State to further the interest in potential life must be
calculated to inform the woman's free choice, not hinder it. And a
statute which, while furthering the interest in potential life or some
other valid state interest, has the effect of placing a substantial
obstacle in the path of a woman's choice cannot be considered a
permissible means of serving its legitimate ends. To the extent that the
opinions of the Court or of individual Justices use the undue burden
standard in a manner that is inconsistent with this analysis, we set out
what in our view should be the controlling standard. Cf. McCleskey v.
Zant (attempting ``to define the doctrine of abuse of the writ with more
precision'' after acknowledging tension among earlier cases). In our
considered judgment, an undue burden is an unconstitutional burden. See
Akron II (opinion of Kennedy, J.). Understood another way, we answer the
question, left open in previous opinions discussing the undue burden
formulation, whether a law designed to further the State's interest in
fetal life which imposes an undue burden on the woman's decision before
fetal viability could be constitutional. See, e. g., Akron I (O'Connor,
J., dissenting). The answer is no.

Some guiding principles should emerge. What is at stake is the woman's
right to make the ultimate decision, not a right to be insulated from
all others in doing so. Regulations which do no more than create a
structural mechanism by which the State, or the parent or guardian of a
minor, may express profound respect for the life of the unborn are
permitted, if they are not a substantial obstacle to the woman's
exercise of the right to choose. See infra (addressing Pennsylvania's
parental consent requirement).

Unless it has that effect on her right of choice, a state measure
designed to persuade her to choose childbirth over abortion will be
upheld if reasonably related to that goal. Regulations designed to
foster the health of a woman seeking an abortion are valid if they do
not constitute an undue burden.

Even when jurists reason from shared premises, some disagreement is
inevitable. That is to be expected in the application of any legal
standard which must accommodate life's complexity. We do not expect it
to be otherwise with respect to the undue burden standard. We give this
summary:

\begin{enumerate}
\def\labelenumi{(\alph{enumi})}
\tightlist
\item
  To protect the central right recognized by Roe v. Wade while at the
  same time accommodating the State's profound interest in potential
  life, we will employ the undue burden analysis as explained in this
  opinion. An undue burden exists, and therefore a provision of law is
  invalid, if its purpose or effect is to place a substantial obstacle
  in the path of a woman seeking an abortion before the fetus attains
  viability.(b) We reject the rigid trimester framework of Roe v. Wade.
  To promote the State's profound interest in potential life, throughout
  pregnancy the State may take measures to ensure that the woman's
  choice is informed, and measures designed to advance this interest
  will not be invalidated as long as their purpose is to persuade the
  woman to choose childbirth over abortion. These measures must not be
  an undue burden on the right.(c) As with any medical procedure, the
  State may enact regulations to further the health or safety of a woman
  seeking an abortion. Unnecessary health regulations that have the
  purpose or effect of presenting a substantial obstacle to a woman
  seeking an abortion impose an undue burden on the right. (d) Our
  adoption of the undue burden analysis does not disturb the central
  holding of Roe v. Wade, and we reaffirm that holding. Regardless of
  whether exceptions are made for particular circumstances, a State may
  not prohibit any woman from making the ultimate decision to terminate
  her pregnancy before viability.(e) We also reaffirm Roe's holding that
  ``subsequent to viability, the State in promoting its interest in the
  potentiality of human life may, if it chooses, regulate, and even
  proscribe, abortion except where it is necessary, in appropriate
  medical judgment, for the preservation of the life or health of the
  mother.'' Roe v. Wade,410 U. S..
\end{enumerate}

These principles control our assessment of the Pennsylvania statute, and
we now turn to the issue of the validity of its challenged provisions.

The Court of Appeals applied what it believed to be the undue burden
standard and upheld each of the provisions except for the husband
notification requirement. We agree generally with this conclusion, but
refine the undue burden analysis in accordance with the principles
articulated above. We now consider the separate statutory sections at
issue.

Because it is central to the operation of various other requirements, we
begin with the statute's definition of medical emergency. Under the
statute, a medical emergency is

``{[}t{]}hat condition which, on the basis of the physician's good faith
clinical judgment, so complicates the medical condition of a pregnant
woman as to necessitate the immediate abortion of her pregnancy to avert
her death or for which a delay will create serious risk of substantial
and irreversible impairment of a major bodily function.'' 18 Pa. Cons.
Stat. § 3203 (1990).

Petitioners argue that the definition is too narrow, contending that it
forecloses the possibility of an immediate abortion despite some
significant health risks. If the contention were correct, we would be
required to invalidate the restrictive operation of the provision, for
the essential holding of Roe forbids a State to interfere with a woman's
choice to undergo an abortion procedure if continuing her pregnancy
would constitute a threat to her health. 410 U. S.. See also Harris v.
McRae.

The District Court found that there were three serious conditions which
would not be covered by the statute: preeclampsia, inevitable abortion,
and premature ruptured membrane. 744 F. Supp.. Yet, as the Court of
Appeals observed, 947 F. 2d, it is undisputed that under some
circumstances each of these conditions could lead to an illness with
substantial and irreversible consequences. While the definition could be
interpreted in an unconstitutional manner, the Court of Appeals
construed the phrase ``serious risk'' to include those circumstances. It
stated: ``{[}W{]}e read the medical emergency exception as intended by
the Pennsylvania legislature to assure that compliance with its abortion
regulations would not in any way pose a significant threat to the life
or health of a woman.'' As we said in Brockett v. Spokane Arcades,
Inc.~(1985): ``Normally, . we defer to the construction of a state
statute given it by the lower federal courts.'' Indeed, we have said
that we will defer to lower court interpretations of state law unless
they amount to ``plain'' error. Palmer v. Hoffman. This
```reflect{[}s{]} our belief that district courts and courts of appeals
are better schooled in and more able to interpret the laws of their
respective States.''' Frisby v. Schultz (citation omitted). We adhere to
that course today, and conclude that, as construed by the Court of
Appeals, the medical emergency definition imposes no undue burden on a
woman's abortion right.

We next consider the informed consent requirement. 18 Pa. Cons. Stat. §
3205 (1990). Except in a medical emergency, the statute requires that at
least 24 hours before performing an abortion a physician inform the
woman of the nature of the procedure, the health risks of the abortion
and of childbirth, and the ``probable gestational age of the unborn
child.'' The physician or a qualified nonphysician must inform the woman
of the availability of printed materials published by the State
describing the fetus and providing information about medical assistance
for childbirth, information about child support from the father, and a
list of agencies which provide adoption and other services as
alternatives to abortion. An abortion may not be performed unless the
woman certifies in writing that she has been informed of the
availability of these printed materials and has been provided them if
she chooses to view them.

Our prior decisions establish that as with any medical procedure, the
State may require a woman to give her written informed consent to an
abortion. See Planned Parenthood of Central Mo. v. Danforth. In this
respect, the statute is unexceptional. Petitioners challenge the
statute's definition of informed consent because it includes the
provision of specific information by the doctor and the mandatory
24-hour waiting period. The conclusions reached by a majority of the
Justices in the separate opinions filed today and the undue burden
standard adopted in this opinion require us to overrule in part some of
the Court's past decisions, decisions driven by the trimester
framework's prohibition of all previability regulations designed to
further the State's interest in fetal life.

In Akron I, we invalidated an ordinance which required that a woman
seeking an abortion be provided by her physician with specific
information ``designed to influence the woman's informed choice between
abortion or childbirth.'' As we later described the Akron I holding in
Thornburghv. American College of Obstetricians and Gynecologists, there
were two purported flaws in the Akron ordinance: the information was
designed to dissuade the woman from having an abortion and the ordinance
imposed ``a rigid requirement that a specific body of information be
given in all cases, irrespective of the particular needs of the patient
. .''

To the extent Akron I and Thornburgh find a constitutional violation
when the government requires, as it does here, the giving of truthful,
nonmisleading information about the nature of the procedure, the
attendant health risks and those of childbirth, and the ``probable
gestational age'' of the fetus, those cases go too far, are inconsistent
with Roe's acknowledgment of an important interest in potential life,
and are overruled. This is clear even on the very terms of Akron I and
Thornburgh. Those decisions, along with Danforth, recognize a
substantial government interest justifying a requirement that a woman be
apprised of the health risks of abortion and childbirth. E. g.,
Danforth. It cannot be questioned that psychological well-being is a
facet of health. Nor can it be doubted that most women considering an
abortion would deem the impact on the fetus relevant, if not
dispositive, to the decision. In attempting to ensure that a woman
apprehend the full consequences of her decision, the State furthers the
legitimate purpose of reducing the risk that a woman may elect an
abortion, only to discover later, with devastating psychological
consequences, that her decision was not fully informed. If the
information the State requires to be made available to the woman is
truthful and not misleading, the requirement may be permissible.

We also see no reason why the State may not require doctors to inform a
woman seeking an abortion of the availability of materials relating to
the consequences to the fetus, even when those consequences have no
direct relation to her health. An example illustrates the point. We
would think it constitutional for the State to require that in order for
there to be informed consent to a kidney transplant operation the
recipient must be supplied with information about risks to the donor as
well as risks to himself or herself. A requirement that the physician
make available information similar to that mandated by the statute here
was described in Thornburgh as ``an outright attempt to wedge the
Commonwealth's message discouraging abortion into the privacy of the
informed-consent dialogue between the woman and her physician.'' 476 U.
S.. We conclude, however, that informed choice need not be defined in
such narrow terms that all considerations of the effect on the fetus are
made irrelevant. As we have made clear, we depart from the holdings of
Akron I and Thornburgh to the extent that we permit a State to further
its legitimate goal of protecting the life of the unborn by enacting
legislation aimed at ensuring a decision that is mature and informed,
even when in so doing the State expresses a preference for childbirth
over abortion. In short, requiring that the woman be informed of the
availability of information relating to fetal development and the
assistance available should she decide to carry the pregnancy to full
term is a reasonable measure to ensure an informed choice, one which
might cause the woman to choose childbirth over abortion. This
requirement cannot be considered a substantial obstacle to obtaining an
abortion, and, it follows, there is no undue burden.

Our prior cases also suggest that the ``straitjacket,''
Thornburgh(quoting Danforth, n.~8), of particular information which must
be given in each case interferes with a constitutional right of privacy
between a pregnant woman and her physician. As a preliminary matter, it
is worth noting that the statute now before us does not require a
physician to comply with the informed consent provisions ``if he or she
can demonstrate by a preponderance of the evidence, that he or she
reasonably believed that furnishing the information would have resulted
in a severely adverse effect on the physical or mental health of the
patient.'' 18 Pa. Cons. Stat. § 3205 (1990). In this respect, the
statute does not prevent the physician from exercising his or her
medical judgment.

Whatever constitutional status the doctor-patient relation may have as a
general matter, in the present context it is derivative of the woman's
position. The doctor-patient relation does not underlie or override the
two more general rights under which the abortion right is justified: the
right to make family decisions and the right to physical autonomy. On
its own, the doctor-patient relation here is entitled to the same
solicitude it receives in other contexts. Thus, a requirement that a
doctor give a woman certain information as part of obtaining her consent
to an abortion is, for constitutional purposes, no different from a
requirement that a doctor give certain specific information about any
medical procedure.

All that is left of petitioners' argument is an asserted First Amendment
right of a physician not to provide information about the risks of
abortion, and childbirth, in a manner mandated by the State. To be sure,
the physician's First Amendment rights not to speak are implicated, see
Wooley v. Maynard, but only as part of the practice of medicine, subject
to reasonable licensing and regulation by the State, cf.~Whalen v. Roe.
We see no constitutional infirmity in the requirement that the physician
provide the information mandated by the State here.

The Pennsylvania statute also requires us to reconsider the holding in
Akron I that the State may not require that a physician, as opposed to a
qualified assistant, provide information relevant to a woman's informed
consent. 462 U. S.. Since there is no evidence on this record that
requiring a doctor to give the information as provided by the statute
would amount in practical terms to a substantial obstacle to a woman
seeking an abortion, we conclude that it is not an undue burden. Our
cases reflect the fact that the Constitution gives the States broad
latitude to decide that particular functions may be performed only by
licensed professionals, even if an objective assessment might suggest
that those same tasks could be performed by others. See Williamson v.
Lee Optical of Okla., Inc.. Thus, we uphold the provision as a
reasonable means to ensure that the woman's consent is informed.

Our analysis of Pennsylvania's 24-hour waiting period between the
provision of the information deemed necessary to informed consent and
the performance of an abortion under the undue burden standard requires
us to reconsider the premise behind the decision in Akron I invalidating
a parallel requirement. In Akron I we said: ``Nor are we convinced that
the State's legitimate concern that the woman's decision be informed is
reasonably served by requiring a 24-hour delay as a matter of course.''
462 U. S.. We consider that conclusion to be wrong. The idea that
important decisions will be more informed and deliberate if they follow
some period of reflection does not strike us as unreasonable,
particularly where the statute directs that important information become
part of the background of the decision. The statute, as construed by the
Court of Appeals, permits avoidance of the waiting period in the event
of a medical emergency and the record evidence shows that in the vast
majority of cases, a 24-hour delay does not create any appreciable
health risk. In theory, at least, the waiting period is a reasonable
measure to implement the State's interest in protecting the life of the
unborn, a measure that does not amount to an undue burden.

Whether the mandatory 24-hour waiting period is nonetheless invalid
because in practice it is a substantial obstacle to a woman's choice to
terminate her pregnancy is a closer question. The findings of fact by
the District Court indicate that because of the distances many women
must travel to reach an abortion provider, the practical effect will
often be a delay of much more than a day because the waiting period
requires that a woman seeking an abortion make at least two visits to
the doctor. The District Court also found that in many instances this
will increase the exposure of women seeking abortions to ``the
harassment and hostility of antiabortion protestors demonstrating
outside a clinic.'' 744 F. Supp.. As a result, the District Court found
that for those women who have the fewest financial resources, those who
must travel long distances, and those who have difficulty explaining
their whereabouts to husbands, employers, or others, the 24-hour waiting
period will be ``particularly burdensome.''

These findings are troubling in some respects, but they do not
demonstrate that the waiting period constitutes an undue burden. We do
not doubt that, as the District Court held, the waiting period has the
effect of ``increasing the cost and risk of delay of abortions,'' but
the District Court did not conclude that the increased costs and
potential delays amount to substantial obstacles. Rather, applying the
trimester framework's strict prohibition of all regulation designed to
promote the State's interest in potential life before viability, see the
District Court concluded that the waiting period does not further the
state ``interest in maternal health'' and ``infringes the physician's
discretion to exercise sound medical judgment,'' Yet, as we have stated,
under the undue burden standard a State is permitted to enact persuasive
measures which favor childbirth over abortion, even if those measures do
not further a health interest. And while the waiting period does limit a
physician's discretion, that is not, standing alone, a reason to
invalidate it. In light of the construction given the statute's
definition of medical emergency by the Court of Appeals, and the
District Court's findings, we cannot say that the waiting period imposes
a real health risk.

We also disagree with the District Court's conclusion that the
``particularly burdensome'' effects of the waiting period on some women
require its invalidation. A particular burden is not of necessity a
substantial obstacle. Whether a burden falls on a particular group is a
distinct inquiry from whether it is a substantial obstacle even as to
the women in that group. And the District Court did not conclude that
the waiting period is such an obstacle even for the women who are most
burdened by it. Hence, on the record before us, and in the context of
this facial challenge, we are not convinced that the 24-hour waiting
period constitutes an undue burden.

We are left with the argument that the various aspects of the informed
consent requirement are unconstitutional because they place barriers in
the way of abortion on demand. Even the broadest reading of Roe,
however, has not suggested that there is a constitutional right to
abortion on demand. See, e. g., Doe v. Bolton. Rather, the right
protected by Roe is a right to decide to terminate a pregnancy free of
undue interference by the State. Because the informed consent
requirement facilitates the wise exercise of that right, it cannot be
classified as an interference with the right Roe protects. The informed
consent requirement is not an undue burden on that right.

Section 3209 of Pennsylvania's abortion law provides, except in cases of
medical emergency, that no physician shall perform an abortion on a
married woman without receiving a signed statement from the woman that
she has notified her spouse that she is about to undergo an abortion.
The woman has the option of providing an alternative signed statement
certifying that her husband is not the man who impregnated her; that her
husband could not be located; that the pregnancy is the result of
spousal sexual assault which she has reported; or that the woman
believes that notifying her husband will cause him or someone else to
inflict bodily injury upon her. A physician who performs an abortion on
a married woman without receiving the appropriate signed statement will
have his or her license revoked, and is liable to the husband for
damages.

The District Court heard the testimony of numerous expert witnesses, and
made detailed findings of fact regarding the effect of this statute.
These included:

"273. The vast majority of women consult their husbands prior to
deciding to terminate their pregnancy. .

. .

"279. The `bodily injury' exception could not be invoked by a married
woman whose husband, if notified, would, in her reasonable belief,
threaten to (a) publicize her intent to have an abortion to family,
friends or acquaintances; (b) retaliate against her in future child
custody or divorce proceedings; (c) inflict psychological intimidation
or emotional harm upon her, her children or other persons; (d) inflict
bodily harm on other persons such as children, family members or other
loved ones; or (e) use his control over finances to deprive of necessary
monies for herself or her children. .

. .

"281. Studies reveal that family violence occurs in two million families
in the United States. This figure, however, is a conservative one that
substantially understates (because battering is usually not reported
until it reaches life-threatening proportions) the actual number of
families affected by domestic violence. In fact, researchers estimate
that one of every two women will be battered at some time in their life.
.

"282. A wife may not elect to notify her husband of her intention to
have an abortion for a variety of reasons, including the husband's
illness, concern about her own health, the imminent failure of the
marriage, or the husband's absolute opposition to the abortion. .

"283. The required filing of the spousal consent form would require
plaintiff-clinics to change their counseling procedures and force women
to reveal their most intimate decision-making on pain of criminal
sanctions. The confidentiality of these revelations could not be
guaranteed, since the woman's records are not immune from subpoena. .

"284. Women of all class levels, educational backgrounds, and racial,
ethnic and religious groups are battered. .

"285. Wife-battering or abuse can take on many physical and
psychological forms. The nature and scope of the battering can cover a
broad range of actions and be gruesome and torturous. .

"286. Married women, victims of battering, have been killed in
Pennsylvania and throughout the United States. .

"287. Battering can often involve a substantial amount of sexual abuse,
including marital rape and sexual mutilation. .

"288. In a domestic abuse situation, it is common for the battering
husband to also abuse the children in an attempt to coerce the wife. .

"289. Mere notification of pregnancy is frequently a flashpoint for
battering and violence within the family. The number of battering
incidents is high during the pregnancy and often the worst abuse can be
associated with pregnancy. . The battering husband may deny parentage
and use the pregnancy as an excuse for abuse. .

"290. Secrecy typically shrouds abusive families. Family members are
instructed not to tell anyone, especially police or doctors, about the
abuse and violence. Battering husbands often threaten their wives or her
children with further abuse if she tells an outsider of the violence and
tells her that nobody will believe her. A battered woman, therefore, is
highly unlikely to disclose the violence against her for fear of
retaliation by the abuser. .

"291. Even when confronted directly by medical personnel or other
helping professionals, battered women often will not admit to the
battering because they have not admitted to themselves that they are
battered. .

. .

"294. A woman in a shelter or a safe house unknown to her husband is
not'reasonably likely' to have bodily harm inflicted upon her by her
batterer, however her attempt to notify her husband pursuant to section
3209 could accidentally disclose her whereabouts to her husband. Her
fear of future ramifications would be realistic under the circumstances.

"295. Marital rape is rarely discussed with others or reported to law
enforcement authorities, and of those reported only few are prosecuted.
.

"296. It is common for battered women to have sexual intercourse with
their husbands to avoid being battered. While this type of coercive
sexual activity would be spousal sexual assault as defined by the Act,
many women may not consider it to be so and others would fear disbelief.
.

"297. The marital rape exception to section 3209 cannot be claimed by
women who are victims of coercive sexual behavior other than
penetration. The 90-day reporting requirement of the spousal sexual
assault statute, 18 Pa. Con. Stat. Ann. § 3218(c), further narrows the
class of sexually abused wives who can claim the exception, since many
of these women may be psychologically unable to discuss or report the
rape for several years after the incident. .

``298. Because of the nature of the battering relationship, battered
women are unlikely to avail themselves of the exceptions to section 3209
of the Act, regardless of whether the section applies to them.'' 744 F.
Supp. (footnote omitted).

These findings are supported by studies of domestic violence. The
American Medical Association (AMA) has published a summary of the recent
research in this field, which indicates that in an average 12-month
period in this country, approximately two million women are the victims
of severe assaults by their male partners. In a 1985 survey, women
reported that nearly one of every eight husbands had assaulted their
wives during the past year. The AMA views these figures as ``marked
underestimates,'' because the nature of these incidents discourages
women from reporting them, and because surveys typically exclude the
very poor, those who do not speak English well, and women who are
homeless or in institutions or hospitals when the survey is conducted.
According to the AMA, ``{[}r{]}esearchers on family violence agree that
the true incidence of partner violence is probably double the above
estimates; or four million severely assaulted women per year. Studies on
prevalence suggest that from one-fifth to one-third of all women will be
physically assaulted by a partner or ex-partner during their lifetime.''
AMA Council on Scientific Affairs, Violence Against Women 7 (1991)
(emphasis in original). Thus on an average day in the United States,
nearly 11,000 women are severely assaulted by their male partners. Many
of these incidents involve sexual assault. -4; Shields \& Hanneke,
Battered Wives' Reactions to Marital Rape, in The Dark Side of Families:
Current Family Violence Research 131, 144 (D. Finkelhor, R. Gelles, G.
Hataling, \& M. Straus eds.~1983). In families where wifebeating takes
place, moreover, child abuse is often present as well. Violence Against
Women.

Other studies fill in the rest of this troubling picture. Physical
violence is only the most visible form of abuse. Psychological abuse,
particularly forced social and economic isolation of women, is also
common. L. Walker, The Battered Woman Syndrome 27-28 (1984). Many
victims of domestic violence remain with their abusers, perhaps because
they perceive no superior alternative. Herbert, Silver, \& Ellard,
Coping with an Abusive Relationship: I. How and Why do Women Stay?, 53
J. Marriage \& the Family 311 (1991). Many abused women who find
temporary refuge in shelters return to their husbands, in large part
because they have no other source of income. Aguirre, Why Do They
Return? Abused Wives in Shelters, 30 J. Nat. Assn. of Social Workers
350, 352 (1985). Returning to one's abuser can be dangerous. Recent
Federal Bureau of Investigation statistics disclose that 8 percent of
all homicide victims in the United States are killed by their spouses.
Mercy \& Saltzman, Fatal Violence Among Spouses in the United States,
1976Am. J. Public Health 595 (1989). Thirty percent of female homicide
victims are killed by their male partners. Domestic Violence: Terrorism
in the Home, Hearing before the Subcommittee on Children, Family, Drugs
and Alcoholism of the Senate Committee on Labor and Human Resources,
101st Cong., 2d Sess., 3 (1990).

The limited research that has been conducted with respect to notifying
one's husband about an abortion, although involving samples too small to
be representative, also supports the District Court's findings of fact.
The vast majority of women notify their male partners of their decision
to obtain an abortion. In many cases in which married women do not
notify their husbands, the pregnancy is the result of an extramarital
affair. Where the husband is the father, the primary reason women do not
notify their husbands is that the husband and wife are experiencing
marital difficulties, often accompanied by incidents of violence. Ryan
\& Plutzer, When Married Women Have Abortions: Spousal Notification and
Marital Interaction, 51 J. Marriage \& the Family 41, 44 (1989).

This information and the District Court's findings reinforce what common
sense would suggest. In well-functioning marriages, spouses discuss
important intimate decisions such as whether to bear a child. But there
are millions of women in this country who are the victims of regular
physical and psychological abuse at the hands of their husbands. Should
these women become pregnant, they may have very good reasons for not
wishing to inform their husbands of their decision to obtain an
abortion. Many may have justifiable fears of physical abuse, but may be
no less fearful of the consequences of reporting prior abuse to the
Commonwealth of Pennsylvania. Many may have a reasonable fear that
notifying their husbands will provoke further instances of child abuse;
these women are not exempt from § 3209's notification requirement. Many
may fear devastating forms of psychological abuse from their husbands,
including verbal harassment, threats of future violence, the destruction
of possessions, physical confinement to the home, the withdrawal of
financial support, or the disclosure of the abortion to family and
friends. These methods of psychological abuse may act as even more of a
deterrent to notification than the possibility of physical violence, but
women who are the victims of the abuse are not exempt from § 3209's
notification requirement. And many women who are pregnant as a result of
sexual assaults by their husbands will be unable to avail themselves of
the exception for spousal sexual assault, § 3209(b)(3), because the
exception requires that the woman have notified law enforcement
authorities within 90 days of the assault, and her husband will be
notified of her report once an investigation begins, § 3128(c). If
anything in this field is certain, it is that victims of spousal sexual
assault are extremely reluctant to report the abuse to the government;
hence, a great many spousal rape victims will not be exempt from the
notification requirement imposed by § 3209.

The spousal notification requirement is thus likely to prevent a
significant number of women from obtaining an abortion. It does not
merely make abortions a little more difficult or expensive to obtain;
for many women, it will impose a substantial obstacle. We must not blind
ourselves to the fact that the significant number of women who fear for
their safety and the safety of their children are likely to be deterred
from procuring an abortion as surely as if the Commonwealth had outlawed
abortion in all cases.

Respondents attempt to avoid the conclusion that § 3209 is invalid by
pointing out that it imposes almost no burden at all for the vast
majority of women seeking abortions. They begin by noting that only
about 20 percent of the women who obtain abortions are married. They
then note that of these women about 95 percent notify their husbands of
their own volition. Thus, respondents argue, the effects of § 3209 are
felt by only one percent of the women who obtain abortions. Respondents
argue that since some of these women will be able to notify their
husbands without adverse consequences or will qualify for one of the
exceptions, the statute affects fewer than one percent of women seeking
abortions. For this reason, it is asserted, the statute cannot be
invalid on its face. See Brief for Respondents 83-86. We disagree with
respondents' basic method of analysis.

The analysis does not end with the one percent of women upon whom the
statute operates; it begins there. Legislation is measured for
consistency with the Constitution by its impact on those whose conduct
it affects. For example, we would not say that a law which requires a
newspaper to print a candidate's reply to an unfavorable editorial is
valid on its face because most newspapers would adopt the policy even
absent the law. See Miami Herald Publishing Co.~v. Tornillo. The proper
focus of constitutional inquiry is the group for whom the law is a
restriction, not the group for whom the law is irrelevant.

Respondents' argument itself gives implicit recognition to this
principle, at one of its critical points. Respondents speak of the one
percent of women seeking abortions who are married and would choose not
to notify their husbands of their plans. By selecting as the controlling
class women who wish to obtain abortions, rather than all women or all
pregnant women, respondents in effect concede that § 3209 must be judged
by reference to those for whom it is an actual rather than an irrelevant
restriction. Of course, as we have said, § 3209's real target is
narrower even than the class of women seeking abortions identified by
the State: it is married women seeking abortions who do not wish to
notify their husbands of their intentions and who do not qualify for one
of the statutory exceptions to the notice requirement. The unfortunate
yet persisting conditions we document above will mean that in a large
fraction of the cases in which § 3209 is relevant, it will operate as a
substantial obstacle to a woman's choice to undergo an abortion. It is
an undue burden, and therefore invalid.

This conclusion is in no way inconsistent with our decisions upholding
parental notification or consent requirements. See, e. g., Akron II;
Bellotti v. Baird (Bellotti II); Planned Parenthood of Central Mo. v.
Danforth. Those enactments, and our judgment that they are
constitutional, are based on the quite reasonable assumption that minors
will benefit from consultation with their parents and that children will
often not realize that their parents have their best interests at heart.
We cannot adopt a parallel assumption about adult women.

We recognize that a husband has a ``deep and proper concern and interest
. in his wife's pregnancy and in the growth and development of the fetus
she is carrying.'' Danforth. With regard to the children he has fathered
and raised, the Court has recognized his ``cognizable and substantial''
interest in their custody. Stanley v. Illinois (1972); see also Quilloin
v. Walcott; Caban v. Mohammed; Lehr v. Robertson. If these cases
concerned a State's ability to require the mother to notify the father
before taking some action with respect to a living child raised by both,
therefore, it would be reasonable to conclude as a general matter that
the father's interest in the welfare of the child and the mother's
interest are equal.

Before birth, however, the issue takes on a very different cast. It is
an inescapable biological fact that state regulation with respect to the
child a woman is carrying will have a far greater impact on the mother's
liberty than on the father's. The effect of state regulation on a
woman's protected liberty is doubly deserving of scrutiny in such a
case, as the State has touched not only upon the private sphere of the
family but upon the very bodily integrity of the pregnant woman. Cf.
Cruzan v. Director, Mo. Dept. of Health. The Court has held that ``when
the wife and the husband disagree on this decision, the view of only one
of the two marriage partners can prevail. Inasmuch as it is the woman
who physically bears the child and who is the more directly and
immediately affected by the pregnancy, as between the two, the balance
weighs in her favor.'' Danforth. This conclusion rests upon the basic
nature of marriage and the nature of our Constitution: ``{[}T{]}he
marital couple is not an independent entity with a mind and heart of its
own, but an association of two individuals each with a separate
intellectual and emotional makeup. If the right of privacy means
anything, it is the right of the individual, married or single, to be
free from unwarranted governmental intrusion into matters so
fundamentally affecting a person as the decision whether to bear or
beget a child.'' Eisenstadt v. Baird (emphasis in original). The
Constitution protects individuals, men and women alike, from unjustified
state interference, even when that interference is enacted into law for
the benefit of their spouses.

There was a time, not so long ago, when a different understanding of the
family and of the Constitution prevailed. In Bradwell v. State, 16 Wall.
130 (1873), three Members of this Court reaffirmed the common-law
principle that ``a woman had no legal existence separate from her
husband, who was regarded as her head and representative in the social
state; and, notwithstanding some recent modifications of this civil
status, many of the special rules of law flowing from and dependent upon
this cardinal principle still exist in full force in most States.''
(Bradley, J., joined by Swayne and Field, JJ., concurring in judgment).
Only one generation has passed since this Court observed that ``woman is
still regarded as the center of home and family life,'' with attendant
``special responsibilities'' that precluded full and independent legal
status under the Constitution. These views, of course, are no longer
consistent with our understanding of the family, the individual, or the
Constitution.

In keeping with our rejection of the common-law understanding of a
woman's role within the family, the Court held in Danforth that the
Constitution does not permit a State to require a married woman to
obtain her husband's consent before undergoing an abortion. 428 U. S..
The principles that guided the Court in Danforth should be our guides
today. For the great many women who are victims of abuse inflicted by
their husbands, or whose children are the victims of such abuse, a
spousal notice requirement enables the husband to wield an effective
veto over his wife's decision. Whether the prospect of notification
itself deters such women from seeking abortions, or whether the husband,
through physical force or psychological pressure or economic coercion,
prevents his wife from obtaining an abortion until it is too late, the
notice requirement will often be tantamount to the veto found
unconstitutional in Danforth. The women most affected by this
law---those who most reasonably fear the consequences of notifying their
husbands that they are pregnant---are in the gravest danger.

The husband's interest in the life of the child his wife is carrying
does not permit the State to empower him with this troubling degree of
authority over his wife. The contrary view leads to consequences
reminiscent of the common law. A husband has no enforceable right to
require a wife to advise him before she exercises her personal choices.
If a husband's interest in the potential life of the child outweighs a
wife's liberty, the State could require a married woman to notify her
husband before she uses a postfertilization contraceptive. Perhaps next
in line would be a statute requiring pregnant married women to notify
their husbands before engaging in conduct causing risks to the fetus.
After all,if the husband's interest in the fetus' safety is a sufficient
predicate for state regulation, the State could reasonably conclude that
pregnant wives should notify their husbands before drinking alcohol or
smoking. Perhaps married women should notify their husbands before using
contraceptives or before undergoing any type of surgery that may have
complications affecting the husband's interest in his wife's
reproductive organs. And if a husband's interest justifies notice in any
of these cases, one might reasonably argue that it justifies exactly
what the Danforth Court held it did not justify---a requirement of the
husband's consent as well. A State may not give to a man the kind of
dominion over his wife that parents exercise over their children.

Section 3209 embodies a view of marriage consonant with the common-law
status of married women but repugnant to our present understanding of
marriage and of the nature of the rights secured by the Constitution.
Women do not lose their constitutionally protected liberty when they
marry. The Constitution protects all individuals, male or female,
married or unmarried, from the abuse of governmental power, even where
that power is employed for the supposed benefit of a member of the
individual's family. These considerations confirm our conclusion that §
3209 is invalid.

We next consider the parental consent provision. Except in a medical
emergency, an unemancipated young woman under 18 may not obtain an
abortion unless she and one of her parents (or guardian) provides
informed consent as defined above. If neither a parent nor a guardian
provides consent, a court may authorize the performance of an abortion
upon a determination that the young woman is mature and capable of
giving informed consent and has in fact given her informed consent, or
that an abortion would be in her best interests.

We have been over most of this ground before. Our cases establish, and
we reaffirm today, that a State may require a minor seeking an abortion
to obtain the consent of a parent or guardian, provided that there is an
adequate judicial bypass procedure. Under these precedents, in our view,
the one-parent consent requirement and judicial bypass procedure are
constitutional.

The only argument made by petitioners respecting this provision and to
which our prior decisions do not speak is the contention that the
parental consent requirement is invalid because it requires informed
parental consent. For the most part, petitioners' argument is a reprise
of their argument with respect to the informed consent requirement in
general, and we reject it for the reasons given above. Indeed, some of
the provisions regarding informed consent have particular force with
respect to minors: the waiting period, for example, may provide the
parent or parents of a pregnant young woman the opportunity to consult
with her in private, and to discuss the consequences of her decision in
the context of the values and moral or religious principles of their
family. See Hodgson (opinion of Stevens, J.).

Under the recordkeeping and reporting requirements of the statute, every
facility which performs abortions is required to file a report stating
its name and address as well as the name and address of any related
entity, such as a controlling or subsidiary organization. In the case of
state-funded institutions, the information becomes public.

For each abortion performed, a report must be filed identifying: the
physician (and the second physician where required); the facility; the
referring physician or agency; the woman's age; the number of prior
pregnancies and prior abortions she has had; gestational age; the type
of abortion procedure; the date of the abortion; whether there were any
pre-existing medical conditions which would complicate pregnancy;
medical complications with the abortion; where applicable, the basis for
the determination that the abortion was medically necessary; the weight
of the aborted fetus; and whether the woman was married, and if so,
whether notice was provided or the basis for the failure to give notice.
Every abortion facility must also file quarterly reports showing the
number of abortions performed broken down by trimester. See 18 Pa. Cons.
Stat. §§ 3207, 3214 (1990). In all events, the identity of each woman
who has had an abortion remains confidential.

In Danforth, we held that recordkeeping and reporting provisions ``that
are reasonably directed to the preservation of maternal health and that
properly respect a patient's confidentiality and privacy are
permissible.'' We think that under this standard, all the provisions at
issue here, except that relating to spousal notice, are constitutional.
Although they do not relate to the State's interest in informing the
woman's choice, they do relate to health. The collection of information
with respect to actual patients is a vital element of medical research,
and so it cannot be said that the requirements serve no purpose other
than to make abortions more difficult. Nor do we find that the
requirements impose a substantial obstacle to a woman's choice. At most
they might increase the cost of some abortions by a slight amount. While
at some point increased cost could become a substantial obstacle, there
is no such showing on the record before us.

Subsection (12) of the reporting provision requires the reporting of,
among other things, a married woman's ``reason for failure to provide
notice'' to her husband. § 3214(a)(12). This provision in effect
requires women, as a condition of obtaining an abortion, to provide the
Commonwealth with the precise information we have already recognized
that many women have pressing reasons not to reveal. Like the spousal
notice requirement itself, this provision places an undue burden on a
woman's choice, and must be invalidated for that reason.

Our Constitution is a covenant running from the first generation of
Americans to us and then to future generations. It is a coherent
succession. Each generation must learn anew that the Constitution's
written terms embody ideas and aspirations that must survive more ages
than one. We accept our responsibility not to retreat from interpreting
the full meaning of the covenant in light of all of our precedents. We
invoke it once again to define the freedom guaranteed by the
Constitution's own promise, the promise of liberty.

The judgment in No.~91-902 is affirmed. The judgment in No.~91-744 is
affirmed in part and reversed in part, and the case is remanded for
proceedings consistent with this opinion, including consideration of the
question of severability.

It is so ordered.

\textbf{Justice Stevens, concurring in part and dissenting in part.} The
portions of the Court's opinion that I have joined are more important
than those with which I disagree. I shall therefore first comment on
significant areas of agreement, and then explain the limited character
of my disagreement.

The Court is unquestionably correct in concluding that the doctrine of
stare decisishas controlling significance in a case of this kind,
notwithstanding an individual Justice's concerns about the merits The
central holding of Roe v. Wade, has been a ``part of our law'' for
almost two decades. Planned Parenthood of Central Mo. v. Danforth
(Stevens, J., concurring in part and dissenting in part). It was a
natural sequel to the protection of individual liberty established in
Griswold v. Connecticut. See also Carey v. Population Services
International, 702 (1977) (White, J., concurring in part and concurring
in result). The societal costs of overruling Roe at this late date would
be enormous. Roe is an integral part of a correct understanding of both
the concept of liberty and the basic equality of men and women.

Stare decisis also provides a sufficient basis for my agreement with the
joint opinion's reaffirmation of Roe's postviability analysis.
Specifically, I accept the proposition that ``{[}i{]}f the State is
interested in protecting fetal life after viability, it may go so far as
to proscribe abortion during that period, except when it is necessary to
preserve the life or health of the mother.'' 410 U. S.; see I also
accept what is implicit in the Court's analysis, namely, a reaffirmation
of Roe`s explanation of why the State's obligation to protect the life
or health of the mother must take precedence over any duty to the
unborn. The Court in Roe carefully considered, and rejected, the State's
argument ``that the fetus is a'person' within the language and meaning
of the Fourteenth Amendment.'' 410 U. S.. After analyzing the usage of
``person'' in the Constitution, the Court concluded that that word ``has
application only postnatally.'' Commenting on the contingent property
interests of the unborn that are generally represented by guardians ad
litem, the Court noted: ``Perfection of the interests involved, again,
has generally been contingent upon live birth. In short, the unborn have
never been recognized in the law as persons in the whole sense.''
Accordingly, an abortion is not ``the termination of life entitled to
Fourteenth Amendment protection.'' From this holding, there was no
dissent, see ; indeed, no Member of the Court has ever questioned this
fundamental proposition. Thus, as a matter of federal constitutional
law, a developing organism that is not yet a ``person'' does not have
what is sometimes described as a ``right to life.'' This has been and,
by the Court's holding today, remains a fundamental premise of our
constitutional law governing reproductive autonomy.

My disagreement with the joint opinion begins with its understanding of
the trimester framework established in Roe. Contrary to the suggestion
of the joint opinion, it is not a ``contradiction'' to recognize that
the State may have a legitimate interest in potential human life and, at
the same time, to conclude that that interest does not justify the
regulation of abortion before viability (although other interests, such
as maternal health, may). The fact that the State's interest is
legitimate does not tell us when, if ever, that interest outweighs the
pregnant woman's interest in personal liberty. It is appropriate,
therefore, to consider more carefully the nature of the interests at
stake.

First, it is clear that, in order to be legitimate, the State's interest
must be secular; consistent with the First Amendment the State may not
promote a theological or sectarian interest. See Thornburgh v. American
College of Obstetricians and Gynecologists (Stevens, J., concurring);
see generally Webster v. Reproductive Health Services (1989) (Stevens,
J., concurring in part and dissenting in part). Moreover, as discussed
above, the state interest in potential human life is not an interest in
loco parentis, for the fetus is not a person.

Identifying the State's interests---which the States rarely articulate
with any precision---makes clear that the interest in protecting
potential life is not grounded in the Constitution. It is, instead, an
indirect interest supported by both humanitarian and pragmatic concerns.
Many of our citizens believe that any abortion reflects an unacceptable
disrespect for potential human life and that the performance of more
than a million abortions each year is intolerable; many find
third-trimester abortions performed when the fetus is approaching
personhood particularly offensive. The State has a legitimate interest
in minimizing such offense. The State may also have a broader interest
in expanding the population,3 believing society would benefit from the
services of additional productive citizens---or that the potential human
lives might include the occasional Mozart or Curie. These are the kinds
of concerns that comprise the State's interest in potential human life.

In counterpoise is the woman's constitutional interest in liberty. One
aspect of this liberty is a right to bodily integrity, a right to
control one's person. See, e. g., Rochin v. California; Skinner v.
Oklahoma ex rel. Williamson. This right is neutral on the question of
abortion: The Constitution would be equally offended by an absolute
requirement that all women undergo abortions as by an absolute
prohibition on abortions. ``Our whole constitutional heritage rebels at
the thought of giving government the power to control men's minds.''
Stanley v. Georgia. The same holds true for the power to control women's
bodies.

The woman's constitutional liberty interest also involves her freedom to
decide matters of the highest privacy and the most personal nature. Cf.
Whalen v. Roe,429 U. S. 589, 598-600 (1977). A woman considering
abortion faces ``a difficult choice having serious and personal
consequences of major importance to her own future---perhaps to the
salvation of her own immortal soul.'' Thornburgh,476 U. S.. The
authority to make such traumatic and yet empowering decisions is an
element of basic human dignity. As the joint opinion so eloquently
demonstrates, a woman's decision to terminate her pregnancy is nothing
less than a matter of conscience.

Weighing the State's interest in potential life and the woman's liberty
interest, I agree with the joint opinion that the State may
``\texttt{"expres{[}s{]}\ a\ preference\ for\ normal\ childbirth,"}''
that the State may take steps to ensure that a woman's choice ``is
thoughtful and informed,'' and that ``States are free to enact laws to
provide a reasonable framework for a woman to make a decision that has
such profound and lasting meaning.'' Serious questions arise, however,
when a State attempts to ``persuade the woman to choose childbirth over
abortion.'' Decisional autonomy must limit the State's power to inject
into a woman's most personal deliberations its own views of what is
best. The State may promote its preferences by funding childbirth, by
creating and maintaining alternatives to abortion, and by espousing the
virtues of family; but it must respect the individual's freedom to make
such judgments.

This theme runs throughout our decisions concerning reproductive
freedom. In general, Roe's requirement that restrictions on abortions
before viability be justified by the State's interest in maternal health
has prevented States from interjecting regulations designed to influence
a woman's decision. Thus, we have upheld regulations of abortion that
are not efforts to sway or direct a woman's choice, but rather are
efforts to enhance the deliberative quality of that decision or are
neutral regulations on the health aspects of her decision. We have, for
example, upheld regulations requiring written informed consent, see
Planned Parenthood of Central Mo. v. Danforth; limited recordkeeping and
reporting, see ; and pathology reports, see Planned Parenthood Assn. of
Kansas City, Mo., Inc.~v. Ashcroft; as well as various licensing and
qualification provisions, see, e. g., Roe; Simopoulos v. Virginia.
Conversely, we have consistently rejected state efforts to prejudice a
woman's choice, either by limiting the information available to her, see
Bigelow v. Virginia, or by ``requir{[}ing{]} the delivery of information
designed'to influence the woman's informed choice between abortion or
childbirth.''' Thornburgh; see also Akron v. Akron Center for
Reproductive Health, Inc.~(1983).

In my opinion, the principles established in this long line of cases and
the wisdom reflected in Justice Powell's opinion for the Court in Akron
(and followed by the Court just six years ago in Thornburgh) should
govern our decision today. Under these principles, Pa. Cons. Stat. §§
3205(a)(2)(i)---(iii) (1990) of the Pennsylvania statute are
unconstitutional. Those sections require a physician or counselor to
provide the woman with a range of materials clearly designed to persuade
her to choose not to undergo the abortion. While the Commonwealth is
free, pursuant to § 3208 of the Pennsylvania law, to produce and
disseminate such material, the Commonwealth may not inject such
information into the woman's deliberations just as she is weighing such
an important choice.

Under this same analysis, §§ 3205(a)(1)(i) and (iii) of the Pennsylvania
statute are constitutional. Those sections, which require the physician
to inform a woman of the nature and risks of the abortion procedure and
the medical risks of carrying to term, are neutral requirements
comparable to those imposed in other medical procedures. Those sections
indicate no effort by the Commonwealth to influence the woman's choice
in any way. If anything, such requirements enhance,rather than skew, the
woman's decisionmaking.

The 24-hour waiting period required by §§ 3205(a)(1)---(2) of the
Pennsylvania statute raises even more serious concerns. Such a
requirement arguably furthers the Commonwealth's interests in two ways,
neither of which is constitutionally permissible.

First, it may be argued that the 24-hour delay is justified by the mere
fact that it is likely to reduce the number of abortions, thus
furthering the Commonwealth's interest in potential life. But such an
argument would justify any form of coercion that placed an obstacle in
the woman's path. The Commonwealth cannot further its interests by
simply wearing down the ability of the pregnant woman to exercise her
constitutional right.

Second, it can more reasonably be argued that the 24-hour delay furthers
the Commonwealth's interest in ensuring that the woman's decision is
informed and thoughtful. But there is no evidence that the mandated
delay benefits women or that it is necessary to enable the physician to
convey any relevant information to the patient. The mandatory delay thus
appears to rest on outmoded and unacceptable assumptions about the
decisionmaking capacity of women. While there are well-established and
consistently maintained reasons for the Commonwealth to view with
skepticism the ability of minors to make decisions, see Hodgson v.
Minnesota,4 none of those reasons applies to an adult woman's
decisionmaking ability. Just as we have left behind the belief that a
woman must consult her husband before undertaking serious matters, see
so we must reject the notion that a woman is less capable of deciding
matters of gravity. Cf. Reed v. Reed.

In the alternative, the delay requirement may be premised on the belief
that the decision to terminate a pregnancy is presumptively wrong. This
premise is illegitimate. Those who disagree vehemently about the
legality and morality of abortion agree about one thing: The decision to
terminate a pregnancy is profound and difficult. No person undertakes
such a decision lightly---and States may not presume that a woman has
failed to reflect adequately merely because her conclusion differs from
the State's preference. A woman who has, in the privacy of her thoughts
and conscience, weighed the options and made her decision cannot be
forced to reconsider all, simply because the State believes she has come
to the wrong conclusion.

Part of the constitutional liberty to choose is the equal dignity to
which each of us is entitled. A woman who decides to terminate her
pregnancy is entitled to the same respect as a woman who decides to
carry the fetus to term. The mandatory waiting period denies women that
equal respect.

In my opinion, a correct application of the ``undue burden'' standard
leads to the same conclusion concerning the constitutionality of these
requirements. A state-imposed burden on the exercise of a constitutional
right is measured both by its effects and by its character: A burden may
be ``undue'' either because the burden is too severe or because it lacks
a legitimate, rational justification.

The 24-hour delay requirement fails both parts of this test. The
findings of the District Court establish the severity of the burden that
the 24-hour delay imposes on many pregnant women. Yet even in those
cases in which the delay is not especially onerous, it is,in my opinion,
``undue'' because there is no evidence that such a delay serves a useful
and legitimate purpose. As indicated above, there is no legitimate
reason to require a woman who has agonized over her decision to leave
the clinic or hospital and return again another day. While a general
requirement that a physician notify her patients about the risks of a
proposed medical procedure is appropriate, a rigid requirement that all
patients wait 24 hours or (what is true in practice) much longer to
evaluate the significance of information that is either common knowledge
or irrelevant is an irrational and, therefore, ``undue'' burden.

The counseling provisions are similarly infirm. Whenever government
commands private citizens to speak or to listen, careful review of the
justification for that command is particularly appropriate. In these
cases, the Pennsylvania statute directs that counselors provide women
seeking abortions with information concerning alternatives to abortion,
the availability of medical assistance benefits, and the possibility of
child-support payments. §§ 3205(a)(2)(i)---(iii). The statute requires
that this information be given to all women seeking abortions, including
those for whom such information is clearly useless, such as those who
are married, those who have undergone the procedure in the past and are
fully aware of the options, and those who are fully convinced that
abortion is their only reasonable option. Moreover, the statute requires
physicians to inform all of their patients of ``{[}t{]}he probable
gestational age of the unborn child.'' § 3205(a)(1)(ii). This
information is of little decisional value in most cases, because 90\% of
all abortions are performed during the first trimester7 when fetal age
has less relevance than when the fetus nears viability. Nor can the
information required by the statute be justified as relevant to any
``philosophic'' or ``social'' argument, either favoring or disfavoring
the abortion decision in a particular case. In light of all of these
facts, I conclude that the information requirements in § 3205(a)(1)(ii)
and §§ 3205(a)(2)(i)---(iii) do not serve a useful purpose and thus
constitute an unnecessary--- and therefore undue---burden on the woman's
constitutional liberty to decide to terminate her pregnancy.

Accordingly, while I disagree with Parts IV, V---B, and V---D of the
joint opinion,8 I join the remainder of the Court's opinion.

It is sometimes useful to view the issue of stare decisis from a
historical perspective. In the last 19 years, 15 Justices have
confronted the basic issue presented in Roe v.Wade. Of those, 11 have
voted as the majority does today: Chief Justice Burger, Justices
Douglas, Brennan, Stewart, Marshall, and Powell, and Justices Blackmun,
O'Connor, Kennedy, Souter, and myself. Only four---all of whom happen to
be on the Court today---have reached the opposite conclusion.

Professor Dworkin has made this comment on the issue: "The suggestion
that states are free to declare a fetus a person. . assumes that a state
can curtail some persons' constitutional rights by adding new persons to
the constitutional population. The constitutional rights of one citizen
are of course very much affected by who or what else also has
constitutional rights, because the rights of others may compete or
conflict with his. So any power to increase the constitutional
population by unilateral decision would be, in effect, a power to
decrease rights the national Constitution grants to others.

``. . If a state could declare trees to be persons with a constitutional
right to life, it could prohibit publishing newspapers or books in spite
of the First Amendment's guarantee of free speech, which could not be
understood as a license to kill. . Once we understand that the
suggestion we are considering has that implication, we must reject it.
If a fetus is not part of the constitutional population, under the
national constitutional arrangement, then states have no power to
overrule that national arrangement by themselves declaring that fetuses
have rights competitive with the constitutional rights of pregnant
women.'' Unenumerated Rights: Whether and How Roe Should be Overruled,
59 U. Chi. L. Rev.~381, 400--- 401 (1992).

The state interest in protecting potential life may be compared to the
state interest in protecting those who seek to immigrate to this
country. A contemporary example is provided by the Haitians who have
risked the perils of the sea in a desperate attempt to become
``persons'' protected by our laws. Humanitarian and practical concerns
would support a state policy allowing those persons unrestricted entry;
countervailing interests in population control support a policy of
limiting the entry of these potential citizens. While the state interest
in population control might be sufficient to justify strict enforcement
of the immigration laws, that interest would not be sufficient to
overcome a woman's liberty interest. Thus, a state interest in
population control could not justify a state-imposed limit on family
size or, for that matter, state-mandated abortions.

As we noted in that opinion, the State's ``legitimate interest in
protecting minor women from their own immaturity'' distinguished that
case from Akron v. Akron Center for Reproductive Health, Inc., which
involved ``a provision that required that mature women, capable of
consenting to an abortion, wait 24 hours after giving consent before
undergoing an abortion.'' Hodgson, n.~35.

The joint opinion's reliance on the indirect effects of the regulation
of constitutionally protected activity, see is misplaced; what matters
is not only the effect of a regulation but also the reason for the
regulation. As I explained in Hodgson: ``In cases involving abortion, as
in cases involving the right to travel or the right to marry, the
identification of the constitutionally protected interest is merely the
beginning of the analysis. State regulation of travel and of marriage is
obviously permissible even though a State may not categorically exclude
nonresidents from its borders, Shapiro v. Thompson, or deny prisoners
the right to marry, Turner v. Safley (1987). But the regulation of
constitutionally protected decisions, such as where a person shall
reside or whom he or she shall marry, must be predicated on legitimate
state concerns other than disagreement with the choice the individual
has made. Cf. Turner v. Safley; Loving v. Virginia. In the abortion
area, a State may have no obligation to spend its own money, or use its
own facilities, to subsidize nontherapeutic abortions for minors or
adults. See, e. g., Maher v. Roe; cf.~Webster v. Reproductive Health
Services (1989) (O'Connor, J., concurring in part and concurring in
judgment). A State's value judgment favoring childbirth over abortion
may provide adequate support for decisions involving such allocation of
public funds, but not for simply substituting a state decision for an
individual decision that a woman has a right to make for herself.
Otherwise, the interest in liberty protected by the Due Process Clause
would be a nullity. A state policy favoring childbirth over abortion is
not in itself a sufficient justification for overriding the woman's
decision or for placing'obstacles---absolute or otherwise---in the
pregnant woman's path to an abortion.'''

The meaning of any legal standard can only be understood by reviewing
the actual cases in which it is applied. For that reason, I discount
both Justice Scalia's comments on past descriptions of the standard, see
post (opinion concurring in judgment in part and dissenting in part),
and the attempt to give it crystal clarity in the joint opinion. The
several opinions supporting the judgment in Griswold v. Connecticut, are
less illuminating than the central holding of the case, which appears to
have passed the test of time. The future may also demonstrate that a
standard that analyzes both the severity of a regulatory burden and the
legitimacy of its justification will provide a fully adequate framework
for the review of abortion legislation even if the contours of the
standard are not authoritatively articulated in any single opinion.

U. S. Dept. of Commerce, Bureau of the Census, Statistical Abstract of
the United States 71 (111th ed.~1991).

Although I agree that a parental-consent requirement (with the
appropriate bypass) is constitutional, I do not join Part V---D of the
joint opinion because its approval of Pennsylvania's informed
parental-consent requirement is based on the reasons given in Part
V---B, with which I disagree. Justice Blackmun, concurring in part,
concurring in the judgment in part, and dissenting in part. I join Parts
I, II, III, V---A, V---C, and of the joint opinion of Justices O'Connor,
Kennedy, and Souter, ante.

Three years ago, in Webster v. Reproductive Health Services, four
Members of this Court appeared poised to ``cas{[}t{]} into darkness the
hopes and visions of every woman in this country'' who had come to
believe that the Constitution guaranteed her the right to reproductive
choice. All that remained between the promise of Roe and the darkness of
the plurality was a single, flickering flame. Decisions since Webster
gave little reason to hope that this flame would cast much light. But
now, just when so many expected the darkness to fall, the flame has
grown bright.

I do not underestimate the significance of today's joint opinion. Yet I
remain steadfast in my belief that the right to reproductive choice is
entitled to the full protection afforded by this Court before Webster.
And I fear for the darkness as four Justices anxiously await the single
vote necessary to extinguish the light.

Make no mistake, the joint opinion of Justices O'Connor, Kennedy, and
Souter is an act of personal courage and constitutional principle. In
contrast to previous decisions in which Justices O'Connor and Kennedy
postponed reconsideration of Roe v. Wade, the authors of the joint
opinion today join Justice Stevens and me in concluding that ``the
essential holding of Roe v. Wade should be retained and once again
reaffirmed.'' In brief, five Members of this Court today recognize that
``the Constitution protects a woman's right to terminate her pregnancy
in its early stages.''

A fervent view of individual liberty and the force of stare decisis have
led the Court to this conclusion. Today a majority reaffirms that the
Due Process Clause of the Fourteenth Amendment establishes ``a realm of
personal liberty which the government may not enter,'' --- a realm whose
outer limits cannot be determined by interpretations of the Constitution
that focus only on the specific practices of States at the time the
Fourteenth Amendment was adopted. See Included within this realm of
liberty is ```the right of the individual, married or single, to be free
from unwarranted governmental intrusion into matters so fundamentally
affecting a person as the decision whether to bear or beget a child.'''
quoting Eisenstadt v. Baird (emphasis in original). ``These matters,
involving the most intimate and personal choices a person may make in a
lifetime, choices central to personal dignity and autonomy, are central
to the liberty protected by the Fourteenth Amendment.'' (emphasis
added). Finally, the Court today recognizes that in the case of
abortion, ``the liberty of the woman is at stake in a sense unique to
the human condition and so unique to the law. The mother who carries a
child to full term is subject to anxieties, to physical constraints, to
pain that only she must bear.''

The Court's reaffirmation of Roe's central holding is also based on the
force of stare decisis. ``{[}N{]}o erosion of principle going to liberty
or personal autonomy has left Roe's central holding a doctrinal remnant;
Roe portends no developments at odds with other precedent for the
analysis of personal liberty; and no changes of fact have rendered
viability more or less appropriate as the point at which the balance of
interests tips.'' Indeed, the Court acknowledges that Roe's limitation
on state power could not be removed ``without serious inequity to those
who have relied upon it or significant damage to the stability of the
society governed by it.'' In the 19 years since Roe was decided, that
case has shaped more than reproductive planning---``{[}a{]}n entire
generation has come of age free to assume Roe's concept of liberty in
defining the capacity of women to act in society, and to make
reproductive decisions.'' The Court understands that, having
``call{[}ed{]} the contending sides . to end their national division by
accepting a common mandate rooted in the Constitution,'' a decision to
overrule Roe ``would seriously weaken the Court's capacity to exercise
the judicial power and to function as the Supreme Court of a Nation
dedicated to the rule of law.'' What has happened today should serve as
a model for future Justices and a warning to all who have tried to turn
this Court into yet another political branch.

In striking down the Pennsylvania statute's spousal notification
requirement, the Court has established a framework for evaluating
abortion regulations that responds to the social context of women facing
issues of reproductive choice In determining the burden imposed by the
challenged regulation, the Court inquires whether the regulation's
``purpose or effect is to place a substantial obstacle in the path of a
woman seeking an abortion before the fetus attains viability.''
(emphasis added). The Court reaffirms: ``The proper focus of
constitutional inquiry is the group for whom the law is a restriction,
not the group for whom the law is irrelevant.'' Looking at this group,
the Court inquires, based on expert testimony, empirical studies, and
common sense, whether ``in a large fraction of the cases in which {[}the
restriction{]} is relevant, it will operate as a substantial obstacle to
a woman's choice to undergo an abortion.'' ``A statute with this purpose
is invalid because the means chosen by the State to further the interest
in potential life must be calculated to inform the woman's free choice,
not hinder it.'' And in applying its test, the Court remains sensitive
to the unique role of women in the decision making process. Whatever may
have been the practice when the Fourteenth Amendment was adopted, the
Court observes, ``{[}w{]}omen do not lose their constitutionally
protected liberty when they marry. The Constitution protects all
individuals, male or female, married or unmarried, from the abuse of
governmental power, even where that power is employed for the supposed
benefit of a member of the individual's family.''

Lastly, while I believe that the joint opinion errs in failing to
invalidate the other regulations, I am pleased that the joint opinion
has not ruled out the possibility that these regulations may be shown to
impose an unconstitutional burden. The joint opinion makes clear that
its specific holdings are based on the insufficiency of the record
before it. See, e. g., I am confident that in the future evidence will
be produced to show that ``in a large fraction of the cases in which
{[}these regulations are{]} relevant, {[}they{]} will operate as a
substantial obstacle to a woman's choice to undergo an abortion.''

Today, no less than yesterday, the Constitution and decisions of this
Court require that a State's abortion restrictions be subjected to the
strictest judicial scrutiny. Our precedents and the joint opinion's
principles require us to subject all non-de-minimis abortion regulations
to strict scrutiny. Under this standard, the Pennsylvania statute's
provisions requiring content-based counseling, a 24-hour delay, informed
parental consent, and reporting of abortion-related information must be
invalidated.

The Court today reaffirms the long recognized rights of privacy and
bodily integrity. As early as 1891, the Court held, ``{[}n{]}o right is
held more sacred, or is more carefully guarded by the common law, than
the right of every individual to the possession and control of his own
person, free from all restraint or interference of others . .'' Union
Pacific R. Co.~v. Botsford. Throughout this century, this Court also has
held that the fundamental right of privacy protects citizens against
governmental intrusion in such intimate family matters as procreation,
childrearing, marriage, and contraceptive choice. See These cases embody
the principle that personal decisions that profoundly affect bodily
integrity, identity, and destiny should be largely beyond the reach of
government. Eisenstadt. In Roe v. Wade, this Court correctly applied
these principles to a woman's right to choose abortion.

State restrictions on abortion violate a woman's right of privacy in two
ways. First, compelled continuation of a pregnancy infringes upon a
woman's right to bodily integrity by imposing substantial physical
intrusions and significant risks of physical harm. During pregnancy,
women experience dramatic physical changes and a wide range of health
consequences. Labor and delivery pose additional health risks and
physical demands. In short, restrictive abortion laws force women to
endure physical invasions far more substantial than those this Court has
held to violate the constitutional principle of bodily integrity in
other contexts. See, e. g., Winston v. Lee (invalidating surgical
removal of bullet from murder suspect); Rochin v. California
(invalidating stomach pumping).

Further, when the State restricts a woman's right to terminate her
pregnancy, it deprives a woman of the right to make her own decision
about reproduction and family planning---critical life choices that this
Court long has deemed central to the right to privacy. The decision to
terminate or continue a pregnancy has no less an impact on a woman's
life than decisions about contraception or marriage. 410 U. S.. Because
motherhood has a dramatic impact on a woman's educational prospects,
employment opportunities, and self-determination, restrictive abortion
laws deprive her of basic control over her life. For these reasons,
``the decision whether or not to beget or bear a child'' lies at ``the
very heart of this cluster of constitutionally protected choices.''
Carey v. Population Services International.

A State's restrictions on a woman's right to terminate her pregnancy
also implicate constitutional guarantees of gender equality. State
restrictions on abortion compel women to continue pregnancies they
otherwise might terminate. By restricting the right to terminate
pregnancies, the State conscripts women's bodies into its service,
forcing women to continue their pregnancies, suffer the pains of
childbirth, and in most instances, provide years of maternal care. The
State does not compensate women for their services; instead, it assumes
that they owe this duty as a matter of course. This assumption---that
women can simply be forced to accept the ``natural'' status and
incidents of motherhood---appears to rest upon a conception of women's
role that has triggered the protection of the Equal Protection Clause.
See, e. g., Mississippi Univ. for Women v. Hogan (1982); Craig v. Boren
(1976) The joint opinion recognizes that these assumptions about women's
place in society ``are no longer consistent with our understanding of
the family, the individual, or the Constitution.''

The Court has held that limitations on the right of privacy are
permissible only if they survive ``strict'' constitutional
scrutiny---that is, only if the governmental entity imposing the
restriction can demonstrate that the limitation is both necessary and
narrowly tailored to serve a compelling governmental interest. Griswold
v. Connecticut. We have applied this principle specifically in the
context of abortion regulations. Roe v. Wade.

Roe implemented these principles through a framework that was designed
``to ensure that the woman's right to choose not become so subordinate
to the State's interest in promoting fetal life that her choice exists
in theory but not in fact,'' Roe identified two relevant state
interests: ``an interest in preserving and protecting the health of the
pregnant woman'' and an interest in ``protecting the potentiality of
human life.'' 410 U. S.. With respect to the State's interest in the
health of the mother, ``the'compelling' point . is at approximately the
end of the first trimester,'' because it is at that point that the
mortality rate in abortion approaches that in childbirth. With respect
to the State's interest in potential life, ``the'compelling' point is at
viability,'' because it is at that point that the fetus ``presumably has
the capability of meaningful life outside the mother's womb.'' In order
to fulfill the requirement of narrow tailoring, ``the State is obligated
to make a reasonable effort to limit the effect of its regulations to
the period in the trimester during which its health interest will be
furthered.'' Akron v. Akron Center for Reproductive Health, Inc..

In my view, application of this analytical framework is no less
warranted than when it was approved by seven Members of this Court in
Roe. Strict scrutiny of state limitations on reproductive choice still
offers the most secure protection of the woman's right to make her own
reproductive decisions, free from state coercion. No majority of this
Court has ever agreed upon an alternative approach. The factual premises
of the trimester framework have not been undermined, see Webster
(Blackmun, J., dissenting), and the Roe framework is far more
administrable, and far less manipulable, than the ``undue burden''
standard adopted by the joint opinion.

Nonetheless, three criticisms of the trimester framework continue to be
uttered. First, the trimester framework is attacked because its key
elements do not appear in the text of the Constitution. My response to
this attack remains the same as it was in Webster:

``Were this a true concern, we would have to abandon most of our
constitutional jurisprudence. {[}T{]}he'critical elements' of countless
constitutional doctrines nowhere appear in the Constitution's text . .
The Constitution makes no mention, for example, of the First
Amendment's'actual malice' standard for proving certain libels, see New
York Times Co.~v. Sullivan.. . Similarly, the Constitution makes no
mention of the rational-basis test, or the specific verbal formulations
of intermediate and strict scrutiny by which this Court evaluates claims
under the Equal Protection Clause. The reason is simple. Like the Roe
framework, these tests or standards are not, and do not purport to be,
rights protected by the Constitution. Rather, they are judge-made
methods for evaluating and measuring the strength and scope of
constitutional rights or for balancing the constitutional rights of
individuals against the competing interests of government.''

The second criticism is that the framework more closely resembles a
regulatory code than a body of constitutional doctrine. Again, my answer
remains the same as in Webster:

"{[}I{]}f this were a true and genuine concern, we would have to abandon
vast areas of our constitutional jurisprudence. . Are {[}the
distinctions entailed in the trimester framework{]} any finer, or
more'regulatory,' than the distinctions we have often drawn in our First
Amendment jurisprudence, where, for example, we have held that a'release
time' program permitting public-school students to leave school grounds
during school hours to receive religious instruction does not violate
the Establishment Clause, even though a release-time program permitting
religious instruction on school grounds does violate the Clause? Compare
Zorach v. Clauson, with Illinois ex rel. Mc- Collum v. Board of
Education of School Dist. No.~71, Champaign County. . Similarly, in a
Sixth Amendment case, the Court held that although an overnight ban on
attorney-client communication violated the constitutionally guaranteed
right to counsel, Geders v. United States, that right was not violated
when a trial judge separated a defendant from his lawyer during a
15-minute recess after the defendant's direct testimony. Perry v. Leeke.

``That numerous constitutional doctrines result in narrow
differentiations between similar circumstances does not mean that this
Court has abandoned adjudication in favor of regulation.'' -550.

The final, and more genuine, criticism of the trimester framework is
that it fails to find the State's interest in potential human life
compelling throughout pregnancy. No Member of this Court---nor for that
matter, the Solicitor General, see Tr. of Oral Arg. 42---has ever
questioned our holding in Roe that an abortion is not ``the termination
of life entitled to Fourteenth Amendment protection.'' 410 U. S..
Accordingly, a State's interest in protecting fetal life is not grounded
in the Constitution. Nor, consistent with our Establishment Clause, can
it be a theological or sectarian interest. See Thornburgh v. American
College of Obstetricians and Gynecologists (Stevens, J., concurring). It
is, instead, a legitimate interest grounded in humanitarian or pragmatic
concerns. See (Stevens, J., concurring in part and dissenting in part).

But while a State has ``legitimate interests from the outset of the
pregnancy in protecting the health of the woman and the life of the
fetus that may become a child,'' legitimate interests are not enough. To
overcome the burden of strict scrutiny, the interests must be
compelling. The question then is how best to accommodate the State's
interest in potential human life with the constitutional liberties of
pregnant women. Again, I stand by the views I expressed in Webster:

``I remain convinced, as six other Members of this Court 16 years ago
were convinced, that the Roe framework, and the viability standard in
particular, fairly, sensibly, and effectively functions to safeguard the
constitutional liberties of pregnant women while recognizing and
accommodating the State's interest in potential human life. The
viability line reflects the biological facts and truths of fetal
development; it marks that threshold moment prior to which a fetus
cannot survive separate from the woman and cannot reasonably and
objectively be regarded as a subject of rights or interests distinct
from, or paramount to, those of the pregnant woman. At the same time,
the viability standard takes account of the undeniable fact that as the
fetus evolves into its postnatal form, and as it loses its dependence on
the uterine environment, the State's interest in the fetus' potential
human life, and in fostering a regard for human life in general, becomes
compelling. As a practical matter, because viability
follows'quickening'---the point at which a woman feels movement in her
womb---and because viability occurs no earlier than 23 weeks gestational
age, it establishes an easily applicable standard for regulating
abortion while providing a pregnant woman ample time to exercise her
fundamental right with her responsible physician to terminate her
pregnancy.'' 492 U. S..

Roe`s trimester framework does not ignore the State's interest in
prenatal life. Like Justice Stevens, I agree that the State may take
steps to ensure that a woman's choice ``is thoughtful and informed,''
and that ``States are free to enact laws to provide a reasonable
framework for a woman to make a decision that has such profound and
lasting meaning.'' But

``{[}s{]}erious questions arise . when a State attempts to persuade the
woman to choose childbirth over abortion. Decisional autonomy must limit
the State's power to inject into a woman's most personal deliberations
its own views of what is best. The State may promote its preferences by
funding childbirth, by creating and maintaining alternatives to
abortion, and by espousing the virtues of family; but it must respect
the individual's freedom to make such judgments.'' (Stevens, J.,
concurring in part and dissenting in part) (internal quotation marks
omitted).

As the joint opinion recognizes, ``the means chosen by the State to
further the interest in potential life must be calculated to inform the
woman's free choice, not hinder it.'' In sum, Roe's requirement of
strict scrutiny as implemented through a trimester framework should not
be disturbed. No other approach has gained a majority, and no other is
more protective of the woman's fundamental right. Lastly, no other
approach properly accommodates the woman's constitutional right with the
State's legitimate interests.

Application of the strict scrutiny standard results in the invalidation
of all the challenged provisions. Indeed, as this Court has invalidated
virtually identical provisions in prior cases, stare decisis requires
that we again strike them down.

This Court has upheld informed- and written-consent requirements only
where the State has demonstrated that they genuinely further important
health-related state concerns. See Planned Parenthood of Central Mo. v.
Danforth (1976). A State may not, under the guise of securing informed
consent, ``require the delivery of information'designed to influence the
woman's informed choice between abortion or childbirth.''' Thornburgh,
quoting Akron,462 U. S.. Rigid requirements that a specific body of
information be imparted to a woman in all cases, regardless of the needs
of the patient, improperly intrude upon the discretion of the pregnant
woman's physician and thereby impose an ```undesired and uncomfortable
straitjacket.''' Thornburgh, quoting Danforth, n.~8.

Measured against these principles, some aspects of the Pennsylvania
informed-consent scheme are unconstitutional. While it is
unobjectionable for the Commonwealth to require that the patient be
informed of the nature of the procedure, the health risks of the
abortion and of childbirth, and the probable gestational age of the
unborn child, compare Pa. Cons. Stat. §§ 3205(a)(i)-(iii) (1990) with
Akron, n.~37, I remain unconvinced that there is a vital state need for
insisting that the information be provided by a physician rather than a
counselor. The District Court found that the physician-only requirement
necessarily would increase costs to the plaintiff clinics, costs that
undoubtedly would be passed on to patients. And because trained women
counselors are often more understanding than physicians, and generally
have more time to spend with patients, see App. 366-387, the
physician-only disclosure requirement is not narrowly tailored to serve
the Commonwealth's interest in protecting maternal health.

Sections 3205(a)(2)(i)-(iii) of the Act further requires that the
physician or a qualified nonphysician inform the woman that printed
materials are available from the Commonwealth that describe the fetus
and provide information about medical assistance for childbirth,
information about child support from the father, and a list of agencies
offering adoption and other services as alternatives to abortion.
Thornburgh invalidated biased patient-counseling requirements virtually
identical to the one at issue here. What we said of those requirements
fully applies in these cases:

"{[}T{]}he listing of agencies in the printed Pennsylvania form presents
serious problems; it contains names of agencies that well may be out of
step with the needs of the particular woman and thus places the
physician in an awkward position and infringes upon his or her
professional responsibilities. Forcing the physician or counselor to
present the materials and the list to the woman makes him or her in
effect an agent of the State in treating the woman and places his or her
imprimatur upon both the materials and the list. All this is, or comes
close to being, state medicine imposed upon the woman, not the
professional medical guidance she seeks, and it officially
structures---as it obviously was intended to do---the dialogue between
the woman and her physician.

``The requirements . that the woman be advised that medical assistance
benefits may be available, and that the father is responsible for
financial assistance in the support of the child similarly are poorly
disguised elements of discouragement for the abortion decision. Much of
this . for many patients, would be irrelevant and inappropriate. For a
patient with a life-threatening pregnancy, the'information' in its very
rendition may be cruel as well as destructive of the physician-patient
relationship. As any experienced social worker or other counselor knows,
theoretical financial responsibility often does not equate with
fulfillment . . Under the guise of informed consent, the Act requires
the dissemination of information that is not relevant to such consent,
and, thus, it advances no legitimate state interest.'' 476 U. S.
(citation omitted).

``This type of compelled information is the antithesis of informed
consent,'' and goes far beyond merely describing the general subject
matter relevant to the woman's decision. ``That the Commonwealth does
not, and surely would not, compel similar disclosure of every possible
peril of necessary surgery or of simple vaccination, reveals the
anti-abortion character of the statute and its real purpose.''

The 24-hour waiting period following the provision of the foregoing
information is also clearly unconstitutional. The District Court found
that the mandatory 24-hour delay could lead to delays in excess of 24
hours, thus increasing health risks, and that it would require two
visits to the abortion provider, thereby increasing travel time,
exposure to further harassment, and financial cost. Finally, the
District Court found that the requirement would pose especially
significant burdens on women living in rural areas and those women that
have difficulty explaining their whereabouts. 744 F. Supp. 1323,
1378-1379 (ED Pa. 1990). In Akron this Court invalidated a similarly
arbitrary or inflexible waiting period because, as here, it furthered no
legitimate state interest.

As Justice Stevens insightfully concludes, the mandatory delay rests
either on outmoded or unacceptable assumptions about the decision making
capacity of women or the belief that the decision to terminate the
pregnancy is presumptively wrong. The requirement that women consider
this obvious and slanted information for an additional 24 hours
contained in these provisions will only influence the woman's decision
in improper ways. The vast majority of women will know this
information---of the few that do not, it is less likely that their minds
will be changed by this information than it will be either by the
realization that the State opposes their choice or the need once again
to endure abuse and harassment on return to the clinic.

Except in the case of a medical emergency, § 3206 requires a physician
to obtain the informed consent of a parent or guardian before performing
an abortion on an unemancipated minor or an incompetent woman. Based on
evidence in the record, the District Court concluded that, in order to
fulfill the informed-consent requirement, generally accepted medical
principles would require an in-person visit by the parent to the
facility. 744 F. Supp.. Although the Court ``has recognized that the
State has somewhat broader authority to regulate the activities of
children than of adults,'' the State nevertheless must demonstrate that
there is a ``significant state interest in conditioning an abortion .
that is not present in the case of an adult.'' Danforth (emphasis
added). The requirement of an in-person visit would carry with it the
risk of a delay of several days or possibly weeks, even where the parent
is willing to consent. While the State has an interest in encouraging
parental involvement in the minor's abortion decision, § 3206 is not
narrowly drawn to serve that interest.

Finally, the Pennsylvania statute requires every facility performing
abortions to report its activities to the Commonwealth. Pennsylvania
contends that this requirement is valid under Danforth, in which this
Court held that recordkeeping and reporting requirements that are
reasonably directed to the preservation of maternal health and that
properly respect a patient's confidentiality are permissible. -81. The
Commonwealth attempts to justify its required reports on the ground that
the public has a right to know how its tax dollars are spent. A
regulation designed to inform the public about public expenditures does
not further the Commonwealth's interest in protecting maternal health.
Accordingly, such a regulation cannot justify a legally significant
burden on a woman's right to obtain an abortion.

The confidential reports concerning the identities and medical judgment
of physicians involved in abortions at first glance may seem valid,
given the Commonwealth's interest in maternal health and enforcement of
the Act. The District Court found, however, that, notwithstanding the
confidentiality protections, many physicians, particularly those who
have previously discontinued performing abortions because of harassment,
would refuse to refer patients to abortion clinics if their names were
to appear on these reports. 744 F. Supp.. The Commonwealth has failed to
show that the name of the referring physician either adds to the pool of
scientific knowledge concerning abortion or is reasonably related to the
Commonwealth's interest in maternal health. I therefore agree with the
District Court's conclusion that the confidential reporting requirements
are unconstitutional insofar as they require the name of the referring
physician and the basis for his or her medical judgment.

In sum, I would affirm the judgment in No.~91-902 and reverse the
judgment in No.~91-744 and remand the cases for further proceedings.

At long last, The Chief Justice and those who have joined him admit it.
Gone are the contentions that the issue need not be (or has not been)
considered. There, on the first page, for all to see, is what was
expected: ``We believe that Roe was wrongly decided, and that it can and
should be overruled consistently with our traditional approach to stare
decisis in constitutional cases.'' Post. If there is much reason to
applaud the advances made by the joint opinion today, there is far more
to fear from The Chief Justice's opinion.

The Chief Justice's criticism of Roe follows from his stunted conception
of individual liberty. While recognizing that the Due Process Clause
protects more than simple physical liberty, he then goes on to construe
this Court's personal-liberty cases as establishing only a laundry list
of particular rights, rather than a principled account of how these
particular rights are grounded in a more general right of privacy. Post.
This constricted view is reinforced by The Chief Justice's exclusive
reliance on tradition as a source of fundamental rights. He argues that
the record in favor of a right to abortion is no stronger than the
record in Michael H. v. Gerald D., where the plurality found no
fundamental right to visitation privileges by an adulterous father, or
in Bowers v. Hardwick, where the Court found no fundamental right to
engage in homosexual sodomy, or in a case involving the ```firing
{[}of{]} a gun . into another person's body.''' Post. In The Chief
Justice's world, a woman considering whether to terminate a pregnancy is
entitled to no more protection than adulterers, murderers, and so-called
sexual deviates Given The Chief Justice's exclusive reliance on
tradition, people using contraceptives seem the next likely candidate
for his list of outcasts.

Even more shocking than The Chief Justice's cramped notion of individual
liberty is his complete omission of any discussion of the effects that
compelled childbirth and motherhood have on women's lives. The only
expression of concern with women's health is purely instrumental---for
The Chief Justice, only women's psychological health is a concern, and
only to the extent that he assumes that every woman who decides to have
an abortion does so without serious consideration of the moral
implications of her decision. Post. In short, The Chief Justice's view
of the State's compelling interest in maternal health has less to do
with health than it does with compelling women to be maternal.

Nor does The Chief Justice give any serious consideration to the
doctrine of stare decisis. For The Chief Justice, the facts that gave
rise to Roe are surprisingly simple: ``women become pregnant, there is a
point somewhere, depending on medical technology, where a fetus becomes
viable, and women give birth to children.'' Post. This characterization
of the issue thus allows The Chief Justice quickly to discard the joint
opinion's reliance argument by asserting that ``reproductive planning
could take virtually immediate account of'' a decision overruling Roe.
Post (internal quotation marks omitted).

The Chief Justice's narrow conception of individual liberty and stare
decisis leads him to propose the same standard of review proposed by the
plurality in Webster.``States may regulate abortion procedures in ways
rationally related to a legitimate state interest. Williamson v. Lee
Optical of Oklahoma, Inc.; cf.~Stanley v. Illinois (1972).'' Post. The
Chief Justice then further weakens the test by providing an
insurmountable requirement for facial challenges: Petitioners must
```show that no set of circumstances exists under which the
{[}provision{]} would be valid.''' Post, quoting Ohio v. Akron Center
for Reproductive Health. In short, in his view, petitioners must prove
that the statute cannot constitutionally be applied to anyone. Finally,
in applying his standard to the spousal-notification provision, The
Chief Justice contends that the record lacks any ``hard evidence'' to
support the joint opinion's contention that a ``large fraction'' of
women who prefer not to notify their husbands involve situations of
battered women and unreported spousal assault. Post, n.~2. Yet
throughout the explication of his standard, The Chief Justice never
explains what hard evidence is, how large a fraction is required, or how
a battered woman is supposed to pursue an as-applied challenge.

Under his standard, States can ban abortion if that ban is rationally
related to a legitimate state interest---a standard which the United
States calls ``deferential, but not toothless.'' Yet when pressed at
oral argument to describe the teeth, the best protection that the
Solicitor General could offer to women was that a prohibition, enforced
by criminal penalties, with no exception for the life of the mother,
``could raise very serious questions.'' Tr. of Oral Arg. 48. Perhaps,
the Solicitor General offered, the failure to include an exemption for
the life of the mother would be ``arbitrary and capricious.'' If, as The
Chief Justice contends, the undue burden test is made out of whole
cloth, the so-called ``arbitrary and capricious'' limit is the Solicitor
General's ``new clothes.''

Even if it is somehow ``irrational'' for a State to require a woman to
risk her life for her child, what protection is offered for women who
become pregnant through rape or incest? Is there anything arbitrary or
capricious about a State's prohibiting the sins of the father from being
visited upon his offspring?

But, we are reassured, there is always the protection of the democratic
process. While there is much to be praised about our democracy, our
country since its founding has recognized that there are certain
fundamental liberties that are not to be left to the whims of an
election. A woman's right to reproductive choice is one of those
fundamental liberties. Accordingly, that liberty need not seek refuge at
the ballot box.

In one sense, the Court's approach is worlds apart from that of The
Chief Justice and Justice Scalia. And yet, in another sense, the
distance between the two approaches is short---the distance is but a
single vote.

I am 83 years old. I cannot remain on this Court forever, and when I do
step down, the confirmation process for my successor well may focus on
the issue before us today. That, I regret, may be exactly where the
choice between the two worlds will be made.

As I shall explain, the joint opinion and I disagree on the appropriate
standard of review for abortion regulations. I do agree, however, that
the reasons advanced by the joint opinion suffice to invalidate the
spousal notification requirement under a strict scrutiny standard.

I also join the Court's decision to uphold the medical emergency
provision. As the Court notes, its interpretation is consistent with the
essential holding ofRoe that ``forbids a State to interfere with a
woman's choice to undergo an abortion procedure if continuing her
pregnancy would constitute a threat to her health.'' As is apparent in
my analysis below, however, this exception does not render
constitutional the provisions which I conclude do not survive strict
scrutiny.

As the joint opinion acknowledges, this Court has recognized the vital
liberty interest of persons in refusing unwanted medical treatment.
Cruzan v. Director, Mo. Dept. of Health. Just as the Due Process Clause
protects the deeply personal decision of the individual to refuse
medical treatment, it also must protect the deeply personal decision to
obtain medical treatment, including a woman's decision to terminate a
pregnancy.

A growing number of commentators are recognizing this point. See, e. g.,
L. Tribe, American Constitutional Law § 15-10, pp.~1353-1359 (2d
ed.~1988); Siegel, Reasoning from the Body: A Historical Perspective on
Abortion Regulation and Questions of Equal Protection, 44 Stan. L.
Rev.~261, 350-380 (1992); Sunstein, Neutrality in Constitutional Law
(With Special Reference to Pornography, Abortion, and Surrogacy), 92
Colum. L. Rev.~1, 31-44 (1992); cf.~Rubenfeld, The Right of Privacy, 102
Harv. L. Rev.~737, 788-791 (1989) (similar analysis under the rubric of
privacy); MacKinnon, Reflections on Sex Equality Under Law, 100 Yale L.
J. 1281, 1308-1324 (1991).

To say that restrictions on a right are subject to strict scrutiny is
not to say that the right is absolute. Regulations can be upheld if they
have no significant impact on the woman's exercise of her right and are
justified by important state health objectives. See, e. g., Planned
Parenthood of Central Mo. v. Danforth, 79-81 (1976) (upholding
requirements of a woman's written consent and record keeping). But the
Court today reaffirms the essential principle of Roe that a woman has
the right ``to choose to have an abortion before viability and to obtain
it without undue interference from the State.'' Under Roe, any more than
de minimis interference is undue.

The joint opinion agrees with Roe`s conclusion that viability occurs or
24 weeks at the earliest. Compare with Roe v. Wade.

While I do not agree with the joint opinion's conclusion that these
provisions should be upheld, the joint opinion has remained faithful to
principles this Court previously has announced in examining counseling
provisions. For example, the joint opinion concludes that the
``information the State requires to be made available to the woman''
must be ``truthful and not misleading.'' Because the State's information
must be ``calculated to inform the woman's free choice, not hinder it,''
the measures must be designed to ensure that a woman's choice is
``mature and informed,'' not intimidated, imposed, or impelled. To this
end, when the State requires the provision of certain information, the
State may not alter the manner of presentation in order to inflict
``psychological abuse,'' designed to shock or unnerve a woman seeking to
exercise her liberty right. This, for example, would appear to preclude
a State from requiring a woman to view graphic literature or films
detailing the performance of an abortion operation. Just as a visual
preview of an operation to remove an appendix plays no part in a
physician's securing informed consent to an appendectomy, a preview of
scenes appurtenant to any major medical intrusion into the human body
does not constructively inform the decision of a woman of the State's
interest in the preservation of the woman's health or demonstrate the
State's ``profound respect for the life of the unborn.''

The Court's decision in Hodgson v. Minnesota, validating a 48-hour
waiting period for minors seeking an abortion to permit parental
involvement does not alter this conclusion. Here the 24-hour delay is
imposed on an adult woman. See -450, n.~35; Ohio v. Akron Center for
Reproductive Health, Inc.. Moreover, the statute in Hodgson did not
require any delay once the minor obtained the affirmative consent of
either a parent or the court.

Because this information is so widely known, I am confident that a
developed record can be made to show that the 24-hour delay, ``in a
large fraction of the cases in which {[}the restriction{]} is relevant,
. will operate as a substantial obstacle to a woman's choice to undergo
an abortion.''

The judicial-bypass provision does not cure this violation. Hodgson is
distinguishable, since these cases involve more than parental
involvement or approval---rather, the Pennsylvania law requires that the
parent receive information designed to discourage abortion in a
face-to-face meeting with the physician. The bypass procedure cannot
ensure that the parent would obtain the information, since in many
instances, the parent would not even attend the hearing. A State may not
place any restriction on a young woman's right to an abortion,however
irrational,simply because it has provided a judicial bypass.

Obviously, I do not share The Chief Justice's views of homosexuality as
sexual deviance. See Bowers, n.~2.

Justice Scalia urges the Court to ``get out of this area,'' post, and
leave questions regarding abortion entirely to the States, post. Putting
aside the fact that what he advocates is nothing short of an abdication
by the Court of its constitutional responsibilities, Justice Scalia is
uncharacteristically naive if he thinks that overruling Roe and holding
that restrictions on a woman's right to an abortion are subject only to
rational-basis review will enable the Court henceforth to avoid
reviewing abortion-related issues. State efforts to regulate and
prohibit abortion in a post-Roe world undoubtedly would raise a host of
distinct and important constitutional questions meriting review by this
Court. For example, does the Eighth Amendment impose any limits on the
degree or kind of punishment a State can inflict upon physicians who
perform, or women who undergo, abortions? What effect would differences
among States in their approaches to abortion have on a woman's right to
engage in interstate travel? Does the First Amendment permit States that
choose not to criminalize abortion to ban all advertising providing
information about where and how to obtain abortions?

\textbf{Chief Justice Rehnquist, with whom Justice White, Justice
Scalia, and Justice Thomas join, concurring in the judgment in part and
dissenting in part.}

The joint opinion, following its newly minted variation on stare
decisis, retains the outer shell of Roe v. Wade, but beats a wholesale
retreat from the substance of that case. We believe that Roe was wrongly
decided, and that it can and should be overruled consistently with our
traditional approach to stare decisis in constitutional cases. We would
adopt the approach of the plurality in Webster v. Reproductive Health
Services, and uphold the challenged provisions of the Pennsylvania
statute in their entirety.

In ruling on this litigation below, the Court of Appeals for the Third
Circuit first observed that ``this appeal does not directly implicate
Roe; this case involves the regulation of abortions rather than their
outright prohibition.'' Accordingly, the court directed its attention to
the question of the standard of review for abortion regulations. In
attempting to settle on the correct standard, however, the court
confronted the confused state of this Court's abortion jurisprudence.
After considering the several opinions in Webster v. Reproductive Health
Servicesand Hodgson v. Minnesota, the Court of Appeals concluded that
Justice O'Connor's ``undue burden'' test was controlling, as that was
the narrowest ground on which we had upheld recent abortion regulations.
947 F. 2d (``When a fragmented court decides a case and no single
rationale explaining the result enjoys the assent of five Justices, the
holding of the Court may be viewed as that position taken by those
Members who concurred in the judgments on the narrowest grounds''
(quoting Marks v. United States (internal quotation marks omitted))).
Applying this standard, the Court of Appeals upheld all of the
challenged regulations except the one requiring a woman to notify her
spouse of an intended abortion.

In arguing that this Court should invalidate each of the provisions at
issue, petitioners insist that we reaffirm our decision in Roe v. Wadein
which we held unconstitutional a Texas statute making it a crime to
procure an abortion except to save the life of the mother We agree with
the Court of Appeals that our decision in Roe is not directly implicated
by the Pennsylvania statute, which does not prohibit, but simply
regulates, abortion. But, as the Court of Appeals found, the state of
our post-Roe decisional law dealing with the regulation of abortion is
confusing and uncertain, indicating that a reexamination of that line of
cases is in order. Unfortunately for those who must apply this Court's
decisions, the reexamination undertaken today leaves the Court no less
divided than beforehand. Although they reject the trimester framework
that formed the underpinning of Roe, Justices O'Connor, Kennedy, and
Souter adopt a revised undue burden standard to analyze the challenged
regulations. We conclude, however, that such an outcome is an
unjustified constitutional compromise, one which leaves the Court in a
position to closely scrutinize all types of abortion regulations despite
the fact that it lacks the power to do so under the Constitution.

In Roe, the Court opined that the State ``does have an important and
legitimate interest in preserving and protecting the health of the
pregnant woman, . and that it has still another important and legitimate
interest in protecting the potentiality of human life.'' 410 U. S.
(emphasis omitted). In the companion case of Doe v. Bolton, the Court
referred to its conclusion in Roe ``that a pregnant woman does not have
an absolute constitutional right to an abortion on her demand.'' 410 U.
S.. But while the language and holdings of these cases appeared to leave
States free to regulate abortion procedures in a variety of ways, later
decisions based on them have found considerably less latitude for such
regulations than might have been expected.

For example, after Roe, many States have sought to protect their young
citizens by requiring that a minor seeking an abortion involve her
parents in the decision. Some States have simply required notification
of the parents, while others have required a minor to obtain the consent
of her parents. In a number of decisions, however, the Court has
substantially limited the States in their ability to impose such
requirements. With regard to parental notice requirements, we initially
held that a State could require a minor to notify her parents before
proceeding with an abortion. H. L. v. Matheson (1981). Recently,
however, we indicated that a State's ability to impose a notice
requirement actually depends on whether it requires notice of one or
both parents. We concluded that although the Constitution might allow a
State to demand that notice be given to one parent prior to an abortion,
it may not require that similar notice be given to two parents, unless
the State incorporates a judicial bypass procedure in that two-parent
requirement. Hodgson v. Minnesota.

We have treated parental consent provisions even more harshly. Three
years after Roe, we invalidated a Missouri regulation requiring that an
unmarried woman under the age of 18 obtain the consent of one of her
parents before proceeding with an abortion. We held that our abortion
jurisprudence prohibited the State from imposing such a ``blanket
provision. . requiring the consent of a parent.'' Planned Parenthood of
Central Mo. v. Danforth. In Bellotti v. Baird, the Court struck down a
similar Massachusetts parental consent statute. A majority of the Court
indicated, however, that a State could constitutionally require parental
consent, if it alternatively allowed a pregnant minor to obtain an
abortion without parental consent by showing either that she was mature
enough to make her own decision, or that the abortion would be in her
best interests. In light of Bellotti, we have upheld one parental
consent regulation which incorporated a judicial bypass option we viewed
as sufficient, see Planned Parenthood Assn. of Kansas City, Mo., Inc.~v.
Ashcroft, but have invalidated another because of our belief that the
judicial procedure did not satisfy the dictates of Bellotti, see Akron
v. Akron Center for Reproductive Health, Inc.~(1983). We have never had
occasion, as we have in the parental notice context, to further parse
our parental consent jurisprudence into one-parent and two-parent
components.

In Roe, the Court observed that certain States recognized the right of
the father to participate in the abortion decision in certain
circumstances. Because neither Roenor Doe involved the assertion of any
paternal right, the Court expressly stated that the case did not disturb
the validity of regulations that protected such a right. Roe v. Wade,
n.~67. But three years later, in Danforth, the Court extended its
abortion jurisprudence and held that a State could not require that a
woman obtain the consent of her spouse before proceeding with an
abortion. Planned Parenthood of Central Mo. v. Danforth.

States have also regularly tried to ensure that a woman's decision to
have an abortion is an informed and well-considered one. In Danforth, we
upheld a requirement that a woman sign a consent form prior to her
abortion, and observed that ``it is desirable and imperative that {[}the
decision{]} be made with full knowledge of its nature and
consequences.'' Since that case, however, we have twice invalidated
state statutes designed to impart such knowledge to a woman seeking an
abortion. In Akron, we held unconstitutional a regulation requiring a
physician to inform a woman seeking an abortion of the status of her
pregnancy, the development of her fetus, the date of possible viability,
the complications that could result from an abortion, and the
availability of agencies providing assistance and information with
respect to adoption and childbirth. Akronv. Akron Center for
Reproductive Health. More recently, in Thornburgh v. American College of
Obstetricians and Gynecologists, we struck down a more limited
Pennsylvania regulation requiring that a woman be informed of the risks
associated with the abortion procedure and the assistance available to
her if she decided to proceed with her pregnancy, because we saw the
compelled information as ``the antithesis of informed consent.'' Even
when a State has sought only to provide information that, in our view,
was consistent with the Roe framework, we concluded that the State could
not require that a physician furnish the information, but instead had to
alternatively allow nonphysician counselors to provide it. Akron v.
Akron Center for Reproductive Health. In Akron as well, we went further
and held that a State may not require a physician to wait 24 hours to
perform an abortion after receiving the consent of a woman. Although the
State sought to ensure that the woman's decision was carefully
considered, the Court concluded that the Constitution forbade the State
to impose any sort of delay. -451.

We have not allowed States much leeway to regulate even the actual
abortion procedure. Although a State can require that second-trimester
abortions be performed in outpatient clinics, see Simopoulos v.
Virginia,we concluded in Akron and Ashcroft that a State could not
require that such abortions be performed only in hospitals. See Akron v.
Akron Center for Reproductive Health; Planned Parenthood Assn. of Kansas
City, Mo., Inc.~v. Ashcroft. Despite the fact that Roe expressly allowed
regulation after the first trimester in furtherance of maternal health,
```present medical knowledge,''' in our view, could not justify such a
hospitalization requirement under the trimester framework. Akron v.
Akron Center for Reproductive Health (quoting Roe v. Wade). And in
Danforth, the Court held that Missouri could not outlaw the saline
amniocentesis method of abortion, concluding that the Missouri
Legislature had ``failed to appreciate and to consider several
significant facts'' in making its decision. 428 U. S..

Although Roe allowed state regulation after the point of viability to
protect the potential life of the fetus, the Court subsequently rejected
attempts to regulate in this manner. In Colautti v. Franklin, the Court
struck down a statute that governed the determination of viability.
-397. In the process, we made clear that the trimester framework
incorporated only one definition of viability---ours---as we forbade
States to decide that a certain objective indicator---``be it weeks of
gestation or fetal weight or any other single factor''---should govern
the definition of viability. In that same case, we also invalidated a
regulation requiring a physician to use the abortion technique offering
the best chance for fetal survival when performing postviability
abortions. See -401; see also Thornburgh v. American College of
Obstetricians and Gynecologists,476 U. S. (invalidating a similar
regulation). In Thornburgh, the Court struck down Pennsylvania's
requirement that a second physician be present at post viability
abortions to help preserve the health of the unborn child, on the ground
that it did not incorporate a sufficient medical emergency exception.
-771. Regulations governing the treatment of aborted fetuses have met a
similar fate. In Akron, we invalidated a provision requiring physicians
performing abortions to ``insure that the remains of the unborn child
are disposed of in a humane and sanitary manner.'' 462 U. S. (internal
quotation marks omitted).

Dissents in these cases expressed the view that the Court was expanding
upon Roe in imposing ever greater restrictions on the States. See
Thornburgh v. American College of Obstetricians and Gynecologists
(Burger, C. J., dissenting) (``The extent to which the Court has
departed from the limitations expressed in Roe is readily apparent'');
(White, J., dissenting) (``{[}T{]}he majority indiscriminately strikes
down statutory provisions that in no way contravene the right recognized
in Roe''). And, when confronted with state regulations of this type in
past years, the Court has become increasingly more divided: The three
most recent abortion cases have not commanded a Court opinion. See Ohio
v. Akron Center for Reproductive Health; Hodgson v. Minnesota; Webster
v. Reproductive Health Services.

The task of the Court of Appeals in the present cases was obviously
complicated by this confusion and uncertainty. Following Marks v. United
States, it concluded that in light of Webster and Hodgson, the strict
scrutiny standard enunciated in Roe was no longer applicable, and that
the ``undue burden'' standard adopted by Justice O'Connor was the
governing principle. This state of confusion and disagreement warrants
reexamination of the ``fundamental right'' accorded to a woman's
decision to abort a fetus in Roe, with its concomitant requirement that
any state regulation of abortion survive ``strict scrutiny.'' See Payne
v. Tennessee (1991) (observing that reexamination of constitutional
decisions is appropriate when those decisions have generated uncertainty
and failed to provide clear guidance, because ``correction through
legislative action is practically impossible'' (internal quotation marks
omitted)); Garcia v. San Antonio Metropolitan Transit Authority, 557
(1985).

We have held that a liberty interest protected under the Due Process
Clause of the Fourteenth Amendment will be deemed fundamental if it is
``implicit in the concept of ordered liberty.'' Palko v. Connecticut.
Three years earlier, in Snyder v. Massachusetts, we referred to a
``principle of justice so rooted in the traditions and conscience of our
people as to be ranked as fundamental.'' ; see also Michael H. v. Gerald
D. (plurality opinion) (citing the language from Snyder). These
expressions are admittedly not precise, but our decisions implementing
this notion of ``fundamental'' rights do not afford any more elaborate
basis on which to base such a classification.

In construing the phrase ``liberty'' incorporated in the Due Process
Clause of the Fourteenth Amendment, we have recognized that its meaning
extends beyond freedom from physical restraint. In Pierce v. Society of
Sisters, we held that it included a parent's right to send a child to
private school; in Meyer v. Nebraska, we held that it included a right
to teach a foreign language in a parochial school. Building on these
cases, we have held that the term ``liberty'' includes a right to marry,
Loving v. Virginia, 388 U. S. 1 (1967); a right to procreate, Skinner v.
Oklahoma ex rel. Williamson; and a right to use contraceptives, Griswold
v. Connecticut; Eisenstadt v. Baird. But a reading of these opinions
makes clear that they do not endorse any all-encompassing ``right of
privacy.''

In Roe v. Wade, the Court recognized a ``guarantee of personal privacy''
which ``is broad enough to encompass a woman's decision whether or not
to terminate her pregnancy.'' 410 U. S.. We are now of the view that, in
terming this right fundamental, the Court in Roe read the earlier
opinions upon which it based its decision much too broadly. Unlike
marriage, procreation, and contraception, abortion ``involves the
purposeful termination of a potential life.'' Harris v. McRae. The
abortion decision must therefore ``be recognized as sui generis,
different in kind from the others that the Court has protected under the
rubric of personal or family privacy and autonomy.'' Thornburgh v.
American College of Obstetricians and Gynecologists (White, J.,
dissenting). One cannot ignore the fact that a woman is not isolated in
her pregnancy, and that the decision to abort necessarily involves the
destruction of a fetus. See Michael H. v. Gerald D., n.~4 (To look ``at
the act which is assertedly the subject of a liberty interest in
isolation from its effect upon other people {[}is{]} like inquiring
whether there is a liberty interest in firing a gun where the case at
hand happens to involve its discharge into another person's body'').

Nor do the historical traditions of the American people support the view
that the right to terminate one's pregnancy is ``fundamental.'' The
common law which we inherited from England made abortion after
``quickening'' an offense. At the time of the adoption of the Fourteenth
Amendment, statutory prohibitions or restrictions on abortion were
commonplace; in 1868, at least 28 of the then-37 States and 8
Territories had statutes banning or limiting abortion. J. Mohr, Abortion
in America 200 (1978). By the turn of the century virtually every State
had a law prohibiting or restricting abortion on its books. By the
middle of the present century, a liberalization trend had set in. But 21
of the restrictive abortion laws in effect in 1868 were still in effect
in 1973 when Roe was decided, and an overwhelming majority of the States
prohibited abortion unless necessary to preserve the life or health of
the mother. Roe v. Wade (Rehnquist, J., dissenting). On this record, it
can scarcely be said that any deeply rooted tradition of relatively
unrestricted abortion in our history supported the classification of the
right to abortion as ``fundamental'' under the Due Process Clause of the
Fourteenth Amendment.

We think, therefore, both in view of this history and of our decided
cases dealing with substantive liberty under the Due Process Clause,
that the Court was mistaken in Roe when it classified a woman's decision
to terminate her pregnancy as a ``fundamental right'' that could be
abridged only in a manner which withstood ``strict scrutiny.'' In so
concluding, we repeat the observation made in Bowers v. Hardwick:

``Nor are we inclined to take a more expansive view of our authority to
discover new fundamental rights imbedded in the Due Process Clause. The
Court is most vulnerable and comes nearest to illegitimacy when it deals
with judge-made constitutional law having little or no cognizable roots
in the language or design of the Constitution.''

We believe that the sort of constitutionally imposed abortion code of
the type illustrated by our decisions following Roe is inconsistent
``with the notion of a Constitution cast in general terms, as ours is,
and usually speaking in general principles, as ours does.'' Webster v.
Reproductive Health Services (plurality opinion). The Court in Roe
reached too far when it analogized the right to abort a fetus to the
rights involved in Pierce, Meyer, Loving, and Griswold, and thereby
deemed the right to abortion fundamental.

The joint opinion of Justices O'Connor, Kennedy, and Souter cannot bring
itself to say that Roe was correct as an original matter, but the
authors are of the view that ``the immediate question is not the
soundness of Roe's resolution of the issue, but the precedential force
that must be accorded to its holding.'' Instead of claiming that Roe was
correct as a matter of original constitutional interpretation, the
opinion therefore contains an elaborate discussion of stare decisis.
This discussion of the principle of stare decisis appears to be almost
entirely dicta, because the joint opinion does not apply that principle
in dealing with Roe. Roe decided that a woman had a fundamental right to
an abortion. The joint opinion rejects that view. Roe decided that
abortion regulations were to be subjected to ``strict scrutiny'' and
could be justified only in the light of ``compelling state interests.''
The joint opinion rejects that view. ; see Roe v. Wade. Roe analyzed
abortion regulation under a rigid trimester framework, a framework which
has guided this Court's decision making for 19 years. The joint opinion
rejects that framework.

Stare decisis is defined in Black's Law Dictionary as meaning ``to abide
by, or adhere to, decided cases.'' Black's Law Dictionary 1406 (6th
ed.~1990). Whatever the ``central holding'' of Roe that is left after
the joint opinion finishes dissecting it is surely not the result of
that principle. While purporting to adhere to precedent, the joint
opinion instead revises it. Roe continues to exist, but only in the way
a storefront on a western movie set exists: a mere facade to give the
illusion of reality. Decisions following Roe, such as Akron v. Akron
Center for Reproductive Health, Inc., and Thornburgh v. American College
of Obstetricians and Gynecologists, are frankly overruled in part under
the ``undue burden'' standard expounded in the joint opinion.

In our view, authentic principles of stare decisis do not require that
any portion of the reasoning in Roe be kept intact. ``Stare decisis is
not . a universal, inexorable command,'' especially in cases involving
the interpretation of the Federal Constitution. Burnet v. Coronado Oil
\& Gas Co.~(Brandeis, J., dissenting). Erroneous decisions in such
constitutional cases are uniquely durable, because correction through
legislative action, save for constitutional amendment, is impossible. It
is therefore our duty to reconsider constitutional interpretations that
``depar{[}t{]} from a proper understanding'' of the Constitution. Garcia
v. San Antonio Metropolitan Transit Authority; see United States v.
Scott (```{[}I{]}n cases involving the Federal Constitution, . {[}t{]}he
Court bows to the lessons of experience and the force of better
reasoning, recognizing that the process of trial and error, so fruitful
in the physical sciences, is appropriate also in the judicial
function''' (quoting Burnet v. Coronado Oil \& Gas Co.~(Brandeis, J.,
dissenting))); Smith v. Allwright. Our constitutional watch does not
cease merely because we have spoken before on an issue; when it becomes
clear that a prior constitutional interpretation is unsound we are
obliged to reexamine the question. See, e. g., West Virginia Bd. of Ed.
v. Barnette; Erie R. Co.~v. Tompkins (1938).

The joint opinion discusses several stare decisis factors which, it
asserts, point toward retaining a portion of Roe. Two of these factors
are that the main ``factual underpinning'' of Roe has remained the same,
and that its doctrinal foundation is no weaker now than it was in 1973.
Of course, what might be called the basic facts which gave rise to Roe
have remained the same---women become pregnant, there is a point
somewhere, depending on medical technology, where a fetus becomes
viable, and women give birth to children. But this is only to say that
the same facts which gave rise to Roe will continue to give rise to
similar cases. It is not a reason, in and of itself, why those cases
must be decided in the same incorrect manner as was the first case to
deal with the question. And surely there is no requirement, in
considering whether to depart from stare decisis in a constitutional
case, that a decision be more wrong now than it was at the time it was
rendered. If that were true, the most outlandish constitutional decision
could survive forever, based simply on the fact that it was no more
outlandish later than it was when originally rendered.

Nor does the joint opinion faithfully follow this alleged requirement.
The opinion frankly concludes that Roe and its progeny were wrong in
failing to recognize that the State's interests in maternal health and
in the protection of unborn human life exist throughout pregnancy. But
there is no indication that these components of Roe are any more
incorrect at this juncture than they were at its inception.

The joint opinion also points to the reliance interests involved in this
context in its effort to explain why precedent must be followed for
precedent's sake. Certainly it is true that where reliance is truly at
issue, as in the case of judicial decisions that have formed the basis
for private decisions, ``{[}c{]}onsiderations in favor of stare decisis
are at their acme.'' Payne v. Tennessee. But, as the joint opinion
apparently agrees, any traditional notion of reliance is not applicable
here. The Court today cuts back on the protection afforded by Roe,and no
one claims that this action defeats any reliance interest in the
disavowed trimester framework. Similarly, reliance interests would not
be diminished were the Court to go further and acknowledge the full
error of Roe, as ``reproductive planning could take virtually immediate
account of'' this action. The joint opinion thus turns to what can only
be described as an unconventional---and unconvincing---notion of
reliance, a view based on the surmise that the availability of abortion
since Roe has led to ``two decades of economic and social developments''
that would be undercut if the error of Roe were recognized. The joint
opinion's assertion of this fact is undeveloped and totally conclusory.
In fact, one cannot be sure to what economic and social developments the
opinion is referring. Surely it is dubious to suggest that women have
reached their ``places in society'' in reliance upon Roe, rather than as
a result of their determination to obtain higher education and compete
with men in the job market, and of society's increasing recognition of
their ability to fill positions that were previously thought to be
reserved only for men.

In the end, having failed to put forth any evidence to prove any true
reliance, the joint opinion's argument is based solely on generalized
assertions about the national psyche, on a belief that the people of
this country have grown accustomed to the Roe decision over the last 19
years and have ``ordered their thinking and living around'' it. As an
initial matter, one might inquire how the joint opinion can view the
``central holding'' of Roe as so deeply rooted in our constitutional
culture, when it so casually uproots and disposes of that same
decision's trimester framework. Furthermore, at various points in the
past, the same could have been said about this Court's erroneous
decisions that the Constitution allowed ``separate but equal'' treatment
of minorities, see Plessy v. Ferguson, or that ``liberty'' under the Due
Process Clause protected ``freedom of contract,'' see Adkins v.
Children's Hospital of District of Columbia; Lochner v. New York. The
``separate but equal'' doctrine lasted 58 years after Plessy, and
Lochner's protection of contractual freedom lasted 32 years. However,
the simple fact that a generation or more had grown used to these major
decisions did not prevent the Court from correcting its errors in those
cases, nor should it prevent us from correctly interpreting the
Constitution here. See Brown v. Board of Education (rejecting the
``separate but equal'' doctrine); West Coast Hotel Co.~v. Parrish
(overruling Adkins v. Children's Hospitalin upholding Washington's
minimum wage law).

Apparently realizing that conventional stare decisis principles do not
support its position, the joint opinion advances a belief that retaining
a portion of Roe is necessary to protect the ``legitimacy'' of this
Court. Because the Court must take care to render decisions ``grounded
truly in principle,'' and not simply as political and social
compromises, the joint opinion properly declares it to be this Court's
duty to ignore the public criticism and protest that may arise as a
result of a decision. Few would quarrel with this statement, although it
may be doubted that Members of this Court, holding their tenure as they
do during constitutional ``good behavior,'' are at all likely to be
intimidated by such public protests.

But the joint opinion goes on to state that when the Court
``resolve{[}s{]} the sort of intensely divisive controversy reflected in
Roe and those rare, comparable cases,'' its decision is exempt from
reconsideration under established principles of stare decisis in
constitutional cases. This is so, the joint opinion contends, because in
those ``intensely divisive'' cases the Court has ``call{[}ed{]} the
contending sides of a national controversy to end their national
division by accepting a common mandate rooted in the Constitution,'' and
must therefore take special care not to be perceived as
``surrender{[}ing{]} to political pressure'' and continued opposition.
This is a truly novel principle, one which is contrary to both the
Court's historical practice and to the Court's traditional willingness
to tolerate criticism of its opinions. Under this principle, when the
Court has ruled on a divisive issue, it is apparently prevented from
overruling that decision for the sole reason that it was incorrect,
unless opposition to the original decision has died away.

The first difficulty with this principle lies in its assumption that
cases that are ``intensely divisive'' can be readily distinguished from
those that are not. The question of whether a particular issue is
``intensely divisive'' enough to qualify for special protection is
entirely subjective and dependent on the individual assumptions of the
Members of this Court. In addition, because the Court's duty is to
ignore public opinion and criticism on issues that come before it, its
Members are in perhaps the worst position to judge whether a decision
divides the Nation deeply enough to justify such uncommon protection.
Although many of the Court's decisions divide the populace to a large
degree, we have not previously on that account shied away from applying
normal rules of stare decisis when urged to reconsider earlier
decisions. Over the past 21 years, for example, the Court has overruled
in whole or in part 34 of its previous constitutional decisions. See
Payne v. Tennessee, and n.~1 (listing cases).

The joint opinion picks out and discusses two prior Court rulings that
it believes are of the ``intensely divisive'' variety, and concludes
that they are of comparable dimension to Roe. (discussing Lochner v. New
Yorkand Plessy v. Ferguson). It appears to us very odd indeed that the
joint opinion chooses as benchmarks two cases in which the Court chose
not to adhere to erroneous constitutional precedent, but instead
enhanced its stature by acknowledging and correcting its error,
apparently in violation of the joint opinion's ``legitimacy'' principle.
See West Coast Hotel Co.~v. Parrish; Brown v. Board of Education. One
might also wonder how it is that the joint opinion puts these, and not
others, in the ``intensely divisive'' category, and how it assumes that
these are the only two lines of cases of comparable dimension to Roe.
There is no reason to think that either Plessy or Lochner produced the
sort of public protest when they were decided that Roe did. There were
undoubtedly large segments of the bench and bar who agreed with the
dissenting views in those cases, but surely that cannot be what the
Court means when it uses the term ``intensely divisive,'' or many other
cases would have to be added to the list. In terms of public protest,
however, Roe, so far as we know, was unique. But just as the Court
should not respond to that sort of protest by retreating from the
decision simply to allay the concerns of the protesters, it should
likewise not respond by determining to adhere to the decision at all
costs lest it seem to be retreating under fire. Public protests should
not alter the normal application of stare decisis, lest perfectly lawful
protest activity be penalized by the Court itself.

Taking the joint opinion on its own terms, we doubt that its distinction
between Roe, on the one hand, and Plessy and Lochner, on the other,
withstands analysis. The joint opinion acknowledges that the Court
improved its stature by overruling Plessy in Brown on a deeply divisive
issue. And our decision in West Coast Hotel,which overruled Adkins v.
Children's Hospitaland Lochner, was rendered at a time when Congress was
considering President Franklin Roosevelt's proposal to ``reorganize''
this Court and enable him to name six additional Justices in the event
that any Member of the Court over the age of 70 did not elect to retire.
It is difficult to imagine a situation in which the Court would face
more intense opposition to a prior ruling than it did at that time, and,
under the general principle proclaimed in the joint opinion, the Court
seemingly should have responded to this opposition by stubbornly
refusing to reexamine the Lochner rationale, lest it lose legitimacy by
appearing to ``overrule under fire.'' The joint opinion agrees that the
Court's stature would have been seriously damaged if in Brown and West
Coast Hotel it had dug in its heels and refused to apply normal
principles of stare decisis to the earlier decisions. But the opinion
contends that the Court was entitled to overrule Plessy and Lochner in
those cases, despite the existence of opposition to the original
decisions, only because both the Nation and the Court had learned new
lessons in the interim. This is at best a feebly supported, post hoc
rationalization for those decisions.

For example, the opinion asserts that the Court could justifiably
overrule its decision in Lochner only because the Depression had
convinced ``most people'' that constitutional protection of contractual
freedom contributed to an economy that failed to protect the welfare of
all. Surely the joint opinion does not mean to suggest that people saw
this Court's failure to uphold minimum wage statutes as the cause of the
Great Depression! In any event, the LochnerCourt did not base its rule
upon the policy judgment that an unregulated market was fundamental to a
stable economy; it simple believed, erroneously, that ``liberty'' under
the Due Process Clause protected the ``right to make a contract.''
Lochner v. New York. Nor is it the case that the people of this Nation
only discovered the dangers of extreme laissez-faire economics because
of the Depression. State laws regulating maximum hours and minimum wages
were in existence well before that time. A Utah statute of that sort
enacted in 1896 was involved in our decision in Holden v. Hardy, and
other States followed suit shortly afterwards, see, e. g., Muller v.
Oregon; Bunting v. Oregon. These statutes were indeed enacted because of
a belief on the part of their sponsors that ``freedom of contract'' did
not protect the welfare of workers, demonstrating that that belief
manifested itself more than a generation before the Great Depression.
Whether ``most people'' had come to share it in the hard times of the
1930's is, insofar as anything the joint opinion advances, entirely
speculative. The crucial failing at that time was not that workers were
not paid a fair wage, but that there was no work available at any wage.

When the Court finally recognized its error in West Coast Hotel, it did
not engage in the post hoc rationalization that the joint opinion
attributes to it today; it did not state that Lochner had been based on
an economic view that had fallen into disfavor, and that it therefore
should be overruled. Chief Justice Hughes in his opinion for the Court
simply recognized what Justice Holmes had previously recognized in his
Lochner dissent, that ``{[}t{]}he Constitution does not speak of freedom
of contract.'' West Coast Hotel Co.~v. Parrish; Lochnerv. New York
(Holmes, J., dissenting) (``{[}A{]} constitution is not intended to
embody a particular economic theory, whether of paternalism and the
organic relation of the citizen to the State or of laissez faire'').
Although the Court did acknowledge in the last paragraph of its opinion
the state of affairs during the then-current Depression, the theme of
the opinion is that the Court had been mistaken as a matter of
constitutional law when it embraced ``freedom of contract'' 32 years
previously.

The joint opinion also agrees that the Court acted properly in rejecting
the doctrine of ``separate but equal'' in Brown. In fact, the opinion
lauds Brown in comparing it to Roe. This is strange, in that under the
opinion's ``legitimacy'' principle the Court would seemingly have been
forced to adhere to its erroneous decision in Plessy because of its
``intensely divisive'' character. To us, adherence to Roe today under
the guise of ``legitimacy'' would seem to resemble more closely
adherence to Plessy on the same ground. Fortunately, the Court did not
choose that option in Brown, and instead frankly repudiated Plessy. The
joint opinion concludes that such repudiation was justified only because
of newly discovered evidence that segregation had the effect of treating
one race as inferior to another. But it can hardly be argued that this
was not urged upon those who decided Plessy, as Justice Harlan observed
in his dissent that the law at issue ``puts the brand of servitude and
degradation upon a large class of our fellow-citizens, our equals before
the law.'' Plessy v. Ferguson. It is clear that the same arguments made
before the Court in Brown were made in Plessy as well. The Court in
Brown simply recognized, as Justice Harlan had recognized beforehand,
that the Fourteenth Amendment does not permit racial segregation. The
rule of Brown is not tied to popular opinion about the evils of
segregation; it is a judgment that the Equal Protection Clause does not
permit racial segregation, no matter whether the public might come to
believe that it is beneficial. On that ground it stands, and on that
ground alone the Court was justified in properly concluding that the
Plessy Court had erred.

There is also a suggestion in the joint opinion that the propriety of
overruling a ``divisive'' decision depends in part on whether ``most
people'' would now agree that it should be overruled. Either the demise
of opposition or its progression to substantial popular agreement
apparently is required to allow the Court to reconsider a divisive
decision. How such agreement would be ascertained, short of a public
opinion poll, the joint opinion does not say. But surely even the
suggestion is totally at war with the idea of ``legitimacy'' in whose
name it is invoked. The Judicial Branch derives its legitimacy, not from
following public opinion, but from deciding by its best lights whether
legislative enactments of the popular branches of Government comport
with the Constitution. The doctrine of stare decisis is an adjunct of
this duty, and should be no more subject to the vagaries of public
opinion than is the basic judicial task.

There are other reasons why the joint opinion's discussion of legitimacy
is unconvincing as well. In assuming that the Court is perceived as
``surrender{[}ing{]} to political pressure'' when it overrules a
controversial decision, the joint opinion forgets that there are two
sides to any controversy. The joint opinion asserts that, in order to
protect its legitimacy, the Court must refrain from overruling a
controversial decision lest it be viewed as favoring those who oppose
the decision. But a decision to adhere to prior precedent is subject to
the same criticism, for in such a case one can easily argue that the
Court is responding to those who have demonstrated in favor of the
original decision. The decision in Roehas engendered large
demonstrations, including repeated marches on this Court and on
Congress, both in opposition to and in support of that opinion. A
decision either way on Roe can therefore be perceived as favoring one
group or the other. But this perceived dilemma arises only if one
assumes, as the joint opinion does, that the Court should make its
decisions with a view toward speculative public perceptions. If one
assumes instead, as the Court surely did in both Brown and West Coast
Hotel, that the Court's legitimacy is enhanced by faithful
interpretation of the Constitution irrespective of public opposition,
such self-engendered difficulties may be put to one side.

Roe is not this Court's only decision to generate conflict. Our
decisions in some recent capital cases, and in Bowers v. Hardwick, have
also engendered demonstrations in opposition. The joint opinion's
message to such protesters appears to be that they must cease their
activities in order to serve their cause, because their protests will
only cement in place a decision which by normal standards of stare
decisis should be reconsidered. Nearly a century ago, Justice David J.
Brewer of this Court, in an article discussing criticism of its
decisions, observed that ``many criticisms may be, like their authors,
devoid of good taste, but better all sorts of criticism than no
criticism at all.'' Justice Brewer on ``The Nation's Anchor,'' 57 Albany
L. J. 166, 169 (1898). This was good advice to the Court then, as it is
today. Strong and often misguided criticism of a decision should not
render the decision immune from reconsideration, lest a fetish for
legitimacy penalize freedom of expression.

The end result of the joint opinion's paeans of praise for legitimacy is
the enunciation of a brand new standard for evaluating state regulation
of a woman's right to abortion--- the ``undue burden'' standard. As
indicated above, Roe v. Wade adopted a ``fundamental right'' standard
under which state regulations could survive only if they met the
requirement of ``strict scrutiny.'' While we disagree with that
standard, it at least had a recognized basis in constitutional law at
the time Roe was decided. The same cannot be said for the ``undue
burden'' standard, which is created largely out of whole cloth by the
authors of the joint opinion. It is a standard which even today does not
command the support of a majority of this Court. And it will not, we
believe, result in the sort of ``simple limitation,'' easily applied,
which the joint opinion anticipates. In sum, it is a standard which is
not built to last.

In evaluating abortion regulations under that standard, judges will have
to decide whether they place a ``substantial obstacle'' in the path of a
woman seeking an abortion. In that this standard is based even more on a
judge's subjective determinations than was the trimester framework, the
standard will do nothing to prevent ``judges from roaming at large in
the constitutional field'' guided only by their personal views. Griswold
v. Connecticut (Harlan, J., concurring in judgment). Because the undue
burden standard is plucked from nowhere, the question of what is a
``substantial obstacle'' to abortion will undoubtedly engender a variety
of conflicting views. For example, in the very matter before us now, the
authors of the joint opinion would uphold Pennsylvania's 24-hour waiting
period, concluding that a ``particular burden'' on some women is not a
substantial obstacle. But the authors would at the same time strike down
Pennsylvania's spousal notice provision, after finding that in a ``large
fraction'' of cases the provision will be a substantial obstacle. And,
while the authors conclude that the informed consent provisions do not
constitute an ``undue burden,'' Justice Stevens would hold that they do.

Furthermore, while striking down the spousal notice regulation, the
joint opinion would uphold a parental consent restriction that certainly
places very substantial obstacles in the path of a minor's abortion
choice. The joint opinion is forthright in admitting that it draws this
distinction based on a policy judgment that parents will have the best
interests of their children at heart, while the same is not necessarily
true of husbands as to their wives. This may or may not be a correct
judgment, but it is quintessentially a legislative one. The ``undue
burden'' inquiry does not in any way supply the distinction between
parental consent and spousal consent which the joint opinion adopts.
Despite the efforts of the joint opinion, the undue burden standard
presents nothing more workable than the trimester framework which it
discards today. Under the guise of the Constitution, this Court will
still impart its own preferences on the States in the form of a complex
abortion code.

The sum of the joint opinion's labors in the name of stare decisis and
``legitimacy'' is this: Roe v. Wade stands as a sort of judicial
Potemkin Village, which may be pointed out to passers-by as a monument
to the importance of adhering to precedent. But behind the facade, an
entirely new method of analysis, without any roots in constitutional
law, is imported to decide the constitutionality of state laws
regulating abortion. Neither stare decisis nor ``legitimacy'' are truly
served by such an effort.

We have stated above our belief that the Constitution does not subject
state abortion regulations to heightened scrutiny. Accordingly, we think
that the correct analysis is that set forth by the plurality opinion in
Webster. A woman's interest in having an abortion is a form of liberty
protected by the Due Process Clause, but States may regulate abortion
procedures in ways rationally related to a legitimate state interest.
Williamson v. Lee Optical of Oklahoma, Inc.; cf.~Stanley v. Illinois
(1972). With this rule in mind, we examine each of the challenged
provisions.

Section 3205 of the Act imposes certain requirements related to the
informed consent of a woman seeking an abortion. 18 Pa. Cons. Stat. §
3205 (1990). Section 3205(a)(1) requires that the referring or
performing physician must inform a woman contemplating an abortion of
(i) the nature of the procedure and the risks and alternatives that a
reasonable patient would find material; (ii) the fetus' probable
gestational age; and (iii) the medical risks involved in carrying her
pregnancy to term. Section 3205(a)(2) requires a physician or a
nonphysician counselor to inform the woman that (i) the state health
department publishes free materials describing the fetus at different
stages and listing abortion alternatives; (ii) medical assistance
benefits may be available for prenatal, childbirth, and neonatal care;
and (iii) the child's father is liable for child support. The Act also
imposes a 24-hour waiting period between the time that the woman
receives the required information and the time that the physician is
allowed to perform the abortion. See Appendix to opinion of O'Connor,
Kennedy, and Souter, JJ.

This Court has held that it is certainly within the province of the
States to require a woman's voluntary and informed consent to an
abortion. See Thornburgh v. American College of Obstetricians and
Gynecologists. Here, Pennsylvania seeks to further its legitimate
interest in obtaining informed consent by ensuring that each woman ``is
aware not only of the reasons for having an abortion, but also of the
risks associated with an abortion and the availability of assistance
that might make the alternative of normal childbirth more attractive
than it might otherwise appear.'' (White, J., dissenting).

We conclude that this provision of the statute is rationally related to
the State's interest in assuring that a woman's consent to an abortion
be a fully informed decision.

Section 3205(a)(1) requires a physician to disclose certain information
about the abortion procedure and its risks and alternatives. This
requirement is certainly no large burden, as the Court of Appeals found
that ``the record shows that the clinics, without exception, insist on
providing this information to women before an abortion is performed.''
947 F. 2d. We are of the view that this information ``clearly is related
to maternal health and to the State's legitimate purpose in requiring
informed consent.'' Akron v. Akron Center for Reproductive Health, Inc..
An accurate description of the gestational age of the fetus and of the
risks involved in carrying a child to term helps to further both those
interests and the State's legitimate interest in unborn human life. See
-446, n.~37 (required disclosure of gestational age of the fetus
``certainly is not objectionable''). Although petitioners contend that
it is unreasonable for the State to require that a physician, as opposed
to a nonphysician counselor, disclose this information, we agree with
the Court of Appeals that a State ``may rationally decide that
physicians are better qualified than counselors to impart this
information and answer questions about the medical aspects of the
available alternatives.'' 947 F. 2d.

Section 3205(a)(2) compels the disclosure, by a physician or a
counselor, of information concerning the availability of paternal child
support and state-funded alternatives if the woman decides to proceed
with her pregnancy. Here again, the Court of Appeals observed that ``the
record indicates that most clinics already require that a counselor
consult in person with the woman about alternatives to abortion before
the abortion is performed.'' -705. And petitioners do not claim that the
information required to be disclosed by statute is in any way false or
inaccurate; indeed, the Court of Appeals found it to be ``relevant,
accurate, and noninflammatory.'' We conclude that this required
presentation of ``balanced information'' is rationally related to the
State's legitimate interest in ensuring that the woman's consent is
truly informed, Thornburgh v. American College of Obstetricians and
Gynecologists (O'Connor, J., dissenting), and in addition furthers the
State's interest in preserving unborn life. That the information might
create some uncertainty and persuade some women to forgo abortions does
not lead to the conclusion that the Constitution forbids the provision
of such information. Indeed, it only demonstrates that this information
might very well make a difference, and that it is therefore relevant to
a woman's informed choice. Cf. (White, J., dissenting) (``{[}T{]}he
ostensible objective of Roe v. Wade is not maximizing the number of
abortions, but maximizing choice''). We acknowledge that in Thornburgh
this Court struck down informed consent requirements similar to the ones
at issue here. See -764. It is clear, however, that while the detailed
framework of Roe led to the Court's invalidation of those informational
requirements, they ``would have been sustained under any traditional
standard of judicial review, . or for any other surgical procedure
except abortion.'' Webster v. Reproductive Health Services (plurality
opinion) (citing Thornburgh v. American College of Obstetricians and
Gynecologists (White, J., dissenting); (Burger, C. J., dissenting)). In
light of our rejection of Roe's ``fundamental right'' approach to this
subject, we do not regard Thornburgh as controlling.

For the same reason, we do not feel bound to follow this Court's
previous holding that a State's 24-hour mandatory waiting period is
unconstitutional. See Akron v. Akron Center for Reproductive Health,
Inc.. Petitioners are correct that such a provision will result in
delays for some women that might not otherwise exist, therefore placing
a burden on their liberty. But the provision in no way prohibits
abortions, and the informed consent and waiting period requirements do
not apply in the case of a medical emergency. See 18 Pa. Cons. Stat. §§
3205(a), (b) (1990). We are of the view that, in providing time for
reflection and reconsideration, the waiting period helps ensure that a
woman's decision to abort is a well-considered one, and reasonably
furthers the State's legitimate interest in maternal health and in the
unborn life of the fetus. It ``is surely a small cost to impose to
ensure that the woman's decision is well considered in light of its
certain and irreparable consequences on fetal life, and the possible
effects on her own.'' (O'Connor, J., dissenting).

In addition to providing her own informed consent, before an
unemancipated woman under the age of 18 may obtain an abortion she must
either furnish the consent of one of her parents, or must opt for the
judicial procedure that allows her to bypass the consent requirement.
Under the judicial bypass option, a minor can obtain an abortion if a
state court finds that she is capable of giving her informed consent and
has indeed given such consent, or determines that an abortion is in her
best interests. Records of these court proceedings are kept
confidential. The Act directs the state trial court to render a decision
within three days of the woman's application, and the entire procedure,
including appeal to Pennsylvania Superior Court, is to last no longer
than eight business days. The parental consent requirement does not
apply in the case of a medical emergency.

This provision is entirely consistent with this Court's previous
decisions involving parental consent requirements. See Planned
Parenthood Assn. of Kansas City, Mo., Inc.~v. Ashcroft (upholding
parental consent requirement with a similar judicial bypass option);
Akron v. Akron Center for Reproductive Health, Inc.~(approving of
parental consent statutes that include a judicial bypass option allowing
a pregnant minor to ``demonstrate that she is sufficiently mature to
make the abortion decision herself or that, despite her immaturity, an
abortion would be in her best interests''); Bellotti v. Baird.

We think it beyond dispute that a State ``has a strong and legitimate
interest in the welfare of its young citizens, whose immaturity,
inexperience, and lack of judgment may sometimes impair their ability to
exercise their rights wisely.'' Hodgson v. Minnesota (opinion of
Stevens, J.). A requirement of parental consent to abortion, like myriad
other restrictions placed upon minors in other contexts, is reasonably
designed to further this important and legitimate state interest. In our
view, it is entirely ``rational and fair for the State to conclude that,
in most instances, the family will strive to give a lonely or even
terrified minor advice that is both compassionate and mature.'' Ohio v.
Akron Center for Reproductive Health (opinion of Kennedy, J.); see also
Planned Parenthood of Central Mo. v. Danforth (Stewart, J., concurring)
(``There can be little doubt that the State furthers a constitutionally
permissible end by encouraging an unmarried pregnant minor to seek the
help and advice of her parents in making the very important decision
whether or not to bear a child''). We thus conclude that Pennsylvania's
parental consent requirement should be upheld.

Section 3209 of the Act contains the spousal notification provision. It
requires that, before a physician may perform an abortion on a married
woman, the woman must sign a statement indicating that she has notified
her husband of her planned abortion. A woman is not required to notify
her husband if (1) her husband is not the father, (2) her husband, after
diligent effort, cannot be located, (3) the pregnancy is the result of a
spousal sexual assault that has been reported to the authorities, or (4)
the woman has reason to believe that notifying her husband is likely to
result in the infliction of bodily injury upon her by him or by another
individual. In addition, a woman is exempted from the notification
requirement in the case of a medical emergency. 18 Pa. Cons. Stat. §
3209 (1990). See Appendix to opinion of O'Connor, Kennedy, and Souter,
JJ.

We first emphasize that Pennsylvania has not imposed a spousal consent
requirement of the type the Court struck down in Planned Parenthood of
Central Mo. v. Danforth. Missouri's spousal consent provision was
invalidated in that case because of the Court's view that it
unconstitutionally granted to the husband ``a veto power exercisable for
any reason whatsoever or for no reason at all.'' But the provision here
involves a much less intrusive requirement of spousal notification, not
consent. Such a law requiring only notice to the husband ``does not give
any third party the legal right to make the {[}woman's{]} decision for
her, or to prevent her from obtaining an abortion should she choose to
have one performed.'' Hodgson v. Minnesota (Kennedy, J., concurring in
judgment in part and dissenting in part); see H. L. v. Matheson, n.~17.
Danforth thus does not control our analysis. Petitioners contend that it
should, however; they argue that the real effect of such a notice
requirement is to give the power to husbands to veto a woman's abortion
choice. The District Court indeed found that the notification provision
created a risk that some woman who would otherwise have an abortion will
be prevented from having one. 947 F. 2d. For example, petitioners argue,
many notified husbands will prevent abortions through physical force,
psychological coercion, and other types of threats. But Pennsylvania has
incorporated exceptions in the notice provision in an attempt to deal
with these problems. For instance, a woman need not notify her husband
if the pregnancy is the result of a reported sexual assault, or if she
has reason to believe that she would suffer bodily injury as a result of
the notification. 18 Pa. Cons. Stat. § 3209(b) (1990). Furthermore,
because this is a facial challenge to the Act, it is insufficient for
petitioners to show that the notification provision ``might operate
unconstitutionally under some conceivable set of circumstances.'' United
States v. Salerno. Thus, it is not enough for petitioners to show that,
in some ``worst case'' circumstances, the notice provision will operate
as a grant of veto power to husbands. Ohio v. Akron Center for
Reproductive Health. Because they are making a facial challenge to the
provision, they must ``show that no set of circumstances exists under
which the {[}provision{]} would be valid.'' (internal quotation marks
omitted). This they have failed to do.

The question before us is therefore whether the spousal notification
requirement rationally furthers any legitimate state interests. We
conclude that it does. First, a husband's interests in procreation
within marriage and in the potential life of his unborn child are
certainly substantial ones. See Planned Parenthood of Central Mo. v.
Danforth (``We are not unaware of the deep and proper concern and
interest that a devoted and protective husband has in his wife's
pregnancy and in the growth and development of the fetus she is
carrying''); (White, J., concurring in part and dissenting in part);
Skinner v. Oklahoma ex rel. Williamson. The State itself has legitimate
interests both in protecting these interests of the father and in
protecting the potential life of the fetus, and the spousal notification
requirement is reasonably related to advancing those state interests. By
providing that a husband will usually know of his spouse's intent to
have an abortion, the provision makes it more likely that the husband
will participate in deciding the fate of his unborn child, a possibility
that might otherwise have been denied him. This participation might in
some cases result in a decision to proceed with the pregnancy. As Judge
Alito observed in his dissent below, ``{[}t{]}he Pennsylvania
legislature could have rationally believed that some married women are
initially inclined to obtain an abortion without their husbands'
knowledge because of perceived problems---such as economic constraints,
future plans, or the husbands' previously expressed opposition---that
may be obviated by discussion prior to the abortion.'' (opinion
concurring in part and dissenting in part).

The State also has a legitimate interest in promoting ``the integrity of
the marital relationship.'' 18 Pa. Cons. Stat. § 3209(a) (1990). This
Court has previously recognized ``the importance of the marital
relationship in our society.'' Planned Parenthood of Central Mo. v.
Danforth. In our view, the spousal notice requirement is a rational
attempt by the State to improve truthful communication between spouses
and encourage collaborative decision making, and thereby fosters marital
integrity. See Labine v. Vincent (``{[}T{]}he power to make rules to
establish, protect, and strengthen family life'' is committed to the
state legislatures). Petitioners argue that the notification requirement
does not further any such interest; they assert that the majority of
wives already notify their husbands of their abortion decisions, and the
remainder have excellent reasons for keeping their decisions a secret.
In the first case, they argue, the law is unnecessary, and in the second
case it will only serve to foster marital discord and threats of harm.
Thus, petitioners see the law as a totally irrational means of
furthering whatever legitimate interest the State might have. But, in
our view, it is unrealistic to assume that every husband-wife
relationship is either (1) so perfect that this type of truthful and
important communication will take place as a matter of course, or (2) so
imperfect that, upon notice, the husband will react selfishly,
violently, or contrary to the best interests of his wife. See Planned
Parenthood of Central Mo. v. Danforth (Stevens, J., concurring in part
and dissenting in part) (making a similar point in the context of a
parental consent statute). The spousal notice provision will admittedly
be unnecessary in some circumstances, and possibly harmful in others,
but ``the existence of particular cases in which a feature of a statute
performs no function (or is even counter productive) ordinarily does not
render the statute unconstitutional or even constitutionally suspect.''
Thornburgh v. American College of Obstetricians and Gynecologists
(White, J., dissenting). The Pennsylvania Legislature was in a position
to weigh the likely benefits of the provision against its likely adverse
effects, and presumably concluded, on balance, that the provision would
be beneficial. Whether this was a wise decision or not, we cannot say
that it was irrational. We therefore conclude that the spousal notice
provision comports with the Constitution. See Harris v. McRae (``It is
not the mission of this Court or any other to decide whether the balance
of competing interests . is wise social policy'').

The Act also imposes various reporting requirements. Section 3214(a)
requires that abortion facilities file a report on each abortion
performed. The reports do not include the identity of the women on whom
abortions are performed, but they do contain a variety of information
about the abortions. For example, each report must include the
identities of the performing and referring physicians, the gestational
age of the fetus at the time of abortion, and the basis for any medical
judgment that a medical emergency existed. See 18 Pa. Cons. Stat. §§
3214(a)(1), (5), (10) (1990). See Appendix to opinion of O'Connor,
Kennedy, and Souter, JJ., The District Court found that these reports
are kept completely confidential. 947 F. 2d. We further conclude that
these reporting requirements rationally further the State's legitimate
interests in advancing the state of medical knowledge concerning
maternal health and prenatal life, in gathering statistical information
with respect to patients, and in ensuring compliance with other
provisions of the Act.

Section 3207 of the Act requires each abortion facility to file a report
with its name and address, as well as the names and addresses of any
parent, subsidiary, or affiliated organizations. 18 Pa. Cons. Stat. §
3207(b) (1990). Section 3214(f) further requires each facility to file
quarterly reports stating the total number of abortions performed,
broken down by trimester. Both of these reports are available to the
public only if the facility received state funds within the preceding 12
months. See Appendix to opinion of O'Connor, Kennedy, and Souter, JJ.,
Petitioners do not challenge the requirement that facilities provide
this information. They contend, however, that the forced public
disclosure of the information given by facilities receiving public funds
serves no legitimate state interest. We disagree. Records relating to
the expenditure of public funds are generally available to the public
under Pennsylvania law. As the Court of Appeals observed, ``{[}w{]}hen a
state provides money to a private commercial enterprise, there is a
legitimate public interest in informing taxpayers who the funds are
benefiting and what services the funds are supporting.'' 947 F. 2d.
These reporting requirements rationally further this legitimate state
interest.

Finally, petitioners challenge the medical emergency exception provided
for by the Act. The existence of a medical emergency exempts compliance
with the Act's informed consent, parental consent, and spousal notice
requirements. See 18 Pa. Cons. Stat. §§ 3205(a), 3206(a), 3209(c)
(1990). The Act defines a ``medical emergency'' as

``{[}t{]}hat condition which, on the basis of the physician's good faith
clinical judgment, so complicates the medical condition of a pregnant
woman as to necessitate the immediate abortion of her pregnancy to avert
her death or for which a delay will create serious risk of substantial
and irreversible impairment of major bodily function.'' § 3203.

Petitioners argued before the District Court that the statutory
definition was inadequate because it did not cover three serious
conditions that pregnant women can suffer--- preeclampsia, inevitable
abortion, and prematurely ruptured membrane. The District Court agreed
with petitioners that the medical emergency exception was inadequate,
but the Court of Appeals reversed this holding. In construing the
medical emergency provision, the Court of Appeals first observed that
all three conditions do indeed present the risk of serious injury or
death when an abortion is not performed, and noted that the medical
profession's uniformly prescribed treatment for each of the three
conditions is an immediate abortion. See 947 F. 2d. Finding that
``{[}t{]}he Pennsylvania legislature did not choose the wording of its
medical emergency exception in a vacuum,'' the court read the exception
as intended ``to assure that compliance with its abortion regulations
would not in any way pose a significant threat to the life or health of
a woman.'' It thus concluded that the exception encompassed each of the
three dangerous conditions pointed to by petitioners.

We observe that Pennsylvania's present definition of medical emergency
is almost an exact copy of that State's definition at the time of this
Court's ruling in Thornburgh, one which the Court made reference to with
apparent approval. 476 U. S. (``It is clear that the Pennsylvania
Legislature knows how to provide a medical-emergency exception when it
chooses to do so'') We find that the interpretation of the Court of
Appeals in these cases is eminently reasonable, and that the provision
thus should be upheld. When a woman is faced with any condition that
poses a ``significant threat to {[}her{]} life or health,'' she is
exempted from the Act's consent and notice requirements and may proceed
immediately with her abortion.

For the reasons stated, we therefore would hold that each of the
challenged provisions of the Pennsylvania statute is consistent with the
Constitution. It bears emphasis that our conclusion in this regard does
not carry with it any necessary approval of these regulations. Our task
is, as always, to decide only whether the challenged provisions of a law
comport with the United States Constitution. If,as we believe,these
do,their wisdom as a matter of public policy is for the people of
Pennsylvania to decide.

Two years after Roe, the West German constitutional court, by contrast,
struck down a law liberalizing access to abortion on the grounds that
life developing within the womb is constitutionally protected. Judgment
of February 25, 1975, 39 BVerfGE 1 (translated in Jonas \& Gorby, West
German Abortion Decision: A Contrast to Roe v. Wade, 9 John Marshall J.
Prac. \& Proc. 605 (1976)). In 1988, the Canadian Supreme Court followed
reasoning similar to that of Roe in striking down a law that restricted
abortion. Morgentaler v. Queen, 1 S. C. R. 30, 44 D. L. R. 4th 385
(1988).

The joint opinion of Justices O'Connor, Kennedy, and Souter appears to
ignore this point in concluding that the spousal notice provision
imposes an undue burden on the abortion decision. In most instances the
notification requirement operates without difficulty. As the District
Court found, the vast majority of wives seeking abortions notify and
consult with their husbands, and thus suffer no burden as a result of
the provision. 744 F. Supp. 1323, 1360 (ED Pa. 1990). In other instances
where a woman does not want to notify her husband, the Act provides
exceptions. For example, notification is not required if the husband is
not the father, if the pregnancy is the result of a reported spousal
sexual assault, or if the woman fears bodily injury as a result of
notifying her husband. Thus, in these instances as well, the
notification provision imposes no obstacle to the abortion decision. The
joint opinion puts to one side these situations where the regulation
imposes no obstacle at all,and instead focuses on the group of married
women who would not otherwise notify their husbands and who do not
qualify for one of the exceptions. Having narrowed the focus, the joint
opinion concludes that in a ``large fraction'' of those cases, the
notification provision operates as a substantial obstacle, and that the
provision is therefore invalid. There are certainly instances where a
woman would prefer not to notify her husband, and yet does not qualify
for an exception. For example, there are the situations of battered
women who fear psychological abuse or injury to their children as a
result of notification; because in these situations the women do not
fear bodily injury, they do not qualify for an exception. And there are
situations where a woman has become pregnant as a result of an
unreported spousal sexual assault; when such an assault is unreported,
no exception is available. But, as the District Court found, there are
also instances where the woman prefers not to notify her husband for a
variety of other reasons. See 744 F. Supp.. For example, a woman might
desire to obtain an abortion without her husband's knowledge because of
perceived economic constraints or her husband's previously expressed
opposition to abortion. The joint opinion concentrates on the situations
involving battered women and unreported spousal assault, and assumes,
without any support in the record, that these instances constitute a
``large fraction'' of those cases in which women prefer not to notify
their husbands (and do not qualify for an exception). This assumption is
not based on any hard evidence, however. And were it helpful to an
attempt to reach a desired result, one could just as easily assume that
the battered women situations form 100 percent of the cases where women
desire not to notify, or that they constitute only 20 percent of those
cases. But reliance on such speculation is the necessary result of
adopting the undue burden standard.

The definition in use at that time provided as follows: ```Medical
emergency.'That condition which, on the basis of the physician's best
clinical judgment, so complicates a pregnancy as to necessitate the
immediate abortion of same to avert the death of the mother or for which
a 24-hour delay will create grave peril of immediate and irreversible
loss of major bodily function.'' Pa. Stat.Ann., Tit § 3203(Purdon 1983).

\textbf{Justice Scalia, with whom The Chief Justice, Justice White, and
Justice Thomas join, concurring in the judgment in part and dissenting
in part.}

My views on this matter are unchanged from those I set forth in my
separate opinions in Webster v. Reproductive Health Services (opinion
concurring in part and concurring in judgment), and Ohio v. Akron Center
for Reproductive Health (Akron II) (concurring opinion). The States may,
if they wish, permit abortion on demand, but the Constitution does not
require them to do so. The permissibility of abortion, and the
limitations upon it, are to be resolved like most important questions in
our democracy: by citizens trying to persuade one another and then
voting. As the Court acknowledges, ``where reasonable people disagree
the government can adopt one position or the other.'' The Court is
correct in adding the qualification that this ``assumes a state of
affairs in which the choice does not intrude upon a protected liberty,''
---but the crucial part of that qualification is the penultimate word. A
State's choice between two positions on which reasonable people can
disagree is constitutional even when (as is often the case) it intrudes
upon a ``liberty'' in the absolute sense. Laws against bigamy, for
example---with which entire societies of reasonable people
disagree---intrude upon men and women's liberty to marry and live with
one another. But bigamy happens not to be a liberty specially
``protected'' by the Constitution.

That is, quite simply, the issue in these cases: not whether the power
of a woman to abort her unborn child is a ``liberty'' in the absolute
sense; or even whether it is a liberty of great importance to many
women. Of course it is both. The issue is whether it is a liberty
protected by the Constitution of the United States. I am sure it is not.
I reach that conclusion not because of anything so exalted as my views
concerning the ``concept of existence, of meaning, of the universe, and
of the mystery of human life.'' Rather, I reach it for the same reason I
reach the conclusion that bigamy is not constitutionally
protected---because of two simple facts: (1) the Constitution says
absolutely nothing about it, and (2) the longstanding traditions of
American society have permitted it to be legally proscribed Akron II
(Scalia, J., concurring).

The Court destroys the proposition, evidently meant to represent my
position, that ``liberty'' includes ``only those practices, defined at
the most specific level, that were protected against government
interference by other rules of law when the Fourteenth Amendment was
ratified,'' (citing Michael H. v. Gerald D.,491 U. S. 110, 127, n.~6
(1989) (opinion of Scalia, J.)). That is not, however, what Michael H.
says; it merely observes that, in defining ``liberty,'' we may not
disregard a specific, ``relevant tradition protecting, or denying
protection to, the asserted right,'' But the Court does not wish to be
fettered by any such limitations on its preferences. The Court's
statement that it is ``tempting'' to acknowledge the authoritativeness
of tradition in order to ``cur{[}b{]} the discretion of federal
judges,'' is of course rhetoric rather than reality; no government
official is ``tempted'' to place restraints upon his own freedom of
action, which is why Lord Act on did not say ``Power tends to purify.''
The Court's temptation is in the quite opposite and more natural
direction---towards systematically eliminating checks upon its own
power; and it succumbs.

Beyond that brief summary of the essence of my position, I will not
swell the United States Reports with repetition of what I have said
before; and applying the rational basis test, I would uphold the
Pennsylvania statute in its entirety. I must, however, respond to a few
of the more outrageous arguments in today's opinion, which it is beyond
human nature to leave unanswered. I shall discuss each of them under a
quotation from the Court's opinion to which they pertain.

``The inescapable fact is that adjudication of substantive due process
claims may call upon the Court in interpreting the Constitution to
exercise that same capacity which by tradition courts always have
exercised: reasoned judgment.'' Assuming that the question before us is
to be resolved at such a level of philosophical abstraction, in such
isolation from the traditions of American society, as by simply applying
``reasoned judgment,'' I do not see how that could possibly have
produced the answer the Court arrived at in Roe v. Wade. Today's opinion
describes the methodology of Roe, quite accurately, as weighing against
the woman's interest the State's ```important and legitimate interest in
protecting the potentiality of human life.''' (quoting Roe). But
``reasoned judgment'' does not begin by begging the question, as Roe and
subsequent cases unquestionably did by assuming that what the State is
protecting is the mere ``potentiality of human life.'' See, e. g., Roe;
Planned Parenthood of Central Mo. v. Danforth; Colautti v. Franklin;
Akron v. Akron Center for Reproductive Health, Inc.~(Akron I); Planned
Parenthood Assn. of Kansas City, Mo., Inc.~v. Ashcroft. The whole
argument of abortion opponents is that what the Court calls the fetus
and what others call the unborn child is a human life. Thus, whatever
answer Roe came up with after conducting its ``balancing'' is bound to
be wrong, unless it is correct that the human fetus is in some critical
sense merely potentially human. There is of course no way to determine
that as a legal matter; it is in fact a value judgment. Some societies
have considered newborn children not yet human, or the incompetent
elderly no longer so.

The authors of the joint opinion, of course, do not squarely contend
that Roe v. Wade was a correct application of ``reasoned judgment'';
merely that it must be followed, because of stare decisis. 871. But in
their exhaustive discussion of all the factors that go into the
determination of when stare decisis should be observed and when
disregarded, they never mention ``how wrong was the decision on its
face?'' Surely, if ``{[}t{]}he Court's power lies .. in its legitimacy,
a product of substance and perception,'' the ``substance'' part of the
equation demands that plain error be acknowledged and eliminated. Roe
was plainly wrong---even on the Court's methodology of ``reasoned
judgment,'' and even more so (of course) if the proper criteria of text
and tradition are applied.

The emptiness of the ``reasoned judgment'' that produced Roe is
displayed in plain view by the fact that, after more than 19 years of
effort by some of the brightest (and most determined) legal minds in the
country, after more than 10 cases upholding abortion rights in this
Court, and after dozens upon dozens of amicus briefs submitted in these
and other cases, the best the Court can do to explain how it is that the
word ``liberty'' must be thought to include the right to destroy human
fetuses is to rattle off a collection of adjectives that simply decorate
a value judgment and conceal a political choice. The right to abort, we
are told, inheres in ``liberty'' because it is among ``a person's most
basic decisions,'' ; it involves a ``most intimate and personal
choic{[}e{]},'' ; it is ``central to personal dignity and autonomy,'' ;
it ``originate{[}s{]} within the zone of conscience and belief,'' ; it
is ``too intimate and personal'' for state interference, ; it reflects
``intimate views'' of a ``deep, personal character,'' ; it involves
``intimate relationships'' and notions of ``personal autonomy and bodily
integrity,'' ; and it concerns a particularly ```important
decisio{[}n{]},''' (citation omitted) But it is obvious to anyone
applying ``reasoned judgment'' that the same adjectives can be applied
to many forms of conduct that this Court (including one of the Justices
in today's majority, see Bowers v. Hardwick) has held are not entitled
to constitutional protection---because, like abortion, they are forms of
conduct that have long been criminalized in American society. Those
adjectives might be applied, for example, to homosexual sodomy,
polygamy, adult incest, and suicide, all of which are equally
``intimate'' and ``deep{[}ly{]} personal'' decisions involving
``personal autonomy and bodily integrity,'' and all of which can
constitutionally be proscribed because it is our unquestionable
constitutional tradition that they are proscribable. It is not reasoned
judgment that supports the Court's decision; only personal predilection.
Justice Curtis's warning is as timely today as it was 135 years ago:

``{[}W{]}hen a strict interpretation of the Constitution, according to
the fixed rules which govern the interpretation of laws, is abandoned,
and the theoretical opinions of individuals are allowed to control its
meaning, we have no longer a Constitution; we are under the government
of individual men, who for the time being have power to declare what the
Constitution is, according to their own views of what it ought to
mean.'' Dred Scott v. Sandford, 19 How. 393, 621 (1857) (dissenting
opinion).

``Liberty finds no refuge in a jurisprudence of doubt.'' One might have
feared to encounter this august and sonorous phrase in an opinion
defending the real Roe v. Wade, rather than the revised version
fabricated today by the authors of the joint opinion. The shortcomings
of Roe did not include lack of clarity: Virtually all regulation of
abortion before the third trimester was invalid. But to come across this
phrase in the joint opinion---which calls upon federal district judges
to apply an ``undue burden'' standard as doubtful in application as it
is unprincipled in origin---is really more than one should have to bear.

The joint opinion frankly concedes that the amorphous concept of ``undue
burden'' has been inconsistently applied by the Members of this Court in
the few brief years since that ``test'' was first explicitly propounded
by Justice O'Connor in her dissent in Akron I. See Because the three
Justices now wish to ``set forth a standard of general application,''
the joint opinion announces that ``it is important to clarify what is
meant by an undue burden.'' I certainly agree with that, but I do not
agree that the joint opinion succeeds in the announced endeavor. To the
contrary, its efforts at clarification make clear only that the standard
is inherently manipulable and will prove hopelessly unworkable in
practice.

The joint opinion explains that a state regulation imposes an ``undue
burden'' if it ``has the purpose or effect of placing a substantial
obstacle in the path of a woman seeking an abortion of a nonviable
fetus.'' ; see also An obstacle is ``substantial,'' we are told, if it
is ``calculated{[},{]} {[}not{]} to inform the woman's free choice,
{[}but to{]} hinder it.'' This latter statement cannot possibly mean
what it says. Any regulation of abortion that is intended to advance
what the joint opinion concedes is the State's ``substantial'' interest
in protecting unborn life will be ``calculated {[}to{]} hinder'' a
decision to have an abortion. It thus seems more accurate to say that
the joint opinion would uphold abortion regulations only if they do not
unduly hinder the woman's decision. That, of course, brings us right
back to square one: Defining an ``undue burden'' as an ``undue
hindrance'' (or a ``substantial obstacle'') hardly ``clarifies'' the
test. Consciously or not, the joint opinion's verbal shell game will
conceal raw judicial policy choices concerning what is ``appropriate''
abortion legislation.

The ultimately standard less nature of the ``undue burden'' inquiry is a
reflection of the underlying fact that the concept has no principled or
coherent legal basis. As The Chief Justice points out, Roe's
strict-scrutiny standard ``at least had a recognized basis in
constitutional law at the time Roe was decided,'' while ``{[}t{]}he same
cannot be said for the'undue burden' standard, which is created largely
out of whole cloth by the authors of the joint opinion,'' The joint
opinion is flatly wrong in asserting that ``our jurisprudence relating
to all liberties save perhaps abortion has recognized'' the
permissibility of laws that do not impose an ``undue burden.'' It argues
that the abortion right is similar to other rights in that a law ``not
designed to strike at the right itself, {[}but which{]} has the
incidental effect of making it more difficult or more expensive to
{[}exercise the right,{]}'' is not invalid. I agree, indeed I have
forcefully urged, that a law of general applicability which places only
an incidental burden on a fundamental right does not infringe that
right, see R. A. V. v. St.~Paul (1992); Employment Div., Dept. of Human
Resources of Ore. v. Smith (1990), but that principle does not establish
the quite different (and quite dangerous) proposition that a law which
directly regulates a fundamental right will not be found to violate the
Constitution unless it imposes an ``undue burden.'' It is that, of
course, which is at issue here: Pennsylvania has consciously and
directly regulated conduct that our cases have held is constitutionally
protected. The appropriate analogy, therefore, is that of a state law
requiring purchasers of religious books to endure a 24-hour waiting
period, or to pay a nominal additional tax of 1¢. The joint opinion
cannot possibly be correct in suggesting that we would uphold such
legislation on the ground that it does not impose a ``substantial
obstacle'' to the exercise of First Amendment rights. The ``undue
burden'' standard is not at all the generally applicable principle the
joint opinion pretends it to be; rather, it is a unique concept created
specially for these cases, to preserve some judicial foothold in this
ill-gotten territory. In claiming otherwise, the three Justices show
their willingness to place all constitutional rights at risk in an
effort to preserve what they deem the ``central holding in Roe.''

The rootless nature of the ``undue burden'' standard, a phrase plucked
out of context from our earlier abortion decisions, see n.~3is further
reflected in the fact that the joint opinion finds it necessary
expressly to repudiate the more narrow formulations used in Justice
O'Connor's earlier opinions. Those opinions stated that a statute
imposes an ``undue burden'' if it imposes ``absolute obstacles or severe
limitations on the abortion decision,'' Akron I (dissenting opinion)
(emphasis added). Those strong adjectives are conspicuously missing from
the joint opinion, whose authors have for some unexplained reason now
determined that a burden is ``undue'' if it merely imposes a
``substantial'' obstacle to abortion decisions. Justice O'Connor has
also abandoned (again without explanation) the view she expressed in
Planned Parenthood Assn. of Kansas City, Mo., Inc.~v. Ashcroft
(dissenting opinion), that a medical regulation which imposes an ``undue
burden'' could nevertheless be upheld if it ``reasonably relate{[}s{]}
to the preservation and protection of maternal health,'' (citation and
internal quotation marks omitted). In today's version, even health
measures will be upheld only ``if they do not constitute an undue
burden,'' (emphasis added). Gone too is Justice O'Connor's statement
that ``the State possesses compelling interests in the protection of
potential human life throughout pregnancy,'' Akron I(dissenting
opinion); instead, the State's interest in unborn human life is
stealthily downgraded to a merely ``substantial'' or ``profound''
interest, (That had to be done, of course, since designating the
interest as ``compelling'' throughout pregnancy would have been, shall
we say, a ``substantial obstacle'' to the joint opinion's determined
effort to reaffirm what it views as the ``central holding'' of Roe. See
Akron I, n.~1.) And ``viability'' is no longer the ``arbitrary''
dividing line previously decried by Justice O'Connor in Akron I, ; the
Court now announces that ``the attainment of viability may continue to
serve as the critical fact,'' It is difficult to maintain the illusion
that we are interpreting a Constitution rather than inventing one, when
we amend its provisions so breezily.

Because the portion of the joint opinion adopting and describing the
undue burden test provides no more useful guidance than the empty
phrases discussed above, one must turn to the 23 pages applying that
standard to the present facts for further guidance. In evaluating
Pennsylvania's abortion law, the joint opinion relies extensively on the
factual findings of the District Court, and repeatedly qualifies its
conclusions by noting that they are contingent upon the record developed
in these cases. Thus, the joint opinion would uphold the 24-hour waiting
period contained in the Pennsylvania statute's informed consent
provision, because ``the record evidence shows that in the vast majority
of cases, a 24-hour delay does not create any appreciable health risk,''
The three Justices therefore conclude that ``on the record before us, we
are not convinced that the 24-hour waiting period constitutes an undue
burden.'' The requirement that a doctor provide the information
pertinent to informed consent would also be upheld because ``there is no
evidence on this record that {[}this requirement{]} would amount in
practical terms to a substantial obstacle to a woman seeking an
abortion.'' Similarly, the joint opinion would uphold the reporting
requirements of the Act, §§ 3207, 3214, because ``there is no showing on
the record before us'' that these requirements constitute a
``substantial obstacle'' to abortion decisions. But at the same time the
opinion pointedly observes that these reporting requirements may
increase the costs of abortions and that ``at some point {[}that fact{]}
could become a substantial obstacle.'' Most significantly, the joint
opinion's conclusion that the spousal notice requirement of the Act, see
§ 3209, imposes an ``undue burden'' is based in large measure on the
District Court's ``detailed findings of fact,'' which the joint opinion
sets out at great length.

I do not, of course, have any objection to the notion that, in applying
legal principles, one should rely only upon the facts that are contained
in the record or that are properly subject to judicial notice But what
is remarkable about the joint opinion's fact-intensive analysis is that
it does not result in any measurable clarification of the ``undue
burden'' standard. Rather, the approach of the joint opinion is, for the
most part, simply to highlight certain facts in the record that
apparently strike the three Justices as particularly significant in
establishing (or refuting) the existence of an undue burden; after
describing these facts, the opinion then simply announces that the
provision either does or does not impose a ``substantial obstacle'' or
an ``undue burden.'' See, e. g., 893, 901. We do not know whether the
same conclusions could have been reached on a different record, or in
what respects the record would have had to differ before an opposite
conclusion would have been appropriate. The inherently standard less
nature of this inquiry invites the district judge to give effect to his
personal preferences about abortion. By finding and relying upon the
right facts, he can invalidate, it would seem, almost any abortion
restriction that strikes him as ``undue''---subject, of course, to the
possibility of being reversed by a court of appeals or Supreme Court
that is as unconstrained in reviewing his decision as he was in making
it.

To the extent I can discern any meaningful content in the ``undue
burden'' standard as applied in the joint opinion, it appears to be that
a State may not regulate abortion in such a way as to reduce
significantly its incidence. The joint opinion repeatedly emphasizes
that an important factor in the ``undue burden'' analysis is whether the
regulation ``prevent{[}s{]} a significant number of women from obtaining
an abortion,'' ; whether a ``significant number of women . are likely to
be deterred from procuring an abortion,'' ; and whether the regulation
often ``deters'' women from seeking abortions, We are not told, however,
what forms of ``deterrence'' are impermissible or what degree of success
in deterrence is too much to be tolerated. If, for example, a State
required a woman to read a pamphlet describing, with illustrations, the
facts of fetal development before she could obtain an abortion, the
effect of such legislation might be to ``deter'' a ``significant number
of women'' from procuring abortions, thereby seemingly allowing a
district judge to invalidate it as an undue burden. Thus, despite
flowery rhetoric about the State's ``substantial'' and ``profound''
interest in ``potential human life,'' and criticism of Roe for
undervaluing that interest, the joint opinion permits the State to
pursue that interest only so long as it is not too successful. As
Justice Blackmun recognizes (with evident hope), the ``undue burden''
standard may ultimately require the invalidation of each provision
upheld today if it can be shown, on a better record, that the State is
too effectively ``express{[}ing{]} a preference for childbirth over
abortion,'' Reason finds no refuge in this jurisprudence of confusion.

``While we appreciate the weight of the arguments . that Roe should be
overruled, the reservations any of us may have in reaffirming the
central holding of Roe are outweighed by the explication of individual
liberty we have given combined with the force of stare decisis.''

The Court's reliance upon stare decisis can best be described as
contrived. It insists upon the necessity of adhering not to all of Roe,
but only to what it calls the ``central holding.'' It seems to me that
stare decisis ought to be applied even to the doctrine of stare decisis,
and I confess never to have heard of this new,
keep-what-you-want-and-throwaway-the-rest version. I wonder whether, as
applied to Marbury v. Madison, 1 Cranch 137 (1803), for example, the new
version of stare decisis would be satisfied if we allowed courts to
review the constitutionality of only those statutes that (like the one
in Marbury) pertain to the jurisdiction of the courts.

I am certainly not in a good position to dispute that the Court has
saved the ``central holding'' of Roe, since to do that effectively I
would have to know what the Court has saved, which in turn would require
me to understand (as I do not) what the ``undue burden'' test means. I
must confess, however, that I have always thought, and I think a lot of
other people have always thought, that the arbitrary trimester
framework, which the Court today discards, was quite as central to Roeas
the arbitrary viability test, which the Court today retains. It seems
particularly ungrateful to carve the trimester framework out of the core
of Roe, since its very rigidity (in sharp contrast to the utter
indeterminability of the ``undue burden'' test) is probably the only
reason the Court is able to say, in urging stare decisis, that Roe``has
in no sense proven'unworkable,''' I suppose the Court is entitled to
call a ``central holding'' whatever it wants to call a ``central
holding''---which is, come to think of it, perhaps one of the
difficulties with this modified version of stare decisis. I thought I
might note, however, that the following portions of Roe have not been
saved:

\begin{itemize}
\item
  Under Roe, requiring that a woman seeking an abortion be provided
  truthful information about abortion before giving informed written
  consent is unconstitutional, if the information is designed to
  influence her choice. Thornburgh,476 U. S.; Akron I. Under the joint
  opinion's ``undue burden'' regime (as applied today, at least) such a
  requirement is constitutional.
\item
  Under Roe, requiring that information be provided by a doctor, rather
  than by nonphysician counselors, is unconstitutional. Akron I. Under
  the ``undue burden'' regime (as applied today, at least) it is not.
\item
  Under Roe, requiring a 24-hour waiting period between the time the
  woman gives her informed consent and the time of the abortion is
  unconstitutional. Akron I. Under the ``undue burden'' regime (as
  applied today, at least) it is not.
\item
  Under Roe, requiring detailed reports that include demographic data
  about each woman who seeks an abortion and various information about
  each abortion is unconstitutional. Thornburgh. Under the ``undue
  burden'' regime (as applied today, at least) it generally is not.
\end{itemize}

``Where, in the performance of its judicial duties, the Court decides a
case in such a way as to resolve the sort of intensely divisive
controversy reflected in Roe . its decision has a dimension that the
resolution of the normal case does not carry. It is the dimension
present whenever the Court's interpretation of the Constitution calls
the contending sides of a national controversy to end their national
division by accepting a common mandate rooted in the Constitution.''

The Court's description of the place of Roe in the social history of the
United States is unrecognizable. Not only did Roe not, as the Court
suggests, resolve the deeply divisive issue of abortion; it did more
than anything else to nourish it, by elevating it to the national level
where it is infinitely more difficult to resolve. National politics were
not plagued by abortion protests, national abortion lobbying, or
abortion marches on Congress before Roe v. Wade was decided. Profound
disagreement existed among our citizens over the issue---as it does over
other issues, such as the death penalty---but that disagreement was
being worked out at the state level. As with many other issues, the
division of sentiment within each State was not as closely balanced as
it was among the population of the Nation as a whole, meaning not only
that more people would be satisfied with the results of state-by-state
resolution, but also that those results would be more stable. Pre-Roe,
moreover, political compromise was possible.

Roe's mandate for abortion on demand destroyed the compromises of the
past, rendered compromise impossible for the future, and required the
entire issue to be resolved uniformly, at the national level. At the
same time, Roe created a vast new class of abortion consumers and
abortion proponents by eliminating the moral opprobrium that had
attached to the act. (``If the Constitution guarantees abortion, how can
it be bad?''---not an accurate line of thought, but a natural one.) Many
favor all of those developments, and it is not for me to say that they
are wrong. But to portray Roe as the states manlike ``settlement'' of a
divisive issue, a jurisprudential Peace of Westphalia that is worth
preserving, is nothing less than Orwellian. Roe fanned into life an
issue that has inflamed our national politics in general, and has
obscured with its smoke the selection of Justices to this Court in
particular, ever since. And by keeping us in the abortion-umpiring
business, it is the perpetuation of that disruption, rather than of any
Pax Roeana, that the Court's new majority decrees.

``{[}T{]}o overrule under fire . would subvert the Court's legitimacy .
.''. . To all those who will be . tested by follow-ing, the Court
implicitly undertakes to remain stead- fast . . The promise of
constancy, once given, binds its maker for as long as the power to stand
by the decision survives and . the commitment {[}is not{]} obsolete. .

``{[}The American people's{]} belief in themselves as . a people {[}who
aspire to live according to the rule of law{]} is not readily separable
from their under- standing of the Court invested with the authority to
decide their constitutional cases and speak before all others for their
constitutional ideals. If the Court's legitimacy should be undermined,
then, so would the country be in its very ability to see itself through
its constitutional ideals.'' The Imperial Judiciary lives. It is
instructive to compare this Nietzschean vision of us unelected,
life-tenured judges--- leading a Volk who will be ``tested by
following,'' and whose very ``belief in themselves'' is mystically bound
up in their ``understanding'' of a Court that ``speak{[}s{]} before all
others for their constitutional ideals''---with the somewhat more modest
role envisioned for these lawyers by the Founders.

``The judiciary . has . no direction either of the strength or of the
wealth of the society, and can take no active resolution whatever. It
may truly be said to have neither Force nor Will, but merely judgment .
.''

The Federalist No.~78, pp.~393-394 (G. Wills ed.~1982). Or, again, to
compare this ecstasy of a Supreme Court in which there is, especially on
controversial matters, no shadow of change or hint of alteration
(``There is a limit to the amount of error that can plausibly be imputed
to prior Courts,'' ), with the more democratic views of a more humble
man:

``{[}T{]}he candid citizen must confess that if the policy of the
Government upon vital questions affecting the whole people is to be
irrevocably fixed by decisions of the Supreme Court, . the people will
have ceased to be their own rulers, having to that extent practically
resigned their Government into the hands of that eminent tribunal.'' A.
Lincoln, First Inaugural Address (Mar.~4, 1861), reprinted in Inaugural
Addresses of the Presidents of the United States, S. Doc. No.~101-10,
p.~139 (1989).

It is particularly difficult, in the circumstances of the present
decision, to sit still for the Court's lengthy lecture upon the virtues
of ``constancy,'' of ``remain{[}ing{]} steadfast,'' , and adhering to
``principle,'' ante, passim. Among the five Justices who purportedly
adhere to Roe, at most three agree upon the principle that constitutes
adherence (the joint opinion's ``undue burden'' standard)---and that
principle is inconsistent with Roe. See 410 U. S. To make matters worse,
two of the three, in order thus to remain steadfast, had to abandon
previously stated positions. See n.~4; see. It is beyond me how the
Court expects these accommodations to be accepted ``as grounded truly in
principle, not as compromises with social and political pressures
having, as such, no bearing on the principled choices that the Court is
obliged to make.'' The only principle the Court ``adheres'' to, it seems
to me, is the principle that the Court must be seen as standing by Roe.
That is not a principle of law (which is what I thought the Court was
talking about), but a principle of Realpolitik---and a wrong one at
that.

I cannot agree with, indeed I am appalled by, the Court's suggestion
that the decision whether to stand by an erroneous constitutional
decision must be strongly influenced--- against overruling, no less---by
the substantial and continuing public opposition the decision has
generated. The Court's judgment that any other course would ``subvert
the Court's legitimacy'' must be another consequence of reading the
error-filled history book that described the deeply divided country
brought together by Roe. In my history book, the Court was covered with
dishonor and deprived of legitimacy by Dred Scott v. Sandford, 19 How.
393 (1857), an erroneous (and widely opposed) opinion that it did not
abandon, rather than by West Coast Hotel Co.~v. Parrish, which produced
the famous ``switch in time'' from the Court's erroneous (and widely
opposed) constitutional opposition to the social measures of the New
Deal. (Both Dred Scott and one line of the cases resisting the New Deal
rested upon the concept of ``substantive due process'' that the Court
praises and employs today. Indeed, Dred Scott was ``very possibly the
first application of substantive due process in the Supreme Court, the
original precedent for Lochner v. New York and Roe v. Wade.'' D. Currie,
The Constitution in the Supreme Court 271 (1985) (footnotes omitted).)

But whether it would ``subvert the Court's legitimacy'' or not, the
notion that we would decide a case differently from the way we otherwise
would have in order to show that we can stand firm against public
disapproval is frightening. It is a bad enough idea, even in the head of
someone like me, who believes that the text of the Constitution, and our
traditions, say what they say and there is no fiddling with them. But
when it is in the mind of a Court that believes the Constitution has an
evolving meaning, see ; that the Ninth Amendment's reference to
``othe{[}r{]}'' rights is not a disclaimer, but a charter for action, ;
and that the function of this Court is to ``speak before all others for
{[}the people's{]} constitutional ideals'' unrestrained by meaningful
text or tradition---then the notion that the Court must adhere to a
decision for as long as the decision faces ``great opposition'' and the
Court is ``under fire'' acquires a character of almost czarist
arrogance. We are offended by these marchers who descend upon us, every
year on the anniversary of Roe, to protest our saying that the
Constitution requires what our society has never thought the
Constitution requires. These people who refuse to be ``tested by
following'' must be taught a lesson. We have no Cossacks, but at least
we can stubbornly refuse to abandon an erroneous opinion that we might
otherwise change---to show how little they intimidate us.

Of course, as The Chief Justice points out, we have been subjected to
what the Court calls ```political pressure''' by both sides of this
issue. Maybe today's decision not to overrule Roe will be seen as
buckling to pressure from that direction. Instead of engaging in the
hopeless task of predicting public perception---a job not for lawyers
but for political campaign managers---the Justices should do what is
legally right by asking two questions: (1) Was Roe correctly decided?
(2) Has Roe succeeded in producing a settled body of law? If the answer
to both questions is no, Roe should undoubtedly be overruled.

In truth, I am as distressed as the Court is---and expressed my distress
several years ago, see Webster---about the ``political pressure''
directed to the Court: the marches, the mail, the protests aimed at
inducing us to change our opinions. How upsetting it is,that so many of
our citizens (good people, not lawless ones, on both sides of this
abortion issue, and on various sides of other issues as well) think that
we Justices should properly take into account their views, as though we
were engaged not in ascertaining an objective law but in determining
some kind of social consensus. The Court would profit, I think, from
giving less attention to the fact of this distressing phenomenon, and
more attention to the cause of it. That cause permeates today's opinion:
a new mode of constitutional adjudication that relies not upon text and
traditional practice to determine the law, but upon what the Court calls
``reasoned judgment,'' which turns out to be nothing but philosophical
predilection and moral intuition. All manner of ``liberties,'' the Court
tells us, inhere in the Constitution and are enforceable by this
Court---not just those mentioned in the text or established in the
traditions of our society. Why even the Ninth Amendment---which says
only that ``{[}t{]}he enumeration in the Constitution, of certain
rights, shall not be construed to deny or disparage others retained by
the people''---is, despite our contrary understanding for almost 200
years, a literally boundless source of additional, unnamed, unhinted-at
``rights,'' definable and enforceable by us, through ``reasoned
judgment.''

What makes all this relevant to the bothersome application of
``political pressure'' against the Court are the twin facts that the
American people love democracy and the American people are not fools. As
long as this Court thought (and the people thought) that we Justices
were doing essentially lawyers' work up here---reading text and
discerning our society's traditional understanding of that text---the
public pretty much left us alone. Texts and traditions are facts to
study, not convictions to demonstrate about. But if in reality our
process of constitutional adjudication consists primarily of making
value judgments; if we can ignore a long and clear tradition clarifying
an ambiguous text, as we did, for example, five days ago in declaring
unconstitutional invocations and benedictions at public high school
graduation ceremonies, Lee v. Weisman; if, as I say, our pronouncement
of constitutional law rests primarily on value judgments, then a free
and intelligent people's attitude towards us can be expected to be
(ought to be) quite different. The people know that their value
judgments are quite as good as those taught in any law school---maybe
better. If, indeed, the ``liberties'' protected by the Constitution are,
as the Court says, undefined and unbounded, then the people should
demonstrate, to protest that we do not implement their values instead of
ours. Not only that, but confirmation hearings for new Justices should
deteriorate into question-and-answer sessions in which Senators go
through a list of their constituents' most favored and most disfavored
alleged constitutional rights, and seek the nominee's commitment to
support or oppose them. Value judgments, after all, should be voted on,
not dictated; and if our Constitution has somehow accidently committed
them to the Supreme Court, at least we can have a sort of plebiscite
each time a new nominee to that body is put forward. Justice Blackmun
not only regards this prospect with equanimity, he solicits it.

There is a poignant aspect to today's opinion. Its length, and what
might be called its epic tone, suggest that its authors believe they are
bringing to an end a troublesome era in the history of our Nation and of
our Court. ``It is the dimension'' of authority, they say, to
``cal{[}l{]} the contending sides of national controversy to end their
national division by accepting a common mandate rooted in the
Constitution.''

There comes vividly to mind a portrait by Emanuel Leutze that hangs in
the Harvard Law School: Roger Brooke Taney, painted in 1859, the 82d
year of his life, the 24th of his Chief Justiceship, the second after
his opinion in Dred Scott. He is all in black, sitting in a shadowed red
armchair, left hand resting upon a pad of paper in his lap, right hand
hanging limply, almost lifelessly, beside the inner arm of the chair. He
sits facing the viewer and staring straight out. There seems to be on
his face, and in his deep-set eyes, an expression of profound sadness
and disillusionment. Perhaps he always looked that way, even when
dwelling upon the happiest of thoughts. But those of us who know how the
lustre of his great Chief Justiceship came to be eclipsed by Dred Scott
cannot help believing that he had that case---its already apparent
consequences for the Court and its soon-to-be-played-out consequences
for the Nation---burning on his mind. I expect that two years earlier
he, too, had thought himself ``call{[}ing{]} the contending sides of
national controversy to end their national division by accepting a
common mandate rooted in the Constitution.''

It is no more realistic for us in this litigation, than it was for him
in that, to think that an issue of the sort they both involved---an
issue involving life and death, freedom and subjugation---can be
``speedily and finally settled'' by the Supreme Court, as President
James Buchanan in his inaugural address said the issue of slavery in the
territories would be. See Inaugural Addresses of the Presidents of the
United States, S. Doc. No.~101-10, p.~126 (1989). Quite to the contrary,
by foreclosing all democratic outlet for the deep passions this issue
arouses, by banishing the issue from the political forum that gives all
participants, even the losers, the satisfaction of a fair hearing and an
honest fight, by continuing the imposition of a rigid national rule
instead of allowing for regional differences, the Court merely prolongs
and intensifies the anguish.

We should get out of this area, where we have no right to be, and where
we do neither ourselves nor the country any good by remaining.

The Court's suggestion, that adherence to tradition would require us to
uphold laws against interracial marriage is entirely wrong. Any
tradition in that case was contradicted by a text---an Equal Protection
Clause that explicitly establishes racial equality as a constitutional
value. See Loving v. Virginia (``In the case at bar, we deal with
statutes containing racial classifications, and the fact of equal
application does not immunize the statute from the very heavy burden of
justification which the Fourteenth Amendment has traditionally required
of state statutes drawn according to race''); see also (Stewart, J.,
concurring in judgment). The enterprise launched in Roe v. Wade, by
contrast, sought to establish ---in the teeth of a clear, contrary
tradition---a value found nowhere in the constitutional text. There is,
of course, no comparable tradition barring recognition of a ``liberty
interest'' in carrying one's child to term free from state efforts to
kill it. For that reason, it does not follow that the Constitution does
not protect childbirth simply because it does not protect abortion. The
Court's contention, that the only way to protect childbirth is to
protect abortion shows the utter bankruptcy of constitutional analysis
deprived of tradition as a validating factor.It drives one to say that
the only way to protect the right to eat is to acknowledge the
constitutional right to starve oneself to death.

Justice Blackmun's parade of adjectives is similarly empty: Abortion is
among
``\texttt{the\ most\ intimate\ and\ personal\ choices,\textquotesingle{}\ "ante;\ it\ is\ a\ matter\ "central\ to\ personal\ dignity\ and\ autonomy,"\ \ ;\ and\ it\ involves\ "personal\ decisions\ that\ profoundly\ affect\ bodily\ integrity,identity,and\ destiny,"\ Justice\ Stevens\ is\ not\ much\ less\ conclusory:\ The\ decision\ to\ choose\ abortion\ is\ a\ matter\ of\ "the\ highest\ privacy\ and\ the\ most\ personal\ nature,"\ ;\ it\ involves\ a\ "}difficult
choice having serious and personal consequences of major importance to
{[}a woman's{]} future,''' ; the authority to make this ``traumatic and
yet empowering decisio{[}n{]}'' is ``an element of basic human
dignity,'' ; and it is ``nothing less than a matter of conscience,''

The joint opinion is clearly wrong in asserting, that ``the Court's
early abortion cases adhered to'' the ``undue burden'' standard. The
passing use of that phrase in Justice Blackmun's opinion for the Court
in Bellotti v. Baird (Bellotti I), was not by way of setting forth the
standard of unconstitutionality, as Justice O'Connor's later opinions
did, but by way of expressing the conclusion of unconstitutionality.
Justice Powell for a time appeared to employ a variant of ``undue
burden'' analysis in several non majority opinions, see, e. g., Bellotti
v. Baird (Bellotti II); Carey v. Population Services International
(opinion concurring in part and concurring in judgment), but he too
ultimately rejected that standard in his opinion for the Court in Akron
v. Akron Center for Reproductive Health, Inc., n.~1 (1983) (Akron I) The
joint opinion's reliance on Maher v. Roe, and Harris v. McRae, is
entirely misplaced, since those cases did not involve regulation of
abortion, but mere refusal to fund it. In any event, Justice O'Connor's
earlier formulations have apparently now proved unsatisfactory to the
three Justices, who---in the name of stare decisis no less---today find
it necessary to devise an entirely new version of ``undue burden''
analysis. See

The joint opinion further asserts that a law imposing an undue burden on
abortion decisions is not a ``permissible'' means of serving
``legitimate'' state interests. This description of the undue burden
standard in terms more commonly associated with the rational-basis test
will come as a surprise even to those who have followed closely our
wanderings in this forsaken wilderness. See, e. g., Akron I (O'Connor,
J., dissenting) (``The'undue burden' . represents the required threshold
inquiry that must be conducted before this Court can require a State to
justify its legislative actions under the exacting'compelling state
interest' standard''); see also Hodgson v. Minnesota (1990) (O'Connor,
J., concurring in part and concurring in judgment in part); Thornburgh
v. American College of Obstetricians and Gynecologists (O'Connor, J.,
dissenting). This confusing equation of the two standards is apparently
designed to explain how one of the Justices who joined the plurality
opinion in Webster v. Reproductive Health Services, which adopted the
rational-basis test, could join an opinion expressly adopting the undue
burden test. See (rejecting the view that abortion is a ``fundamental
right,'' instead inquiring whether a law regulating the woman's
``liberty interest'' in abortion is ``reasonably designed'' to further
``legitimate'' state ends). The same motive also apparently underlies
the joint opinion's erroneous citation of the plurality opinion in Ohio
v. Akron Center for Reproductive Health (Akron II) (opinion of Kennedy,
J.), as applying the undue burden test. See (using this citation to
support the proposition that ``two of us''---i. e., two of the authors
of the joint opinion---have previously applied this test). In fact,
Akron

does not mention the undue burden standard until the conclusion of the
opinion, when it states that the statute at issue ``does not impose an
undue, or otherwise unconstitutional, burden.'' 497 U. S. (emphasis
added). I fail to see how anyone can think that saying a statute does
not impose an unconstitutional burden under any standard, including the
undue burden test, amounts to adopting the undue burden test as the
exclusive standard. The Court's citation of Hodgson as reflecting
Justice Kennedy's and Justice O'Connor's ``shared premises,'' is
similarly inexplicable, since the word ``undue'' was never even used in
the former's opinion in that case.I joined Justice Kennedy's opinions in
both Hodgson and Akron II; I should be grateful, I suppose, that the
joint opinion does not claim that I, too, have adopted the undue burden
test.

Of course Justice O'Connor was correct in her former view. The
arbitrariness of the viability line is confirmed by the Court's
inability to offer any justification for it beyond the conclusory
assertion that it is only at that point that the unborn child's life
``can in reason and all fairness'' be thought to override the interests
of the mother. Precisely why is it that, at the magical second when
machines currently in use (though not necessarily available to the
particular woman) are able to keep an unborn child alive apart from its
mother, the creature is suddenly able (under our Constitution) to be
protected by law, whereas before that magical second it was not? That
makes no more sense than according infants legal protection only after
the point when they can feed themselves.

The joint opinion is not entirely faithful to this principle, however.
In approving the District Court's factual findings with respect to the
spousal notice provision, it relies extensively on nonrecord materials,
and in reliance upon them adds a number of factual conclusions of its
own. Because this additional factfinding pertains to matters that surely
are ``subject to reasonable dispute,'' Fed. Rule Evid. 201(b), the joint
opinion must be operating on the premise that these are ``legislative''
rather than ``adjudicative'' facts, see Rule 201(a). But if a court can
find an undue burden simply by selectively string-citing the right
social science articles, I do not see the point of emphasizing or
requiring ``detailed factual findings'' in the District Court.

Justice Blackmun's effort to preserve as much of Roe as possible leads
him to read the joint opinion as more ``constan{[}t{]}'' and
``steadfast'' than can be believed. He contends that the joint opinion's
``undue burden'' standard requires the application of strict scrutiny to
``all non-de-minimis'' abortion regulations, but that could only be true
if a ``substantial obstacle,'' (joint opinion), were the same thing as a
non-de-minimis obstacle---which it plainly is not. 2

\hypertarget{reynolds-v.-sims}{%
\subsubsection{Reynolds v. Sims}\label{reynolds-v.-sims}}

377 U.S. 533 (1964)

\emph{Mr.~Chief Justice WARREN delivered the opinion of the Court.}

Involved in these cases are an appeal and two cross-appeals from a
decision of the Federal District Court for the Middle District of
Alabama holding invalid, under the Equal Protection Clause of the
Federal Constitution, the existing and two legislative proposed plans
for the apportionment of seats in the two houses of the Alabama
Legislature, and ordering into effect a temporary reapportionment plan
comprised of parts of the proposed but judicially disapproved measures

On August 26, 1961, the original plaintiffs (appellees in No.~23),
residents, taxpayers and voters of Jefferson County, Alabama, filed a
complaint in the United States District Court for the Middle District of
Alabama, in their own behalf and on behalf of all similarly situated
Alabama voters, challenging the apportionment of the Alabama
Legislature. Defendants below (appellants in No.~23), sued in their
representative capacities, were various state and political party
officials charged with the performance of certain duties in connection
with state elections The complaint alleged a deprivation of rights under
the Alabama Constitution and under the Equal Protection Clause of the
Fourteenth Amendment, and asserted that the District Court had
jurisdiction under provisions of the Civil Rights Act, 42 U.S.C. §§
1983, 1988, as well as under 28 U.S.C. § 1343(3).

The complaint stated that the Alabama Legislature was composed of a
Senate of 35 members and a House of Representatives of 106 members. It
set out relevant portions of the 1901 Alabama Constitution, which
prescribe the number of members of the two bodies of the State
Legislature and the method of apportioning the seats among the State's
67 counties, and provide as follows:

Art. IV, Sec. 50. `The legislature shall consist of not more than
thirty-five senators, and not more than one hundred and five members of
the house of representatives, to be apportioned among the several
districts and counties, as prescribed in this Constitution; provided
that in addition to the above number of representatives, each new county
hereafter created shall be entitled to one representative.'

Art. IX, Sec. 197. `The whole number of senators shall be not less than
one-fourth or more than one-third of the whole number of
representatives.'

Art. IX, Sec. 198. `The house of representatives shall consist of not
more than one hundred and five members, unless new counties shall be
created, in which event each new county shall be entitled to one
representative. The members of the house of representatives shall be
apportioned by the legislature among the several counties of the state,
according to the number of inhabitants in them, respectively, as
ascertained by the decennial census of the United States, which
apportionment, when made, shall not be subject to alteration until the
next session of the legislature after the next decennial census of the
United States shall have been taken.'

Art. IX, Sec. 199. `It shall be the duty of the legislature at its first
session after the taking of the decennial census of the United States in
the year nineteen hundred and ten, and after each subsequent decennial
census, to fix by law the number of representatives and apportion them
among the several counties of the state, according to the number of
inhabitants in them, respectively; provided, that each county shall be
entitled to at least one representative.'

Art. IX, Sec. 200. `It shall be the duty of the legislature at its first
session after taking of the decennial census of the United States in the
year nineteen hundred and ten, and after each subsequent decennial
census, to fix by law the number of senators, and to divide the state
into as many senatorial districts as there are senators, which districts
shall be as nearly equal to each other in the number of inhabitants as
may be, and each shall be entitled to one senator, and no more; and such
districts, when formed, shall not be changed until the next apportioning
session of the legislature, after the next decennial census of the
United States shall have been taken; provided, that counties created
after the next preceding apportioning session of the legislature may be
attached to senatorial districts. No county shall be divided between two
districts, and no district shall be made up of two or more counties not
contiguous to each other.'

Art. XVIII, Sec. 284. `Representation in the legislature shall be based
upon population, and such basis of representation shall not be changed
by constitutional amendments.'

The maximum size of the Alabama House was increased from 105 to 106 with
the creation of a new county in 1903, pursuant to the constitutional
provision which states that, in addition to the prescribed 105 House
seats, each county thereafter created shall be entitled to one
representative. Article IX, §§ 202 and 203, of the Alabama Constitution
established precisely the boundaries of the State's senatorial and
representative districts until the enactment of a new reapportionment
plan by the legislature. These 1901 constitutional provisions,
specifically describing the composition of the senatorial districts and
detailing the number of House seats allocated to each county, were
periodically enacted as statutory measures by the Alabama Legislature,
as modified only by the creation of an additional county in 1903, and
provided the plan of legislative apportionment existing at the time this
litigation was commenced.

Plaintiffs below alleged that the last apportionment of the Alabama
Legislature was based on the 1900 federal census, despite the
requirement of the State Constitution that the legislature be
reapportioned decennially. They asserted that, since the population
growth in the State from 1900 to 1960 had been uneven, Jefferson and
other counties were now victims of serious discrimination with respect
to the allocation of legislative representation. As a result of the
failure of the legislature to reapportion itself, plaintiffs asserted,
they were denied `equal suffrage in free and equal elections and the
equal protection of the laws' in violation of the Alabama Constitution
and the Fourteenth Amendment to the Federal Constitution. The complaint
asserted that plaintiffs had no other adequate remedy, and that they had
exhausted all forms of relief other than that available through the
federal courts. They alleged that the Alabama Legislature had
established a pattern of prolonged inaction from 1911 to the present
which `clearly demonstrates that no reapportionment shall be effected';
that representation at any future constitutional convention would be
established by the legislature, making it unlikely that the membership
of any such convention would be fairly representative; and that, while
the Alabama Supreme Court had found that the legislature had not
complied with the State Constitution in failing to reapportion according
to population decennially,4 that court had nevertheless indicated that
it would not interfere with matters of legislative reapportionment.

Plaintiffs requested that a three-judge District Court be convened With
respect to relief, they sought a declaration that the existing
constitutional and statutory provisions, establishing the present
apportionment of seats in the Alabama Legislature, were unconstitutional
under the Alabama and Federal Constitutions, and an injunction against
the holding of future elections for legislators until the legislature
reapportioned itself in accordance with the State Constitution. They
further requested the issuance of a mandatory injunction, effective
until such time as the legislature properly reapportioned, requiring the
conducting of the 1962 election for legislators at large over the entire
State, and any other relief which `may seem just, equitable and proper.'

A three-judge District Court was convened, and three groups of voters,
taxpayers and residents of Jefferson, Mobile, and Etowah Counties were
permitted to inter- vene in the action as intervenor-plaintiffs. Two of
the groups are cross-appellants in Nos. 27 and 41. With minor
exceptions, all of the intervenors adopted the allegations of and sought
the same relief as the original plaintiffs.

On March 29, 1962, just three days after this Court had decided Baker v.
Carrplaintiffs moved for a preliminary injunction requiring defendants
to conduct at large the May 1962 Democratic primary election and the
November 1962 general election for members of the Alabama Legislature.
The District Court set the motion for hearing in an order stating its
tentative views that an injunction was not required before the May 1962
primary election to protect plaintiffs' constitutional rights, and that
the Court should take no action which was not `absolutely essential' for
the protection of the asserted constitutional rights before the Alabama
Legislature had had a `further reasonable but prompt opportunity to
comply with its duty' under the Alabama Constitution.

On April 14, 1962, the District Court, after reiterating the views
expressed in its earlier order, reset the case for hearing on July 16,
noting that the importance of the case, together with the necessity for
effective action within a limited period of time, required an early
announcement of its views. 205 F.Supp. 245. Relying on our decision in
Baker v. Carr, the Court found jurisdiction, justiciability and
standing. It stated that it was taking judicial notice of the facts that
there had been population changes in Alabama's counties since 1901, that
the present representation in the State Legislature w § not on a
population basis, and that the legislature had never reapportioned its
membership as required by the Alabama Constitution Continuing, the Court
stated that if the legislature complied with the Alabama constitutional
provision requiring legislative representation to be based on population
there could be no objection on federal constitutional grounds to such an
apportionment. The Court further indicated that, if the legislature
failed to act, or if its actions did not meet constitutional standards,
it would be under a `clear duty' to take some action on the matter prior
to the November 1962 general election. The District Court stated that
its `present thinking' was to follow an approach suggested by MR.
JUSTICE CLARK in his concurring opinion in Baker v. Carr---awarding
seats released by the consolidation or revamping of existing districts
to counties suffering `the most egregious discrimination,' thereby
releasing the strangle hold on the legislature sufficiently so as to
permit the newly elected body to enact a constitutionally valid and
permanent reapportionment plan, and allowing eventual dismissal of the
case. Subsequently, plaintiffs were permitted to amend their complaint
by adding a further prayer for relief, which asked the District Court to
reapportion the Alabama Legislature provisionally so that the rural
strangle hold would be relaxed enough to permit it to reapportion
itself.

On July 12, 1962, an extraordinary session of the Alabama Legislature
adopted two reapportionment plans to take effect for the 1966 elections.
One was a proposed constitutional amendment, referred to as the
`67-Senator Amendment.' 9 It provided for a House of Representatives
consisting of 106 members, apportioned by giving one seat to each of
Alabama's 67 counties and distributing the others according to
population by the `equal proportions' method Using this formula, the
constitutional amendment specified the number of representatives
allotted to each county until a new apportionment could be made on the
basis of the 1970 census. The Senate was to be composed of 67 members,
one from each county. The legislation provided that the proposed
amendment should be submitted to the voters for ratification at the
November 1962 general election.

The other reapportionment plan was embodied in a statutory measure
adopted by the legislature and signed into law by the Alabama Governor,
and was referred to as the `Crawford-Webb Act.'11 It was enacted as
standby legislation to take effect in 1966 if the proposed
constitutional amendment should fail of passage by a majority of the
State's voters, or should the federal courts refuse to accept the
proposed amendment (though not rejected by the voters) as effective
action in compliance with the requirements of the Fourteenth Amendment.
The act provided for a Senate consisting of 35 members, representing 35
senatorial districts established along county lines, and altered only a
few of the former districts. In apportioning the 106 seats in the
Alabama House of Representatives, the statutory measure gave each county
one seat, and apportioned the remaining 39 on a rough population basis,
under a formula requiring increasingly more population for a county to
be accorded additional seats. The Crawford-Webb Act also provided that
it would be effective 'until the legislature is reapportioned according
to law,' but provided no standards for such a reapportionment. Future
apportionments would presumably be based on the existing provisions of
the Alabama Constitution which the statute, unlike the proposed
constitutional amendment, would not affect.

The evidence adduced at trial before the three-judge panel consisted
primarily of figures showing the population of each Alabama county and
senatorial district according to the 1960 census, and the number of
representatives allocated to each county under each of the three plans
at issue in the litigation---the existing apportionment (under the 1901
constitutional provisions and the current statutory measures
substantially reenacting the same plan), the proposed 67-Senator
constitutional amendment, and the Crawford-Webb Act. Under all three
plans, each senatorial district would be represented by only one
senator.

On July 21, 1962, the District Court held that the inequality of the
existing representation in the Alabama Legislature violated the Equal
Protection Clause of the Fourteenth Amendment, a finding which the Court
noted had been `generally conceded' by the parties to the litigation,
since population growth and shifts had converted the 1901 scheme, as
perpetuated some 60 years later, into an invidiously discriminatory plan
completely lacking in rationality. 208 F.Supp. 431. Under the existing
provisions, applying 1960 census figures, only 25 \% of the State's
totel population resided in districts represented by a majority of the
members of the Senate, and only 25 \% lived in counties which could
elect a majority of the members of the House of Representatives.
Population-variance ratios of up to about 41-to-1 existed in the Senate,
and up to about 16-to-1 in the House. Bullock County, with a population
of only 13,462, and Henry County, with a population of only 15,286, each
were allocated two seats in the Alabama House, whereas Mobile County,
with a population of 314,301, was given only three seats, and Jefferson
County, with 634,864 people, had only seven representatives With respect
to senatorial apportionment, since the pertinent Alabama constitutional
provisions had been consistently construed as prohibiting the giving of
more than one Senate seat to any one county,13 Jefferson County, with
over 600,000 people, was given only one senator, as was Lowndes County,
with a 1960 population of only 15,417, and Wilcox County, with only
18,739 people.

The Court then considered both the proposed constitutional amendment and
the Crawford-Webb Act to ascer- tain whether the legislature had taken
effective action to remedy the unconstitutional aspects of the existing
apportionment. In initially summarizing the result which it had reached,
the Court stated:

`This Court has reached the conclusion that neither the '67-Senator
Amendment,' nor the `Crawford-Webb Act' meets the necessary
constitutional requirements. We find that each of the legislative acts,
when considered as a whole, is so obviously discriminatory, arbitrary
and irrational that it becomes unnecessary to pursue a detailed
development of each of the relevant factors of the (federal
constitutional) test.'

The Court stated that the apportionment of one senator to each county,
under the proposed constitutional amendment, would `make the
discrimination in the Senate even more invidious than at present.' Under
the 67-Senator Amendment, as pointed out by the court below, `(t)he
present control of the Senate by members representing 25 \% of the
people of Alabama would be reduced to control by members representing 19
\% of the people of the State,' the 34 smallest counties, with a total
population of less than that of Jefferson County, would have a majority
of the senatorial seats, and senators elected by only about 14\% of the
State's population could prevent the submission to the electorate of any
future proposals to amend the State Constitution (since a vote of
two-fifths of the members of one house can defeat a proposal to amend
the Alabama Constitution). Noting that the `only conceivable
rationalization' of the senatorial apportionment scheme is that it was
based on equal representation of political subdivisions within the State
and is thus analogous to the Federal Senate, the District Court rejected
the analogy on the ground that Alabama counties are merely involuntary
political units of the State created by statute to aid in the
administration of state government. In finding the so-called federal
analogy irrelevant, the District Court stated:

`The analogy cannot survive the most superficial examination into the
history of the requirement of the Federal Constitution and the
diametrically opposing history of the requirement of the Alabama
Constitution that representation shall be based on population. Nor can
it survive a comparison of the different political natures of states and
counties.'

The Court also noted that the senatorial apportionment proposal `may not
have complied with the State Constitution,' since not only is it
explicitly provided that the population basis of legislative
representation 'shall not be changed by constitutional amendments,'17
but the Alabama Supreme Court had previously indicated that that
requirement could probably be altered only by constitutional convention
The Court concluded, however, that the apportionment of seats in the
Alabama House, under the proposed constitutional amendment, was 'based
upon reason, with a rational regard for known and accepted standards of
apportionment.'19 Under the proposed apportionment of representatives,
each of the 67 counties was given one seat and the remaining 39 were
allocated on a population basis. About 43\% of the State's total
population would live in counties which could elect a majority in that
body. And, under the provisions of the 67-Senator Amendment, while th
maximum population-variance ratio was increased to about 59-to-1 in the
Senate, it was significantly reduced to about 4 -to-1 in the House of
Representatives. Jefferson County was given 17 House seats, an addition
of 10, and Mobile County was allotted eight, an increase of five. The
increased representation of the urban counties was achieved primarily by
limiting the State's 55 least populous counties to one House seat each,
and the net effect was to take 19 seats away from rural counties and
allocate them to the more populous counties. Even so, serious
disparities from a population-based standard remained. Montgomery
County, with 169,210 people, was given only four seats, while Coosa
County, with a population of only 10,726, and Cleburne County, with only
10,911, were each allocated one representative.

Turning next to the provisions of the Crawford-Webb Act, the District
Court found that its apportionment of the 106 seats in the Alabama House
of Representatives, by allocating one seat to each county and
distributing the remaining 39 to the more populous counties in
diminishing ratio to their populations, was `totally unacceptable.'20
Under this plan, about 37\% of the State's total population would reside
in counties electing a majority of the members of the Alabama House,
with a maximum population-variance ratio of about 5-to-1. Each
representative from Jefferson and Mobile Counties would represent over
52,000 persons while representatives from eight rural counties would
each represnet less than 20,000 people. The Court regarded the
senatorial apportionment provided in the Crawford-Webb Act as 'a step in
the right direction, but an extremely short step,' and but a 'slight
improvement over the present system of representation.'21 The net effect
of combining a few of the less populous counties into two-county
districts and splitting up several of the larger districts into smaller
ones would be merely to increase the minority which would be represented
by a majority of the members of the Senate from 25 \% to only 27 \% of
the State's population The Court pointed out that, under the
Crawford-Webb Act, the vote of a person in the senatorial district
consisting of Bibb and Perry Counties would be worth 20 times that of a
citizen in Jefferson County, and that the vote of a citizen in the six
smallest districts would be worth 15 or more times that of a Jefferson
County voter. The Court concluded that the Crawford-

Webb Act was `totally unacceptable' as a `piece of permanent
legislation' which, under the Alabama Constitution, would have remained
in effect without alteration at least until after the next decennial
census.

Under the detailed requirements of the various constitutional provisions
relating to the apportionment of seats in the Alabama Senate and House
of Representatives, the Court found, the membership of neither house can
be apportioned solely on a population basis, despite the provision in
Art. XVIII, § 284, which states that `(r)epresentation in the
legislature shall be based upon population.' In dealing with the
conflicting and somewhat paradoxical requirements (under which the
number of seats in the House in limited to 106 but each of the 67
counties is required to be given at least one representative, and the
size of the Senate is limited to 35 but it is required to have at least
one-fourth of the members of the House, although no county can be given
more than one senator), the District Court stated its view that `the
controlling or dominant provision of the Alabama Constitution on the
subject of representation in the Legislature' is the previously referred
to language of § 284. The Court stated that the detailed requirements of
Art. IX, §§ 197---200, 'make it obvious that in neither the House nor
the Senate can representation be based strictly and entirely upon
population.

The result may well be that representation according to population to
some extent must be required in both Houses if invidious discrimination
in the legislative systems as a whole is to be avoided. Indeed, it is
the policy and theme of the Alabama Constitution to require
representation according to population in both Houses as nearly as may
be, while still complying with more detailed provisions.'

The District Court then directed its concern to the providing of an
effective remedy. It indicated that it was adopting and ordering into
effect for the November 1962 election a provisional and temporary
reapportionment plan composed of the provisions relating to the House of
Representatives contained in the 67-Senator Amendment and the provisions
of the Crawford-Webb Act relating to the Senate. The Court noted,
however, that `(t)he proposed reapportionment of the Senate in the
'Crawford-Webb Act,' unacceptable as a piece of permanent legislation,
may not even break the strangle hold.' Stating that it was retaining
jurisdiction and deferring any hearing on plaintiffs' motion for a
permanent injunction 'until the Legislature, as provisionally
reapportioned * * *, has an opportunity to provide for a true
reapportionment of both Houses of the Alabama Legislature,' the Court
emphasized that its `moderate' action was designed to break the strangle
hold by the smaller counties on the Alabama Legislature and would not
suffice as a permanent reapportionment. On July 25, 1962, the Court
entered its decree in accordance with its previously stated
determinations, concluding that `plaintiffs are denied equal protection
by virtue of the debasement of their votes, since the Legislature of the
State of Alabama has failed and continues to fail to reapportion itself
(as required by law).' It enjoined the defendant state officials from
holding any future elections under any of the apportionment plans that
it had found invalid, and stated that the 1962 election of Alabam
legislators could validly be conducted only under the apportionment
scheme specified in the Court's order.

After the District Court's decision, new primary elections were held
pursuant to legislation enacted in 1962 at the same special session as
the proposed constitutional amendment and the Crawford-Webb Act, to be
effective in the event the Court itself ordered a particular
reapportionment plan into immediate effect. The November 1962 general
election was likewise conducted on the basis of the District Court's
ordered apportionment of legislative seats, as Mr.~Justice Black refused
to stay the District Court's order. Consequently, the present Alabama
Legislature is apportioned in accordance with the temporary plan
prescribed by the District Court's decree. All members of both houses of
the Alabama Legislature serve four-year terms, so that the next
regularly scheduled election of legislators will not be held until 1966.
The 1963 regular session of the Alabama Legislature produced no
legislation relating to legislative apportionment,24 and the
legislature, which meets biennially, will not hold another regular
session until 1965.

No effective political remedy to obtain relief against the alleged
malapportionment of the Alabama Legislature appears to have been
available. 25 No initiative procedure exists under Alabama law.
Amendment of the State Constitution can be achieved only after a
proposal is adopted by three-fifths of the members of both houses of the
legislature and is approved by a majority of the people,26 or as a
result of a constitutional convention convened after approval by the
people of a convention call initiated by a majority of both houses of
the Alabama Legislature.

Notices of appeal to this Court from the District Court's decision were
timely filed by defendants below (appellants in No.~23) and by two
groups of intervenor-plaintiffs (cross-appellants in Nos. 27 and 41).
Appellants in No.~23 contend that the District Court erred in holding
the existing and the two proposed plans for the apportionment of seats
in the Alabama Legislature unconstitutional, and that a federal court
lacks the power to affirmatively reapportion seats in a state
legislature. Cross-appellants in No.~27 assert that the court below
erred in failing to compel reapportionment of the Alabama Senate on a
population basis as allegedly required by the Alabama Constitution and
the Equal Protection Clause of the Federal Constitution.
Cross-appellants in No.~41 contend that the District Court should have
required and ordered into effect the apportionment of seats in both
houses of the Alabama Legislature on a population basis. We noted
probable jurisdiction on June 10,

Undeniably the Constitution of the United States protects the right of
all qualified citizens to vote, in state as well as in federal
elections. A consistent line of decisions by this Court in cases
involving attempts to deny or restrict the right of suffrage has made
this indelibly clear. It has been repeatedly recognized that all
qualified voters have a constitutionally protected right to vote, Ex
parte Yarbroughand to have their votes counted, United States v.
MosleyIn Mosley the Court stated that it is `as equally unquestionable
that the right to have one's vote counted is as open to protection as
the right to put a ballot in a box.' 238 U.S., The right to vote can
neither be denied outright, Guinn v. United StatesLane v. Wilsonnor
destroyed by alteration of ballots, see United States v. Classic, nor
diluted by ballot-box stuffing Ex parte SieboldL.Ed. 717, United States
v. SaylorAs the Court stated in Classic, 'Obviously included within the
right to choose, secured by the Constitution, is the right of qualified
voters within a state to cast their ballots and have them counted * *
*.' 313 U.S., Racially based gerrymandering, Gomillion v. Lightfootand
the conducting of white primaries, Nixon v. HerndonNixon v. CondonSmith
v. AllwrightTerry v. Adamsboth of which result in denying to some
citizens their right to vote, have been held to be constitutionally
impermissible. And history has seen a continuing expansion of the scope
of the right of suffrage in this country. 28 The right to vote freely
for the candidate of one's choice is of the essence of a democratic
society, and any restrictions on that right strike at the heart of
representative government. And the right of suffrage can be denied by a
debasement or dilution of the weight of a citizen's vote just as
effectively as by wholly prohibiting the free exercise of the franchise.

In Baker v. Carrwe held that a claim asserted under the Equal Protection
Clause challenging the constitutionality of a State's apportionment of
seats in its legislature, on the ground that the right to vote of
certain citizens was effectively impaired since debased and diluted, in
effect presented a justiciable controversy subject to adjudication by
federal courts. The spate of similar cases filed and decided by lower
courts since our decision in Baker amply shows that the problem of state
legislative malapportionment is one that is perceived to exist in a
large number of the States In Baker, a suit involving an attack on the
apportionment of seats in the Tennessee Legislature, we remanded to the
District Court, which had dismissed the action, for consideration on the
merits. We intimated no view as to the proper constitutional standards
for evaluating the validity of a state legislative apportionment scheme.
Nor did we give any consideration to the question of appropriate
remedies. Rather, we simply stated:

`Beyond noting that we have no cause at this stage to doubt the District
Court will be able to fashion relief if violations of constitutional
rights are found, it is improper now to consider what remedy would be
most appropriate if appellants prevail at the trial.'

We indicated in Baker, however, that the Equal Protection Clause
provides discoverable and manageable standards for use by lower courts
in determining the constitutionality of a state legislative
apportionment scheme, and we stated:

`Nor need the appellants, in order to succeed in this action, ask the
Court to enter upon policy determinations for which judicially
manageable standards are lacking. Judicial standards under the Equal
Protection Clause are well developed and familiar, and it has been open
to courts since the enactment of the Fourteenth Amendment to determine,
if on the particular facts they must, that a discrimination reflects no
policy, but simply arbitrary and capricious action.'

Subsequent to Baker, we remanded several cases to the courts below for
reconsideration in light of that decision.

In Gray v. Sanderswe held that the Georgia county unit system,
applicable in statewide primary elections, was unconstitutional since it
resulted in a dilution of the weight of the votes of certain Georgia
voters merely because of where they resided. After indicating that the
Fifteenth and Nineteenth Amendments prohibit a State from overweighting
or diluting votes on the basis of race or sex, we stated:

`How then can one person be given twice or 10 times the voting power of
another person in a statewide election merely because he lives in a
rural area or because he lives in the smallest rural county? Once the
geographical unit for which a representative is to be chosen is
designated, all who participate in the election are to have an equal
vote whatever their race, whatever their sex, whatever their occu-
pation, whatever their income, and wherever their home may be in that
geographical unit. This is required by the Equal Protection Clause of
the Fourteenth Amendment. The concept of 'we the people' under the
Constitution visualizes no preferred class of voters but equality among
those who meet the basic qualifications. The idea that every voter is
equal to every other voter in his State, when he casts his ballot in
favor of one of several competing candidates, underlies many of our
decisions.'

Continuing, we stated that `there is no indication in the Constitution
that homesite or occupation affords a permissible basis for
distinguishing between qualified voters within the State.' And, finally,
we concluded: `The conception of political equality from the Declaration
of Independence, to Lincoln's Gettysburg Address, to the Fifteenth,
Seventeenth, and Nineteenth Amendments can mean only one thing---one
person, one vote.'

We stated in Gray, however, that that case, 'unlike Baker v. Carr, does
not involve a question of the degree to which the Equal Protection
Clause of the Fourteenth Amendment limits the authority of a State
legislatu e in designing the geographical districts from which
representatives are chosen either for the State Legislature or for the
Federal House of Representatives.

Nor does it present the question, inherent in the bicameral form of our
Federal Government, whether a State may have one house chosen without
regard to population.'

Of course, in these cases we are faced with the problem not presented in
Gray---that of determining the basic standards and stating the
applicable guidelines for implementing our decision in Baker v. Carr.

In Wesberry v. Sandersdecided earlier this Term, we held that attacks on
the constitutionality of congressional districting plans enacted by
state legislatures do not present nonjusticiable questions and should
not be dismissed generally for `want of equity.' We determine that the
constitutional test for the validity of congressional districting
schemes was one of substantial equality of population among the various
districts established by a state legislature for the election of members
of the Federal House of Representatives.

In that case we decided that an apportionment of congressional seats
which `contracts the value of some votes and expands that of others' is
unconstitutional, since 'the Federal Constitution intends that when
qualified voters elect members of Congress each vote be given as much
weight as any other vote * * *.' We concluded that the constitutional
prescription for election of members of the House of Representatives `by
the People,' construed in its historical context, `means that as nearly
as is practicable one man's vote in a congressional election is to be
worth as much as another's.' We further stated:

`It would defeat the principle solemnly embodied in the Great
Compromise---equal representation in the House for equal numbers of
people---for us to hold that, within the States, legislatures may draw
the lines of congressional districts in such a way as to give some
voters a greater voice in choosing a Congressman than others.'

We found further, in Wesberry, that `our Constitution's plain objective'
was that 'of making equal repre- sentation for equal numbers of people
the fundamental goal * * *.' We concluded by stating:

`No right is more precious in a free country than that of having a voice
in the election of those who make the laws under which, as good
citizens, we must live. Other rights, even the most basic, are illusory
if the right to vote is undermined. Our Constitution leaves no room for
classification of people in a way that unnecessarily abridges this
right.'

Gray and Wesberry are of course not dispositive of or directly
controlling on our decision in these cases involving state legislative
apportionment controversies. Admittedly, those decisions, in which we
held that, in statewide and in congressional elections, one person's
vote must be counted equally with those of all other voters in a State,
were based on different constitutional considerations and were addressed
to rather distinct problems. But neither are they wholly inapposite.
Gray, though not determinative here since involving the weighting of
votes in statewide elections, established the basic principle of
equality among voters within a State, and held that voters cannot be
classified, constitutionally, on the basis of where they live, at least
with respect to voting in statewide elections. And our decision in
Wesberry was of course grounded on that language of the Constitution
which prescribes that members of the Federal House of Representatives
are t be chosen `by the People,' while attacks on state legislative
apportionment schemes, such as that involved in the instant cases, are
principally based on the Equal Protection Clause of the Fourteenth
Amendment. Nevertheless, Wesberry clearly established that the
fundamental principle of representative government in this country is
one of equal representation for equal numbers of people, without regard
to race, sex, economic status, or place of residence within a State. Our
problem, then, is to ascertain, in the insant cases, whether there are
any constitutionally cognizable principles which would justify
departures from the basic standard of equality among voters in the
apportionment of seats in state legislatures.

A predominant consideration in determining whether a State's legislative
apportionment scheme constitutes an invidious discrimination violative
of rights asserted under the Equal Protection Clause is that the rights
allegedly impaired are individual and personal in nature. As stated by
the Court in United States v. Bathgate, '(t)he right to vote is personal
* * *.'39 While the result of a court decision in a state legislative
apportionment controversy may be to require the restructuring of the
geographical distribution of seats in a state legislature, the judicial
focus must be concentrated upon ascertaining whether there has been any
discrimination against certain of the State's citizens which constitutes
an impermissible impairment of their constitutionally protected right to
vote. Like Skinner v. Oklahomasuch a case `touches a sensitive and
important area of human rights,' and `involves one of the basic civil
rights of man,' presenting questions of alleged `invidious
discriminations against groups or types of individuals in violation of
the constitutional guaranty of just and equal laws.' 316 U.S., 541,
Undoubtedly, the right of suffrage is a fundamental mat- ter in a free
and democratic society. Especially since the right to exercise the
franchise in a free and unimpaired manner is preservative of other basic
civil and political rights, any alleged infringement of the right of
citizens to vote must be carefully and meticulously scrutinized. Almost
a century ago, in Yick Wo v. Hopkinsthe Court referred to `the political
franchise of voting' as `a fundamental political right, because
preservative of all rights.'

Legislators represent people, not trees or acres. Legislators are
elected by voters, not farms or cities or economic interests. As long as
ours is a representative form of government, and our legislatures are
those instruments of government elected directly by and directly
representative of the people, the right to elect legislators in a free
and unimpaired fashion is a bedrock of our political system. It could
hardly be gainsaid that a constitutional claim had been asserted by an
allegation that certain otherwise qualified voters had been entirely
prohibited from voting for members of their state legislature. And, if a
State should provide that the votes of citizens in one part of the State
should be given two times, or five times, or 10 times the weight of
votes of citizens in another part of the State, it could hardly be
contended that the right to vote of those residing in the disfavored
areas had not been effectively diluted. It would appear extraordinary to
suggest that a State could be constitutionally permitted to enact a law
providing that certain of the State's voters could ote two, five, or 10
times for their legislative representatives, while voters living
elsewhere could vote only once. And it is inconceivable that a state law
to the effect that, in counting votes for legislators, the votes of
citizens in one part of the State would be multiplied by two, five, or
10, while the votes of persons in another area would be counted only at
face value, could be constitutionally sustainable. Of course, the effect
of state legislative districting schemes which give the same number of
representatives to unequal numbers of constituents is identical
Overweighting and overvaluation of the votes of those living here has
the certain effect of dilution and undervaluation of the votes of those
living there. The resulting discrimination against those individual
voters living in disfavored areas is easily demonstrable mathematically.
Their right to vote is simply not the same right to vote as that of
those living in a favored part of the State. Two, five, or 10 of them
must vote before the effect of their voting is equivalent to that of
their favored neighbor. Weighting the votes of citizens differently, by
any method or means, merely because of where they happen to reside,
hardly seems justifiable. One must be ever aware that the Constitution
forbids `sophisticated as well as simpleminded modes of discrimination.'
Lane v. Wilson, ; Gomillion v. Lightfoot, As we stated in Wesberry v.
Sanders:

`We do not believe that the Framers of the Constitution intended to
permit the same vote-diluting discrimination to be accomplished through
the device of districts containing widely varied numbers of inhabitants.
To say that a vote is worth more in one district than in another would
run counter to our fundamental ideas of democratic government.'

State legislatures are, historically, the fountainhead of representative
government in this country. A number of them have their roots in
colonial times, and substantially antedate the creation of our Nation
and our Federal Government. In fact, the first formal stirrings of
American political independence are to be found, in large part, in the
views and actions of several of the colonial legislative bodies. With
the birth of our National Government, and the adoption and ratification
of the Federal

Constitution, state legislatures retained a most important place in our
Nation's governmental structure. But representative government is in
essence self-government through the medium of elected representatives of
the people, and each and every citizen has an inalienable right to full
and effective participation in the political processes of his State's
legislative bodies. Most citizens can achieve this participation only as
qualified voters through the election of legislators to represent them.
Full and effective participation by all citizens in state government
requires, therefore, that each citizen have an equally effective voice
in the election of members of his state legislature. Modern and viable
state government needs, and the Constitution demands, no less.

Logically, in a society ostensibly grounded on representative
government, it would seem reasonable that a majority of the people of a
State could elect a majority of that State's legislators. To conclude
differently, and to sanction minority control of state legislative
bodies, would appear to deny majority rights in a way that far surpasses
any possible denial of minority rights that might otherwise be thought
to result. Since legislatures are responsible for enacting laws by which
all citizens are to be governed, they should be bodies which are
collectively responsive to the popular will. And the concept of equal
protection has been traditionally viewed as requiring the uniform
treatment of persons standing in the same relation to the governmental
action questioned or challenged. With respect to the allocation of
legislative representation, all voters, as citizens of a State, stand in
the same relation regardless of where they live. Any suggested criteria
for the differentiation of citizens are insufficient to justify any
discrimination, as to the weight of their votes, unless relevant to the
permissible purposes of legislative apportionment. Since the achieving
of fair and effective representation for all citi- zens is concededly
the basic aim of legislative apportionment, we conclude that the Equal
Protection Clause guarantees the opportunity for equal participation by
all voters in the election of state legislators. Diluting the weight of
votes because of place of residence impairs basic constitutional rights
under the Fourteenth Amendment just as much as invidious discriminations
based upon factors such as race, Brown v. Board of Educationor economic
status, Griffin v. People of State of IllinoisDouglas v. People of State
of CaliforniaOur constitutional system amply provides for the protection
of minorities by means other than giving them majority control of state
legislatures. And the democratic ideals of equality and majority rule,
which have served this Nation so well in the past, are hardly of any
less significance for the present and the future.

We are told that the matter of apportioning representation in a state
legislature is a complex and many-faceted one. We are advised that
States can rationally consid r factors other than population in
apportioning legislative representation. We are admonished not to
restrict the power of the States to impose differing views as to
political philosophy on their citizens. We are cautioned about the
dangers of entering into political thickets and mathematical quagmires.
Our answer is this: a denial of constitutionally protected rights
demands judicial protection; our oath and our office require no less of
us. As stated in Gomillion v. Lightfoot:

`When a State exercises power wholly within the domain of state
interest, it is insulated from federal judicial review. But such
insulation is not carried over when state power is used as an instrument
for circumventing a federally protected right.'

To the extent that a citizen's right to vote is debased, he is that much
less a citizen. The fact that an individual lives here or there is not a
legitimate reason for overweighting or diluting the efficacy of his
vote. The complexions of societies and civilizations change, often with
amazing rapidity. A nation once primarily rural in character becomes
predominantly urban Representation schemes once fair and equitable
become archaic and outdated. But the basic principle of representative
government remains, and must remain, unchanged---the weight of a
citizen's vote cannot be made to depend on where he lives. Population
is, of necessity, the starting point for consideration and the
controlling criterion for judgment in legislative apportionment
controversies.

A citizen, a qualified voter, is no more nor no less so because he lives
in the city or on the farm. This is the clear and strong command of our
Constitution's Equal Protection Clause. This is an essential part of the
concept of a government of laws and not men. This is at the heart of
Lincoln's vision of `government of the people, by the people, (and) for
the people.' The Equal Protection Clause demands no less than
substantially equal state legislative representation for all citizens,
of all places as well as of all races.

\begin{enumerate}
\def\labelenumi{\Roman{enumi}.}
\setcounter{enumi}{3}
\item
\end{enumerate}

We hold that, as a basic constitutional standard, the Equal Protection
Clause requires that the sea § in both houses of a bicameral state
legislature must be apportioned on a population basis. Simply stated, an
individual's right to vote for state legislators is unconstitutionally
impaired when its weight is in a substantial fashion diluted when
compared with votes of citizens living on other parts of the State.
Since, under neither the existing apportionment provisions nor either of
the proposed plans was either of the houses of the Alabama Legislature
apportioned on a population basis, the District Court correctly held
that all three of these schemes were constitutionally invalid.
Furthermore, the existing apportionment, and also to a lesser extent the
apportionment under the Crawford-Webb Act, presented little more than
crazy quilts, completely lacking in rationality, and could be found
invalid on that basis alone Al- though the District Court presumably
found the apportionment of the Alabama House of Representatives under
the 67-Senator Amendment to be acceptable, we conclude that the
deviations from a strict population basis are too egregious to permit us
to find that that body, under this proposed plan, was apportioned
sufficiently on a population basis so as to permit the arrangement to be
constitutionally sustained. Although about 43\% of the State's total
population would be required to comprise districts which could elect a
majority in that body, only 39 of the 106 House seats were actually to
be distributed on a population basis, as each of Alabama's 67 counties
was given at least one representative, and population-variance ratios of
close to 5-to-1 would have existed. While mathematical nicety is not a
constitutional requisite, one could hardly conclude that the Alabama
House, under the proposed constitutional amendment, had been apportioned
sufficiently on a population basis to be sustainable under the
requirements of the Equal Protection Clause. And none of the other
apportionments of seats in either of the bodies of the Alabama
Legislature under the three plans considered by the District Court, came
nearly as close to approaching the required constitutional standard as
did that of the House of Representatives under the 67-Senator Amendment.

Legislative apportionment in Alabama is signally illustrative and
symptomatic of the seriousness of this problem in a number of the
States. At the time this litigation was commenced, there had been no
reappor- tionment of seats in the Alabama Legislature for over 60 years
Legislative inaction, coupled with the unavailability of any political
or judicial remedy,47 had resulted, with the passage of years, in the
perpetuated scheme becoming little more than an irrational anachronism.
Consistent failure by the Alabama Legislature to comply with state
constitutional requirements as to the frequency of reapportionment and
the bases of legislative representation resulted in a minority strangle
hold on the State Legislature. Inequality of representation in one house
added to the inequality in the other. With the crazy-quilt existing
apportionment virtually conceded to be invalid, the Alabama Legislature
offered two proposed plans for consideration by the District Court,
neither of which was to be effective until 1966 and neither of which
provided for the apportionment of even one of the two houses on a
population basis. We find that the court below did not err in holding
that neither of these proposed reapportionment schemes, considered as a
whole `meets the necessary constitutional requirements.' And we conclude
that the District Court acted properly in considering these two proposed
plans, although neither was to become effective until the 1966 election
and the proposed constitutional amendment was scheduled to be submitted
to the State's voters in November 1962.

Consideration by the court below of the two proposed plans was clearly
necessary in determining whether the Alabama Legislature had acted
effectively to correct the admittedly existing malapportionment, and in
ascertaining what sort of judicial relief, if any, should be afforded.

\begin{enumerate}
\def\labelenumi{\Alph{enumi}.}
\setcounter{enumi}{21}
\item
\end{enumerate}

Since neither of the houses of the Alabama Legislature, under any of the
three plans considered by the District Court, was apportioned on a
population basis, we would be justified in proceeding no further.
However, one of the proposed plans, that contained in the so-called
67-Senator Amendment, at least superficially resembles the scheme of
legislative representation folowed in the Federal Congress. Under this
plan, each of Alabama's 67 counties is allotted one senator, and no
counties are given more than one Senate seat. Arguably, this is
analogous to the allocation of two Senate seats, in the Federal
Congress, to each of the 50 States, regardless of population. Seats in
the Alabama House, under the proposed constitutional amendment, are
distributed by giving each of the 67 counties at least one, with the
remaining 39 seats being allotted among the more populous counties on a
population basis. This scheme, at least at first glance, appears to
resemble that prescribed for the Federal House of Representatives, where
the 435 seats are distributed among the States on a population basis,
although each State, regardless of its population, is given at least one
Congressman. Thus, although there are substantial differences in
underlying rationale and result,49 the 67-Senator Amendment, as proposed
by the Alabama Legislature, at least arguably presents for consideration
a scheme analogous to that used for apportioning seats in Congress.

Much has been written since our decision in Baker v. Carr about the
applicability of the so-called federal analogy to state legislative
apportionment arrangements After considering the matter, the court below
concluded that no conceivable analogy could be drawn between the federal
scheme and the apportionment of seats in the Alabama Legislature under
the proposed con- stitutional amendment We agree with the District
Court, and find the federal analogy inapposite and irrelevant to state
legislative districting schemes. Attempted reliance on the federal
analogy appears often to be little more than an after-the-fact
rationalization offered in defense of maladjusted state apportionment
arrangements. The original constitutions of 36 of our States provided
that representation in both houses of the state legislatures would be
based completely, or predominantly, on population And the Founding
Fathers clearly had no intention of establishing a pattern or model for
the apportionment of seats in state legislatures when the system of
representation in the Federal Congress was adopted Demonstrative of this
is the fact that the Northwest Ordinance, adopted in the same year,
1787, as the Federal Constitution, provided for the apportionment of
seats in territorial legislatures solely on the basis of population.

The system of representation in the two Houses of the Federal Congress
is one ingrained in our Constitution, as part of the law of the land. It
is one conceived out of compromise and concession indispensable to the
establishment of our federal republic Arising from unique historical cir
umstances, it is based on the consideration that in establishing our
type of federalism a group of formerly independent States bound
themselves together under one national government. Admittedly, the
original 13 States surrendered some of their sovereignty in agreeing to
join together `to form a more perfect Union.' But at the heart of our
constitutional system remains the concept of separate and distinct
governmental entities which have delegated some, but not all, of their
formerly held powers to the single national government. The fact that
almost three-fourths of our present States were never in fact
independently sovereign does not detract from our view that the
so-called federal analogy is inapplicable as a sustaining precedent for
state legislative apportionments. The developing history and growth of
our republic cannot cloud the fact that, at the time of the inception of
the system of representation in the Federal Congress, a compromise
between the larger and smaller States on this matter averted a deadlock
in the Constitutional Convention which had threatened to abort the birth
of our Nation. In rejecting an asserted analogy to the federal electoral
college in Gray v. Sanderswe stated: an analogous system by a State in a
statewide election. No such specific accommodation of the latter was
ever undertaken, and therefore no validation of its numerical inequality
ensued.'

Political subdivisions of States---counties, cities, or whatever---never
were and never have been considered as sovereign entities. Rather, they
have been traditionally regarded as subordinate governmental
instrumentalities created by the State to assist in the carrying out of
state governmental functions. As stated by the Court in Hunter v. City
of Pittsburgh, these governmental units are `created as convenient
agencies for exercising such of the governmental powers of the state as
may be entrusted to them,' and the `number, nature, and duration of the
powers conferred upon (them) and the territory over which they shall be
exercised rests in the absolute discretion of the state.' The
relationship of the States to the Federal Government could hardly be
less analogous.

Thus, we conclude that the plan contained in the 67-Senator Amendment
for apportioning seats in the Alabama Legislature cannot be sustained by
recourse to the so-called federal analogy. Nor can any other inequitable
state legislative apportionment scheme be justified on such an asserted
basis. This does not necessarily mean that such a plan is irrational or
involves something other than a `republican form of government.' We
conclude simply that such a plan is impermissible for the States under
the Equal Protection Clause, since perforce resulting, in virtually
every case, in submergence of the equal-population principle in at least
one house of a state legislature.

Since we find the so-called federal analogy inapposite to a
consideration of the constitutional validity of state legislative
apportionment schemes, we necessarily hold that the Equal Protection
Clause requires both houses of a state legislature to be apportioned on
a population basis. The right of a citizen to equal representation and
to have his vote weighted equally with those of all other citizens in
the election of members of one house of a bicameral state legislature
would amount to little if States could effectively submerge the
equal-population principle in the apportionment of seats in the other
house. If such a scheme were permissible, an individual citizen's
ability to exercise an effective voice in the only instrument of state
government directly representative of the people might be almost as
effectively thwarted as if neither house were apportioned on a
population basis. Deadlock between the two bodies might result in
compromise and concession on some issues. But in all too many cases the
more probable result would be frustration of the majority will through
minority veto in the house not apportioned on a population basis,
stemming directly from the failure to accord adequate overall
legislative representation to all of the State's citizens on a
nondiscriminatory basis. In summary, we can perceive no constitutional
difference, with respect to the geographical distribution of state
legislative representation, between the two houses of a bicameral state
legislature.

We do not believe that the concept of bicameralism is rendered
anachronistic and meaningless when the predominant basis of
representation in the two state legislative bodies is required to be the
same---population. A prime reason for bicameralism, modernly considered,
is to insure mature and deliberate consideration of, and to prevent
precipitate action on, proposed legislative measures. Simply because the
controlling criterion for apportioning representation is required to be
the same in both houses does not mean that there will be no differences
in the composition and complexion of the two bodies. Different
constituencies can be represented in the two houses. One body could be
composed of single-member districts while the other could have at least
some multimember districts. The length of terms of the legislators in
the separate bodies could differ. The numerical size of the two bodies
could be made to differ, even significantly, and the geographical size
of districts from which legislators are elected could also be made to
differ. And apportionment in one house could be arranged so as to
balance off minor inequities in the representation of certain areas in
the other house. In summary, these and other factors could be, and are
presently in many States, utilized to engender differing complexions and
collective attitudes in the two bodies of a state legislature, although
both are apportioned substantially on a population basis. VI.

By holding that as a federal constitutional requisite both houses of a
state legislature must be apportioned on a population basis, we mean
that the Equal Protection Clause requires that a State make an honest
and good faith effort to construct districts, in both houses of its
legislature, as nearly of equal population as is practicable. We realize
that it is a practical impossibility to arrange legislative districts so
that each one has an identical number of residents, or citizens, or
voters. Mathematical exactness or precision is hardly a workable
constitutional requirement.

In Wesberry v. Sandersthe Court stated that congressional representation
must be based on population as nearly as is practicable. In implementing
the basic constitutional principle of representative government as
enunciated by the Court in Wesberry---equality of popu- lation among
districts---some distinctions may well be made between congressional and
state legislative representation. Since, almost invariably, there is a
significantly larger number of seats in state legislative bodies to be
distributed within a State than congressional seats, it may be feasible
to use political subdivision lines to a greater extent in establishing
state legislative districts than in congressional districting while
still affording adequate representation to all parts of the State. To do
so would e constitutionally valid, so long as the resulting
apportionment was one based substantially on population and the
equal-population principle ws not diluted in any significant way.
Somewhat more flexibility may therefore be constitutionally permissible
with respect to state legislative apportionment than in congressional
districting. Lower courts can and assuredly will work out more concrete
and specific standards for evaluating state legislative apportionment
schemes in the context of actual litigation. For the present, we deem it
expedient not to attempt to spell out any precise constitutional tests.
What is marginally permissible in one State may be unsatisfactory in
another, depending on the particular circumstances of the case.
Developing a body of doctrine on a case-by-case basis appears to us to
provide the most satisfactory means of arriving at detailed
constitutional requirements in the area of state legislative
apportionment. Cf. Slaughter-House Cases, 16 Wall. 36, 78---79, Thus, we
proceed to state here only a few rather general considerations which
appear to us to be relevant.

A State may legitimately desire to maintain the integrity of various
political subdivisions, insofar as possible, and provide for compact
districts of contiguous territory in designing a legislative
apportionment scheme. Valid considerations may underlie such aims.
Indiscriminate districting, without any regard for political subdivision
or natural or historical boundary lines, may be little more than an open
invitation to partisan gerrymandering. Single-member districts may be
the rule in one State, while another State might desire to achieve some
flexibility by creating multimember58 or floterial districts Whatever
the means of accomplishment, the overriding objective must be
substantial equality of population among the various districts, so that
the vote of any citizen is approximately equal in weight to that of any
other citizen in the State.

History indicates, however, that many States have deviated, to a greater
or lesser degree, from the equal-population principle in the
apportionment of seats in at least one house of their legislatures So
long as the divergences from a strict population standard are based on
legitimate considerations incident to the effectuation of a rational
state policy, some deviations from the equal-population principle are
constitutionally permissible with respect to the apportionment of seats
in either or both of the two houses of a bicameral state legislature.
But neither history alone,61 nor economic or other sorts of group
interests, are permissible factors in attempting to justify disparties
from population-based representation. Citizens, not history or economic
interests, cast votes. Considerations of area alone provide an
insufficient justification for deviations from the equal-population
principle. Again, people, not land or trees or pastures, vote. Modern
developments and improvements in transportation and communications make
rather hollow, in the mid-1960's, most claims that deviations from
population-based representation can validly be based solely on
geographical considerations. Arguments for allowing such deviations in
order to insure effective representation for sparsely settled areas and
to prevent legislative districts from becoming so large that the
availability of access of citizens to their representatives is impaired
are today, for the most part, unconvincing.

A consideration that appears to be of more substance in justifying some
deviations from population-based representation in state legislatures is
that of insuring some voice to political subdivisions, as political
subdivisions. Several factors make more than insubstantial claims that a
State can rationally consider according political subdivisions some
independent representation in at least one body of the state
legislature, as long as the basic standard of equality of population
among districts is maintained. Local governmental entities are
frequently charged with various responsibilities incident to the
operation of state government. In many States much of the legislature's
activity involves the enactment of so-called local legislation, directed
only to the concerns of particular political subdivisions. And a State
may legitimately desire to construct districts along political
subdivision lines to deter the possibilities of gerrymandering. However,
permitting deviations from population-based representation does not mean
that each local governmental unit or political subdivision can be given
separate representation, regardless of population. Carried too far, a
scheme of giving at least one seat in one house to each political
subdivision (for example, to each county) could easily result, in many
States, in a total subversion of the equal-population principle in that
legislative body This would be especially true in a State where the
number of counties is large and many of them are sparsely populated, and
the number of seats in the legislative body being apportioned does not
significantly exceed the number of counties Such a result, we conclude,
would be constitutionally impermissible. And careful judicial scrutiny
must of course be given, in evaluating state apportionment schemes, to
the character as well as the degree of deviations from a strict
population basis. But if, even as a result of a clearly rational state
policy of according some legislative representation to political
subdivisions, population is submerged as the controlling consideration
in the apportionment of seats in the particular legislative body, then
the right of all of the State's citizens to cast an effective and
adequately weighted vote would be unconstitutionally impaired.

One of the arguments frequently offered as a basis for upholding a
State's legislative apportionment arrangement, despite substantial
disparities from a population basis in either or both houses, is
grounded on congressional approval, incident to admitting States into
the Union, of state apportionment plans containing deviations from the
equal-population principle. Proponents of this argument contend that
congressional approval of such schemes, despite their disparities from
population-bas d representation, indicates that such arrangements are
plainly sufficient as establishing a `republican form of government.' As
we stated in Baker v. Carr, some questions raised under the Guaranty
Clause are nonjusticiable, where `political' in nature and where there
is a clear absence of judicially manageable standards Nevertheless, it
is not inconsistent with this view to hold that, despite congressional
approval of state legislative apportionment plans at the time of
admission into the Union, even though deviating from the
equal-population principle here enunciated, the Equal Protection Clause
can and does require more. And an apportionment scheme in which both
houses are based on population can hardly be considered as failing to
satisfy the Guaranty Clause requirement. Congress presumably does not
assume, in admitting States into the Union, to pass on all
constitutional questions relating to the character of state governmental
organization. In any event, congressional approval, however
well-considered, could hardly validate an unconstitutional state
legislative apportionment. Congress simply lacks the constitutional
power to insulate States from attack with respect to alleged
deprivations of individual constitutional rights.

That the Equal Protection Clause requires that both houses of a state
legislature be apportioned on a population basis does not mean that
States cannot adopt some reasonable plan for periodic revision of their
apportionment schemes. Decennial reapportionment appears to be a
rational approach to readjustment of legislative representation in order
to take into account population shifts and growth. Reallocation of
legislative seats every 10 years coincides with the prescribed practice
in 41 of the States,65 often honored more in the breach than the
observance, however, Illustratively, the Alabama Constitution requires
decennial reapportionment, yet the last reapportionment of the Alabama
Legislature, when this suit was brought, was in 1901. Limitations on the
frequency of reapportionment are justified by the need for stability and
continuity in the organization of the legislative system, although
undoubtedly reapportioning no more frequently than every 10 years leads
to some imbalance in the population of districts toward the end of the
decennial period and also to the development of resistance to change on
the part of some incumbent legislators. In substance, we do not regard
the Equal Protection Clause as requiring daily, monthly, annual or
biennial reapportionment, so long as a State has a reasonably conceived
plan for periodic readjustment of legislative representation. While we
do not intend to indicate that decennial reapportionment is a
constitutional requisite, compliance with such an approach would clearly
meet the minimal requirements for maintaining a reasonably current
scheme of legislative representation. And we do not mean to intimate
that more frequent reapportionment would not be constitutionally
permissible or practicably desirable. But if reapportionment were
accomplished with less frequency, it would assuredly be constitutionally
suspect.

Although general provisions of the Alabama Constitution provide that the
apportionment of seats in both houses of the Alabama Legislature should
be on a population basis, other more detailed provisions clearly make
compliance with both sets of requirements impossible. With respect to
the operation of the Equal Protection Clause, it makes no difference
whether a State's apportionment schem is embodied in its constitution or
in statutory provisions. In those States where the alleged
malapportionment has resulted from noncompliance with state
constitutional provisions which, if complied with, would result in an
apportionment valid under the Equal Protection Clause, the judicial task
of providing effective relief would appear to be rather simple. We agree
with the view of the District Court that state constitutional provisions
should be deemed violative of the Federal Constitution only when validly
asserted constitutional rights could not otherwise be protected and
effectuated. Clearly, courts should attempt to accommodate the relief
ordered to the apportionment provisions of state constitutions insofar
as is possible. But it is also quite clear that a state legislative
apportionment scheme is no less violative of the Federal Constitution
when it is based on state constitutional provisions which have been
consistently complied with than when resulting from a noncompliance with
state constitutional requirements. When there is an unavoidable conflict
between the Federal and a State Constitution, the Supremacy Clause of
course controls.

We do not consider here the difficult question of the proper remedial
devices which federal courts should utilize in state legislative
apportionment cases. 66 Remedial techniques in this new and developing
area of the law will probably often differ with the circumstances of the
challenged apportionment and a variety of local conditions. It is enough
to say now that, once a State's legislative apportionment scheme has
been found to be unconstitutional, it would be the unusual case in which
a court would be justified in not taking appropriate action to insure
that no further elections are conducted under the invalid plan. However,
under certain circumstances, such as where an impending election is
imminent and a State's election machinery is already in progress,
equitable considerations might justify a court in withholding the
granting of immediately effective relief in a legislative apportionment
case, even though the existing apportionment scheme was found invalid.
In awarding or withholding immediate relief, a court is entitled to and
should consider the proximity of a forthcoming election and the
mechanics and complexities of state election laws, and should act and
rely upon general equitable principles. With respect to the timing of
relief, a court can reasonably endeavor to avoid a disruption of the
election process which might result from requiring precipitate changes
that could make unreasonable or embarrassing demands on a State in
adjusting to the requirements of the court's decree. As stated by MR.
JUSTICE DOUGLAS, concurring in Baker v. Carr, `any relief accorded can
be fashioned in the light of well-known principles of equity.'

We feel that the District Court in this case acted in a most proper and
commendable manner. It initially acted wisely in declining to stay the
impending primary election in Alabama, and properly refrained from
acting further until the Alabama Legislature had been given an
opportunity to remedy the admitted discrepancies in the State's
legislative apportionment scheme, while initially stating some of its
views to provide guidelines for legislative action. And it correctly
recognized that legislative reapportionment is primarily a matter for
legislative consideration and determination, and that judicial relief
becomes appropriate only when a legislature fails to reapportion
according to federal constitutional requisites in a timely fashion after
having had an adequate opportunity to do so. Additionally, the court
below acted with proper judicial restraint, after the Alabama
Legislature had failed to act effectively in remedying he constitutional
deficiencies in the State's legislative apportionment scheme, in
ordering its own temporary reapportionment plan into effect, at a time
sufficiently early to permit the holding of elections pursuant to that
plan without great difficulty, and in prescribing a plan admittedly
provisional in purpose so as not to usurp the primary responsibility for
reapportionment which rests with the legislature.

We find, therefore, that the action taken by the District Court in this
case, in ordering into effect a reapportionment of both houses of the
Alabama Legislature for purposes of the 1962 primary and general
elections, by using the best parts of the two proposed plans which it
had found, as a whole, to be invalid,68 was an appropriate and
well-considered exercise of judicial power. Admittedly, the lower
court's ordered plan was intended only as a temporary and provisional
measure and the District Court correctly indicated that the plan was
invalid as a permanent apportionment. In retaining jurisdiction while
deferring a hearing on the issuance of a final injunction in order to
give the provisionally reapportioned legislature an opportunity to act
effectively, the court below proceeded in a proper fashion. Since the
District Court evinced its realization that its ordered reapportionment
could not be sustained as the basis for conducting the 1966 election of
Alabama legislators, and avowedly intends to take some further action
should the reapportioned Alabama Legislature fail to enact a
constitutionally valid, permanent apportionment scheme in the interim,
we affirm the judgment below and remand the cases for further
proceedings consistent with the views stated in this opinion. It is so
ordered.

Affirmed and remanded.

\textbf{Mr.~Justice CLARK, concurring in the affirmance.} The Court goes
much beyond the necessities of this case in laying down a new `equal
population' principle for state legislative apportionment. This
principle seems to be an offshoot of Gray v. Sanders, i.e., `one person,
one vote,' modified by the `nearly as is practicable' admonition of
Wesberry v. Sanders, .* Whether `nearly as is practicable' means `one
person, one vote' qualified by `approximately equal' or `some
deviations' or by the impossibility of `mathematical nicety' is not
clear from the majority's use of these vague and meaningless phrases.
But whatever the standard, the Court applies it to each house of the
State Legislature.

It seems to me that all that the Court need say in this case is that
each plan considered by the trial court is `a crazy quilt,' clearly
revealing invidious discrimination in each house of the Legislature and
therefore violative of the Equal Protection Clause. See my concurring
opinion in Baker v. Carr,

I, therefore, do not reach the question of the so-called `federal
analogy.' But in my view, if one house of the State Legislature meets
the population standard, representation in the other house might include
some departure from it so as to take into account, on a rational basis,
other factors in order to afford some representation to the various
elements of the State.

All of the parties have agreed with the District Court's finding that
legislative inaction for some 60 years in the face of growth and shifts
in population has converted Alabama's legislative apportionment plan
enacted in 1901 into one completely lacking in rationality. Accordingly,
for the reasons stated in my dissenting opinion in Lucas v. Forty-Fourth
General Assembly of Colorado, 377 U.S., p.~744, 84 S.Ct., p.~1477. I
would affirm the judgment of the District Court holding that this
apportionment violated the Equal Protection Clause.

I also agree with the Court that it was proper for the District Court,
in framing a remedy, to adhere as closely as practicable to the
apportionments approved by the representatives of the people of Alabama,
and to afford the State of Alabama full opportunity, consistent with the
requirements of the Federal Constitution, to devise its own system of
legislative apportionment.

\textbf{Mr.~Justice HARLAN, dissenting.}

In these cases the Court holds that seats in the legislatures of six
States 1 are apportioned in ways that violate the Federal Constitution.
Under the Court's ruling it is bound to follow that the legislatures in
all but a few of the other 44 States will meet the same fate These
decisions, with Wesberry v. Sandersinvolving congressional districting
by the States, and Gray v. Sandersrelating to elections for statewide
office, have the effect of placing basic aspects of state political
systems under the pervasive overlordship of the federal judiciary. Once
again,3 I must register my protest.

PRELIMINARY STATEMENT.

Today's holding is that the Equal Protection Clause of the Fourteenth
Amendment requires every State to structure its legislature so that all
the members of each house represent substantially the same number of
people; other factors may be given play only to the extent that they do
not significantly encroach on this basic `population' principle.
Whatever may be thought of this holding as a piece of political
ideology---and even on that score the political history and practices of
this country from its earliest beginnings leave wide room for debate
(see the dissenting opinion of Frankfurter, J., in Baker v. Carr, 301,
)---I think it demonstrable that the Fourteenth Amendment does not
impose this political tenet on the States or authorize this Court to do
so.

The Court's constitutional discussion, found in its opinion in the Ala
ama cases (Nos. 23, 27, 41, ante, p.~533) and more particularly at pages
561---568 thereof, is remarkable (as, indeed, is that found in the
separate opinions of my Brothers STEWART and CLARK, ante, pp.~588, 587,)
for its failure to address itself at all to the Fourteenth Amendment as
a whole or to the legislative history of the Amendment pertinent to the
matter at hand. Stripped of aphorisms, the Court's argument boils down
to the assertion that appellees' right to vote has been invidiously
`debased' or `diluted' by systems of apportionment which entitle them to
vote for fewer legislators than other voters, an assertion which is tied
to the Equal Protection Clause only by the constitutionally frail
tautology that `equal' means `equal.'

Had the Court paused to probe more deeply into the matter, it would have
found that the Equal Protection Clause was never intended to inhibit the
States in choos- ing any democratic method they pleased for the
apportionment of their legislatures. This is shown by the language of
the Fourteenth Amendment taken as a whole, by the understanding of those
who proposed and ratified it, and by the political practices of the
States at the time the Amendment was adopted. It is confirmed by
numerous state and congressional actions since the adoption of the
Fourteenth Amendment, and by the common understanding of the Amendment
as evidenced by subsequent constitutional amendments and decisions of
this Court before Baker v. Carrmade an abrupt break with the past in
1962.

The failure of the Court to consider any of these matters cannot be
excused or explained by any concept of `developing' constitutionalism.
It is meaningless to speak of constitutional `development' when both the
language and history of the controlling provisions of the Constitution
are wholly ignored. Since it can, I think, be shown beyond doubt that
state legislative apportionments, as such, are wholly free of
constitutional limitations, save such as may be imposed by the
Republican Form of Government Clause (Const., Art. IV, § 4), 4 the
Court's action now bringing them within the purview of the Fourteenth
Amendment amounts to nothing less than an exercise of the amending power
by this Court.

So far as the Federal Constitution is concerned, the complaints in these
cases should all have been dismissed below for failure to state a cause
of action, because what has been alleged or proved shows no violation of
any constitutional right.

Before proceeding to my argument it should be observed that nothing done
in Baker v. Carror in the two cases that followed in its wake, Gray v.
Sanders and Wesberry v. Sandersfrom which the Court quotes at some
length, forecloses the conclusion which I reach.

Baker decided only that claims such as those made here are within the
competence of the federal courts to adjudicate. Although the Court
stated as its conclusion that the allegations of a denial of equal
protection presented `a justiciable constitutional cause of action', it
is evident from the Court's opinion that it was concerned all but
exclusively with justiciability and gave no serious attention to the
question whether the Equal Protection Clause touches state legislative
apportionments Neither the opinion of the Court nor any of the
concurring opinions considered the relevant text of the Fourteenth
Amendm nt or any of the historical materials bearing on that question.
None of the materials was briefed or otherwise brought to the Court's
attention.

In the Gray case the Court expressly laid aside the applicability to
state legislative apportionments of the `one person, one vote' theory
there found to require the striking down of the Georgia county unit
system. See 372 U.S., and the concurring opinion of STEWART, J., joined
by CLARK, J., U.S.---382, In Wesberry, involving congressional
districting, the decision rested on Art. I, § 2, of the Constitution.
The Court expressly did not reach the arguments put forward concerning
the Equal Protection Clause. See 376 U.S., note 10,

Thus it seems abundantly clear that the Court is entirely free to deal
with the cases presently before it in light of materials now called to
its attention for the first time. To these I now turn.

A. The Language of the Fourteenth Amendment.

The Court relies exclusively on that portion of § 1 of the Fourteenth
Amendment which provides that no State shall `deny to any person within
its jurisdiction the equal protection of the laws,' and disregards
entirely the significance of § 2, which reads:

`Representatives shall be apportioned among the several States according
to their respective numbers, counting the whole number of persons in
each State, excluding Indians not taxed. But when the right to vote at
any election for the choice of electors for President and Vice President
of the United States, Representatives in Congress, the Executive and
Judicial officers of a State, or the members of the Legislature thereof,
is denied to any of the male inhabitants of such State, being twenty-one
years of age, and citizens of the United States, or in any way abridged,
except for participation in rebellion, or other crime, the basis of
representation therein shall be reduced in the proportion which the
number of such male citizens shall bear to the whole number of male
citizens twenty-one years of age in such State.' (Emphasis added.)

The Amendment is a single text. It was introduced and discussed as such
in the Reconstruction Committee,7 which reported it to the Congress. It
was discussed as a unit in Congress and proposed as a unit to the
States,8 which ratified it as a unit. A proposal to split up the
Amendment and submit each section to the States as a separate amendment
was rejected by the Senate Whatever one might take to be the application
of these cases of the Equal Protection Clause if it stood alone, I am
unable to understand the Court's utter disregard of the second section
which expressly recognizes the States' power to deny `or in any way'
abridge the right of their inhabitants to vote for `the members of the
(State) Legislature,' and its express provision of a remedy for such
denial or abridgment. The comprehensive scope of the second section and
its particular reference to the state legislatures preclude the sugg
stion that the first section was intended to have the result reached by
the Court today. If indeed the words of the Fourteenth Amendment speak
for themselves, as the majority's disregard of history seems to imply,
they speak as clearly as may be against the construction which the
majority puts on them. But we are not limited to the language of the
Amendment itself.

B. Proposal and Ratification of the Amendment.

The history of the adoption of the Fourteenth Amendment provides
conclusive evidence that neither those who proposed nor those who
ratified the Amendment believed that the Equal Protection Clause limited
the power of the States to apportion their legislatures as they saw fit.
Moreover, the history demonstrate that the intention to leave this power
undisturbed was deliberate and was widely believed to be essential to
the adoption of the Amendment.

\begin{enumerate}
\def\labelenumi{(\roman{enumi})}
\tightlist
\item
  Proposal of the amendment in Congress.---A resolution proposing what
  became the Fourteenth Amendment was reported to both houses of
  Congress by the Reconstruction Committee of Fifteen on April 30, 1866
  The first two sections of the proposed amendment read: shall bear to
  the whole number of malecitizens not less than twenty-one years of
  age.'
\end{enumerate}

In the House, Thaddeus Stevens introduced debate on the resolution on
May 8. In his opening remarks, Stevens explained why he supported the
resolution although it fell `far short' of his wishes:

`I believe it is all that can be obtained in the present state of public
opinion. Not only Congress but the several States are to be consulted.
Upon a careful survey of the whole ground, we did not believe that
nineteen of the loyal States could be induced to ratify and proposition
more stringent than this.'

In explanation of this belief, he asked the House to remember `that
three months since, and more, the committee reported and the House
adopted a proposed amendment fixing the basis of representation in such
way as would surely have secured the enfranchisement of every citizen at
no distant period,' but that proposal had been rejected by the Senate.

He then ex lained the impact of the first section of the proposed
Amendment, particularly the Equal Protection Clause.

`This amendment allows Congress to correct the unjust legislation of the
States, so far that the law which operates upon one man shall operate
equally upon all. Whatever law punishes a white man for a crime shall
punish the black man precisely in the same way and to the same degree.
Whatever law protects the white man shall afford 'equal' protection to
the black man. Whatever means of redress is afforded to one shall be
afforded to all. Whatever law allows the white man to testify in court
shall allow the man of color to do the same. These are great advantages
over their present codes. Now different degrees of punishment are
inflicted, not on account of the magnitude of the crime, but according
to the color of the skin. Now color disqualifies a man from testifying
in courts, or being tried in the same way as white men. I need not
enumerate these partial and oppressive laws. Unless the Constitution
should restrain them those States will all, I fear, keep up this
discrimination, and crush to death the hated freedmen.'

He turned next to the second section, which he said he considered 'the
most important in the article.'15 Its effect, he said, was to fix 'the
basis of representation in Congress.'16 In unmistakable terms, he
recognized the power of a State to withhold the right to vote:

`If any State shall exclude any of her adult male citizens from the
elective franchise, or abridge that right, she shall forfeit her right
to representation in the same proportion. The effect of this provision
will be either to compel the States to grant universal suffrage or so to
shear them of their power as to keep them forever in a hopeless minority
in the national Government, both legislative and executive.'

Closing his discussion of the second section, he noted his dislike for
the fact that it allowed `the States to discriminate (with respect to
the right to vote) among the same class, and receive proportionate
credit inrepresentation.'

Toward the end of the debate three days later, Mr.~Bingham, the author
of the first section in the Reconstruction Committee and its leading
proponent,19 concluded his discussion of it with the following:

'Allow me, Mr.~Speaker, in passing, to say that this amendment takes
from no State any right that ever pertained to it. No State ever had the
right, under the forms of law or otherwise, to deny to any freeman the
equal protection of the laws or to abridge the privileges or immunities
of any citizen of the Republic, although many of them have assumed and
exercised the power, and that without remedy. The amendment does not
give, as the second section shows, the power to Congress of regulating
suffrage in the several States.'20 (Emphasis added.)

He immediately continued:

'The second section excludes the conclusion that by the first section
suffrage is subjected to congressional law; save, indeed, with this
exception, that as the right in the people of each State to a republican
government and to choose their Representatives in Congress is of the
guarantees of the Constitution, by this amendment a remedy might be
given directly for a case supposed by Madison, where treason might
change a State government from a republican to a despotic government,
and thereby deny suffrage to the people.'21 (Emphasis added.)

He stated at another point in his remarks:

'To be sure we all agree, and the great body of the peopl of this
country agree, and the committee thus far in reporting measures of
reconstruction agree, that the exercise of the elective franchise,
though it be one of the privileges of a citizen of the Republic, is
exclusively under the control of the States.'22 (Emphasis added.)

In the three days of debate which separate the opening and closing
remarks, both made by members of the Reconstruction Committee, every
speaker on the resolution, with a single doubtful exception,23 assumed
without question that, as Mr.~Bingham said'the second section excludes
the conclusion that by the first section suffrage is subjected to
congressional law.' The assumption was neither inadvertent nor silent.
Much of the debate concerned the change in the basis of representation
effected by the second section, and the speakers stated repeatedly, in
express terms or by unmistakable implication, that the States retained
the power to regulate suffrage within their borders. Attached as
Appendix A hereto are some of those statements. The resolution was
adopted by the House without change on May 10.

Debate in the Senate began on May 23, and followed the same pattern.
Speaking for the Senate Chairman of the Reconstruction Committee, who
was ill, Senator Howard, also a member of the Committee, explained the
meaning of the Equal Protection Clause as follows:

'The last two clauses of the first section of the amendment disable a
State from depriving not merely a citizen of the United States, but any
person, whoever he may be, of life, liberty, or property without due
process of law, or from denying to him the equal protection of the laws
of the State. This abolishes all class legislation in the States and
does away with the injustice of subjecting one caste of persons to a
code not applicable to another. It prohibits the hanging of a black man
for a crime for which the white man is not to be hanged. It protects the
black man in his fundamental rights as a citizen with the same shield
which it throws over the white man. Is it not time, Mr.~President, that
we extend to the black man, I had almost called it the poor privilege of
the equal protection of the law?

'But, sir, the first section of the proposed amendment does not give to
either of these classes the right of voting. The right of suffrage is
not, in law, one of the privileges or immunities thus secured by the
Constitution. It is merely the creature of law. It has always been
regarded in this country as the result of positive local law, not
regarded as one of those fundamental rights lying at the basis of all
society and without which a people cannot exist except as slaves,
subject to a depotism (sic).'25 (Emphasis added.)

Discussing the second section, he expressed his regret that it did 'not
recognize the authority of the United States over the question of
suffrage in the several States at all * * *.'26 He justified the limited
purpose of the Amendment in this regard as follows:

'But, sir, it is not the question here what will we do; it is not the
question what you, or I, or half a dozen other members of the Senate may
prefer in respect to colored suffrage; it is not entirely the question
what measure we can pass through the two Houses; but the question really
is, what will the Legislatures of the various States to whom these
amendments are to be submitted do in the premises; what is it likely
will meet the general approbation of the people who are to elect the
Legislatures, three fourths of whom must ratify our propositions before
they have the force of constitutional provisions?

'The committee were of opinion that the States are not ye prepared to
sanction so fundamental a change as would be the concession of the right
of suffrage to the colored race. We may as well state it plainly and
fairly, so that there shall be no misunderstanding on the subject. It
was our opinion that three fourths of the States of this Union could not
be induced to vote to grant the right of suffrage, even in any degree or
under any restriction, to the colored race.

'The second section leaves the right to regulate the elective franchise
still with the States, and does not meddle with that right.'27 (Emphasis
added.)

There was not in the Senate, as there had been in the House, a closing
speech in explanation of the Amendment. But because the Senate
considered, and finally adopted, several changes in the first and second
sections, even more attention was given to the problem of voting rights
there than had been given in the House. In the Senate, it was fully
understood by everyone that neither the first nor the second section
interfered with the right of the States to regulate the elective
franchise. Attached as Appendix B hereto are representative statements
from the debates to that effect. After having changed the proposed
amendment to the form in which it was adopted, the Senate passed the
resolution on June 8, 1866 As changed, it passed in the House on June
13.

\begin{enumerate}
\def\labelenumi{(\roman{enumi})}
\setcounter{enumi}{1}
\tightlist
\item
  Ratification by the `loyal' States.---Reports of the debates in the
  state legislatures on the ratification of the Fourteenth Amendment are
  not generally available There is, however, compelling indirect
  evidence. Of the 23 loyal States which ratified the Amendment before
  1870, five had constitutional provisions for apportionment of at least
  one house of their respective legislatures which wholly disregarded
  the spread of population.
\end{enumerate}

Ten more had constitutional provisions which gave primary emphasis to
population, but which applied also other principles, such as partial
ratios and recognition of political subdivisions, which were intended to
favor sparsely settled areas. 32 Can it be seriously contended that the
legislatures of these States, almost two-thirds of those concerned,
would have ratified an amendment which might render their own States'
constitutions unconstitutional?

Nor were these state constitutional provisions merely theoretical. In
New Jersey, for example, Cape May County, with a population of 8,349,
and Ocean County, with a population of 13,628, each elected one State
Senator, as did Essex and Hudson Counties, with populations of 143,839
and 129,067, respectively In the House, each county was entitled to one
representative, which left 39 seats to be apportioned according to
population. 34 Since there were 12 counties besides the two already
mentioned which had populations over 30,000,35 it is evident that there
were serious disproportions in the House also. In

New York, each of the 60 counties except Hamilton County was entitled to
one of the 128 seats in the Assembly This left 69 seats to be
distributed among counties the populations of which ranged from 15,420
to 942,292 With seven more counties having populations over 100,000 and
13 others having populations over 50,000,38 the disproportion in the
Assembly was necessarily large. In Vermont, after each county had been
allocated one Senator, there were 16 seats remaining to be distributed
among the larger counties The smallest county had a population of 4,082;
the largest had a population of 40,651 and there were 10 other counties
with populations over 20,000.

\begin{enumerate}
\def\labelenumi{(\roman{enumi})}
\setcounter{enumi}{2}
\tightlist
\item
  Ratification by the `reconstructed' States.---Each of the 10
  `reconstructed' States was required to ratify the Fourteenth Amendment
  before it was readmitted to the Union The Constitution of each was
  scrutinized in Congress Debates over readmission were extensive In at
  least one instance, the problem of state legislative apportionment was
  expressly called to the attention of Congress. Objecting to the
  inclusion of Florida in the Act of June 25, 1868, Mr.~Farnsworth
  stated on the floor of the House: to a representative in the
  Legislature; while the populous counties are entitled to only one
  representative each, with an additional representative for every
  thousand inhabitants.'
\end{enumerate}

The response of Mr.~Butler is particularly illuminating:

`All these arguments, all these statements, all the provisions of this
constitution have been submitted to the Judiciary Committee of the
Senate, and they have found the constitution republican and proper. This
constitution has been submitted to the Senate, and they have found it
republican and proper. It has been submitted to your own Committee on
Reconstruction, and they have found it republican and proper, and have
reported it to this House.'

The Constitutions of six of the 10 States Contained provisions departing
substantially from the method of apportionment now held to be required
by the Amendment And, as in the North, the departures were as real in
fact as in theory. In North Carolina, 90 of the 120 representatives were
apportioned among the counties without regard to population, leaving 30
seats to be distributed by numbers Since there were seven counties with
populations under 5,000 and 26 counties with populations over 15,000,
the disproportions must have been widespread and substantial In South
Carolina, Charleston, with a population of 88,863, elected two Senators;
each of the other counties, with populations ranging from 10,269 to Page

42,486, elected one Senator In Florida, each of the 39 counties was
entitled to elect one Representative; no county was entitled to more
than four These principles applied to Dade County, with a population of
85, and to Alachua County and Leon County, with populations of 17,328
and 15,236, respectively.

It is incredible that Congress would have exacted ratification of the
Fourteenth Amendment as the price of readmission, would have studied the
State Constitutions for compliance with the Amendment, and would then
have disregarded violations of it.

The facts recited above show beyond any possible doubt:

`(1) that Congress, with full awareness of and attention to the
possibility that the States would not afford full equality in voting
rights to all their citizens, nevertheless deliberately chose not to
interfere with the States' plenary power in this regard when it proposed
the Fourteenth Amendment;

\begin{enumerate}
\def\labelenumi{(\arabic{enumi})}
\setcounter{enumi}{1}
\item
  that Congress did not include in the Fourteenth Amendment restrictions
  on the States' power to control voting rights because it believed that
  if such restrictions were included, the Amendment would not be
  adopted; and
\item
  that at least a substantial majority, if not all, of the States which
  ratified the Fourteenth Amendment did not consider that in so doing,
  they were accepting limitations on their freedom, never before
  questioned, to regulate voting rights as they chose.
\end{enumerate}

Even if one were to accept the majority's belief that it is proper
entirely to disregard the unmistakable implica- tions of the second
section of the Amendment in construing the first section, one is
confounded by its disregard of all this history. There is here none of
the difficulty which may attend the application of basic principles to
situations not contemplated or understood when the principles were
framed. The problems which concern the Court now were problems when the
Amendment was adopted. By the deliberate choice of those responsible for
the Amendment, it left those problems untouched.

C. After 1868.

The years following 1868, far from indicating a developing awareness of
the applicability of the Fourteenth Amendment to problems of
apportionment, demonstrate precisely the reverse: that the States
retained and exercised the power independently to apportion their
legislatures. In its Constitutions of 1875 and 1901, Alabama carried
forward earlier provisions guaranteeing each county at least one
representative and fixing an upper limit to the number of seats in the
House Florida's Constitution of 1885 continued the guarantee of one
representative for each county and reduced the maximum number of
representatives per county from four to three Georgia, in 1877,
continued to favor the smaller counties Louisiana, in 1879, guaranteed
each parish at least one representative in the House In 1890,
Mississippi guaranteed each county one representative, established a
maximum number of representatives, and provided that specified groups of
counties should each have approximately one-third of the seats in the
House, what- ever the spread of population Missouri's

Constitution of 1875 gave each county one representative and otherwise
favored less populous areas Montana's original Constitution of 1889
apportioned the State Senate by counties In 1877, New Hampshire amended
its Constitution's provisions for apportionment, but continued to favor
sparsely settled areas in the House and to apportion seats in the Senate
according to direct taxes paid the same was true of New Hampshire's
Constitution of 1902.

In 1894, New York adopted a Constitution the peculiar apportionment
provisions of which were obviously intended to prevent representation
according to population: no county was allowed to have more than
one-third of all the Senators, no two countries which were adjoining or
`separated only by public waters' could have more than one-half of all
the Senators, and whenever any county became entitled to mo e than three
Senators, the total number of Senators was increased, thus preserving to
the small counties their original number of seats In addition, each
county except Hamilton was guaranteed a seat in the Assembly The North
Carolina Constitution of 1876 gave each county at least one
representative and fixed a maximum number of representatives for the
whole House Oklahoma's Constitution at the time of its admission to the
Union (1907) favored small counties by the use of partial ratios and a
maximum number of seats in the House; in addition, no county was
permitted to `take part' in the election of more than seven
representatives Pennsylvania, in 1873, continued to guarantee each
county one representative in the House The same was true of South
Carolina's Constitution of 1895, which provided also that each county
should elect one and only one Senator Utah's original Constitution of
1895 assured each county of one representative in the House Wyoming,
when it entered the Union in 1889, guaranteed each county at least one
Senator and one representative.

D. Today.

Since the Court now invalidates the legislative apportionments in six
States, and has so far upheld the apportionment in none, it is scarcely
necessary to comment on the situation in the States today, which is of
course, as fully contrary to the Court's decision as is the record of
every prior period in this Nation's history. As of 1961, the
Constitutions of all but 11 States, roughly 20\% of the total,
recognized bases of apportionment other than geographic spread of
population, and to some extent favored sparsely populated areas by a
variety of devices, ranging from straight area representation or
guaranteed minimum area representation to complicated schemes of the
kind exemplified by the provisions of New York's Constitution of 1894,
still in effect until struck down by the Court today in No.~20, 377
U.S., p.~633, 84 S.Ct., p.~633 Since

Tennessee, which was the subject of Baker v. Carr, and Virginia,
scrutinized and disapproved today in No.~69, 377 U.S., p.~678, 84 S.Ct.,
p.~1441, are among the 11 States whose own Constitutions are sound from
the standpoint of the Federal Constitution as construed today, it is
evident that the actual practice of the States is even more uniformly
than their theory opposed to the Court's view of what is
constitutionally permissible.

E. Other Factors.

In this summary of what the majority ignores, note should be taken of
the Fifteenth and Nineteenth Amendments. The former prohibited the
States from denying or abridging the right to vote `on account of race,
color, or previous condition of servitude.' The latter, certified as
part of the Constitution in 1920, added sex to the prohibited
classifications. In Minor v. Happersett, 21 Wall. 162, this Court
considered the claim that the right of women to vote was protected by
the Privileges and Immunities Clause of the Fourteenth Amendment. The
Court's discussion there of the significance of the Fifteenth Amendment
is fully applicable here with respect to the Nineteenth Amendment as
well.

`And still again, after the adoption of the fourteenth amendment, it was
deemed necessary to adopt a fifteenth, as follows: 'The ight of citizens
of the United States to vote shall not be denied or abridged by the
United States, or by any State, on account of race, color, or previous
condition of servitude.' The fourteenth amendment had already provided
that no State should make or enforce any law which should abridge the
privileges or immunities of citizens of the United States. If suffirage
was one of these privileges or immunities, why amend the Constitution to
prevent its being denied on account of race, \&c.? Nothing is more
evident than that the greater must include the less, and if all were
already protected why go through with the form of amending the
Constitution to protect a part?' Wall..

In the present case, we can go still further. If constitutional
amendment was the only means by which all men and, later, women, could
be guaranteed the right to vote at all, even for federal officers, how
can it be that the far less obvious right to a particular kind of
apportionment of state legislatures a right to which is opposed a far
more plausible conflicting interest of the State than the interest which
opposes the general right to vote---can be conferred by judicial
construction of the Fourteenth Amendment? 70 Yet, unless one takes the
highly implausible view that the Fourteenth Amendment controls methods
of apportionment but leaves the right to vote itself suprotected, the
conclusion is inescapable that the Court has, for purposes of these
cases, relegated the Fifteenth and Nineteenth Amendments to the same
limbo of constitutional anachronisms to which the second section of the
Fourteenth Amendment has been assigned.

Mention should be made finally of the decisions of this Court which are
disregarded or, more accurately, silently overruled today. Minor v.
Happersettin which the Court held that the Fourteenth Amendment did not
confer the right to vote on anyone, has already been noted. Other cases
are more directly in point. In Colegrove v. Barrettthis Court dismissed
`for want of a substantial federal question' an appeal from the
dismissal of a complaint alleging that the Illinois legislative
apportionment resulted in `gross inequality in voting power' and `gross
and arbitrary and atrocious discrimination in voting' which denied the
plaintiffs equal protection of the laws In Remmey v. Smith, 102 F.Supp.
708 (D.C.E.D.Pa.), a three-judge District Court dismissed a complaint
alleging that the apportionment of the Pennsylvania Legislature deprived
the plaintiffs of `constitutional rights guaranteed to them by the
Fourteenth Amendment'. F.Supp.. The District Court stated that it was
aware that the plaintiffs' allegations were `notoriously true' and that
`(t)he practical disenfranchisement of qualified electors in certain of
the election districts in Philadelphia County is a matter of common
knowledge.' F.Supp.. This Court dismissed the appeal `for the want of a
substantial federal question.'

In Kidd v. McCanless, 200 Tenn. 273, 292 S.W d 40, the Supreme Court of
Tennessee dismissed an action for a declaratory judgment that the
Tennessee Apportionment Act of 1901 was unconstitutional. The complaint
alleged that `a minority of approximately 37\% of the voting population
of the State now elects and controls 20 of the 33 members of the Senate;
that a minority of 40\% of the voting population of the State now
controls 63 of the 99 members of the House of Representatives.' 292 S.W
d.~Without dissent, this Court granted the motion to dismiss the appeal.
352 U.S. 920, In Radford v. Gary, 145 F.Supp. 541 (D.C.W.D.Okla.), a
three-judge District Court was convened to consider `the complaint of
the plaintiff to the effect that the existing apportionment statutes of
the State of Oklahoma violate the plain mandate of the Oklahoma
Constitution and operate to deprive him of the equal protection of the
laws guaranteed by the Fourteenth Amendment to the Constitution of the
United States.' F.Supp.. The plaintiff alleged that he was a resident
and voter in the most populous county of the State, which had about 15\%
of the total population of the State but only about 2\% of the seats in
the State Senate and less than 4\% of the seats in the House. The
complaint recited the unwillingness or inability of the branches of the
state government to provide relief and alleged that there was no state
remedy available. The District Court granted a motion to dismiss. This
Court affirmed without dissent.

Each of these recent cases is distinguished on some ground or other in
Baker v. Carr. See 369 U.S.S.Ct. 720. Their summary dispositions prevent
consideration whether these after-the-fact distinctions are real or
imaginary. The fact remains, however, that between 1947 and 1957, four
cases raising issues precisely the same as those decided today were
presented to the Court. Three were dismissed because the issues
presented were thought insubstantial and in the fourth the lower court's
dismissal was affirmed.

I have tried to make the catalogue complete, yet to keep it within the
manageable limits of a judicial opinion. In my judgment, today's
decisions are refuted by the language of the Amendment which they
construe and by the inference fairly to be drawn from subsequently
enacted Amendments. They are unequivocally refuted by history and by
consistent theory and practice from the time of the adop ion of the
Fourteenth Amendment until today.

The Court's elaboration of its new `constitutional' doctrine indicates
how far---and how unwisely---it has strayed from the appropriate bounds
of its authority. The consequence of today's decision is that in all but
the handful of States which may already satisfy the new requirements the
local District Court or, it may be, the state courts, are given blanket
authority and the constitutional duty to supervise apportionment of the
State Legislatures. It is difficult to imagine a more intolerable and
inappropriate interference by the judiciary with the independent
legislatures of the States.

In the Alabama cases (Nos. 23, 27, 41), the District Court held invalid
not only existing provisions of the State Constitution---which this
Court lightly dismisses with a wave of the Supremacy Clause and the
remark that `it makes no difference whether a State's apportionment
scheme is embodied in its constitution or in statutory provisions,'
ante, p.~584 but also a proposed amendment to the Alabama Constitution
which had never been submitted to the voters of Alabama for
ratification, and `standby' legislation which was not to become
effective unless the amendment was rejected (or declared
unconstitutional) and in no event before 1966. Sims v. Frink, D.C., 208
F.Supp. 431. See ante, pp.~543---551. Both of these measures had been
adopted only nine days before,73 at an Extraordinary Session of the
Alabama Legislature, convened pursuant to what was very nearly a
directive of the District Court, see Sims v. Frink, D.C., 205 F.Supp.
245, 248. The District Court formulated its own plan for the
apportionment of the Alabama Legislature, by picking and choosing among
the provisions of the legislative measures. 208 F.Supp.---442. See ante,
p.~552. Beyond that, the court warned the legislature that there would
be still further judicial reapportionment unless the legislature, like
it or not, undertook the task for itself. 208 F.Supp.. This Court now
states that the District Court acted in `a most proper and commendable
manner,' ante, p.~586, and approves the District Court's avowed
intention of taking `some further action' unless the State Legislature
acts by 1966, ante, p.~587.

In the Maryland case (No.~29S.Ct. 1429), the State Legislature was
called into Special Session and enacted a temporary reapportionment of
the House of Delegates, under pressure from the state courts Thereafter,
the

Maryland Court of Appeals held that the Maryland Senate was
constitutionally apportioned. Maryland Committee for Fair Representation
v. Tawes, 229 Md. 406, 184 A d 715. This Court now holds that neither
branch of the State Legislature meets constitutional requirements. 377
U.S., p.~674, 84 S.Ct., p.~1439. The Court presumes that since 'the
Maryland constitutional provisions relating to legislative apportionment
(are) hereby held unconstitutional, the Maryland Legislature

has the inherent power to enact at least temporary reapportionment
legislation pending adoption of state constitutional provisions' which
satisfy the Federal Constitution, U.S., On this premise, the Court
concludes that the Maryland courts need not `feel obliged to take
further affirmative action' now, but that `under no circumstances should
the 1966 election of members of the Maryland Legislature be permitted to
be conducted pursuant to the existing or any other unconstitutional
plan.' U.S., In the Virginia case (No.~69, 377 U.S., p.~678, 84 S.Ct.,
p.~1441), the State Legislature in 1962 complied with the state
constitutional requirement of regular reapportionment Two days later, a
complaint was filed in the District Court Eight months later, the
legislative reapportionment was declared unconstitutional. declared
unconstitutional. Mann v. Davis, D.C., 213 F.Supp. 577. The District
Court gave the State Legislature two months within which to reapportion
itself in special session, under penalty of being reapportioned by the
court Only a stay granted by a member of this Court slowed the process
it is plain that no stay will be forthcoming in the future. The Virginia
Legislature is to be given `an adequate opportunity to enact a valid
plan'; but if it fails `to act promptly in remedying the constitutional
defects in the State's legislative apportionment plan,' the District
Court is to `take further action.' 377 U.S. p.~693, 84 S.Ct. p.~1449.

In Delaware (No.~307S.Ct. 1449), the District Court entered an order on
July 25, 1962, which stayed proceedings until August 7, 1962, `in the
hope and expectation' that the General Assembly would take `some
appropriate action' in the intervening 13 days. Sincock v. Terry, 207
F.Supp. 205, 207. By way of prodding, presumably, the court noted that
if no legislative action were taken and the court sustained the
plaintiffs' claim, `the present General Assembly and any subsequent
General Assembly, the members of which were elected pursuant to Section
2 of Article 2 (the challenged provisions of the Delaware Constitution),
might be held not to be a de jure legislature and its legislative acts
might be held invalid and unconstitutional.' F.Supp.---206. Five days
later, on July 30, 1962, the General Assembly approved a proposed
amendment to the State Constitution. On August 7, 1962, the District
Court entered an order denying the defendants' motion to dismiss. The
court said that it did not wish to substitute its judgment `for the
collective wisdom of the General Assembly of Delaware', but that `in the
light of all the circumstances', it had to proceed promptly. 210 F.Supp.
395, 396. On October 16, 1962, the court declined to enjoin the conduct
of elections in November. 210 F.Supp. 396. The court went on to express
its regret that the General Assembly had not adopted the court's
suggestion, see 207 F.Supp.---207, that the Delaware Constitution be
amended to make apportionment a statutory rather than a constitutional
matter, so as to facilitate further changes in apportionment which might
be required. 210 F.Supp.. In January 1963, the General Assembly again
approved the proposed amendment of the apportionment provisions of the
Delaware Constitution, which thereby became effe tive on January 17,
1963 Three months later, on April 17, 1963, the District Court reached
`the reluctant conclusion' that Art. II, § 2, of the Delaware
Constitution was unconstitutional, with or without the 1963 amendment.
Sincock v. Duffy, D.C., 215 F.Supp. 169, 189. Observing that `(t)he
State of Delaware, the General Assembly, and this court all seem to be
trapped in a kind of box of time,' F.Supp., the court gave that General
Assembly until October 1, 1963, to adopt acceptable provisions for
apportionment. On May 20, 1963, the District Court enjoined the
defendants from conducting any elections, including the general election
scheduled for November 1964, pursuant to the old or the new
constitutional provisions This Court now approves all these proceedings,
noting particularly that in allowing the 1962 elections to go forward,
`the District Court acted in a wise and temperate manner.'

Records such as these in the cases decided today are sure to be
duplicated in most of the other States if they have not been already.
They present a jarring picture of courts threatening to take action in
an area which they have no business entering, inevitably on the basis of
political judgments which they are incompetent to make. They show
legislatures of the States meeting in haste and deliberating and
deciding in haste to avoid the threat of judicial interference. So far
as I can tell, the Court's only response to this unseemingly state of
affairs is ponderous insistence that `a denial of constitutionally
protected rights demands judicial protection,' ante, p.~566. By thus
refusing to recognize the bearing which a potential for conflict of this
kind may have on the question whether the claimed rights are in fact
constitutionally entitled to judicial protection, the Court assumes,
rather than supports, its conclusion.

It should by now be obvious that these cases do not mark the end of
reapportionment problems in the courts. Predictions once made that the
courts would never have to face the problem of actually working out an
apportionment have proved false. This Court, however, ontinues to avoid
the consequences of its decisions, simply assuring us that the lower
courts `can and will work out more concrete and specific standards,'
ante, p.~578. Deeming it `expedient' not to spell out `precise
constitutional tests,' the Court contents itself with stating `only a
few rather general considerations.'

Generalities cannot obscure the cold truth that cases of this type are
not amenable to the development of judicial standards. No set of
standards can guide a court which has to decide how many legislative
districts a State shall have, or what the shape of the districts shall
be, or where to draw a particular district line. No judicially
manageable standard can determine whether a State should have
single-member districts or multimember districts or some combination of
both. No such standard can control the balance between keeping up with
population shifts and having stable districts. In all these respects,
the courts will be called upon to make particular decisions with respect
to which a principle of equally populated districts will be of no
assistance whatsoever. Quite obviously, there are limitless
possibilities for districking consistent with such a principle. Nor can
these problems be avoided by judicial reliance on legislative judgments
so far as possible. Reshaping or combining one or two districts, or
modifying just a few district lines, is no less a matter of choosing
among many possible solutions, with varying political consequences, than
reapportionment broadside.

The Court ignores all this, saying only that `what is marginally
permissible in one State may be unsatisfactory in another, depending on
the particular circumstances of the case,' ante, p.~578. It is well to
remember that the product of today's decisions will not be readjustment
of a few districts in a few States which most glaringly depart from the
principle of equally populated districts. It will be a redetermination,
extensive in many cases, of legislative districts in all but a few
States.

Although the Court---necessarily, as I believe---provides only
generalities in elaboration of its main thesis, its opinion nevertheless
fully demonstrates how far removed these problems are from fields of
judicial competence. Recognizing that `indiscriminate districting' is an
invitation to `partisan gerrymandering,' ante, pp.~578-579, the Court
nevertheless excludes virtually every basis for the formation of
electoral districts other than `indiscriminate districting.' In one or
another of today's opinions, the Court declares it unconstitutional for
a State to give effective consideration to any of the following in
establishing legislative districts:

\begin{enumerate}
\def\labelenumi{\arabic{enumi}.}
\item
  history;
\item
  ``economic or other sorts of group interests'';{]}
\item
  area;
\item
  geographical considerations;
\item
  a desire ``to insure effective representation for sparsely settled
  areas''
\item
  ``availability of access of citizens to their representatives'';
\item
  theories of bicameralism (except those approved by the Court);
\item
  occupation;
\item
  ``an attempt to balance urban and rural power.''
\item
  the preference of a majority of voters in the state.
\end{enumerate}

So far as presently appears, the only factor which a State may consider,
apart from numbers, is political subdivisions. But even `a clearly
rational state policy' recognizing this factor is unconstitutional if
'population is submerged as the controlling consideration * * *.'

I now of no principle of logic or practical or theoretical politics,
still less any constitutional principle, which establishes all or any of
these exclusions. Certain it is that the Court's opinion does not
establish them. So far as the Court says anything at all on this score,
it says only that `legislators represent people, not trees or acres,'
ante, p.~1382; that `citizens, not history or economic interests, cast
votes,' ante, p.~580; that `people, not land or trees or pastures,
vote,' ibid All this may be conceded. But it is surely equally obvious,
and, in the context of elections, more meaningful to note that people
are not ciphers and that legislators can represent their electors only
by speak- ing for their interests---economic, social, political---many
of which do reflect the place where the electors live. The Court does
not establish, or indeed even attempt to make a case for the proposition
that conflicting interests within a State can only be adjusted by
disregarding them when voters are grouped for purposes of
representation.

With these cases the Court approaches the end of the third round set in
motion by the complaint filed in Baker v. Carr. What is done today
deepens my conviction that judicial entry into this realm is profoundly
ill-advised and constitutionally impermissible. As I have said before,
Wesberry v. Sanders U.S., I believe that the vitality of our political
system, on which in the last analysis all else depends, is weakened by
reliance on the judiciary for political reform; in time a complacent
body politic may result.

These decisions also cut deeply into the fabric of our federalism. What
must follow from them may eventually appear to be the product of state
legislatures. Nevertheless, no thinking person can fail to recognize
that the aftermath of these cases, however desirable it may be thought
in itself, will have been achieved at the cost of a radical alteration
in the relationship between the States and the Federal Government, more
particularly the Federal Judiciary. Only one who has an overbearing
impatience with the federal system and its political processes will
believe that that cost was not too high or was inevitable.

Finally, these decisions give support to a current mistaken view of the
Constitution and the constitutional function of this Court. This view,
in a nutshell, is that every major social ill in this country can find
its cure in some constitutional `principle,' and that this Court should
`take the lead' in promoting reform when other branches of government
fail to act. The Constitution is not a panacea for every blot upon the
public welfare, nor should this Court, ordained as a judicial body, the
though of as a general haven for reform movements. The Constitution is
an instrument of government, fundamental to which is the premise that in
a diffusion of governmental authority lies the greatest promise that
this Nation will realize liberty for all its citizens. This Court,
limited in function in accordance with that premise, does not serve its
high purpose when it exceeds its authority, even to satisfy justified
impatience with the slow workings of the political process. For when, in
the name of constitutional interpretation, the Court adds something to
the Constitution that was deliberately excluded from it, the Court in
reality substitutes its view of what should be so for the amending
process.

I dissent in each of these cases, believing that in none of them have
the plaintiffs stated a cause of action. To the extent that Baker v.
Carr, expressly or by implication, went beyond a discussion of
jurisdictional doctrines independent of the substantive issues involved
here, it hould be limited to what it in fact was: an experiment in
venturesome constitutionalism. I would reverse the judgments of the
District Courts in Nos. 23, 27, and 41 (Alabama), No.~69 (Virginia), and
No.~307 (Delaware), and remand with directions to dismiss the
complaints. I would affirm the judgments of the District Courts in
No.~20 (New York), and No.~508 (Colorado), and of the Court of Appeals
of Maryland in No.~29.

APPENDIX A TO OPINION OF MR. JUSTICE HARLAN, DISSENTING.

Statements made in the House of Representatives during the debate on the
resolution proposing the Fourteenth Amendment.

`As the nearest approach to justice which we are likely to be able to
make, I approve of the second section that bases representation upon
voters.' 2463 (Mr.~Garfield).

`Would it not be a most unprecedented thing that when this (former
slave) population are not permitted where they reside to enter into the
basis of representation in their own State, we should receive it as an
element of representation here; that when they will not count them in
apportioning their own legislative districts, we are to count them as
five fifths (no longer as three fifths, for that is out of the question)
as soon as you make a new apportionment?' 2464---2465 (Mr.~Thayer).

`The second section of the amendment is ostensibly intended to remedy a
supposed inequality in the basis of representation. The real object is
to reduce the number of southern representatives in Congress and in the
Electoral College; and also to operate as a standing inducement to negro
suffrage.' 2467 (Mr.~Boyer).

`Shall the pardoned rebels of the South include in the basis of
representation four million people to whom they deny political rights,
and to no one of whom is allowed a vote in the selection of a
Representative?' 2468 (Mr.~Kelley).

`I shall, Mr.~Speaker, vote for this amendment; not because I approve
it. Could I have controlled the report of the committee of fifteen, it
would have proposed to give the right of suffrage to every loyal man in
the country.' 2469 (Mr.~Kelley).

'But I will ask, why should not the representation of the States be
limited as the States themselves limit suffrage?

If the negroes of the South are not to be counted as a political element
in the government of the South in the States, why should they be counted
as a political element in the government of the country in the Union?'
2498 (Mr.~Broomall).

`It is now proposed to base representation upon suffrage, upon the
number of voters, instead of upon the aggregate population in every
State of the Union.' 2502 (Mr.~Raymond).

`We admit equality of representation based upon the exercise of the
elective franchise by the people. The proposition in the matter of
suffrage falls short of what I desire, but so far as it goes it tends to
the equalization of the inequality at present existing; and while I
demand and shall continue to demand the franchise for all loyal male
citizens of this country---and I cannot but admit the possibility that
ultimately those eleven States may be restored to representative power
without the right of franchise being conferred upon the colored
people---I should feel myself doubly humiliated and disgraced, and
criminal even, if I hesitated to do what I can for a proposition which
equalizes representation.' 2508 (Mr.~Boutwell).

`Now, conceding to each State the right to regulate the right of
suffrage, they ought not to have a representation for male citizens not
less than twenty-one years of age, whether white or black, who are
deprived of the exercise of suffrage. This amendment will settle the
complication in regard to suffrage and representation, leaving each
State to regulate that for itself, so that it will be for it to decide
whether or not it shall have a representation for all its male citizens
not less than twenty-one years of age.' 2510 (Mr.~Miller).

`Manifestly no State should have its basis of nat onal representation
enlarged by reason of a portion of citizens within its borders to which
the elective franchise is denied. If political power shall be lost
because of such denial, not imposed because of participation in
rebellion or other crime, it is to be hoped that political interests may
work in the line of justice, and that the end will be the impartial
enfranchisement of all citizens not disqualified by crime. Whether that
end shall be attained or not, this will be secured: that the measure of
political power of any State shall be determined by that portion of its
citizens which can speak and act at the polls, and shall not be enlarged
because of the residence within the State of portions of its citizens
denied the right of franchise. So much for the second section of the
amendment. It is not all that I wish and would demand; but odious
inequalities are removed by it and representation will be equalized, and
the political rights of all citizens will under its operation be, as we
believe, unlitimately recognized and admitted.' 2511 (Mr.~Eliot).

`I have no doubt that the Government of the United States has full power
to extend the elective franchise to the colored population of the
insurgent States. I mean authority; I said power. I have no doubt that
the Government of the United States has authority to do this under the
Constitution; but I do not think they have the power. The distinction I
make between authority and power is this: we have, in the nature of our
Government, the right to do it; but the public opinion of the country is
such at this precise moment as to make it impossible we should do it. It
was therefore most wise on the part of the committee on reconstruction
to waive this matter in deference to public opinion. The situa- tion of
opinion in these States compels us to look to other means to protect the
Government against the enemy.' 2532 (Mr.~Banks).

`If you deny to any portion of the loyal citizens of your State the
right to vote for Representatives you shall not assume to represent
them, and, as you have done for so long a time, misrepresent and oppress
them. This is a step in the right direction; and although I should
prefer to see incorporated into the Constitution a guarantee of
universal suffrage, as we cannot get the required two thirds for that, I
cordially support this proposition as the next best.' 2539 2540
(Mr.~Farnsworth).

APPENDIX B TO OPINION OF MR. JUSTICE HARLAN, DISSENTING.

Statements made in the Senate during the debate on the resolution
proposing the Fourteenth Amendment.*

`The second section of the constitutional amendment proposed by the
committee can be justified upon no other theory than that the negroes
ought to vote; and negro suffrage must be vindicated before the people
in sustaining that section, for it does not exclude the non-voting
population of the North, because it is admitted that there is no wrong
in excluding from suffrage aliens, females, and minors. But we say, if
the negro is excluded from suffrage he shall also be excluded from the
basis of representation. Why this inequality? Why this injustice? For
injustice it would be unless there be some good reason for this
discrimination against the South in excluding her non-voting population
from the basis of representation. The only defense that we can make to
this apparent injustice is that the South commits an outrage upon human
rights when she denies the ballot to the blacks, and we will not allow
her to take advantage of her own wrong or profit by this outrage. Does
any one suppose it possible to avoid this plain issue before the people?
For if they will sustain you in reducing the representation of the South
because she does not allow the negro to vote, they will do so because
they think it is wrong to disfranchise him.' 2800 (Senator Stewart).

`It (the second section of the proposed amendment) relieves him (the
Negro) from misrepresentation in Congress by denying him any represe
tation whatever.' 2801 (Senator Stewart).

`But I will again venture the opinion that it (the second section) means
as if it read thus: no State shall be allowed a representation on a
colored population unless the right of voting is given to the
negroes---presenting to the States the alternative of loss of
representation or the enfranchisement of the negroes, and their
political equality.' 2939 (Senator Hendricks).

'I should be much better satisfied if the right of suffrage had been
given at once to the more intelligent of them (the Negroes) and such as
had served in our Army. But it is believed by wiser ones than myself
that this amendment will very soon produce some grant of suffrage to
them, and that the craving for political power will ere long give them
universal suffrage.

Believing that this amendment probably goes as far in favor of suffrage
to the negro as is practicable to accomplish now, and hoping it may in
the end accomplish all I desire in this respect, I shall vote for its
adoption, although I should be glad to go further.' 2963---2964 (Senator
Poland). `What is to be the operation of this amendment? Just this: your
whip is held over Pennsylvania, and you say to her that she must either
allow her negroes to vote or have one member of Congress less.' 2987
(Senator Cowan).

`Now, sir, in all the States---certainly in mine, and no doubt in
all---there are local as contradistinguished from State elections. There
are city elections, county elections, and district or borough elections;
and those city and county and district elections are held under some law
of the State in which the city or county or district or borough may be;
and in those elections, according to the laws of the States, certain
qualifications are prescribed, residence within the limits of the
locality and a property qualification in some. Now, is it proposed to
say that if every man in a State is not at liberty to vote at a city or
a country or a borough election that is to affect the basis of
representation?' 2991 (Senator Johnson).

'Again, Mr.~President, the measure upon the table, like the first
proposition submitted to the Senate from the committee of fifteen,
concedes to the States not only the right, but the exclusive right, to
regulate the franchise.

It says that each of the southern States, and, of course, each other
State in the Union, has a right to regulate for itself the franchise,
and that consequently, as far as the Government of the United States is
concerned, if the black man is not permitted the right to the franchise,
it will be a wrong (if a wrong) which the Govern- ment of the United
States will be impotent to redress.' 3027 (Senator Johnson).

`The amendment fixes representation upon numbers, precisely as the
Constitution new does, but when a State denies or abridges the elective
franchise to any of its male inhabitants who are citizens of the United
States and not less than twenty-one years of age, except for
participation in rebellion or other crime, then such State will lose its
representation in Congress in the proportion which the male citizen so
excluded bears to the whole number of male citizens not less than
twenty-one years of age in the State.' 3033 (Senator Henderson).

\hypertarget{from-due-process-to-equal-protection}{%
\section{From Due Process to Equal
Protection}\label{from-due-process-to-equal-protection}}

\hypertarget{the-constitution-of-the-united-states-is-it-pro-slavery-or-anti-slavery}{%
\subsubsection{The Constitution of the United States: Is It Pro-Slavery
or
Anti-Slavery?}\label{the-constitution-of-the-united-states-is-it-pro-slavery-or-anti-slavery}}

\emph{``The Constitution of the United States: Is It Pro-Slavery or
Anti-Slavery'' is a speech by Frederick Douglass delivered in Glasgow in
1860. The text below is copied from Philip Foner, ed., Frederick
Douglass: Selected Speeches and Writings (Lawrence Hill Books 1999).}

I proceed to the discussion. And first a word about the question. Much
will be gained at the outset if we fully and clearly understand the real
question under discussion. Indeed, nothing is or can be understood till
this is understood. Things are often confounded and treated as the same,
for no better reason than that they resemble each other, even while they
are in their nature and character totally distinct and even directly
opposed to each other. This jumbling up things is a sort of
dust-throwing which is often indulged in by small men who argue for
victory rather than for truth. Thus, for instance, the American
Government and the American Constitution are spoken of in a manner which
would naturally lead the hearer to believe that the one is identical
with the other; when the truth is, they are as distinct in character as
is a ship and a compass. The one may point right and the other steer
wrong. A chart is one thing, the course of the vessel is another. The
Constitution may be right, the Government wrong. If the Government has
been governed by mean, sordid, and wicked passions, it does not follow
that the Constitution is mean, sordid, and wicked. What, then, is the
question? I will state it. But first let me state what is not the
question. It is not whether slavery existed in the United States at the
time of the adoption of the Constitution; it is not whether slaveholders
took part in framing the Constitution; it is not whether those
slaveholders, in their hearts, intended to secure certain advantages in
that instrument for slavery; it is not whether the American Government
has been wielded during seventy-two years in favour of the propagation
and permanence of slavery; it is not whether a pro-slavery
interpretation has been put upon the Constitution by the American Courts
-- all these points may be true or they may be false, they may be
accepted or they may be rejected, without in any wise affecting the real
question in debate. The real and exact question between myself and the
class of persons represented by the speech at the City Hall may be
fairly stated thus: -- 1st, Does the United States Constitution
guarantee to any class or description of people in that country the
right to enslave, or hold as property, any other class or description of
people in that country? 2nd, Is the dissolution of the union between the
slave and free States required by fidelity to the slaves, or by the just
demands of conscience? Or, in other words, is the refusal to exercise
the elective franchise, and to hold office in America, the surest,
wisest, and best way to abolish slavery in America?

To these questions the Garrisonians say Yes. They hold the Constitution
to be a slaveholding instrument, and will not cast a vote or hold
office, and denounce all who vote or hold office, no matter how
faithfully such persons labour to promote the abolition of slavery. I,
on the other hand, deny that the Constitution guarantees the right to
hold property in man, and believe that the way to abolish slavery in
America is to vote such men into power as will use their powers for the
abolition of slavery. This is the issue plainly stated, and you shall
judge between us. Before we examine into the disposition, tendency, and
character of the Constitution, I think we had better ascertain what the
Constitution itself is. Before looking for what it means, let us see
what it is. Here, too, there is much dust to be cleared away. What,
then, is the Constitution? I will tell you. It is no vague, indefinite,
floating, unsubstantial, ideal something, coloured according to any
man's fancy, now a weasel, now a whale, and now nothing. On the
contrary, it is a plainly written document, not in Hebrew or Greek, but
in English, beginning with a preamble, filled out with articles,
sections, provisions, and clauses, defining the rights, powers, and
duties to be secured, claimed, and exercised under its authority. It is
not even like the British Constitution, which is made up of enactments
of Parliament, decisions of Courts, and the established usages of the
Government. The American Constitution is a written instrument full and
complete in itself. No Court in America, no Congress, no President, can
add a single word thereto, or take a single word therefrom. It is a
great national enactment done by the people, and can only be altered,
amended, or added to by the people. I am careful to make this statement
here; in America it would not be necessary. It would not be necessary
here if my assailant had showed the same desire to set before you the
simple truth, which he manifested to make out a good case for himself
and friends. Again, it should be borne in mind that the mere text, and
only the text, and not any commentaries or creeds written by those who
wished to give the text a meaning apart from its plain reading, was
adopted as the Constitution of the United States. It should also be
borne in mind that the intentions of those who framed the Constitution,
be they good or bad, for slavery or against slavery, are to be respected
so far, and so far only, as we find those intentions plainly stated in
the Constitution. It would be the wildest of absurdities, and lead to
endless confusion and mischiefs, if, instead of looking to the written
paper itself, for its meaning, it were attempted to make us search it
out, in the secret motives, and dishonest intentions, of some of the men
who took part in writing it. It was what they said that was adopted by
the people, not what they were ashamed or afraid to say, and really
omitted to say. Bear in mind, also, and the fact is an important one,
that the framers of the Constitution sat with closed doors, and that
this was done purposely, that nothing but the result of their labours
should be seen, and that that result should be judged of by the people
free from any of the bias shown in the debates. It should also be borne
in mind, and the fact is still more important, that the debates in the
convention that framed the Constitution, and by means of which a
pro-slavery interpretation is now attempted to be forced upon that
instrument, were not published till more than a quarter of a century
after the presentation and the adoption of the Constitution.

These debates were purposely kept out of view, in order that the people
should adopt, not the secret motives or unexpressed intentions of any
body, but the simple text of the paper itself. Those debates form no
part of the original agreement. I repeat, the paper itself, and only the
paper itself, with its own plainly-written purposes, is the
Constitution. It must stand or fall, flourish or fade, on its own
individual and self-declared character and objects. Again, where would
be the advantage of a written Constitution, if, instead of seeking its
meaning in its words, we had to seek them in the secret intentions of
individuals who may have had something to do with writing the paper?
What will the people of America a hundred years hence care about the
intentions of the scriveners who wrote the Constitution? These men are
already gone from us, and in the course of nature were expected to go
from us. They were for a generation, but the Constitution is for ages.
Whatever we may owe to them, we certainly owe it to ourselves, and to
mankind, and to God, to maintain the truth of our own language, and to
allow no villainy, not even the villainy of holding men as slaves --
which Wesley says is the sum of all villainies -- to shelter itself
under a fair-seeming and virtuous language. We owe it to ourselves to
compel the devil to wear his own garments, and to make wicked laws speak
out their wicked intentions. Common sense, and common justice, and sound
rules of interpretation all drive us to the words of the law for the
meaning of the law. The practice of the Government is dwelt upon with
much fervour and eloquence as conclusive as to the slaveholding
character of the Constitution. This is really the strong point, and the
only strong point, made in the speech in the City Hall. But good as this
argument is, it is not conclusive. A wise man has said that few people
have been found better than their laws, but many have been found worse.
To this last rule America is no exception. Her laws are one thing, her
practice is another thing. We read that the Jews made void the law by
their tradition, that Moses permitted men to put away their wives
because of the hardness of their hearts, but that this was not so at the
beginning. While good laws will always be found where good practice
prevails, the reverse does not always hold true. Far from it. The very
opposite is often the case. What then? Shall we condemn the righteous
law because wicked men twist it to the support of wickedness? Is that
the way to deal with good and evil? Shall we blot out all distinction
between them, and hand over to slavery all that slavery may claim on the
score of long practice? Such is the course commended to us in the City
Hall speech. After all, the fact that men go out of the Constitution to
prove it pro-slavery, whether that going out is to the practice of the
Government, or to the secret intentions of the writers of the paper, the
fact that they do go out is very significant. It is a powerful argument
on my side. It is an admission that the thing for which they are looking
is not to be found where only it ought to be found, and that is in the
Constitution itself. If it is not there, it is nothing to the purpose,
be it wheresoever else it may be. But I shall have more to say on this
point hereafter.

The very eloquent lecturer at the City Hall doubtless felt some
embarrassment from the fact that he had literally to give the
Constitution a pro-slavery interpretation; because upon its face it of
itself conveys no such meaning, but a very opposite meaning. He thus
sums up what he calls the slaveholding provisions of the Constitution. I
quote his own words: -- ``Article I, section 9, provides for the
continuance of the African slave trade for 20 years, after the adoption
of the Constitution. Art˙ 4, section 9, provides for the recovery from
other States of fugitive slaves. Art˙ I, section 2, gives the slave
States a representation of three-fifths of all the slave population; and
Art˙ I, section 8, requires the President to use the military, naval,
ordnance, and militia resources of the entire country for the
suppression of slave insurrection, in the same manner as he would employ
them to repel invasion.'' Now any man reading this statement, or hearing
it made with such a show of exactness, would unquestionably suppose that
the speaker or writer had given the plain written text of the
Constitution itself. I can hardly believe that he intended to make any
such impression. It would be a scandalous imputation to say he did. And
yet what are we to make of it? How can we regard it? How can he be
screened from the charge of having perpetrated a deliberate and
point-blank misrepresentation? That individual has seen fit to place
himself before the public as my opponent, and yet I would gladly find
some excuse for him. I do not wish to think as badly of him as this
trick of his would naturally lead me to think. Why did he not read the
Constitution? Why did he read that which was not the Constitution? He
pretended to be giving chapter and verse, section and clause, paragraph
and provision. The words of the Constitution were before him. Why then
did he not give you the plain words of the Constitution? Oh, sir, I fear
that that gentleman knows too well why he did not. It so happens that no
such words as ``African slave trade,'' no such words as ``slave
representation,'' no such words as ``fugitive slaves,'' no such words as
``slave insurrections,'' are anywhere used in that instrument. These are
the words of that orator, and not the words of the Constitution of the
United States. Now you shall see a slight difference between my manner
of treating this subject and that which my opponent has seen fit, for
reasons satisfactory to himself, to pursue. What he withheld, that I
will spread before you: what he suppressed, I will bring to light: and
what he passed over in silence, I will proclaim: that you may have the
whole case before you, and not be left to depend upon either his, or
upon my inferences or testimony. Here then are the several provisions of
the Constitution to which reference has been made. I read them word for
word just as they stand in the paper, called the United States
Constitution, Art˙ I, sec˙ 2. ``Representatives and direct taxes shall
be apportioned among the several States which may be included in this
Union, according to their respective numbers, which shall be determined
by adding to the whole number of free persons, including those bound to
service for a term of years, and excluding Indians not taxed,
three-fifths of all other persons; Art˙ I, sec˙ 9. The migration or
importation of such persons as any of the States now existing shall
think fit to admit, shall not be prohibited by the Congress prior to the
year one thousand eight hundred and eight, but a tax or duty may be
imposed on such importation, not exceeding ten dollars for each person;
Art˙ 4, sec˙ 2. No person held to service or labour in one State, under
the laws thereof, escaping into another shall, in consequence of any law
or regulation therein, be discharged from such service or labour; but
shall be delivered up on claim of the party to whom such service or
labour may be due; Art˙ I, sec˙ 8. To provide for calling for the
militia to execute the laws of the Union, suppress insurrections, and
repel invasions.'' Here, then, are those provisions of the Constitution,
which the most extravagant defenders of slavery can claim to guarantee a
right of property in man. These are the provisions which have been
pressed into the service of the human fleshmongers of America. Let us
look at them just as they stand, one by one. Let us grant, for sake of
the argument, that the first of these provisions, referring to the basis
of representation and taxation, does refer to slaves. We are not
compelled to make that admission, for it might fairly apply to aliens --
persons living in the country, but not naturalized. But giving the
provisions the very worst construction, what does it amount to? I answer
-- It is a downright disability laid upon the slaveholding States; one
which deprives those States of two-fifths of their natural basis of
representation. A black man in a free State is worth just two-fifths
more than a black man in a slave State, as a basis of political power
under the Constitution. Therefore, instead of encouraging slavery, the
Constitution encourages freedom by giving an increase of ``two-fifths''
of political power to free over slave States. So much for the
three-fifths clause; taking it at its worst, it still leans to freedom,
not to slavery; for, be it remembered that the Constitution nowhere
forbids a coloured man to vote. I come to the next, that which it is
said guaranteed the continuance of the African slave trade for twenty
years. I will also take that for just what my opponent alleges it to
have been, although the Constitution does not warrant any such
conclusion. But, to be liberal, let us suppose it did, and what follows?
why, this -- that this part of the Constitution, so far as the slave
trade is concerned, became a dead letter more than 50 years ago, and now
binds no man's conscience for the continuance of any slave trade
whatever. Mr˙ Thompson is just 52 years too late in dissolving the Union
on account of this clause. He might as well dissolve the British
Government, because Queen Elizabeth granted to Sir John Hawkins to
import Africans into the West Indies 300 years ago! But there is still
more to be said about this abolition of the slave trade. Men, at that
time, both in England and in America, looked upon the slave trade as the
life of slavery. The abolition of the slave trade was supposed to be the
certain death of slavery. Cut off the stream, and the pond will dry up,
was the common notion at that time.

Wilberforce and Clarkson, clear-sighted as they were, took this view;
and the American statesmen, in providing for the abolition of the slave
trade, thought they were providing for the abolition of slavery. This
view is quite consistent with the history of the times. All regarded
slavery as an expiring and doomed system, destined to speedily disappear
from the country. But, again, it should be remembered that this very
provision, if made to refer to the African slave trade at all, makes the
Constitution anti-slavery rather than for slavery, for it says to the
slave States, the price you will have to pay for coming into the
American Union is, that the slave trade, which you would carry on
indefinitely out of the Union, shall be put an end to in twenty years if
you come into the Union. Secondly, if it does apply, it expired by its
own limitation more than fifty years ago. Thirdly, it is anti-slavery,
because it looked to the abolition of slavery rather than to its
perpetuity. Fourthly, it showed that the intentions of the framers of
the Constitution were good, not bad. I think this is quite enough for
this point. I go to the ``slave insurrection'' clause, though, in truth,
there is no such clause. The one which is called so has nothing whatever
to do with slaves or slaveholders any more than your laws for the
suppression of popular outbreaks has to do with making slaves of you and
your children. It is only a law for suppression of riots or
insurrections. But I will be generous here, as well as elsewhere, and
grant that it applies to slave insurrections. Let us suppose that an
anti-slavery man is President of the United States (and the day that
shall see this the case is not distant) and this very power of
suppressing slave insurrection would put an end to slavery. The right to
put down an insurrection carries with it the right to determine the
means by which it shall be put down. If it should turn out that slavery
is a source of insurrection, that there is no security from insurrection
while slavery lasts, why, the Constitution would be best obeyed by
putting an end to slavery, and an anti-slavery Congress would do that
very thing. Thus, you see, the so-called slave-holding provisions of the
American Constitution, which a little while ago looked so formidable,
are, after all, no defence or guarantee for slavery whatever. But there
is one other provision. This is called the ``Fugitive Slave Provision.''
It is called so by those who wish to make it subserve the interest of
slavery in America, and the same by those who wish to uphold the views
of a party in this country. It is put thus in the speech at the City
Hall: -- ``Let us go back to 1787, and enter Liberty Hall, Philadelphia,
where sat in convention the illustrious men who framed the Constitution
-- with George Washington in the chair. On the 27th of September, Mr˙
Butler and Mr˙ Pinckney, two delegates from the State of South Carolina,
moved that the Constitution should require that fugitive slaves and
servants should be delivered up like criminals, and after a discussion
on the subject, the clause, as it stands in the Constitution, was
adopted. After this, in the conventions held in the several States to
ratify the Constitution, the same meaning was attached to the words. For
example, Mr˙ Madison (afterwards President), when recommending the
Constitution to his constituents, told them that the clause would secure
them their property in slaves.'' I must ask you to look well to this
statement. Upon its face, it would seem a full and fair statement of the
history of the transaction it professes to describe and yet I declare
unto you, knowing as I do the facts in the case, my utter amazement at
the downright untruth conveyed under the fair seeming words now quoted.
The man who could make such a statement may have all the craftiness of a
lawyer, but who can accord to him the candour of an honest debater? What
could more completely destroy all confidence in his statements? Mark
you, the orator had not allowed his audience to hear read the provision
of the Constitution to which he referred. He merely characterized it as
one to ``deliver up fugitive slaves and servants like criminals,'' and
tells you that that provision was adopted as it stands in the
Constitution. He tells you that this was done ``after discussion.'' But
he took good care not to tell you what was the nature of that
discussion. He would have spoiled the whole effect of his statement had
he told you the whole truth. Now, what are the facts connected with this
provision of the Constitution? You shall have them. It seems to take two
men to tell the truth. It is quite true that Mr˙ Butler and Mr˙ Pinckney
introduced a provision expressly with a view to the recapture of
fugitive slaves: it is quite true also that there was some discussion on
the subject -- and just here the truth shall come out. These illustrious
kidnappers were told promptly in that discussion that no such idea as
property in man should be admitted into the Constitution. The speaker in
question might have told you, and he would have told you but the simple
truth, if he had told you that the proposition of Mr˙ Butler and Mr˙
Pinckney -- which he leads you to infer was adopted by the convention
that framed the Constitution -- was, in fact, promptly and indignantly
rejected by that convention. He might have told you, had it suited his
purpose to do so, that the words employed in the first draft of the
fugitive clause were such as applied to the condition of slaves, and
expressly declared that persons held to ``servitude'' should be given
up; but that the word ``servitude'' was struck from the provision, for
the very reason that it applied to slaves. He might have told you that
that same Mr˙ Madison declared that that word was struck out because the
convention would not consent that the idea of property in men should be
admitted into the Constitution. The fact that Mr˙ Madison can be cited
on both sides of this question is another evidence of the folly and
absurdity of making the secret intentions of the framers the criterion
by which the Constitution is to be construed. But it may be asked -- if
this clause does not apply to slaves, to whom does it apply?

I answer, that when adopted, it applies to a very large class of persons
-- namely, redemptioners -- persons who had come to America from
Holland, from Ireland, and other quarters of the globe -- like the
Coolies to the West Indies -- and had, for a consideration duly paid,
become bound to ``serve and labour'' for the parties to whom their
service and labour was due. It applies to indentured apprentices and
others who had become bound for a consideration, under contract duly
made, to serve and labour. To such persons this provision applies, and
only to such persons. The plain reading of this provision shows that it
applies, and that it can only properly and legally apply, to persons
``bound to service.'' Its object plainly is, to secure the fulfilment of
contracts for ``service and labour.'' It applies to indentured
apprentices, and any other persons from whom service and labour may be
due. The legal condition of the slave puts him beyond the operation of
this provision. He is not described in it. He is a simple article of
property. He does not owe and cannot owe service. He cannot even make a
contract. It is impossible for him to do so. He can no more make such a
contract than a horse or an ox can make one. This provision, then, only
respects persons who owe service, and they only can owe service who can
receive an equivalent and make a bargain. The slave cannot do that, and
is therefore exempted from the operation of this fugitive provision. In
all matters where laws are taught to be made the means of oppression,
cruelty, and wickedness, I am for strict construction. I will concede
nothing. It must be shown that it is so nominated in the bond. The pound
of flesh, but not one drop of blood. The very nature of law is opposed
to all such wickedness, and makes it difficult to accomplish such
objects under the forms of law. Law is not merely an arbitrary enactment
with regard to justice, reason, or humanity. Blackstone defines it to be
a rule prescribed by the supreme power of the State commanding what is
right and forbidding what is wrong. The speaker at the City Hall laid
down some rules of legal interpretation. These rules send us to the
history of the law for its meaning. I have no objection to such a course
in ordinary cases of doubt. But where human liberty and justice are at
stake, the case falls under an entirely different class of rules. There
must be something more than history -- something more than tradition.
The Supreme Court of the United States lays down this rule, and it meets
the case exactly -- ``Where rights are infringed -- where the
fundamental principles of the law are overthrown -- where the general
system of the law is departed from, the legislative intention must be
expressed with irresistible clearness.'' The same court says that the
language of the law must be construed strictly in favour of justice and
liberty. Again, there is another rule of law. It is -- Where a law is
susceptible of two meanings, the one making it accomplish an innocent
purpose, and the other making it accomplish a wicked purpose, we must in
all cases adopt that which makes it accomplish an innocent purpose.
Again, the details of a law are to be interpreted in the light of the
declared objects sought by the law. I set these rules down against those
employed at the City Hall. To me they seem just and rational. I only ask
you to look at the American Constitution in the light of them, and you
will see with me that no man is guaranteed a right of property in man,
under the provisions of that instrument. If there are two ideas more
distinct in their character and essence than another, those ideas are
``persons'' and ``property,'' ``men'' and ``things.'' Now, when it is
proposed to transform persons into ``property'' and men into beasts of
burden, I demand that the law that contemplates such a purpose shall be
expressed with irresistible clearness. The thing must not be left to
inference, but must be done in plain English. I know how this view of
the subject is treated by the class represented at the City Hall. They
are in the habit of treating the Negro as an exception to general rules.
When their own liberty is in question they will avail themselves of all
rules of law which protect and defend their freedom; but when the black
man's rights are in question they concede everything, admit everything
for slavery, and put liberty to the proof. They reverse the common law
usage, and presume the Negro a slave unless he can prove himself free.
I, on the other hand, presume him free unless he is proved to be
otherwise. Let us look at the objects for which the Constitution was
framed and adopted, and see if slavery is one of them. Here are its own
objects as set forth by itself: -- ``We, the people of these United
States, in order to form a more perfect union, establish justice, ensure
domestic tranquillity, provide for the common defence, promote the
general welfare, and secure the blessings of liberty to ourselves and
our posterity, do ordain and establish this Constitution for the United
States of America.'' The objects here set forth are six in number:
union, defence, welfare, tranquillity, justice, and liberty. These are
all good objects, and slavery, so far from being among them, is a foe of
them all. But it has been said that Negroes are not included within the
benefits sought under this declaration. This is said by the slaveholders
in America -- it is said by the City Hall orator -- but it is not said
by the Constitution itself. Its language is ``we the people;'' not we
the white people, not even we the citizens, not we the privileged class,
not we the high, not we the low, but we the people; not we the horses,
sheep, and swine, and wheel-barrows, but we the people, we the human
inhabitants; and, if Negroes are people, they are included in the
benefits for which the Constitution of America was ordained and
established. But how dare any man who pretends to be a friend to the
Negro thus gratuitously concede away what the Negro has a right to claim
under the Constitution? Why should such friends invent new arguments to
increase the hopelessness of his bondage? This, I undertake to say, as
the conclusion of the whole matter, that the constitutionality of
slavery can be made out only by disregarding the plain and common-sense
reading of the Constitution itself; by discrediting and casting away as
worthless the most beneficent rules of legal interpretation; by ruling
the Negro outside of these beneficent rules; by claiming everything for
slavery; by denying everything for freedom; by assuming that the
Constitution does not mean what it says, and that it says what it does
not mean; by disregarding the written Constitution, and interpreting it
in the light of a secret understanding. It is in this mean,
contemptible, and underhand method that the American Constitution is
pressed into the service of slavery. They go everywhere else for proof
that the Constitution is pro-slavery but to the Constitution itself. The
Constitution declares that no person shall be deprived of life, liberty,
or property without due process of law; it secures to every man the
right of trial by jury, the privilege of the writ of habeas corpus --
that great writ that put an end to slavery and slave-hunting in England
-- it secures to every State a republican form of government. Any one of
these provisions, in the hands of abolition statesmen, and backed up by
a right moral sentiment, would put an end to slavery in America. The
Constitution forbids the passing of a bill of attainder: that is, a law
entailing upon the child the disabilities and hardships imposed upon the
parent. Every slave law in America might be repealed on this very
ground. The slave is made a slave because his mother is a slave. But to
all this it is said that the practice of the American people is against
my view. I admit it. They have given the Constitution a slaveholding
interpretation. I admit it. They have committed innumerable wrongs
against the Negro in the name of the Constitution. Yes, I admit it all;
and I go with him who goes farthest in denouncing these wrongs. But it
does not follow that the Constitution is in favour of these wrongs
because the slaveholders have given it that interpretation. To be
consistent in his logic, the City Hall speaker must follow the example
of some of his brothers in America -- he must not only fling away the
Constitution, but the Bible. The Bible must follow the Constitution, for
that, too, has been interpreted for slavery by American divines. Nay,
more, he must not stop with the Constitution of America, but make war
upon the British Constitution, for, if I mistake not, that gentleman is
opposed to the union of Church and State. In America he called himself a
Republican. Yet he does not go for breaking down the British
Constitution, although you have a Queen on the throne, and bishops in
the House of Lords.

My argument against the dissolution of the American Union is this: It
would place the slave system more exclusively under the control of the
slaveholding States, and withdraw it from the power in the Northern
States which is opposed to slavery. Slavery is essentially barbarous in
its character. It, above all things else, dreads the presence of an
advanced civilisation. It flourishes best where it meets no reproving
frowns, and hears no condemning voices. While in the Union it will meet
with both. Its hope of life, in the last resort, is to get out of the
Union. I am, therefore, for drawing the bond of the Union more closely,
and bringing the Slave States more completely under the power of the
Free States. What they most dread, that I most desire. I have much
confidence in the instincts of the slaveholders. They see that the
Constitution will afford slavery no protection when it shall cease to be
administered by slaveholders. They see, moreover, that if there is once
a will in the people of America to abolish slavery, there is no word, no
syllable in the Constitution to forbid that result. They see that the
Constitution has not saved slavery in Rhode Island, in Connecticut, in
New York, or Pennsylvania; that the Free States have increased from one
up to eighteen in number, while the Slave States have only added three
to their original number. There were twelve Slave States at the
beginning of the Government: there are fifteen now. There was one Free
State at the beginning of the Government: there are eighteen now. The
dissolution of the Union would not give the North a single advantage
over slavery, but would take from it many. Within the Union we have a
firm basis of opposition to slavery. It is opposed to all the great
objects of the Constitution. The dissolution of the Union is not only an
unwise but a cowardly measure -- 15 millions running away from three
hundred and fifty thousand slaveholders. Mr˙ Garrison and his friends
tell us that while in the Union we are responsible for slavery. He and
they sing out ``No Union with slaveholders,'' and refuse to vote. I
admit our responsibility for slavery while in the Union, but I deny that
going out of the Union would free us from that responsibility. There now
clearly is no freedom from responsibility for slavery to any American
citizen short of the abolition of slavery. The American people have gone
quite too far in this slaveholding business now to sum up their whole
business of slavery by singing out the cant phrase, ``No union with
slaveholders.'' To desert the family hearth may place the recreant
husband out of the presence of his starving children, but this does not
free him from responsibility. If a man were on board of a pirate ship,
and in company with others had robbed and plundered, his whole duty
would not be performed simply by taking the longboat and singing out
``No union with pirates.'' His duty would be to restore the stolen
property. The American people in the Northern States have helped to
enslave the black people. Their duty will not have been done till they
give them back their plundered rights. Reference was made at the City
Hall to my having once held other opinions, and very different opinions
to those I have now expressed. An old speech of mine delivered fourteen
years ago was read to show -- I know not what. Perhaps it was to show
that I am not infallible. If so, I have to say in defence, that I never
pretended to be. Although I cannot accuse myself of being remarkably
unstable, I do not pretend that I have never altered my opinion both in
respect to men and things. Indeed, I have been very much modified both
in feeling and opinion within the last fourteen years. When I escaped
from slavery, and was introduced to the Garrisonians, I adopted very
many of their opinions, and defended them just as long as I deemed them
true. I was young, had read but little, and naturally took some things
on trust. Subsequent experience and reading have led me to examine for
myself. This has brought me to other conclusions. When I was a child, I
thought and spoke as a child. But the question is not as to what were my
opinions fourteen years ago, but what they are now. If I am right now,
it really does not matter what I was fourteen years ago. My position now
is one of reform, not of revolution. I would act for the abolition of
slavery through the Government -- not over its ruins. If slaveholders
have ruled the American Government for the last fifty years, let the
anti-slavery men rule the nation for the next fifty years. If the South
has made the Constitution bend to the purposes of slavery, let the North
now make that instrument bend to the cause of freedom and justice. If
350,000 slaveholders have, by devoting their energies to that single
end, been able to make slavery the vital and animating spirit of the
American Confederacy for the last 72 years, now let the freemen of the
North, who have the power in their own hands, and who can make the
American Government just what they think fit, resolve to blot out for
ever the foul and haggard crime, which is the blight and mildew, the
curse and the disgrace of the whole United States.

\hypertarget{dred-scott-v.-sandford}{%
\subsubsection{Dred Scott v. Sandford}\label{dred-scott-v.-sandford}}

60 U.S. 393 (1856)

\textbf{Chief Justice TANEY delivered the opinion of the court.}

This case has been twice argued. After the argument at the last term,
differences of opinion were found to exist among the members of the
court; and as the questions in controversy are of the highest
importance, and the court was at that time much pressed by the ordinary
business of the term, it was deemed advisable to continue the case, and
direct a re-argument on some of the points, in order that we might have
an opportunity of giving to the whole subject a more deliberate
consideration. It has accordingly been again argued by counsel, and
considered by the court; and I now proceed to deliver its opinion.

There are two leading questions presented by the record:

\begin{enumerate}
\def\labelenumi{\arabic{enumi}.}
\item
  Had the Circuit Court of the United States jurisdiction to hear and
  determine the case between these parties? And
\item
  If it had jurisdiction, is the judgment it has given erroneous or not?
\end{enumerate}

The plaintiff in error, who was also the plaintiff in the court below,
was, with his wife and children, held as slaves by the defendant, in the
State of Missouri; and he brought this action in the Circuit Court of
the United States for that district, to assert the title of himself and
his family to freedom.

The declaration is in the form usually adopted in that State to try
questions of this description, and contains the averment necessary to
give the court jurisdiction; that he and the defendant are citizens of
different States; that is, that he is a citizen of Missouri, and the
defendant a citizen of New York.

The defendant pleaded in abatement to the jurisdiction of the court,
that the plaintiff was not a citizen of the State of Missouri, as
alleged in his declaration, being a negro of African descent, whose
ancestors were of pure African blood, and who were brought into this
country and sold as slaves.

That plea denies the right of the plaintiff to sue in a court of the
United States, for the reasons therein stated.

If the question raised by it is legally before us, and the court should
be of opinion that the facts stated in it disqualify the plaintiff from
becoming a citizen, in the sense in which that word is used in the
Constitution of the United States, then the judgment of the Circuit
Court is erroneous, and must be reversed.

The question is simply this: Can a negro, whose ancestors were imported
into this country, and sold as slaves, become a member of the political
community formed and brought into existence by the Constitution of the
United States, and as such become entitled to all the rights, and
privileges, and immunities, guarantied by that instrument to the
citizen? One of which rights is the privilege of suing in a court of the
United States in the cases specified in the Constitution.

It will be observed, that the plea applies to that class of persons only
whose ancestors were negroes of the African race, and imported into this
country, and sold and held as slaves. The only matter in issue before
the court, therefore, is, whether the descendants of such slaves, when
they shall be emancipated, or who are born of parents who had become
free before their birth, are citizens of a State, in the sense in which
the word citizen is used in the Constitution of the United States. And
this being the only matter in dispute on the pleadings, the court must
be understood as speaking in this opinion of that class only, that is,
of those persons who are the descendants of Africans who were imported
into this country, and sold as slaves.

The situation of this population was altogether unlike that of the
Indian race. The latter, it is true, formed no part of the colonial
communities, and never amalgamated with them in social connections or in
government. But although they were uncivilized, they were yet a free and
independent people, associated together in nations or tribes, and
governed by their own laws. Many of these political communities were
situated in territories to which the white race claimed the ultimate
right of dominion. But that claim was acknowledged to be subject to the
right of the Indians to occupy it as long as they thought proper, and
neither the English nor colonial Governments claimed or exercised any
dominion over the tribe or nation by whom it was occupied, nor claimed
the right to the possession of the territory, until the tribe or nation
consented to cede it. These Indian Governments were regarded and treated
as foreign Governments, as much so as if an ocean had separated the red
man from the white; and their freedom has constantly been acknowledged,
from the time of the first emigration to the English colonies to the
present day, by the different Governments which succeeded each other.
Treaties have been negotiated with them, and their alliance sought for
in war; and the people who compose these Indian political communities
have always been treated as foreigners not living under our Government.
It is true that the course of events has brought the Indian tribes
within the limits of the United States under subjection to the white
race; and it has been found necessary, for their sake as well as our
own, to regard them as in a state of pupilage, and to legislate to a
certain extent over them and the territory they occupy. But they may,
without doubt, like the subjects of any other foreign Government, be
naturalized by the authority of Congress, and become citizens of a
State, and of the United States; and if an individual should leave his
nation or tribe, and take up his abode among the white population, he
would be entitled to all the rights and privileges which would belong to
an emigrant from any other foreign people.

We proceed to examine the case as presented by the pleadings.

The words `people of the United States' and `citizens' are synonymous
terms, and mean the same thing. They both describe the political body
who, according to our republican institutions, form the sovereignty, and
who hold the power and conduct the Government through their
representatives. They are what we familiarly call the `sovereign
people,' and every citizen is one of this people, and a constituent
member of this sovereignty. The question before us is, whether the class
of persons described in the plea in abatement compose a portion of this
people, and are constituent members of this sovereignty? We think they
are not, and that they are not included, and were not intended to be
included, under the word `citizens' in the Constitution, and can
therefore claim none of the rights and privileges which that instrument
provides for and secures to citizens of the United States. On the
contrary, they were at that time considered as a subordinate and
inferior class of beings, who had been subjugated by the dominant race,
and, whether emancipated or not, yet remained subject to their
authority, and had no rights or privileges but such as those who held
the power and the Government might choose to grant them.

It is not the province of the court to decide upon the justice or
injustice, the policy or impolicy, of these laws. The decision of that
question belonged to the political or law-making power; to those who
formed the sovereignty and framed the Constitution. The duty of the
court is, to interpret the instrument they have framed, with the best
lights we can obtain on the subject, and to administer it as we find it,
according to its true intent and meaning when it was adopted.

In discussing this question, we must not confound the rights of
citizenship which a State may confer within its own limits, and the
rights of citizenship as a member of the Union. It does not by any means
follow, because he has all the rights and privileges of a citizen of a
State, that he must be a citizen of the United States. He may have all
of the rights and privileges of the citizen of a State, and yet not be
entitled to the rights and privileges of a citizen in any other State.
For, previous to the adoption of the Constitution of the United States,
every State had the undoubted right to confer on whomsoever it pleased
the character of citizen, and to endow him with all its rights. But this
character of course was confined to the boundaries of the State, and
gave him no rights or privileges in other States beyond those secured to
him by the laws of nations and the comity of States. Nor have the
several States surrendered the power of conferring these rights and
privileges by adopting the Constitution of the United States. Each State
may still confer them upon an alien, or any one it thinks proper, or
upon any class or description of persons; yet he would not be a citizen
in the sense in which that word is used in the Constitution of the
United States, nor entitled to sue as such in one of its courts, nor to
the privileges and immunities of a citizen in the other States. The
rights which he would acquire would be restricted to the State which
gave them. The Constitution has conferred on Congress the right to
establish an uniform rule of naturalization, and this right is evidently
exclusive, and has always been held by this court to be so.
Consequently, no State, since the adoption of the Constitution, can by
naturalizing an alien invest him with the rights and privileges secured
to a citizen of a State under the Federal Government, although, so far
as the State alone was concerned, he would undoubtedly be entitled to
the rights of a citizen, and clothed with all the rights and immunities
which the Constitution and laws of the State attached to that character.

It is very clear, therefore, that no State can, by any act or law of its
own, passed since the adoption of the Constitution, introduce a new
member into the political community created by the Constitution of the
United States. It cannot make him a member of this community by making
him a member of its own. And for the same reason it cannot introduce any
person, or description of persons, who were not intended to be embraced
in this new political family, which the Constitution brought into
existence, but were intended to be excluded from it.

The question then arises, whether the provisions of the Constitution, in
relation to the personal rights and privileges to which the citizen of a
State should be entitled, embraced the negro African race, at that time
in this country, or who might afterwards be imported, who had then or
should afterwards be made free in any State; and to put it in the power
of a single State to make him a citizen of the United States, and endue
him with the full rights of citizenship in every other State without
their consent? Does the Constitution of the United States act upon him
whenever he shall be made free under the laws of a State, and raised
there to the rank of a citizen, and immediately clothe him with all the
privileges of a citizen in every other State, and in its own courts?

The court think the affirmative of these propositions cannot be
maintained. And if it cannot, the plaintiff in error could not be a
citizen of the State of Missouri, within the meaning of the Constitution
of the United States, and, consequently, was not entitled to sue in its
courts.

It is true, every person, and every class and description of persons,
who were at the time of the adoption of the Constitution recognised as
citizens in the several States, became also citizens of this new
political body; but none other; it was formed by them, and for them and
their posterity, but for no one else. And the personal rights and
privileges guarantied to citizens of this new sovereignty were intended
to embrace those only who were then members of the several State
communities, or who should afterwards by birthright or otherwise become
members, according to the provisions of the Constitution and the
principles on which it was founded. It was the union of those who were
at that time members of distinct and separate political communities into
one political family, whose power, for certain specified purposes, was
to extend over the whole territory of the United States. And it gave to
each citizen rights and privileges outside of his State which he did not
before possess, and placed him in every other State upon a perfect
equality with its own citizens as to rights of person and rights of
property; it made him a citizen of the United States.

It becomes necessary, therefore, to determine who were citizens of the
several States when the Constitution was adopted. And in order to do
this, we must recur to the Governments and institutions of the thirteen
colonies, when they separated from Great Britain and formed new
sovereignties, and took their places in the family of independent
nations. We must inquire who, at that time, were recognised as the
people or citizens of a State, whose rights and liberties had been
outraged by the English Government; and who declared their independence,
and assumed the powers of Government to defend their rights by force of
arms.

In the opinion of the court, the legislation and histories of the times,
and the language used in the Declaration of Independence, show, that
neither the class of persons who had been imported as slaves, nor their
descendants, whether they had become free or not, were then acknowledged
as a part of the people, nor intended to be included in the general
words used in that memorable instrument.

It is difficult at this day to realize the state of public opinion in
relation to that unfortunate race, which prevailed in the civilized and
enlightened portions of the world at the time of the Declaration of
Independence, and when the Constitution of the United States was framed
and adopted. But the public history of every European nation displays it
in a manner too plain to be mistaken.

They had for more than a century before been regarded as beings of an
inferior order, and altogether unfit to associate with the white race,
either in social or political relations; and so far inferior, that they
had no rights which the white man was bound to respect; and that the
negro might justly and lawfully be reduced to slavery for his benefit.
He was bought and sold, and treated as an ordinary article of
merchandise and traffic, whenever a profit could be made by it. This
opinion was at that time fixed and universal in the civilized portion of
the white race. It was regarded as an axiom in morals as well as in
politics, which no one thought of disputing, or supposed to be open to
dispute; and men in every grade and position in society daily and
habitually acted upon it in their private pursuits, as well as in
matters of public concern, without doubting for a moment the correctness
of this opinion.

And in no nation was this opinion more firmly fixed or more uniformly
acted upon than by the English Government and English people. They not
only seized them on the coast of Africa, and sold them or held them in
slavery for their own use; but they took them as ordinary articles of
merchandise to every country where they could make a profit on them, and
were far more extensively engaged in this commerce than any other nation
in the world.

The opinion thus entertained and acted upon in England was naturally
impressed upon the colonies they founded on this side of the Atlantic.
And, accordingly, a negro of the African race was regarded by them as an
article of property, and held, and bought and sold as such, in every one
of the thirteen colonies which united in the Declaration of
Independence, and afterwards formed the Constitution of the United
States. The slaves were more or less numerous in the different colonies,
as slave labor was found more or less profitable. But no one seems to
have doubted the correctness of the prevailing opinion of the time.

The legislation of the different colonies furnishes positive and
indisputable proof of this fact.

It would be tedious, in this opinion, to enumerate the various laws they
passed upon this subject. It will be sufficient, as a sample of the
legislation which then generally prevailed throughout the British
colonies, to give the laws of two of them; one being still a large
slaveholding State, and the other the first State in which slavery
ceased to exist.

The province of Maryland, in 1717, (ch.~13, s. 5,) passed a law
declaring `that if any free negro or mulatto intermarry with any white
woman, or if any white man shall intermarry with any negro or mulatto
woman, such negro or mulatto shall become a slave during life, excepting
mulattoes born of white women, who, for such intermarriage, shall only
become servants for seven years, to be disposed of as the justices of
the county court, where such marriage so happens, shall think fit; to be
applied by them towards the support of a public school within the said
county. And any white man or white woman who shall intermarry as
aforesaid, with any negro or mulatto, such white man or white woman
shall become servants during the term of seven years, and shall be
disposed of by the justices as aforesaid, and be applied to the uses
aforesaid.'

The other colonial law to which we refer was passed by Massachusetts in
1705, (chap.~6.) It is entitled `An act for the better preventing of a
spurious and mixed issue,' \&c.; and it provides, that `if any negro or
mulatto shall presume to smite or strike any person of the English or
other Christian nation, such negro or mulatto shall be severely whipped,
at the discretion of the justices before whom the offender shall be
convicted.'

And `that none of her Majesty's English or Scottish subjects, nor of any
other Christian nation, within this province, shall contract matrimony
with any negro or mulatto; nor shall any person, duly authorized to
solemnize marriage, presume to join any such in marriage, on pain of
forfeiting the sum of fifty pounds; one moiety thereof to her Majesty,
for and towards the support of the Government within this province, and
the other moiety to him or them that shall inform and sue for the same,
in any of her Majesty's courts of record within the province, by bill,
plaint, or information.'

We give both of these laws in the words used by the respective
legislative bodies, because the language in which they are framed, as
well as the provisions contained in them, show, too plainly to be
misunderstood, the degraded condition of this unhappy race. They were
still in force when the Revolution began, and are a faithful index to
the state of feeling towards the class of persons of whom they speak,
and of the position they occupied throughout the thirteen colonies, in
the eyes and thoughts of the men who framed the Declaration of
Independence and established the State Constitutions and Governments.
They show that a perpetual and impassable barrier was intended to be
erected between the white race and the one which they had reduced to
slavery, and governed as subjects with absolute and despotic power, and
which they then looked upon as so far below them in the scale of created
beings, that intermarriages between white persons and negroes or
mulattoes were regarded as unnatural and immoral, and punished as
crimes, not only in the parties, but in the person who joined them in
marriage. And no distinction in this respect was made between the free
negro or mulatto and the slave, but this stigma, of the deepest
degradation, was fixed upon the whole race.

We refer to these historical facts for the purpose of showing the fixed
opinions concerning that race, upon which the statesmen of that day
spoke and acted. It is necessary to do this, in order to determine
whether the general terms used in the Constitution of the United States,
as to the rights of man and the rights of the people, was intended to
include them, or to give to them or their posterity the benefit of any
of its provisions.

The language of the Declaration of Independence is equally conclusive:

It begins by declaring that, `when in the course of human events it
becomes necessary for one people to dissolve the political bands which
have connected them with another, and to assume among the powers of the
earth the separate and equal station to which the laws of nature and
nature's God entitle them, a decent respect for the opinions of mankind
requires that they should declare the causes which impel them to the
separation.'

It then proceeds to say: `We hold these truths to be self-evident: that
all men are created equal; that they are endowed by their Creator with
certain unalienable rights; that among them is life, liberty, and the
pursuit of happiness; that to secure these rights, Governments are
instituted, deriving their just powers from the consent of the
governed.'

The general words above quoted would seem to embrace the whole human
family, and if they were used in a similar instrument at this day would
be so understood. But it is too clear for dispute, that the enslaved
African race were not intended to be included, and formed no part of the
people who framed and adopted this declaration; for if the language, as
understood in that day, would embrace them, the conduct of the
distinguished men who framed the Declaration of Independence would have
been utterly and flagrantly inconsistent with the principles they
asserted; and instead of the sympathy of mankind, to which they so
confidently appealed, they would have deserved and received universal
rebuke and reprobation.

Yet the men who framed this declaration were great men---high in
literary acquirements---high in their sense of honor, and incapable of
asserting principles inconsistent with those on which they were acting.
They perfectly understood the meaning of the language they used, and how
it would be understood by others; and they knew that it would not in any
part of the civilized world be supposed to embrace the negro race,
which, by common consent, had been excluded from civilized Governments
and the family of nations, and doomed to slavery. They spoke and acted
according to the then established doctrines and principles, and in the
ordinary language of the day, and no one misunderstood them. The unhappy
black race were separated from the white by indelible marks, and laws
long before established, and were never thought of or spoken of except
as property, and when the claims of the owner or the profit of the
trader were supposed to need protection.

This state of public opinion had undergone no change when the
Constitution was adopted, as is equally evident from its provisions and
language.

The brief preamble sets forth by whom it was formed, for what purposes,
and for whose benefit and protection. It declares that it is formed by
the people of the United States; that is to say, by those who were
members of the different political communities in the several States;
and its great object is declared to be to secure the blessings of
liberty to themselves and their posterity. It speaks in general terms of
the people of the United States, and of citizens of the several States,
when it is providing for the exercise of the powers granted or the
privileges secured to the citizen. It does not define what description
of persons are intended to be included under these terms, or who shall
be regarded as a citizen and one of the people. It uses them as terms so
well understood, that no further description or definition was
necessary.

But there are two clauses in the Constitution which point directly and
specifically to the negro race as a separate class of persons, and show
clearly that they were not regarded as a portion of the people or
citizens of the Government then formed.

One of these clauses reserves to each of the thirteen States the right
to import slaves until the year 1808, if it thinks proper. And the
importation which it thus sanctions was unquestionably of persons of the
race of which we are speaking, as the traffic in slaves in the United
States had always been confined to them. And by the other provision the
States pledge themselves to each other to maintain the right of property
of the master, by delivering up to him any slave who may have escaped
from his service, and be found within their respective territories. By
the first above-mentioned clause, therefore, the right to purchase and
hold this property is directly sanctioned and authorized for twenty
years by the people who framed the Constitution. And by the second, they
pledge themselves to maintain and uphold the right of the master in the
manner specified, as long as the Government they then formed should
endure. And these two provisions show, conclusively, that neither the
description of persons therein referred to, nor their descendants, were
embraced in any of the other provisions of the Constitution; for
certainly these two clauses were not intended to confer on them or their
posterity the blessings of liberty, or any of the personal rights so
carefully provided for the citizen.

No one of that race had ever migrated to the United States voluntarily;
all of them had been brought here as articles of merchandise. The number
that had been emancipated at that time were but few in comparison with
those held in slavery; and they were identified in the public mind with
the race to which they belonged, and regarded as a part of the slave
population rather than the free. It is obvious that they were not even
in the minds of the framers of the Constitution when they were
conferring special rights and privileges upon the citizens of a State in
every other part of the Union.

Indeed, when we look to the condition of this race in the several States
at the time, it is impossible to believe that these rights and
privileges were intended to be extended to them.

It is very true, that in that portion of the Union where the labor of
the negro race was found to be unsuited to the climate and unprofitable
to the master, but few slaves were held at the time of the Declaration
of Independence; and when the Constitution was adopted, it had entirely
worn out in one of them, and measures had been taken for its gradual
abolition in several others. But this change had not been produced by
any change of opinion in relation to this race; but because it was
discovered, from experience, that slave labor was unsuited to the
climate and productions of these States: for some of the States, where
it had ceased or nearly ceased to exist, were actively engaged in the
slave trade, procuring cargoes on the coast of Africa, and transporting
them for sale to those parts of the Union where their labor was found to
be profitable, and suited to the climate and productions. And this
traffic was openly carried on, and fortunes accumulated by it, without
reproach from the people of the States where they resided. And it can
hardly be supposed that, in the States where it was then countenanced in
its worst form---that is, in the seizure and transportation---the people
could have regarded those who were emancipated as entitled to equal
rights with themselves.

And we may here again refer, in support of this proposition, to the
plain and unequivocal language of the laws of the several States, some
passed after the Declaration of Independence and before the Constitution
was adopted, and some since the Government went into operation.

We need not refer, on this point, particularly to the laws of the
present slaveholding States. Their statute books are full of provisions
in relation to this class, in the same spirit with the Maryland law
which we have before quoted. They have continued to treat them as an
inferior class, and to subject them to strict police regulations,
drawing a broad line of distinction between the citizen and the slave
races, and legislating in relation to them upon the same principle which
prevailed at the time of the Declaration of Independence. As relates to
these States, it is too plain for argument, that they have never been
regarded as a part of the people or citizens of the State, nor supposed
to possess any political rights which the dominant race might not
withhold or grant at their pleasure.

And as long ago as 1822, the Court of Appeals of Kentucky decided that
free negroes and mulattoes were not citizens within the meaning of the
Constitution of the United States; and the correctness of this decision
is recognized, and the same doctrine affirmed, in 1 Meigs's Tenn.
Reports, 331.

And if we turn to the legislation of the States where slavery had worn
out, or measures taken for its speedy abolition, we shall find the same
opinions and principles equally fixed and equally acted upon.

Thus, Massachusetts, in 1786, passed a law similar to the colonial one
of

The legislation of the States therefore shows, in a manner not to be
mistaken, the inferior and subject condition of that race at the time
the Constitution was adopted, and long afterwards, throughout the
thirteen States by which that instrument was framed; and it is hardly
consistent with the respect due to these States, to suppose that they
regarded at that time, as fellow-citizens and members of the
sovereignty, a class of beings whom they had thus stigmatized; whom, as
we are bound, out of respect to the State sovereignties, to assume they
had deemed it just and necessary thus to stigmatize, and upon whom they
had impressed such deep and enduring marks of inferiority and
degradation; or, that when they met in convention to form the
Constitution, they looked upon them as a portion of their constituents,
or designed to include them in the provisions so carefully inserted for
the security and protection of the liberties and rights of their
citizens. It cannot be supposed that they intended to secure to them
rights, and privileges, and rank, in the new political body throughout
the Union, which every one of them denied within the limits of its own
dominion. More especially, it cannot be believed that the large
slaveholding States regarded them as included in the word citizens, or
would have consented to a Constitution which might compel them to
receive them in that character from another State. For if they were so
received, and entitled to the privileges and immunities of citizens, it
would exempt them from the operation of the special laws and from the
police regulations which they considered to be necessary for their own
safety. It would give to persons of the negro race, who were recognised
as citizens in any one State of the Union, the right to enter every
other State whenever they pleased, singly or in companies, without pass
or passport, and without obstruction, to sojourn there as long as they
pleased, to go where they pleased at every hour of the day or night
without molestation, unless they committed some violation of law for
which a white man would be punished; and it would give them the full
liberty of speech in public and in private upon all subjects upon which
its own citizens might speak; to hold public meetings upon political
affairs, and to keep and carry arms wherever they went. And all of this
would be done in the face of the subject race of the same color, both
free and slaves, and inevitably producing discontent and insubordination
among them, and endangering the peace and safety of the State.

It is impossible, it would seem, to believe that the great men of the
slaveholding States, who took so large a share in framing the
Constitution of the United States, and exercised so much influence in
procuring its adoption, could have been so forgetful or regardless of
their own safety and the safety of those who trusted and confided in
them.

Besides, this want of foresight and care would have been utterly
inconsistent with the caution displayed in providing for the admission
of new members into this political family. For, when they gave to the
citizens of each State the privileges and immunities of citizens in the
several States, they at the same time took from the several States the
power of naturalization, and confined that power exclusively to the
Federal Government. No State was willing to permit another State to
determine who should or should not be admitted as one of its citizens,
and entitled to demand equal rights and privileges with their own
people, within their own territories. The right of naturalization was
therefore, with one accord, surrendered by the States, and confided to
the Federal Government. And this power granted to Congress to establish
an uniform rule of naturalization is, by the well-understood meaning of
the word, confined to persons born in a foreign country, under a foreign
Government. It is not a power to raise to the rank of a citizen any one
born in the United States, who, from birth or parentage, by the laws of
the country, belongs to an inferior and subordinate class. And when we
find the States guarding themselves from the indiscreet or improper
admission by other States of emigrants from other countries, by giving
the power exclusively to Congress, we cannot fail to see that they could
never have left with the States a much more important power---that is,
the power of transforming into citizens a numerous class of persons, who
in that character would be much more dangerous to the peace and safety
of a large portion of the Union, than the few foreigners one of the
States might improperly naturalize. The Constitution upon its adoption
obviously took from the States all power by any subsequent legislation
to introduce as a citizen into the political family of the United States
any one, no matter where he was born, or what might be his character or
condition; and it gave to Congress the power to confer this character
upon those only who were born outside of the dominions of the United
States. And no law of a State, therefore, passed since the Constitution
was adopted, can give any right of citizenship outside of its own
territory.

But it is said that a person may be a citizen, and entitled to that
character, although he does not possess all the rights which may belong
to other citizens; as, for example, the right to vote, or to hold
particular offices; and that yet, when he goes into another State, he is
entitled to be recognised there as a citizen, although the State may
measure his rights by the rights which it allows to persons of a like
character or class resident in the State, and refuse to him the full
rights of citizenship.

This argument overlooks the language of the provision in the
Constitution of which we are speaking.

Undoubtedly, a person may be a citizen, that is, a member of the
community who form the sovereignty, although he exercises no share of
the political power, and is incapacitated from holding particular
offices. Women and minors, who form a part of the political family,
cannot vote; and when a property qualification is required to vote or
hold a particular office, those who have not the necessary qualification
cannot vote or hold the office, yet they are citizens.

So, too, a person may be entitled to vote by the law of the State, who
is not a citizen even of the State itself. And in some of the States of
the Union foreigners not naturalized are allowed to vote. And the State
may give the right to free negroes and mulattoes, but that does not make
them citizens of the State, and still less of the United States. And the
provision in the Constitution giving privileges and immunities in other
States, does not apply to them.

Neither does it apply to a person who, being the citizen of a State,
migrates to another State. For then he becomes subject to the laws of
the State in which he lives, and he is no longer a citizen of the State
from which he removed. And the State in which he resides may then,
unquestionably, determine his status or condition, and place him among
the class of persons who are not recognised as citizens, but belong to
an inferior and subject race; and may deny him the privileges and
immunities enjoyed by its citizens.

But so far as mere rights of person are concerned, the provision in
question is confined to citizens of a State who are temporarily in
another State without taking up their residence there. It gives them no
political rights in the State, as to voting or holding office, or in any
other respect. For a citizen of one State has no right to participate in
the government of another. But if he ranks as a citizen in the State to
which he belongs, within the meaning of the Constitution of the United
States, then, whenever he goes into another State, the Constitution
clothes him, as to the rights of person, will all the privileges and
immunities which belong to citizens of the

State. And if persons of the African race are citizens of a State, and
of the United States, they would be entitled to all of these privileges
and immunities in every State, and the State could not restrict them;
for they would hold these privileges and immunities under the paramount
authority of the Federal Government, and its courts would be bound to
maintain and enforce them, the Constitution and laws of the State to the
contrary notwithstanding. And if the States could limit or restrict
them, or place the party in an inferior grade, this clause of the
Constitution would be unmeaning, and could have no operation; and would
give no rights to the citizen when in another State. He would have none
but what the State itself chose to allow him. This is evidently not the
construction or meaning of the clause in question. It guaranties rights
to the citizen, and the State cannot withhold them. And these rights are
of a character and would lead to consequences which make it absolutely
certain that the African race were not included under the name of
citizens of a State, and were not in the contemplation of the framers of
the Constitution when these privileges and immunities were provided for
the protection of the citizen in other States.

No one, we presume, supposes that any change in public opinion or
feeling, in relation to this unfortunate race, in the civilized nations
of Europe or in this country, should induce the court to give to the
words of the Constitution a more liberal construction in their favor
than they were intended to bear when the instrument was framed and
adopted. Such an argument would be altogether inadmissible in any
tribunal called on to interpret it. If any of its provisions are deemed
unjust, there is a mode prescribed in the instrument itself by which it
may be amended; but while it remains unaltered, it must be construed now
as it was understood at the time of its adoption. It is not only the
same in words, but the same in meaning, and delegates the same powers to
the Government, and reserves and secures the same rights and privileges
to the citizen; and as long as it continues to exist in its present
form, it speaks not only in the same words, but with the same meaning
and intent with which it spoke when it came from the hands of its
framers, and was voted on and adopted by the people of the United
States. Any other rule of construction would abrogate the judicial
character of this court, and make it the mere reflex of the popular
opinion or passion of the day. This court was not created by the
Constitution for such purposes. Higher and graver trusts have been
confided to it, and it must not falter in the path of duty.

What the construction was at that time, we think can hardly admit of
doubt. We have the language of the Declaration of Independence and of
the Articles of Confederation, in addition to the plain words of the
Constitution itself; we have the legislation of the different States,
before, about the time, and since, the Constitution was adopted; we have
the legislation of Congress, from the time of its adoption to a recent
period; and we have the constant and uniform action of the Executive
Department, all concurring together, and leading to the same result. And
if anything in relation to the construction of the Constitution can be
regarded as settled, it is that which we now give to the word `citizen'
and the word `people.'

And upon a full and careful consideration of the subject, the court is
of opinion, that, upon the facts stated in the plea in abatement, Dred
Scott was not a citizen of Missouri within the meaning of the
Constitution of the United States, and not entitled as such to sue in
its courts; and, consequently, that the Circuit Court had no
jurisdiction of the case, and that the judgment on the plea in abatement
is erroneous.

\hypertarget{equal-protection}{%
\section{Equal Protection}\label{equal-protection}}

\hypertarget{foundations-and-race}{%
\subsection{Foundations, and Race}\label{foundations-and-race}}

\hypertarget{plessy-v.-ferguson}{%
\subsubsection{Plessy v. Ferguson}\label{plessy-v.-ferguson}}

163 U.S. 537 (1896)

\textbf{Mr.~Justice BROWN, after stating the facts in the foregoing
language, delivered the opinion of the court.}

This case turns upon the constitutionality of an act of the general
assembly of the state of Louisiana, passed in 1890, providing for
separate railway carriages for the white and colored races. Acts 1890,
No.~111, p.~152.

The first section of the statute enacts `that all railway companies
carrying passengers in their coaches in this state, shall provide equal
but separate accommodations for the white, and colored races, by
providing two or more passenger coaches for each passenger train, or by
dividing the passenger coaches by a partition so as to secure separate
accommodations: provided, that this section shall not be construed to
apply to street railroads. No person or persons shall be permitted to
occupy seats in coaches, other than the ones assigned to them, on
account of the race they belong to.'

By the second section it was enacted `that the officers of such
passenger trains shall have power and are hereby required to assign each
passenger to the coach or compartment used for the race to which such
passenger belongs; any passenger insisting on going into a coach or
compartment to which by race he does not belong, shall be liable to a
fine of twenty-five dollars, or in lieu thereof to imprisonment for a
period of not more than twenty days in the parish prison, and any
officer of any railroad insisting on assigning a passenger to a coach or
compartment other than the one set aside for the race to which said
passenger belongs, shall be liable to a fine of twenty-five dollars, or
in lieu thereof to imprisonment for a period of not more than twenty
days in the parish prison; and should any passenger refuse to occupy the
coach or compartment to which he or she is assigned by the officer of
such railway, said officer shall have power to refuse to carry such
passenger on his train, and for such refusal neither he nor the railway
company which he represents shall be liable for damages in any of the
courts of this state.'

\begin{enumerate}
\def\labelenumi{\arabic{enumi}.}
\setcounter{enumi}{1}
\tightlist
\item
  By the fourteenth amendment, all persons born or naturalized in the
  United States, and subject to the jurisdiction thereof, are made
  citizens of the United States and of the state wherein they reside;
  and the states are forbidden from making or enforcing any law which
  shall abridge the privileges or immunities of citizens of the United
  States, or shall deprive any person of life, liberty, or property
  without due process of law, or deny to any person within their
  jurisdiction the equal protection of the laws.
\end{enumerate}

The object of the amendment was undoubtedly to enforce the absolute
equality of the two races before the law, but, in the nature of things,
it could not have been intended to abolish distinctions based upon
color, or to enforce social, as distinguish d from political, equality,
or a commingling of the two races upon terms unsatisfactory to either.
Laws permitting, and even requiring, their separation, in places where
they are liable to be brought into contact, do not necessarily imply the
inferiority of either race to the other, and have been generally, if not
universally, recognized as within the competency of the state
legislatures in the exercise of their police power. The most common
instance of this is connected with the establishment of separate schools
for white and colored children, which have been held to be a valid
exercise of the legislative power even by courts of states where the
political rights of the colored race have been longest and most
earnestly enforced.

So far, then, as a conflict with the fourteenth amendment is concerned,
the case reduces itself to the question whether the statute of Louisiana
is a reasonable regulation, and with respect to this there must
necessarily be a large discretion on the part of the legislature. In
determining the question of reasonableness, it is at liberty to act with
reference to the established usages, customs, and traditions of the
people, and with a view to the promotion of their comfort, and the
preservation of the public peace and good order. Gauged by this
standard, we cannot say that a law which authorizes or even requires the
separation of the two races in public conveyances is unreasonable, or
more obnoxious to the fourteenth amendment than the acts of congress
requiring separate schools for colored children in the District of
Columbia, the constitutionality of which does not seem to have been
questioned, or the corresponding acts of state legislatures.

We consider the underlying fallacy of the plaintiff's argument to
consist in the assumption that the enforced separation of the two races
stamps the colored race with a badge of inferiority. If this be so, it
is not by reason of anything found in the act, but solely because the
colored race chooses to put that construction upon it. The argument
necessarily assumes that if, as has been more than once the case, and is
not unlikely to be so again, the colored race should become the dominant
power in the state legislature, and should enact a law in precisely
similar terms, it would thereby relegate the white race to an inferior
position. We imagine that the white race, at least, would not acquiesce
in this assumption. The argument also assumes that social prejudices may
be overcome by legislation, and that equal rights cannot be secured to
the negro except by an enforced commingling of the two races. We cannot
accept this proposition. If the two races are to meet upon terms of
social equality, it must be the result of natural affinities, a mutual
appreciation of each other's merits, and a voluntary consent of
individuals. As was said by the court of appeals of New York in People
v. Gallagher, 93 N. Y. 438, 448:

`This end can neither be accomplished nor promoted by laws which
conflict with the general sentiment of the community upon whom they are
designed to operate. When the government, therefore, has secured to each
of its citizens equal rights before the law, and equal opportunities for
improvement and progress, it has accomplished the end for which it was
organized, and performed all of the functions respecting social
advantages with which it is endowed.'

Legislation is powerless to eradicate racial instincts, or to abolish
distinctions based upon physical differences, and the attempt to do so
can only result in accentuating the difficulties of the present
situation. If the civil and political rights of both races be equal, one
cannot be inferior to the other civilly or politically. If one race be
inferior to the other socially, the constitution of the United States
cannot put them upon the same plane.

\textbf{Mr.~Justice HARLAN dissenting.}

However apparent the injustice of such legislation may be, we have only
to consider whether it is consistent with the constitution of the United
States.

In respect of civil rights, common to all citizens, the constitution of
the United States does not, I think, permit any public authority to know
the race of those entitled to be protected in the enjoyment of such
rights. Every true man has pride of race, and under appropriate
circumstances, when the rights of others, his equals before the law, are
not to be affected, it is his privilege to express such pride and to
take such action based upon it as to him seems proper. But I deny that
any legislative body or judicial tribunal may have regard to the race of
citizens when the civil rights of those citizens are involved. Indeed,
such legislation as that here in question is inconsistent not only with
that equality of rights which pertains to citizenship, national and
state, but with the personal liberty enjoyed by every one within the
United States.

The thirteenth amendment does not permit the withholding or the
deprivation of any right necessarily inhering in freedom. It not only
struck down the institution of slavery as previously existing in the
United States, but it prevents the imposition of any burdens or
disabilities that constitute badges of slavery or servitude. It decreed
universal civil freedom in this country. This court has so adjudged.
But, that amendment having been found inadequate to the protection of
the rights of those who had been in slavery, it was followed by the
fourteenth amendment, which added greatly to the dignity and glory of
American citizenship, and to the security of personal liberty, by
declaring that `all persons born or naturalized in the United States,
and subject to the jurisdiction thereof, are citizens of the United
States and of the state wherein they reside,' and that `no state shall
make or enforce any law which shall abridge the privileges or immunities
of citizens of the United States; nor shall any state deprive any person
of life, liberty or property without due process of law, nor deny to any
person within its jurisdiction the equal protection of the laws.' These
two amendments, if enforced according to their true intent and meaning,
will protect all the civil rights that pertain to freedom and
citizenship. Finally, and to the end that no citizen should be denied,
on account of his race, the privilege of participating in the political
control of his country, it was declared by the fifteenth amendment that
`the right of citizens of the United States to vote shall not be denied
or abridged by the United States or by any state on account of race,
color or previous condition of servitude.'

These notable additions to the fundamental law were welcomed by the
friends of liberty throughout the world. They removed the race line from
our governmental systems. They had, as this court has said, a common
purpose, namely, to secure `to a race recently emancipated, a race that
through many generations have been held in slavery, all the civil rights
that the superior race enjoy.' They declared, in legal effect, this
court has further said, `that the law in the states shall be the same
for the black as for the white; that all persons, whether colored or
white, shall stand equal before the laws of the states; and in regard to
the colored race, for whose protection the amendment was primarily
designed, that no discrimination shall be made against them by law
because of their color.' We also said: `The words of the amendment, it
is true, are prohibitory, but they contain a necessary implication of a
positive immunity or right, most valuable to the colored race,---the
right to exemption from unfriendly legislation against them
distinctively as colored; exemption from legal discriminations, implying
inferiority in civil society, lessening the security of their enjoyment
of the rights which others enjoy; and discriminations which are steps
towards reducing them to the condition of a subject race.' It was,
consequently, adjudged that a state law that excluded citizens of the
colored race from juries, because of their race, however well qualified
in other respects to discharge the duties of jurymen, was repugnant to
the fourteenth amendment. Strauder v. West Virginia, 307; Virginia v.
Rives, ; Ex parte Virginia, ; Neal v. Delaware; Bush v. Com., 1 Sup. Ct.
625. At the present term, referring to the previous adjudications, this
court declared that `underlying all of those decisions is the principle
that the constitution of the United States, in its present form,
forbids, so far as civil and political rights are concerned,
discrimination by the general government or the states against any
citizen because of his race. All citizens are equal before the law.'
Gibson v. StateSup. Ct. 904.

The decisions referred to show the scope of the recent amendments of the
constitution. They also show that it is not within the power of a state
to prohibit colored citizens, because of their race, from participating
as jurors in the administration of justice.

It was said in argument that the statute of Louisiana does not
discriminate against either race, but prescribes a rule applicable alike
to white and colored citizens. But this argument does not meet the
difficulty. Every one knows that the statute in question had its origin
in the purpose, not so much to exclude white persons from railroad cars
occupied by blacks, as to exclude colored people from coaches occupied
by or assigned to white persons. Railroad corporations of Louisiana did
not make discrimination among whites in the matter of commodation for
travelers. The thing to accomplish was, under the guise of giving equal
accommodation for whites and blacks, to compel the latter to keep to
themselves while traveling in railroad passenger coaches. No one would
be so wanting in candor as to assert the contrary. The fundamental
objection, therefore, to the statute, is that it interferes with the
personal freedom of citizens. `Personal liberty,' it has been well said,
`consists in the power of locomotion, of changing situation, or removing
one's person to whatsoever places one's own inclination may direct,
without imprisonment or restraint, unless by due course of law.' 1 Bl.
Comm. If a white man and a black man choose to occupy the same public
conveyance on a public highway, it is their right to do so; and no
government, proceeding alone on grounds of race, can prevent it without
infringing the personal liberty of each.

It is one thing for railroad carriers to furnish, or to be required by
law to furnish, equal accommodations for all whom they are under a legal
duty to carry. It is quite another thing for government to forbid
citizens of the white and black races from traveling in the same public
conveyance, and to punish officers of railroad companies for permitting
persons of the two races to occupy the same passenger coach. If a state
can prescribe, as a rule of civil conduct, that whites and blacks shall
not travel as passengers in the same railroad coach, why may it not so
regulate the use of the streets of its cities and towns as to compel
white citizens to keep on one side of a street, and black citizens to
keep on the other? Why may it not, upon like grounds, punish whites and
blacks who ride together in street cars or in open vehicles on a public
road or street? Why may it not require sheriffs to assign whites to one
side of a court room, and blacks to the other? And why may it not also
prohibit the commingling of the two races in the galleries of
legislative halls or in public assemblages convened for the
consideration of the political questions of the day? Further, if this
statute of Louisiana is consistent with the personal liberty of
citizens, why may not the state require the separation in railroad
coaches of native and naturalized citizens of the United States, or of
Protestants and Roman Catholics?

The answer given at the argument to these questions was that regulations
of the kind they suggest would be unreasonable, and could not,
therefore, stand before the la Is it meant that the determination of
questions of legislative power depends upon the inquiry whether the
statute whose validity is questioned is, in the judgment of the courts,
a reasonable one, taking all the circumstances into consideration? A
statute may be unreasonable merely because a sound public policy forbade
its enactment. But I do not understand that the courts have anything to
do with the policy or expediency of legislation. A statute may be valid,
and yet, upon grounds of public policy, may well be characterized as
unreasonable. Mr.~Sedgwick correctly states the rule when he says that,
the legislative intention being clearly ascertained, `the courts have no
other duty to perform than to execute the legislative will, without any
regard to their views as to the wisdom or justice of the particular
enactment.' Sedg. St.~\& Const. Law, 324. There is a dangerous tendency
in these latter days to enlarge the functions of the courts, by means of
judicial interference with the will of the people as expressed by the
legislature. Our institutions have the distinguishing characteristic
that the three departments of government are co-ordinate and separate.
Each much keep within the limits defined by the constitution. And the
courts best discharge their duty by executing the will of the law-making
power, constitutionally expressed, leaving the results of legislation to
be dealt with by the people through their representatives. Statutes must
always have a reasonable construction. Sometimes they are to be
construed strictly, sometimes literally, in order to carry out the
legislative will. But, however construed, the intent of the legislature
is to be respected if the particular statute in question is valid,
although the courts, looking at the public interests, may conceive the
statute to be both unreasonable and impolitic. If the power exists to
enact a statute, that ends the matter so far as the courts are
concerned. The adjudged cases in which statutes have been held to be
void, because unreasonable, are those in which the means employed by the
legislature were not at all germane to the end to which the legislature
was competent.

The white race deems itself to be the dominant race in this country. And
so it is, in prestige, in achievements, in education, in wealth, and in
power. So, I doubt not, it will continue to be for all time, if it
remains true to its great heritage, and holds fast to the principles of
constitutional liberty. But in view of the constitution, in the eye of
the law, there is in this country no superior, dominant, ruling class of
citizens. There is no caste here. Our constitution is color-blind, and
neither knows nor tolerates classes among citizens. In respect of civil
rights, all citizens are equal before the law. The humblest is the peer
of the most powerful. The law regards man as man, and takes no account
of his surroundings or of his color when his civil rights as guarantied
by the supreme law of the land are involved. It is therefore to be
regretted that this high tribunal, the final expositor of the
fundamental law of the land, has reached the conclusion that it is
competent for a state to regulate the enjoyment by citizens of their
civil rights solely upon the basis of race.

In my opinion, the judgment this day rendered will, in time, prove to be
quite as pernicious as the decision made by this tribunal in the Dred
Scott Case.

It was adjudged in that case that the descendants of Africans who were
imported into this country, and sold as slaves, were not included nor
intended to be included under the word `citizens' in the constitution,
and could not claim any of the rights and privileges which that
instrument provided for and secured to citizens of the United States;
that, at time of the adoption of the constitution, they were `considered
as a subordinate and inferior class of beings, who had been subjugated
by the dominant race, and, whether emancipated or not, yet remained
subject to their authority, and had no rights or privileges but such as
those who held the power and the government might choose to grant them.'
17 How. 393, 404. The recent amendments of the constitution, it was
supposed, had eradicated these principles from our institutions. But it
seems that we have yet, in some of the states, a dominant race,---a
superior class of citizens,---which assumes to regulate the enjoyment of
civil rights, common to all citizens, upon the basis of race. The
present decision, it may well be apprehended, will not only stimulate
aggressions, more or less brutal and irritating, upon the admitted
rights of colored citizens, but will encourage the belief that it is
possible, by means of state enactments, to defeat the beneficent
purposes which the people of the United States had in view when they
adopted the recent amendments of the constitution, by one of which the
blacks of this country were made citizens of the United States and of
the states in which they respectively reside, and whose privileges and
immunities, as citizens, the states are forbidden to abridge. Sixty
millions of whites are in no danger from the presence here of eight
millions of blacks. The destinies of the two races, in this country, are
indissolubly linked together, and the interests of both require that the
common government of all shall not permit the seeds of race hate to be
planted under the sanction of law. What can more certainly arouse race
hate, what more certainly create and perpetuate a feeling of distrust
between these races, than state enactments which, in fact, proceed on
the ground that colored citizens are so inferior and degraded that they
cannot be allowed to sit in public coaches occupied by white citizens?
That, as all will admit, is the real meaning of such legislation as was
enacted in Louisiana.

The sure guaranty of the peace and security of each race is the clear,
distinct, unconditional recognition by our governments, national and
state, of every right that inheres in civil freedom, and of the equality
before the law of all citizens of the United States, without regard to
race. State enactments regulating the enjoyment of civil rights upon the
basis of race, and cunningly devised to defeat legitimate results of the
war, under the pretense of recognizing equality of rights, can have no
other result than to render permanent peace impossible, and to keep
alive a conflict of races, the continuance of which must do harm to all
concerned. This question is not met by the suggestion that social
equality cannot exist between the white and black races in this country.
That argument, if it can be properly regarded as one, is scarcely worthy
of consideration; for social equality no more exists between two races
when traveling in a passenger coach or a public highway than when
members of the same races sit by each other in a street car or in the
jury box, or stand or sit with each other in a political assembly, or
when they use in common the streets of a city or town, or when they are
in the same room for the purpose of having their names placed on the
registry of voters, or when they approach the ballot box in order to
exercise the high privilege of voting.

\hypertarget{brown-v.-board-of-education}{%
\subsubsection{Brown v. Board of
Education}\label{brown-v.-board-of-education}}

347 U.S. 483 (1954)

\textbf{Mr.~Chief Justice WARREN delivered the opinion of the Court.}
These cases come to us from the States of Kansas, South Carolina,
Virginia, and Delaware. They are premised on different facts and
different local conditions, but a common legal question justifies their
consideration together in this consolidated opinion In each of the
cases, minors of the Negro race, through their legal representatives,
seek the aid of the courts in obtaining admission to the public schools
of their community on a nonsegregated basis. In each instance, they have
been denied admission to schools attended by white children under laws
requiring or permitting segregation according to race. This segregation
was alleged to deprive the plaintiffs of the equal protection of the
laws under the Fourteenth Amendment. In each of the cases other than the
Delaware case, a three-judge federal district court denied relief to the
plaintiffs on the so-called `separate but equal' doctrine announced by
this Court in Plessy v. FergusonUnder that doctrine, equality of
treatment is accorded when the races are provided substantially equal
facilities, even though these facilities be separate. In the Delaware
case, the Supreme Court of Delaware adhered to that doctrine, but
ordered that the plaintiffs be admitted to the white schools because of
their superiority to the Negro schools.

The plaintiffs contend that segregated public schools are not `equal'
and cannot be made `equal,' and that hence they are deprived of the
equal protection of the laws. Because of the obvious importance of the
question presented, the Court took jurisdiction Argument was heard in
the 1952 Term, and reargument was heard this Term on certain questions
propounded by the Court Reargument was largely devoted to the
circumstances surrounding the adoption of the Fourteenth Amendment in
1868. It covered exhaustively consideration of the Amendment in
Congress, ratification by the states, then existing practices in racial
segregation, and the views of proponents and opponents of the Amendment.
This discussion and our own investigation convince us that, although
these sources cast some light, it is not enough to resolve the problem
with which we are faced. At best, they are inconclusive. The most avid
proponents of the post-War Amendments undoubtedly intended them to
remove all legal distinctions among `all persons born or naturalized in
the United States.' Their opponents, just as certainly, were
antagonistic to both the letter and the spirit of the Amendments and
wished them to have the most limited effect. What others in Congress and
the state legislatures had in mind cannot be determined with any degree
of certainty.

An additional reason for the inconclusive nature of the Amendment's
history, with respect to segregated schools, is the status of public
education at that time In the South, the movement toward free common
schools, supported by general taxation, had not yet taken hold.
Education of white children was largely in the hands of private groups.
Education of Negroes was almost nonexistent, and practically all of the
race were illiterate. In fact, any education of Negroes was forbidden by
law in some states. Today, in contrast, many Negroes have achieved
outstanding success in the arts and sciences as well as in the business
and professional world. It is true that public school education at the
time of the Amendment had advanced further in the North, but the effect
of the Amendment on Northern States was generally ignored in the
congressional debates. Even in the North, the conditions of public
education did not approximate those existing today. The curriculum was
usually rudimentary; ungraded schools were common in rural areas; the
school term was but three months a year in many states; and compulsory
school attendance was virtually unknown. As a consequence, it is not
surprising that there should be so little in the history of the
Fourteenth Amendment relating to its intended effect on public
education.

In the first cases in this Court construing the Fourteenth Amendment,
decided shortly after its adoption, the Court interpreted it as
proscribing all state-imposed discriminations against the Negro race The
doctrine of ``separate but equal'' did not make its appearance in this
court until 1896 in the case of Plessy v. Fergusoninvolving not
education but transportation American courts have since labored with the
doctrine for over half a century. In this Court, there have been six
cases involving the `separate but equal' doctrine in the field of public
education In Cumming v. Board of Education of Richmond Countyand Gong
Lum v. Ricethe validity of the doctrine itself was not challenged. 8 In
more recent cases, all on the graduate school level, inequality was
found in that specific benefits enjoyed by white students were denied to
Negro students of the same educational qualifications. State of Missouri
ex rel. Gaines v. Canada; Sipuel v. Board of Regents of University of
Oklahoma; Sweatt v. Painters.Ct. 848, ; McLaurin v. Oklahoma State
RegentsIn none of these cases was it necessary to re-examine the
doctrine to grant relief to the Negro plaintiff. And in Sweatt v.
Painterthe Court expressly reserved decision on the question whether
Plessy v. Ferguson should be held inapplicable to public education.

In the instant cases, that question is directly presented. Here, unlike
Sweatt v. Painter, there are findings below that the Negro and white
schools involved have been equalized, or are being equalized, with
respect to buildings, curricula, qualifications and salaries of
teachers, and other `tangible' factors. 9 Our decision, therefore,
cannot turn on merely a comparison of these tangible factors in the
Negro and white schools involved in each of the cases. We must look
instead to the effect of segregation itself on public education.

In approaching this problem, we cannot turn the clock back to 1868 when
the Amendment was adopted, or even to 1896 when Plessy v. Ferguson was
written. We must consider public education in the light of its full
development and its present place in American life throughout the
Nation. Only in this way can it be determined if segregation in public
schools deprives these plaintiffs of the equal protection of the laws.

Today, education is perhaps the most important function of state and
local governments. Compulsory school attendance laws and the great
expenditures for education both demonstrate our recognition of the
importance of education to our democratic society. It is required in the
performance of our most basic public responsibilities, even service in
the armed forces. It is the very foundation of good citizenship. Today
it is a principal instrument in awakening the child to cultural values,
in preparing him for later professional training, and in helping him to
adjust normally to his environment. In these days, it is doubtful that
any child may reasonably be expected to succeed in life if he is denied
the opportunity of an education. Such an opportunity, where the state
has undertaken to provide it, is a right which must be made available to
all on equal terms.

We come then to the question presented: Does segregation of children in
public schools solely on the basis of race, even though the physical
facilities and other `tangible' factors may be equal, deprive the
children of the minority group of equal educational opportunities? We
believe that it does.

In Sweatt v. Painter(339 U.S. 629, ), in finding that a segregated law
school for Negroes could not provide them equal educational
opportunities, this Court relied in large part on `those qualities which
are incapable of objective measurement but which make for greatness in a
law school.' In McLaurin v. Oklahoma State Regents(339 U.S. 637, ), the
Court, in requiring that a Negro admitted to a white graduate school be
treated like all other students, again resorted to intangible
considerations: '

his ability to study, to engage in discussions and exchange views with
other students, and, in general, to learn his profession.' Such
considerations apply with added force to children in grade and high
schools. To separate them from others of similar age and qualifications
solely because of their race generates a feeling of inferiority as to
their status in the community that may affect their hearts and minds in
a way unlikely ever to be undone. The effect of this separation on their
educational opportunities was well stated by a finding in the Kansas
case by a court which nevertheless felt compelled to rule against the
Negro plaintiffs:

`Segregation of white and colored children in public schools has a
detrimental effect upon the colored children. The impact is greater when
it has the sanction of the law; for the policy of separating the races
is usually interpreted as denoting the inferiority of the negro group. A
sense of inferiority affects the motivation of a child to learn.
Segregation with the sanction of law, therefore, has a tendency to
(retard) the educational and mental development of Negro children and to
deprive them of some of the benefits they would receive in a racial(ly)
integrated school system.'

Whatever may have been the extent of psychological knowledge at the time
of Plessy v. Ferguson, this finding is amply supported by modern
authority Any language in Plessy v. Ferguson contrary to this finding is
rejected.

We conclude that in the field of public education the doctrine of
`separate but equal' has no place. Separate educational facilities are
inherently unequal. Therefore, we hold that the plaintiffs and others
similarly situated for whom the actions have been brought are, by reason
of the segregation complained of, deprived of the equal protection of
the laws guaranteed by the Fourteenth Amendment. This disposition makes
unnecessary any discussion whether such segregation also violates the
Due Process Clause of the Fourteenth Amendment.

Because these are class actions, because of the wide applicability of
this decision, and because of the great variety of local conditions, the
formulation of decrees in these cases presents problems of considerable
complexity. On reargument, the consideration of appropriate relief was
necessarily subordinated to the primary question---the constitutionality
of segregation in public education. We have now announced that such
segregation is a denial of the equal protection of the laws. In order
that we may have the full assistance of the parties in formulating
decrees, the cases will be restored to the docket, and the parties are
requested to present further argument on Questions 4 and 5 previously
propounded by the Court for the reargument this Term The Attorney
General of the United States is again invited to participate. The
Attorneys General of the states requiring or permitting segregation in
public education will also be permitted to appear as amici curiae upon
request to do so by September 15, 1954, and submission of briefs by
October 1, 1954.

It is so ordered.

\emph{From the Footnotes}:

K. B. Clark, Effect of Prejudice and Discrimination on Personality
Development (Midcentury White House Conference on Children and Youth,
1950); Witmer and Kotinsky, Personality in the Making (1952), c.~VI;
Deutscher and Chein, The Psychological Effects of Enforced Segregation:
A Survey of Social Science Opinion, 26 J.Psychol. 259 (1948); Chein,
What are the Psychological Effects of Segregation Under Conditions of
Equal Facilities?, 3 Int. J. Opinion and Attitude Res. 229 (1949);
Brameld, Educational Costs, in Discrimination and National Welfare
(MacIver, ed., 1949), 44---48; Frazier, The Negro in the United States
(1949), 674---681. And see generally Myrdal, An American Dilemma (1944).

\hypertarget{loving-v.-virginia}{%
\subsubsection{Loving v. Virginia}\label{loving-v.-virginia}}

388 U.S. 1 (1967)

\textbf{Mr.~Chief Justice WARREN delivered the opinion of the Court.}
This case presents a constitutional question never addressed by this
Court: whether a statutory scheme adopted by the State of Virginia to
prevent marriages between persons solely on the basis of racial
classifications violates the Equal Protection and Due Process Clauses of
the Fourteenth Amendment For reasons which seem to us to reflect the
central meaning of those constitutional commands, we conclude that these
statutes cannot stand consistently with the Fourteenth Amendment.

In June 1958, two residents of Virginia, Mildred Jeter, a Negro woman,
and Richard Loving, a white man, were married in the District of
Columbia pursuant to its laws. Shortly after their marriage, the Lovings
returned to Virginia and established their marital abode in Caroline
County. At the October Term, 1958, of the Circuit Court of Caroline
County, a grand jury issued an indictment charging the Lovings with
violating Virginia's ban on interracial marriages. On January 6, 1959,
the Lovings pleaded guilty to the charge and were sentenced to one year
in jail; however, the trial judge suspended the sentence for a period of
25 years on the condition that the Lovings leave the State and not
return to Virginia together for 25 years. He stated in an opinion that:

`Almighty God created the races white, black, yellow, malay and red, and
he placed them on separate continents. And but for the interference with
his arrangement there would be no cause for such marriages. The fact
that he separated the races shows that he did not intend for the races
to mix.'

After their convictions, the Lovings took up residence in the District
of Columbia. On November 6, 1963, they filed a motion in the state trial
court to vacate the judgment and set aside the sentence on the ground
that the statutes which they had violated were repugnant to the
Fourteenth Amendment. The motion not having been decided by October 28,
1964, the Lovings instituted a class action in the United States
District Court for the Eastern District of Virginia requesting that a
three-judge court be convened to declare the Virginia antimiscegenation
statutes unconstitutional and to enjoin state officials from enforcing
their convictions. On January 22, 1965, the state trial judge denied the
motion to vacate the sentences, and the Lovings perfected an appeal to
the Supreme Court of Appeals of Virginia. On February 11, 1965, the
three-judge District Court continued the case to allow the Lovings to
present their constitutional claims to the highest state court.

The Supreme Court of Appeals upheld the constitutionality of the
antimiscegenation statutes and, after modifying the sentence, affirmed
the convictions.

The two statutes under which appellants were convicted and sentenced are
part of a comprehensive statutory scheme aimed at prohibiting and
punishing interracial marriages. The Lovings were convicted of violating
§ 20---58 of the Virginia Code:

`Leaving State to evade law.---If any white person and colored person
shall go out of this State, for the purpose of being married, and with
the intention of returning, and be married out of it, and afterwards
return to and reside in it, cohabiting as man and wife, they shall be
punished as provided in § 20---59, and the marriage shall be governed by
the same law as if it had been solemnized in this State. The fact of
their cohabitation here as man and wife shall be evidence of their
marriage.'

Section 20---59, which defines the penalty for miscegenation, provides:
`Punishment for marriage.---If any white person intermarry with a
colored person, or any colored person intermarry with a white person, he
shall be guilty of a felony and shall be punished by confinement in the
penitentiary for not less than one nor more than five years.'

Other central provisions in the Virginia statutory scheme are § 20---57,
which automatically voids all marriages between `a white person and a
colored person' without any judicial proceeding,3 and §§ 20---54 and
1---14 which, respectively, define `white persons' and `colored persons
and Indians' for purposes of the statutory prohibitions The Lovings have
never disputed in the course of this litigation that Mrs.~Loving is a
`colored person' or that Mr.~Loving is a `white person' within the
meanings given those terms by the Virginia statutes.

Virginia is now one of 16 States which prohibit and punish marriages on
the basis of racial classifications Penalties for miscegenation arose as
an incident to slavery and have been common in Virginia since the
colonial period. 6 The present statutory scheme dates from the adoption
of the Racial Integrity Act of 1924, passed during the period of extreme
nativism which followed the end of the First World War. The central
features of this Act, and current Virginia law, are the absolute
prohibition of a `white person' marrying other than another `white
person,' a prohibition against issuing marriage licenses until the
issuing official is satisfied that the applicants' statements as to
their race are correct,8 certificates of `racial composition' to be kept
by both local and state registrars,9 and the carrying forward of earlier
prohibitions against racial intermarriage.

In upholding the constitutionality of these provisions in the decision
below, the Supreme Court of Appeals of Virginia referred to its 1955
decision in Naim v. Naim, 197 Va. 80, 87 S.E d 749, as stating the
reasons supporting the validity of these laws. In Naim, the state court
concluded that the State's legitimate purposes were `to preserve the
racial integrity of its citizens,' and to prevent `the corruption of
blood,' `a mongrel breed of citizens,' and `the obliteration of racial
pride,' obviously an endorsement of the doctrine of White Supremacy. 87
S.E d.~The court also reasoned that marriage has traditionally been
subject to state regulation without federal intervention, and,
consequently, the regulation of marriage should be left to exclusive
state control by the Tenth Amendment.

While the state court is no doubt correct in asserting that marriage is
a social relation subject to the State's police power, Maynard v.
Hillthe State does not contend in its argument before this Court that
its powers to regulate marriage are unlimited notwithstanding the
commands of the Fourteenth Amendment. Nor could it do so in light of
Meyer v. State of Nebraskaand Skinner v. State of OklahomaInstead, the
State argues that the meaning of the Equal Protection Clause, as
illuminated by the statements of the Framers, is only that state penal
laws containing an interracial element as part of the definition of the
offense must apply equally to whites and Negroes in the sense that
members of each race are punished to the same degree. Thus, the State
contends that, because its miscegenation statutes punish equally both
the white and the Negro participants in an interracial marriage, these
statutes, despite their reliance on racial classifications do not
constitute an invidious discrimination based upon race. The second
argument advanced by the State assumes the validity of its equal
application theory. The argument is that, if the Equal Protection Clause
does not outlaw miscegenation statutes because of their reliance on
racial classifications, the question of constitutionality would thus
become whether there was any rational basis for a State to treat
interracial marriages differently from other marriages. On this
question, the State argues, the scientific evidence is substantially in
doubt and, consequently, this Court should defer to the wisdom of the
state legislature in adopting its policy of discouraging interracial
marriages.

Because we reject the notion that the mere `equal application' of a
statute containing racial classifications is enough to remove the
classifications from the Fourteenth Amendment's proscription of all
invidious racial discriminations, we do not accept the State's
contention that these statutes should be upheld if there is any possible
basis for concluding that they serve a rational purpose. The mere fact
of equal application does not mean that our analysis of these statutes
should follow the approach we have taken in cases involving no racial
discrimination where the Equal Protection Clause has been arrayed
against a statute discriminating between the kinds of advertising which
may be displayed on trucks in New York City, Railway Express Agency,
Inc.~v. People of State of New York§ .Ct. 463, or an exemption in Ohio's
ad valorem tax for merchandise owned by a non-resident in a storage
warehouse, Allied Stores of Ohio, Inc.~v. BowersIn these cases,
involving distinctions not drawn according to race, the Court has merely
asked whether there is any rational foundation for the discriminations,
and has deferred to the wisdom of the state legislatures. In the case at
bar, however, we deal with statutes containing racial classifications,
and the fact of equal application does not immunize the statute from the
very heavy burden of justification which the Fourteenth Amendment has
traditionally required of state statutes drawn according to race. The
State argues that statements in the Thirty-ninth Congress about the time
of the passage of the Fourteenth Amendment indicate that the Framers did
not intend the Amendment to make unconstitutional state miscegenation
laws. Many of the statements alluded to by the State concern the debates
over the Freedmen's Bureau Bill, which President Johnson vetoed, and the
Civil Rights Act of 1866, 14 Stat. 27, enacted over his veto. While
these statements have some relevance to the intention of Congress in
submitting the Fourteenth Amendment, it must be understood that the
pertained to the passage of specific statutes and not to the broader,
organic purpose of a constitutional amendment. As for the various
statements directly concerning the Fourteenth Amendment, we have said in
connection with a related problem, that although these historical
sources `cast some light' they are not sufficient to resolve the
problem; `(a)t best, they are inconclusive. The most avid proponents of
the post-War Amendments undoubtedly intended them to remove all legal
distinctions among 'all persons born or naturalized in the United
States.' Their opponents, just as certainly, were antagonistic to both
the letter and the spirit of the Amendments and wished them to have the
most limited effect.' Brown v. Board of Education of Topeka, See also
Strauder v. State of West Virginia, We have rejected the proposition
that the debates in the Thirty-ninth Congress or in the state
legislatures which ratified the Fourteenth Amendment supported the
theory advanced by the State, that the requirement of equal protection
of the laws is satisfied by penal laws defining offenses based on racial
classifications so long as white and Negro participants in the offense
were similarly punished.

The State finds support for its `equal application' theory in the
decision of the Court in Pace v. State of AlabamaIn that case, the Court
upheld a conviction under an Alabama statute forbidding adultery or
fornication between a white person and a Negro which imposed a greater
penalty than that of a statute proscribing similar conduct by members of
the same race. The Court reasoned that the statute could not be said to
discriminate against Negroes because the punishment for each participant
in the offense was the same. However, as recently as the 1964 Term, in
rejecting the reasoning of that case, we stated `Pace represents a
limited view of the Equal Protection Clause which has not withstood
analysis in the subsequent decisions of this Court.' McLaughlin v.
Florida U.S., As we there demonstrated, the Equal Protection Clause
requires the consideration of whether the classifications drawn by any
statute constitute an arbitrary and invidious discrimination. The clear
and central purpose of the Fourteenth Amendment was to eliminate all
official state sources of invidious racial discrimination in the States.

There can be no question but that Virginia's miscegenation statutes rest
solely upon distinctions drawn according to race. The statutes proscribe
generally accepted conduct if engaged in by members of different races.
Over the years, this Court has consistently repudiated `(d)istinctions
between citizens solely because of their ancestry' as being `odious to a
free people whose institutions are founded upon the doctrine of
equality.' Hirabayashi v. United States, At the very least, the Equal
Protection Clause demands that racial classifications, especially
suspect in criminal statutes, be subjected to the `most rigid scrutiny,'
Korematsu v. United States, and, if they are ever to be upheld, they
must be shown to be necessary to the accomplishment of some permissible
state objective, independent of the racial discrimination which it was
the object of the Fourteenth Amendment to eliminate. Indeed, two members
of this Court have already stated that they `cannot conceive of a valid
legislative purpose which makes the color of a person's skin the test of
whether his conduct is a criminal offense.'

There is patently no legitimate overriding purpose independent of
invidious racial discrimination which justifies this classification. The
fact that Virginia prohibits only interracial marriages involving white
persons demonstrates that the racial classifications must stand on their
own justification, as measures designed to maintain White Supremacy We
have consistently denied the constitutionality of measures which
restrict the rights of citizens on account of race. There can be no
doubt that restricting the freedom to marry solely because of racial
classifications violates the central meaning of the Equal Protection
Clause.

These statutes also deprive the Lovings of liberty without due process
of law in violation of the Due Process Clause of the Fourteenth
Amendment. The freedom to marry has long been recognized as one of the
vital personal rights essential to the orderly pursuit of happiness by
free men.

Marriage is one of the `basic civil rights of man,' fundamental to our
very existence and survival. Skinner v. State of Oklahoma, See also
Maynard v. HillTo deny this fundamental freedom on so unsupportable a
basis as the racial classifications embodied in these statutes,
classifications so directly subversive of the principle of equality at
the heart of the Fourteenth Amendment, is surely to deprive all the
State's citizens of liberty without due process of law. The Fourteenth
Amendment requires that the freedom of choice to marry not be restricted
by invidious racial discriminations. Under our Constitution, the freedom
to marry or not marry, a person of another race resides with the
individual and cannot be infringed by the State.

These convictions must be reversed. It is so ordered. Reversed.

\emph{From the footnotes}:

Section 20---57 of the Virginia Code provides: `Marriages void without
decree.---All marriages between a white person and a colored person
shall be absolutely void without any decree of divorce or other legal
process.' Va.Code Ann. § 20---57 (1960 Repl.Vol.).

Section 20---54 of the Virginia Code provides: `Intermarriage
prohibited; meaning of term 'white persons.' It shall hereafter be
unlawful for any white person in this State to marry any save a white
person, or a person with no other admixture of blood than white and
American Indian. For the purpose of this chapter, the term `white
person' shall apply only to such person as has no trace whatever of any
blood other than Caucasian; but persons who have one-sixteenth or less
of the blood of the American Indian and have no other non-Caucasic blood
shall be deemed to be white persons. All laws heretofore passed and now
in effect regarding the intermarriage of white and colored persons shall
apply to marriages prohibited by this chaper.' Va.Code Ann. § 20---54
(1960 Repl.Vol.).

The exception for persons with less than one-sixteenth `of the blood of
the American Indian' is apparently accounted for, in the words of a
tract issued by the Registrar of the State Bureau of Vital Statistics,
by 'the desire of all to recognize as an integral and honored part of
the white race the descendants of John Rolfe and Pocahontas * * *.'
Plecker, The New Family and Race Improvement, 17 Va.Health Bull., Extra
No.~12---26 (New Family Series No.~5, 1925), cited in Wadlington, The
Loving Case; Virginia's Anti-Miscegenation Statute in Historical
Perspective, 52 Va.L.Rev. 1189, 1202, n.~93 (1966).

Section 1---14 of the Virginia Code provides:

Colored persons and Indians defined.---Every person in whom there is
ascertainable any Negro blood shall be deemed and taken to be a colored
person, and every person not a colored person having one fourth or more
of American Indian blood shall be deemed an American Indian; except that
members of Indian tribes existing in this Commonwealth having one fourth
or more of Indian blood and less than one sixteenth of Negro blood shall
be deemed tribal Indians.' Va.Code Ann. § 1---14 (1960 Repl.Vol.).

Appellants point out that the State's concern in these statutes, as
expressed in the words of the 1924 Act's title, `An Act to Preserve
Racial Integrity,' extends only to the integrity of the white race.
While Virginia prohibits whites from marrying any nonwhite (subject to
the exception for the descendants of Pocahontas), Negroes, Orientals,
and any other racial class may intermarry without statutory
interference. Appellants contend that this distinction renders
Virginia's miscegenation statutes arbitrary and unreasonable even
assuming the constitutional validity of an official purpose to preserve
`racial integrity.' We need not reach this contention because we find
the racial classifications in these statutes repugnant to the Fourteenth
Amendment, even assuming an even-handed state purpose to protect the
`integrity' of all races.

\hypertarget{johnson-v.-california}{%
\subsubsection{Johnson v. California}\label{johnson-v.-california}}

543 U.S. 499 (2005)

\textbf{Justice O'Connor delivered the opinion of the Court.}

The California Department of Corrections (CDC) has an unwritten policy
of racially segregating prisoners in double cells in reception centers
for up to 60 days each time they enter a new correctional facility. We
consider whether strict scrutiny is the proper standard of review for an
equal protection challenge to that policy.

CDC institutions house all new male inmates and all male inmates
transferred from other state facilities in reception centers for up to
60 days upon their arrival. During that time, prison officials evaluate
the inmates to determine their ultimate placement. Double-cell
assignments in the reception centers are based on a number of factors,
predominantly race. In fact, the CDC has admitted that the chances of an
inmate being assigned a cellmate of another race are `` `{[}p{]}retty
close' '' to zero percent. App. to Pet. for Cert. 3a. The CDC further
subdivides prisoners within each racial group. Thus, Japanese-Americans
are housed separately from Chinese-Americans, and northern California
Hispanics are separated from southern California Hispanics.

The CDC's asserted rationale for this practice is that it is necessary
to prevent violence caused by racial gangs. Brief for Respondents 1-6.
It cites numerous incidents of racial violence in CDC facilities and
identifies five major prison gangs in the State: Mexican Mafia, Nuestra
Familia, Black Guerilla Family, Aryan Brotherhood, and Nazi Low Riders.
The CDC also notes that prison-gang culture is violent and murderous. An
associate warden testified that if race were not considered in making
initial housing assignments, she is certain there would be racial
conflict in the cells and in the yard. App. 215a. Other prison officials
also expressed their belief that violence and conflict would result if
prisoners were not segregated. See, e. g., a-306a. The CDC claims that
it must therefore segregate all inmates while it determines whether they
pose a danger to others. See Brief for Respondents 29.

With the exception of the double cells in reception areas, the rest of
the state prison facilities --- dining areas, yards, and cells --- are
fully integrated. After the initial 60-day period\^{} prisoners are
allowed to choose their own cellmates. The CDC usually grants inmate
requests to be housed together, unless there are security reasons for
denying them.

We have held that ``all racial classifications {[}imposed by
government{]} . must be analyzed by a reviewing court under strict
scrutiny.'' Adarand Constructors, Inc.~v. Peña (emphasis added). Under
strict scrutiny, the government has the burden of proving that racial
classifications ``are narrowly tailored measures that further compelling
governmental interests.'' We have insisted on strict scrutiny in every
context, even for so-called ``benign'' racial classifications, such as
race-conscious university admissions policies, see Grutter v. Bollinger,
race-based preferences in government contracts, see Adarand, and
race-based districting intended to improve minority representation, see
Shaw v. Reno.

The reasons for strict scrutiny are familiar. Racial classifications
raise special fears that they are motivated by an invidious purpose.
Thus, we have admonished time and again that, ``{[}ajbsent searching
judicial inquiry into the justification for such race-based measures,
there is simply no way of determining \ldots{} what classifications are
in fact motivated by illegitimate notions of racial inferiority or
simple racial politics.'' Richmond v. J. A. Croson Co.~(plurality
opinion). We therefore apply strict scrutiny to all racial
classifications to ``\,`smoke out' illegitimate uses of race by assuring
that {[}government{]} is pursuing a goal important enough to warrant use
of a highly suspect tool.''

The CDC claims that its policy should be exempt from our categorical
rule because it is ``neutral'' --- that is, it ``neither benefits nor
burdens one group or individual more than any other group or
individual.'' Brief for Respondents 16. In other words, strict scrutiny
should not apply because all prisoners are ``equally'' segregated. The
CDC's argument ignores our repeated command that ``racial
classifications receive close scrutiny even when they may be said to
burden or benefit the races equally.'' Shaw. Indeed, we rejected the
notion that separate can ever be equal --- or ``neutral'' --- 50 years
ago in Brown v. Board of Education, and we refuse to resurrect it today.
See also Powers v. Ohio (rejecting the argument that race-based
peremptory challenges were permissible because they applied equally to
white and black jurors and holding that ``{[}i{]}t is axiomatic that
racial classifications do not become legitimate on the assumption that
all persons suffer them in equal degree'').

We have previously applied a heightened standard of review in evaluating
racial segregation in prisons. In Lee v. Washington (per curiam), we
upheld a three-judge court's decision striking down Alabama's policy of
segregation in its prisons. -334. Alabama had argued that desegregation
would undermine prison security and discipline, but we rejected that
contention. Three Justices concurred ``to make explicit something that
is left to be gathered only by implication from the Court's opinion''
--- ``that prison authorities have the right, acting in good faith and
in particularized circumstances, to take into account racial tensions in
maintaining security, discipline, and good order in prisons and jails.''
Ibid, (emphasis added). The concurring Justices emphasized that they
were ``unwilling to assume that state or local prison authorities might
mistakenly regard such an explicit pronouncement as evincing any
dilution of this Court's firm commitment to the Fourteenth Amendment's
prohibition of racial discrimination.''

The need for strict scrutiny is no less important here, where prison
officials cite racial violence as the reason for their policy. As we
have recognized in the past, racial classifications ``threaten to
stigmatize individuals by reason of their membership in a racial group
and to incite racial hostility.'' Shaw (citing J. A. Croson
Co.~(plurality opinion); emphasis added). Indeed, by insisting that
inmates be housed only with other inmates of the same race, it is
possible that prison officials will breed further hostility among
prisoners and reinforce racial and ethnic divisions. By perpetuating the
notion that race matters most, racial segregation of inmates ``may
exacerbate the very patterns of {[}violence that it is{]} said to
counteract.'' Shaw; see also Trulson \& Marquart, The Caged Melting Pot:
Toward an Understanding of the Consequences of Desegregation in Prisons,
36 Law \& Soc. Rev.~743, 774 (2002) (in a study of prison desegregation,
finding that ``over {[}10 years{]} the rate of violence between inmates
segregated by race in double cells surpassed the rate among those
racially integrated''). See also Brief for Former State Corrections
Officials as Amici Curiae 19 (opinion of former corrections officials
from six States that ``racial integration of cells tends to diffuse
racial tensions and thus diminish interracial violence'' and that ``a
blanket policy of racial segregation of inmates is contrary to sound
prison management'').

The CBC's policy is unwritten. Although California claimed at oral
argument that two other States follow a similar policy, see Tr. of Oral
Arg. 30-31, this assertion was unsubstantiated, and we are unable to
confirm or deny its accuracy Virtually all other States and the Federal
Government manage their prison systems without reliance on racial
segregation. See Brief for United States as Amicus Curiae 24. Federal
regulations governing the Federal Bureau of Prisons (BOP) expressly
prohibit racial segregation. 28 CFR § 551 (2004) (``{[}BOP{]} staff
shall not discriminate against inmates on the basis of race, religion,
national origin, sex, disability, or political belief. This includes the
making of administrative decisions and providing access to work, housing
and programs''). The United States contends that racial integration
actually ``leads to less violence in BOP's institutions and better
prepares inmates for re-entry into society.'' Brief for United States as
Amicus Curiae 25. Indeed, the United States argues, based on its
experience with the BOP, that it is possible to address ``concerns of
prison security through individualized consideration without the use of
racial segregation, unless warranted as a necessary and temporary
response to a race riot or other serious threat of race-related
violence.'' As to transferees, in particular, whom the CDC has already
evaluated at least once, it is not clear why more individualized
determinations are not possible.

Because the CDC's policy is an express racial classification, it is
``immediately suspect.'' Shaw; see also Washington v. Seattle School
Dist. No.~1. We therefore hold that the Court of Appeals erred when it
failed to apply strict scrutiny to the CDC's policy and to require the
CDC to demonstrate that its policy is narrowly tailored to serve a
compelling state interest.

The CDC invites us to make an exception to the rule that strict scrutiny
applies to all racial classifications, and instead to apply the
deferential standard of review articulated in Turner v. Safley, because
its segregation policy applies only in the prison context. We decline
the invitation. In Turner, we considered a claim by Missouri prisoners
that regulations restricting inmate marriages and inmate-to-inmate
correspondence were unconstitutional. We rejected the prisoners'
argument that the regulations should be subject to strict scrutiny,
asking instead whether the regulation that burdened the prisoners'
fundamental rights was ``reasonably related'' to ``legitimate
penological interests.''

We have never applied Turner to racial classifications. Turner itself
did not involve any racial classification, and it cast no doubt on Lee.
We think this unsurprising, as we have applied Turner's
reasonable-relationship test only to rights that are ``inconsistent with
proper incarceration.'' Overton v. Bazzetta; see also Pell v. Procunier
(``{[}A{]} prison inmate retains those First Amendment rights that are
not inconsistent with his status as a prisoner or with the legitimate
penological objectives of the corrections system''). This is because
certain privileges and rights must necessarily be limited in the prison
context. See O'Lone v. Estate of Shabazz (`` `{[}Ljawful incarceration
brings about the necessary withdrawal or limitation of many privileges
and rights, a retraction justified by the considerations underlying our
penal system' '' (quoting Price v. Johnston)). Thus, for example, we
have relied on Turner in addressing First Amendment challenges to prison
regulations, including restrictions on freedom of association, Overton;
limits on inmate correspondence, Shaw v. Murphy; restrictions on
inmates' access to courts, Lewis v. Casey; restrictions on receipt of
subscription publications, Thornburgh v. Abbott; and work rules limiting
prisoners' attendance at religious services, Shabazz. We have also
applied Turner to some due process claims, such as involuntary
medication of mentally ill prisoners, Washington v. Harper; and
restrictions on the right to marry, Turner.

The right not to be discriminated against based on one's race is not
susceptible to the logic of Turner. It is not a right that need
necessarily be compromised for the sake of proper prison administration.
On the contrary, compliance with the Fourteenth Amendment's ban on
racial discrimination is not only consistent with proper prison
administration, but also bolsters the legitimacy of the entire criminal
justice system. Race discrimination is ``especially pernicious in the
administration of justice.'' Rose v. Mitchell. And public respect for
our system of justice is undermined when the system discriminates based
on race. Cf. Batson v. Kentucky (``{[}Pjublic respect for our criminal
justice system and the rule of law will be strengthened if we ensure
that no citizen is disqualified from jury service because of his
race''). When government officials are permitted to use race as a proxy
for gang membership and violence without demonstrating a compelling
government interest and proving that their means are narrowly tailored,
society as a whole suffers. For similar reasons, we have not used Turner
to evaluate Eighth Amendment claims of cruel and unusual punishment in
prison. We judge violations of that Amendment under the ``deliberate
indifference'' standard, rather than Turner's ``reasonably related''
standard. See Hope v. Pelzer (asking whether prison officials displayed
`` `deliberate indifference' to the inmates' health or safety'' where an
inmate claimed that they violated his rights under the Eighth Amendment
(quoting Hudson v. McMillian)). This is because the integrity of the
criminal justice system depends on full compliance with the Eighth
Amendment. See Spain v. Procunier, -194 (CA9 1979) (Kennedy, J.)
(``{[}T{]}he full protections of the eighth amendment most certainly
remain in force {[}in prison{]}. The whole point of the amendment is to
protect persons convicted of crimes. . Mechanical deference to the
findings of state prison officials in the context of the eighth
amendment would reduce that provision to a nullity in precisely the
context where it is most necessary'').

In the prison context, when the government's power is at its apex, we
think that searching judicial review of racial classifications is
necessary to guard against invidious discrimination. Granting the CDC an
exemption from the rule that strict scrutiny applies to all racial
classifications would undermine our ``unceasing efforts to eradicate
racial prejudice from our criminal justice system.'' McCleskey v. Kemp
(internal quotation marks omitted).

We did not relax the standard of review for racial classifications in
prison in Lee, and we refuse to do so today. Rather, we explicitly
reaffirm what we implicitly held in Lee: The ``necessities of prison
security and discipline,'' 390 U. S., are a compelling government
interest justifying only those uses of race that are narrowly tailored
to address those necessities. See Grutter (Thomas, J., concurring in
part and dissenting in part) (citing Lee for the principle that
``protecting prisoners from violence might justify narrowly tailored
racial discrimination''); J. A. Croson Co.~(Scalia, J., concurring in
judgment) (citing Lee for the proposition that ``only a social emergency
rising to the level of imminent danger to life and limb --- for example,
a prison race riot, requiring temporary segregation of inmates --- can
justify an exception to the principle embodied in the Fourteenth
Amendment that `{[}o{]}ur Constitution is color-blind, and neither knows
nor tolerates classes among citizens'\,'' (quoting Plessy v. Ferguson
(Harlan, J., dissenting))); see also Pell (``{[}C{]}entral to all other
corrections goals is the institutional consideration of internal
security within the corrections facilities themselves'').

The CDC protests that strict scrutiny will handcuff prison
administrators and render them unable to address legitimate problems of
race-based violence in prisons. See also post, 546-547 (Thomas, J.,
dissenting). Not so. Strict scrutiny is not ``strict in theory, but
fatal in fact.'' Adarand (internal quotation marks omitted); Grutter
(``Although all governmental uses of race are subject to strict
scrutiny, not all are invalidated by it''). Strict scrutiny does not
preclude the ability of prison officials to address the compelling
interest in prison safety. Prison administrators, however, will have to
demonstrate that any race-based policies are narrowly tailored to that
end. See (``When race-based action is necessary to further a compelling
governmental interest, such action does not violate the constitutional
guarantee of equal protection so long as the narrow-tailoring
requirement is also satisfied'').

The fact that strict scrutiny applies ``says nothing about the ultimate
validity of any particular law; that determination is the job of the
court applying strict scrutiny.'' Adarand. At this juncture, no such
determination has been made. On remand, the CDC will have the burden of
demonstrating that its policy is narrowly tailored with regard to new
inmates as well as transferees. Prisons are dangerous places, and the
special circumstances they present may justify racial classifications in
some contexts. Such circumstances can be considered in applying strict
scrutiny, which is designed to take relevant differences into account.

We do not decide whether the CDC's policy violates the Equal Protection
Clause. We hold only that strict scrutiny is the proper standard of
review and remand the case to allow the Court of Appeals for the Ninth
Circuit, or the District Court, to apply it in the first instance. See
Consolidated Rail Corporation v. Gottshall (1994) (reversing and
remanding for the lower court to apply the correct legal standard in the
first instance); Lucas v. South Carolina Coastal Council (1992) (same).
The judgment of the Court of Appeals is reversed, and the case is
remanded for further proceedings consistent with this opinion.

\hypertarget{yick-wo-v.-hopkins}{%
\subsubsection{Yick Wo v. Hopkins}\label{yick-wo-v.-hopkins}}

118 U.S. 356 (1886)

The ordinances for the violation of which he had been found guilty were
set out as follows: Order No.~1569, passed May 26, 1880, prescribing the
kind of buildings in which laundries may be- located.

`` The people of the city and county of San Francisco do ordain as
follows: `` Sec. 1. It shall be unlawful, from and after the passage of
this order, for any person or persons to establish, maintain, or carry
on a laundry within the corporate limits of the city and county of San
Francisco without having first obtained the consent of the board of
supervisors, except the same be located in a building constructed either
of brick or stone.

It was alleged in the petition, that `` your petitioner and more than
one hundred and fifty of his countrymen have been arrested upon the
charge of carrying on business without having such special consent,
while those who are not subjects of China, and who are conducting,
eighty odd laundries under similar conditions, are left unmolested and
free to enjoy the enhanced trade and profits arising from this hurtful
and unfair discrimination. The business of your petitioner, and of those
of his countrymen similarly .situated, is greatly impaired, and in many
cases practically ruined by this system of oppression to one kind of men
and favoritism to all others.''

It was also admitted `` that petitioner and 200 of his countrymen
similarly situated petitipne'd the board of supervisors for permission
to continue their business in the various houses which they had been
occupying and using for laundries for more than twenty years, and such
petitions -were denied, and all the petitions of those who were not
Chinese, with one exception of Mrs.~Mary Meagles, were granted.''

\textbf{Mr.~Justice Matthews delivered the opinion of the court.}

We are consequently constrained, at the outset, to differ from the
Supreme Court of California upon the real meaning of the ordinances in
question. That court considered these ordinances as vesting in the board
of supervisors a not unusual discretion in granting or withholding their
assent to the use of wooden buildings as laundries, to be exercised in
reference to the circumstances of each case, with a view to the
protection of the public against the dangers of fire. We are not able to
concur in that interpretation of the power conferred upon the
supervisors. There is nothing in the ordinances which points to such a
regulation of the business of keeping and conducting laundries. They
seem intended to confer, and actually do confer, not a discretion to be
exercised upon a consideration of the circumstances of each case, but a
naked and arbitrary power to give or withhold consent, not only as to
places, but as to persons. So that, if an applicant for such consent,
being in every way a competent and qualified person, and having complied
with every reasonable condition demanded by any public \^{}interest,
should, failing to obtain the requisite consent of the supervisors to
the prosecution of his business, apply for redress by the judicial
process of mandamus, to require the supervisors to consider and act upon
his case, it would be a sufficient answer for them to say that the law
had conferred upon them authority to withhold their assent, without
reason and without responsibility. The power given to them is not
confided to their discretion in the legal sense of that term, but is
granted to their mere will. It is purely arbitrary, and acknowledges
neither guidance nor restraint.

The rights of the petitioners, as affected by the proceedings of which
they complain, are not less, because they are aliens and subjects of the
Emperor of China. By the third article of the treaty between this
Government and that of China, concluded November 17, 1880, 22 Stat. 827,
it is stipulated: `` If Chinese laborers, or Chinese of any other class,
now either permanently or temporarily residing in the territory of the
United States, meet with ill treatment at the hands of any other
persons, the Government of the United States will exert all its powers
to devise measures for their protection, and to secure to them the same
rights, privileges, immunities and exemptions as may be enjoyed by the
citizens or subjects of the most favored nation, and to which they are
entitled by treaty.''

The Fourteenth Amendment to the Constitution is not confined to the
protection of citizens. It says : `` Nor shall any State deprive any
person of life, liberty, or property without due process of law ; nor
deny to any person within its jurisdiction the equal protection of the
laws.'' These provisions are universal in their application, to all
persons within the territorial jurisdiction, without regard to any
differences of race, of color, or of nationality ; and the equal
protection of the laws is a pledge of the protection of equal laws. It
is accordingly enacted by § 1977 of the Kevised Statutes, that `` all
persons within the jurisdiction of the United States shall have the same
right in every State and Territory to make and enforce contracts, to
sue, be parties, give evidence, and to the full and equal benefit of all
laws and proceedings .for the security of persons and property as is
enjoyed by white citizens and shall be subject to like punishment,
pains, penalties, taxes, licenses, and exactions of every kind, and to
no other.'' The questions we have to consider and decide in these cases,
therefore, are to be treated as involving the rights .of every citizen
of the United States equally with those of the strangers and aliens who
now invoke the jurisdiction of the court.

It is contended on the part of the petitioners, that the ordinances for
violations of which they are severally sentenced to imprisonment, are
void on their face, as being within the prohibitions of the Fourteenth
Amendment; and, in the alternative, if not so, that they are void by
reason of their administration, operating unequally, so as to punish in
the present petitioners what is permitted to others as lawful, without
any distinction of circumstances --- an unjust and illegal
discrimination, it is claimed, which, though not made expressly by the
ordinances is made possible by them.

When we consider the nature and the theory of our institutions of
government, the principles upon which they are sup.posed to rest, and
review the history of their development, we are constrained to conclude
that they do not mean to leave room for the play and action of purely
personal and arbitrary power. Sovereignty itself is, of course, not
subject to law, for it is the author and source of law; but in our
system, while sovereign powers are delegated to the agencies of
government, sovereignty itself remains with the people, by whom and for
whom all government exists and acts. And the law is the definition and
limitation of power. It is, indeed, quite true, that there must always
be lodged somewhere, and in some person or body, the authority of final
decision; and in many cases of mere administration the responsibility is
purely'political, no appeal lying except to the ultimate tribunal of the
public judgment, exercised either in the pressure of opinion or by means
of the suffrage. But the fundamental rights to life, liberty, and the
pursuit of happiness, considered as individual possessions, are secured
by those maxims of constitutional law which are the monuments showing
the victorious progress of the race in securing to men the blessings of
civilization under the reign of just and equal laws, so that, in the
famous language of the Massachusetts Bill of Rights, the government of
the commonwealth ``may be a government of laws and not of men.'' For,
the very idea that one man may be compelled to hold his life, or the
means of living, or any material right essential to the enjoyment of
life, at the mere will of another, seems to be intolerable in any
country where freedom prevails, as being the essence of slavery itself.

In the present cases we are not obliged to reason from the probable to
the actual, and pass upon the validity of the ordinances complained of,
as tried merely by the opportunities which their terms afford, of
unequal and unjust discrimination in their administration. For the cases
present the ordinances in actual operation, and the facts shown
establish an administration directed so exclusively against a particular
class1 of persons as to warrant and require the conclusion, that,
whatever may have been the intent of the ordinances as adopted, they are
applied by the public authorities charged with their administration, and
thus representing the State itself, wfth a mind so unequal and
oppressive as to amount to a practical denial by the State of that equal
protection of the laws which is secured to the petitioners, as to all
other persons, by the broad and benign provisions of the Fourteenth
Amendment to the Constitution of the United States. Though the law
itself be fair on its face and impartial in appearance, yet, if it is
applied and administered by public authority with an evil eye and an
unequal hand, so as practically to make unjust and illegal
discriminations between persons in similar circumstances, material to
their rights, the denial of equal justice is still within the
prohibition of the Constitution. This principle of interpretation has
been sanctioned by this court in Henderson v. Mayor of New York; Chy
Lung v. Freeman; Ex parte Virginia; Neal v. Delaware; and Soon Hing v.
Crowley.

The judgment of the Supreme Court of California in the case of Yick Wo,
and that of the Girouit Court of the United States for the District of
Odlifornia ini the case of Wo Lee, are severally reversed, and the cases
remanded, each to the pi'oper court, with directions to discharge the
petitioners from custody cmd imprisonment.

\hypertarget{washington-v.-davis}{%
\subsubsection{Washington v. Davis}\label{washington-v.-davis}}

426 U.S. 229 (1976)

\textbf{Mr.~Justice WHITE delivered the opinion of the Court.}

This case involves the validity of a qualifying test administered to
applicants for positions as police officers in the District of Columbia
Metropolitan Police Department. The test was sustained by the District
Court but invalidated by the Court of Appeals. We are in agreement with
the District Court and hence reverse the judgment of the Court of
Appeals.

This action began on April 10, 1970, when two Negro police officers
filed suit against the then Commissioner of the District of Columbia,
the Chief of the District's Metropolitan Police Department, and the
Commissioners of the United States Civil Service Commission An amended
complaint, filed December 10, alleged that the promotion policies of the
Department were racially discriminatory and sought a declaratory
judgment and an injunction. The respondents Harley and Sellers were
permitted to intervene, their amended complaint assert- ing that their
applications to become officers in the Department had been rejected, and
that the Department's recruiting procedures discriminated on the basis
of race against black applicants by a series of practices including, but
not limited to, a written personnel test which excluded a
disproportionately high number of Negro applicants. These practices were
asserted to violate respondents' rights ``under the due process clause
of the Fifth Amendment to the United States Constitution, under 42
U.S.C. § 1981 and under D.C.Code § 1-320.'' Defendants answered, and
discovery and various other proceedings followed Respondents then filed
a motion for partial summary judgment with respect to the recruiting
phase of the case, seeking a declaration that the test administered to
those applying to become police officers is ``unlawfully discriminatory
and thereby in violation of the due process clause of the Fifth
Amendment . .'' No issue under any statute or regulation was raised by
the motion. The District of Columbia defendants, petitioners here, and
the federal parties also filed motions for summary judgment with respect
to the recruiting aspects of the case, asserting that respondents were
entitled to relief on neither constitutional nor statutory grounds The
District Court granted petitioners' and denied respondents' motions. 348
F.Supp. 15 (DC1972).

According to the findings and conclusions of the District Court, to be
accepted by the Department and to enter an intensive 17-week training
program, the police recruit was required to satisfy certain physical and
character standards, to be a high school graduate or its equivalent, and
to receive a grade of at least 40 out of 80 on ``Test 21,'' which is
``an examination that is used generally throughout the federal
service,'' which ``was developed by the Civil Service Commission, not
the Police Department,'' and which was ``designed to test verbal
ability, vocabulary, reading and comprehension.''

The validity of Test 21 was the sole issue before the court on the
motions for summary judgment. The District Court noted that there was no
claim of ``an intentional discrimination or purposeful discriminatory
acts'' but only a claim that Test 21 bore no relationship to job
performance and ``has a highly discriminatory impact in screening out
black candidates.'' Respondents' evidence, the District Court said,
warranted three conclusions: ``(a) The number of black police officers,
while substantial, is not proportionate to the population mix of the
city. (b) A higher percentage of blacks fail the Test than whites. (c)
The Test has not been validated to establish its reliability for
measuring subsequent job performance.'' This showing was deemed
sufficient to shift the burden of proof to the defendants in the action,
petitioners here; but the court nevertheless concluded that on the
undisputed facts respondents were not entitled to relief. The District
Court relied on several factors. Since August 1969, 44\% Of new police
force recruits had been black; that figure also represented the
proportion of blacks on the total force and was roughly equivalent to
20- to 29-year-old blacks in the 50-mile radius in which the recruiting
efforts of the Police Department had been concentrated. It was
undisputed that the Department had systematically and affirmatively
sought to enroll black officers many of whom passed the test but failed
to report for duty. The District Court rejected the assertion that Test
21 was culturally slanted to favor whites and was ``satisfied that the
undisputable facts prove the test to be reasonably and directly related
to the requirements of the police recruit training program and that it
is neither so designed nor operates (Sic ) to discriminate against
otherwise qualified blacks' It was thus not necessary to show that Test
21 was not only a useful indicator of training school performance but
had also been validated in terms of job performance''The lack of job
performance validation does not defeat the Test, given its direct
relationship to recruiting and the valid part it plays in this process."
The District Court ultimately concluded that ``(t)he proof is wholly
lacking that a police officer qualifies on the color of his skin rather
than ability'' and that the Department ``should not be required on this
showing to lower standards or to abandon efforts to achieve
excellence.''

Having lost on both constitutional and statutory issues in the District
Court, respondents brought the case to the Court of Appeals claiming
that their summary judgment motion, which rested on purely
constitutional grounds, should have been granted. The tendered
constitutional issue was whether the use of Test 21 invidiously
discriminated against Negroes and hence denied them due process of law
contrary to the commands of the Fifth Amendment. The Court of Appeals,
addressing that issue, announced that it would be guided by Griggs v.
Duke Power Co.a case involving the interpretation and application of
Title VII of the Civil Rights Act of 1964, and held that the statutory
standards elucidated in that case were to govern the due process
question tendered in this one.

The court went on to declare that lack of discriminatory intent in
designing and administering Test 21 was irrelevant; the critical fact
was rather that a far greater proportion of blacks four times as many
failed the test than did whites. This disproportionate impact, standing
alone and without regard to whether it indicated a discriminatory
purpose, was held sufficient to establish a constitutional violation,
absent proof by petitioners that the test was an adequate measure of job
performance in addition to being an indicator of probable success in the
training program, a burden which the court ruled petitioners had failed
to discharge. That the Department had made substantial efforts to
recruit blacks was held beside the point and the fact that the racial
distribution of recent hirings and of the Department itself might be
roughly equivalent to the racial makeup of the surrounding community,
broadly conceived, was put aside as a ``comparison (not) material to
this appeal.'' n.~24, 512 F d n.~24. The Court of Appeals, over a
dissent, accordingly reversed the judgment of the District Court and
directed that respondents' motion for partial summary judgment be
granted. We granted the petition for certiorarifiled by the District of
Columbia officials.

Because the Court of Appeals erroneously applied the legal standards
applicable to Title VII cases in resolving the constitutional issue
before it, we reverse its judgment in respondents' favor. Although the
petition for certiorari did not present this ground for reversal,8 our
Rule 40(1)(d)(2) provides that we ``may notice a plain error not
presented''; 9 and this is an appropriate occasion to invoke the Rule.

As the Court of Appeals understood Title VII, employees or applicants
proceeding under it need not concern themselves with the employer's
possibly discriminatory purpose but instead may focus solely on the
racially differential impact of the challenged hiring or promotion
practices. This is not the constitutional rule. We have never held that
the constitutional standard for adjudicating claims of invidious racial
discrimination is identical to the standards applicable under Title VII,
and we decline to do so today.

The central purpose of the Equal Protection Clause of the Fourteenth
Amendment is the prevention of official conduct discriminating on the
basis of race. It is also true that the Due Process Clause of the Fifth
Amendment contains an equal protection component prohibiting the United
States from invidiously discriminating between individuals or groups.
Bolling v. SharpeBut our cases have not embraced the proposition that a
law or other official act, without regard to whether it reflects a
racially discriminatory purpose, is unconstitutional Solely because it
has a racially disproportionate impact.

Almost 100 years ago, Strauder v. West VirginiaL.Ed. 664 (1880),
established that the exclusion of Negroes from grand and petit juries in
criminal proceedings violated the Equal Protection Clause, but the fact
that a particular jury or a series of juries does not statistically
reflect the racial composition of the community does not in itself make
out an invidious discrimination forbidden by the Clause. ``A purpose to
discriminate must be present which may be proven by systematic exclusion
of eligible jurymen of the proscribed race or by unequal application of
the law to such an extent as to show intentional discrimination.'' Akins
v. Texas, 1696 (1945). A defendant in a criminal case is entitled ``to
require that the State not deliberately and systematically deny to
members of his race the right to participate as jurors in the
administration of justice.'' Alexander v. Louisiana.

The rule is the same in other contexts. Wright v. Rockefellerupheld a
New York congressional apportionment statute against claims that
district lines had been racially gerrymandered. The challenged districts
were made up predominantly of whites or of minority races, and their
boundaries were irregularly drawn. The challengers did not prevail
because they failed to prove that the New York Legislature ``was either
motivated by racial considerations or in fact drew the districts on
racial lines''; the plaintiffs had not shown that the statute ``was the
product of a state contrivance to segregate on the basis of race or
place of origin.'' 58, 11 L.Ed d.~The dissenters were in agreement that
the issue was whether the ``boundaries . were purposefully drawn on
racial lines.''

The school desegregation cases have also adhered to the basic equal
protection principle that the invidious quality of a law claimed to be
racially discriminatory must ultimately be traced to a racially
discriminatory purpose. That there are both predominantly black and
predominantly white schools in a community is not alone violative of the
Equal Protection Clause. The essential element of De jure segregation is
``a current condition of segregation resulting from intentional state
action. Keyes v. School Dist. No.~1, The differentiating factor between
De jure segregation and so-called De facto segregation . is Purpose or
Intent to segregate.'' 37 L.Ed d.~See also 211, 213, 2698, 2699, 37 L.Ed
d, 564, 566. The Court has also recently rejected allegations of racial
discrimination based solely on the statistically disproportionate racial
impact of various provisions of the Social Security Act because ``(t)he
acceptance of appellants' constitutional theory would render suspect
each difference in treatment among the grant classes, however lacking in
racial motivation and however otherwise rational the treatment might
be.'' Jefferson v. Hackney, 297 (1972).

This is not to say that the necessary discriminatory racial purpose must
be express or appear on the face of the statute, or that a law's
disproportionate impact is irrelevant in cases involving
Constitution-based claims of racial discrimination. A statute, otherwise
neutral on its face, must not be applied so as invidiously to
discriminate on the basis of race. Yick Wo v. HopkinsIt is also clear
from the cases dealing with racial discrimination in the selection of
juries that the systematic exclusion of Negroes is itself such an
``unequal application of the law . as to show intentional
discrimination.'' Akins v. Texas L.Ed.. Smith v. Texas; Pierre v.
Louisiana; Neal v. DelawareL.Ed. 567 (1881). A prima facie case of
discriminatory purpose may be proved as well by the absence of Negroes
on a particular jury combined with the failure of the jury commissioners
to be informed of eligible Negro jurors in a community, Hill v. Texas,
1562 (1942), or with racially non-neutral selection procedures,
Alexander v. Louisiana; Avery v. Georgia; Whitus v. GeorgiaWith a prima
facie case made out, ``the burden of proof shifts to the State to rebut
the presumption of unconstitutional action by showing that permissible
racially neutral selection criteria and procedures have produced the
monochromatic result.''

Necessarily, an invidious discriminatory purpose may often be inferred
from the totality of the relevant facts, including the fact, if it is
true, that the law bears more heavily on one race than another. It is
also not infrequently true that the discriminatory impact in the jury
cases for example, the total or seriously disproportionate exclusion of
Negroes from jury venires may for all practical purposes demonstrate
unconstitutionality because in various circumstances the discrimination
is very difficult to explain on nonracial grounds. Nevertheless, we have
not held that a law, neutral on its face and serving ends otherwise
within the power of government to pursue, is invalid under the Equal
Protection Clause simply because it may affect a greater proportion of
one race than of another. Disproportionate impact is not irrelevant, but
it is not the sole touchstone of an invidious racial discrimination
forbidden by the Constitution. Standing alone, it does not trigger the
rule, McLaughlin v. Florida that racial classifications are to be
subjected to the strictest scrutiny and are justifiable only by the
weightiest of considerations.

There are some indications to the contrary in our cases. In Palmer v.
Thompsonthe city of Jackson, Miss., following a court decree to this
effect, desegregated all of its public facilities save five swimming
pools which had been operated by the city and which, following the
decree, were closed by ordinance pursuant to a determination by the city
council that closure was necessary to preserve peace and order and that
integrated pools could not be economically operated. Accepting the
finding that the pools were closed to avoid violence and economic loss,
this Court rejected the argument that the abandonment of this service
was inconsistent with the outstanding desegregation decree and that the
otherwise seemingly permissible ends served by the ordinance could be
impeached by demonstrating that racially invidious motivations had
prompted the city council's action. The holding was that the city was
not overtly or covertly operating segregated pools and was extending
identical treatment to both whites and Negroes. The opinion warned
against grounding decision on legislative purpose or motivation, thereby
lending support for the proposition that the operative effect of the law
rather than its purpose is the paramount factor. But the holding of the
case was that the legitimate purposes of the ordinance to preserve peace
and avoid deficits were not open to impeachment by evidence that the
councilmen were actually motivated by racial considerations. Whatever
dicta the opinion may contain, the decision did not involve, much less
invalidate, a statute or ordinary having neutral purposes but
disproportionate racial consequences.

Wright v. Council of City of Emporia also indicates that in proper
circumstances, the racial impact of a law, rather than its
discriminatory purpose, is the critical factor. That case involved the
division of a school district. The issue was whether the division was
consistent with an outstanding order of a federal court to desegregate
the dual school system found to have existed in the area. The
constitutional predicate for the District Court's invalidation of the
divided district was ``the enforcement until 1969 of racial segregation
in a public school system of which Emporia had always been a part.''
There was thus no need to find ``an independent constitutional
violation.'' Citing Palmer v. Thompson, we agreed with the District
Court that the division of the district had the effect of interfering
with the federal decree and should be set aside.

That neither Palmer Nor Wright was understood to have changed the
prevailing rule is apparent from Keyes v. School Dist. No.~1where the
principal issue in litigation was whether to what extent there had been
purposeful discrimination resulting in a partially or wholly segregated
school system. Nor did other later cases, Alexander v. Louisianaand
Jefferson v. Hackneyindicate that either Palmer or Wright had worked a
fundamental change in equal protection law.

Both before and after Palmer v. Thompson, however, various Courts of
Appeals have held in several contexts, including public employment, that
the substantially disproportionate racial impact of a statute or
official practice standing alone and without regard to discriminatory
purpose, suffices to prove racial discrimination violating the Equal
Protection Clause absent some justification going substantially beyond
what would be necessary to validate most other legislative
classifications The cases impressively demonstrate that there is another
side to the issue; but, with all due respect, to the extent that those
cases rested on or expressed the view that proof of discriminatory
racial purpose is unnecessary in making out an equal protection
violation, we are in disagreement.

As an initial matter, we have difficulty understanding how a law
establishing a racially neutral qualification for employment is
nevertheless racially discriminatory and denies ``any person . equal
protection of the laws'' simply because a greater proportion of Negroes
fail to qualify than members of other racial or ethnic groups. Had
respondents, along with all others who had failed Test 21, whether white
or black, brought an action claiming that the test denied each of them
equal protection of the laws as compared with those who had passed with
high enough scores to qualify them as police recruits, it is most
unlikely that their challenge would have been sustained. Test 21, which
is administered generally to prospective Government employees,
concededly seeks to ascertain whether those who take it have acquired a
particular level of verbal skill; and it is untenable that the
Constitution prevents the Government from seeking modestly to upgrade
the communicative abilities of its employees rather than to be satisfied
with some lower level of competence, particularly where the job requires
special ability to communicate orally and in writing. Respondents, as
Negroes, could no more successfully claim that the test denied them
equal protection than could white applicants who also failed. The
conclusion would not be different in the face of proof that more Negroes
than whites had been disqualified by Test 21. That other Negroes also
failed to score well would, alone, not demonstrate that respondents
individually were being denied equal protection of the laws by the
application of an otherwise valid qualifying test being administered to
prospective police recruits.

Nor on the facts of the case before us would the disproportionate impact
of Test 21 warrant the conclusion that it is a purposeful device to
discriminate against Negroes and hence an infringement of the
constitutional rights of respondents as well as other black applicants.
As we have said, the test is neutral on its face and rationally may be
said to serve a purpose the Government is constitutionally empowered to
pursue. Even agreeing with the District Court that the differential
racial effect of Test 21 called for further inquiry, we think the
District Court correctly held that the affirmative efforts of the
Metropolitan Police Department to recruit black officers, the changing
racial composition of the recruit classes and of the force in general,
and the relationship of the test to the training program negated any
inference that the Department discriminated on the basis of race or that
``a police officer qualifies on the color of his skin rather than
ability.'' 348 F.Supp..

Under Title VII, Congress provided that when hiring and promotion
practices disqualifying substantially disprortionate numbers of blacks
are challenged, discriminatory purpose need not be proved, and that it
is an insufficient response to demonstrate some rational basis for the
challenged practices. It is necessary, in addition, that they be
``validated'' in terms of job performance in any one of several ways,
perhaps by ascertaining the minimum skill, ability, or potential
necessary for the position at issue and determining whether the
qualifying tests are appropriate for the selection of qualified
applicants for the job in question However this process proceeds, it
involves a more probing judicial review of, and less deference to, the
seemingly reasonable acts of administrators and executives than is
appropriate under the Constitution where special racial impact, without
discriminatory purpose, is claimed. We are not disposed to adopt this
more rigorous standard for the purposes of applying the Fifth and the
Fourteenth Amendments in cases such as this

A rule that a statute designed to serve neutral ends is nevertheless
invalid, absent compelling justification, if in practice it benefits or
burdens one race more than another would be far-reaching and would raise
serious questions about, and perhaps invalidate, a whole range of tax,
welfare, public service, regulatory, and licensing statutes that may be
more burdensome to the poor and to the average black than to the more
affluent white.

Given that rule, such consequences would perhaps be likely to follow.
However, in our view, extension of the rule beyond those areas where it
is already applicable by reason of statute, such as in the field of
public employment, should await legislative prescription.

As we have indicated, it was error to direct summary judgment for
respondents based on the Fifth Amendment.

We also hold that the Court of Appeals should have affirmed the judgment
of the District Court granting the motions for summary judgment filed by
petitioners and the federal parties. Respondents were entitled to relief
on neither constitutional nor statutory grounds.

The submission of the defendants in the District Court was that Test 21
complied with all applicable statutory as well as constitutional
requirements; and they appear not to have disputed that under the
statutes and regulations governing their conduct standards similar to
those obtaining under Title VII had to be satisfied The District Court
also assumed that Title VII standards were to control the case
identified the determinative issue as whether Test 21 was sufficiently
job related and proceeded to uphold use of the test because it was
``directly related to a determination of whether the applicant possesses
sufficient skills requisite to the demands of the curriculum a recruit
must master at the police academy.'' 348 F.Supp.. The Court of Appeals
reversed because the relationship between Test 21 and training school
success, if demonstrated at all, did not satisfy what it deemed to be
the crucial requirement of a direct relationship between performance on
Test 21 and performance on the policeman's job.

We agree with petitioners and the federal parties that this was error.
The advisability of the police recruit training course informing the
recruit about his upcoming job, acquainting him with its demands, and
attempting to impart a modicum of required skills seems conceded. It is
also apparent to us, as it was to the District Judge, that some minimum
verbal and communicative skill would be very useful, if not essential,
to satisfactory progress in the training regimen. Based on the evidence
before him, the District Judge concluded that Test 21 was directly
related to the requirements of the police training program and that a
positive relationship between the test and training-course performance
was sufficient to validate the former, wholly aside from its possible
relationship to actual performance as a police officer. This conclusion
of the District Judge that training-program validation may itself be
sufficient is supported by regulations of the Civil Service Commission,
by the opinion evidence placed before the District Judge, and by the
current views of the Civil Service Commissioners who were parties to the
case Nor is the conclusion closed by either Griggs or Albemarle Paper
Co.~v. Moody; and it seems to us the much more sensible construction of
the job-relatedness requirement.

The District Court's accompanying conclusion that Test 21 was in fact
directly related to the requirements of the police training program was
supported by a validation study, as well as by other evidence of record;
17 and we are not convinced that this conclusion was erroneous.

The federal parties, whose views have somewhat changed since the
decision of the Court of Appeals and who still insist that
training-program validation is sufficient, now urge a remand to the
District Court for the purpose of further inquiry into whether the
training-program test scores, which were found to correlate with Test 21
scores, are themselves an appropriate measure of the trainee's
mastership of the material taught in the course and whether the training
program itself is sufficiently related to actual performance of the
police officer's task. We think a remand is inappropriate. The District
Court's judgment was warranted by the record before it, and we perceive
no good reason to reopen it, particularly since we were informed at oral
argument that although Test 21 is still being administered, the training
program itself has undergone substantial modification in the course of
this litigation. If there are now deficiencies in the recruiting
practices under prevailing Title VII standards, those deficiencies are
to be directly addressed in accordance with appropriate procedures
mandated under that Title.

The judgment of the Court of Appeals accordingly is reversed.

So ordered.

\textbf{Mr.~Justice STEVENS, concurring.}

The requirement of purposeful discrimination is a common thread running
through the cases summarized in Part II. These cases include criminal
convictions which were set aside because blacks were excluded from the
grand jury, a reapportionment case in which political boundaries were
obviously influenced to some extent by racial considerations, a school
desegregation case, and a case involving the unequal administration of
an ordinance purporting to prohibit the operation of laundries in frame
buildings. Although it may be proper to use the same language to
describe the constitutional claim in each of these contexts, the burden
of proving a prima facie case may well involve differing evidentiary
considerations. The extent of deference that one pays to the trial
court's determination of the factual issue, and indeed, the extent to
which one characterizes the intent issue as a question of fact or a
question of law, will vary in different contexts.

Frequently the most probative evidence of intent will be objective
evidence of what actually happened rather than evidence describing the
subjective state of mind of the actor. For normally the actor is
presumed to have intended the natural consequences of his deeds. This is
particularly true in the case of governmental action which is frequently
the product of compromise, of collective decisionmaking, and of mixed
motivation. It is unrealistic, on the one hand, to require the victim of
alleged discrimination to uncover the actual subjective intent of the
decisionmaker or, conversely, to invalidate otherwise legitimate action
simply because an improper motive affected the deliberation of a
participant in the decisional process. A law conscripting clerics should
not be invalidated because an atheist voted for it.

My point in making this observation is to suggest that the line between
discriminatory purpose and discriminatory impact is not nearly as
bright, and perhaps not quite as critical, as the reader of the Court's
opinion might assume. I agree, of course, that a constitutional issue
does not arise every time some disproportionate impact is shown. On the
other hand, when the disproportion is as dramatic as in Gomillion v.
Lightfootor Yick Wo v. Hopkinsit really does not matter whether the
standard is phrased in terms of purpose or effect. Therefore, although I
accept the statement of the general rule in the Court's opinion, I am
not yet prepared to indicate how that standard should be applied in the
many cases which have formulated the governing standard in different
language.*

My agreement rests on a ground narrower than the Court describes. I do
not rely at all on the evidence of good-faith efforts to recruit black
police officers. In my judgment, neither those efforts nor the
subjective good faith of the District administration, would save Test 21
if it were otherwise invalid.

There are two reasons why I am convinced that the challenge to Test 21
is insufficient. First, the test serves the neutral and legitimate
purpose of requiring all applicants to meet a uniform minimum standard
of literacy. Reading ability is manifestly relevant to the police
function, there is no evidence that the required passing grade was set
at an arbitrarily high level, and there is sufficient disparity among
high schools and high school graduates to justify the use of a separate
uniform test. Second, the same test is used throughout the federal
service. The applicants for employment in the District of Columbia
Police Department represent such a small fraction of the total number of
persons who have taken the test that their experience is of minimal
probative value in assessing the neutrality of the test itself. That
evidence, without more, is not sufficient to overcome the presumption
that a test which is this widely used by the Federal Government is in
fact neutral in its effect as well as its ``purposes'' that term is used
in constitutional adjudication.

\hypertarget{milliken-v.-bradley}{%
\subsubsection{Milliken v. Bradley}\label{milliken-v.-bradley}}

418 U.S. 717 (1974)

\textbf{Mr.~Chief Justice BURGER delivered the opinion of the Court.}

We granted certiorari in these consolidated cases to determine whether a
federal court may impose a multidistrict, areawide remedy to a
single-district de jure segregation problem absent any finding that the
other included school districts have failed to operate unitary school
systems within their districts, absent any claim or finding that the
boundary lines of any affected school district were established with the
purpose of fostering racial segregation in public schools, absent any
finding that the included districts committed acts which effected
segregation within the other districts, and absent a meaningful
opportunity for the included neighboring school districts to present
evidence or be heard on the propriety of a multidistrict remedy or on
the question of constitutional violations by those neighboring
districts.

The action was commenced in August 1970 by the respondents, the Detroit
Branch of the National Association for the Advancement of Colored
People2 and individual parents and students, on behalf of a class later
defined by order of the United States District Court for the Eastern
District of Michigan, dated February 16, 1971, to included `all school
children in the City of Detroit, Michigan, and all Detroit resident
parents who have children of school age.' The named defendants in the
District Court included the Governor of Michigan, the Attorney General,
the State Board of Education, the State Superintendent of Public
Instruction, the Board of Education of the city of Detroit, its members,
the city's and its former superintendent of schools. The State of
Michigan as such is not a party to this litigation and references to the
State must be read as references to the public officials, state and
local, through whom the State is alleged to have acted. In their
complaint respondents attacked the constitutionality of a statute of the
State of Michigan known as Act 48 of the 1970 Legislature on the ground
that it put the State of Michigan in the position of unconstitutionally
interfering with the execution and operation of a voluntary plan of
partial high school desegregation, known as the April 7, 1970, Plan,
which had been adopted by the Detroit Board of Education to be effective
beginning with the fall 1970 semester. The complaint also alleged that
the Detroit Public School System was and is segregated on the basis of
race as a result of the official policies and actions of the defendants
and their predecessors in office, and called for the implementation of a
plan that would eliminate `the racial identity of every school in the
(Detroit) system and . maintain now and hereafter a unitary, nonracial
school system.'

Initially the matter was tried on respondents' motion for a preliminary
injunction to restrain in enforcement of Act 48 so as to permit the
April 7 Plan to be implemented. On that issue, the District Court ruled
that respondents were not entitled to a preliminary injunction since at
that stage there was no proof that Detroit had a dual segregated school
system. On appeal, the Court of Appeals found that the `implementation
of the April 7 plan was (unconstitutionally) thwarted by State action in
the form of the Act of the Legislature of Michigan,' CA6 1970), and that
such action could not be interposed to delay, obstruct, or nullify steps
lawfully taken for the purpose of protecting rights guaranteed by the
Fourteenth Amendment. The case was remanded to the District Court for an
expedited trial on the merits.

On remand, the respondents moved for immediate implementation of the
April 7 Plan in order to remedy the deprivation of the claimed
constitutional rights. In response, the School Board suggested two other
plans, along with the April 7 Plan, and urged that top priority be
assigned to the so-called `Magnet Plan' which was `designed to attract
children to a school because of its superior curriculum.' The District
Court approved the Board's Magnet Plan, and respondents again appealed
to the Court of Appeals, moving for summary reversal. The Court of
Appeals refused to pass on the merits of the Magnet Plan and ruled that
the District Court had not abused its discretion in refusing to adopt
the April 7 Plan without an evidentiary hearing. The case was again
remanded with instructions to proceed immediately to a trial on the
merits of respondents' substantive allegations concerning the Detroit
school system. CA6 1971).

The trial of the issue of segregation in the Detroit school system began
on April 6, 1971, and continued through July 22, 1971, consuming some 41
trial days. On September 27, 1971, the District Court issued its
findings and conclusions on the issue of segregation, finding that
`Governmental actions and inaction at all levels, federal, state and
local, have combined, with those of private organizations, such as
loaning institutions and real estate associations and brokerage firms,
to establish and to maintain the pattern of residential segregation
throughout the Detroit metropolitan area.' 338 F.Supp. 582, 587 (ED Mich
). While still addressing a Detroit-only violation, the District Court
reasoned:

`While it would be unfair to charge the present defendants with what
other governmental officers or agencies have done, it can be said that
the actions or the failure to act by the responsible school authorities,
both city and state, were linked to that of these other governmental
units. When we speak of governmental action we should not view the
different agencies as a collection of unrelated units. Perhaps the most
that can be said is that all of them, including the school authorities,
are, in part, responsible for the segregated condition which exists. And
we note that just as there is an interaction between residential
patterns and the racial composition of the schools, so there is a
corresponding effect on the residential pattern by the racial
composition of the schools.'

The District Court found that the Detroit Board of Education created and
maintained optional attendance zones3 within Detroit neighborhoods
undergoing racial transition and between high school attendance areas of
opposite predominant racial compositions. These zones, the court found,
had the `natural, probable, foreseeable and actual effect' of allowing
white pupils to escape identifiably Negro schools. Similarly, the
District Court found that Detroit school attendance zones had been drawn
along north-south boundary lines despite the Detroit Board's awareness
that drawing boundary lines in an east-west direction would result in
significantly greater desegregation. Again, the District Court
concluded, the natural and actual effect of these acts was the creation
and perpetuation of school segregation within Detroit.

The District Court found that in the operation of its school
transportation program, which was designed to relieve overcrowding, the
Detroit Board had admittedly bused Negro Detroit pupils to predominantly
Negro schools which were beyond or away from closer white schools with
available space This practice was found to have continued in recent
years despite the Detroit Board's avowed policy, adopted in 1967, of
utilizing transportation to increase desegregation:

`With one exception (necessitated by the burning of a white school),
defendant Board has never bused white children to predominantly black
schools. The Board has not bused white pupils to black schools despite
the enormous amount of space available in inner-city schools. There were
22,961 vacant seats in schools 90\% or more black.'

With respect to the Detroit Board of Education's practices in school
construction, the District Court found that Detroit school construction
generally tended to have a segregative effect with the great majority of
schools being built in either overwhelmingly all-Negro or all-white
neighborhoods so that the new schools opened as predominantly one-race
schools. Thus, of the 14 schools which opened for use in 1970opened over
90\% Negro and one opened less than 10\% Negro.

The District Court also found that the State of Michigan had committed
several constitutional violations with respect to the exercise of its
general responsibility for, and supervision of, public education The
State, for example, was found to have failed, until the 1971 Session of
the Michigan Legislature, to provide authorization or funds for the
transportation of pupils within Detroit regardless of their poverty or
distance from the school to which they were assigned; during this same
period the State provided many neighboring, mostly white, suburban
districts the full range of state-supported transportation.

The District Court found that the State, through Act 48, acted to
`impede, delay and minimize racial integration in Detroit schools.' The
first sentence of § 12 of Act 48 was designed to delay the April 7,
1970, desegregation plan originally adopted by the Detroit Board. The
remainder of § 12 sought to prescribe for each school in the eight
districts criteria of `free choice' and `neighborhood schools,' which,
the District Court found, `had as their purpose and effect the
maintenance of segregation.' 338 F.Supp..

The District Court also held that the acts of the Detroit Board of
Education, as a subordinate entity of the State, were attributable to
the State of Michigan, thus creating a vicarious liability on the part
of the State. Under Michigan law, Mich.Comp.Laws § 388 (1970), for
example, school building construction plans had to be approved by the
State Board of Education, and, prior to 1962, the State Board had
specific statutory authority to supervise school-site selection. The
proofs concerning the effect of Detroit's school construction program
were, therefore, found to be largely applicable to show state
responsibility for the segregative results.

Turning to the question of an appropriate remedy for these several
constitutional violations, the District Court deferred a pending motion8
by intervening parent de- fendants to join as additional parties
defendant the 85 outlying school districts in the three-county Detroit
metropolitan area on the ground that effective relief could not be
achieved without their presence. 9 The District Court concluded that
this motion to join was `premature,' since it `has to do with relief'
and no reasonably specific desegregation plan was before the court. 338
F.Supp.. Accordingly, the District Court proceeded to order the Detroit
Board of Education to submit desegregation plans limited to the
segregation problems found to be existing within the city of Detroit. At
the same time, however, the state defendants were directed to submit
desegregation plans encompassing the three-county metropolitan area 10
despite the fact that the 85 outlying school districts of these three
counties were not parties to the action and despite the fact that there
had been no claim that these outlying districts had committed
constitutional violations An effort to appeal these orders to the Court
of Appeals was dismissed on the ground that the orders were not
appealable. CA6), cert. deniedThe sequence of the ensuing actions and
orders of the District Court are significant factors and will therefore
be catalogued in some detail.

Following the District Court's abrupt announcement that it planned to
consider the implementation of a multidistrict, metropolitan area remedy
to the segregation problems identified within the city of Detroit, the
District Court was again requested to grant the outlying school
districts intervention as of right on the ground that the District
Court's new request for multidistrict plans `may, as a practical matter,
impair or impede (the intervenors') ability to protect' the welfare of
their students. The District Court took the motions to intervene under
advisement pending submission of the requested desegregation plans by
Detroit and the state officials. On March 7, 1972, the District Court
notified all parties and the petitioner school districts seeking
intervention, that March 14, 1972, was the deadline for submission of
recommendations for conditions of intervention and the date of the
commencement of hearings on Detroit-only desegregation plans. On the
second day of the scheduled hearings, March 15, 1972, the District Court
granted the motions of the intervenor school districts12 subject, inter
alia, to the following conditions:

'1. No intervenor will be permitted to assert any claim or defense
previously adjudicated by the court.

'2. No intervenor shall reopen any question or issue which has
previously been decided by the court.

`7. New intervenors are granted intervention for two principal purposes:
(a) To advise the court, by brief, of the legal propriety or impropriety
of considering a metropolitan plan; (b) To review any plan or plans for
the desegregation of the so-called larger Detroit Metropolitan area, and
submitting objections, modifications or alternatives to it or them, and
in accordance with the requirements of the United States Constitution
and the prior orders of this court.' 1 Joint Appendix 206 (hereinafter
App.).

Upon granting the motion to intervene, on March 15, 1972, the District
Court advised the petitioning intervenors that the court had previously
set March 22, 1972, as the date for the filing of briefs on the legal
propriety of a `metropolitan' plan of desegregation and, accordingly,
that the intervening school districts would have one week to muster
their legal arguments on the issue.

Thereafter, and following the completion of hearings on the Detroit-only
desegregation plans, the District Court issued the four rulings that
were the principal issues in the Court of Appeals.

\begin{enumerate}
\def\labelenumi{(\alph{enumi})}
\tightlist
\item
  On March 24, 1972, two days after the intervenors' briefs were due,
  the District Court issued its ruling on the question of whether it
  could `consider relief in the form of a metropolitan plan,
  encompassing not only the City of Detroit, but the larger Detroit
  metropolitan area.' It rejected the state defendants' arguments that
  no state action caused the segregation of the Detroit schools, and the
  intervening suburban districts' contention that interdistrict relief
  was inappropriate unless the suburban districts themselves had
  committed violations. The court concluded:
\end{enumerate}

`(I)t is proper for the court to consider metropolitan plans directed
toward the desegregation of the Detroit public schools as an alternative
to the present intra-city desegregation plans before it and, in the
event that the court finds such intra-city plans inadequate to
desegregate such schools, the court is of the opinion that it is
required to consider a metropolitan remedy for desegregation.' Pet.App.
51a.

\begin{enumerate}
\def\labelenumi{(\alph{enumi})}
\setcounter{enumi}{1}
\item
  On March 28, 1972, the District Court issued its findings and
  conclusions on the three Detroit-only plans submitted by the city
  Board and the respondents. It found that the best of the three plans
  `would make the Detroit school system more identifiably Black .
  thereby increasing the flight of Whites from the city and the system.'
  a. From this the court concluded that the plan `would not accomplish
  desegregation . within the corporate geographical limits of the city.'
  a. Accordingly, the District Court held that it `must look beyond the
  limits of the Detroit school district for a solution to the problem,'
  and that `(s)chool district lines are simply matters of political
  convenience and may not be used to deny constitutional rights.' a.
\item
  During the period from March 28 to April 14, 1972, the District Court
  conducted hearings on a metropolitan plan. Counsel for the petitioning
  intervenors was allowed to participate in these hearings, but he was
  ordered to confine his argument to `the size and expanse of the
  metropolitan plan' without addressing the intervenors' opposition to
  such a remedy or the claim that a finding of a constitutional
  violation by the intervenor districts was an essential predicate to
  any remedy involving them. Thereafter, on June 14, 1972, the District
  Court issued its ruling on the `desegregation area' and related
  findings and conclusions. The court acknowledged at the outset that it
  had `taken no proofs with respect to the establishment of the
  boundaries of the 86 public school districts in the counties (in the
  Detroit area), nor on the issue of whether, with the exclusion of the
  city of Detroit school districts, such school districts have committed
  acts of de jure segregation.' Nevertheless, the court designated 53 of
  the 85 suburban school districts plus Detroit as the `desegregation
  area' and appointed a panel to prepare and submit `an effective
  desegregation plan' for the Detroit schools that would encompass the
  entire desegregation area The plan was to be based on 15 clusters,
  each containing part of the Detroit system and two or more suburban
  districts, and was to `achieve the greatest degree of actual
  desegregation to the end that, upon implementation, no school, grade
  or classroom (would be) substantially disproportionate to the overall
  pupil racial composition.' 345 F.Supp. 914, 918 (ED Mich ).
\item
  On July 11, 1972, and in accordance with a recommendation by the
  court-appointed desegregation panel, the District Court ordered the
  Detroit Board of Education to purchase or lease `at least' 295 school
  buses for the purpose of providing transportation under an interim
  plan to be developed for the 1972 1973 school year. The costs of this
  acquisition were to be borne by the state defendants. Pet.App.
  106a---107a.
\end{enumerate}

On June 12, 1973, a divided Court of Appeals, sitting en banc, affirmed
in part, vacated in part, and remanded for further proceedings. CA6) The
Court of Appeals held, first, that the record supported the District
Court's findings and conclusions on the constitutional violations
committed by the Detroit Board, ---238, and by the state defendants,
---241 It stated that the acts of racial discrimina- tion shown in the
record are `causally related to the substantial amount of segregation
found in the Detroit school system,' and that `the District Court was
therefore authorized and required to take effective measures to
desegregate the Detroit Public School System.'

The Court of Appeals also agreed with the District Court that `any less
comprehensive a solution than a metropolitan area plan would result in
an all black school system immediately surrounded by practically all
white suburban school systems, with an overwhelmingly white majority
population in the total metropolitan area.' The court went on to state
that it could `(not) see how such segregation can be any less harmful to
the minority students than if the same result were accomplished within
one school district.'

Accordingly, the Court of Appeals concluded that `the only feasible
desegregation plan involves the crossing of the boundary lines between
the Detroit School District and adjacent or nearby school districts for
the limited purpose of providing an effective desegregation plan.' It
reasoned that such a plan would be appropriate because of the State's
violations, and could be implemented because of the State's authority to
control local school districts. Without further elaboration, and without
any discussion of the claims that no constitutional violation by the
outlying districts had been shown and that no evidence on that point had
been allowed, the Court of Appeals held:

`(T)he State has committed de jure acts of segregation and . the State
controls the instrumentalities whose action is necessary to remedy the
harmful effects of the State acts.'

An interdistrict remedy was thus held to be `within the equity powers of
the District Court.'

The Court of Appeals expressed no views on the propriety of the District
Court's composition of the metropolitan `desegregation area.' It held
that all suburban school districts that might be affected by any
metropolitanwide remedy should, under Fed.Rule Civ.Proc. 19, be made
parties to the case on remand and be given an opportunity to be heard
with respect to the scope and implementation of such a remedy. 484 F
d---252. Under the terms of the remand, however, the District Court was
not `required' to receive further evidence on the issue of segregation
in the Detroit schools or on the propriety of a Detroit-only remedy, or
on the question of whether the affected districts had committed any
violation of the constitutional rights of Detroit pupils or others.
Finally, the Court of Appeals vacated the District Court's order
directing the acquisition of school buses, subject to the right of the
District Court to consider reimposing the order `at the appropriate
time.'

Ever since Brown v. Board of Educationjudicial consideration of school
desegregation cases has begun with the standard:

`(I)n the field of public education the doctrine of 'separate but equal'
has no place. Separate educational facilities are inherently unequal.'

This has been reaffirmed time and again as the meaning of the
Constitution and the controlling rule of law.

The target of the Brown holding was clear and forthright: the
elimination of state-mandated or deliberately maintained dual school
systems with certain schools for Negro pupils and others for white
pupils. This duality and racial segregation were held to violate the
Constitution in the cases subsequent to 1954, including particularly
Green v. County School Board of New Kent County; Raney v. Board of
Education; Monroe v. Board of Comm'rs; Swann v. Charlotte-Mecklenburg
Board of Education; Wright v. Council of the City of Emporia; United
States v. Scotland Neck City Board of Education.

The Swann case, of course, dealt `with the problem of defining in more
precise terms than heretofore the scope of the duty of school
authorities and district courts in implementing Brown I and the mandate
to eliminate dual systems and establish unitary systems at once.'

In Brown v. Board of Education (Brown II), the Court's first encounter
with the problem of remedies in school desegregation cases, the Court
noted:

`In fashioning and effectuating the decrees, the courts will be guided
by equitable principles. Tra- ditionally, equity has been characterized
by a practical flexibility in shaping its remedies and by a facility for
adjusting and reconciling public and private needs.'

In further refining the remedial process, Swann held, the task is to
correct, by a balancing of the individual and collective interests, `the
condition that offends the Constitution.' A federal remedial power may
be exercised `only on the basis of a constitutional violation' and,
`(a)s with any equity case, the nature of the violation determines the
scope of the remedy.'

Proceeding from these basic principles, we first note that in the
District Court the complainants sought a remedy aimed at the condition
alleged to offend the Constitution---the segregation within the Detroit
City School District. 18 The court acted on this theory of the case and
in its initial ruling on the `Desegregation Area' stated:

`The task before this court, therefore, is now, and . has always been,
now to desegregate the Detroit public schools.' 345 F.Supp..

Thereafter, however, the District Court abruptly rejected the proposed
Detroit-only plans on the ground that `while (they) would provide a
racial mix more in keeping with the Black-White proportions of the
student population (they) would accentuate the racial identifiability of
the (Detroit) district as a Black school system, and would not
accomplish desegregation.' Pet.App., 56a. `(T)he racial composition of
the student body is such,' said the court, `that the plan's
implementation would clearly make the entire Detroit public school
system racially identifiable' (Id.a), `leav(ing) many of its schools 75
to 90 per cent Black.' a. Consequently, the court reasoned, it was
imperative to `look beyond the limits of the Detroit school district for
a solution to the problem of segregation in the Detroit public schools .
.' since `(s)chool district lines are simply matters of political
convenience and may not be used to deny constitutional rights.' a.
Accordingly, the District Court proceeded to redefine the relevant area
to include areas of predominantly white pupil population in order to
ensure that `upon implementation, no school, grade or classroom (would
be) substantially disproportionate to the overall pupil racial
composition' of the entire metropolitan area.

While specifically acknowledging that the District Court's findings of a
condition of segregation were limited to Detroit, the Court of Appeals
approved the use of a metropolitan remedy largely on the grounds that it
is

`impossible to declare 'clearly erroneous' the District Judge's
conclusion that any Detroit only segregation plan will lead directly to
a single segregated Detroit school district overwhelmingly black in all
of its schools, surrounded by a ring of suburbs and suburban school
districts overwhelmingly white in composition in a State in which the
racial composition is 87 per cent white and 13 per cent black.' 484 F d.

Viewing the record as a whole, it seems clear that the District Court
and the Court of Appeals shifted the pri- mary focus from a Detroit
remedy to the metropolitan area only because of their conclusion that
total desegregation of Detroit would not produce the racial balance
which they perceived as desirable. Both courts proceeded on an
assumption that the Detroit schools could not be truly desegregated---in
their view of what constituted desegregation---unless the racial
composition of the student body of each school substantially reflected
the racial composition of the population of the metropolitan area as a
whole. The metropolitan area was then defined as Detroit plus 53 of the
outlying school districts. That this was the approach the District Court
expressly and frankly employed is shown by the order which expressed the
court's view of the constitutional standard:

`Within the limitations of reasonable travel time and distance factors,
pupil reassignments shall be effected within the clusters described in
Exhibit P.M. 12 so as to achieve the greatest degree of actual
desegregation to the end that, upon implementation, no school, grade or
classroom (will be) substantially disproportionate to the overall pupil
racial composition.' 345 F.Supp., st 918 (emphasis added).

In Swann, which arose in the context of a single independent school
district, the Court held:

`If we were to read the holding of the District Court to require, as a
matter of substantive constitutional right, any particular degree of
racial balance or mixing, that approach would be disapproved and we
would be obliged to reverse.'

The clear import of this language from Swann is that desegregation, in
the sense of dismantling a dual school system, does not require any
particular racial balance in each 'school, grade or classroom.'19 See
Spencer v. Kugler

Here the District Court's approach to what constituted `actual
desegregation' raises the fundamental question, not presented in Swann,
as to the circumstances in which a federal court may order desegregation
relief that embraces more than a single school district. The court's
analytical starting point was its conclusion that school district lines
are no more than arbitrary lines on a map drawn `for political
convenience.' Boundary lines may be bridged where there has been a
constitutional violation calling for interdistrict relief, but the
nation that school district lines may be casually ignored or treated as
a mere administrative convenience is contrary to the history of public
education in our country. No single tradition in public education is
more deeply rooted than local control over the operation of schools;
local autonomy has long been thought essential both to the maintenance
of community concern and support for public schools and to quality of
the educational process. See Wright v. Council of the City of Emporia,
Thus, in San Antonio Independent School District v. Rodriguez, we
observed that local control over the educational process affords
citizens an opportunity to participate in decision-making, permits the
structuring of school programs to fit local needs, and encourages
`experimentation, innovation, and a healthy competition for educational
excellence.'

The Michigan educational structure involved in this case, in common with
most States, provides for a large measure of local control,20 and a
review of the scope and character of these local powers indicates the
extent to which the interdistrict remedy approved by the two courts
could disrupt and alter the structure of public edu- cation in Michigan.
The metropolitan remedy would require, in effect, consolidation of 54
independent school districts historically administered as separate units
into a vast new super school district. See n.~10. Entirely apart from
the logistical and other serious problems attending large-scale
transportation of students, the consolidation would give rise to an
array of other problems in financing and operating this new school
system. Some of the more obvious questions would be: What would be the
status and authority of the present popularly elected school boards?
Would the children of Detroit be within the jurisdiction and operating
control of a school board elected by the parents and residents of other
districts? What board or boards would levy taxes for school operations
in these 54 districts constituting the consolidated metropolitan area?
What provisions could be made for assuring substantial equality in tax
levies among the 54 districts, if this were deemed requisite? What
provisions would be made for financing? Would the validity of long-term
bonds be jeopardized unless approved by all of the component districts
as well as the State? What body would determine that portion of the
curricula now left to the discretion of local school boards? Who would
establish attendance zones, purchase school equipment, locate and
construct new schools, and indeed attend to all the myriad day-to-day
decisions that are necessary to school operations affecting potentially
more than three-quarters of a million pupils? See n.~10.

It may be suggested that all of these vital operational problems are yet
to be resolved by the District Court, and that this is the purpose of
the Court of Appeals' proposed remand. But it is obvious from the scope
of the interdistrict remedy itself that absent a complete restructuring
of the laws of Michigan relating to school districts the District Court
will become first, a de facto Page

`legislative authority' to resolve these complex questions, and then the
`school superintendent' for the entire area. This is a task which few,
if any, judges are qualified to perform and one which would deprive the
people of control of schools through their elected representatives.

Of course, no state law is above the Constitution. School district lines
and the present laws with respect to local control, are not sacrosanct
and if they conflict with the Fourteenth Amendment federal courts have a
duty to prescribe appropriate remedies. See, e.g., Wright v. Council of
the City of Emporia; United States v. Scotland Neck City Board of
Education (state or local officials prevented from carving out a new
school district from an existing district that was in process of
dismantling a dual school system); cf.~Haney v. County Board of
Education of Sevier County, CA8 1970) (State contributed to separation
of races by drawing of school district lines); United States v. Texas,
321 F.Supp. 1043 (ED Tex ), aff'd, CA5 1971), cert. denied sub nom.
Edgar v. United States (one or more school districts created and
maintained for one race). But our prior holdings have been confined to
violations and remedies within a single school district. We therefore
turn to address, for the first time, the validity of a remedy mandating
cross-district or interdistrict consolidation to remedy a condition of
segregation found to exist in only one district.

The controlling principle consistently expounded in our holdings is that
the scope of the remedy is determined by the nature and extent of the
constitutional violation. Swann, Before the boundaries of separate and
autonomous school districts may be set aside by consolidating the
separate units for remedial purposes or by imposing a cross-district
remedy, it must first be shown that there has been a constitutional
violation within one district that produces a significant segregative
effect in another district. Specifically, it must be shown that racially
discriminatory acts of the state or local school districts, or of a
single school district have been a substantial cause of interdistrict
segregation. Thus an interdistrict remedy might be in order where the
racially discriminatory acts of one or more school districts caused
racial segregation in an adjacent district, or where district lines have
been deliberately drawn on the basis of race. In such circumstances an
interdistrict remedy would be appropriate to eliminate the interdistrict
segregation directly caused by the constitutional violation. Conversely,
without an interdistrict violation and interdistrict effect, there is no
constitutional wrong calling for an interdistrict remedy.

The record before us, voluminous as it is, contains evidence of de jure
segregated conditions only in the Detroit schools; indeed, that was the
theory on which the litigation was initially based and on which the
District Court took evidence. See---726. With no showing of significant
violation by the 53 outlying school districts and no evidence of any
interdistrict violation or effect, the court went beyond the original
theory of the case as framed by the pleadings and mandated a
metropolitan area remedy. To approve the remedy ordered by the court
would impose on the outlying districts, not shown to have committed any
constitutional violation, a wholly impermissible remedy based on a
standard not hinted at in Brown I and or any holding of this Court.

In dissent, Mr.~Justice WHITE and Mr.~Justice MARSHALL undertake to
demonstrate that agencies having statewide authority participated in
maintaining the dual school system found to exist in Detroit. They are
apparently of the view that once such participation is shown, the
District Court should have a relatively free hand to reconstruct school
districts outside of Detroit in fashioning relief. Our assumption,
arguendo, see infra, p.~748, that state agencies did participate in the
maintenance of the Detroit system, should make it clear that it is not
on this point that we part company. 21 The difference between us arises
instead from established doctrine laid down by our cases. Brown; Green;
Swann; Scotland Neck; and Emporiaeach addressed the issue of
constitutional wrong in terms of an established geographic and
administrative school system populated by both Negro and white children.
In such a context, terms such as `unitary' and `dual' systems, and
`racially identifiable schools,' have meaning, and the necessary federal
authority to remedy the constitutional wrong is firmly established. But
the remedy is necessarily designed, as all remedies are, to restore the
victims of discriminatory conduct to the position they would have
occupied in the absence of such conduct. Disparate treatment of white
and Negro students occurred within the Detroit school system, and not
elsewhere, and on this record the remedy must be limited to that system.

The constitutional right of the Negro respondents residing in Detroit is
to attend a unitary school system in that district. Unless petitioners
drew the district lines in a discriminatory fashion. or arranged for
white students residing in the Detroit district to attend schools in
Oakland and Macomb Counties, they were under no constitutional duty to
make provisions for Negro students to do so. The view of the dissenters,
that the existence of a dual system in Detroit can be made the basis for
a decree requiring cross-district transportation of pupils, cannot be
supported on the grounds that it represents merely the devising of a
suitably flexible remedy for the violation of rights already established
by our prior decisions. It can be supported only by drastic expansion of
the constitutional right itself, an expansion without any support in
either constitutional principle or precedent.

We recognize that the six-volume record presently under consideration
contains language and some specific incidental findings thought by the
District Court to afford a basis for interdistrict relief. However,
these comparatively isolated findings and brief comments concern only
one possible interdistrict violation and are found in the context of a
proceeding that, as the District Court conceded, included no proof of
segregation practiced by any of the 85 suburban school districts
surrounding Detroit. The Court of Appeals, for example, relied on five
factors which, it held, amounted to unconstitutional state action with
respect to the violations found in the Detroit system:

\begin{enumerate}
\def\labelenumi{(\arabic{enumi})}
\item
  It held the State derivatively responsible for the Detroit Board's
  violations on the theory that actions of Detroit as a political
  subdivision of the State were attributable to the State. Accepting,
  arguendo, the correctness of this finding of state responsibility for
  the segregated conditions within the city of Detroit, it does not
  follow that an interdistrict remedy is constitutionally justified or
  required. With a single exception, discussed later, there has been no
  showing that either the State or any of the 85 outlying districts
  engaged in activity that had a cross-district effect. The boundaries
  of the Detroit School District, which are coterminous with the
  boundaries of the city of Detroit, were established over a century ago
  by neutral legislation when the city was incorporated; there is no
  evidence in the record, nor is there any suggestion by the
  respondents, that either the original boundaries of the Detroit School
  District, or any other school district in Michigan, were established
  for the purpose of creating, maintaining, or perpetuating segregation
  of races. There is no claim and there is no evidence hinting that
  petitioner outlying schools districts and their processors, or the
  30-odd other school districts in the tricounty area---but outside the
  District Court's `desegregation area'---have ever maintained or
  operated anything but unitary school systems. Unitary school systems
  have been required for more than a century by the Michigan
  Constitution as implemented by state law White the schools of only one
  district have been affected, there is no constitutional power in the
  courts to decree relief balancing the racial composition of that
  district's schools with those of the surrounding districts.
\item
  There was evidence introduced at trial that, during the late 1950's,
  Carver School District, a predominantly Negro suburban district,
  contracted to have Negro high school students sent to a predominantly
  Negro school in Detroit. At the time, Carver was an independent school
  district that had no high school because, according to the trial
  evidence, `Carver District . did not have a place for adequate high
  school facilities.' 484 F d.. Accordingly, arrangements were made with
  Northern High School in the abutting Detroit School District so that
  the Carver high school students could obtain a secondary school
  education. In 1960 the Oak Park School District, a predominantly white
  suburban district, annexed the predominantly Negro Carver School
  District, through the initiative of local officials.
\end{enumerate}

There is, of course, no claim that the 1960 annexation had a segregative
purpose or result or that Oak Park now maintains a dual system.

According to the Court of Appeals, the arrangement during the late
1950's which allowed Carver students to be educated within the Detroit
District was dependent upon the `tacit or express' approval of the State
Board of Education and was the result of the refusal of the white
suburban districts to accept the Carver students. Although there is
nothing in the record supporting the Court of Appeals' supposition that
suburban white schools refused to accept the Carver students, it appears
that this situation, whether with or without the State's consent, may
have had a segregative effect on the school populations of the two
districts involved. However, since `the nature of the violation
determines the scope of the remedy,' Swann, this isolated instance
effecting two of the school districts would not justify the broad
metropolitanwide remedy contemplated by the District Court and approved
by the Court of Appeals, particularly since it embraced potentially 52
districts having no responsibility for the arrangement and involved
503,000 pupils in addition to Detroit's 276,000 students.

\begin{enumerate}
\def\labelenumi{(\arabic{enumi})}
\setcounter{enumi}{2}
\item
  The Court of Appeals cited the enactment of state legislation (Act 48)
  which had the effect of rescinding Detroit's voluntary desegregation
  plan (the April 7 Plan). That plan, however, affected only 12 of 21
  Detroit high schools and had no causal connection with the
  distribution of pupils by race between Detroit and the other school
  districts within the tricounty area.
\item
  The court relied on the State's authority to supervise schoolsite
  selection and to approve building construction as a basis for holding
  the State responsible for the segregative results of the school
  construction program in Detroit. Specifically, the Court of Appeals
  asserted that during the period between 1949 and 1962 the State Board
  of Education exercised general authority as overseer of site
  acquisitions by local boards for new school construction, and
  suggested that this state-approved school construction `fostered
  segregation throughout the Detroit Metropolitan area.' 484 F d.~This
  brief comment, however, is not supported by the evidence taken at
  trial since that evidence was specifically limited to proof that
  schoolsite acquisition and school construction within the city of
  Detroit produced de jure segregation within the city itself. ---238.
  Thus, there was no evidence suggesting that the State's activities
  with respect to either school construction or site acquisition within
  Detroit affected the racial composition of the school population
  outside Detroit or, conversely, that the State's school construction
  and site acquisition activities within the outlying districts affected
  the racial composition of the schools within Detroit.
\item
  The Court of Appeals also relied upon the District Court's finding:
\end{enumerate}

`This and other financial limitations, such as those on bonding and the
working of the state aid formula whereby suburban districts were able to
make far larger per pupil expenditures despite less tax effort, have
created and perpetuated systematic educational inequalities.'

However, neither the Court of Appeals nor the District Court offered any
indication in the record or in their opinions as to how, if at all, the
availability of state-financed aid for some Michigan students outside
Detroit, but not for those within Detroit, might have affected the
racial character of any of the State's school districts. Furthermore, as
the respondents recognize, the application of our recent ruling in San
Antonio School District v. Rodriguezto this state education financing
system is questionable, and this issue was not addressed by either the
Court of Appeals or the District Court. This, again, underscores the
crucial fact that the theory upon which the the case proceeded related
solely to the establishment of Detroit city violations as a basis for
desegregating Detroit schools and that, at the time of trial, neither
the parties nor the trial judge was concerned with a foundation for
interdistrict relief.

Petitioners have urged that they were denied due process by the manner
in which the District Court limited their participation after
intervention was allowed, thus precluding adequate opportunity to
present evidence that they had committed no acts having a segregative
effect in Detroit. In light of our holding that, absent an interdistrict
violation, there is no basis for an interdistrict remedy, we need not
reach these claims. It is clear, however, that the District Court, with
the approval of the Court of Appeals, has provided an interdistrict
remedy in the face of a record which shows no constitutional violations
that would call for equitable relief except within the city of Detroit.
In these circumstances there was no occasion for the parties to address,
or for the District Court to consider whether there were racially
discriminatory acts for which any of the 53 outlying districts were
responsible and which had direct and significant segregative effect on
schools of more than one district.

We conclude that the relief ordered by the District Court and affirmed
by the Court of Appeals was based upon an erroneous standard and was
unsupported by record evidence that acts of the outlying districts
effected the discrimination found to exist in the schools of De- troit.
Accordingly, the judgment of the Court of Appeals is reversed and the
case is remanded for further proceedings consistent with this opinion
leading to prompt formulation of a decree directed to eliminating the
segregation found to exist in Detroit city schools, a remedy which has
been delayed since 1970.

Reversed and remanded.

\emph{From the footnotes:}

\begin{enumerate}
\def\labelenumi{\arabic{enumi}.}
\setcounter{enumi}{3}
\item
  The Court of Appeals found record evidence that in at least one
  instance during the period 1957---1958, Detroit served a suburban
  school district by contracting with it to educate its Negro high
  school students by transporting them away from nearby suburban white
  high schools, and past Detroit high schools which were predominantly
  white, to all-Negro or predominantly Negro Detroit schools. 484 F d.
\item
  School districts in the State of Michigan are instrumentalities of the
  State and subordinate to its State Board of Education and legislature.
  The Constitution of the State of Michigan, Art. 8, § 2, provides in
  relevant part: `The legislature shall maintain and support a system of
  free public elementary and secondary schools as defined by law.'
  Similarly, the Michigan Supreme Court has stated: `The school district
  is a State agency. Moreover, it is of legislative creation. . .'
  Attorney General ex rel. Kies v. Lowrey, 131 Mich. 639, 644, 92 N.W.
  289, 290 (1902): ``Education in Michigan belongs to the State. It is
  no part of the local self-government inherent in the township or
  municipality, except so far as the Legislature may choose to make it
  such. The Constitution has turned the whole subject over to the
  Legislature. . .'' Attorney General ex rel. Zacharias v. Detroit Board
  of Education, 154 Mich. 584, 590, 118 N.W. 606, 609 (1908).
\item
  The District Court briefly alluded to the possibility that the State,
  along with private persons, had caused, in part, the housing patterns
  of the Detroit metropolitan area which, in turn, produced the
  predominantly white and predominantly Negro neighborhoods that
  characterize Detroit: `It is no answer to say that restricted
  practices grew gradually (as the black population in the area
  increased between 1920 and 1970), or that since 1948 racial
  restrictions on the ownership of real property have been removed. The
  policies pursued by both government and private persons and agencies
  have a continuing and present effect upon the complexion of the
  community as we know, the choice of a residence is a relatively
  infrequent affair. For many years FHA and VA openly advised and
  advocated the maintenance of 'harmonious' neighborhoods, i.e.,
  racially and economically harmonious. The conditions created
  continue.' 338 F.Supp. 582, 587 (ED Mich ).
\end{enumerate}

Thus, the District Court concluded:

`The affirmative obligation of the defendant Board has been and is to
adopt and implement pupil assignment practices and policies that
compensate for and avoid incorporation into the school system the
effects of residential racial segregation.' The Court of Appeals,
however, expressly noted that: `In affirming the District Judge's
findings of constitutional violations by the Detroit Board of Education
and by the State defendants resulting in segregated schools in Detroit,
we have not relied at all upon testimony pertaining to segregated
housing except as school construction programs helped cause or maintain
such segregation.' 484 F d..

Accordingly, in its present posture, the case does not present any
question concerning possible state housing violations. 8. On March 22,
1971, a group of Detroit residents, who were parents of children
enrolled in the Detroit public schools, were permitted to intervene as
parties defendant. On June 24, 1971, the District Judge alluded to the
`possibility' of a metropolitan school system stating: `(A)s I have said
to several witnesses in this case: 'How do you desegrate a black city,
or a black school system." Petitioners' Appendix 243a (hereinafter
Pat.App.). Subsequently, on July 16, 1971, various parents filed a
motion to require joinder of all of the 85 outlying independent school
districts within the tri-county area. 9. The respondents, as plaintiffs
below, opposed the motion to join the additional school districts,
arguing that the presence of the state defendants was sufficient and all
that was required, even if, in shaping a remedy, the affairs of these
other districts was to be affected. 338 F.Supp.. 10. At the time of the
1970 census, the population of Michigan was 8,875,083, almost half of
which, 4,199,931, resided in the tri-county area of Wayne, Oakland, and
Macomb. Oakland and Macomb Counties abut Wayne County to the north, and
Oakland County abuts Macomb County to the west. These counties cover
1,952 square miles, Michigan Statistical Abstract (9th ed.~1972), and
the area is approximately the size of the State of Delaware (2,057
square miles), more than half again the size of the State of Rhode
Island (1,214 square miles) and almost 30 times the size of the District
of Columbia (67 square miles). Statistical Abstract of the United States
(93d ed.~1972). The populations of Wayne, Oakland, and Macomb Counties
were 2,666,751; 907,871; and 625,309, respectively, in 1970. Detroit,
the State's largest city, is located in Wayne County.

In the 1970---1971 school year, there were 2,157,449 children enrolled
in school districts in Michigan. There are 86 independent, legally
distinct school districts within the tri-county area, having a total
enrollment of approximately 1,000,000 children. In 1970, the Detroit
Board of Education operated 319 schools with approximately 276,000
students. 11. In its formal opinion, subsequently announced, the
District Court candidly recognized:

\begin{enumerate}
\def\labelenumi{\arabic{enumi}.}
\setcounter{enumi}{21}
\tightlist
\item
  The suggestion in the dissent of Mr.~Justice MARSHALL that schools
  which have a majority of Negro students are not `desegregated,'
  whatever the racial makeup of the school district's population and
  however neutrally the district lines have been drawn and administered,
  finds no support in our prior cases. In Green v. County School Board
  of New Kent Countyfor example, this Court approved a desegregation
  plan which would have resulted in each of the schools within the
  district having a racial composition of 57\% Negro and 43\% White. In
  Wright v. Council of the City of Emporiathe optimal desegregation plan
  would have resulted in the schools' being 66\% Negro and 34\% white,
  substantially the same percentages as could be obtained under one of
  the plans involved in this case. And in United States v. Scotland Neck
  City Board of Educationn. 5, a desegregation plan was implicitly
  approved for a school district which had a racial composition of 77\%
  Negro and 22\% white. In none of these cases was it even intimated
  that `actual desegregation' could not be accomplished as long as the
  number of Negro students was greater than the number of white
  students. The dissents also seem to attach importance to the
  metropolitan character of Detroit and neighboring school districts.
  But the constitutional principles applicable in school desegregation
  cases cannot vary in accordance with the size or population dispersal
  of the particular city, county, or school district as compared with
  neighboring areas.
\end{enumerate}

\textbf{Mr.~Justice STEWART, concurring.}

In joining the opinion of the Court, I think it appropriate, in view of
some of the extravagant language of the dissenting opinions, to state
briefly my understanding of what it is that the Court decides today.

The respondents commenced this suit in 1970, claiming only that a
constitutionally impermissible allocation of educational facilities
along racial lines had occurred in public schools within a single school
district whose lines were coterminous with those of the city of Detroit.
In the course of the subsequent proceedings, the District Court found
that public school officials had contributed to racial segregation
within that district by means of improper use of zoning and attendance
patterns, optional-attendance areas, and building and site selection.
This finding of a violation of the Equal Protection Clause was upheld by
the Court of Appeals, and is accepted by this Court today. See n.~18. In
the present posture of the case, therefore, the Court does not deal with
questions of substantive constitutional law. The basic issue now before
the Court concerns, rather, the appropriate exercise of federal equity
jurisdiction.

No evidence was adduced and no findings were made in the District Court
concerning the activities of school officials in districts outside the
city of Detroit, and no school officials from the outside districts even
participated in the suit until after the District Court had made the
initial determination that is the focus of today's decision. In spite of
the limited scope of the inquiry and the findings, the District Court
concluded that the only effective remedy for the constitutional
violations found to have existed within the city of Detroit was a
desegregation plan calling for busing pupils to and from school
districts outside the city. The District Court found that any
desegregation plan operating wholly ``within the corporate geographical
limits of the city'' would be deficient since it ``would clearly make
the entire Detroit public school system racially identifiable as
Black.'' 243. The Court of Appeals, in affirming the decision that an
interdistrict remedy was necessary, noted that a plan limited to the
city of Detroit `would result in an all black school system immediately
surrounded by practically all white suburban school systems, with an
overwhelmingly white majority population in the total metropolitan
area.'

The courts were in error for the simple reason that the remedy they
thought necessary was not commensurate with the constitutional violation
found. Within a single school district whose officials have been shown
to have engaged in unconstitutional racial segregation, a remedial
decree that affects every individual school may be dictated by `common
sense,' see Keyes v. School District No.~1, Denver, Colorado, and indeed
may provide the only effective means to eliminate segregation `root and
branch,' Green v. County School Board of New Kent County, and to
`effectuate a transition to a racially nondiscriminatory school system.'
Brown v. Board of Education, See KeyesS.Ct.---2696. But in this case the
Court of Appeals approved the concept of a remedial decree that would go
beyond the boundaries of the district where the constitutional violation
was found, and include schools and schoolchildren in many other school
districts that have presumptively been administered in complete accord
with the Constitution.

The opinion of the Court convincingly demonstrates, ---743, that
traditions of local control of schools, together with the difficulty of
a judicially supervised restructuring of local administration of
schools, render improper and inequitable such an interdistrict response
to a constitutional violation found to have occurred only within a
single school district.

This is not to say, however, that an interdistrict remedy of the sort
approved by the Court of Appeals would not be proper, or even necessary,
in other factual situations. Were it to be shown, for example, that
state officials had contributed to the separation of the races by
drawing or redrawing school district lines, see Haney v. County Board of
Education of Sevier County, ; cf.~Wright v. Council of the City of
Emporia; United States v. Scotland Neck City Board of Education; by
transfer of school units between districts, United States v. Texas, 321
F.Supp. 1043, aff'd, ; Turner v. Warren County Board of Education, 313
F.Supp. 380; or by purposeful racially discriminatory use of state
housing or zoning laws, then a decree calling for transfer of pupils
across district lines or for restructuring of district lines might well
be appropriate.

In this case, however, no such interdistrict violation was shown.
Indeed, no evidence at all concerning the administration of schools
outside the city of Detroit was presented other than the fact that these
schools contained a higher proportion of white pupils than did the
schools within the city. Since the mere fact of different racial
compositions in contiguous districts does not itself imply or constitute
a violation of the Equal Protection Clause in the absence of a showing
that such disparity was imposed, fostered, or encouraged by the State or
its political subdivisions, it follows that no interdistrict violation
was shown in this case The formulation of an inter-distrit remedy was
thus simply not responsive to the factual record before the District
Court and was an abuse of that court's equitable powers.

In reversing the decision of the Court of Appeals this Court is in no
way turning its back on the proscription of state-imposed segregation
first voiced in Brown v. Board of Educationor on the delineation of
remedial powers and duties most recently expressed in Swann v.
Charlotte-Mecklenburg Board of EducationIn Swann the Court addressed
itself to the range of equitable remedies available to the courts to
effectuate the desegregation mandated by Brown and its progeny, noting
that the task in choosing appropriate relief is `to correct . the
condition that offends the Constitution,' and that `the nature of the
violation determines the scope of the remedy . .' The disposition of
this case thus falls squarely under these principles. The only
`condition that offends the Constitution' found by the District Court in
this case is the existence of officially supported segregation in and
among public schools in Detroit itself. There were no findings that the
differing racial composition between schools in the city and in the
outlying suburbs was caused by official activity of any sort. It follows
that the decision to include in the desegregation plan pupils from
school districts outside Detroit was not predicated upon any
constitutional violation involving those school districts. By approving
a remedy that would reach beyond the limits of the city of Detroit to
correct a constitutional violation found to have occurred solely within
that city the Court of Appeals thus went beyond the governing equitable
principles established in this Court's decisions.

\emph{From the footnotes}:

\begin{enumerate}
\def\labelenumi{\arabic{enumi}.}
\setcounter{enumi}{1}
\tightlist
\item
  My Brother MARSHALL seems to ignore this fundamental fact when he
  states, post, that `the most essential finding (made by the District
  Court) was that Negro children in Detroit had been confined by
  intentional acts of segregation to a growing core of Negro schools
  surrounded by a receding ring of white schools.' This conclusion is
  simply not substantiated by the record presented in this case. The
  record here does support the claim made by the respondents that white
  and Negro students within Detroit who otherwise would have attended
  school together were separated by acts of the State or its
  subdivision. However, segregative acts within the city alone cannot be
  presumed to have produced---and no factual showing was made that they
  did produce---an increase in the number of Negro students in the city
  as a whole. It is this essential fact of a predominantly Negro school
  population in Detroit---caused by unknown and perhaps unknowable
  factors such as in-migration, birth rates, economic changes, or
  cumulative acts of private racial fears---that accounts for the
  `growing core of Negro schools,' a `core' that has grown to include
  virtually the entire city. The Constitution simply does not allow
  federal courts to attempt to change that situation unless and until it
  is shown that the State, or its political subdivisions, have
  contributed to cause the situation to exist. No record has been made
  in this case showing that the racial composition of the Detroit school
  population or that residential patterns within Detroit and in the
  surrounding areas were in any significant measure caused by
  governmental activity, and it follows that the situation over which my
  dissenting Brothers express concern cannot serve as the predicate for
  the remedy adopted by the District Court and approved by the Court of
  Appeals.
\end{enumerate}

\textbf{Mr.~Justice DOUGLAS, dissenting.}

The Court of Appeals has acted responsibly in these cases and we should
affirm its judgment. This was the fourth time the case was before it
over a span of less than three years. The Court of Appeals affirmed the
District

Court on the issue of segregation and on the `Detroit-only' plans of
desegregation. The Court of Appeals also approved in principle the use
of a metropolitan area plan, vacating and remanding only to allow the
other affected school districts to be brought in as parties, and in
other minor respects.

We have before us today no plan for integration. The only orders entered
so far are interlocutory. No new principles of law are presented here.
Metropolitan treatment of metropolitan problems is commonplace. If this
were a sewage problem or a water problem, or an energy problem, there
can be no doubt that Michigan would stay well within federal
constitutional bounds if it sought a metropolitan remedy. In Bradley v.
School Board of City of Richmond, 4 Cir., aff'd by an equally divided
Courtwe had a case involving the Virginia school system where local
school boards had `exclusive jurisdiction' of the problem, not `the
State Board of Education,' 462 F d.~Here the Michigan educational system
is unitary, maintained and supported by the legislature and under the
general supervision of the State Board of Education The State controls
the boundaries of school districts The State supervises schoolsite
selection The construction is done through municipal bonds approved by
several state agencies Education in Michigan is a state project with
very little completely local control,5 except that the schools are
financed locally, not on a statewide basis. Indeed the proposal to put
school funding in Michigan on a statewide basis was defeated at the
polls in November 1972. 6 Yet the school districts by state law are
agencies of the State State action is indeed challenged as violating the
Equal Protection Clause. Whatever the reach of that claim may be, it
certainly is aimed at discrimination based on race.

Therefore as the Court of Appeals held there can be no doubt that as a
matter of Michigan law the State itself has the final say as to where
and how school district lines should be drawn.

When we rule against the metropolitan area remedy we take a step that
will likely put the problems of the blacks and our society back to the
period that antedated the `separate but equal' regime of Plessy v.
FergusonThe reason is simple.

The inner core of Detroit is now rather solidly black and the blacks, we
know, in many instances are likely to be poorer,10 just as were the
Chicanos in San Antonio School District v. RodriguezBy that decision the
poorer school districts11 must pay their own way. It is therefore a
foregone conclusion that we have now given the States a formula whereby
the poor must pay their own way.

Today's decision, given Rodriguez, means that there is no violation of
the Equal Protection Clause though the schools are segregated by race
and though the black schools are not only `separate' but `inferior.'

So far as equal protection is concerned we are now in a dramatic retreat
from the 7-to-1 decision in 1896 that blacks could be segregated in
public facilities, provided they received equal treatment.

As I indicated in Keyes v. School District No.~1, Denver, Colorado,
there is so far as the school cases go no constitutional difference
between de facto and de jure segregation. Each school board performs
state action for Fourteenth Amendment purposes when it draws the lines
that confine it to a given area, when it builds schools at particular
sites, or when it allocates students. The creation of the school
districts in Metropolitan Detroit either maintained existing segregation
or caused additional segregation. Restrictive covenants maintained by
state action or inaction build black ghettos. It is state action when
public funds are dispensed by housing agencies to build racial ghettos.
Where a community is racially mixed and school authorities segregate
schools, or assign black teachers to black schools or close schools in
fringe areas and build new schools in black areas and in more distant
white areas, the State creates and nurtures a segregated school system,
just as surely as did those States involved in Brown v. Board of
Educationwhen they maintained dual school systems.

All these conditions and more were found by the District Court to exist.
The issue is not whether there should be racial balance but whether the
State's use of various devices that end up with black schools and white
schools brought the Equal Protection Clause into effect. Given the
State's control over the educational system in Michigan, the fact that
the black schools are in one district and the white schools are in
another is not controlling---either constitutionally or equitably No
specific plan has yet been adopted. We are still at an interlocutory
stage of a long drawn-out judicial effort at school desegregation. It is
conceivable that ghettos develop on their own without any hint of state
action. But since Michigan by one device or another has over the years
created black school districts and white school districts, the task of
equity is to provide a unitary system for the affected area where, as
here, the State washes its hands of its own creations.

\textbf{Mr.~Justice WHITE, with whom Mr.~Justice DOUGLAS, Mr.~Justice
BRENNAN, and Mr.~Justice MARSHALL join, dissenting.}

The District Court and the Court of Appeals found that over a long
period of years those in charge of the Michigan public schools engaged
in various practices calculated to effect the segregation of the Detroit
school system. The Court does not question these findings, nor could it
reasonably do so. Neither does it question the obligation of the federal
courts to devise a feasible and effective remedy. But it promptly
cripples the ability of the judiciary to perform this task, which is of
fundamental importance to our constitutional system, by fashioning a
strict rule that remedies in school cases must stop at the school
district line unless certain other conditions are met. As applied here,
the remedy for unquestioned violations of the protection rights of
Detroit's Negroes by the Detroit School Board and the State of Michigan
must be totally confined to the limits of the school district and may
not reach into adjoining or surrounding districts unless and until it is
proved there has been some sort of `interdistrict violation'---unless
unconstitutional actions of the Detroit School Board have had a
segregative impact on other districts, or unless the segregated
condition of the Detroit schools has itself been influenced by
segregative practices in those surrounding districts into which it is
proposed to extend the remedy.

Regretfully, and for several reasons, I can join neither the Court's
judgment nor its opinion. The core of my disagreement is that deliberate
acts of segregation and their consequences will go unremedied, not
because a remedy would be infeasible or unreasonable in terms of the
usual criteria governing school desegregation cases, but because an
effective remedy would cause what the Court considers to be undue
administrative inconvenience to the State. The result is that the State
of Michigan, the entity at which the Fourteenth Amendment is directed,
has successfully insulated itself from its duty to provide effective
desegregation remedies by vesting sufficient power over its public
schools in its local school districts. If this is the case in Michigan,
it will be the case in most States.

There are undoubted practical as well as legal limits to the remedial
powers of federal courts in school desegregation cases. The Court has
made it clear that the achievement of any particular degree of racial
balance in the school system is not required by the Constitution; nor
may it be the primary focus of a court in devising an acceptable remedy
for de jure segregation. A variety of procedures and techniques are
available to a district court engrossed in fashioning remedies in a case
such as this; but the courts must keep in mind that they are dealing
with the process of educating the young, including the very young. The
task is not to devise a system of pains and penalties to punish
constituttional violations brought to light. Rather, it is to
desegregate an educational system in which the races have been kept
apart, without, at the same time, losing sight of the central
educational function of the schools.

Viewed in this light, remedies calling for school zoning, pairing, and
pupil assignments, become more and more suspect as they require that
schoolchildren spend more and more time in buses going to and from
school and that more and more educational dollars be diverted to
transportation systems. Manifestly, these considerations are of
immediate and urgent concern when the issue is the desegregation of a
city school system where residential patterns are predominantly
segregated and the respective areas occupied by blacks and whites are
heavily populated and geographically extensive. Thus, if one postulates
a metropolitan school system covering a sufficiently large area, with
the population evenly divided between whites and Negroes and with the
races occupying identifiable residential areas, there will be very real
practical limits on the extent to which racially identifiable schools
can be eliminated within the school district. It is also apparent that
the larger the proportion of Negroes in the area, the more difficult it
would be to avoid having a substantial number of all-black or nearly
all-black schools.

The Detroit school district is both large and heavily populated. It
covers 139 square miles, encircles two entirely separate cities and
school districts, and surrounds a third city on three sides. Also,
whites and Negroes live in identifiable areas in the city. The 1970
public school enrollment in the city school district totaled 289,763 and
was 63 \% Negro and 34 \% white. 1 If `racial balance' were achieved in
every school in the district, each school would be approximately 64\%
Negro. A remedy confined to the district could achieve no more
desegregation. Furthermore, the proposed intracity remedies were beset
with practical problems. None of the plans limited to the school
district was satisfactory to the District Court. The most promising
proposal, submitted by respondents, who were the plaintiffs in the
District Court, would `leave many of its schools 75 to 90 per cent
Black.' CA6 1973) Transportation on a `vast scale' would be required;
900 buses would have to be purchased for the transportation of pupils
who are not now bused. The District Court also found that the plan
`would change a school system which is now Black and White to one that
would be perceived as Black, thereby increasing the flight of Whites
from the city and the system, thereby increasing the Black student
population.' For the District Court, '(t)he conclusion, under the
evidence in this case, is inescapable that relief of segregation in the
public schools of the

City of Detroit cannot be accomplished within the corporate geographical
limits of the city.'

The District Court therefore considered extending its remedy of the
suburbs. After hearings, it concluded that a much more effective
desegregation plan could be implemented if the suburban districts were
included. In proceeding to design its plan on the basis that student bus
rides to and from school should not exceed 40 minutes each way as a
general matter, the court's express finding was that `(f)or all the
reasons stated heretofore including time, distance, and transportation
factors---desegregation within the area described is physically easier
and more practicable and feasible, than desegregation efforts limited to
the corporate geographic limits of the city of Detroit.' 345 F.Supp.
914, 930 (ED Mich ).

The Court of Appeals agreed with the District Court that the remedy must
extend beyond the city limits of Detroit. It concluded that `(i)n the
instant case the only feasible desegregation plan involves the crossing
of the boundary lines between the Detroit School District and adjacent
or nearby school districts for the limited purpose of providing an
effective desegregation plan.' 484 F d.~(Emphasis added.) It also agreed
that `any Detroit only desegregation plan will lead directly to a single
segregated Detroit school district overwhelmingly black in all of its
schools, surrounded by a ring of suburbs and suburban school districts
overwhelmingly white in composition in a State in which the racial
composition is 87 per cent white and 13 per cent black.' There was `more
than ample support for the District Judge's findings of unconstitutional
segregation by race resulting in major part from action and inaction of
public authorities, both local and State. . Under this record a remedial
order of a court of equity which left the Detroit school system
overwhelmingly black (for the fore- seeable future) surrounded by
suburban school systems overwhelmingly white cannot correct the
constitutional violations herein found.' To conclude otherwise, the
Court of Appeals announced, would call up `haunting memories of the now
long overruled and discredited 'separate but equal doctrine' of Plessy
v. Ferguson) . (1896),' and `would be opening a way to nullify Brown v.
Board of Education which overruled Plessy. . .' 484 F d.

This Court now reverses the Court of Appeals. It does not question the
District Court's findings that any feasible Detroit-only plan would
leave many schools 75 to 90 percent black and that the district would
become progressively more black as whites left the city. Neither does
the Court suggest that including the suburbs in a desegregation plan
would be impractical or infeasible because of educational
considerations, because of the number of children requiring
transportation, or because of the length of their rides. Indeed, the
Court leaves unchallenged the District Court's conclusion that a plan
including the suburbs would be physically easier and more practical and
feasible than a Detroit-only plan. Whereas the most promising
Detroit-only plan, for example, would have entailed the purchase of 900
buses, the metropolitan plan would involve the acquisition of no more
than 350 new vehicles.

Despite the fact that a metropolitan remedy, if the findings of the
District Court accepted by the Court of Appeals are to be credited,
would more effectively desegregate the Detroit schools, would prevent
resegregation,3 and would be easier and more feasible from many
standpoints, the Court fashions out of whole cloth an arbitrary rule
that remedies for constitutional violations occurring in a single
Michigan school district must stop at the school district line.
Apparently, no matter how much less burdensome or more effective and
efficient in many respects, such as transportation, the metropolitan
plan might be, the school district line may not be crossed. Otherwise,
it seems, there would be too much disruption of the Michigan scheme for
managing its educational system, too much confusion, and too much
administrative burden.

The District Court, on the scene and familiar with local conditions, had
a wholly different view. The Court of Appeals also addressed itself at
length to matters of local law and to the problems that interdistrict
remedies might present to the State of Michigan. Its conclusion, flatly
contrary to that of this Court, was that `the constitutional right to
equality before the law (is not) hemmed in by the boundaries of a school
district' and that an interdistrict remedy

`is supported by the status of school districts under Michigan law and
by the historical control exercised over local school districts by the
legislature of Michigan and by State agencies and officials . .. (I)t is
well established under the Constitution and laws of Michigan that the
public school system is a State function and that local school districts
are instrumentalities of the State created for administrative
convenience.' 484 F d---246.

I am surprised that the Court, sitting at this distance from the State
of Michigan, claims better insight than the Court of Appeals and the
District Court as to whether an interdistrict remedy for equal
protection violations practiced by the State of Michigan would involve
undue difficulties for the State in the management of its public
schools. In the area of what constitutes an acceptable desegregation
plan, `we must of necessity rely to a large extent, as this Court has
for more than 16 years, on the informed judgment of the district courts
in the first instance and on courts of appeals.' Swann v.
Charlotte-Mecklenburg Board of Education, Obviously, whatever
difficulties there might be, they are surmountable; for the Court itself
concedes that, had there been sufficient evidence of an interdistrict
violation, the District Court could have fashioned a single remedy for
the districts implicated rather than a different remedy for each
district in which the violation had occurred or had an impact.

I am even more mystified as to how the Court can ignore the legal
reality that the constitutional violations, even if occurring locally,
were committed by governmental entities for which the State is
responsible and that it is the State that must respond to the command of
the Fourteenth Amendment. An interdistrict remedy for the infringements
that occurred in this case is well within the confines and powers of the
State, which is the governmental entity ultimately responsible for
desegregating its schools. The Michigan Supreme Court has observed that
`(t)he school district is a State agency,' Attorney General ex rel. Kies
v. Lowrey, 131 Mich. 639, 644, 92 N.W. 289, 290 (1902), and that
``(e)ducation in Michigan belongs to the State. It is no part of the
local self-government inherent in the township or municipality, except
so far as the legislature may choose to make it such. The Constitution
has turned the whole subject over to the legislature. . .'' Attorney
General ex rel. Lacharias v. Detroit Board of Education, 154 Mich. 584,
590, 118 N.W. 606, 609 (1908).

It is unnecessary to catalogue at length the various public misdeeds
found by the District Court and the Court of Appeals to have contributed
to the present segregation of the Detroit public schools. The
legislature contributed directly by enacting a statute overriding a
partial high school desegregation plan voluntarily adopted by the
Detroit Board of Education. Indirectly, the trial court found the State
was accountable for the thinly disguised, pervasive acts of segregation
committed by the Detroit Board,5 for Detroit's school construction plans
that would promote segregation, and for the Detroit school district's
not having funds for pupil transportation within the district. The State
was also chargeable with responsibility for the transportation of Negro
high school students in the late 1950's from the suburban Ferndale
School District, past closer suburban and Detroit high schools with
predominantly white student bodies, to a predominantly Negro high school
within Detroit. Swann v. Charlotte-Mecklenburg Board of Education---21,
and Keyes v. School District No.~1, Denver, Coloradomake abundantly
clear that the tactics employed by the Detroit Board of Education, a
local instrumentality of the State, violated the constitutional rights
of the Negro students in Detroit's public schools and required equitable
relief sufficient to accomplish the maximum, practical desegregation
within the power of the political body against which the Fourteenth
Amendment directs its proscriptions. No `State' may deny any individual
the equal protection of the laws; and if the Constitution and the
Supremacy Clause are to have any substance at all, the courts must be
free to devise workable remedies against the political entity with the
effective power to determine local choice. It is also the case here that
the State's legislative interdiction of Detroit's voluntary effort to
desegregate its school system was unconstitutional.

The Court draws the remedial line at the Detroit school district
boundary, even though the Fourteenth Amendment is addressed to the State
and even though the State denies equal protection of the laws when its
public agencies, acting in its behalf, invidiously discriminate. The
State's default is `the condition that offends the Constitution,' Swann
v. Charlotte-Mecklenburg Board of Educationand state officials may
therefore be ordered to take the necessary measures to completely
eliminate from the Detroit public schools `all vestiges of state-imposed
segregation.' I cannot understand, nor does the majority satisfactorily
explain, why a federal court may not order an appropriate interdistrict
remedy, if this is necessary or more effective to accomplish this
constitutionally mandated task. As the Court unanimously observed in
Swann: `Once a right and a violation have been shown, the scope of a
district court's equitable powers to remedy past wrongs is broad, for
breadth and flexibility are inherent in equitable remedies.' In this
case, both the right and the State's Fourteenth Amendment violation have
concededly been fully established, and there is no acceptable reason for
permitting the party responsible for the constitutional violation to
contain the remedial powers of the federal court within administrative
boundaries over which the transgressor itself has plenary power.

The unwavering decisions of this Court over the past 20 years support
the assumption of the Court of Appeals that the District Court's
remedial power does not cease at the school district line. The Court's
first formulation of the remedial principles to be followed in
disestablishing racially discriminatory school systems recognized the
variety of problems arising from different local school conditions and
the necessity for that `practical flexibility' traditionally associated
with courts of equity. Brown v. Board of Education, (1955) (Brown II).
Indeed, the district courts to which the Brown cases were remanded for
the formulation of remedial decrees were specifically instructed that
they might consider, inter alia, `revision of school districts and
attendance areas into compact units to achieve a system of determining
admission to the public schools on a nonracial basis . ..' ---301, The
malady addressed in Brown was the statewide policy of requiring or
permitting school segregation on the basis of race, while the record
here concerns segregated schools only in the city of Detroit. The
obligation to rectify the unlawful condition nevertheless rests on the
State. The permissible revision of school districts contemplated in
Brown rested on the State's responsibility for desegregating its
unlawfully segregated schools, not on any segregative effect which the
condition of segregation in one school district might have had on the
schools of a neighboring district. The same situation obtains here and
the same remedial power is available to the District Court.

Later cases reinforced the clearly essential rules that state officials
are fully answerable for unlawfully caused conditions of school
segregation which can effectively be controlled only by steps beyond the
authority of local school districts to take, and that the equity power
of the district courts includes the ability to order such measures
implemented. When the highest officials of the State of Arkansas impeded
a federal court order to desegregate the public schools under the
immediate jurisdiction of the Little Rock School Board, this Court
refused to accept the local board's assertion of its good faith as a
legal excuse for delay in implementing the desegregation order. The
Court emphasized that `from the point of view of the Fourteenth
Amendment, they (the local school board members) stand in this
litigation as the agents of the State.' Cooper v. Aaron, Per- haps more
importantly for present purposes, the Court went on to state:

`The record before us clearly establishes that the growth of the Board's
difficulties to a magnitude beyond its unaided power to control is the
product of state action. Those difficulties . can also be brought under
control by state action.'

In the context of dual school systems, the Court subsequently made clear
the `affirmative duty to take whatever steps might be necessary to
convert to a unitary system in which racial discrimination would be
eliminated root and branch' and to come forward with a desegregation
plan that `promises realistically to work now.' Green v. County School
Board of New Kent County, `Freedom of choice' plans were rejected as
acceptable desegregation measures where `reasonably available other ways
. promising speedier and more effective conversion to a unitary,
nonracial school system . .' exist. Imperative insistence on immediate
full desegregation of dual school systems `to operate now and hereafter
only unitary schools' was reiterated in Alexander v. Holmes County Board
of Education, and Carter v. West Feliciana Parish School Board.

The breadth of the equitable authority of the district courts to
accomplish these comprehensive tasks was reaffirmed in much greater
detail in Swann v. Charlotte-Mecklenburg Board of Educationand the
companion case of Davis v. School Comm'rs of Mobile Countywhere there
was unanimous assent to the following propositions:

`Having once found a violation, the district judge or school authorities
should make every effort to achieve the greatest possible degree of
actual desegregation, taking into account the practicalities of the
situation. A district court may and should consider the use of all
available techniques including restructuring of attendance zones and
both contiguous and noncontiguous attendance zones. . The measure of any
desegregation plan is its effectiveness.'

No suggestion was made that interdistrict relief was not an available
technique. In Swann v. Charlotte-Mecklenburg Board of Education itself,
the Court, without dissent, recognized that the District Judge, in
fulfilling his obligation to `make every effort to achieve the greatest
possible degree of actual desegregation(,) will thus necessarily be
concerned with the elimination of one-race schools.' 402 U.S., Nor was
there any dispute that to break up the dual school system, it was within
the District Court's `broad remedial powers' to employ a `frank---and
sometimes drastic---gerrymandering of school districts and attendance
zones,' as well as `pairing, 'clustering,' or `grouping' of schools,' to
desegregate the `formerly all-Negro schools,' despite the fact that
these zones might not be compact or contiguous and might be `on opposite
ends of the city.' The school board in that case had jurisdiction over a
550-square-mile area encompassing the city of Charlotte and surrounding
Mecklenburg County, North Carolina. The Mobile County, Alabama, board in
Davis embraced a 1,248-squaremile area, including the city of Mobile.
Yet the Court approved the District Court's authority to award
countywide relief in each case in order to accomplish desegregation of
the dual school system.

Even more recently, the Court specifically rejected the claim that a new
school district, which admittedly would operate a unitary school system
within its borders, was beyond the reach of a court-ordered
desegregation plan for other school districts, where the effectiveness
of the plan as to the other districts depended upon the availability of
the facilities and student population of the new district. In Wright v.
Council of City of Emporia, we held `that a new school district may not
be created where its effect would be to impede the process of
dismantling a dual system.' Mr.~Justice Stewart's opinion for the Court
made clear that if a proposal to erect new district boundary lines
`would impede the dismantling of the (pre-existing) dual system, then a
district court, in the exercise of its remedial discretion, may enjoin
it from being carried out.' In United States v. Scotland Neck Board of
Educationthis same standard was applied to forbid North Carolina from
creating a new city school district within a larger district which was
in the process of dismantling a dual school system. The Court noted that
if establishment of the new district were permitted, the `traditional
racial identities of the schools in the area would be maintained.'

Until today, the permissible contours of the equitable authority of the
district courts to remedy the unlawful establishment of a dual school
system have been extensive, adaptable, and fully responsive to the
ultimate goal of achieving `the greatest possible degree of actual
desegregation.' There are indeed limitations on the equity powers of the
federal judiciary, but until now the Court had not accepted the
proposition that effective enforcement of the Fourteenth Amendment could
be limited by political or administrative boundary lines demarcated by
the very State responsible for the constitutional violation and for the
disestablishment of the dual system. Until now the Court has instead
looked to practical considerations in effectuating a desegregation
decree, such as excessive distance, transportation time, and hazards to
the safety of the schoolchildren involved in a proposed plan. That these
broad principles have developed in the context of dual school systems
compelled or authorized by state statute at the time of Brown v. Board
of Education (Brown I), does not lessen their current applicability to
dual systems found to exist in other contexts, like that in Detroit,
where intentional school segregation does not stem from the compulsion
of state law, but from deliberate individual actions of local and state
school authorities directed at a particular school system. The majority
properly does not suggest that the duty to eradicate completely the
resulting dual system in the latter context is any less than in the
former. But its reason for incapacitating the remedial authority of the
federal judiciary in the presence of school district perimeters in the
latter context is not readily apparent.

The result reached by the Court certainly cannot be supported by the
theory that the configuration of local governmental units is immune from
alteration when necessary to redress constitutional violations. In
addition to the well-established principles already noted, the Court has
elsewhere required the public bodies of a State to restructure the
State's political subdivisions to remedy infringements of the
constitutional rights of certain members of its populace, notably in the
reapportionment cases. In Reynolds v. Simsfor example, which held that
equal protection of the laws demands that the seats in both houses of a
bicameral state legislature be apportioned on a population basis, thus
necessitating wholesale revision of Alabama's voting districts, the
Court remarked:

`Political subdivisions of States---counties, cities, or
whatever---never were and never have been con- sidered as sovereign
entities. Rather, they have been traditionally regarded as subordinate
governmental instrumentalities created by the State to assist in the
carrying out of state governmental functions.' And even more pointedly,
the Court declared in Gomillion v. Lightfoot---345, that '(l) egislative
control of municipalities, no less than other state power, lies within
the scope of relevant limitations imposed by the United States
Constitution.

Nor does the Court's conclusion follow from the talismanic invocation of
the desirability of local control over education. Local autonomy over
school affairs, in the sense of the community's participation in the
decisions affecting the education of its children, is, of course, an
important interest. But presently constituted school district lines do
not delimit fixed and unchangeable areas of a local educational
community. If restructuring is required to meet constitutional
requirements, local authority may simply be redefined in terms of
whatever configuration is adopted, with the parents of the children
attending schools in the newly demarcated district or attendance zone
continuing their participation in the policy management of the schools
with which they are concerned most directly. The majority's suggestion
that judges should not attempt to grapple with the administrative
problems attendant on a reorganization of school attendance patterns is
wholly without foundation. It is precisely this sort of task which the
district courts have been properly exercising to vindicate the
constitutional rights of Negro students since Brown I and which the
Court has never suggested they lack the capacity to perform.
Intradistrict revisions of attendance zones, and pairing and grouping of
schools, are techniques unanimously approved in Swann v.
Charlotte-Mecklenburg Board of Education which entail the same
sensitivity to the interet of parents in the education their children
receive as would an interditrict plan which is likely to employ the very
same methods. There is no reason to suppose that the District Court,
which has not yet adopted a final plan of desegregation, would not be as
capable of giving or as likely to give sufficient weight to the interest
in community participation in schools in an interdistrict setting,
consistent with the dictates of the Fourteenth Amendment. The majority's
assumption that the District Court would act otherwise is a radical
departure from the practical flexibility previously left to the equity
powers of the federal judiciary.

Finally, I remain wholly unpersuaded by the Court's assertion that `the
remedy is necessarily designed, as all remedies are, to restore the
victims of discriminatory conduct to the position they would have
occupied in the absence of such conduct.' Ante, p.~746. In the first
place, under this premise the Court's judgment is itself infirm; for had
the Detroit school system not followed an official policy of segregation
throughout the 1950's and 1960's, Negroes and whites would have been
going to school together. There would have been no, or at least not as
many, recognizable Negro schools and no, or at least not as many, white
schools, but `just schools,' and neither Negroes nor whites would have
suffered from the effects of segregated education, will all its
shortcomings. Surely the Court's remedy will not restore to the Negro
community, stigmatized as it was by the dual school system, what it
would have enjoyed over all or most of this period if the remedy is
confined to present-day Detroit; for the maximum remedy available within
that area will leave many of the schools almost totally black, and the
system itself will be predominantly black and will become increasingly
so. Moreover, when a State has engaged in acts of official segregation
over a lengthy period of time, as in the case before us, it is
unrealistic to suppose that the children who were victims of the State's
unconstitutional conduct could now be provided the benefits of which
they were wrongfully deprived. Nor can the benefits which accrue to
school systems in which schoolchildren have not been officially
segregated, and to the communities supporting such school systems, be
fully and immediately restored after a substantial period of unlawful
segregation. The education of children of different races in a
desegregated environment has unhappily been lost, along with the social,
economic, and political advantages which accompany a desegregated school
system as compared with an unconstitutionally segregated system. It is
for these reasons that the Court has consistently followed the course of
requiring the effects of past official segregation to be eliminated
`root and branch' by imposing, in the present, the duty to provide a
remedy which will achieve `the greatest possible degree of actual
desegregation, taking into account the practicalities of the situation.'
It is also for these reasons that once a constitutional violation has
been found, the district judge obligated to provide such a remedy `will
thus necessarily be concerned with the elimination of one-race schools.'
These concerns were properly taken into account by the District Judge in
this case. Confining the remedy to the boundaries of the Detroit
district is quite unrelated either to the goal of achieving maximum
desegregation or to those intensely practical considerations, such as
the extent and expense of transportation, that have imposed limits on
remedies in cases such as this. The Court's remedy, in the end, is
essentially arbitrary and will leave serious violations of the
Constitution substantially unremedied.

I agree with my Brother DOUGLAS that the Court of Appeals has acted
responsibly in these cases. Regre- tably, the majority's arbitrary
limitation on the equitable power of federal district courts, based on
the invisible borders of local school districts, is unrelated to the
State's responsibility for remedying the constitutional wrongs visited
upon the Negro schoolchildren of Detroit. It is oblivious to the
potential benefits of metropolitan relief, to the noneducational
communities of interest among neighborhoods located in and sometimes
bridging different school districts, and to the considerable
interdistrict cooperation already existing in various educational areas.
Ultimately, it is unresponsive to the goal of attaining the utmost
actual desegregation consistent with restraints of practicability and
thus augurs the frequent frustration of the remedial powers of the
federal courts.

Here the District Court will be forced to impose an intracity
desegregation plan more expensive to the district, more burdensome for
many of Detroit's Negro students, and surely more conductive to white
flight than a metropolitan plan would be---all of this merely to avoid
what the Detroit School Board, the District Court, and the en banc Court
of Appeals considered to be the very manageable and quite surmountable
difficulties that would be involved in extending the desegregation
remedy to the suburban school districts.

I am therefore constrained to record my disagreement and dissent.

\textbf{Mr.~Justice MARSHALL, with whom Mr.~Justice DOUGLAS, Mr.~Justice
BRENNAN, and Mr.~Justice WHITE join, dissenting.}

In Brown v. Board of Education this Court held that segregation of
children in public schools on the basis of race deprives minority group
childen of equal educational opportunities and therefore denies them the
equal protection of the laws under the Fourteenth Amendment. This Court
recognized then that remedying decades of segregation in public
education would not be an easy task. Subsequent events, unfortunately,
have seen that prediction bear bitter fruit. But however imbedded old
ways, however ingrained old prejudices, this Court has not been diverted
from its appointed task of making `a living truth' of our constitutional
ideal of equal justice under law. Cooper v. Aaron,

After 20 years of small, often difficult steps toward that great end,
the Court today takes a giant step backwards. Notwithstanding a record
showing widespread and pervasive racial segregation in the educational
system provided by the State of Michigan for children in Detroit, this
Court holds that the District Court was powerless to require the State
to remedy its constitutional violation in any meaningful fashion.
Ironically purporting to base its result on the principle that the scope
of the remedy in a desegregation case should be determined by the nature
and the extent of the constitutional violation, the Court's answer is to
provide no remedy at all for the violation proved in this case, thereby
guaranteeing that Negro children in Detroit will receive the same
separate and inherently unequal education in the future as they have
been unconstitutionally afforded in the past.

I cannot subscribe to this emasculation of our constitutional guarantee
of equal protection of the laws and must respectfully dissent. Our
precedents, in my view, firmly establish that where, as here,
state-imposed segregation has been demonstrated, it becomes the duty of
the State to eliminate root and branch all vestiges of racial
discrimination and to achieve the greatest possible degree of actual
desegregation. I agree with both the District Court and the Court of
Appeals that, under the facts of this case, this duty cannot be
fulfilled unless the State of Michigan involves outlying metropolitan
area school districts in its desegregation remedy. Furthermore, I
perceive no basis either in law or in the practicalities of the
situation justifying the State's interposition of school district
boundaries as absolute barriers to the implementation of an effective
desegregation remedy. Under established and frequently used Michigan
procedures, school district lines are both flexible and permeable for a
wide variety of purposes, and there is no reason why they must now stand
in the way of meaningful desegregation relief.

The rights at issue in this case are too fundamental to be abridged on
grounds as superficial as those relied on by the majority today. We deal
here with the right of all of our children, whatever their race, to an
equal start in life and to an equal opportunity to reach their full
potential as citizens. Those children who have been denied that right in
the past deserve better than to see fences thrown up to deny them that
right in the future. Our Nation, I fear, will be ill served by the
Court's refusal to remedy separate and unequal education, for unless our
children begin to learn together, there is little hope that our people
will ever learn to live together.

The great irony of the Court's opinion and, in my view, its most serious
analytical flaw may be gleaned from its concluding sentence, in which
the Court remands for `prompt formulation of a decree directed to
eliminating the segregation found to exist in Detroit city schools, a
remedy which has been delayed since 1970.' The majority, however, seems
to have forgotten the District Court's explicit finding that a
Detroit-only decree, the only remedy permitted under today's decision,
`would not accomplish desegregation.'

Nowhere in the Court's opinion does the majority confront, let alone
respond to, the District Court's conclusion that a remedy limited to the
city of Detroit would not effectively desegregate the Detroit city
schools. I, for one, find the District Court's conclusion well supported
by the record and its analysis compelled by our prior cases. Before
turning to these questions, however, it is best to begin by laying to
rest some mischaracterizations in the Court's opinion with respect to
the basis for the District Court's decision to impose a metropolitan
remedy.

The Court maintains that while the initial focus of this lawsuit was the
condition of segregation within the Detroit city schools, the District
Court abruptly shifted focus in mid-course and altered its theory of the
case. This new theory, in the majority's words, was `equating racial
imbalance with a constitutional violation calling for a remedy.' n.~19.
As the following review of the District Court's handling of the case
demonstrates, however, the majority's characterization is totally
inaccurate. Nowhere did the District Court indicate that racial
imbalance between school districts in the Detroit metropolitan area or
within the Detroit School District constituted constitutional violation
calling for interdistrict relief. The focus of this case was from the
beginning, and has remained, the segregated system of education in the
Detroit city schools and the steps necessary to cure that condition
which offends the Fourteenth Amendment.

The District Court's consideration of this case began with its finding,
which the majority accepts, that the State of Michigan, through its
instrumentality, the Detroit Board of Education, engaged in widespread
purposeful acts of racial segregation in the Detroit School District.
Without belaboring the details, it is sufficient to note that the
various techniques used in Detroit were typical of methods employed to
segregate students by race in areas where no statutory dual system of
education has existed. See, e.g., Keyes v. School District No.~1,
Denver, ColoradoExacerbating the effects of extensive residential
segregation between Negroes and whites, the school board consciously
drew attendance zones along lines which maximized the segregation of the
races in schools as well. Optional attendance zones were created for
neighborhoods undergoing racial transition so as to allow whites in
these areas to escape integration. Negro students in areas with
overcrowded schools were transported past or away from closer white
schools with available space to more distant Negro schools. Grade
structures and feeder-school patterns were created and maintained in a
manner which had the foreseeable and actual effect of keeping Negro and
white pupils in separate schools. Schools were also constructed in
locations and in sizes which ensured that they would open with
predominantly one-race student bodies. In sum, the evidence adduced
below showed that Negro children had been intentionally confined to an
expanding core of virtually all-Negro schools immediately surrounded by
a receding band of all-white schools.

Contrary to the suggestions in the Court's opinion, the basis for
affording a desegregation remedy in this case was not some perceived
racial imbalance either between schools within a single school district
or between independent school districts. What we confront here is `a
systematic program of segregation affecting a substantial portion of the
students, schools . and facilities within the school system . ..' U.S.,
The constitutional violation found here was not some de facto racial
imbalance, but rather the purposeful, intentional, massive, de jure
segregation of the Detroit city schools, which under our decision in
Keyes, forms `a predicate for a finding of the existence of a dual
school system,' , and justifies `all-out desegregation.'

Having found a de jure segregated public school system in operation in
the city of Detroit, the District Court turned next to consider which
officials and agencies should be assigned the affirmative obligation to
cure the constitutional violation. The court concluded that
responsibility for the segregation in the Detroit city schools rested
not only with the Detroit Board of Education, but belonged to the State
of Michigan itself and the state defendants in this case that is, the
Governor of Michigan, the Attorney General, the State Board of
Education, and the State Superintendent of Public Instruction. While the
validity of this conclusion will merit more extensive analysis below,
suffice it for now to say that it was based on three considerations.
First, the evidence at trial showed that the State itself had taken
actions contributing to the segregation within the Detroit schools.
Second, since the Detroit Board of Education was an agency of the State
of Michigan, its acts of racial discrimination were acts of the State
for purposes of the Fourteenth Amendment. Finally, the District Court
found that under Michigan law and practice, the system of education was
in fact a state school system, characterized by relatively little local
control and a large degree of centralized state regulation, with respect
to both educational policy and the structure and operation of school
districts.

Having concluded, then, that the school system in the city of Detroit
was a de jure segregated system and that the State of Michigan had the
affirmative duty to remedy that condition of segregation, the District
Court then turned to the difficult task of devising an effective remedy.
It bears repeating that the District Court's focus at this stage of the
litigation remained what it had been at the beginning---the condition of
segregation within the Detroit city schools. As the District Court
stated: `From the initial ruling (on segregation) to this day, the basis
of the proceedings has been and remains the violation: de jure school
segregation. . The task before this court, therefore, is now, and . has
always been, how to desegregate the Detroit public schools.'

The District Court first considered three desegregation plans limited to
the geographical boundaries of the city of Detroit. All were rejected as
ineffective to desegregate the Detroit city schools. Specifically, the
District Court determined that the racial composition of the Detroit
student body is such that implementation of any Detroit-only plan `would
clearly make the entire Detroit public school system racially
identifiable as Black' and would `leave many of its schools 75 to 90 per
cent Black.' The District Court also found that a Detroit-only plan
`would change a school system which is now Black and White to one that
would be perceived as Black, thereby increasing the flight of Whites
from the city and the system, thereby increasing the Black student
population.' Based on these findings, the District Court reasoned that
`relief of segregation in the public schools of the City of Detroit
cannot be accomplished within the corporate geographical limits of the
city' because a Detroit-only decree `would accentuate the racial
identifiability of the district as a Black school system, and would not
accomplish desegregation.' The District Court therefore concluded that
it `must look beyond the limits of the Detroit school district for a
solution to the problem of segregation in the Detroit public schools .
..'

In seeking to define the appropriate scope of that expanded
desegregation area, however, the District Court continued to maintain as
its sole focus the condition shown to violate the Constitution in this
case---the segregation of the Detroit school system. As it stated, the
primary question `remains the determination of the area necessary and
practicable effectively to eliminate 'root and branch' the effects of
state-imposed and supported segregation and to desegregate the Detroit
public schools.'

There is simply no foundation in the record, then, for the majority's
accusation that the only basis for the District Court's order was some
desire to achieve a racial balance in the Detroit metropolitan area In
fact, just the contrary is the case. In considering proposed
desegregation areas, the District Court had occasion to criticize one of
the State's proposals specifically because it had no basis other than
its `particular racial ratio' and did not focus on `relevant factors,
like eliminating racially identifiable schools (and) accomplishing
maximum actual desegregation of the Detroit public schools.' Similarly,
in rejecting the Detroit School Board's proposed desegregation area,
even though it included more all-white districts and therefore achieved
a higher white-Negro ratio, the District Court commented:

`There is nothing in the record which suggests that these districts need
be included in the desegregation area in order to disestablish the
racial identifiability of the Detroit public schools. From the evidence,
the primary reason for the Detroit School Board's interest in the
inclusion of these school districts is not racial desegregation but to
increase the average socio-economic balance of all the schools in the
abutting regions and clusters.'

The Court also misstates the basis for the District Court's order by
suggesting that since the only segregation proved at trial was within
the Detroit school system, any relief which extended beyond the
jurisdiction of the Detroit Board of Education would be inappropriate
because it would impose a remedy on outlying districts `not shown to
have committed any constitutional violation.' The essential foundation
of interdistrict relief in this case was not to correct conditions
within outlying districts which themselves engaged in purposeful
segregation. Instead, interdistrict relief was seen as a necessary part
of any meaningful effort by the State of Michigan to remedy the
state-caused segregation within the city of Detroit.

Rather than consider the propriety of interdistrict relief on this
basis, however, the Court has conjured up a largely fictional account of
what the District Court was attempting to accomplish. With all due
respect, the Court, in my view, does a great disservice to the District
Judge who labored long and hard with this complex litigation by accusing
him of changing horses in midstream and shifting the focus of this case
from the pursuit of a remedy for the condition of segregation within the
Detroit school system to some unprincipled attempt to impose his own
philosophy of racial balance on the entire Detroit metropolitan area.
See ---739. The focus of this case has always been the segregated system
of education in the city of Detroit. The District Court determined that
interdistrict relief was necessary and appropriate only because it found
that the condition of segregation within the Detroit school system could
not be cured with a Detroit-only remedy. It is on this theory that the
interdistrict relief must stand or fall. Unlike the Court, I perceive my
task to be to review the District Court's order for what it is, rather
than to criticize it for what it manifestly is not.

As the foregoing demonstrates, the District Court's decision to expand
its desegregation decree beyond the geographical limits of the city of
Detroit rested in large part on its conclusions (A) that the State of
Michigan was ultimately responsible for curing the condition of
segregation within the Detroit city schools, and (B) that a Detroit-only
remedy would not accomplish this task. In my view, both of these
conclusions are well supported by the facts of this case and by this
Court's precedents.

To begin with, the record amply supports the District Court's findings
that the State of Michigan, through state officers and state agencies,
had engaged in purposeful acts which created or aggravated segregation
in the Detroit schools. The State Board of Education, for example, prior
to 1962, exercised its authority to supervise local schoolsite selection
in a manner which contributed to segregation. CA6 1973). Furthermore,
the State's continuing authority, after 1962, to approve school building
construction plans3 had intertwined the State with site-selection
decisions of the Detroit Board of Education which had the purpose and
effect of maintaining segregation.

The State had also stood in the way of past efforts to desegregate the
Detroit city schools. In 1970, for example, the Detroit School Board had
begun implementation of its own desegregation plan for its high schools,
despite considerable public and official resistance. The State
Legislature intervened by enacting Act 48 of the Public Acts of 1970,
specifically prohibiting implementation of the desegregation plan and
thereby continuing the growing segregation of the Detroit school system.
Adequate desegregation of the Detroit system was also hampered by
discriminatory restrictions placed by the State on the use of
transportation within Detroit. While state aid for transportation was
provided by statute for suburban districts, many of which were highly
urbanized, aid for intracity transportation was excepted. One of the
effects of this restriction was to encourage the construction of small
walk-in neighborhood schools in Detroit, thereby lending aid to the
intentional policy of creating a school system which reflected, to the
greatest extent feasible, extensive residential segregation. Indeed,
that one of the purposes of the transportation restriction was to impede
desegregation was evidenced when the Michigan Legislature amended the
State Transportation Aid Act to cover intracity transportation but
expressly prohibited the allocation of funds for cross-busing of
students within a school district to achieve racial balance.

Also significant was the State's involvement during the 1950's in the
transportation of Negro high school students from the Carver
School-District past a closer white high school in the Oak Park District
to a more distant Negro high school in the Detroit system. Certainly the
District Court's finding that the State Board of Education had knowledge
of this action and had given its tacit or express approval was not
clearly erroneous. Given the comprehensive statutory powers of the State
Board of Education over contractual arrangements between school
districts in the enrollment of students on a nonresident tuition basis,
including certification of the number of pupils involved in the transfer
and the amount of tuition charged, over the review of transportation
routes and distances, and over the disbursement of transportation
funds,5 the State Board inevitably knew and understood the significance
of this discriminatory act.

Aside from the acts of purposeful segregation committed by the State
Legislature and the State Board of Education, the District Court also
concluded that the State was responsible for the many intentional acts
of segregation committed by the Detroit Board of Education, an agency of
the State. The majority is only willing to accept this finding arguendo.
See I have no doubt, however, as to its validity under the Fourteenth
Amendment.

`The command of the Fourteenth Amendment,' it should be recalled, `is
that no 'State' shall deny to any person within its jurisdiction the
equal protection of the laws.' Cooper v. Aaron, While a State can act
only through `the officers or agents by whom its powers are exerted,' Ex
parte Virginia, actions by an agent or officer of the State are
encompassed by the Fourteenth Amendment for, `as he acts in the name and
for the State, and is clothed with the State's power, his act is that of
the State.' See also Cooper v. Aaron; Virginia v. Rives, ; Shelley v.
Kraemer, Under Michigan law a `school district is an agency of the City
of State government.' School District of Lansing v. State Board of
Education, 367 Mich. 591, 600, 116 N.W d 866, 870 (1962). It is `a legal
division of territory, created by the State for educational purposes, to
which the State has granted such powers as are deemed necessary to
permit the district to function as a State agency.' Detroit Board of
Education v. Superintendent of Public Instruction, 319 Mich. 436, 450,
29 N.W d 902, 908 (1947). Racial discrimination by the school district,
an agency of the State, is therefore racial discrimination by the State
itself, forbidden by the Fourteenth Amendment.

We recognized only last Term in Keyes that it was the State itself which
was ultimately responsible for de jure acts of segregation committed by
a local school board. A deliberate policy of segregation by the local
board, we held, amounted to `state-imposed segregation.' 413 U.S.,
Wherever a dual school system exists, whether compelled by state statute
or created by a local board's systematic program of segregation, `the
State automatically assumes an affirmative duty 'to effectuate a
transition to a racially nondiscriminatory school system' (and) to
eliminate from the public schools within their school system 'all
vestiges of state-imposed segregation."

Vesting responsibility with the State of Michigan for Detroit's
segregated schools is particularly appropriate as

Michigan, unlike some other States, operates a single statewide system
of education rather than several separate and independent local school
systems. The majority's emphasis on local governmental control and local
autonomy of school districts in Michigan will come as a surprise to
those with any familiarity with that State's system of education. School
districts are not separate and distinct sovereign entities under
Michigan law, but rather are ``auxiliaries of the State,'' subject to
its `absolute power.' Attorney General of Michigan ex rel. Kies v.
Lowrey, The courts of the State have repeatedly emphasized that
education in Michigan is not a local governmental concern, but a state
function.

`Unlike the delegation of other powers by the legislature to local
governments, education is not inherently a part of the local
self-government of a municipality . .. Control of our public school
system is a State matter delegated and lodged in the State legislature
by the Constitution. The policy of the State has been to retain control
of its school system, to be administered throughout the State under
State laws by local State agencies organized with plenary powers to
carry out the delegated functions given (them) by the legislature.'
School District of the City of Lansing v. State Board of Education, 116
N.W d.

The Supreme Court of Michigan has noted the deep roots of this policy:

`It has been settled by the Ordinance of 1787, the several Constitutions
adopted in this state, by its uniform course of legislation, and by the
decisions of this court, that education in Michigan is a matter of state
concern, that it is no part of the local self-government of a particular
township or munic- ipality . .. The legislature has always dictated the
educational policy of the state.' In re School District No.~6, 284 Mich.
132, 145N.W. 792, 797 (1938).

The State's control over education is reflected in the fact that,
contrary to the Court's implication, there is little or no relationship
between school districts and local political units. To take the 85
outlying local school districts in the Detroit metropolitan area as
examples, 17 districts lie in two counties, two in three counties. One
district serves five municipalities; other suburban municipalities are
fragmented into as many as six school districts. Nor is there any
apparent state policy with regard to the size of school districts, as
they now range from 2,000 to 285,000 students.

Centralized state control manifests itself in practice as well as in
theory. The State controls the financing of education in several ways.
The legislature contributes a substantial portion of most school
districts' operating budgets with funds appropriated from the State's
General Fund revenues raised through statewide taxation The State's
power over the purse can be and is in fact used to enforce the State's
powers over local districts In addition, although local districts obtain
funds through local property taxation, the State has assumed the
responsibility to ensure equalized property valuations throughout the
State The State also establishes standards for teacher certification and
teacher tenure determines part of the required curriculum; sets the
minimum school term approves bus routes, equipment, and drivers approves
textbooks and establishes procedures for student discipline The State
Superintendent of Public Instruction and the State Board of Education
have the power to remove local school board members from office for
neglect of their duties.

Most significantly for present purposes, the State has wide-ranging
powers to consolidate and merge school districts, even without the
consent of the districts themselves or of the local citizenry See, e.g.,
Attorney General ex rel. Kies, v. Lowrey, 131 Mich. 639, 92 N.W. 289
(1902), aff'dIndeed, recent years have witnessed an accelerated program
of school district consolidations, mergers, and annexations, many of
which were state imposed. Whereas the State had 7,362 local districts in
1912, the number had been reduced to 1,438 in 1964 and to 738 in 1968 By
June 1972, only 608 school districts remained. Furthermore, the State
has broad powers to transfer property from one district to another,
again without the consent of the local school districts affected by the
transfer.

Whatever may be the history of public education in other parts of our
Nation, it simply files in the face of reality to say, as does the
majority, that in Michigan, `(n)o single tradition in public education
is more deeply rooted than local control over the operation of schools .
..' Ante, as 741. As the State's Supreme Court has said: `We have
repeatedly held that education in this state is not a matter of local
concern, but belongs to the state at large.' Indeed, a study prepared
for the 1961 Michigan Constitutional Convention noted that the Michigan
Constitution's articles on education had resulted in `the establishment
of a state system of education in contrast to a series of local school
systems.'

In sum, several factors in this case coalesce to support the District
Court's ruling that it was the State of Michigan itself, not simply the
Detroit Board of Education, which bore the obligation of curing the
condition of segregation within the Detroit city schools. The actions of
the State itself directly contributed to Detroit's segregation. Under
the Fourteenth Amendment, the State is ultimately responsible for the
actions of its local agencies. And, finally, given the structure of
Michigan's educational system, Detroit's segregation cannot be viewed as
the problem of an independent and separate entity. Michigan operates a
single statewide system of education, a substantial part of which was
shown to be segregated in this case.

What action, then, could the District Court require the State to take in
order to cure Detroit's condition of segregation? Our prior cases have
not minced words as to what steps responsible officials and agencies
must take in order to remedy segregation in the public schools. Not only
must distinctions on the basis of race be terminated for the future, but
school officials are also `clearly charged with the affirmative duty to
take whatever steps might be necessary to convert to a unitary system in
which racial discrimination would be eliminated root and branch.' Green
v. County School Board of New Kent County---438, See also Lee v. Macon
County Board of Education, 267 F.Supp. 458 (MD Ala.), aff'd sub nom.
Wallace v. United StatesNegro students are not only entitled to neutral
nondiscriminatory treatment in the future. They must receive `what Brown
promised them: a school system in which all vestiges of enforced racial
segregation have been eliminated.' Wright v. Council of the City of
Emporia, See also Swann v. Charlotte-Mecklenburg Board of Education,
These remedial standards are fully applicable not only to school
districts where a dual system was compelled by statute, but also where,
as here, a dual system was the product of purposeful and intentional
state action.

After examining three plans limited to the city of Detroit, the District
Court correctly concluded that none would eliminate root and branch the
vestiges of unconstitutional segregation. The plans' effectiveness, of
course, had to be evaluated in the context of the District Court's
findings as to the extent of segregation in the Detroit city schools. As
indicated earlier, the most essential finding was that Negro children in
Detroit had been confined by intentional acts of segregation to a
growing core of Negro schools surrounded by a receding ring of white
schools Thus, in 1960, of Detroit's 251 regular attendance schools, 100
were 90\% or more white and 71 were 90\% or more Negro. In 1970, of
Detroit's 282 regular attendance schools, 69 were 90\% or more white and
133 were 90\% or more Negro. While in 1960, 68\% of all schools were
90\% or more one race, by 1970, 71 \% of the schools fell into that
category. The growing core of all-Negro schools was further evidenced in
total school district population figures. In 1960 the Detroit system had
46\% Negro students and 54\% white students, but by 1970, 64\% of the
students were Negro and only 36\% were white. This increase in the
proportion of Negro students was the highest of any major Northern city.

It was with these figures in the background that the District Court
evaluated the adequacy of the three Detroit-only plans submitted by the
parties. Plan A, proposed by the Detroit Board of Education,
desegregated the high schools and about a fifth of the middle-level
schools. It was deemed inadequate, however, because it did not
desegregate elementary schools and left the middle-level schools not
included in the plan more segregated than ever. Plan C, also proposed by
the Detroit Board, was deemed inadequate because it too covered only
some grade levels and would leave elementary schools segregated. Plan B,
the plaintiffs' plan, though requiring the transportation of 82,000
pupils and the acquisition of 900 school buses, would make little
headway in rooting out the vestiges of segregation. To begin with,
because of practical limitations, the District Court found that the plan
would leave many of the Detroit city schools 75 to 90\% Negro. More
significantly, the District Court recognized that in the context of a
community which historically had a school system marked by rigid de jure
segregation, the likely effect of a Detroit-only plan would be to
`change a school system which is now Black and White to one that would
be perceived as Black . ..' The result of this changed perception, the
District Court found, would be to increase the flight of whites from the
city to the outlying suburbs, compounding the effects of the present
rate of increase in the proportion of Negro students in the Detroit
system. Thus, even if a plan were adopted which, at its outset, provided
in every school a 65\% Negro-35\% white racial mix in keeping with the
Negro-white proportions of the total student population, such a system
would, in short order, devolve into an all-Negro system. The net result
would be a continuation of the all-Negro schools which were the
hallmarks of Detroit's former dual system of one-race schools.

Under our decisions, it was clearly proper for the District Court to
take into account the so-called `white flight' from the city schools
which would be forthcoming from any Detroit-only decree. The court's
prediction of white flight was well supported by expert testimony based
on past experience in other cities undergoing desegregation relief. We
ourselves took the possibility of white flight into account in
evaluating the effectiveness of a desegregation plan in Wrightwhere we
relied on the District Court's finding that if the city of Emporia were
allowed to withdraw from the existing system, leaving a system with a
higher proportion of Negroes, it `may be anticipated that the pro-
portion of whites in county schools may drop as those who can register
in private academies' . ..' 407 U.S., One cannot ignore the white-flight
problem, for where legally imposed segregation has been established, the
District Court has the responsibility to see to it not only that the
dual system is terminated at once but also that future events do not
serve to perpetuate or re-establish segregation.

We held in Swannthat where de jure segregation is shown, school
authorities must make `every effort to achieve the greatest possible
degree of actual desegregation.' 402 U.S., This is the operative
standard re-emphasized in Davis v. School Comm'rs of Mobile County, If
these words have any meaning at all, surely it is that school
authorities must, to the extent possible, take all practicable steps to
ensure that Negro and white children in fact go to school together. This
is, in the final analysis, what desegregation of the public schools is
all about.

Because of the already high and rapidly increasing percentage of Negro
students in the Detroit system, as well as the prospect of white flight,
a Detroit-only plan simply has no hope of achieving actual
desegregation. Under such a plan white and Negro students will not go to
school together. Instead, Negro children will continue to attend
all-Negro schools. The very evil that Brown I was aimed at will not be
cured, but will be perpetuated for the future.

Racially identifiable schools are one of the primary vestiges of
state-imposed segregation which an effective desegregation decree must
attempt to eliminate. In Swannfor example, we held that `(t)he district
judge or school authorities . will thus necessarily be concerned with
the elimination of one-race schools.' There is `a presumption,' we
stated, `against schools that are substantially disproportionate in
their racial composition.' And in evaluating the effectiveness of
desegregation plans in prior cases, we ourselves have considered the
extent to which they discontinued racially identifiable schools. See,
e.g., Green v. County School Board of New Kent County; Wright v. Council
of the City of Emporia. For a principal end of any desegregation remedy
is to ensure that it is no longer `possible to identify a 'white school'
or a `Negro school." Swann U.S., 18, The evil to be remedied in the
dismantling of a dual system is the'(r) acial identification of the
system's schools.' GreenThe goal is a system without white schools or
Negro schools---a system with `just schools.' A school authority's
remedial plan or a district court's remedial decree is to be judged by
its effectiveness in achieving this end.

We cautioned in Swann, of course, that the dismantling of a segregated
school system does not mandate any particular racial balance. 402 U.S.,
We also concluded that a remedy under which there would remain a small
number of racially identifiable schools was only presumptively
inadequate and might be justified. But this is a totally different case.
The flaw of a Detroit-only decree is not that it does not reach some
ideal degree of racial balance or mixing. It simply does not promise to
achieve actual desegregation at all. It is one thing to have a system
where a small number of students remain in racially identifiable
schools. It is something else entirely to have a system where all
students continue to attend such schools.

The continued racial identifiability of the Detroit schools under a
Detroit-only remedy is not simply a reflection of their high percentage
of Negro students.

What is or is not a racially identifiable vestige of de jure segregation
must necessarily depend on several factors. Cf. Keyes, Foremost among
these should be the relationship between the schools in question and the
neighboring community. For these purposes the city of Detroit and its
surrounding suburbs must be viewed as a single community. Detroit is
closely connected to its suburbs in many ways, and the metropolitan area
is viewed as a single cohesive unit by its residents. About 40\% of the
residents of the two suburban counties included in the desegregation
plan work in Wayne County, in which Detroit is situated. Many residents
of the city work in the suburbs. The three counties participate in a
wide variety of cooperative governmental ventures on a metropolitan-wide
basis, including a metropolitan transit system, park authority, water
and sewer system, and council of governments. The Federal Government has
classified the tri-county area as a Standard Metropolitan Statistical
Area, indicating that it is an area of `economic and social
integration.' United States v. Connecticut National Bank,

Under a Detroit-only decree, Detroit's schools will clearly remain
racially identifiable in comparison with neighboring schools in the
metropolitan community. Schools with 65\% and more Negro students will
stand in sharp and obvious contrast to schools in neighboring districts
with less than 2\% Negro enrollment. Negro students will continue to
perceive their schools as segregated educational facilities and this
perception will only be increased when whites react to a Detroit-only
decree by fleeing to the suburbs to avoid integration. School district
lines, however innocently drawn, will surely be perceived as fences to
separate the races when, under a Detroit-only decree, white parents
withdraw their chil- dren from the Detroit city schools and move to the
suburbs in order to continue them in all-white schools. The message of
this action will not escape the Negro children in the city of Detroit.
See Wright, It will be of scant significance to Negro children who have
for years been confined by de jure acts of segregation to a growing core
of all-Negro schools surrounded by a ring of all-white schools that the
new dividing line between the races is the school district boundary.

Nor can it be said that the State is free from any responsibility for
the disparity between the racial makeup of Detroit and its surrounding
suburbs. The State's creation, through de jure acts of segregation, of a
growing core of all-Negro schools inevitably acted as a magnet to
attract Negroes to the areas served by such schools and to deter them
from settling either in other areas of the city or in the suburbs. By
the same token, the growing core of all-Negro schools inevitably helped
drive whites to other areas of the city or to the suburbs. As we
recognized in Swann:

`People gravitate toward school facilities, just as schools are located
in response to the needs of people. The location of schools may thus
influence the patterns of residential development of a metropolitan area
and have important impact on composition of inner-city neighborhoods. .
(Action taken) to maintain the separation of the races with a minimum
departure from the formal principles of 'neighborhood zoning' . does
more than simply influence the short-run composition of the student body
. .. It may well promote segregated residential patterns which, when
combined with `neighborhood zoning,' further lock the school system into
the mold of separation of the races. Upon a proper showing a district
court may consider this in fashioning a remedy.' The rippling effects on
residential patterns caused by purposeful acts of segregation do not
automatically subside at the school district border. With rare
exceptions, these effects naturally spread through all the residential
neighborhoods within a metropolitan area.

The State must also bear part of the blame for the white flight to the
suburbs which would be forthcoming from a Detroit-only decree and would
render such a remedy ineffective. Having created a system where whites
and Negroes were intentionally kept apart so that they could not become
accustomed to learning together, the State is responsible for the fact
that many whites will react to the dismantling of that segregated system
by attempting to flee to the suburbs. Indeed, by limiting the District
Court to a Detroit-only remedy and allowing that flight to the suburbs
to succeed, the Court today allows the State to profit from its own
wrong and to perpetuate for years to come the separation of the races it
achieved in the past by purposeful state action.

The majority asserts, however, that involvement of outlying districts
would do violence to the accepted principle that `the nature of the
violation determines the scope of the remedy.' SwannSee Not only is the
majority's attempt to find in this single phrase the answer to the
complex and difficult questions presented in this case hopelessly
simplistic, but more important, the Court reads these words in a manner
which perverts their obvious meaning. The nature of a violation
determines the scope of the remedy simply because the function of any
remedy is to cure the violation to which it is addressed. In school
segregation cases, as in other equitable causes, a remedy which
effectively cures the violation is what is required. See Green, 88
S.Ct.; Davis, No more is necessary, but we can tolerate no less. To read
this principle as barring a district court from imposing the only
effective remedy for past segregation and remitting the court to a
patently ineffective alternative is, in my view, to turn a simple
commonsense rule into a cruel and meaningless paradox. Ironically, by
ruling out an interdistrict remedy, the only relief which promises to
cure segregation in the Detroit public schools, the majority flouts the
very principle on which it purports to rely.

Nor should it be of any significance that the suburban school districts
were not shown to have themselves taken any direct action to promote
segregation of the races. Given the State's broad powers over local
school districts, it was well within the State's powers to require those
districts surrounding the Detroit school district to participate in a
metropolitan remedy. The State's duty should be no different here than
in cases where it is shown that certain of a State's voting districts
are malapportioned in violation of the Fourteenth Amendment.
Overrepresented electoral districts are required to participate in
reapportionment although their only `participation' in the violation was
to do nothing about it. Similarly, electoral districts which themselves
meet representation standards must frequently be redrawn as part of a
remedy for other over-and under-inclusive districts. No finding of fault
on the part of each electoral district and no finding of a
discriminatory effect on each district is a prerequisite to its
involvement in the constitutionally required remedy. By the same logic,
no finding of fault on the part of the suburban school districts in this
case and no finding of a discriminatory effect on each district should
be a prerequisite to their involvement in the constitutionally required
remedy.

It is the State, after all, which bears the responsibility under Brown
of affording a nondiscriminatory system of education. The State, of
course, is ordinarily free to choose any decentralized framework for
education it wishes, so long as it fulfills that Fourteenth Amendment
obligation. But the State should no more be allowed to hide behind its
delegation and compartmentalization of school districts to avoid its
constitutional obligations to its children than it could hide behind its
political subdivisions to avoid its obligations to its voters.

It is a hollow remedy indeed where `after supposed 'desegregation' the
schools remained segregated in fact.' We must do better than
``substitute . one segregated school system for another segregated
school system.'' Wright, To suggest, as does the majority, that a
Detroitonly plan somehow remedies the effects of de jure segregation of
the races is, in my view, to make a solemn mockery of Brown I's holding
that separate educational facilities are inherently unequal and of
Swann's unequivocal mandate that the answer to de jure segregation is
the greatest possible degree of actual desegregation.

One final set of problems remains to be considered. We recognized in
Brown II, and have re-emphasized ever since, that in fashioning relief
in desegregation cases, `the courts will be guided by equitable
principles. Traditionally, equity has been characterized by a practical
flexibility in shaping its remedies and by a facility for adjusting and
reconciling public and private needs.' Brown II, See also Swann.

Though not resting its holding on this point, the majority suggests that
various equitable considerations militate against interdistrict relief.
The Court, for example, refers to financing and administrative problems,
the logistical problems attending large-scale transportation of
students, and the prospect of the District Court's becoming a `de facto
'legislative authority" and "school superintendent' for the entire
area.' The entangling web of problems woven by the Court, however,
appears on further consideration to be constructed of the flimsiest of
threads.

I deal first with the last of the problems posed by the Court the
specter of the District Court qua `school superintendent' and
`legislative authority'---for analysis of this problem helps put the
other issues in proper perspective. Our cases, of course, make clear
that the initial responsibility for devising an adequate desegregation
plan belongs with school authorities, not with the District Court. The
court's primary role is to review the adequacy of the school
authorities' efforts and to substitute its own plan only if and to the
extent they default. See Swann, 91 S.Ct.; Green, Contrary to the
majority's suggestions, the District Judge in this case consistently
adhered to these procedures and there is every indication that he would
have continued to do so. After finding de jure segregation the court
ordered the parties to submit proposed Detroit-only plans. The state
defendants were also ordered to submit a proposed metropolitan plan
extending beyond Detroit's boundaries. As the District Court stated,
`the State defendants . bear the initial burden of coming forward with a
proposal that promises to work.' The state defendants defaulted in this
obligation, however.

Rather than submit a complete plan, the State Board of Education
submitted six proposals, none of which was in fact a desegregation plan.
It was only upon this default that the District Court began to take
steps to develop its own plan. Even then the District Court maximized
school authority participation by appointing a panel representing both
plaintiffs and defendants to develop a plan. Pet.App. 99a---100a.
Furthermore, the District Court still left the state defendants the
initial responsibility for developing both interim and final financial
and administrative arrangements to implement interdistrict relief.
A---105a. The Court of Appeals further protected the interests of local
school authorities by ensuring that the outlying suburban districts
could fully participate in the proceedings to develop a metropolitan
remedy.

These processes have not been allowed to run their course. No final
desegregation plan has been proposed by the panel of experts, let alone
approved by the District Court. We do not know in any detail how many
students will be transported to effect a metropolitan remedy, and we do
not know how long or how far they will have to travel. No
recommendations have yet been submitted by the state defendants on
financial and administrative arrangements. In sum, the practicality of a
final metropolitan plan is simply not before us at the present time.
Since the State and the panel of experts have not yet had an opportunity
a come up with a workable remedy, there is no foundation for the
majority's suggestion of the impracticality of interdistrict relief.
Furthermore, there is no basis whatever for assuming that the District
Court will inevitably be forced to assume the role of legislature or
school superintendent.

Were we to hold that it was its constitutional duty to do so, there is
every indication that the State of Michigan would fulfill its obligation
and develop a plan which is workable, administrable, financially sound,
and, most important, in the best interest of quality education for all
of the children in the Detroit metropolitan area.

Since the Court chooses, however, to speculate on the feasibility of a
metropolitan plan, I feel constrained to comment on the problem areas it
has targeted. To begin with, the majority's question concerning the
practicality of consolidation of school districts need not give us
pause. The State clearly has the power, under existing law, to effect a
consolidation if it is ultimately determined that this offers the best
prospect for a workable and stable desegregation plan. See---797. And
given the 1,000 or so consolidations of school districts which have
taken place in the past, it is hard to believe that the State has not
already devised means of solving most, if not all, of the practical
problems which the Court suggests consolidation would entail.

Furthermore, the majority ignores long-established Michigan procedures
under which school districts may enter into contractual agreements to
educate their pupils in other districts using state or local funds to
finance nonresident education Such agreements could form an easily
administrable framework for interdistrict relief short of outright
consolidation of the school districts. The District Court found that
interdistrict procedures like these were frequently used to provide
special educational services for handicapped children, and extensive
statutory provision is also made for their use in vocational education
Surely if school districts are willing to engage in interdistrict
programs to help those unfortunate children crippled by physical or
mental handicaps, school districts can be required to participate in an
inter-district program to help those children in the city of Detroit
whose educations and very futures have been crippled by purposeful state
segregation.

Although the majority gives this last matter only fleeting reference, it
is plain that one of the basic emotional and legal issues underlying
these cases concerns the propriety of transportation of students to
achieve desegregation. While others may have retreated from its
standards, see, e.g., Keyes, 93 S.Ct. (Powell, J., concurring in part
and dissenting in part), I continue to adhere to the guidelines set
forth in Swann on this issue. See 402 U.S.S.Ct.---1283. And though no
final desegregation plan is presently before us, to the extent the
outline of such a plan is now visible, it is clear that the
transportation it would entail will be fully consistent with these
guidelines.

First of all, the metropolitan plan would not involve the busing of
substantially more students than already ridebuses. The District Court
found that, statewide, 35\%---40\% of all students already arrive at
school on a bus. In those school districts in the tri-county Detroit
metropolitan area eligible for state reimbursement of transportation
costs, 42\%---52\% of all students rode buses to school. In the
tri-county areas as a whole, ap- proximately 300,000 pupils arrived at
school on some type of bus, with about 60,000 of these apparently using
regular public transit. In comparison, the desegregation plan, according
to its present rough outline, would involve the transportation of
310,000 students, about 40\% of the population within the desegregation
area.

With respect to distance and amount of time traveled, 17 of the outlying
school districts involved in the plan are contiguous to the Detroit
district. The rest are all within 8 miles of the Detroit city limits.
The trial court, in defining the desegregation area, placed a ceiling of
40 minutes one way on the amount of travel time, and many students will
obviously travel for far shorter periods. As to distance, the average
statewide bus trip is 8 1/2 miles one way, and in some parts of the
tri-county area, students already travel for one and a quarter hours or
more each way. In sum, with regard to both the number of students
transported and the time and distances involved, the outlined
desegregation plan `compares favorably with the transportation plan
previously operated . ..'

As far as economics are concerned, a metropolitan remedy would actually
be more sensible than a Detroit-only remedy. Because of prior
transportation aid restrictions, see, Detroit largely relied on public
transport, at student expense, for those students who lived too far away
to walk to school. Since no inventory of school buses existed, a
Detroit-only plan was estimated to require the purchase of 900 buses to
effectuate the necessary transportation. The tri-county area, in
contrast, already has an inventory of 1,800 buses, many of which are now
under-utilized. Since increased utilization of the existing inventory
can take up much of the increase in transportation involved in the
interdistrict remedy, the District Court found that only 350 additional
buses would probably be needed, almost two-thirds fewer than a
Detroit-only remedy. Other features of an interdistrict remedy bespeak
its practicality, such as the possibility of pairing up Negro schools
near Detroit's boundary with nearby white schools on the other side of
the present school district line.

Some disruption, of course, is the inevitable product of any
desegregation decree, whether it operates within one district or on an
interdistrict basis. As we said in Swann, however:

`Absent a constitutional violation there would be no basis for
judicially ordering assignment of students on a racial basis. All things
being equal, with no history of discrimination, it might well be
desirable to assign pupils to schools nearest their homes. But all
things are not equal in a system that has been deliberately constructed
and maintained to enforce racial segregation. The remedy for such
segregation may be administratively awkward, inconvenient, and even
bizarre in some situations and may impose burdens on some; but all
awkwardness and inconvenience cannot be avoided . ..'

Desegregation is not and was never expected to be an easy task. Racial
attitudes ingrained in our Nation's childhood and adolescence are not
quickly thrown aside in its middle years. But just as the inconvenience
of some cannot be allowed to stand in the way of the rights of others,
so public opposition, no matter how strident, cannot be permitted to
divert this Court from the enforcement of the constitutional principles
at issue in this case. Today's holding, I fear, is more a reflection of
a perceived public mood that we have gone far enough in enforcing the
Constitution's guarantee of equal justice than it is the product of
neutral principles of law. In the short run, it may seem to be the
easier course to allow our great metropolitan areas to be divided up
each into two cities---one white, the other black---but it is a course,
I predict, our people will ultimately regret. I dissent.

\emph{From the footnotes}:

\begin{enumerate}
\def\labelenumi{\arabic{enumi}.}
\setcounter{enumi}{18}
\tightlist
\item
  Despite Mr.~Justice STEWART's claim to the contrary, n.~2, of his
  concurring opinion, the record fully supports my statement that Negro
  students were intentionally confined to a core of Negro schools within
  the city of Detroit. See, e.g.. Indeed, Mr.~Justice STEWART
  acknowledges that intentional acts of segregation by the State have
  separated white and Negro students within the city, and that the
  resulting core of all-Negro schools has grown to encompass most of the
  city. In suggesting that my approval of an interdistrict remedy rests
  on a further conclusion that the State or its political subdivisions
  have been responsible for the increasing percentage of Negro students
  in Detroit, my Brother STEWART misconceives the thrust of this
  dissent. In light of the high concentration of Negro students in
  Detroit, the District Judge's finding that a Detroit-only remedy
  cannot effectively cure the constitutional violation within the city
  should be enough to support the choice of an interdistrict remedy.
  Whether state action is responsible for the growth of the core of
  all-Negro schools in Detroit is, in my view, quite irrelevant.
\end{enumerate}

The difficulty with Mr.~Justice STEWART's position is that he, like the
Court, confuses the inquiry required to determine whether there has been
a substantive constitutional violation with that necessary to formulate
an appropriate remedy once a constitutional violation has been shown.
While a finding of state action is of course a prerequisite to finding a
violation, we have never held that after unconstitutional state action
has been shown, the District Court at the remedial stage must engage in
a second inquiry to determine whether additional state action exists to
justify a particular remedy. Rather, once a constitutional violation has
been shown, the District Court is duty-bound to formulate an effective
remedy and, in so doing, the court is entitled---indeed, it is
required---to consider all the factual circumstances relevant to the
framing of an effective decree. Thus, in Swann v. Charlotte-Mecklenburg
Board of Education we held that the District Court must take into
account the existence of extensive residential segregation in
determining whether a racially neutral `neighborhood school' attendance
plan was an adequate desegregation remedy, regardless of whether this
residential segregation was caused by state action. So here, the
District Court was required to consider the facts that the Detroit
school system was already predominantly Negro and would likely become
all-Negro upon issuance of a Detroit-only decree in framing an effective
desegregation remedy, regardless of state responsibility for this
situation.

\hypertarget{san-antonio-independent-school-district-v.-rodriguez}{%
\subsubsection{San Antonio Independent School District v.
Rodriguez}\label{san-antonio-independent-school-district-v.-rodriguez}}

411 U.S. 1 (1973)

\textbf{Mr.~Justice POWELL delivered the opinion of the Court.}

This suit attacking the Texas system of financing public education was
initiated by Mexican-American parents whose children attend the
elementary and sec- ondary schools in the Edgewood Independent School
District, an urban school district in San Antonio, Texas They brought a
class action on behalf of schoolchildren throughout the State who are
members of minority groups or who are poor and reside in school
districts having a low property tax base. Named as defendants2 were the
State Board of Education, the Commissioner of Education, the State
Attorney General, and the Bexar County (San Antonio) Board of Trustees.
The com- plaint was filed in the summer of 1968 and a three-judge court
was impaneled in January 1969. In December 1971 the panel rendered its
judgment in a per curiam opinion holding the Texas school finance system
unconstitutional under the Equal Protection Clause of the Fourteenth
Amendment The State appealed, and we noted probable jurisdiction to
consider the far-reaching constitutional questions presented. 406 U.S.
966, For the reasons stated in this opinion, we reverse the decision of
the District Court.

The first Texas State Constitution, promulgated upon Texas' entry into
the Union in 1845, provided for the establishment of a system of free
schools. 6 Early in its history, Texas adopted a dual approach to the
financing of its schools, relying on mutual participation by the local
school districts and the State. As early as 1883, the state constitution
was amended to provide for the creation of local school districts
empowered to levy ad valorem taxes with the consent of local taxpayers
for the `erection . of school buildings' and for the `further
maintenance of public free schools.' Such local funds as were raised
were supplemented by funds distributed to each district from the State's
Permanent and Available School Funds The Permanent School Fund, its
predecessor established in 1854 with \$2,000,000 realized from an
annexation settlement,9 was thereafter endowed with millions of acres of
public land set aside to assure a continued source of income for school
support The Available School Fund, which received income from the
Permanent School Fund as well as from a state ad valorem property tax
and other designated taxes,11 served as the disbursing arm for most
state educational funds throughout the late 1800's and first half of
this century. Additionally, in 1918 an increase in state property taxes
was used to finance a program providing free textbooks throughout the
State.

Until recent times, Texas was a predominantly rural State and its
population and property wealth were spread relatively evenly across the
State Sizable differences in the value of assessable property between
local school districts became increasingly evident as the State became
more industrialized and as rural-to-urban population shifts became more
pronounced The location of commercial and industrial property began to
play a significant role in determining the amount of tax resources
available to each school district. These growing disparities in
population and taxable property between districts were responsible in
part for increasingly notable differences in levels of local expenditure
for education.

In due time it became apparent to those concerned with financing public
education that contributions from the Available School Fund were not
sufficient to ameliorate these disparities Prior to 1939, the Available
School Fund contributed money to every school district at a rate of \$17
per school-age child Although the amount was increased several times in
the early 1940's,18 the Fund was providing only \$46 per student by
1945.

Recognizing the need for increased state funding to help offset
disparities in local spending and to meet Texas' changing educational
requirements, the state legislature in the late 1940's undertook a
thorough evaluation of public education with an eye toward major reform.
In 1947, an 18-member committee, composed of educators and legislators,
was appointed to explore alternative systems in other States and to
propose a funding scheme that would guarantee a minimum or basic
educational offering to each child and that would help overcome
interdistrict disparities in taxable resources. The Committee's efforts
led to the passage of the Gilmer-Aikin bills, named for the Committee's
co-chairmen, establishing the Texas Minimum Foundation School Program20.
Today, this Program accounts for approximately half of the total
educational expenditures in Texas.

The Program calls for state and local contributions to a fund earmarked
specifically for teacher salaries, operating expenses, and
transportation costs. The State, supplying funds from its general
revenues, finances approximately 80\% of the Program, and the school
districts are responsible---as a unit---for providing the remaining
20\%. The districts' share, known as the Local Fund Assignment, is
apportioned among the school districts under a formula designed to
reflect each district's relative taxpaying ability. The Assignment is
first divided among Texas' 254 counties pursuant to a complicated
economic index that takes into account the relative value of each
county's contribution to the State's total income from manufacturing,
mining, and agricultural activities. It also considers each county's
relative share of all payrolls paid within the State and, to a lesser
extent, considers each county's share of all property in the State Each
county's assignment is then divided among its school districts on the
basis of each district's share of assessable property within the county
The district, in turn, finances its share of the Assignment out of
revenues from local property taxation.

The design of this complex system was twofold. First, it was an attempt
to assure that the Foundation Program would have an equalizing influence
on expenditure levels between school districts by placing the heaviest
burden on the school districts most capable of paying. Second, the
Program's architects sought to establish a Local Fund Assignment that
would force every school district to contribute to the education of its
children24 but that would not by itself exhaust any district's resources
Today every school district does impose a property tax from which it
derives locally expendable funds in excess of the amount necessary to
satisfy its Local Fund Assignment under the Foundation Program.

In the years since this program went into operation in 1949,
expenditures for education---from state as well as local sources have
increased steadily. Between 1949 and 1967, expenditures increased
approximately 500\% In the last decade alone the total public school
budget rose from \$750 million to.\$2 billion27 and these increases have
been reflected in consistently rising perpupil expenditures throughout
the State Teacher salaries, by far the largest item in any school's
budget, have increased dramatically---the state-supported minimum salary
for teachers possessing college degrees has risen from \$2,400 to
\$6,000 over the last 20 years.

The school district in which appellees reside, the Edgewood Independent
School District, has been compared throughout this litigation with the
Alamo Heights Independent School District. This comparison between the
least and most affluent districts in the San Antonio area serves to
illustrate the manner in which the dual system of finance operates and
to indicate the extent to which substantial disparities exist despite
the State's impressive progress in recent years. Edgewood is one of
seven public school districts in the metropolitan are enrolled in its 25
elementary and secondary schools. The district is are enrolled in its 25
elementary situated in the core-city sector of San Antonio in a
residential neighborhood that has little commercial or industrial
property. The residents are predominantly of Mexican-American descent:
approximately 90\% of the student population is Mexican-American and
over 6\% is Negro. The average assessed property value per pupil is
\$5,960---the lowest in the metropolitan area---and the median family
income (\$4,686) is also the lowest At an equalized tax rate of \$1 per
\$100 of assessed property the highest in the metropolitan area---the
district contributed \$26 to the education of each child for the
1967---1968 school year above its Local Fund Assignment for the Minimum
Foundation Program. The Foundation Program contributed \$222 per pupil
for a state-local total of \$248 Federal funds added another \$108 for a
total of \$356 per pupil.

Alamo Heights is the most affluent school district in San Antonio. Its
six schools, housing approximately 5,000 students, are situated in a
residential community quite unlike the Edgewood District. The school
population is predominantly `Anglo,' having only 18\% Mexican-Amer-
icans and less than 1\% Negroes. The assessed property value per pupil
exceeds \$49,000,33 and the median family income is \$8,001. In
1967---1968 the local tax rate of \$ per \$100 of valuation yielded
\$333 per pupil over and above its contribution to the Foundation
Program. Coupled with the \$225 provided from that Program, the district
was able to supply \$558 per student. Supplemented by a \$36 per-pupil
grant from federal sources, Alamo Heights spent \$594 per pupil.

Although the 1967---1968 school year figures provide the only complete
statistical breakdown for each category of aid,34 more recent partial
statistics indicate that the previously noted trend of increasing state
aid has been significant. For the 1970---1971 school year, the
Foundation School Program allotment for Edgewood was \$356 per pupil, a
62\% increase over the 1967---68 school year. Indeed, state aid alone in
1970---1971 equaled Edgewood's entire 1967---1968 school budget from
local, state, and federal sources. Alamo Heights enjoyed a similar
increase under the Foundation Program, netting \$491 per pupil in
1970---1971 These recent

figures also reveal the extent to which these two districts' allotments
were funded from their own required contributions to the Local Fund
Assignment. Alamo Heights, because of its relative wealth, was required
to contribute out of its local property tax collections approximately
\$100 per pupil, or about 20\% of its Foundation grant. Edgewood, on the
other hand, paid only \$8 per pupil, which is about 2 \% of its grant It
appears then that, at least as to these two districts, the Local Fund
Assignment does reflect a rough approximation of the relative taxpaying
potential of each.

Despite these recent increases, substantial interdistrict disparities in
school expenditures found by the District Court to prevail in San
Antonio and in varying degrees throughout the State38 still exist. And
it was these disparities, largely attributable to differences in the
amounts of money collected through local property taxation, that led the
District Court to conclude that Texas' dual system of public school
financing violated the Equal Protection Clause. The District Court held
that the Texas system discriminates on the basis of wealth in the manner
in which education is provided for its people. 337 F.Supp.. Finding that
wealth is a `suspect' classification and that education is a
`fundamental' interest, the District Court held that the Texas system
could be sustained only if the State could show that it was premised
upon some compelling state interest. ---284. On this issue the court
concluded that `(n)ot only are defendants unable to demonstrate
compelling state interests . they fail even to establish a reasonable
basis for these classifications.'

Texas virtually concedes that its historically rooted dual system of
financing education could not withstanding the strict judicial scrutiny
that this Court has found appropriate in reviewing legislative judgments
that interfere with fundamental constitutional rights39 or that involve
suspect classifications If, as previous decisions have indicated, strict
scrutiny means that the State's system is not entitled to the usual
presumption of validity, that the State rather than the complainants
must carry a `heavy burden of justification,' that the State must
demonstrate that its educational system has been structured with
`precision,' and is `tailored' narrowly to serve legitimate objectives
and that it has selected the `less drastic means' for effectuating its
objectives,41 the Texas financing system and its counterpart in
virtually every other State will not pass muster. The State candidly
admits that `(n)o one familiar with the Texas system would contend that
it has yet achieved perfection.'42 Apart from its concession that
educational financing in Texas has 'defects'43 and 'imperfections,'44
the State defends the system's rationality with vigor and disputes the
District Court's finding that it lacks a 'reasonable basis.'

This, then, establishes the framework for our analysis. We must decide,
first, whether the Texas system of financing public education operates
to the disadvantage of some suspect class or impinges upon a fundamental
right explicitly or implicitly protected by the Constitution, thereby
requiring strict judicial scrutiny. If so, the judgment of the District
Court should be affirmed. If not, the Texas scheme must still be
examined to determine whether it rationally furthers some legitimate,
articulated state purpose and therefore does not constitute an invidious
discrimination in violation of the Equal Protection Clause of the
Fourteenth Amendment.

The District Court's opinion does not reflect the novelty and complexity
of the constitutional questions posed by appellees' challenge to Texas'
system of school financing. In concluding that strict judicial scrutiny
was required, that court relied on decisions dealing with the rights of
indigents to equal treatment in the criminal trial and appellate
processes,45 and on cases disapproving wealth restrictions on the right
to vote Those cases, the District Court concluded, established wealth as
a suspect classification. Finding that the local property tax system
discriminated on the basis of wealth, it regarded those precedents as
controlling. It then reasoned, based on decisions of this Court
affirming the undeniable importance of education, 47 that there is a
fundamental right to education and that, absent some compelling state
justification, the Texas system could not stand.

We are unable to agree that this case, which in significant aspects is
sui generis, may be so neatly fitted into the conventional mosaic of
constitutional analysis under the Equal Protection Clause. Indeed, for
the several reasons that follow, we find neither the
suspect-classification not the fundamental-interest analysis persuasive.

\begin{enumerate}
\def\labelenumi{\Alph{enumi}.}
\item
\end{enumerate}

The wealth discrimination discovered by the District Court in this case,
and by several other courts that have recently struck down
school-financing laws in other States,48 is quite unlike any of the
forms of wealth dis- crimination heretofore reviewed by this Court.
Rather than focusing on the unique features of the alleged
discrimination, the courts in these cases have virtually assumed their
findings of a suspect classification through a simplistic process of
analysis: since, under the traditional systems of financing public
schools, some poorer people receive less expensive educations than other
more affluent people, these systems discriminate on the basis of wealth.
This approach largely ignores the hard threshold questions, including
whether it makes a difference for purposes of consideration under the
Constitution that the class of disadvantaged `poor' cannot be identified
or defined in customary equal protection terms, and whether the
relative---rather than absolute---nature of the asserted deprivation is
of significant consequence. Before a State's laws and the justifications
for the classifications they create are subjected to strict judicial
scrutiny, we think these threshold considerations must be analyzed more
closely than they were in the court below.

The case comes to us with no definitive description of the classifying
facts or delineation of the disfavored class. Examination of the
District Court's opinion and of appellees' complaint, briefs, and
contentions at oral argument suggests, however, at least three ways in
which the discrimination claimed here might be described. The Texas
system of school financing might be regarded as discriminating (1)
against `poor' persons whose incomes fall below some identifiable level
of poverty or who might be characterized as functionally 'indigent,'49
or (2) against those who are relatively poorer than others, 50 or (3)
against all those who, irrespective of their personal incomes, happen to
reside in relatively poorer school districts Our task must be to
ascertain whether, in fact, the Texas system has been shown to
discriminate on any of these possible bases and, if so, whether the
resulting classification may be regarded as suspect.

The precedents of this Court provide the proper starting point. The
individuals, or groups of individuals, who constituted the class
discriminated against in our prior cases shared two distinguishing
characteristics: because of their impecunity they were completely unable
to pay for some desired benefit, and as a consequence, they sustained an
absolute deprivation of a meaningful opportunity to enjoy that benefit.
In Griffin v. Illinois,

Likewise, in Douglas v. California, a decision establishing an indigent
defendant's right to court-appointed counsel on direct appeal, the Court
dealt only with defendants who could not pay for counsel from their own
resources and who had no other way of gaining representation. Douglas
provides no relief for those on whom the burdens of paying for a
criminal defense are relatively speaking, great but not insurmountable.
Nor does it deal with relative differences in the quality of counsel
acquired by the less wealthy.

Williams v. Illinoisand Tate v. Short struck down criminal penalties
that subjected indigents to incarceration simply be- cause of their
inability to pay a fine. Again, the disadvantaged class was composed
only of persons who were totally unable to pay the demanded sum. Those
cases do not touch on the question whether equal protection is denied to
persons with relatively less money on whom designated fines impose
heavier burdens. The Court has not held that fines must be structured to
reflect each person's ability to pay in order to avoid disproportionate
burdens. Sentencing judges may, and often do, consider the defendant's
ability to pay, but in such circumstances they are guided by sound
judicial discretion rather than by constitutional mandate.

Finally, in Bullock v. Carter the Court invalidated the Texas filing-fee
requirement for primary elections. Both of the relevant classifying
facts found in the previous cases were present there. The size of the
fee, often running into the thousands of dollars and, in at least one
case, as high as \$8,900, effectively barred all potential candidates
who were unable to pay the required fee. As the system provided `no
reasonable alternative means of access to the ballot' (id., 92 S.Ct.),
inability to pay occasioned an absolute denial of a position on the
primary ballot.

Only appellees' first possible basis for describing the class
disadvantaged by the Texas school-financing

system---discrimination against a class of definably `poor'
persons---might arguably meet the criteria established in these prior
cases. Even a cursory examination, however, demonstrates that neither of
the two distinguishing characteristics of wealth classifications can be
found here. First, in support of their charge that the system
discriminates against the `poor,' appellees have made no effort to
demonstrate that it operates to the peculiar disadvantage of any class
fairly definable as indigent, or as composed of persons whose incomes
are beneath any designated poverty level. Indeed, there is reason to
believe that the poorest families are not necessarily clustered in the
poorest property districts. A recent and exhaustive study of school
districts in Connecticut concluded that `(i)t is clearly incorrect . to
contend that the 'poor' live in `poor' districts . .. Thus, the major
factual assumption of Serrano---that the educational financing system
discriminates against the `poor'---is simply false in Connecticut.'53
Defining `poor' families as those below the Bureau of the Census
'poverty level,'54 the Connecticut study found, not surprisingly, that
the poor were clustered around commercial and industrial areas---those
same areas that provide the most attractive sources of property tax
income for school districts Whether a similar pattern would be
discovered in Texas is not known, but there is no basis on the record in
this case for assuming that the poorest people---defined by reference to
any level of absolute impecunity---are concentrated in the poorest
districts.

Second, neither appellees nor the District Court addressed the fact
that, unlike each of the foregoing cases, lack of personal resources has
not occasioned an absolute deprivation of the desired benefit. The
argument here is not that the children in districts having relatively
low assessable property values are receiving no public education;
rather, it is that they are receiving a poorer quality education than
that available to children in districts having more assessable wealth.
Apart from the unsettled and disputed question whether the quality of
education may be determined by the amount of money expended for it,56 a
sufficient answer to appellees' argument is that, at least where wealth
is involved, the Equal Protection Clause does not require absolute
equality or precisely equal advantages Nor indeed, in view of the
infinite variables affecting the educational process, can any system
assure equal quality of education except in the most relative sense.
Texas asserts that the Minimum Foundation Program provides an `adequate'
education for all children in the State. By providing 12 years of free
public-school education, and by assuring teachers, books,
transportation, and operating funds, the Texas Legislature has
endeavored to 'guarantee, for the welfare of the state as a whole, that
all people shall have at least an adequate program of education. This is
what is meant by 'A Minimum Foundation Program of Education."58 The
State repeatedly asserted in its briefs in this Court that it has
fulfilled this desire and that it now assures 'every child in every
school district an adequate education.'59 No proof was offered at trial
persuasively discrediting or refuting the State's assertion.

For these two reasons---the absence of any evidence that the financing
system discriminates against any definable category of `poor' people or
that it results in the absolute deprivation of education---the
disadvantaged class is not susceptible of identification in traditional
terms.

As suggested above, appellees and the District Court may have embraced a
second or third approach, the second of which might be characterized as
a theory of relative or comparative discrimination based on family
income. Appellees sought to prove that a direct correlation exists
between the wealth of families within each district and the expenditures
therein for education. That is, along a continuum, the poorer the family
the lower the dollar amount of education received by the family's
children.

The principal evidence adduced in support of this
comparative-discrimination claim is an affidavit submitted by Professor
Joele S. Berke of Syracuse University's Educational Finance Policy
Institute. The District Court, relying in major part upon this affidavit
and apparently accepting the substance of appellees' theory, noted,
first, a positive correlation between the wealth of school districts,
measured in terms of assessable property per pupil, and their levels of
per-pupil expenditures. Second, the court found a similar correlation
between district wealth and the personal wealth of its residents,
measured in terms of median family income. 337 F.Supp. n.~3.

If, in fact, these correlations could be sustained, then it might be
argued that expenditures on education---equated by appellees to the
quality of education---are dependent on personal wealth. Appellees'
comparative-discrimination theory would still face serious unanswered
questions, including whether a bare positive correlation or some higher
degree of correlation61 is necessary to provide a basis for concluding
that the financing system is designed to operate to the peculiar
disadvantage of the comparatively poor, and whether a class of this size
and diversity could ever claim the special protection accorded `suspect'
classes. These questions need not be addressed in this case, however,
since appellees' proof fails to support their allegations or the
District Court's conclusions.

Professor Berke's affidavit is based on a survey of approximately 10\%
of the school districts in Texas. His findings, previously set out in
the margin, 63 show only that the wealthiest few districts in the sample
have the highest median family incomes and spend the most on education,
and that the several poorest districts have the lowest family incomes
and devote the least amount of money to education. For the remainder of
the districts---96 districts composing almost 90\% of the sample the
correlation is inverted, i.e., the districts that spend next to the most
money on education are populated by families having next to the lowest
median family incomes while the districts spending the least have the
highest median family incomes. It is evident that, even if the
conceptual questions were answered favorably to appellees, no factual
basis exists upon which to found a claim of comparative wealth
discrimination.

This brings us, then, to the third way in which the classification
scheme might be defined---district wealth discrimination. Since the only
correlation indicated by the evidence is between district property
wealth and expenditures, it may be argued that discrimination might be
found without regard to the individual income characteristics of
district residents. Assuming a perfect correlation between district
property wealth and expenditures from top to to bottom, the
disadvantaged class might be viewed as encompassing every child in every
district except the district that has the most assessable wealth and
spends the most on education Alternatively, as suggested in Mr.~Justice
MARSHALL's dissenting opinion, post, the class might be defined more
restrictively to include children in districts with assessable property
which falls below the statewide average, or median, or below some other
artificially defined level.

However described, it is clear that appellees' suit asks this Court to
extend its most exacting scrutiny to review a system that allegedly
discriminates against a large, diverse, and amorphous class, unified
only by the common factor of residence in districts that happen to have
less taxable wealth than other districts The system of alleged
discrimination and the class it defines have none of the traditional
indicia of suspectness: the class is not saddled with such disabilities,
or subjected to such a history of purposeful unequal treatment, or
relegated to such a position of political powerlessness as to command
extraordinary protection from the majoritarian political process.

We thus conclude that the Texas system does not operate to the peculiar
disadvantage of any suspect class.

But in recognition of the fact that this Court has never heretofore held
that wealth discrimination alone provides an adequate basis for invoking
strict scrutiny, appellees have not relied solely on this contention
They also assert that the State's system impermissibly interferes with
the exercise of a `fundamental' right and that accordingly the prior
decisions of this Court require the application of the strict standard
of judicial review. Graham v. Richardson; Kramer v. Union Free School
District; Shapiro v. ThompsonIt is this question---whether education is
a fundamental right, in the sense that it is among the rights and
liberties protected by the Constitution---which has so consumed the
attention of courts and commentators in recent years.

In Brown v. Board of Educationa unanimous Court recognized that
`education is perhaps the most important function of state and local
governments.' What was said there in the context of racial
discrimination has lost none of its vitality with the passage of time:

`Compulsory school attendance laws and the great expenditures for
education both demonstrate our recognition of the importance of
education to our democratic society. It is required in the performance
of our most basic public responsibilities, even service in the armed
forces. It is the very foundation of good citizenship. Today it is a
principal instrument in awakening the child to cultural values, in
preparing him for later professional training, and in helping him to
adjust normally to his environment. In these days, it is doubtful that
any child may reasonably be expected to succeed in life if he is denied
the opportunity of an education. Such an opportunity, where the state
has undertaken to provide it, is a right which must be made available to
all on equal terms.'

This theme, expressing an abiding respect for the vital role of
education in a free society, may be found in numerous opinions of
Justices of this Court writing both before and after Brown was decided.
Wisconsin v. Yoder; Abington School Dist. v. Schempp, (Brennan, J.);
People of State of Illinois ex rel. McCollum v. Board of Education,
(Frankfurter, J.); Pierce v. Society of Sisters; Meyer v. Nebraska;
Interstate Consolidated Street R. Co.~v. Massachusetts.

Nothing this Court holds today in any way detracts from our historic
dedication to public education. We are in complete agreement with the
conclusion of the three-judge panel below that `the grave significance
of education both to the individual and to our society' cannot be
doubted But the importance of a service performed by the State does not
determine whether it must be regarded as fundamental for purposes of
examination under the Equal Protection Clause. Mr.~Justice Harlan,
dissenting from the Court's application of strict scrutiny to a law
impinging upon the right of interstate travel, admonished that
`(v)irtually every state statute affects important rights.' Shapiro v.
Thompson. In his view, if the degree of judicial scrutiny of state
legislation fluctuated, depending on a majority's view of the importance
of the interest affected, we would have gone 'far toward making this
Court a 'super-legislature." We would, indeed, then be assuming a
legislative role and one for which the Court lacks both authority and
competence. But Mr.~Justice Stewart's response in Shapiro to Mr.~Justice
Harlan's concern correctly articulates the limits of the
fundamental-rights rationale employed in the Court's equal protection
decisions:

`The Court today does not 'pick out particular human activities,
characterize them as 'fundamental,' and give them added protection . ..'
To the contrary, the Court simply recognizes, as it must, an established
constitutional right, and gives to that right no less protection than
the Constitution itself demands.' (Emphasis in original.)

Mr.~Justice Stewart's statement serves to underline what the opinion of
the Court in Shapiro makes clear. In subjecting to strict judicial
scrutiny state welfare eligibility statutes that imposed a one-year
durational residency requirement as a precondition to receiving AFDC
benefits, the Court explained:

`(I)n moving from State to State . appellees were exercising a
constitutional right, and any classification which serves to penalize
the exercise of that right, unless shown to be necessary to promote a
compelling governmental interest, is unconstitutional.' (Emphasis in
original.)

The right to interstate travel had long been recognized as a right of
constitutional significance,70 and the Court's decision, therefore, did
not require an ad hoc determination as to the social or economic
importance of that right.

Lindsey v. Normetdecided only last Term, firmly reiterates that social
importance is not the critical determinant for subjecting state
legislation to strict scrutiny. The complainants in that case, involving
a challenge to the procedural limitations imposed on tenants in suits
brought by landlords under Oregon's Forcible Entry and Wrongful Detainer
Law, urged the Court to examine the operation of the statute under `a
more stringent standard than mere rationality.' The tenants argued that
the statutory limitations implicated `fundamental interests which are
particularly important to the poor,' such as the ``need for decent
shelter'' and the ``right to retain peaceful possession of one's home.''
Mr.~Justice White's analysis, in his opinion for the Court is
instructive:

`We do not denigrate the importance of decent, safe and sanitary
housing. But the Constitution does not provide judicial remedies for
every social and economic ill.~We are unable to perceive in that
document any constitutional guarantee of access to dwellings of a
particular quality or any recognition of the right of a tenant to occupy
the real property of his landlord beyond the term of his lease, without
the payment of rent . .. Absent constitutional mandate, the assurance of
adequate housing and the definition of landlord-tenant relationships are
legislative, not judicial, functions.' (Emphasis supplied.)

Similarly, in Dandridge v. Williams the Court's explicit recognition of
the fact that the `administration of public welfare assistance .
involves the most basic economic needs of impoverished human beings,' 72
provided no basis for departing from the settled mode of constitutional
analysis of legislative classifications involving questions of economic
and social policy. As in the case of housing, the central importance of
welfare benefits to the poor was not an adequate foundation for
requiring the State to justify its law by showing some compelling state
interest. See also Jefferson v. Hackney; Richardson v. Belcher.

The lesson of these cases in addressing the question now before the
Court is plain. It is not the province of this Court to create
substantive constitutional rights in the name of guaranteeing equal
protection of the laws. Thus, the key to discovering whether education
is `fundamental' is not to be found in comparisons of the relative
societal significance of education as opposed to subsistence or housing.
Nor is it to be found by weighing whether education is as important as
the right to travel. Rather, the answer lies in assessing whether there
is a right to education explicitly or implicitly guaranteed by the
Constitution.

Education, of course, is not among the rights afforded explicit
protection under our Federal Constitution. Nor do we find any basis for
saying it is implicitly so protected. As we have said, the undisputed
importance of education will not alone cause this Court to depart from
the usual standard for reviewing a State's social and economic
legislation. It is appellees' contention, however, that education is
distinguishable from other services and benefits provided by the State
because it bears a peculiarly close relationship to other rights and
liberties accorded protection under the Constitution. Specifically, they
insist that education is itself a fundamental personal right because it
is essential to the effective exercise of First Amendment freedoms and
to intelligent utilization of the right to vote. In asserting a nexus
between speech and education, appellees urge that the right to speak is
meaningless unless the speaker is capable of articulating his thoughts
intelligently and persuasively. The `marketplace of ideas' is an empty
forum for those lacking basic communicative tools. Likewise, they argue
that the corollary right to receive information77 becomes little more
than a hollow privilege when the recipient has not been taught to read,
assimilate, and utilize available knowledge.

A similar line of reasoning is pursued with respect to the right to
vote. 78 Exercise of the franchise, it is contended, cannot be divorced
from the educational foun- dation of the voter. The electoral process,
if reality is to conform to the democratic ideal, depends on an informed
electorate: a voter cannot cast his ballot intelligently unless his
reading skills and thought processes have been adequately developed.

We need not dispute any of these propositions. The Court has long
afforded zealous protection against unjustifiable governmental
interference with the individual's rights to speak and to vote. Yet we
have never presumed to possess either the ability or the authority to
guarantee to the citizenry the most effective speech or the most
informed electoral choice. That these may be desirable goals of a system
of freedom of expression and of a representative form of government is
not to be doubted These are indeed goals to be pursued by a people whose
thoughts and beliefs are freed from governmental interference. But they
are not values to be implemented by judicial instrusion into otherwise
legitimate state activities.

Even if it were conceded that some identifiable quantum of education is
a constitutionally protected prerequisite to the meaningful exercise of
either right, we have no indication that the present levels of
educational expendi tures in Texas provide an education that falls
short. Whatever merit appellees' argument might have if a State's
financing system occasioned an absolute denial of educational
opportunities to any of its children, that argument provides no basis
for finding an interference with fundamental rights where only relative
differences in spending levels are involved and where---as is true in
the present case---no charge fairly could be made that the system fails
to provide each child with an opportunity to acquire the basic minimal
skills necessary for the enjoyment of the rights of speech and of full
participation in the political process.

Furthermore, the logical limitations on appellees' nexus theory are
difficult to perceive. How, for instance, is education to be
distinguished from the significant personal interests in the basics of
decent food and shelter? Empirical examination might well buttress an
assumption that the ill-fed, ill-clothed, and ill-housed are among the
most ineffective participants in the political process, and that they
derive the least enjoyment from the benefits of the First Amendment If
so, appellees' thesis would cast serious doubt on the authority of
Dandridge v. Williamsand Lindsey v. Normer.

We have carefully considered each of the arguments supportive of the
District Court's finding that education is a fundamental right or
liberty and have found those arguments unpersuasive. In one further
respect we find this a particularly inappropriate case in which to
subject state action to strict judicial scrutiny. The present case, in
another basic sense, is significantly different from any of the cases in
which the Court has applied strict scrutiny to state or federal
legislation touching upon constitutionally protected rights. Each of our
prior cases involved legislation which `deprived,' `infringed,' or
`interfered' with the free exercise of some such fundamental personal
right or liberty. See Skinner v. Oklahoma, ex rel. Williamson U.S., 62
S.Ct.; Shapiro v. Thompson U.S., 89 S.Ct.; Dunn v. Blumstein
U.S.S.Ct.---1004. A critical distinction between those cases and the one
now before us lies in what Texas is endeavoring to do with respect to
education. Mr.~Justice Brennan, writing for the Court in Katzenbach v.
Morganexpresses well the salient point:

`This is not a complaint that Congress . has unconstitutionally denied
or diluted anyone's right to vote but rather that Congress violated the
Constitution by not extending the relief effected (to others similarly
situated). (The federal law in question) does not restrict or deny the
franchise but in effect extends the franchise to persons who otherwise
would be denied it by state law. . We need only decide whether the
challenged limitation on the relief effected .. was permissible. In
deciding that question, the principle that calls for the closest
scrutiny of distinctions in laws denying fundamental rights . is
inapplicable; for the distinction challenged by appellees is presented
only as a limitation on a reform measure aimed at eliminating an
existing barrier to the exercise of the franchise. Rather, in deciding
the constitutional propriety of the limitations in such a reform measure
we are guided by the familiar principles that a 'statute is not invalid
under the Constitution because it might have gone farther than it did,'
. that a legislature need not `strike at all evils at the same time,' .
and that 'reform may take one step at a time, addressing itself to the
phase of the problem which seems most acute to the legislative mind .
.." (Emphasis in original.)

The Texas system of school financing is not unlike the federal
legislation involved in Katzenbach in this regard. Every step leading to
the establishment of the system Texas utilizes today---including the
decisions permitting localities to tax and expend locally, and creating
and continuously expanding the state aid---was implemented in an effort
to extend public education and to improve its quality Of course, every
reform that benefits some more than others may be criticized for what it
fails to accomplish. But we think it plain that, in substance, the
thrust of the Texas system is affirmative and reformatory and,
therefore, should be scrutinized under judicial principles sensitive to
the nature of the State's efforts and to the rights reserved to the
States under the Constitution.

It should be clear, for the reasons stated above and in accord with the
prior decisions of this Court, that this is not a case in which the
challenged state action must be subjected to the searching judicial
scrutiny reserved for laws that create suspect classifications or
impinge upon constitutionally protected rights.

We need not rest our decision, however, solely on the inappropriateness
of the strict-scrutiny test. A century of Supreme Court adjudication
under the Equal Protection Clause affirmatively supports the application
of the traditional standard of review, which requires only that the
State's system be shown to bear some rational relationship to legitimate
state purposes. This case represents far more than a challenge to the
manner in which Texas provides for the education of its children. We
have here nothing lass than a direct attack on the way in which Texas
has chosen to raise and disburse state and local tax revenues. We are
asked to condemn the State's judgment in conferring on political
subdivisions the power to tax local property to supply revenues for
local interests. In so doing, appellees would have the Court intrude in
an area in which it has traditionally deferred to state legislatures
This Court has often admonished against such interferences with the
State's fiscal policies under the Equal Protection Clause:

`The broad discretion as to classification possessed by a legislature in
the field of taxation has long been recognized. . (T)he passage of time
has only served to underscore the wisdom of that recognition of the
large area of discretion which is needed by a legislature in formulating
sound tax poli- cies. . It has . been pointed out that in taxation, even
more than in other fields, legislatures possess the greatest freedom in
classification. Since the members of a legislature necessarily enjoy a
familiarity with local conditions which this Court cannot have, the
presumption of constitutionality can be overcome only by the most
explicit demonstration that a classification is a hostile and oppressive
discrimination against particular persons and classes. . .'

Thus, we stand on familiar grounds when we continue to acknowledge that
the Justices of this Court lack both the expertise and the familiarity
with local problems so necessary to the making of wise decisions with
respect to the raising and disposition of public revenues. Yet, we are
urged to direct the States either to alter drastically the present
system or to throw out the property tax altogether in favor of some
other form of taxation. No scheme of taxation, whether the tax is
imposed on property, income, or purchases of goods and services, has yet
been devised which is free of all discriminatory impact. In such a
complex arena in which no perfect alternatives exist, the Court does
well not to impose too rigorous a standard of scrutiny lest all local
fiscal schemes become subjects of criticism under the Equal Protection
Clause.

In addition to matters of fiscal policy, this case also involves the
most persistent and difficult questions of educational policy, another
area in which this Court's lack of specialized knowledge and experience
counsels against premature interference with the informed judgments made
at the state and local levels. Education, perhaps even more than welfare
assistance, presents a myriad of `intractable economic, social, and even
philosophical problems.' Dandridge v. Williams, The very complexity of
the problems of financing and managing a statewide public school system
suggests that `there will be more than one constitutionally permissible
method of solving them,' and that, within the limits of rationality,
`the legislature's efforts to tackle the problems' should be entitled to
respect. Jefferson v. Hackney---547, On even the most basic questions in
this area the scholars and educational experts are divided. Indeed, one
of the major sources of controversy concerns the extent to which there
is a demonstrable correlation between educational expenditures and the
quality of education86---an assumed correlation underlying virtually
every legal conclusion drawn by the District Court in this case. Related
to the questioned relationship between cost and quality is the equally
unsettled controversy as to the proper goals of a system of public
education And the question regarding the most effective relationship
between state boards of education and local school boards, in terms of
their respective responsibilities and degrees of control, is now
undergoing searching re-examination. The ultimate wisdom as to these and
related problems of education is not likely to be divined for all time
even by the scholars who now so earnestly debate the issues. In such
circumstances, the judiciary is well advised to refrain from imposing on
the States inflexible constitutional restraints that could circumscribe
or handicap the continued research and experimentation so vital to
finding even partial solutions to educational problems and to keeping
abreast of ever-changing conditions.

It must be remembered, also, that every claim arising under the Equal
Protection Clause has implications for the relationship between national
and state power under our federal system. Questions of federalism are
always inherent in the process of determining whether a State's laws are
to be accorded the traditional presumption of constitutionality, or are
to be subjected instead to rigorous judicial scrutiny. While '(t)he
maintenance of the principles of federalism is a foremost consideration
in interpreting any of the pertinent constitutional provisions under
which this Court examines state action,'88 it would be difficult to
imagine a case having a greater potential impact on our federal system
than the one now before us, in which we are urged to abrogate systems of
financing public education presently in existence in virtually every
State.

The foregoing considerations buttress our conclusion that Texas' system
of public school finance is an inappropriate candidate for strict
judicial scrutiny. These same considerations are relevant to the
determination whether that system, with its conceded imperfections,
nevertheles bears some rational relationship to a legitimate state
purpose. It is to this question that we next turn our attention.

The basic contours of the Texas school finance system have been traced
at the outset of this opinion. We will now describe in more detail that
system and how it operates, as these facts bear directly upon the
demands of the Equal Protection Clause.

Apart from federal assistance, each Texas school receives its funds from
the State and from its local school district. On a statewide average, a
roughly comparable amount of funds is derived from each source The
State's contribution, under the Minimum Foundation Program, was designed
to provide an adequate minimum educational offering in every school in
the State. Funds are distributed to assure that there will be one
teacher---compensated at the statesupported minimum salary---for every
25 students Each school

district's other supportive personnel are provided for: one principal
for every 30 teachers one `special service' teacher---librarian, nurse,
doctor, etc.---for every 20 teachers superintendents, vocational
instructors, counselors, and educators for exceptional children are also
provided Additional funds are earmarked for current operating expenses,
for student transportation,94 and for free textbooks.

The program is administered by the State Board of Education and by the
Central Education Agency, which also have responsibility for school
accreditation96 and for monitoring the statutory teacher-qualification
standards As reflected by the 62\% increase in funds allotted to the
Edgewood School District over the last three years,98 the State's
financial contribution to education is steadily increasing. None of
Texas' school districts, how- ever, has been content to rely alone on
funds from the Foundation Program.

By virtue of the obligation to fulfill its Local Fund Assignment, every
district must impose an ad valorem tax on property located within its
borders. The Fund Assignment was designed to remain sufficiently low to
assure that each district would have some ability to provide a more
enriched educational program Every district supplements its Foundation
grant in this manner. In some districts, the local property tax
contribution is insubstantial, as in Edgewood where the supplement was
only \$26 per pupil in 1967. In other districts, the local share may far
exceed even the total Foundation grant. In part, local differences are
attributable to differences in the rates of taxation or in the degree to
which the market value for any category of property varies from its
assessed value The greatest interdistrict disparities, however, are
attributable to differences in the amount of assessable property
available within any district. Those districts that have more property,
or more valuable property, have a greater capability for supplementing
state funds. In large measure, these additional local revenues are
devoted to paying higher salaries to more teachers. Therefore, the
primary distinguishing attributes of schools in property-affluent
districts are lower pupil-teacher ratios and higher salary schedules.

This, then, is the basic outline of the Texas school financing
structure. Because of differences in expenditure levels occasioned by
disparities in property tax income, appellees claim that children in
less affluent districts have been made the subject of invidious
discrimination. The District Court found that the State had failed even
`to establish a reasonable basis' for a system that results in different
levels of per-pupil expenditure. 337 F.Supp.. We disagree.

In its reliance on state as well as local resources, the Texas system is
comparable to the systems employed in virtually every other State The
power to tax local property for educational purposes has been recognized
in Texas at least since 1883 When the growth of commercial and
industrial centers and accompanying shifts in population began to create
disparities in local resources, Texas undertook a program calling for a
considerable investment of state funds.

The `foundation grant' theory upon which Texas legislators and educators
based the Gilmer-Aikin bills, was a product of the pioneering work of
two New York educational reformers in the 1920's, George D. Strayer and
Robert M. Haig. 104 Their efforts were devoted to establishing a means
of guaranteeing a minimum statewide educational program without
sacrificing the vital element of local participation. The Strayer-Haig
thesis represented an accommodation between these two competing forces.
As articulated by Professor Coleman:

`The history of education since the industrial revolution shows a
continual struggle between two forces: the desire by members of society
to have educational opportunity for all children, and the desire of each
family to provide the best education it can afford for its own
children.'

The Texas system of school finance is responsive to these two forces.
While assuring a basis education for every child in the State, it
permits and encourages a large measure of participation in and control
of each district's schools at the local level. In an era that has
witnessed a consistent trend toward centralization of the functions of
government, local sharing of responsibility for public education has
survived. The merit of local control was recognized last Term in both
the majority and dissenting opinions in Wright v. Council of the City of
EmporiaMr. Justice Stewart stated there that `(d)irect control over
decisions vitally affecting the education of one's children is a need
that is strongly felt in our society.' The Chief Justice, in his
dissent, agreed that `(l)ocal control is not only vital to continued
public support of the schools, but it is of overriding importance from
an educational standpoint as well.'

The persistence of attachment to government at the lowest level where
education is concerned reflects the depth of commitment of its
supporters. In part, local control means, as Professor Coleman suggests,
the freedom to devote more money to the education of one's children.
Equally important, however, is the opportunity it offers for
participation in the decisionmaking process that determines how those
local tax dollars will be spent. Each locality is free to tailor local
programs to local needs. Pluralism also affords some opportunity for
experimentation, innovation, and a healthy competition for educational
excellence. An analogy to the Nation-State relationship in our federal
system seems uniquely appropriate. Mr.~Justice Brandeis identified as
one of the peculiar strengths of our form of government each State's
freedom to 'serve as a laboratory; and try novel social and economic
experiments.'106 No area of social concern stands to profit more from a
multiplicity of viewpoints and from a diversity of approaches than does
public education.

Appellees do not question the propriety of Texas' dedication to local
control of education. To the contrary, they attack the school-financing
system precisely because, in their view, it does not provide the same
level of local control and fiscal flexibility in all districts.
Appellees suggest that local control could be preserved and promoted
under other financing systems that resulted in more equality in
education expenditures. While it is no doubt true that reliance on local
property taxation for school revenues provides less freedom of choice
with respect to expenditures for some districts than for others,107 the
existence of `some inequality' in the manner in which the State's
rationale is achieved is not alone a sufficient basis for striking down
the entire system. McGowan v. Maryland, It may not be condemned simply
because it imperfectly effectuates the State's goals. Dandridge v.
Williams, Nor must the financing system fail because, as appellees
suggest, other methods of satisfying the State's interest, which
occasion `less drastic' disparities in expenditures, might be conceived.
Only where state action impinges on the exercise of fundamental
constitutional rights or liberties must it be found to have chosen the
least restrictive alternative. Cf. Dunn v. Blumstein, 92 S.Ct.; Shelton
v. Tucker, It is also well to remember that even those districts that
have reduced ability to make free decisions with respect to how much
they spend on education still retain under the present system a large
measure of authority as to how available funds will be allocated. They
further enjoy the power to make numerous other decisions with respect to
the operation of the schools The people of Texas may be justified in
believing that other systems of school financing, which place more of
the financial responsibility in the hands of the State, will result in a
comparable lessening of desired local autonomy. That is, they may
believe that along with increased control of the purse strings at the
state level will go increased control over local policies.

Appellees further urge that the Texas system is unconstitutionally
arbitrary because it allows the availability of local taxable resources
to turn on `happenstance.' They see no justification for a system that
allows, as they contend, the quality of education to fluctuate on the
basis of the fortuitous positioning of the boundary lines of political
subdivisions and the location of valuable commercial and industrial
property. But any scheme of local taxation---indeed the very existence
of identifiable local governmental units---requires the establishment of
jurisdictional boundaries that are inevitably arbitrary. It is equally
inevitable that some localities are going to be blessed with more
taxable assets than others Nor is local wealth a static quantity.
Changes in the level of taxable wealth within any district may result
from any number of events, some of which local residents can and do
influence. For instance, commercial and industrial enterprises may be
encouraged to locate within a district by various actions---public and
private.

Moreover, if local taxation for local expenditures were an
unconstitutional method of providing for education then it might be an
equally impermissible means of providing other necessary services
customarily financed largely from local property taxes, including local
police and fire protection, public health and hospitals, and public
utility facilities of various kinds. We perceive no justification for
such a severe denigration of local property taxation and control as
would follow from appellees' contentions. It has simply never been
within the constitutional prerogative of this Court to nullify statewide
measures for financing public services merely because the burdens or
benefits thereof fall unevenly depending upon the relative wealth of the
political subdivisions in which citizens live.

In sum, to the extent that the Texas system of school financing results
in unequal expenditures between chil- dren who happen to reside in
different districts, we cannot say that such disparities are the product
of a system that is so irrational as to be invidiously discriminatory.
Texas has acknowledged its shortcomings and has persistently
endeavored---not without some success---to ameliorate the differences in
levels of expenditures without sacrificing the benefits of local
participation. The Texas plan is not the result of hurried,
ill-conceived legislation. It certainly is not the product of purposeful
discrimination against any group or class. On the contrary, it is rooted
in decades of experience in Texas and elsewhere, and in major part is
the product of responsible studies by qualified people. In giving
substance to the presumption of validity to which the Texas system is
entitled, Lindsley v. Natural Carbonic Gas Co., it is important to
remember that at every stage of its development it has constituted a
`rough accommodation' of interests in an effort to arrive at practical
and workable solutions. Metropolis Theatre Co.~v. City of Chicago---70,
One also must remember that the system here challenged is not peculiar
to Texas or to any other State. In its essential characteristics, the
Texas plan for financing public education reflects what many educators
for a half century have thought was an enlightened approach to a problem
for which there is no perfect solution. We are unwilling to assume for
ourselves a level of wisdom superior to that of legislators, scholars,
and educational authorities in 50 States, especially where the
alternatives proposed are only recently conceived and nowhere yet
tested. The constitutional standard under the Equal Protection Clause is
whether the challenged state action rationally furthers a legitimate
state purpose or interest. McGinnis v. Royster, We hold that the Texas
plan abundantly satisfies this standard.

In light of the considerable attention that has focused on the District
Court opinion in this case and on its California predecessor, Serrano v.
Priest, 5 Cal d 584, 96 Cal.Rptr. 601, 487 P d 1241 (1971), a cautionary
postscript seems appropriate. It cannot be questioned that the
constitutional judgment reached by the District Court and approved by
our dissenting Brothers today would occasion in Texas and elsewhere an
unprecedented upheaval in public education. Some commentators have
concluded that, whatever the contours of the alternative financing
programs that might be devised and approved, the result could not avoid
being a beneficial one. But, just as there is nothing simple about the
constitutional issues involved in these cases, there is nothing simple
or certain about predicting the consequences of massive change in the
financing and control of public education. Those who have devoted the
most thoughtful attention to the practical ramifications of these cases
have found no clear or dependable answers and their scholarship reflects
no such unqualified confidence in the desirability of completely
uprooting the existing system.

The complexity of these problems is demonstrated by the lack of
consensus with respect to whether it may be said with any assurance that
the poor, the racial minorities, or the children in over-burdened
core-city school districts would be benefited by abrogation of
traditional modes of financing education. Unless there is to be a
substantial increase in state expenditures on education across the
board---an event the likelihood of which is open to considerable
question 111---these groups stand to realize gains in terms of increased
per-pupil expenditures only if they reside in districts that presently
spend at relatively low levels, i.e., in those districts that would
benefit from the redistribution of existing resources. Yet, recent
studies have indicated that the poorest families are not invariably
clustered in the most impecunious school districts Nor does it now
appear that there is any more than a random chance that racial
minorities are concentrated in property-poor districts Additionally,
several research projects have concluded that any financing alternative
designed to achieve a greater equality of expenditures is likely to lead
to higher taxation and lower educational expenditures in the major urban
centers,114 a result that would exacerbate rather than ameliorate
existing conditions in those areas.

These practical considerations, of course, play no role in the
adjudication of the constitutional issues presented here. But they serve
to highlight the wisdom of the traditional limitations on this Court's
function. The consideration and initiation of fundamental reforms with
respect to state taxation and education are matters reserved for the
legislative processes of the various States, and we do no violence to
the values of federalism and separation of powers by staying our hand.
We hardly need add that this Court's action today is not to be viewed as
placing its judicial imprimatur on the status quo. The need is apparent
for reform in tax systems which may well have relied too long and too
heavily on the local property tax. And certainly innovative thinking as
to public education, its methods, and its funding is necessary to assure
both a higher level of quality and greater uniformity of opportunity.
These matters merit the continued attention of the scholars who already
have contributed much by their challenges. But the ultimate solutions
must come from the lawmakers and from the democractic pressures of those
who elect them.

Reversed.

\emph{From the footnotes to majority opinion}:

\begin{enumerate}
\def\labelenumi{\arabic{enumi}.}
\setcounter{enumi}{77}
\item
  Since the right to vote, per se, is not a constitutionally protected
  right, we assume that appellees' references to that right are simply
  shorthand references to the protected right, implicit in our
  constitutional system, to participate in state elections on an equal
  basis with other qualified voters whenever the State has adopted an
  elective process for determining who will represent any segment of the
  State's population.
\item
  Those who urge that the present system be invalidated offer little
  guidance as to what type of school financing should replace it. The
  most likely result of rejection of the existing system would be
  state-wide financing of all public education with funds derived from
  taxation of property or from the adoption or expansion of sales and
  income taxes. See Simonn. 62. The authors of Private Wealth and Public
  Educationn. 13---242, suggest an alternative scheme, known as
  `district power equalizing.' In simplest terms, the State would
  guarantee that at any particular rate of property taxation the
  district would receive a stated number of dollars regardless of the
  district's tax base. To finance the subsidies to `poorer' districts,
  funds would be taken away from the `wealthier' districts that, because
  of their higher property values, collect more than the stated amount
  at any given rate. This is not the place to weigh the arguments for an
  against `district power equalizing,' beyond noting that commentators
  are in disagreement as to whether it is feasible, how it would work,
  and indeed whether it would violate the equal protection theory
  underlying appellees' case.
\end{enumerate}

\textbf{Mr.~Justice STEWART, concurring.}

The method of financing public schools in Texas, as in almost every
other State, has resulted in a system of public education that can
fairly be described as chaotic and unjust It does not follow, however,
and I cannot find, that this system violates the Constitution of the
United States. I join the opinion and judgment of the Court because I am
convinced that any other course would mark an extraordinary departure
from principled adjudication under the Equal Protection Clause of the
Fourteenth Amendment. The unchartered directions of such a departure are
suggested, I think, by the imaginative dissenting opinion my Brother
MARSHALL has filed today.

Unlike other provisions of the Constitution, the Equal Protection Clause
confers no substantive rights and creates no substantive liberties The
function of the Equal Protection Clause, rather, is simply to measure
the validity of classifications created by state laws.

There is hardly a law on the books that does not affect some people
differently from others. But the basic concern of the Equal Protection
Clause is with state legislation whose purpose or effect is to create
discrete and objectively identifiable classes And with respect to such
legislation, it has long been settled that the Equal Protection Clause
is offended only by laws that are invidiously discriminatory---only by
classifications that are wholly arbitrary or capricious. See, e.g.,
Rinaldi v. YeagerThis settled principle of constitutional law was
compendiously stated in Mr.~Chief Justice Warren's opinion for the Court
in McGowan v. Maryland---426, in the following words:

`Although no precise formula has been developed, the Court has held that
the Fourteenth Amendment permits the States a wide scope of discretion
in enacting laws which affect some groups of citizens differently than
others. The constitutional safeguard is offended only if the
classification rests on grounds wholly irrelevant to the achievement of
the State's objective. State legislatures are presumed to have acted
within their constitutional power despite the fact that, in practice,
their laws result in some inequality. A statutory discrimination will
not be set aside if any state of facts reasonably may be conceived to
justify it.'

This doctrine is no more than a specific application of one of the first
principles of constitutional adjudication---the basic presumption of the
constitutional validity of a duly enacted state or federal law. See
Thayer, The Origin and Scope of the American Doctrine of Constitutional
Law, 7 Harv.L.Rev. 129 (1893).

Under the Equal Protection Clause, this presumption of constitutional
validity disappears when a State has enacted legislation whose purpose
or effect is to create classes based upon criteria that, in a
constitutional sense, are inherently `suspect.' Because of the historic
purpose of the Fourteenth Amendment, the prime example of such a
`suspect' classification is one that is based upon race. See, e.g.,
Brown v. Board of Education; McLaughlin v. FloridaBut there are other
classifications that, at least in some settings, are also
`suspect'---for example, those based upon national origin, alienage,
indigency, or illegitimacy.

Moreover, quite apart from the Equal Protection Clause, a state law that
impinges upon a substantive right or liberty created or conferred by the
Constitution is, of course, presumptively invalid, whether or not the
law's purpose or effect is to create any classifications. For example, a
law that provided that newspapers could be published only by people who
had resided in the State for five years could be superficially viewed as
invidiously discriminating against an identifiable class in violation of
the Equal Protection Clause. But, more basically, scuch a law would be
invalid simply because it abridged the freedom of the press. Numerous
cases in this Court illustrate this principle.

In refusing to invalidate the Texas system of financing its public
schools, the Court today applies with thoughtfulness and understanding
the basic principles I have so sketchily summarized. First, as the Court
points out, the Texas system has hardly created the kind of objectively
identifiable classes that are cognizable under the Equal Protection
Clause Second, even assuming the existence of such discernible
categories, the classifications are in no sense based upon
constitutionally `suspect' criteria. Third, the Texas system does not
rest `on grounds wholly irrelevant to the achievement of the State's
objective.' Finally, the Texas system impinges upon no substantive
constitutional rights or liberties. It follows, therefore, under the
established principle reaffirmed in Mr.~Chief Justice Warren's opinion
for the Court in McGowan v. Marylandthat the judgment of the District
Court must be reversed.

\textbf{Mr.~Justice BRENNAN, dissenting.}

Although I agree with my Brother WHITE that the Texas statutory scheme
is devoid of any rational basis, and for that reason is violative of the
Equal Protection Clause, I also record my disagreement with the Court's
rather distressing assertion that a right may be deemed `fundamental'
for the purposes of equal protection analysis only if it is `explicitly
or implicitly guaranteed by the Constitution.' As my Brother MARSHALL
convincingly demonstrates, our prior cases stand for the proposition
that `fundamentality' is, in large measure, a function of the right's
importance in terms of the effectuation of those rights which are in
fact constitutionally guaranteed. Thus, `(a)s the nexus between the
specific constitutional guarantee and the nonconstitutional interest
draws closer, the non- constitutional interest becomes more fundamental
and the degree of judicial scrutiny applied when the interest is
infringed on a discriminatory basis must be adjusted accordingly.'

Here, there can be no doubt that education is inextricably linked to the
right to participate in the electoral process and to the rights of free
speech and association guaranteed by the First Amendment. This being so,
any classification affecting education must be subjected to strict
judicial scrutiny, and since even the State concedes that the statutory
scheme now before us cannot pass constitutional muster under this
stricter standard of review, I can only conclude that the Texas
school-financing scheme is constitutionally invalid.

\textbf{Mr.~Justice WHITE, with whom Mr.~Justice DOUGLAS and Mr.~Justice
BRENNAN join, dissenting.}

The Texas public schools are financed through a combination of state
funding, local property tax revenue, and some federal funds Concededly,
the system yields wide disparity in per-pupil revenue among the various
districts. In a typical year, for example, the Alamo Heights district
had total revenues of \$594 per pupil, while the Edgewood district had
only \$356 per pupil The majority and the State concede, as they must,
the existence of major disparities in spendable funds. But the State
contends that the disparities do not invidiously discriminate against
children and families in districts such as Edgewood, because the Texas
scheme is designed `to provide an adequate education for all, with local
autonomy to go beyond that as individual school districts desire and are
able . .. It leaves to the people of each district the choice whether to
go beyond the minimum and, if so, by how much.' The majority advances
this rationalization: `While assuring a basic education for every child
in the State, it permits and encourages a large measure of participation
in and control of each district's schools at the local level.'

I cannot disagree with the proposition that local control and local
decisionmaking play an important part in our democratic system of
government. Cf. James v. ValtierraMuch may be left to local option, and
this case would be quite different if it were true that the Texas
system, while insuring minimum educational expenditures in every
district through state funding, extended a meaningful option to all
local districts to increase their per-pupil expenditures and so to
improve their children's education to the extent that increased funding
would achieve that goal. The system would then arguably provide a
rational and sensible method of achieving the stated aim of preserving
an area for local initiative and decision.

The difficulty with the Texas system, however, is that it provides a
meaningful option to Alamo Heights and like school districts but almost
none to Edgewood and those other districts with a low per-pupil real
estate tax base. In these latter districts, no matter how desirous
parents are of supporting their schools with greater revenues, it is
impossible to do so through the use of the real estate property tax. In
these districts, the Texas system utterly fails to extend a realistic
choice to parents because the property tax, which is the only
revenue-raising mechanism extended to school districts, is practically
and legally unavailable. That this is the situation may be readily
demonstrated.

Local school districts in Texas raise their portion of the Foundation
School Program---the Local Fund Assignment---by levying ad valorem taxes
on the property located within their boundaries. In addition, the
districts are authorized, by the state constitution and by statute, to
levy ad valorem property taxes in order to raise revenues to support
educational spending over and above the expenditure of Foundation School
Program funds.

Both the Edgewood and Alamo Heights districts are located in Bexar
County, Texas. Student enrollment in Alamo Heights is 5,432, in Edgewood
22,862. The per-pupil market value of the taxable property in Alamo
Heights is \$49,078, in Edgewood \$5,960. In a typical relevant year,
Alamo Heights had a maintenance tax rate of \$1 and a debt service
(bond) tax rate of 20¢ per \$100 assessed evaluation, while Edgewood had
a maintenance rate of 52¢ and a bond rate of 67¢. These rates, when
applied to the respective tax bases, yielded Alamo Heights \$1,433,473
in maintenance dollars and \$236,074 in bond dollars, and Edgewood
\$223,034 in maintenance dollars and \$279,023 in bond dollars. As is
readily apparent, because of the variance in tax bases between the
districts, results, in terms of revenues, do not correlate with effort,
in terms of tax rate. Thus, Alamo Heights, with a tax base approximately
twice the size of Edgewood's base, realized approximately six times as
many maintenance dollars as Edgewood by using a tax rate only
approximately two and one-half times larger. Similarly, Alamo Heights
realized slightly fewer bond dollars by using a bond tax rate less than
one-third of that used by Edgewood.

Nor is Edgewood's revenue-raising potential only deficient when compared
with Alamo Heights. North East District has taxable property with a
per-pupil market value of approximately \$31,000, but total taxable
property approximately four and one-half times that of Edgewood.
Applying a maintenance rate of \$1, North East yielded \$2,818,148.
Thus, because of its superior tax base, North East was able to apply a
tax rate slightly less than twice that applied by Edgewood and yield
more than 10 times the maintenance dollars. Similarly, North East, with
a bond rate of 45¢, yielded \$1,249,159---more than four times
Edgewood's yield with two-thirds the rate.

Plainly, were Alamo Heights or North East to apply the Edgewood tax rate
to its tax base, it would yield far greater revenues than Edgewood is
able to yield applying those same rates to its base. Conversely, were
Edgewood to apply the Alamo Heights or North East rates to its base, the
yield would be far smaller than the Alamo Heights or North East yields.
The disparity is, therefore, currently operative and its impact on
Edgewood is undeniably serious. It is evident from statistics in the
record that show that, applying an equalized tax rate of 85¢ per \$100
assessed valuation, Alamo Heights was able to provide approximately
\$330 per pupil in local revenues over and above the Local Fund
Assignment. In Edgewood, on the other hand, with an equalized tax rate
of \$1 per \$100 of assessed valuation, \$26 per pupil was raised beyond
the Local Fund Assignment As previously noted in Alamo Heights, total
per-pupil revenues from local, state, and federal funds was \$594 per
pupil, in Edgewood \$356.

In order to equal the highest yield in any other Bexar County district,
Alamo Heights would be required to tax at the rate of 68 per \$100 of
assessed valuation. Edgewood would be required to tax at the prohibitive
rate of \$5 per \$100. But state law places a \$1 per \$100 ceiling on
the maintenance tax rate, a limit that would surely be reached long
before Edgewood attained an equal yield. Edgewood is thus precluded in
law, as well as in fact, from achieving a yield even close to that of
some other districts.

The Equal Protection Clause permits discriminations between classes but
requires that the classification bear some rational relationship to a
permissible object sought to be attained by the statute. It is not
enough that the Taxas system before us seeks to achieve the valid,
rational purpose of maximizing local initiative; the means chosen by the
State must also be rationally ralated to the end sought to be achieved.
As the Court stated just lat Term in Weber v. Aetna Casualty \& Surety
Co.

`The tests to determine the validity of state statutes under the Equal
Protection Clause have been variously expressed, but this Court
requires, at a minimum, that a statutory classification bear some
rational relationship to a legitimate state purpose. Morey v. Doud;
Williamson v. Lee Optical Co.; Gulf Colorado \& Santa Fe Ry. v. Ellis;
Yick Wo v. Hopkins.'

Neither Taxas nor the majority heeds this rule. If the State aims at
maximizing local initiative and local choice, by permitting school
districts to resort to the real property tax if they choose to do so, it
utterly fails in achieving its purpose in districts with property tax
bases so low that there is little if any opportunity for interested
parents, rich or poor, to augment school district revenues. Requiring
the State to establish only that unequal treatment is in furtherance of
a permissible goal, without also requiring the State to show that the
means chosen to effectuate that goal are rationally related to its
achievement, makes equal protection analysis no more than an empty
gesture In my view, the parents and children in Edgewood, and in like
districts, suffer from an invidious discrimination violative of the
Equal Protection Clause.

This does not, of course, mean that local control may not be a
legitimate goal of a school-financing system. Nor does it mean that the
State must guarantee each district an equal per-pupil revenue from the
state school-financing system. Nor does it mean, as the majority appears
to believe, that, by affirming the decision below, this Court would be
`imposing on the States inflexible constitutionl restraints that could
circumscribe or handicap the continued research and experimentation so
vital to finding even partial solutions to educational problems and to
keeping abreast of ever-changing conditions.' On the contrary, it would
merely mean that the State must fashion a financing scheme which
provides a rational basis for the maximization of local control, if
local control is to remain a goal of the system, and not a scheme with
`different treatment be(ing) accorded to persons placed by a statute
into different classes on the basis of criteria wholly unrelated to the
objective of that statute.' Reed v. Reed---76,

Perhaps the majority believes that the major disparity in revenues
provided and permitted by the Texas system is inconsequential. I cannot
agree, however, that the difference of the magnitude appearing in this
case can sensibly be ignored, particularly since the State itself
considers it so important to provide opportunities to exceed the minimum
state educational expenditures.

There is no difficulty in identifying the class that is subject to the
alleged discrimination and that is entitled to the benefits of the Equal
Protection Clause. I need go no further than the parents and children in
the Edgewood district, who are plaintiffs here and who assert that they
are entitled to the same choice as Alamo Heights to augment local
expenditures for schools but are denied that choice by state law. This
group constitutes a class sufficiently definite to invoke the protection
of the Constitution. They are as entitled to the protection of the Equal
Protection Clause as were the voters in allegedly underrepresented
counties in the reapportionment cases. See, e.g., Baker v. Carr, ; Gray
v. Sanders, ; Reynolds v. Sims, And in Bullock v. Carterwhere a
challenge to the

Texas candidate filing fee on equal protection grounds was upheld, we
noted that the victims of alleged discrimination wrought by the filing
fee `cannot be described by reference to discrete and precisely defined
segments of the community as is typical of inequities challenged under
the Equal Protection Clause,' but concluded that `we would ignore
reality were we not to recognize that this system falls with unequal
weight on voters, as well as candidates, according to their economic
status.' Similarly, in the present case we would blink reality to ignore
the fact that school districts, and students in the end, are
differentially affected by the Texas school-financing scheme with
respect to their capability to supplement the Minimum Foundation School
Program. At the very least, the law discriminates against those children
and their parents who live in districts where the per-pupil tax base is
sufficiently low to make impossible the provision of comparable school
revenues by resort to the real property tax which is the only device the
State extends for this purpose.

\textbf{Mr.~Justice MARSHALL, with whom Mr.~Justice DOUGLAS concurs,
dissenting.}

The Court today decides, in effect, that a State may constitutionally
vary the quality of education which it offers its children in accordance
with the amount of taxable wealth located in the school districts within
which they reside. The majority's decision represents an abrupt
departure from the mainstream of recent state and federal court
decisions concerning the unconstitutionality of state educational
financing schemes dependent upon taxable local wealth More
unfortunately, though, the majority's holding can only be seen as a
retreat from our historic commitment to equality of educational
opportunity and as unsupportable acquiescence in a system which deprives
children in their earliest years of the chance to reach their full
potential as citizens. The Court does this despite the absence of any
substantial justification for a scheme which arbitrarily channels
educational resources in accordance with the fortuity of the amount of
taxable wealth within each district.

In my judgment, the right of every American to an equal start in life,
so far as the provision of a state service as important as education is
concerned, is far too vital to permit state discrimination on grounds as
tenuous as those presented by this record. Nor can I accept the notion
that it is sufficient to remit these appellees to the vagaries of the
political process which, contrary to the majority's suggestion, has
proved singularly unsuited to the task of providing a remedy for this
discrimination I, for one, am unsatisfied with the hope of an ultimate
`political' solution sometime in the indefinite future while, in the
meantime, countless children unjustifiably receive inferior educations
that may affect their hearts and minds in a way unlikely ever to be
undone.' Brown v. Board of Education, 98 l.Ed. 873 (1954). I must
therefore respectfully dissent. I

The Court acknowledges that `substantial interdistrict disparities in
school expenditures' exist in Texas, and that these disparities are
`largely attributable to differences in the amounts of money collected
through local property taxation,' But instead of closely examining the
seriousness of these disparities and the invidiousness of the Texas
financing scheme, the Court undertakes an elaborate exploration of the
efforts Texas has purportedly made to close the gaps between its
districts in terms of levels of district wealth and resulting
educational funding. Yet, however praiseworthy Texas' equalizing
efforts, the issue in this case is not whether Texas is doing its best
to ameliorate the worst features of a discriminatory scheme but, rather,
whether the scheme itself is in fact unconstitutionally discriminatory
in the face of the Fourteenth Amendment's guarantee of equal protection
of the laws. When the Texas financing scheme is taken as a whole, I do
not think it can be doubted that it produces a discriminatory impact on
substantial numbers of the schoolage children of the State of Texas.

Funds to support public education in Texas are derived from three
sources: local ad valorem property taxes; the Federal Government; and
the state government It is enlightening to consider these in order.

Under Texas law, the only mechanism provided the local school district
for raising new, unencumbered revenues is the power to tax property
located within its boundaries At the same time, the Texas financing
scheme effectively restricts the use of monies raised by local property
taxation to the support of public education within the boundaries of the
district in which they are raised, since any such taxes must be approved
by a majority of the property-taxpaying voters of the district.

The significance of the local property tax element of the Texas
financing scheme is apparent from the fact that it provides the funds to
meet some 40\% of the cost of public education for Texas as a whole Yet
the amount of revenue that any particular Texas district can raise is
dependent on two factors---its tax rate and its amount of taxable
property. The first factor is determined by the property-taxpaying
voters of the district But, regardless of the enthusiasm of the local
voters for public education, the second factor---the taxable property
wealth of the district---necessarily restricts the district's ability to
raise funds to support public education. 8 Thus, even though the voters
of two Texas districts may be willing to make the same tax effort, the
results for the districts will be substantially different if one is
property rich while the other is property poor. The necessary effect of
the Texas local property tax is, in short, to favor property-rich
districts and to disfavor property-poor ones.

The seriously disparate consequences of the Texas local property tax,
when that tax is considered alone, are amply illustrated by data
presented to the District Court by appellees. These data included a
detailed study of a sample of 110 Texas school districts9 for the
1967---1968 school year conducted by Professor Joel S. Berke of Syracuse
University's Educational Finance Policy Institute. Among other things,
this study revealed that the 10 richest districts examined, each of
which had more than \$100,000 in taxable property per pupil, raised
through local effort an average of \$610 per pupil, whereas the four
poorest districts studied, each of which had less than \$10,000 in
taxable property per pupil, were able to raise only an average of \$63
per pupil And, as the Court effectively recognizes, this correlation
between the amount of taxable property per pupil and the amount of local
revenues per pupil holds true for the 96 districts in between the
richest and poorest districts.

It is clear, moreover, that the disparity of per-pupil revenues cannot
be dismissed as the result of lack of local effort that is, lower tax
rates---by property-poor districts. To the contrary, the data presented
below indicate that the poorest districts tend to have the highest tax
rates and the richest districts tend to have the lowest tax rates Yet,
despite the apparent extra effort being made by the poorest districts,
they are unable even to begin to match the richest districts in terms of
the production of local revenues. For example, the 10 richest districts
studied by Professor Berke were able to produce \$585 per pupil with an
equalized tax rate of 31¢ on \$100 of equalized valuation, but the four
poorest districts studied, with an equalized rate of 70¢ on \$100 of
equalized valuation, were able to produce only \$60 per pupil. 13
Without more, this stateimposed system of educational funding presents a
serious picture of widely varying treatment of Texas school districts,
and thereby of Texas schoolchildren, in terms of the amount of funds
available for public education.

Nor are these funding variations corrected by the other aspects of the
Texas financing scheme. The Federal Government provides funds sufficient
to cover only some 10\% of the total cost of public education in Texas
Furthermore, while these federal funds are not distributed in Texas
solely on a per-pupil basis, appellants do not here contend that they
are used in such a way as to ameliorate signiticantly the widely varying
consequences for Texas school districts and schoolchildren of the local
property tax element of the state financing scheme.

State funds provide the remaining some 50\% of the monies spent on
public education in Texas Technically, they are distributed under two
programs. The first is the Available School Fund, for which provision is
made in the Texas Constitution The Available

School Fund is composed of revenues obtained from a number of sources,
including receipts from the state ad valorem property tax, one-fourth of
all monies collected by the occupation tax, annual contributions by the
legislature from general revenues, and the revenues derived from the
Permanent School Fund For the 1970---1971 school year the Available
School Fund contained \$296,000,000. The Texas Constitution requires
that this money be distributed annually on a per capita basis19 to the
local school districts. Obviously, such a flat grant could not alone
eradicate the funding differentials atrributable to the local property
tax. Moreover, today the Available School Fund is in reality simply one
facet of the second state financing program, the Minimum Foundation
School Program,20 since each district's annual share of the Fund is
deducted from the sum to which the district is entitled under the
Foundation Program.

The Minimum Foundation School Program provides funds for three specific
purposes: professional salaries, current operating expenses, and
transportation expenses The State pays, on an overall basis, for
approximately 80\% of the cost of the Program; the remaining 20\% is
distributed among the local school districts under the

Local Fund Assignment Each district's share of the Local Fund Assignment
is determined by a complex `economic index' which is designed to
allocate a larger share of the costs to property-rich districts than to
property-poor districts Each district pays its share with revenues
derived from local property taxation.

The stated purpose of the Minimum Foundation School Program is to
provide certain basic funding for each local Texas school district At
the same time, the Program was apparently intended to improve, to some
degree, the financial position of property-poor districts relative to
property-rich districts, since through the use of the economic
index---an effort is made to charge a disproportionate share of the
costs of the Program to rich districts. 26 It bears noting, however,
that substantial criticism has been leveled at the practical
effectiveness of the economic index system of local cost allocation In
theory, the index is designed to ascertain the relative ability of each
district to contribute to the Local Fund Assignment from local property
taxes. Yet the index is not developed simply on the basis of each
district's taxable wealth. It also takes into account the district's
relative income from manufacturing, mining, and agriculture, its
payrolls, and its scholastic population.

It is difficult to discern precisely how these latter factors are
predictive of a district's relative ability to raise revenues through
local property taxes. Thus, in 1966, one of the consultants who
originally participated in the development of the Texas economic index
adopted in 1949 told the Governor's Committee on Public School
Education: `The Economic Index approach to evaluating local ability
offers a little better measure than sheer chance, but not much.'

Moreover, even putting aside these criticisms of the economic index as a
device for achieving meaningful district wealth equalization through
cost allocation, poor districts still do not necessarily receive more
state aid than property-rich districts. For the standards which
currently determine the amount received from the Foundation School
Program by any particular district 30 favor property-rich districts
Thus, focusing on the same

Edgewood Independent and Alamo Heights School Districts which the
majority uses for purposes of illustration, we find that in 1967 1968
property-rich Alamo Heights,32 which raised \$333 per pupil on an
equalized tax rate of 85¢ per \$100 valuation, received \$225 per pupil
from the Foundation School Program, while property-poor Edgewood,33
which raised only \$26 per pupil with an equalized tax rate of \$1 per
\$100 valuation, received only \$222 per pupil from the Foundation
School Program And, more recent data, which indicate that for the
1970---1971 school year Alamo Heights received \$491 per pupil from the
Program while Edgewood received only \$356 per pupil, hardly suggest
that the wealth gap between the districts is being narrowed by the State
Program. To the contrary, whereas in 1967 1968 Alamo Heights received
only \$3 per pupil, or about 1\%, more than Edgewood in state aid, by
1970---1971 the gap had widened to a difference of \$135 per pupil, or
about 38\% It was data of this character that prompted the District
Court to observe that `the current (state aid) system tends to subsidize
the rich at the expense of the poor, rather than the other way
around.'36 337 F.Supp. 280, 282. And even the appellants go no further
here than to venture that the Minimum Foundation School Program has 'a
mildly equalizing effect.'

Despite these facts, the majority continually emphasized how much state
aid has, in recent years, been given to property-poor Texas school
districts. What the Court fails to emphasize is the cruel irony of how
much more state aid is being given to property-rich Texas school
districts on top of their already substantial local property tax
revenues Under any view, then, it is apparent that the state aid
provided by the Foundation School Program fails to compensate for the
large funding variations attributable to the local property tax element
of the Texas financing scheme. And it is these stark differences in the
treatment of Texas school districts and school children inherent in the
Texas financing schement, not the absolute amount of state aid provided
to any particular school district, that are the crux of this case. There
can, moreover, be no escaping the conclusion that the local property tax
which is dependent upon taxable district property wealth is an essential
feature of the Texas scheme for financing public education.

The appellants do not deny the disparities in educational funding caused
by variations in taxable district property wealth. They do contend,
however, that whatever the differences in per-pupil spending among Texas
districts, there are no discriminatory consequences for the children of
the disadvantaged districts. They recognize that what is at stake in
this case is the quality of the public education provided Texas children
in the districts in which they live. But appellants reject the
suggestion that the quality of education in any particular district is
determined by money beyond some minimal level of funding which they
believe to be assured every Texas district by the Minimum Foundation
School Program. In their view, there is simply no denial of equal
educational opportunity to any Texas school children as a result of the
widely varying per-pupil spending power provided districts under the
current financing scheme.

In my view, though, even an unadorned restatement of this contention is
sufficient to reveal its absurdity. Authorities concerned with
educational quality no doubt disagree as to the significance of
variations in per-pupil spending Indeed, conflicting expert testimony
was presented to the District Court in this case concerning the effect
of spending variations on educational achievement We sit, however, not
to resolve disputes over educational theory but to enforce our
Constitution. It is an inescapable fact that if one district has more
funds available per pupil than another district, the former will have
greater choice in educational planning than will the latter. In this
regard, I believe the question of discrimination in educational quality
must be deemed to be an objective one that looks to what the State
provides its children, not to what the children are able to do with what
they receive. That a child forced to attend an underfunded school with
poorer physical facilities, less experienced teachers, larger classes,
and a narrower range of courses than a school with substantially more
funds---and thus with greater choice in educational planning may
nevertheless excel is to the credit of the child, not the State,
cf.~Missouri ex rel. Gaines v. Canada, Indeed, who can ever measure for
such a child the opportuntiies lost and the talents wasted for want of a
broader, more enriched education? Discrimination in the opportunity to
learn that is afforded a child must be our standard.

Hence, even before this Court recognized its duty to tear down the
barriers of state-enforced racial segregation in public education, it
acknowledged that inequality in the educational facilities provided to
students may be discriminatory state action as contemplated by the Equal
Protection Clause. As a basis for striking down state-enforced
segregation of a law school, the Court in Sweatt v. Painter---634,
stated:

`(W)e cannot find substantial equality in the educational opportunities
offered white and Negro law students by the State. In terms of number of
the faculty, variety of courses and opportunity for specialization, size
of the student body, scope of the library, availability of law review
and similar activities, the (whites only) Law School is superior. . It
is difficult to believe that one who had a free choice between these law
schools would consider the question close.'

See also McLaurin v. Oklahoma State Regents for Higher
EducationLikewise, it is difficult to believe that if the children of
Texas had a free choice, they would choose to be educated in districts
with fewer resources, and hence with more antiquated plants, less
experienced teachers, and a less diversified curriculum. In fact, if
financing variations are so insignificant to educational quality, it is
difficult to understand why a number of our country's wealthiest school
districts, which have no legal obligation to argue in support of the
constitutionality of the Texas legislation, have nevertheless zealously
pursued its cause before this Court.

The consequences, in terms of objective educational input, of the
variations in district funding caused by the Texas financing scheme are
apparent from the data introduced before the District Court. For
example, in 1968\% of the teachers in the property-rich Alamo Heights
School District had college degrees By contrast, during the same school
year only 80 \% of the teachers had college degrees in the property poor
Edgewood Independent School District Also, in 1968---1969, approximately
47\% of the teachers in the Edgewood District were on emergency teaching
permits, whereas only 11\% of the teachers in Alamo Heights were on such
permits This is undoubtedly a reflection of the fact that the top of
Edgewood's teacher salary scale was approximately 80\% of Alamo Heights
And, not surprisingly, the teacher-student ratio varies significantly
between the two districts In other words, as might be expected, a
difference in the funds available to districts results in a difference
in educational inputs available for a child's public education in Texas.
For constitutional purposes, I believe this situation, which is directly
attributable to the Texas financing scheme, raises a grave question of
state-created discrimination in the provision of public education. Cf.
Gaston County v. United States,

At the very least, in view of the substantial interdistrict disparities
in funding and in resulting educational inputs shown by appellees to
exist under the Texas financing scheme, the burden of proving that these
disparities do not in fact affect the quality of children's education
must fall upon the appellants. Cf. Hobson v. Hansen, 327 F.Supp. 844,
860---861 (D.C.D.C ). Yet appellants made no effort in the District
Court to demonstrate that educational quality is not affected by
variations in funding and in resulting inputs. And, in this Court, they
have argued no more than that the relationship is ambiguous. This is
hardly sufficient to overcome appellees' prima facie showing of
state-created discrimination between the schoolchildren of Texas with
respect to objective educational opportunity.

Nor can I accept the appellants' apparent suggestion that the Texas
Minimum Foundation School Program effectively eradicates any
discriminatory effects otherwise resulting from the local property tax
element of the

Texas financing scheme. Appellants assert that, despite its
imperfections, the Program `does guarantee an adequate education to
every child.'48 The majority, in considering the constitutionality of
the Texas financing scheme, seems to find substantial merit in this
contention, for it tells us that the Foundation Program 'was designed to
provide an adequate minimum educational offering in every school in the
State,' and that the Program `assur(es) a basic education for every
child,' But I fail to understand how the constitutional problems
inherent in the financing scheme are eased by the Foundation Program.
Indeed, the precise thrust of the appellants' and the Court's remarks
are not altogether clear to me.

The suggestion may be that the state aid received via the Foundation
Program sufficiently improves the position of property-poor districts
vis-a-vis property-rich districts---in terms of educational funds---to
eliminate any claim of interdistrict discrimination in available
educational resources which might otherwise exist if educational funding
were dependent solely upon local property taxation. Certainly the Court
has recognized that to demand precise equality of treatment is normally
unrealistic, and thus minor differences inherent in any practical
context usually will not make out a substantial equal protection claim.
See, e.g., Mayer v. City of Chicago, ; Draper v. Washington, ; Bain
Peanut Co.~v. Pinson, But, as has already been seen, we are hardly
presented here with some de minimis claim of discrimination resulting
from the play necessary in any functioning system; to the contrary, it
is clear that the Foundation Program utterly fails to ameliorate the
seriously discriminatory effects of the local property tax.

Alternatively, the appellants and the majority may believe that the
Equal Protection Clause cannot be offended by substantially unequal
state treatment of persons who are similarly situated so long as the
State provides everyone with some unspecified amount of education which
evidently is 'enough.'50 The basis for such a novel view is far from
clear. It is, of course, true that the Constitution does not require
precise equality in the treatment of all persons. As Mr.~Justice
Frankfurter explained:

`The equality at which the 'equal protection' clause aims is not a
disembodied equality. The Fourteenth Amendment enjoins `the equal
protection of the laws', and laws are not abstract propositions. . The
Constitution does not require things which are different in fact or
opinion to be treated in law as though they were the same.' Tigner v.
Texas.

But this Court has never suggested that because some `adequate' level of
benefits is provided to all, discrimination in the provision of services
is therefore constitutionally excusable. The Equal Protection Clause is
not addressed to the minimal sufficiency but rather to the unjustifiable
inequalities of state action. It mandates nothing less than that `all
persons similarly circumstanced shall be treated alike.' F. S. Royster
Guano Co.~v. Virginia,

Even if the Equal Protection Clause encompassed some theory of
constitutional adequacy, discrimination in the provision of educational
opportunity would certainly seem to be a poor candidate for its
application. Neither the majority nor appellants inform us how
judicially manageable standards are to be derived for determining how
much education is `enough' to excuse constitutional discrimination. One
would think that the majority would heed its own fervent affirmation of
judicial self-restraint before undertaking the complex task of
determining at large what level of education is constitutionally
sufficient. Indeed, the majority's apparent reliance upon the adequacy
of the educational opportunity assured by the Texas Minimum Foundation
School Program seems fundamentally inconsistent with its own recognition
that educational authorities are unable to agree upon what makes for
educational quality, see ---43, and n.~86 and n.~101. If, as the
majority stresses, such authorities are uncertain as to the impact of
various levels of funding on educational quality, I fail to see where it
finds the expertise to divine that the particular levels of funding
provided by the Program assure an adequate educational
opportunity---much less an education substantially equivalent in quality
to that which a higher level of funding might provide. Certainly
appellants' mere assertion before this Court of the adequacy of the
education guaranteed by the Minimum

Foundation School Program cannot obscure the constitutional implications
of the discrimination in educational funding and objective educational
inputs resulting from the local property tax particularly since the
appellees offered substantial uncontroverted evidence before the
District Court impugning the now muchtouted `adequacy' of the education
guaranteed by the Foundation Program.

In my view, then, it is inequality---not some notion of gross
inadequacy---of educational opportunity that raises a question of denial
of equal protection of the laws. I find any other approach to the issue
unintelligible and without directing principle. Here, appellees have
made a substantial showing of wide variations in educational funding and
the resulting educational opportunity afforded to the schoolchildren of
Texas. This discrimination is, in large measure, attributable to
significant disparities in the taxable wealth of local Texas school
districts. This is a sufficient showing to raise a substantial question
of discriminatory state action in violation of the Equal Protection
Clause.

Despite the evident discriminatory effect of the Texas financing scheme,
both the appellants and the majority raise substantial questions
concerning the precise character of the disadvantaged class in this
case. The District Court concluded that the Texas financing scheme draws
`distinction betwen groups of citizens depending upon the wealth of the
district in which they live' and thus creates a disadvantaged class
composed of persons living in property-poor districts. See 337 F.Supp..
See also In light of the data introduced before the District Court, the
conclusion that the schoolchildren of property-poor districts constitute
a sufficient class for our purposes seems indisputable to me.

Appellants contend, however, that in constitutional terms this case
involves nothing more than discrimination against local school
districts, not against individuals, since on its face the state scheme
is concerned only with the provision of funds to local districts. The
result of the Texas financing scheme, appellants suggest, is merely that
some local districts have more available revenues for education; others
have less. In that respect, they point out, the States have broad
discretion in drawing reasonable distinctions between their political
subdivisions.

But this Court has consistently recognized that where there is in fact
discrimination against individual interests, the constitutional
guarantee of equal protection of the laws is not inapplicable simply
because the discrimination is based upon some group characteristic such
as geographic location. See Gordon v. Lance, ; Reynolds v. Sims; Gray v.
Sanders, Texas has chosen to provide free public education for all its
citizens, and it has embodied that decision in its constitution. 53 Yet,
having established public education for its citizens, the State, as a
direct consequence of the variations in local property wealth endemic to
Texas' financing scheme, has provided some Texas schoolchildren with
substantially less resources for their education than others. Thus,
while on its face the Texas scheme may merely discriminate between local
districts, the impact of that discrimination falls directly upon the
children whose educational opportunity is dependent upon where they
happen to live. Consequently, the District Court correctly concluded
that the Texas financing scheme discriminates, from a constitutional
perspective, between school children on the basis of the amount of
taxable property located within their local districts.

In my Brother STEWART's view, however, such a description of the
discrimination inherent in this case is apparently not sufficient, for
it fails to define the `kind of objectively identifiable classes' that
he evidentlyperceives to be necessary for a claim to be `cognizable
under the Equal Protection Clause,' He asserts that this is also the
view of the majority, but he is unable to cite, nor have I been able to
find, any portion of the Court's opinion which remotely suggests that
there is no objectively identifiable or definable class in this case. In
any event, if he means to suggest that an essential predicate to equal
protection analysis is the precise identification of the particular
individuals who compose the disadvantaged class, I fail to find the
source from which he derives such a requirement. Certainly such
precision is not analytically necessary. So long as the basis of the
discrimination is clearly identified, it is possible to test it against
the State's purpose for such discrimination---whatever the standard of
equal protection analysis employed This is clear from our decision only
last Term in Bullock v. Carterwhere the Court, in striking down Texas'
primary filing fees as violative of equal protection, found no
impediment to equal protection analysis in the fact that the members of
the disadvantaged class could not be readily identified. The Court
recognized that the filing-fee system tended `to deny some voters the
opportunity to vote for a candidate of their choosing; at the same time
it gives the affluent the power to place on the ballot their own names
or the names of persons they favor.' The

Court also recognized that `(t)his disparity in voting power based on
wealth cannot be described by reference to discrete and precisely
defined segments of the community as is typical of inequities challenged
under the Equal Protection Clause . ..' Nevertheless, it concluded that
`we would ignore reality were we not to recognize that this system falls
with unequal weight on voters . according to their economic status.' The
nature of the classification in Bullock was clear, although the precise
membership of the disadvantaged class was not. This was enough in
Bullock for purposes of equal protection analysis. It is enough here.

It may be, though, that my Brother STEWART is not in fact demanding
precise identification of the membership of the disadvantaged class for
purposes of equal protection analysis, but is merely unable to discern
with sufficient clarity the nature of the discrimination charged in this
case. Indeed, the Court itself displays some uncertainty as to the exact
nature of the discrimination and the resulting disadvantaged class
alleged to exist in this case. It is, of course, essential to equal
protection analysis to have a firm grasp upon the nature of the
discrimination at issue. In fact, the absence of such a clear,
articulable understanding of the nature of alleged discrimination in a
particular instance may well suggest the absence of any real
discrimination. But such is hardly the case here.

A number of theories of discrimination have, to be sure, been considered
in the course of this litigation. Thus, the District Court found that in
Texas the poor and minority group members tend to live in property-poor
districts, suggesting discrimination on the basis of both personal
wealth and race. The Court goes to great lengths to discredit the data
upon which the District Court relied, and thereby its conclusion that
poor people live in property-poor dis- tricts Although I have serious
doubts as to the correctness of the Court's analysis in rejecting the
data submitted below,56 I have no need to join issue on these factual
disputes.

I believe it is sufficient that the overarching form of discrimination
in this case is between the schoolchildren of Texas on the basis of the
taxable property wealth of the districts in which they happen to live.
To understand both the precise nature of this discrimination and the
parameters of the disadvantaged class it is sufficient to consider the
constitutional principle which appellees contend is controlling in the
context of educational financing. In their complaint appellees asserted
that the Constitution does not permit local district wealth to be
determinative of educational opportunity This is simply another way of
saying, as the District Court concluded, that consistent with the
guarantee of equal protection of the laws, `the quality of public
education may not be a function of wealth, other than the wealth of the
state as a whole.' 337 F.Supp.. Under such a principle, the children of
a district are excessively advantaged if that district has more taxable
property per pupil than the average amount of taxable property per pupil
considering the State as a whole. By contrast, the children of a
district are disadvantaged if that district has less taxable property
per pupil than the state average. The majority attempts to disparage
such a definition of the disadvantaged class as the product of an
`artificially defined level' of district wealth. But such is clearly not
the case, for this is the definition unmistakably dictated by the
constitutional principle for which appellees have argued throughout the
course of this litigation. And I do not believe that a clearer
definition of either the disadvantaged class of Texas schoolchildren or
the allegedly unconstitutional discrimination suffered by the members of
that class under the present Texas financing scheme could be asked for,
much less needed Whether this discrimination, against the schoolchildren
of property-poor districts, inherent in the Texas financing scheme, is
violative of the Equal Protection Clause is the question to which the
must now turn.

To avoid having the Texas financing scheme struck down because of the
interdistrict variations in taxable property wealth, the District Court
determined that it was insufficient for appellants to show merely that
the State's scheme was rationally related to some legitimate state
purpose; rather, the discrimination inherent in the scheme had to be
shown necessary to promote a `compelling state interest' in order to
withstand constitutional scrutiny. The basis for this determination was
twofold: first, the financing scheme divides citizens on a wealth basis,
a classification which the District Court viewed as highly suspect; and
second, the discriminatory scheme directly affects what it considered to
be a `fundamental interest,' namely, education.

This Court has repeatedly held that state discrimination which either
adversely affects a `fundamental interest,' see, e.g., Dunn v.
Blumstein, ; Shapiro v. Thompson, or is based on a distinction of a
suspect character, see, e.g., Graham v. Richardson; McLaughlin v.
Florida, must be carefully scrutinized to ensure that the scheme is
necessary to promote a substantial, legitimate state interest. See,
e.g., Dunn v. BlumsteinS.Ct.---1004; Shapiro v. ThompsonThe majority
today concludes, however, that the Texas scheme is not subject to such a
strict standard of review under the Equal Protection Clause. Instead, in
its view, the Texas scheme must be tested by nothing more than that
lenient standard of rationality which we have traditionally applied to
discriminatory state action in the context of economic and commercial
matters. See, e.g., McGowan v. MarylandS.Ct.---1105; Morey v. Doud; F.
S. Royster Guano Co.~v. Virginia, 40 S.Ct.; Lindsley v. Natural Carbonic
Gas Co., By so doing, the Court avoids the telling task of searching for
a substantial state interest which the Texas financing scheme, with its
variations in taxable district property wealth, is necessary to further.
I cannot accept such an emasculation of the Equal Protection Clause in
the context of this case. A.

To begin, I must once more voice my disagreement with the Court's
rigidified approach to equal protection analysis. See Dandridge v.
Williams, (dissenting opinion); Richardson v. Belcher, (dissenting
opinion). The Court apparently seeks to establish today that equal
protection cases fall into one of two neat categories which dictate the
appropriate standard of review---strict scrutiny or mere rationality.
But this Court's decisions in the field of equal protection defy such
easy categorization. A principled reading of what this Court has done
reveals that it has applied a spectrum of standards in reviewing
discrimination allegedly violative of the Equal Protec- tion Clause.
This spectrum clearly comprehends variations in the degree of care with
which the Court will scrutinize particular classifications, depending, I
believe, on the constitutional and societal importance of the interest
adversely affected and the recognized invidiousness of the basis upon
which the particular classification is drawn. I find in fact that many
of the Court's recent decisions embody the very sort of reasoned
approach to equal protection analysis for which I previously
argued---that is, an approach in which `concentration (is) placed upon
the character of the classification in question, the relative importance
to individuals in the class discriminated against of the governmental
benefits that they do not receive, and the asserted state interests in
support of the classification.' Dandridge v. Williams (dissenting
opinion).

I therefore cannot accept the majority's labored efforts to demonstrate
that fundamental interests, which call for strict scrutiny of the
challenged classification, encompass only established rights which we
are somehow bound to recognize from the text of the Constitution itself.
To be sure, some interests which the Court has deemed to be fundamental
for purposes of equal protection analysis are themselves
constitutionally protected rights. Thus, discrimination against the
guaranteed right of freedom of speech has called for strict judicial
scrutiny. See Police Dept. of City of Chicago v. MosleyFurther, every
citizen's right to travel interstate, although nowhere expressly
mentioned in the Constitution, has long been recognized as implicit in
the premises underlying that document: the right `was conceived from the
beginning to be a necessary concomitant of the stronger Union the
Constitution created.' United States v. Guest, See also Crandall v.
Nevada, 6 Wall. 35, 48, Consequently, the Court has required that a
state classification affecting theconstitutionally protected right to
travel must be `shown to be necessary to promote a compelling
governmental interest.' Shapiro v. Thompson, But it will not do to
suggest that the `answer' to whether an interest is fundamental for
purposes of equal protection analysis is always determined by whether
that interest `is a right . explicitly or implicitly guaranteed by the
Constitution,'

I would like to know where the Constitution guarantees the right to
procreate, Skinner v. Oklahoma ex rel. Williamson, or the right to vote
in state elections, e.g., Reynolds v. Simsor the right to an appeal from
a criminal conviction, e.g., Griffin v. IllinoisThese are instances in
which, due to the importance of the interests at stake, the Court has
displayed a strong concern with the existence of discriminatory state
treatment. But the Court has never said or indicated that these are
interests which independently enjoy fullblown constitutional protection.

Thus, in Buck v. Bellthe Court refused to recognize a substantive
constitutional guarantee of the right to procreate. Nevertheless, in
Skinner v. Oklahoma ex rel. Williamsonthe Court, without impugning the
continuing validity of Buck v. Bell, held that `strict scrutiny' of
state discrimination affecting procreation `is essential' for
`(m)arriage and procreation are fundamental to the very existence and
survival of the race.' Recently, in Roe v. Wade, the importance of
procreation has indeed been explained on the basis of its intimate
relationship with the constitutional right of privacy which we have
recognized. Yet the limited stature thereby accorded any `right' to
procreate is evident from the fact that at the same time the Court
reaffirmed its initial decision in Buck v. Bell. See Roe v. Wade.

Similarly, the right to vote in state elections has been recognized as a
`fundamental political right,' because the Court concluded very early
that it is `preservative of all rights.' Yick Wo v. Hopkins, ; see,
e.g., Reynolds v. SimsS.Ct.---1382. For this reason, `this Court has
made clear that a citizen has a constitutionally protected right to
participate in elections on an equal basis with other citizens in the
jurisdiction.' Dunn v. Blumstein, 92 S.Ct. (emphasis added). The final
source of such protection from inequality in the provision of the state
franchise is, of course, the Equal Protection Clause. Yet it is clear
that whatever degree of importance has been attached to the state
electoral process when unequally distributed, the right to vote in state
elections has itself never been accorded the statute of an independent
constitutional guarantee. 60 See Oregon v. Mitchell; Kramer v. Union
Free School District No.~15, ; Harper v. Virginia Board of Elections,

Finally, it is likewise `true that a State is not required by the
Federal Constitution to provide appellate courts or a right to appellate
review at all.' Griffin v. Illinois, Nevertheless, discrimination
adversely affecting access to an appellate process which a State has
chosen to provide has been considered to require close judicial
scrutiny. See, e.g., Griffin v. Illinois; Douglas v. California.

The majority is, of course, correct when it suggests that the process of
determining which interests are fundamental is a difficult one. But I do
not think the problem is insurmountable. And I certainly do not accept
the view that the process need necessarily degenerate into an
unprincipled, subjective `picking-and-choosing' between various
interests or that it must involve this Court in creating `substantive
constitutional rights in the name of guaranteeing equal protection of
the laws,' Although not all fundamental interests are constitutionally
guaranteed, the determination of which interests are fundamental should
be firmly rooted in the text of the Constitution. The task in every case
should be to determine the extent to which constitutionally guaranteed
rights are dependent on interests not mentioned in the Constitution. As
the nexus between the specific constitutional guarantee and the
nonconstitutional interest draws closer, the nonconstitutional interest
becomes more fundamental and the degree of judicial scrutiny applied
when the interest is infringed on a discriminatory basis must be
adjusted accordingly. Thus, it cannot be denied that interests such as
procreation, the exercise of the state franchise, and access to criminal
appellate processes are not fully guaranteed to the citizen by our
Constitution. But these interests have nonetheless been afforded special
judicial consideration in the face of discrimination because they are,
to some extent, interrelated with constitutional guarantees. Procreation
is now understood to be important because of its interaction with the
established constitutional right of privacy. The exercise of the state
franchise is closely tied to basic civil and political rights inherent
in the First Amendment. And access to criminal appellate processes
enhances the integrity of the range of rights62 implicit in the
Fourteenth Amendment guarantee of due process of law. Only if we closely
protect the related interests from state discrimination do we ultimately
ensure the integrity of the constitutional guarantee itself. This is the
real lesson that must be taken from our previous decisions involving
interests deemed to be fundamental.

The effect of the interaction of individual interests with established
constitutional guarantees upon the degree of care exercised by this
Court in reviewing state discrimination affecting such interests is
amply illustrated by our decision last Term in Eisenstadt v. BairdIn
Baird, the Court struck down as violative of the Equal Protection Clause
a state statute which denied unmarried persons access to contraceptive
devices on the same basis as married persons. The Court purported to
test the statute under its traditional standard whether there is some
rational basis for the discrimination effected. S.Ct.---1035. In the
context of commercial regulation, the Court has indicated that the Equal
Protection Clause `is offended only if the classification rests on
grounds wholly irrelevant to the achievement of the State's objective.'
See, e.g., McGowan v. Maryland, ; Kotch v. Board of River Port Pilot
Comm'rs, And this lenient standard is further weighted in the State's
favor by the fact that `(a) statutory discrimination will not be set
aside if any state of facts reasonably may be conceived (by the Court)
to justify it.' McGowan v. MarylandBut in Baird the Court clearly did
not adhere to these highly tolerant standards of traditional rational
review. For although there were conceivable state interests intended to
be advanced by the statute---e.g., deterrence of premarital sexual
activity and regulation of the dissemination of potentially dangerous
articles---the Court was not prepared to accept these interests on their
face, but instead proceeded to test their substantiality by independent
analysis. See 405 U.S.S.Ct.---1039. Such close scrutiny of the State's
interests was hardly characteristic of the deference shown state
classifications in the context of economic interests. See, e.g.,
Goesaert v. Cleary; Kotch v. Board of River Port Pilot Comm'rs. Yet I
think the Court's action was entirely appropriate, for access to and use
of contraceptives bears a close relationship to the individual's
constitutional right of privacy. See 405 U.S. 454; S.Ct. 1038---1039;
---1044 (White, J., concurring in result). See also Roe v.
WadeS.Ct.---727.

A similar process of analysis with respect to the invidiousness of the
basis on which a particular classification is drawn has also influenced
the Court as to the appropriate degree of scrutiny to to accorded any
particular case. The highly suspect character of classifications based
on race, nationality, or alienage is well established. The reasons why
such classifications call for close judicial scrutiny are manifold.
Certain racial and ethnic groups have frequently been recognized as
`discrete and insular minorities' who are relatively powerless to
protect their interests in the political process. See Graham v.
Richardson, 91 S.Ct.; United States v. Carolene Products Co.---153,
n.~4, 783---784, Moreover, race, nationality, or alienage is "in most
circumstances irrelevant' to any constitutionally acceptable legislative
purpose, Kiyoshi Hirabayashi v. United States, .' McLaughlin v. Florida,
Instead, lines drawn on such bases are frequently the reflection of
historic prejudices rather than legislative rationality. It may be that
all of these considerations, which make for particular judicial
solicitude in the face of discrimination on the basis of race,
nationality, or alienage, do not coalesce---or at least not to the same
degree---in other forms of discrimination. Nevertheless, these
considerations have undoubtedly influenced the care with which the Court
has scrutinized other forms of discrimination.

In James v. Strangethe Court held unconstitutional a state statute which
provided for recoupment from indigent convicts of legal defense fees
paid by the State. The Court found that the statute impermissibly
differentiated between indigent criminals in debt to the State and civil
judgment debtors, since criminal debtors were denied various protective
exemptions afforded civil judgment debtors The Court suggested that in
reviewing the statute under the Equal Protection Clause, it was merely
applying the traditional requirement that there be ``some rationality''
in the line drawn between the different types of debtors. Yet it then
proceeded to scrutinize the statute with less than traditional deference
and restraint. Thus, the Court recognized `that state recoupment
statutes may betoken legitimate state interests' in recovering expenses
and discouraging fraud. Nevertheless, Mr.~Justice Powell, speaking for
the Court, concluded that

`these interests are not thwarted by requiring more even treatment of
indigent criminal defendants with other classes of debtors to whom the
statute itself repeatedly makes reference. State recoupment laws,
notwithstanding the state interests they may serve, need not blight in
such discriminatory fashion the hopes of indigents for self sufficiency
and self respect.'

The Court, in short, clearly did not consider the problems of fraud and
collection that the state legislature might have concluded were peculiar
to indigent criminal defendants to be either sufficiently important or
at least sufficiently substantiated to justify denial of the protective
exemptions afforded to all civil judgment debtors, to a class composed
exclusively of indigent criminal debtors.

Similarly, in Reed v. Reed the Court, in striking down a state statute
which gave men preference over women when persons of equal entitlement
apply for assignment as an administrator of a particular estate,
resorted to a more stringent standard of equal protecting review than
that employed in cases involving commercial matters. The Court indicated
that it was testing the claim of sex discrimination by nothing more than
whether the line drawn bore `a rational relationship to a state
objective,' which it recognized as a legitimate effort to reduce the
work of probate courts in choosing between competing applications for
letters of administration. Accepting such a purpose, the Idaho Supreme
Court had thought the classification to be sustainable on the basis that
the legislature might have reasonably concluded that, as a rule, men
have more experience than women in business matters relevant to the
administration of an estate. This Court, however, concluded that '(t)o
give a mandatory preference to members of either sex over members of the
other, merely to accomplish the elimination of hearings on the merits,
is to make the very kind of arbitrary legislative choice forbidden by
the Equal Protection Clause of the Fourteenth Amendment. This Court, in
other words, was unwilling to consider a theoretical and unsubstantiated
basis for distinction---however reasonable it might appear---sufficient
to sustain a statute discriminating on the basis of sex.

James and Reed can only be understood as instances in which the
particularly invidious character of the classification caused the Court
to pause and scrutinize with more than traditional care the rationality
of state discrimination. Discrimination on the basis of past criminality
and on the basis of sex posed for the Court the spector of forms of
discrimination which it implicitly recognized to have deep social and
legal roots without necessarily having any basis in actual differences.
Still, the Court's sensitivity to the invidiousness of the basis for
discrimination is perhaps most apparent in its decisions protecting the
interests of children born out of wedlock from discriminatory state
action. See Weber v. Aetna Casualty \& Surety Co., 406 U.S., 164, ; Levy
v. Louisiana In Weber, the Court struck down a portion of a state
workmen's compensation statute that relegated unacknowledged
illegitimate children of the deceased to a lesser status with respect to
benefits than that occupied by legitimate children of the deceased. The
Court acknowledged the true nature of its inquiry in cases such as
these: `What legitimate state interest does the classification promote?
What fundamental personal rights might the classification endanger?'
U.S., Embarking upon a determination of the relative substantiality of
the State's justifications for the classification, the Court rejected
the contention that the classifications reflected what might be presumed
to have been the deceased's preference of beneficiaries as `not
compelling . where dependency on the deceased is a prerequisite to
anyone's recovery . ..' Likewise, it deemed the relationship between the
State's interest in encouraging legitimate family relationships and the
burden placed on the illegitimates too tenuous to permit the
classification to stand. A clear insight into the basis of the Court's
action is provided by its conclusion:

'(I)mposing disabilities on the illegitimate child is contrary to the
basic concept of our system that legal burdens should bear some
relationship to individual responsibility or wrongdoing. Obviously, no
child is responsible for his birth and penalizing the illegitimate child
is an ineffectual---as well as an unjust---way of deterring the parent.
Courts are powerless to prevent the social opprobrium suffered by these
hapless children, but the Equal Protection Clause does enable us to
strike down discriminatory laws relating to status of birth.

Status of birth, like the color of one's skin, is something which the
individual cannot control, and should generally be irrelevant in
legislative considerations. Yet illegitimacy has long been stigmatized
by our society. Hence, discrimination on the basis of
birth---particularly when it affects innocent children warrants special
judicial consideration.

In summary, it seems to me inescapably clear that this Court has
consistently adjusted the care with which it will review state
discrimination in light of the constitutional significance of the
interests affected and the invidiousness of the particular
classification. In the context of economic interests, we find that
discriminatory state action is almost always sustained, for such
interests are generally far removed from constitutional guarantees.
Moreover, `(t)he extremes to which the Court has gone in dreaming up
rational bases for state regulation in that area may in many instances
be ascribed to a healthy revulsion from the Court's earlier excesses in
using the Constitution to protect interests that have more than enough
power to protect themselves in the legislative halls.' Dandridge v.
Williams, (dissenting opinion). But the situation differs markedly when
discrimination against important individual interests with
constitutional implications and against particularly disadvantaged or
powerless classes is involved. The majority suggests, however, that a
variable standard of review would give this Court the appearance of a
`super-legislature.' I cannot agree. Such an approach seems to me a part
of the guarantees of our Constitution and of the historic experiences
with oppression of and discrimination against discrete, powerless
minorities which underlie that document. In truth, the Court itself will
be open to the criticism raised by the majority so long as it continues
on its present course of effectively selecting in private which cases
will be afforded special consideration without acknowledging the true
basis of its action.

Opinions such as those in Reed and James seem drawn more as efforts to
shield rather than to reveal the true basis of the Court's decisions.
Such obfuscated action may be appropriate to a political body such as a
legislature, but it is not appropriate to this Court. Open debate of the
bases for the Court's action is essential to the rationality and
consistency of our decisionmaking process. Only in this way can we avoid
the label of legislature and ensure the integrity of the judicial
process.

Nevertheless, the majority today attempts to force this case into the
same category for purposes of equal protection analysis as decisions
involving discrimination affecting commercial interests. By so doing,
the majority singles this case out for analytic treatment at odds with
what seems to me to be the clear trend of recent decisions in this
Court, and thereby ignores the constitutional importance of the interest
at stake and the invidiousness of the particular classification, factors
that call for far more than the lenient scrutiny of the Texas financing
scheme which the majority pursues. Yet if the discrimination inherent in
the Texas scheme is scrutinized with the care demanded by the interest
and classification present in this case, the unconstitutionality of that
scheme is unmistakable.

Since the Court now suggests that only interests guaranteed by the
Constitution are fundamental for purposes of equal protection analysis,
and since it rejects the contention that public education is
fundamental, it follows that the Court concludes that public education
is not constitutionally guaranteed. It is true that this Court has never
deemed the provision of free public education to be required by the
Constitution. Indeed, it has on occasion suggested that state-supported
education is a privilege bestowed by a State on its citizens. See
Missouri ex rel. Gaines v. Canada, Nevertheless, the fundamental
importance of education is amply indicated by the prior decisions of
this Court, by the unique status accorded public education by our
society, and by the close relationship between education and some of our
most basic constitutional values.

The special concern of this Court with the educational process of our
country is a matter of common knowledge. Undoubtedly, this Court's most
famous statement on the subject is that contained in Brown v. Board of
Education, 74 S.Ct.:

`Today, education is perhaps the most important function of state and
local governments. Compulsory school attendance laws and the great
expenditures for education both demonstrate our recognition of the
importance of education to our democratic society. It is required in the
performance of our most basic public responsibilities, even service in
the armed forces. It is the very foundation of good citizenship. Today
it is a principal instrument in awakening the child to cultural values,
in preparing him for later professional training, and in helping him to
adjust normally to his environment. . .'

Only last Term, the Court recognized that `(p)roviding public schools
ranks at the very apex of the function of a State.' Wisconsin v. Yoder,
This is clearly borne out by the fact that in 48 of our 50 States the
provision of public education is mandated by the state constitution No
other state function is so uniformly recognized69 as an essential
element of our society's well-being. In large measure, the explanation
for the special importance attached to education must rest, as the Court
recognized in Yoder, on the facts that `some degree of education is
necessary to prepare citizens to participate effectively and
intelligently in our open political system . .,' and that `education
prepares individuals to be self-reliant and self-sufficient participants
in society.' Both facets of this observation are suggestive of the
substantial relationship which education bears to guarantees of our
Constitution.

Education directly affects the ability of a child to exercise his First
Amendment rights, both as a source and as a receiver of information and
ideas, whatever interests he may pursue in life. This Court's decision
in Sweezy v. New Hampshire, speaks of the right of students `to inquire,
to study and to evaluate, to gain new maturity and understanding . .'
Thus, we have not casually described the classroom as the ``marketplace
of ideas.'' Keyishian v. Board of Regents, The opportunity for formal
education may not necessarily be the essential determinant of an
individual's ability to enjoy throughout his life the rights of free
speech andassociation guaranteed to him by the First Amendment. But such
an opportunity may enhance the individual's enjoyment of those rights,
not only during but also following school attendance. Thus, in the final
analysis, `the pivotal position of education to success in American
society and its essential role in opening up to the individual the
central experiences of our culture lend it an importance that is
undeniable.'

Of particular importance is the relationship between education and the
political process. `Americans regard the public schools as a most vital
civic institution for the preservation of a democratic system of
government.' School District of Abington Township v. Schempp, (Brennan,
J., concurring). Education serves the essential function of instilling
in our young an understanding of and appreciation for the principles and
operation of our governmental processes. 71 Education may instill the
interest and provide the tools necessary for political discourse and
debate. Indeed, it has frequently been suggested that education is the
dominant factor affecting political consciousness and participation A
system of `(c)ompetition in ideas andgovernmental policies is at the
core of our electoral process and of the First Amendment freedoms.'
Williams v. Rhodes, But of most immediate and direct concern must be the
demonstrated effect of education on the exercise of the franchise by the
electorate. The right to vote in federal elections is conferred by Art.
I, § 2, and the Seventeenth Amendment of the Constitution, and access to
the state franchise has been afforded special protection because it is
`preservative of other basic civil and political rights,' Reynolds v.
Sims, Data from the Presidential Election of 1968 clearly demonstrate a
direct relationship between participation in the electoral process and
level of educational attainment and, as this Court recognized in Gaston
County v. United States, the quality of education offered may influence
a child's decision to `enter or remain in school.' It is this very sort
of intimate relationship between a particular personal interest and
specific constitutional guarantees that has heretofore caused the Court
to attach special significance, for purposes of equal protection
analysis, to individual interests such as procreation and the exercise
of the state franchise.

While ultimately disputing little of this, the majority seeks refuge in
the fact that the Court has `never presumed to possess either the
ability or the authority to guarantee to the citizenry the most
effective speech or the most informed electoral choice.' Ante. This
serves only to blur what is in fact at stake. With due respect, the
issue is neither provision of the most effective speech nor of the most
informed vote. Appellees do not now seek the best education Texas might
provide. They do seek, however, an end to state discrimination resulting
from the unequal distribution of taxable district property wealth that
directly impairs the ability of some districts to provide the same
educational opportunity that other districts can provide with the same
or even substantially less tax effort. The issue is, in other words, one
of discrimination that affects the quality of the education which Texas
has chosen to provide its children; and, the precise question here is
what importance should attach to education for purposes of equal
protection analysis of that discrimination. As this Court held in Brown
v. Board of Education, the opportunity of education, `where the state
has undertaken to provide it, is a right which must be made available to
all on equal terms.' The factors just considered, including the
relationship between education and the social and political interests
enshrined within the Constitution, compel us to recognize the
fundamentality of education and to scrutinize with appropriate care the
bases for state discrimination affecting equality of educational
opportunity in Texas' school districts75---aconclusion which is only
strengthened when we consider the character of the classification in
this case.

The District Court found that in discriminating between Texas
schoolchildren on the basis of the amount of taxable property wealth
located in the district in which they live, the Texas financing scheme
created a form of wealth discrimination. This Court has frequently
recognized that discrimination on the basis of wealth may create a
classification of a suspect character and thereby call for exacting
judicial scrutiny. See, e.g., Griffin v. Illinois; Douglas v.
California; McDonald v. Board of Election Comm'rs of Chicago, The
majority, however, considers any wealth classification in this case to
lack certain essential characteristics which it contends are common to
the instances of wealth discrimination that this Court has heretofore
recognized. We are told that in every prior case involving a wealth
classification, the members of the disadvantaged class have `shared two
distinguishing characteristics: because of their impecunity they were
completely unable to pay for some desired benefit, and as a consequence,
they sustained an absolute deprivation of a meaningful opportunity to
enjoy that benefit.' I cannot agree. The Court's distinctions may be
sufficient to explain the decisions in Williams v. Illinois; Tate v.
Short; and even Bullock v. CarterBut they are not in fact consistent
with the decisions in Harper v. Virginia Board of Electionsor Griffin v.
Illinoisor Douglas v. California.

In Harper, the Court struck down as violative of the Equal Protection
Clause an annual Virginia poll tax of \$1 payment of which by persons
over the age of 21 was a prerequisite to voting in Virginia elections.
In part, the Court relied on the fact that the poll tax interfered with
a fundamental interest---the exercise of the state franchise. In
addition, though, the Court emphasized that `(l)ines drawn on the basis
of wealth or property . are traditionally disfavored.' 383 U.S., Under
the first part of the theory announced by the majority, the
disadvantaged class in Harper, in terms of a wealth analysis, should
have consisted only of those too poor to afford the \$1 necessary to
vote. But the Harper Court did not see it that way. In its view, the
Equal Protection Clause `bars a system which excludes (from the
franchise) those unable to pay a fee to vote or who fail to pay.'
(Emphasis added.) So far as the Court was concerned, the `degree of the
discrimination (was) irrelevant.' Thus, the Court struck down the poll
tax in toto; it did not order merely that those too poor to pay the tax
be exempted; complete impecunity clearly was not determinative of the
limits of the disadvantaged class, nor was it essential to make an equal
protection claim.

Similarly, Griffin and Douglas refute the majority's contention that we
have in the past required an absolute deprivation before subjecting
wealth classifications to strict scrutiny. The Court characterizes
Griffin as a case concerned simply with the denial of a transcript or an
adequate substitute therefor, and Douglas as involving the denial
counsel. But in both cases the question was in fact whether `a State
that (grants) appellate review can do so in a way that discriminates
against some convicted defendants on account of their proverty.' Griffin
v. Illinois S.Ct. (emphasis added). In that regard, the Court concluded
that inability to purchase a transcript denies `the poor an adequate
appellate review accorded to all who have money enough to pay the costs
in advance,' (emphasis added), and that `the type of an appeal a person
is afforded . hinges upon whether or not he can pay for the assistance
of counsel,' Douglas v. CaliforniaS.Ct. (emphasis added). The right of
appeal itself was not absolutely denied to those too poor to pay; but
because of the cost of a transcript and of counsel, the appeal was a
substantially less meaningful right for the poor than for the rich It
was on these terms that the Court a denial of equal protection, and
those terms clearly encompassed degrees of discrimination on the basis
of wealth which do not amount to outright denial of the affected right
or interest.

This is not to say that the form of wealth classification in this case
does not differ significantly from those recognized in the previous
decisions of this Court. Our prior cases have dealt essentially with
discrimination on the basis of personal wealth Here, by contrast, the
children of the disadvantaged Texas school districts are being
discriminated against not necessarily because of their personal wealth
or the wealth of their families, but because of the taxable property
wealth of the residents of the district in which they happen to live.
The appropriate question, then, is whether the same degree of judicial
solicitude and scrutiny that has previously been afforded wealth
classifications is warranted here.

As the Court points out, ---29, no previous decision has deemed the
presence of just a wealth classification to be sufficient basis to call
forth rigorous judicial scrutiny of allegedly discriminatory state
action. Compare, e.g., Harper v. Virginia Board of Electionswith, e.g.,
James v. ValtierraThat wealth classifications alone have not necessarily
been considered to bear the same high degree of suspectness as have
classifications based on, for instance, race or alienage may be
explainable on a number of grounds. The `poor' may not be seen as
politically powerless as certain discrete and insular minority groups.
79 Personal proverty may entail much the same social stigma as
historically attached to certain racial or ethnic groups But personal
poverty is not a permanent disability; its shackles may be escaped.
Perhaps most importantly, though, personal wealth may not necessarily
share the general irrelevance as a basis for legislative action that
race or nationality is recognized to have. While the `poor' have
frequently been a legally disadvantaged group,81 it cannot be ignored
that social legislation must frequently take cognizance of the economic
status of our citizens. Thus, we have generally gauged the invidiousness
of wealth classifications with an awareness of the importance of the
interests being affected and the relevance of personal wealth to those
interests. See Harper v. Virginia Board of Elections.

When evaluated with these considerations in mind, it seems to me that
discrimination on the basis of group wealth in this case likewise calls
for careful judicial scrutiny. First, it must be recognized that while
local district wealth may serve other interests,82 it bears no
relationship whatsoever to the interest of Texas schoolchildren in the
educational opportunity afforded them by the State of Texas. Given the
importance of that interest, we must be particularly sensitive to the
invidious characteristics of any form of discrimination that is not
clearly intended to serve it, as opposed to some other distinct state
interest. Discrimination on the basis of group wealth may not, to be
sure, reflect the social stigma frequently attached to personal poverty.
Nevertheless, insofar as group wealth discrimination involves wealth
over which the disadvantaged individual has no significant control,83 it
represents in fact a more serious basis of discrimination than does
personal wealth. For such discrimination is no reflection of the
individual's characteristics or his abilities. And thus---particularly
in the context of a disadvantaged class composed of children---we have
previously treated discrimination on a basis which the individual cannot
control as constitutionally disfavored. Cf. Weber v. Aetna Casualty \&
Surety Co.; Levy v. Louisiana The disability of the disadvantaged class
in this case extends as well into the political processes upon which we
ordinarily rely as adequate for the protection and promotion of all
interests. Here legislative reallocation of the State's property wealth
must be sought in the face of inevitable opposition from significantly
advantaged districts that have a strong vested interest in the
preservation of the status quo, a problem not completely dissimilar to
that faced by underrepresented districts prior to the Court's
intervention in the process of reapportionment.

Nor can we ignore the extent to which, in contrast to our prior
decisions, the State is responsible for the wealth discrimination in
this instance. Griffin, Douglas, Williams, Tate, and our other prior
cases have dealt with discrimination on the basis of indigency which was
attributable to the operation of the private sector. But we have no such
simple de facto wealth discrimination here. The means for financing
public education in Texas are selected and specified by the State. It is
the State that has created local school districts, and tied educational
funding to the local property tax and thereby to local district wealth.
At the same time, governmentally imposed land use controls have
undoubtedly encouraged and rigidified natural trends in the allocation
of particular areas for residential or commercial use,85 and thus
determined each district's amount of taxable property wealth. In short,
this case, in contrast to the Court's previous wealth discrimination
decisions, can only be seen as `unusual in the extent to which
governmental action is the cause of the wealth classifications.'

In the final anaylsis, then The invidious characteristics of the group
wealth classification present in this case merely serve to emphasize the
need for careful judicial scrutiny of the State's justifications for the
resulting interdistrict discrimination in the educational opportunity
afforded to the schoolchildren of Texas.

The nature of our inquiry into the justifications for state
discrimination is essentially the same in all equal protection cases: We
must consider the substantiality of the state interests sought to be
served, and we must scrutinize the reasonableness of the means by which
the State has sought to advance its interests. See Police Dept. of City
of Chicago v. Mosley, Differences in the application of this test are,
in my view, a function of the constitutional importance of the interests
at stake and the invidiousness of the particular classification. In
terms of the asserted state interests, the Court has indicated that it
will require, for instance, a `compelling,' Shapiro v. Thompson, or a
`substantial' or `important,' Dunn v. Blumstein, state interest to
justify discrimination affecting individual interests of constitutional
significance. Whatever the differences, if any, in these descriptions of
the character of the state interest necessary to sustain such
discrimination, basic to each is, I believe, a concern with the
legitimacy and the reality of the asserted state interests. Thus, when
interests of constitutional importance are at stake, the Court does not
stand ready to credit the State's classification with any conceivable
legitimate purpose,87 but demands a clear showing that there are
legitimate state interests which the classification was in fact intended
to serve. Beyond the question of the adequacy of the State's purpose for
the classification, the Court traditionally has become increasingly
sensitive to the means by which a State chooses at act as its action
affects more directly interests of constitutional significance. See,
e.g., United States v. Robel, ; Shelton v. Tucker, Thus, by now, `less
restrictive alternatives' analysis is firmly established in equal
protection jurisprudence. See Dunn v. Blumstein S.Ct.; Kramer v. Union
Free School District No.~15, It seems to me that the range of choice we
are willing to accord the State in selecting the means by which it will
act, and the care with which we scrutinize the effectiveness of the
means which the State selects, also must reflect the constitutional
importance of the interest affected and the invidiousness of the
particular classification. Here, both the nature of the interest and the
classification dictate close judicial scrutiny of the purposes which
Texas seeks to serve with its present educational financing scheme and
of the means it has selected to serve that purpose.

The only justification offered by appellants to sustain the
discrimination in educational opportunity caused by the Texas financing
scheme is local educational control. Presented with this justification,
the District Court concluded that `(n)ot only are defendants unable to
demonstrate compelling state interests for their classifications based
upon wealth, they fail even to establish a reasonable basis for these
classifications.' 337 F.Supp.. I must agree with this conclusion.

At the outset, I do not question that local control of public education,
as an abstract matter, constitutes a very substantial state interest. We
observed only last Term that `(d)irect control over decisions vitally
affecting the education of one's children is a need that is strongly
felt in our society.' Wright v. Council of the City of Emporia. The
State's interest in local educational control---which certainly includes
questions of educational funding---has deep roots in the inherent
benefits of community support for public education. Consequently, true
state dedication to local control would present, I think, a substantial
justification to weigh against simply interdistrict variations in the
treatment of a State's schoolchildren. But I need not now decide how I
might ultimately strike the balance were we confronted with a situation
where the State's sincere concern for local control inevitably produced
educational inequality. For, on this record, it is apparent that the
State's purported concern with local control is offered primarily as an
excuse rather than as a justification for interdistrict inequality.

In Texas, statewide laws regulate in fact the most minute details of
local public education. For example, the State prescribes required
courses All textbooks must be submitted for state approval,89 and only
approved textbooks may be used The State has established the
qualifications necessary for teaching in Texas public schools and the
procedures for obtaining certification The State has even legislated on
the length of the school day Texas' own courts have said:

`As a result of the acts of the Legislature our school system is not of
mere local concern but it is statewide. While a school district is local
in territorial limits, it is an integral part of the vast school system
which is coextensive with the confines of the State of Texas.' Treadaway
v. Whitney Independent School District, 205 S.W d 97, 99 (Tex.Civ.App ).

Moreover, even if we accept Texas' general dedication to local control
in educational matters, it is difficult to find any evidence of such
dedication with respect to fiscal matters. It ignores reality to
suggest---as the Court does, that the local property tax element of the
Texas financing scheme reflects a conscious legislative effort to
provide school districts with local fiscal control. If Texas had a
system truly dedicated to local fiscal control, one would expect the
quality of the educational opportunity provided in each district to vary
with the decision of the voters in that district as to the level of
sacrifice they wish to make for public education. In fact, the Texas
scheme produces precisely the opposite result. Local school districts
cannot choose to have the best education in the State by imposing the
highest tax rate. Instead, the quality of the educational opportunity
offered by any particular district is largely determined by the amount
of taxable property located in the district---a factor over which local
voters can exercise no control.

The study introduced in the District Court showed a direct inverse
relationship between equalized taxable district property wealth and
district tax effort with the result that the property-poor districts
making the highest tax effort obtained the lowest per-pupil yield The
implications of this situation for local choice are illustrated by again
comparing the Edgewood and Alamo Heights School Districts. In
1967---1968, Edgewood, after contributing its share to the Local Fund
Assignment, raised only \$26 per pupil through its local property tax,
whereas Alamo Heights was able to raise \$333 per pupil. Since the funds
received through the Minimum Foundation School Program are to be used
only for minimum professional salaries, transportation costs, and
operating expenses, it is not hard to see the lack of local choice with
respect to higher teacher salaries to attract more and better teachers,
physical facilities, library books, and facilities, special courses, or
participation in special state and federal matching funds
programs---under which a property-poor district such as Edgewood is
forced to labor In fact, because of the difference in taxable local
property wealth, Edgewood would have to tax itself almost nine times as
heavily to obtain the same yield as Alamo Heights At present, then,
local control is a myth for many of the local school districts in Texas.
As one district court has observed, `rather than reposing in each school
district the economic power to fix its own level of per pupil
expenditure, the State has so arranged the structure as to guarantee
that some districts will spend low (with high taxes) while others will
spend high (with low taxes).' Van Dusartz v. Hatfield, 334 F.Supp. 870,
876 (D.C.Minn ).

In my judgment, any substantial degree of scrutiny of the operation of
the Texas financing scheme reveals that the State has selected means
wholly inappropriate to secure its purported interest in assuring its
school districts local fiscal control At the same time, appellees have
pointed out a variety of alternative financing schemes which may serve
the State's purported interest in local control as well as, if not
better than, the present scheme without the current impairment of the
educational opportunity of vast numbers of Texas schoolchildren I see no
need, however, to explore the practical or constitutional merits of
those suggested alternatives at this time for, whatever their positive
or negative features, experience with the present financing scheme
impugns any suggestion that it constitutes a serious effort to provide
local fiscal control. If for the sake of local education control, this
Court is to sustain interdistrict discrimination in the educational
opportunity afforded Texas school children, it should require that the
State present something more than the mere sham now before us.

In conclusion, it is essential to recognize that an end to the wide
variations in taxable district property wealth inherent in the Texas
financing scheme would entail none of the untoward consequences
suggested by the Court or by the appellants.

First, affirmance of the District Court's decisions would hardly sound
the death knell for local control of education. It would mean neither
centralized decisionmaking nor federal court intervention in the
operation of public schools. Clearly, this suit has nothing to do with
local decisionmaking with respect to educational policy or even
educational spending. It involves only a narrow aspect of local
control---namely, local control over the raising of educational funds.
In fact, in striking down interdistrict disparities in taxable local
wealth, the District Court took the course which is most likely to make
true local control over educational decision-making a reality for all
Texas school districts.

Nor does the District Court's decision even necessarily eliminate local
control of educational funding. The District Court struck down nothing
more than the continued interdistrict wealth discrimination inherent in
the present property tax. Both centralized and decentralized plans for
educational funding not involving such interdistrict discrimination have
been put forward The choice among these or other alternatives would
remain with the State, not with the federal courts. In this regard, it
should be evident that the degree of federal intervention in matters of
local concern would be substantially less in this context than in
previous decisions in which we have been asked effectively to impose a
particular scheme upon the States under the guise of the Equal
Protection Clause. See, e.g., Dandridge v. Williams; Cf. Richardson v.
Belcher Still, we are told that this case requires us `to condemn the
State's judgment in conferring on political subdivisions the power to
tax local property to supply revenues for local interests.' Yet no one
in the course of this entire litigation has ever questioned the
constitutionality of the local property tax as a device for raising
educational funds. The District Court's decision, at most, restricts the
power of the State to make educational funding dependent exclusively
upon local property taxation so long as there exists interdistrict
disparities in taxable property wealth. But it hardly eliminates the
local property tax as a source of educational funding or as a means of
providing local fiscal control.

The Court seeks solace for its action today in the possibility of
legislative reform. The Court's suggestions of legislative redress and
experimentation will doubtless be of great comfort to the schoolchildren
of Texas' disadvantaged districts, but considering the vested interests
of wealthy school districts in the preservation of the status quo, they
are worth little more. The possibility of legislative action is, in all
events, no answer to this Court's duty under the Constitution to
eliminate unjustified state discrimination. In this case we have been
presented with an instance of such discrimination, in a particularly
invidious form, against an individual interest of large constitutional
and practical importance. To support the demonstrated discrimination in
the provision of educational opportunity the State has offered a
justification which, on analysis, takes on at best an ephemeral
character. Thus, I believe that the wide disparities in taxable district
property wealth inherent in the local property tax element of the Texas
financing scheme render that scheme violative of the Equal Protection
Clause.

I would therefore affirm the judgment of the District Court.

\hypertarget{fisher-v.-univ.-of-tex.-at-austin}{%
\subsubsection{Fisher v. Univ. of Tex. at
Austin}\label{fisher-v.-univ.-of-tex.-at-austin}}

579 U.S. \_\_\_ (2016)

\textbf{Justice KENNEDY delivered the opinion of the Court.} The Court
is asked once again to consider whether the race-conscious admissions
program at the University of Texas is lawful under the Equal Protection
Clause.

The University of Texas at Austin (or University) relies upon a complex
system of admissions that has undergone significant evolution over the
past two decades. Until 1996, the University made its admissions
decisions primarily based on a measure called ``Academic Index'' (or
AI), which it calculated by combining an applicant's SAT score and
academic performance in high school. In assessing applicants, preference
was given to racial minorities.

In 1996, the Court of Appeals for the Fifth Circuit invalidated this
admissions system, holding that any consideration of race in college
admissions violates the Equal Protection Clause. See Hopwood v. Texas,
--935, 948.

One year later the University adopted a new admissions policy. Instead
of considering race, the University began making admissions decisions
based on an applicant's AI and his or her ``Personal Achievement Index''
(PAI). The PAI was a numerical score based on a holistic review of an
application. Included in the number were the applicant's essays,
leadership and work experience, extracurricular activities, community
service, and other ``special characteristics'' that might give the
admissions committee insight into a student's background. Consistent
with Hopwood, race was not a consideration in calculating an applicant's
AI or PAI.

The Texas Legislature responded to Hopwood as well. It enacted H.B. 588,
commonly known as the Top Ten Percent Law. Tex. Educ.Code Ann. § 51
(West Cum. Supp. 2015). As its name suggests, the Top Ten Percent Law
guarantees college admission to students who graduate from a Texas high
school in the top 10 percent of their class. Those students may choose
to attend any of the public universities in the State.

The University implemented the Top Ten Percent Law in 1998. After first
admitting any student who qualified for admission under that law, the
University filled the remainder of its incoming freshman class using a
combination of an applicant's AI and PAI scores---again, without
considering race.

The University used this admissions system until 2003, when this Court
decided the companion cases of Grutter v. Bollingerand Gratz v.
BollingerIn Gratz, this Court struck down the University of Michigan's
undergraduate system of admissions, which at the time allocated
predetermined points to racial minority candidates. See 539 U.S.,
275--276, In Grutter, however, the Court upheld the University of
Michigan Law School's system of holistic review---a system that did not
mechanically assign points but rather treated race as a relevant feature
within the broader context of a candidate's application. See 539 U.S.,
343--344, In upholding this nuanced use of race, Grutter implicitly
overruled Hopwood 's categorical prohibition.

In the wake of Grutter, the University embarked upon a year-long study
seeking to ascertain whether its admissions policy was allowing it to
provide ``the educational benefits of a diverse student body \ldots{} to
all of the University's undergraduate students.'' App. 481a--482a
(affidavit of N. Bruce Walker ¶ 11 (Walker Aff.)); see also a--447a. The
University concluded that its admissions policy was not providing these
benefits. Supp. App. 24a--25a.

To change its system, the University submitted a proposal to the Board
of Regents that requested permission to begin taking race into
consideration as one of ``the many ways in which {[}an{]} academically
qualified individual might contribute to, and benefit from, the rich,
diverse, and challenging educational environment of the University.'' a.
After the board approved the proposal, the University adopted a new
admissions policy to implement it. The University has continued to use
that admissions policy to this day.

Although the University's new admissions policy was a direct result of
Grutter, it is not identical to the policy this Court approved in that
case. Instead, consistent with the State's legislative directive, the
University continues to fill a significant majority of its class through
the Top Ten Percent Plan (or Plan). Today, up to 75 percent of the
places in the freshman class are filled through the Plan. As a practical
matter, this 75 percent cap, which has now been fixed by statute, means
that, while the Plan continues to be referenced as a ``Top Ten Percent
Plan,'' a student actually needs to finish in the top seven or eight
percent of his or her class in order to be admitted under this category.

The University did adopt an approach similar to the one in Grutter for
the remaining 25 percent or so of the incoming class. This portion of
the class continues to be admitted based on a combination of their AI
and PAI scores. Now, however, race is given weight as a subfactor within
the PAI. The PAI is a number from 1 to 6 (6 is the best) that is based
on two primary components. The first component is the average score a
reader gives the applicant on two required essays. The second component
is a full-file review that results in another 1--to--6 score, the
``Personal Achievement Score'' or PAS. The PAS is determined by a
separate reader, who (1) rereads the applicant's required essays, (2)
reviews any supplemental information the applicant submits (letters of
recommendation, resumes, an additional optional essay, writing samples,
artwork, etc.), and (3) evaluates the applicant's potential
contributions to the University's student body based on the applicant's
leadership experience, extracurricular activities, awards/honors,
community service, and other ``special circumstances.''

``Special circumstances'' include the socioeconomic status of the
applicant's family, the socioeconomic status of the applicant's school,
the applicant's family responsibilities, whether the applicant lives in
a single-parent home, the applicant's SAT score in relation to the
average SAT score at the applicant's school, the language spoken at the
applicant's home, and, finally, the applicant's race. See App.
218a--220a, 430a.

Both the essay readers and the full-file readers who assign applicants
their PAI undergo extensive training to ensure that they are scoring
applicants consistently. Deposition of Brian Breman 9--14, Record in
No.~1: 08--CV--00263, (WD Tex.), Doc. 96--3. The Admissions Office also
undertakes regular ``reliability analyses'' to ``measure the frequency
of readers scoring within one point of each other.'' App. 474a
(affidavit of Gary M. Lavergne ¶ 8); see also a (deposition of Kedra
Ishop (Ishop Dep.)). Both the intensive training and the reliability
analyses aim to ensure that similarly situated applicants are being
treated identically regardless of which admissions officer reads the
file.Once the essay and full-file readers have calculated each
applicant's AI and PAI scores, admissions officers from each school
within the University set a cutoff PAI/AI score combination for
admission, and then admit all of the applicants who are above that
cutoff point. In setting the cutoff, those admissions officers only know
how many applicants received a given PAI/AI score combination. They do
not know what factors went into calculating those applicants' scores.
The admissions officers who make the final decision as to whether a
particular applicant will be admitted make that decision without knowing
the applicant's race. Race enters the admissions process, then, at one
stage and one stage only---the calculation of the PAS.

Therefore, although admissions officers can consider race as a positive
feature of a minority student's application, there is no dispute that
race is but a ``factor of a factor of a factor'' in the holistic-review
calculus. 645 F.Supp d 587, 608 (W.D.Tex ). Furthermore, consideration
of race is contextual and does not operate as a mechanical plus factor
for underrepresented minorities. (``Plaintiffs cite no evidence to show
racial groups other than African--Americans and Hispanics are excluded
from benefitting from UT's consideration of race in admissions. As the
Defendants point out, the consideration of race, within the full context
of the entire application, may be beneficial to any UT Austin
applicant---including whites and Asian--Americans''); see also Brief for
Asian American Legal Defense and Education Fund et al.~as Amici Curiae
12 (the contention that the University discriminates against
Asian--Americans is ``entirely unsupported by evidence in the record or
empirical data''). There is also no dispute, however, that race, when
considered in conjunction with other aspects of an applicant's
background, can alter an applicant's PAS score. Thus, race, in this
indirect fashion, considered with all of the other factors that make up
an applicant's AI and PAI scores, can make a difference to whether an
application is accepted or rejected.

Petitioner Abigail Fisher applied for admission to the University's 2008
freshman class. She was not in the top 10 percent of her high school
class, so she was evaluated for admission through holistic, full-file
review. Petitioner's application was rejected.

Petitioner then filed suit alleging that the University's consideration
of race as part of its holistic-review process disadvantaged her and
other Caucasian applicants, in violation of the Equal Protection Clause.
See U.S. Const., Amdt. 14, § 1 (no State shall ``deny to any person
within its jurisdiction the equal protection of the laws''). The
District Court entered summary judgment in the University's favor, and
the Court of Appeals affirmed.

This Court granted certiorari and vacated the judgment of the Court of
Appeals, Fisher v. University of Tex. at Austin, 570 U.S. --------,
(Fisher I ), because it had applied an overly deferential ``good-faith''
standard in assessing the constitutionality of the University's program.
The Court remanded the case for the Court of Appeals to assess the
parties' claims under the correct legal standard.

Without further remanding to the District Court, the Court of Appeals
again affirmed the entry of summary judgment in the University's favor.
C.A 2014). This Court granted certiorari for a second time, 576 U.S.
--------, and now affirms.

Fisher I set forth three controlling principles relevant to assessing
the constitutionality of a public university's affirmative-action
program. First, ``because racial characteristics so seldom provide a
relevant basis for disparate treatment,'' Richmond v. J.A. Croson Co.,
``{[}r{]}ace may not be considered {[}by a university{]} unless the
admissions process can withstand strict scrutiny,'' Fisher I, 570 U.S.,
at --------, Strict scrutiny requires the university to demonstrate with
clarity that its " `purpose or interest is both constitutionally
permissible and substantial, and that its use of the classification is
necessary \ldots{} to the accomplishment of its purpose.' "

Second, Fisher I confirmed that ``the decision to pursue `the
educational benefits that flow from student body diversity' \ldots{} is,
in substantial measure, an academic judgment to which some, but not
complete, judicial deference is proper.'' --------, A university cannot
impose a fixed quota or otherwise ``define diversity as `some specified
percentage of a particular group merely because of its race or ethnic
origin.'\,'' Once, however, a university gives ``a reasoned, principled
explanation'' for its decision, deference must be given ``to the
University's conclusion, based on its experience and expertise, that a
diverse student body would serve its educational goals.'' (internal
quotation marks and citation omitted).

Third, Fisher I clarified that no deference is owed when determining
whether the use of race is narrowly tailored to achieve the university's
permissible goals. --------, 133 S.Ct.--2420. A university, Fisher I
explained, bears the burden of proving a ``nonracial approach'' would
not promote its interest in the educational benefits of diversity
``about as well and at tolerable administrative expense.'' --------, 133
S.Ct. (internal quotation marks omitted). Though ``{[}n{]}arrow
tailoring does not require exhaustion of every conceivable race-neutral
alternative'' or ``require a university to choose between maintaining a
reputation for excellence {[}and{]} fulfilling a commitment to provide
educational opportunities to members of all racial groups,'' Grutter, it
does impose ``on the university the ultimate burden of demonstrating''
that ``race-neutral alternatives'' that are both ``available'' and
``workable'' ``do not suffice.'' Fisher I, 570 U.S., at --------,

Fisher I set forth these controlling principles, while taking no
position on the constitutionality of the admissions program at issue in
this case. The Court held only that the District Court and the Court of
Appeals had ``confined the strict scrutiny inquiry in too narrow a way
by deferring to the University's good faith in its use of racial
classifications.'' --------, 133 S.Ct. The Court remanded the case, with
instructions to evaluate the record under the correct standard and to
determine whether the University had made ``a showing that its plan is
narrowly tailored to achieve'' the educational benefits that flow from
diversity. --------, On remand, the Court of Appeals determined that the
program conformed with the strict scrutiny mandated by Fisher I See
--660. Judge Garza dissented.

The University's program is sui generis Unlike other approaches to
college admissions considered by this Court, it combines holistic review
with a percentage plan. This approach gave rise to an unusual
consequence in this case: The component of the University's admissions
policy that had the largest impact on petitioner's chances of admission
was not the school's consideration of race under its holistic-review
process but rather the Top Ten Percent Plan. Because petitioner did not
graduate in the top 10 percent of her high school class, she was
categorically ineligible for more than three-fourths of the slots in the
incoming freshman class. It seems quite plausible, then, to think that
petitioner would have had a better chance of being admitted to the
University if the school used race-conscious holistic review to select
its entire incoming class, as was the case in Grutter

Despite the Top Ten Percent Plan's outsized effect on petitioner's
chances of admission, she has not challenged it. For that reason,
throughout this litigation, the Top Ten Percent Plan has been taken,
somewhat artificially, as a given premise.

Petitioner's acceptance of the Top Ten Percent Plan complicates this
Court's review. In particular, it has led to a record that is almost
devoid of information about the students who secured admission to the
University through the Plan. The Court thus cannot know how students
admitted solely based on their class rank differ in their contribution
to diversity from students admitted through holistic review.

In an ordinary case, this evidentiary gap perhaps could be filled by a
remand to the district court for further factfinding. When petitioner's
application was rejected, however, the University's combined
percentage-plan/holistic-review approach to admission had been in effect
for just three years. While studies undertaken over the eight years
since then may be of significant value in determining the
constitutionality of the University's current admissions policy, that
evidence has little bearing on whether petitioner received equal
treatment when her application was rejected in 2008. If the Court were
to remand, therefore, further factfinding would be limited to a narrow
3--year sample, review of which might yield little insight.

Furthermore, as discussed above, the University lacks any authority to
alter the role of the Top Ten Percent Plan in its admissions process.
The Plan was mandated by the Texas Legislature in the wake of Hopwood,
so the University, like petitioner in this litigation, has likely taken
the Plan as a given since its implementation in 1998. If the University
had no reason to think that it could deviate from the Top Ten Percent
Plan, it similarly had no reason to keep extensive data on the Plan or
the students admitted under it---particularly in the years before Fisher
I clarified the stringency of the strict-scrutiny burden for a school
that employs race-conscious review.

Under the circumstances of this case, then, a remand would do nothing
more than prolong a suit that has already persisted for eight years and
cost the parties on both sides significant resources. Petitioner long
since has graduated from another college, and the University's
policy---and the data on which it first was based---may have evolved or
changed in material ways.

The fact that this case has been litigated on a somewhat artificial
basis, furthermore, may limit its value for prospective guidance. The
Texas Legislature, in enacting the Top Ten Percent Plan, cannot much be
criticized, for it was responding to Hopwood, which at the time was
binding law in the State of Texas. That legislative response, in turn,
circumscribed the University's discretion in crafting its admissions
policy. These circumstances refute any criticism that the University did
not make good-faith efforts to comply with the law.

That does not diminish, however, the University's continuing obligation
to satisfy the burden of strict scrutiny in light of changing
circumstances. The University engages in periodic reassessment of the
constitutionality, and efficacy, of its admissions program. See Supp.
App. 32a; App. 448a. Going forward, that assessment must be undertaken
in light of the experience the school has accumulated and the data it
has gathered since the adoption of its admissions plan.

As the University examines this data, it should remain mindful that
diversity takes many forms. Formalistic racial classifications may
sometimes fail to capture diversity in all of its dimensions and, when
used in a divisive manner, could undermine the educational benefits the
University values. Through regular evaluation of data and consideration
of student experience, the University must tailor its approach in light
of changing circumstances, ensuring that race plays no greater role than
is necessary to meet its compelling interest. The University's
examination of the data it has acquired in the years since petitioner's
application, for these reasons, must proceed with full respect for the
constraints imposed by the Equal Protection Clause. The type of data
collected, and the manner in which it is considered, will have a
significant bearing on how the University must shape its admissions
policy to satisfy strict scrutiny in the years to come. Here, however,
the Court is necessarily limited to the narrow question before it:
whether, drawing all reasonable inferences in her favor, petitioner has
shown by a preponderance of the evidence that she was denied equal
treatment at the time her application was rejected.

In seeking to reverse the judgment of the Court of Appeals, petitioner
makes four arguments. First, she argues that the University has not
articulated its compelling interest with sufficient clarity. According
to petitioner, the University must set forth more precisely the level of
minority enrollment that would constitute a ``critical mass.'' Without a
clearer sense of what the University's ultimate goal is, petitioner
argues, a reviewing court cannot assess whether the University's
admissions program is narrowly tailored to that goal.

As this Court's cases have made clear, however, the compelling interest
that justifies consideration of race in college admissions is not an
interest in enrolling a certain number of minority students. Rather, a
university may institute a race-conscious admissions program as a means
of obtaining ``the educational benefits that flow from student body
diversity.'' Fisher I, 570 U.S., at --------, 133 S.Ct. (internal
quotation marks omitted); see also Grutter, As this Court has said,
enrolling a diverse student body ``promotes cross-racial understanding,
helps to break down racial stereotypes, and enables students to better
understand persons of different races.'' (internal quotation marks and
alteration omitted). Equally important, ``student body diversity
promotes learning outcomes, and better prepares students for an
increasingly diverse workforce and society.'' Ibid (internal quotation
marks omitted).

Increasing minority enrollment may be instrumental to these educational
benefits, but it is not, as petitioner seems to suggest, a goal that can
or should be reduced to pure numbers. Indeed, since the University is
prohibited from seeking a particular number or quota of minority
students, it cannot be faulted for failing to specify the particular
level of minority enrollment at which it believes the educational
benefits of diversity will be obtained. On the other hand, asserting an
interest in the educational benefits of diversity writ large is
insufficient. A university's goals cannot be elusory or amorphous---they
must be sufficiently measurable to permit judicial scrutiny of the
policies adopted to reach them.

The record reveals that in first setting forth its current admissions
policy, the University articulated concrete and precise goals. On the
first page of its 2004 ``Proposal to Consider Race and Ethnicity in
Admissions,'' the University identifies the educational values it seeks
to realize through its admissions process: the destruction of
stereotypes, the " `promot{[}ion of{]} cross-racial understanding,' "
the preparation of a student body " `for an increasingly diverse
workforce and society,' " and the " `cultivat{[}ion of{]} a set of
leaders with legitimacy in the eyes of the citizenry.' " Supp. App. 1a;
see also a; App. 314a--315a (deposition of N. Bruce Walker (Walker
Dep.)), 478a--479a (Walker Aff. ¶ 4) (setting forth the same goals).
Later in the proposal, the University explains that it strives to
provide an ``academic environment'' that offers a ``robust exchange of
ideas, exposure to differing cultures, preparation for the challenges of
an increasingly diverse workforce, and acquisition of competencies
required of future leaders.'' Supp. App. 23a. All of these objectives,
as a general matter, mirror the ``compelling interest'' this Court has
approved in its prior cases.

The University has provided in addition a ``reasoned, principled
explanation'' for its decision to pursue these goals. Fisher I--------,
The University's 39--page proposal was written following a year-long
study, which concluded that ``{[}t{]}he use of race-neutral policies and
programs ha{[}d{]} not been successful'' in ``provid{[}ing{]} an
educational setting that fosters cross-racial understanding,
provid{[}ing{]} enlightened discussion and learning, {[}or{]}
prepar{[}ing{]} students to function in an increasingly diverse
workforce and society.'' Supp. App. 25a; see also App. 481a--482a
(Walker Aff. ¶¶ 8--12) (describing the ``thoughtful review'' the
University undertook when it faced the ``important decision \ldots{}
whether or not to use race in its admissions process''). Further support
for the University's conclusion can be found in the depositions and
affidavits from various admissions officers, all of whom articulate the
same, consistent ``reasoned, principled explanation.'' See, e.g., a
(Ishop Dep.), 314a--318a, 359a (Walker Dep.), 415a--416a (Defendant's
Statement of Facts), 478a--479a, 481a--482a (Walker Aff. ¶¶ 4, 10--13).
Petitioner's contention that the University's goal was insufficiently
concrete is rebutted by the record.

Second, petitioner argues that the University has no need to consider
race because it had already ``achieved critical mass'' by 2003 using the
Top Ten Percent Plan and race-neutral holistic review. Brief for
Petitioner 46. Petitioner is correct that a university bears a heavy
burden in showing that it had not obtained the educational benefits of
diversity before it turned to a race-conscious plan. The record reveals,
however, that, at the time of petitioner's application, the University
could not be faulted on this score. Before changing its policy the
University conducted ``months of study and deliberation, including
retreats, interviews, {[}and{]} review of data,'' App. 446a, and
concluded that ``{[}t{]}he use of race-neutral policies and programs
ha{[}d{]} not been successful in achieving'' sufficient racial diversity
at the University, Supp. App. 25a. At no stage in this litigation has
petitioner challenged the University's good faith in conducting its
studies, and the Court properly declines to consider the extrarecord
materials the dissent relies upon, many of which are tangential to this
case at best and none of which the University has had a full opportunity
to respond to. See, e.g., post (opinion of ALITO, J.) (describing a 2015
report regarding the admission of applicants who are related to
``politically connected individuals'').

The record itself contains significant evidence, both statistical and
anecdotal, in support of the University's position. To start, the
demographic data the University has submitted show consistent stagnation
in terms of the percentage of minority students enrolling at the
University from 1996 to 2002. In 1996, for example, 266
African--American freshmen enrolled, a total that constituted 4 percent
of the incoming class. In 2003, the year Grutter was decided, 267
African--American students enrolled---again, 4 percent of the incoming
class. The numbers for Hispanic and Asian--American students tell a
similar story. See Supp. App. 43a. Although demographics alone are by no
means dispositive, they do have some value as a gauge of the
University's ability to enroll students who can offer underrepresented
perspectives.

In addition to this broad demographic data, the University put forward
evidence that minority students admitted under the Hopwood regime
experienced feelings of loneliness and isolation. See, e.g., App.
317a--318a.

This anecdotal evidence is, in turn, bolstered by further, more nuanced
quantitative data. In 2002, 52 percent of undergraduate classes with at
least five students had no African--American students enrolled in them,
and 27 percent had only one African--American student. Supp. App. 140a.
In other words, only 21 percent of undergraduate classes with five or
more students in them had more than one African--American student
enrolled. Twelve percent of these classes had no Hispanic students, as
compared to 10 percent in 1996. a, 140a. Though a college must
continually reassess its need for race-conscious review, here that
assessment appears to have been done with care, and a reasonable
determination was made that the University had not yet attained its
goals.

Third, petitioner argues that considering race was not necessary because
such consideration has had only a " `minimal impact' in advancing the
{[}University's{]} compelling interest." Brief for Petitioner 46; see
also Tr. of Oral Arg. 23:10--12; 24:13--25:2, 25:24--26:3. Again, the
record does not support this assertion. In 2003, 11 percent of the Texas
residents enrolled through holistic review were Hispanic and 3 percent
were African--American. Supp. App. 157a. In 2007, by contrast, 16
percent of the Texas holistic-review freshmen were Hispanic and 6
percent were African--American. Those increases---of 54 percent and 94
percent, respectively---show that consideration of race has had a
meaningful, if still limited, effect on the diversity of the
University's freshman class.

In any event, it is not a failure of narrow tailoring for the impact of
racial consideration to be minor. The fact that race consciousness
played a role in only a small portion of admissions decisions should be
a hallmark of narrow tailoring, not evidence of unconstitutionality.

Petitioner's final argument is that ``there are numerous other available
race-neutral means of achieving'' the University's compelling interest.
Brief for Petitioner 47. A review of the record reveals, however, that,
at the time of petitioner's application, none of her proposed
alternatives was a workable means for the University to attain the
benefits of diversity it sought. For example, petitioner suggests that
the University could intensify its outreach efforts to African--American
and Hispanic applicants. But the University submitted extensive evidence
of the many ways in which it already had intensified its outreach
efforts to those students. The University has created three new
scholarship programs, opened new regional admissions centers, increased
its recruitment budget by half-a-million dollars, and organized over
1,000 recruitment events. Supp. App. 29a--32a; App. 450a--452a (citing
affidavit of Michael Orr ¶¶ 4--20). Perhaps more significantly, in the
wake of Hopwood, the University spent seven years attempting to achieve
its compelling interest using race-neutral holistic review. None of
these efforts succeeded, and petitioner fails to offer any meaningful
way in which the University could have improved upon them at the time of
her application.

Petitioner also suggests altering the weight given to academic and
socioeconomic factors in the University's admissions calculus. This
proposal ignores the fact that the University tried, and failed, to
increase diversity through enhanced consideration of socioeconomic and
other factors. And it further ignores this Court's precedent making
clear that the Equal Protection Clause does not force universities to
choose between a diverse student body and a reputation for academic
excellence. Grutter, Petitioner's final suggestion is to uncap the Top
Ten Percent Plan, and admit more---if not all---the University's
students through a percentage plan. As an initial matter, petitioner
overlooks the fact that the Top Ten Percent Plan, though facially
neutral, cannot be understood apart from its basic purpose, which is to
boost minority enrollment. Percentage plans are ``adopted with racially
segregated neighborhoods and schools front and center stage.'' Fisher I,
(GINSBURG, J., dissenting). ``It is race consciousness, not blindness to
race, that drives such plans.'' Consequently, petitioner cannot assert
simply that increasing the University's reliance on a percentage plan
would make its admissions policy more race neutral.

Even if, as a matter of raw numbers, minority enrollment would increase
under such a regime, petitioner would be hard-pressed to find convincing
support for the proposition that college admissions would be improved if
they were a function of class rank alone. That approach would sacrifice
all other aspects of diversity in pursuit of enrolling a higher number
of minority students. A system that selected every student through class
rank alone would exclude the star athlete or musician whose grades
suffered because of daily practices and training. It would exclude a
talented young biologist who struggled to maintain above-average grades
in humanities classes. And it would exclude a student whose
freshman-year grades were poor because of a family crisis but who got
herself back on track in her last three years of school, only to find
herself just outside of the top decile of her class.

These are but examples of the general problem. Class rank is a single
metric, and like any single metric, it will capture certain types of
people and miss others. This does not imply that students admitted
through holistic review are necessarily more capable or more desirable
than those admitted through the Top Ten Percent Plan. It merely reflects
the fact that privileging one characteristic above all others does not
lead to a diverse student body. Indeed, to compel universities to admit
students based on class rank alone is in deep tension with the goal of
educational diversity as this Court's cases have defined it. See
Grutter, (explaining that percentage plans ``may preclude the university
from conducting the individualized assessments necessary to assemble a
student body that is not just racially diverse, but diverse along all
the qualities valued by the university''); (pointing out that the Top
Ten Percent Law leaves out students ``who fell outside their high
school's top ten percent but excelled in unique ways that would enrich
the diversity of {[}the University's{]} educational experience'' and
``leaves a gap in an admissions process seeking to create the
multi-dimensional diversity that {[}Regents of Univ. of Cal v. Bakke,{]}
envisions''). At its center, the Top Ten Percent Plan is a blunt
instrument that may well compromise the University's own definition of
the diversity it seeks.

In addition to these fundamental problems, an admissions policy that
relies exclusively on class rank creates perverse incentives for
applicants. Percentage plans ``encourage parents to keep their children
in low-performing segregated schools, and discourage students from
taking challenging classes that might lower their grade point
averages.'' Gratz, n.~10, (GINSBURG, J., dissenting).

For all these reasons, although it may be true that the Top Ten Percent
Plan in some instances may provide a path out of poverty for those who
excel at schools lacking in resources, the Plan cannot serve as the
admissions solution that petitioner suggests. Wherever the balance
between percentage plans and holistic review should rest, an effective
admissions policy cannot prescribe, realistically, the exclusive use of
a percentage plan.

In short, none of petitioner's suggested alternatives---nor other
proposals considered or discussed in the course of this
litigation---have been shown to be ``available'' and ``workable'' means
through which the University could have met its educational goals, as it
understood and defined them in 2008. Fisher I--------, The University
has thus met its burden of showing that the admissions policy it used at
the time it rejected petitioner's application was narrowly tailored.

A university is in large part defined by those intangible ``qualities
which are incapable of objective measurement but which make for
greatness.'' Sweatt v. Painter, Considerable deference is owed to a
university in defining those intangible characteristics, like student
body diversity, that are central to its identity and educational
mission. But still, it remains an enduring challenge to our Nation's
education system to reconcile the pursuit of diversity with the
constitutional promise of equal treatment and dignity.

In striking this sensitive balance, public universities, like the States
themselves, can serve as ``laboratories for experimentation.'' United
States v. Lopez, (KENNEDY, J., concurring); see also New State Ice
Co.~v. Liebmann, (Brandeis, J., dissenting). The University of Texas at
Austin has a special opportunity to learn and to teach. The University
now has at its disposal valuable data about the manner in which
different approaches to admissions may foster diversity or instead
dilute it. The University must continue to use this data to scrutinize
the fairness of its admissions program; to assess whether changing
demographics have undermined the need for a race-conscious policy; and
to identify the effects, both positive and negative, of the
affirmative-action measures it deems necessary.

The Court's affirmance of the University's admissions policy today does
not necessarily mean the University may rely on that same policy without
refinement. It is the University's ongoing obligation to engage in
constant deliberation and continued reflection regarding its admissions
policies.

The judgment of the Court of Appeals is affirmed.

\textbf{Justice ALITO, with whom THE CHIEF JUSTICE and Justice THOMAS
join, dissenting.} Something strange has happened since our prior
decision in this case. See Fisher v. University of Tex. at Austin, 570
U.S. --------, (Fisher I ). In that decision, we held that strict
scrutiny requires the University of Texas at Austin (UT or University)
to show that its use of race and ethnicity in making admissions
decisions serves compelling interests and that its plan is narrowly
tailored to achieve those ends. Rejecting the argument that we should
defer to UT's judgment on those matters, we made it clear that UT was
obligated (1) to identify the interests justifying its plan with enough
specificity to permit a reviewing court to determine whether the
requirements of strict scrutiny were met, and (2) to show that those
requirements were in fact satisfied. On remand, UT failed to do what our
prior decision demanded. The University has still not identified with
any degree of specificity the interests that its use of race and
ethnicity is supposed to serve. Its primary argument is that merely
invoking ``the educational benefits of diversity'' is sufficient and
that it need not identify any metric that would allow a court to
determine whether its plan is needed to serve, or is actually serving,
those interests. This is nothing less than the plea for deference that
we emphatically rejected in our prior decision. Today, however, the
Court inexplicably grants that request.

To the extent that UT has ever moved beyond a plea for deference and
identified the relevant interests in more specific terms, its efforts
have been shifting, unpersuasive, and, at times, less than candid. When
it adopted its race-based plan, UT said that the plan was needed to
promote classroom diversity. See Supp. App. 1a, 24a--25a, 39a; App.
316a. It pointed to a study showing that African--American, Hispanic,
and Asian--American students were underrepresented in many classes. See
Supp. App. 26a. But UT has never shown that its race-conscious plan
actually ameliorates this situation. The University presents no evidence
that its admissions officers, in administering the ``holistic''
component of its plan, make any effort to determine whether an
African--American, Hispanic, or Asian--American student is likely to
enroll in classes in which minority students are underrepresented. And
although UT's records should permit it to determine without much
difficulty whether holistic admittees are any more likely than students
admitted through the Top Ten Percent Law, Tex. Educ.Code Ann. § 51 (West
Cum. Supp. 2015), to enroll in the classes lacking racial or ethnic
diversity, UT either has not crunched those numbers or has not revealed
what they show. Nor has UT explained why the underrepresentation of
Asian--American students in many classes justifies its plan, which
discriminates against those students.

At times, UT has claimed that its plan is needed to achieve a ``critical
mass'' of African--American and Hispanic students, but it has never
explained what this term means. According to UT, a critical mass is
neither some absolute number of African--American or Hispanic students
nor the percentage of African--Americans or Hispanics in the general
population of the State. The term remains undefined, but UT tells us
that it will let the courts know when the desired end has been achieved.
See App. 314a--315a. This is a plea for deference---indeed, for blind
deference---the very thing that the Court rejected in Fisher I.

UT has also claimed at times that the race-based component of its plan
is needed because the Top Ten Percent Plan admits the wrong kind of
African--American and Hispanic students, namely, students from poor
families who attend schools in which the student body is predominantly
African--American or Hispanic. As UT put it in its brief in Fisher I,
the race-based component of its admissions plan is needed to admit
``{[}t{]}he African--American or Hispanic child of successful
professionals in Dallas.'' Brief for Respondents, O.T. 2012,
No.~11--345, p.~34.

After making this argument in its first trip to this Court, UT
apparently had second thoughts, and in the latest round of briefing UT
has attempted to disavow ever having made the argument. See Brief for
Respondents 2 (``Petitioner's argument that UT's interest is favoring
`affluent' minorities is a fabrication''); see also But it did, and the
argument turns affirmative action on its head. Affirmative-action
programs were created to help disadvantaged students.

Although UT now disowns the argument that the Top Ten Percent Plan
results in the admission of the wrong kind of African--American and
Hispanic students, the Fifth Circuit majority bought a version of that
claim. As the panel majority put it, the Top Ten African--American and
Hispanic admittees cannot match the holistic African--American and
Hispanic admittees when it comes to ``records of personal achievement,''
a ``variety of perspectives'' and ``life experiences,'' and ``unique
skills.'' All in all, according to the panel majority, the Top Ten
Percent students cannot ``enrich the diversity of the student body'' in
the same way as the holistic admittees. As Judge Garza put it in
dissent, the panel majority concluded that the Top Ten Percent admittees
are ``somehow more homogenous, less dynamic, and more undesirably
stereotypical than those admitted under holistic review.'' (Garza, J.,
dissenting).

The Fifth Circuit reached this conclusion with little direct evidence
regarding the characteristics of the Top Ten Percent and holistic
admittees. Instead, the assumption behind the Fifth Circuit's reasoning
is that most of the African--American and Hispanic students admitted
under the race-neutral component of UT's plan were able to rank in the
top decile of their high school classes only because they did not have
to compete against white and Asian--American students. This insulting
stereotype is not supported by the record. African--American and
Hispanic students admitted under the Top Ten Percent Plan receive higher
college grades than the African--American and Hispanic students admitted
under the race-conscious program. See Supp. App. 164a--165a.

It should not have been necessary for us to grant review a second time
in this case, and I have no greater desire than the majority to see the
case drag on. But that need not happen. When UT decided to adopt its
race-conscious plan, it had every reason to know that its plan would
have to satisfy strict scrutiny and that this meant that it would be its
burden to show that the plan was narrowly tailored to serve compelling
interests. UT has failed to make that showing. By all rights, judgment
should be entered in favor of petitioner.

But if the majority is determined to give UT yet another chance, we
should reverse and send this case back to the District Court. What the
majority has now done---awarding a victory to UT in an opinion that
fails to address the important issues in the case---is simply wrong.

Over the past 20 years, UT has frequently modified its admissions
policies, and it has generally employed race and ethnicity in the most
aggressive manner permitted under controlling precedent.

Before 1997, race was considered directly as part of the general
admissions process, and it was frequently a controlling factor.
Admissions were based on two criteria: (1) the applicant's Academic
Index (AI), which was computed from standardized test scores and high
school class rank, and (2) the applicant's race. In 1996, the last year
this race-conscious system was in place, 4 \% of enrolled freshmen were
African--American, 14 \% were Asian--American, and 14 \% were Hispanic.
Supp. App. 43a.

The Fifth Circuit's decision in Hopwood v. Texas, prohibited UT from
using race in admissions. In response to Hopwood, beginning with the
1997 admissions cycle, UT instituted a ``holistic review'' process in
which it considered an applicant's AI as well as a Personal Achievement
Index (PAI) that was intended, among other things, to increase minority
enrollment. The race-neutral PAI was a composite of scores from two
essays and a personal achievement score, which in turn was based on a
holistic review of an applicant's leadership qualities, extracurricular
activities, honors and awards, work experience, community service, and
special circumstances. Special consideration was given to applicants
from poor families, applicants from homes in which a language other than
English was customarily spoken, and applicants from single-parent
households. Because this race-neutral plan gave a preference to
disadvantaged students, it had the effect of ``disproportionately''
benefiting minority candidates. 645 F.Supp d 587, 592 (W.D.Tex ).The
Texas Legislature also responded to Hopwood. In 1997, it enacted the Top
Ten Percent Plan, which mandated that UT admit all Texas seniors who
rank in the top 10\% of their high school classes. This facially
race-neutral law served to equalize competition between students who
live in relatively affluent areas with superior schools and students in
poorer areas served by schools offering fewer opportunities for academic
excellence. And by benefiting the students in the latter group, this
plan, like the race-neutral holistic plan already adopted by UT, tended
to benefit African--American and Hispanic students, who are often
trapped in inferior public schools. --653.

Starting in 1998, when the Top Ten Percent Plan took effect, UT's
holistic, race-neutral AI/PAI system continued to be used to fill the
seats in the entering class that were not taken by Top Ten Percent
students. The AI/PAI system was also used to determine program placement
for all incoming students, including the Top Ten Percent students.

``The University's revised admissions process, coupled with the
operation of the Top Ten Percent Law, resulted in a more racially
diverse environment at the University.'' Fisher I, 570 U.S., at
--------, In 2000, UT announced that its ``enrollment levels for African
American and Hispanic freshmen have returned to those of 1996, the year
before the Hopwood decision prohibited the consideration of race in
admissions policies.'' App. 393a; see also Supp. App. 23a--24a
(pre-Hopwood diversity levels were ``restored'' in 1999); App.
392a--393a (``The `Top 10 Percent Law' is Working for Texas'' and ``has
enabled us to diversify enrollment at UT Austin with talented students
who succeed''). And in 2003, UT proclaimed that it had ``effectively
compensated for the loss of affirmative action.'' a; see also a
(``Diversity efforts at The University of Texas at Austin have brought a
higher number of freshman minority students---African Americans,
Hispanics and Asian--Americans---to the campus than were enrolled in
1996, the year a court ruling ended the use of affirmative action in the
university's enrollment process''). By 2004---the last year under the
holistic, race-neutral AI/PAI system---UT's entering class was 4 \%
African--American, 17 \% Asian--American, and 16 \% Hispanic. Supp. App.
156a. The 2004 entering class thus had a higher percentage of
African--Americans, Asian--Americans, and Hispanics than the class that
entered in 1996, when UT had last employed racial preferences.

Notwithstanding these lauded results, UT leapt at the opportunity to
reinsert race into the process. On June 23, 2003, this Court decided
Grutter v. Bollingerwhich upheld the University of Michigan Law School's
race-conscious admissions system. In Grutter, the Court warned that a
university contemplating the consideration of race as part of its
admissions process must engage in ``serious, good faith consideration of
workable race-neutral alternatives that will achieve the diversity the
university seeks.'' Nevertheless, on the very day Grutter was handed
down, UT's president announced that ``{[}t{]}he University of Texas at
Austin will modify its admissions procedures'' in light of Grutter,
including by ``implementing procedures at the undergraduate level that
combine the benefits of the Top 10 Percent Law with affirmative action
programs.'' App. 406a--407a (emphasis added). UT purports to have later
engaged in ``almost a year of deliberations,'' a, but there is no
evidence that the reintroduction of race into the admissions process was
anything other than a foregone conclusion following the president's
announcement.

``The University's plan to resume race-conscious admissions was given
formal expression in June 2004 in an internal document entitled Proposal
to Consider Race and Ethnicity in Admissions'' (Proposal). Fisher
I--------, The Proposal stated that UT needed race-conscious admissions
because it had not yet achieved a ``critical mass of racial diversity.''
Supp. App. 25a. In support of this claim, UT cited two pieces of
evidence. First, it noted that there were ``significant differences
between the racial and ethnic makeup of the University's undergraduate
population and the state's population.'' a. Second, the Proposal
``relied in substantial part,'' Fisher I--------, on a study of a subset
of undergraduate classes containing at least five students, see Supp.
App. 26a. The study showed that among select classes with five or more
students, 52\% had no African--Americans, 16\% had no Asian--Americans,
and 12\% had no Hispanics. Moreover, the study showed, only 21\% of
these classes had two or more African--Americans, 67\% had two or more
Asian--Americans, and 70\% had two or more Hispanics. See Based on this
study, the Proposal concluded that UT ``has not reached a critical mass
at the classroom level.'' a. The Proposal did not analyze the
backgrounds, life experiences, leadership qualities, awards,
extracurricular activities, community service, personal attributes, or
other characteristics of the minority students who were already being
admitted to UT under the holistic, race-neutral process.

``To implement the Proposal the University included a student's race as
a component of the PAI score, beginning with applicants in the fall of
2004.'' Fisher I, 570 U.S., at --------, ``The University asks students
to classify themselves from among five predefined racial categories on
the application.'' ``Race is not assigned an explicit numerical value,
but it is undisputed that race is a meaningful factor.'' UT decided to
use racial preferences to benefit African--American and Hispanic
students because it considers those groups ``underrepresented
minorities.'' Supp. App. 25a; see also App. 445a--446a (defining
``underrepresented minorities'' as ``Hispanic{[}s{]} and African
Americans''). Even though UT's classroom study showed that more classes
lacked Asian--American students than lacked Hispanic students, Supp.
App. 26a, UT deemed Asian--Americans ``overrepresented'' based on state
demographics, 645 F.Supp d ; see also (``It is undisputed that UT
considers African--Americans and Hispanics to be underrepresented but
does not consider Asian--Americans to be underrepresented'').

Although UT claims that race is but a ``factor of a factor of a factor
of a factor,'' UT acknowledges that ``race is the only one of {[}its{]}
holistic factors that appears on the cover of every application,'' Tr.
of Oral Arg. 54 (Oct.~10, 2012). ``Because an applicant's race is
identified at the front of the admissions file, reviewers are aware of
it throughout the evaluation.'' 645 F.Supp d ; see also (``{[}A{]}
candidate's race is known throughout the application process'').
Consideration of race therefore pervades every aspect of UT's admissions
process. See App. 219a (``We are certainly aware of the applicant's
race. It's on the front page of the application that's being read
{[}and{]} is used in context with everything else that's part of the
applicant's file''). This is by design, as UT considers its use of
racial classifications to be a benign form of ``social engineering.''
Powers, Why Schools Still Need Affirmative Action, National L. J.,
Aug.~4, 2014, p.~22 (editorial by Bill Powers, President of UT from
2006--2015) (``Opponents accuse defenders of race-conscious admissions
of being in favor of `social engineering,' to which I believe we should
reply, `Guilty as charged'\,'').

Notwithstanding the omnipresence of racial classifications, UT claims
that it keeps no record of how those classifications affect its process.
``The university doesn't keep any statistics on how many students are
affected by the consideration of race in admissions decisions,'' and it
``does not know how many minority students are affected in a positive
manner by the consideration of race.'' App. 337a. According to UT, it
has no way of making these determinations. See a--322a. UT says that it
does not tell its admissions officers how much weight to give to race.
See Deposition of Gary Lavergne 43--45, Record in No.~1:08--CV--00263
(WD Tex.), Doc. 94--9 (Lavergne Deposition). And because the influence
of race is always ``contextual,'' UT claims, it cannot provide even a
single example of an instance in which race impacted a student's odds of
admission. See App. 220a (``Q. Could you give me an example where race
would have some impact on an applicant's personal achievement score? A.
To be honest, not really\ldots. {[}I{]}t's impossible to say---to give
you an example of a particular student because it's all contextual'').
Accordingly, UT asserts that it has no idea which students were admitted
as a result of its race-conscious system and which students would have
been admitted under a race-neutral process. UT thus makes no effort to
assess how the individual characteristics of students admitted as the
result of racial preferences differ (or do not differ) from those of
students who would have been admitted without them.

UT's race-conscious admissions program cannot satisfy strict scrutiny.
UT says that the program furthers its interest in the educational
benefits of diversity, but it has failed to define that interest with
any clarity or to demonstrate that its program is narrowly tailored to
achieve that or any other particular interest. By accepting UT's
rationales as sufficient to meet its burden, the majority licenses UT's
perverse assumptions about different groups of minority students---the
precise assumptions strict scrutiny is supposed to stamp out.

``The moral imperative of racial neutrality is the driving force of the
Equal Protection Clause.'' Richmond v. J.A. Croson Co., (KENNEDY, J.,
concurring in part and concurring in judgment). ``At the heart of the
Constitution's guarantee of equal protection lies the simple command
that the Government must treat citizens as individuals, not as simply
components of a racial, religious, sexual or national class.'' Miller v.
Johnson, (internal quotation marks omitted). ``Race-based assignments
embody stereotypes that treat individuals as the product of their race,
evaluating their thoughts and efforts---their very worth as
citizens---according to a criterion barred to the Government by history
and the Constitution.'' (internal quotation marks omitted). Given our
constitutional commitment to ``the doctrine of equality,'' "
`{[}d{]}istinctions between citizens solely because of their ancestry
are by their very nature odious to a free people.' " Rice v. Cayetano,
(quoting Hirabayashi v. United States, ).

``{[}B{]}ecause racial characteristics so seldom provide a relevant
basis for disparate treatment, the Equal Protection Clause demands that
racial classifications \ldots{} be subjected to the most rigid
scrutiny.'' Fisher I, 570 U.S., at --------, 133 S.Ct. (internal
quotation marks and citations omitted). ``{[}J{]}udicial review must
begin from the position that `any official action that treats a person
differently on account of his race or ethnic origin is inherently
suspect.'\,'' ; see also Grutter, (KENNEDY, J., dissenting) (" `Racial
and ethnic distinctions of any sort are inherently suspect and thus call
for the most exacting judicial examination' ``). Under strict scrutiny,
the use of race must be''necessary to further a compelling governmental
interest," and the means employed must be " `specifically and narrowly'
" tailored to accomplish the compelling interest. 333, (O'Connor, J.,
for the Court).

The ``higher education dynamic does not change'' this standard. Fisher
I--------, ``Racial discrimination {[}is{]} invidious in all contexts,''
Edmonson v. Leesville Concrete Co., and " `{[}t{]}he analysis and level
of scrutiny applied to determine the validity of {[}a racial{]}
classification do not vary simply because the objective appears
acceptable,' " Fisher I--------, Nor does the standard of review "
`depen{[}d{]} on the race of those burdened or benefited by a particular
classification.' " Gratz v. Bollinger, (quoting Adarand Constructors,
Inc.~v. Peã, ); see also Miller, (``This rule obtains with equal force
regardless of `the race of those burdened or benefited by a particular
classification'\,'' (quoting Croson, (plurality opinion of O'Connor,
J.))). ``Thus, `any person, of whatever race, has the right to demand
that any governmental actor subject to the Constitution justify any
racial classification subjecting that person to unequal treatment under
the strictest of judicial scrutiny.'\,'' Gratz, (quoting Adarand, ).

In short, in ``all contexts,'' Edmonson, racial classifications are
permitted only ``as a last resort,'' when all else has failed, Croson,
(opinion of KENNEDY, J.). ``Strict scrutiny is a searching examination,
and it is the government that bears the burden'' of proof. Fisher I, 570
U.S., at --------, To meet this burden, the government must
``demonstrate with clarity that its `purpose or interest is both
constitutionally permissible and substantial, and that its use of the
classification is necessary \ldots{} to the accomplishment of its
purpose.'\,'' --------, 133 S.Ct. (emphasis added).

Here, UT has failed to define its interest in using racial preferences
with clarity. As a result, the narrow tailoring inquiry is impossible,
and UT cannot satisfy strict scrutiny.

When UT adopted its challenged policy, it characterized its compelling
interest as obtaining a " `critical mass' " of underrepresented
minorities. --------, The 2004 Proposal claimed that ``{[}t{]}he use of
race-neutral policies and programs has not been successful in achieving
a critical mass of racial diversity.'' Supp. App. 25a; see Fisher v.
University of Tex. at Austin, C.A 2011) (``{[}T{]}he 2004 Proposal
explained that UT had not yet achieved the critical mass of
underrepresented minority students needed to obtain the full educational
benefits of diversity''). But to this day, UT has not explained in
anything other than the vaguest terms what it means by ``critical
mass.'' In fact, UT argues that it need not identify any interest more
specific than ``securing the educational benefits of diversity.'' Brief
for Respondents 15.

UT has insisted that critical mass is not an absolute number. See Tr. of
Oral Arg. 39 (Oct.~10, 2012) (declaring that UT is not working toward
any particular number of African--American or Hispanic students); App.
315a (confirming that UT has not defined critical mass as a number and
has not projected when it will attain critical mass). Instead, UT
prefers a deliberately malleable ``we'll know it when we see it'' notion
of critical mass. It defines ``critical mass'' as ``an adequate
representation of minority students so that the \ldots{} educational
benefits that can be derived from diversity can actually happen,'' and
it declares that it ``will \ldots{} know {[}that{]} it has reached
critical mass'' when it ``see{[}s{]} the educational benefits
happening.'' a--315a. In other words: Trust us.

This intentionally imprecise interest is designed to insulate UT's
program from meaningful judicial review. As Judge Garza explained:

``{[}T{]}o meet its narrow tailoring burden, the University must explain
its goal to us in some meaningful way. We cannot undertake a rigorous
ends-to-means narrow tailoring analysis when the University will not
define the ends. We cannot tell whether the admissions program closely
`fits' the University's goal when it fails to objectively articulate its
goal. Nor can we determine whether considering race is necessary for the
University to achieve `critical mass,' or whether there are effective
race-neutral alternatives, when it has not described what `critical
mass' requires.'' (dissenting opinion). Indeed, without knowing in
reasonably specific terms what critical mass is or how it can be
measured, a reviewing court cannot conduct the requisite ``careful
judicial inquiry'' into whether the use of race was " `necessary.' "
Fisher I--------.

To be sure, I agree with the majority that our precedents do not require
UT to pinpoint ``an interest in enrolling a certain number of minority
students.'' But in order for us to assess whether UT's program is
narrowly tailored, the University must identify some sort of concrete
interest ``Classifying and assigning'' students according to race
``requires more than \ldots{} an amorphous end to justify it.'' Parents
Involved in Community Schools v. Seattle School Dist. No.~1, Because UT
has failed to explain ``with clarity,'' Fisher I--------, why it needs a
race-conscious policy and how it will know when its goals have been met,
the narrow tailoring analysis cannot be meaningfully conducted. UT
therefore cannot satisfy strict scrutiny.

The majority acknowledges that ``asserting an interest in the
educational benefits of diversity writ large is insufficient,'' and that
``{[}a{]} university's goals cannot be elusory or amorphous---they must
be sufficiently measurable to permit judicial scrutiny of the policies
adopted to reach them.'' According to the majority, however, UT has
articulated the following ``concrete and precise goals'': ``the
destruction of stereotypes, the promot{[}ion of{]} cross-racial
understanding, the preparation of a student body for an increasingly
diverse workforce and society, and the cultivat{[}ion of{]} a set of
leaders with legitimacy in the eyes of the citizenry.'' (internal
quotation marks omitted).

These are laudable goals, but they are not concrete or precise, and they
offer no limiting principle for the use of racial preferences. For
instance, how will a court ever be able to determine whether stereotypes
have been adequately destroyed? Or whether cross-racial understanding
has been adequately achieved? If a university can justify racial
discrimination simply by having a few employees opine that racial
preferences are necessary to accomplish these nebulous goals, see --
2211 (citing only self-serving statements from UT officials), then the
narrow tailoring inquiry is meaningless. Courts will be required to
defer to the judgment of university administrators, and
affirmative-action policies will be completely insulated from judicial
review.

By accepting these amorphous goals as sufficient for UT to carry its
burden, the majority violates decades of precedent rejecting blind
deference to government officials defending " `inherently suspect' "
classifications. Miller, (citing Regents of Univ. of Cal. v. Bakke,
(opinion of Powell, J.)); see also, e.g., Miller, (``Our presumptive
skepticism of all racial classifications \ldots{} prohibits us \ldots{}
from accepting on its face the Justice Department's conclusion''
(citation omitted)); Croson, (``{[}T{]}he mere recitation of a `benign'
or legitimate purpose for a racial classification is entitled to little
or no weight''); (``The history of racial classifications in this
country suggests that blind judicial deference to legislative or
executive pronouncements of necessity has no place in equal protection
analysis''). Most troublingly, the majority's uncritical deference to
UT's self-serving claims blatantly contradicts our decision in the prior
iteration of this very case, in which we faulted the Fifth Circuit for
improperly ``deferring to the University's good faith in its use of
racial classifications.'' Fisher I, 570 U.S., at --------, As we
emphasized just three years ago, our precedent ``ma{[}kes{]} clear that
it is for the courts, not for university administrators, to ensure
that'' an admissions process is narrowly tailored. --------, A court
cannot ensure that an admissions process is narrowly tailored if it
cannot pin down the goals that the process is designed to achieve. UT's
vague policy goals are ``so broad and imprecise that they cannot
withstand strict scrutiny.'' Parents Involved, (KENNEDY, J., concurring
in part and concurring in judgment).

Although UT's primary argument is that it need not point to any interest
more specific than ``the educational benefits of diversity,'' Brief for
Respondents 15, it has---at various points in this
litigation---identified four more specific goals: demographic parity,
classroom diversity, intraracial diversity, and avoiding racial
isolation. Neither UT nor the majority has demonstrated that any of
these four goals provides a sufficient basis for satisfying strict
scrutiny. And UT's arguments to the contrary depend on a series of
invidious assumptions.

First, both UT and the majority cite demographic data as evidence that
African--American and Hispanic students are ``underrepresented'' at UT
and that racial preferences are necessary to compensate for this
underrepresentation. See, e.g., Supp. App. 24a; -- 2212. But neither UT
nor the majority is clear about the relationship between Texas
demographics and UT's interest in obtaining a critical mass.

Does critical mass depend on the relative size of a particular group in
the population of a State? For example, is the critical mass of
African--Americans and Hispanics in Texas, where African--Americans are
about 11 \% of the population and Hispanics are about 37 \%, different
from the critical mass in neighboring New Mexico, where the
African--American population is much smaller (about 2 \%) and the
Hispanic population constitutes a higher percentage of the State's total
(about 46 \%)? See United States Census Bureau, QuickFacts, online at
https://www.census.gov/quickfacts/table/PST045215/35,48 (all Internet
materials as last visited June 21, 2016).

UT's answer to this question has veered back and forth. At oral argument
in Fisher I, UT's lawyer indicated that critical mass ``could'' vary
``from group to group'' and from ``state to state.'' See Tr. of Oral
Arg. 40 (Oct.~10, 2012). And UT initially justified its race-conscious
plan at least in part on the ground that ``significant differences
between the racial and ethnic makeup of the University's undergraduate
population and the state's population prevent the University from fully
achieving its mission.'' Supp. App. 24a; see also a (``{[}A{]} critical
mass in Texas is necessarily larger than a critical mass in Michigan,''
because ``{[}a{]} majority of the college-age population in Texas is
African American or Hispanic''); Fisher, 631 F d--226, 236 (concluding
that UT's reliance on Texas demographics reflects ``measured attention
to the community it serves''); Brief for Respondents in No.~11--345
(noting that critical mass may hinge, in part, on ``the communities that
universities serve''). UT's extensive reliance on state demographics is
also revealed by its substantial focus on increasing the representation
of Hispanics, but not Asian--Americans, see, e.g., 645 F.Supp d ; Supp.
App. 25a; App. 445a--446a, because Hispanics, but not Asian--Americans,
are underrepresented at UT when compared to the demographics of the
State. On the other hand, UT's counsel asserted that the critical mass
for the University is ``not at all'' dependent on the demographics of
Texas, and that UT's ``concept {[}of{]} critical mass isn't tied to
demographic{[}s{]}.'' Tr. of Oral Arg. 40, 49 (Oct.~10, 2012). And UT's
Fisher I brief expressly agreed that ``a university cannot look to
racial demographics---and then work backward in its admissions process
to meet a target tied to such demographics.'' Brief for Respondents in
No.~11--345; see also Brief for Respondents 26--27 (disclaiming any
interest in demographic parity).

To the extent that UT is pursuing parity with Texas demographics, that
is nothing more than ``outright racial balancing,'' which this Court has
time and again held ``patently unconstitutional.'' Fisher I, 570 U.S.,
at --------, 133 S.Ct. ; see Grutter, (``{[}O{]}utright racial balancing
\ldots{} is patently unconstitutional''); Freeman v. Pitts, (``Racial
balance is not to be achieved for its own sake''); Croson, (rejecting
goal of ``outright racial balancing''); Bakke, (opinion of Powell, J.)
(``If petitioner's purpose is to assure within its student body some
specified percentage of a particular group merely because of its race or
ethnic origin, such a preferential purpose must be rejected \ldots{} as
facially invalid''). An interest ``linked to nothing other than
proportional representation of various races \ldots{} would support
indefinite use of racial classifications, employed first to obtain the
appropriate mixture of racial views and then to ensure that the
{[}program{]} continues to reflect that mixture.'' Metro Broadcasting,
Inc.~v. FCC, (O'Connor, J., dissenting). And as we held in Fisher I, "
`{[}r{]}acial balancing is not transformed from ``patently
unconstitutional'' to a compelling state interest simply by relabeling
it ``racial diversity.''\,' " 570 U.S., at --------, 133 S.Ct. (quoting
Parents Involved, ).

The record here demonstrates the pitfalls inherent in racial balancing.
Although UT claims an interest in the educational benefits of diversity,
it appears to have paid little attention to anything other than the
number of minority students on its campus and in its classrooms. UT's
2004 Proposal illustrates this approach by repeatedly citing numerical
assessments of the racial makeup of the student body and various classes
as the justification for adopting a race-conscious plan. See, e.g.,
Supp. App. 24a--26a, 30a. Instead of focusing on the benefits of
diversity, UT seems to have resorted to a simple racial census.

The majority, for its part, claims that ``{[}a{]}lthough demographics
alone are by no means dispositive, they do have some value as a gauge of
the University's ability to enroll students who can offer
underrepresented perspectives.'' But even if UT merely ``view{[}s{]} the
demographic disparity as cause for concern,'' Brief for United States as
Amicus Curiae 29, and is seeking only to reduce---rather than
eliminate---the disparity, that undefined goal cannot be properly
subjected to strict scrutiny. In that case, there is simply no way for a
court to know what specific demographic interest UT is pursuing, why a
race-neutral alternative could not achieve that interest, and when that
demographic goal would be satisfied. If a demographic discrepancy can
serve as ``a gauge'' that justifies the use of racial discrimination, --
2212, then racial discrimination can be justified on that basis until
demographic parity is reached. There is no logical stopping point short
of patently unconstitutional racial balancing. Demographic disparities
thus cannot be used to satisfy strict scrutiny here. See Croson,
(rejecting a municipality's assertion that its racial set-aside program
was justified in light of past discrimination because that assertion had
" `no logical stopping point' " and could continue until the percentage
of government contracts awarded to minorities ``mirrored the percentage
of minorities in the population as a whole''); Wygant v. Jackson Bd. of
Ed., (plurality opinion) (rejecting the government's asserted interest
because it had ``no logical stopping point'').

The other major explanation UT offered in the Proposal was its desire to
promote classroom diversity. The Proposal stressed that UT ``has not
reached a critical mass at the classroom level .'' Supp. App. 24a
(emphasis added); see also a, 25a, 39a; App. 316a. In support of this
proposition, UT relied on a study of select classes containing five or
more students. As noted above, the study indicated that 52\% of these
classes had no African--Americans, 16\% had no Asian--Americans, and
12\% had no Hispanics. Supp. App. 26a. The study further suggested that
only 21\% of these classes had two or more African--Americans, 67\% had
two or more Asian--Americans, and 70\% had two or more Hispanics. See
Based on this study, UT concluded that it had a ``compelling educational
interest'' in employing racial preferences to ensure that it did not
``have large numbers of classes in which there are no students---or only
a single student---of a given underrepresented race or ethnicity.'' a.

UT now equivocates, disclaiming any discrete interest in classroom
diversity. See Brief for Respondents 26--27. Instead, UT has taken the
position that the lack of classroom diversity was merely a ``red flag
that UT had not yet fully realized'' ``the constitutionally permissible
educational benefits of diversity.'' Brief for Respondents in
No.~11--345. But UT has failed to identify the level of classroom
diversity it deems sufficient, again making it impossible to apply
strict scrutiny. A reviewing court cannot determine whether UT's
race-conscious program was necessary to remove the so-called ``red
flag'' without understanding the precise nature of that goal or knowing
when the ``red flag'' will be considered to have disappeared.

Putting aside UT's effective abandonment of its interest in classroom
diversity, the evidence cited in support of that interest is woefully
insufficient to show that UT's race-conscious plan was necessary to
achieve the educational benefits of a diverse student body. As far as
the record shows, UT failed to even scratch the surface of the available
data before reflexively resorting to racial preferences. For instance,
because UT knows which students were admitted through the Top Ten
Percent Plan and which were not, as well as which students enrolled in
which classes, it would seem relatively easy to determine whether Top
Ten Percent students were more or less likely than holistic admittees to
enroll in the types of classes where diversity was lacking. But UT never
bothered to figure this out. See (acknowledging that UT submitted no
evidence regarding ``how students admitted solely based on their class
rank differ in their contribution to diversity from students admitted
through holistic review''). Nor is there any indication that UT
instructed admissions officers to search for African--American and
Hispanic applicants who would fill particular gaps at the classroom
level. Given UT's failure to present such evidence, it has not
demonstrated that its race-conscious policy would promote classroom
diversity any better than race-neutral options, such as expanding the
Top Ten Percent Plan or using race-neutral holistic admissions.

Moreover, if UT is truly seeking to expose its students to a diversity
of ideas and perspectives, its policy is poorly tailored to serve that
end. UT's own study---which the majority touts as the best ``nuanced
quantitative data'' supporting UT's position, ---demonstrated that
classroom diversity was more lacking for students classified as
Asian--American than for those classified as Hispanic. Supp. App. 26a.
But the UT plan discriminates against Asian--American students. UT is
apparently unconcerned that Asian--Americans ``may be made to feel
isolated or may be seen as \ldots{} `spokesperson{[}s{]}' of their race
or ethnicity.'' a; see a. And unless the University is engaged in
unconstitutional racial balancing based on Texas demographics (where
Hispanics outnumber Asian--Americans), see Part II--C--1it seemingly
views the classroom contributions of Asian--American students as less
valuable than those of Hispanic students. In UT's view, apparently,
``Asian Americans are not worth as much as Hispanics in promoting
`cross-racial understanding,' breaking down `racial stereotypes,' and
enabling students to `better understand persons of different races.'\,''
Brief for Asian American Legal Foundation et al.~as Amici Curiae 11
(representing 117 Asian--American organizations). The majority opinion
effectively endorses this view, crediting UT's reliance on the classroom
study as proof that the University assessed its need for racial
discrimination (including racial discrimination that undeniably harms
Asian--Americans) ``with care.'' While both the majority and the Fifth
Circuit rely on UT's classroom study, see ; --659, they completely
ignore its finding that Hispanics are better represented than
Asian--Americans in UT classrooms. In fact, they act almost as if
Asian--American students do not exist. See -- 2212 (mentioning
Asian--Americans only a single time outside of parentheticals, and not
in the context of the classroom study); (mentioning Asian--Americans
only a single time). Only the District Court acknowledged the impact of
UT's policy on Asian--American students. But it brushed aside this
impact, concluding---astoundingly---that UT can pick and choose which
racial and ethnic groups it would like to favor. According to the
District Court, ``nothing in Grutter requires a university to give equal
preference to every minority group,'' and UT is allowed ``to exercise
its discretion in determining which minority groups should benefit from
the consideration of race.'' 645 F.Supp d.

This reasoning, which the majority implicitly accepts by blessing UT's
reliance on the classroom study, places the Court on the ``tortuous''
path of ``decid{[}ing{]} which races to favor.'' Metro Broadcasting
(KENNEDY, J., dissenting). And the Court's willingness to allow this
``discrimination against individuals of Asian descent in UT admissions
is particularly troubling, in light of the long history of
discrimination against Asian Americans, especially in education.'' Brief
for Asian American Legal Foundation et al.~as Amici Curiae 6; see also,
e.g., --17 (discussing the placement of Chinese--Americans in "
`separate but equal' " public schools); Gong Lum v. Rice--82, (holding
that a 9--year--old Chinese--American girl could be denied entry to a
``white'' school because she was ``a member of the Mongolian or yellow
race''). In sum, ``{[}w{]}hile the Court repeatedly refers to the
preferences as favoring `minorities,' \ldots{} it must be emphasized
that the discriminatory policies upheld today operate to exclude''
Asian--American students, who ``have not made {[}UT's{]} list'' of
favored groups. Metro Broadcasting (KENNEDY, J., dissenting).

Perhaps the majority finds discrimination against Asian--American
students benign, since Asian--Americans are ``overrepresented'' at UT.
645 F.Supp d.~But ``{[}h{]}istory should teach greater humility.'' Metro
Broadcasting, (O'Connor, J., dissenting). " `{[}B{]}enign' carries with
it no independent meaning, but reflects only acceptance of the current
generation's conclusion that a politically acceptable burden, imposed on
particular citizens on the basis of race, is reasonable." Where, as
here, the government has provided little explanation for why it needs to
discriminate based on race, " `there is simply no way of determining
what classifications are ``benign'' \ldots{} and what classifications
are in fact motivated by illegitimate notions of racial inferiority or
simple racial politics.' " Parents Involved, (opinion of KENNEDY, J.)
(quoting Croson, (plurality opinion of O'Connor, J.)). By accepting the
classroom study as proof that UT satisfied strict scrutiny, the majority
``move{[}s{]} us from `separate but equal' to `unequal but benign.'\,''
Metro Broadcasting (KENNEDY, J., dissenting).

In addition to demonstrating that UT discriminates against
Asian--American students, the classroom study also exhibits UT's use of
a few crude, overly simplistic racial and ethnic categories. Under the
UT plan, both the favored and the disfavored groups are broad and
consist of students from enormously diverse backgrounds. See Supp. App.
30a; see also Fisher I, 570 U.S., at --------, 133 S.Ct. (``five
predefined racial categories''). Because ``{[}c{]}rude measures of this
sort threaten to reduce {[}students{]} to racial chits,'' Parents
Involved, (opinion of KENNEDY, J.), UT's reliance on such measures
further undermines any claim based on classroom diversity statistics,
see (majority opinion) (criticizing school policies that viewed race in
rough ``white/nonwhite'' or ``black/`other'\,'' terms); (opinion of
KENNEDY, J.) (faulting government for relying on ``crude racial
categories''); Metro Broadcasting, n.~1 (KENNEDY, J., dissenting)
(concluding that " `the very attempt to define with precision a
beneficiary's qualifying racial characteristics is repugnant to our
constitutional ideals,' " and noting that if the government " `is to
make a serious effort to define racial classes by criteria that can be
administered objectively, it must study precedents such as the First
Regulation to the Reichs Citizenship Law of November 14, 1935' ").

For example, students labeled ``Asian American,'' Supp. App. 26a,
seemingly include ``individuals of Chinese, Japanese, Korean,
Vietnamese, Cambodian, Hmong, Indian and other backgrounds comprising
roughly 60\% of the world's population,'' Brief for Asian American Legal
Foundation et al.~as Amici Curiae, O.T. 2012, No.~11--345, p.~28. It
would be ludicrous to suggest that all of these students have similar
backgrounds and similar ideas and experiences to share. So why has UT
lumped them together and concluded that it is appropriate to
discriminate against Asian--American students because they are
``overrepresented'' in the UT student body? UT has no good answer. And
UT makes no effort to ensure that it has a critical mass of, say,
``Filipino Americans'' or ``Cambodian Americans.'' Tr. of Oral Arg. 52
(Oct.~10, 2012). As long as there are a sufficient number of ``Asian
Americans,'' UT is apparently satisfied.

UT's failure to provide any definition of the various racial and ethnic
groups is also revealing. UT does not specify what it means to be
``African--American,'' ``Hispanic,'' ``Asian American,'' ``Native
American,'' or ``White.'' Supp. App. 30a. And UT evidently labels each
student as falling into only a single racial or ethnic group, see, e.g.,
a--13a, 30a, 43a--44a, 71a, 156a--157a, 169a--170a, without explaining
how individuals with ancestors from different groups are to be
characterized. As racial and ethnic prejudice recedes, more and more
students will have parents (or grandparents) who fall into more than one
of UT's five groups. According to census figures, individuals describing
themselves as members of multiple races grew by 32\% from 2000 to 2010.
A recent survey reported that 26\% of Hispanics and 28\% of
Asian--Americans marry a spouse of a different race or ethnicity. UT's
crude classification system is ill suited for the more integrated
country that we are rapidly becoming. UT assumes that if an applicant
describes himself or herself as a member of a particular race or
ethnicity, that applicant will have a perspective that differs from that
of applicants who describe themselves as members of different groups.
But is this necessarily so? If an applicant has one grandparent,
great-grandparent, or great-great-grandparent who was a member of a
favored group, is that enough to permit UT to infer that this student's
classroom contribution will reflect a distinctive perspective or set of
experiences associated with that group? UT does not say. It instead
relies on applicants to ``classify themselves.'' Fisher I, 570 U.S., at
--------, This is an invitation for applicants to game the system.

Finally, it seems clear that the lack of classroom diversity is
attributable in good part to factors other than the representation of
the favored groups in the UT student population. UT offers an enormous
number of classes in a wide range of subjects, and it gives
undergraduates a very large measure of freedom to choose their classes.
UT also offers courses in subjects that are likely to have special
appeal to members of the minority groups given preferential treatment
under its challenged plan, and this of course diminishes the number of
other courses in which these students can enroll. See, e.g., Supp. App.
72a--73a (indicating that the representation of African--Americans and
Hispanics in UT classrooms varies substantially from major to major).
Having designed an undergraduate program that virtually ensures a lack
of classroom diversity, UT is poorly positioned to argue that this very
result provides a justification for racial and ethnic discrimination,
which the Constitution rarely allows.

UT's purported interest in intraracial diversity, or ``diversity within
diversity,'' Brief for Respondents 34, also falls short. At bottom, this
argument relies on the unsupported assumption that there is something
deficient or at least radically different about the African--American
and Hispanic students admitted through the Top Ten Percent Plan.

Throughout this litigation, UT has repeatedly shifted its position on
the need for intraracial diversity. Initially, in the 2004 Proposal, UT
did not rely on this alleged need at all. Rather, the Proposal
``examined two metrics---classroom diversity and demographic
disparities---that it concluded were relevant to its ability to provide
{[}the{]} benefits of diversity.'' Brief for United States as Amicus
Curiae 27--28. Those metrics looked only to the numbers of
African--Americans and Hispanics, not to diversity within each group.

On appeal to the Fifth Circuit and in Fisher I, however, UT began to
emphasize its intraracial diversity argument. UT complained that the Top
Ten Percent Law hinders its efforts to assemble a broadly diverse class
because the minorities admitted under that law are drawn largely from
certain areas of Texas where there are majority-minority schools. These
students, UT argued, tend to come from poor, disadvantaged families, and
the University would prefer a system that gives it substantial leeway to
seek broad diversity within groups of underrepresented minorities. In
particular, UT asserted a need for more African--American and Hispanic
students from privileged backgrounds. See, e.g., Brief for Respondents
in No.~11--345 (explaining that UT needs race-conscious admissions in
order to admit ``{[}t{]}he African--American or Hispanic child of
successful professionals in Dallas''); (claiming that privileged
minorities ``have great potential for serving as a `bridge' in promoting
cross-racial understanding, as well as in breaking down racial
stereotypes''); (intimating that the underprivileged minority students
admitted under the Top Ten Percent Plan ``reinforc{[}e{]}''
``stereotypical assumptions''); Tr. of Oral Arg. 43--45 (Oct.~10, 2012)
(``{[}A{]}lthough the percentage plan certainly helps with minority
admissions, by and large, the---the minorities who are admitted tend to
come from segregated, racially-identifiable schools,'' and ``we want
minorities from different backgrounds''). Thus, the Top Ten Percent Law
is faulted for admitting the wrong kind of African--American and
Hispanic students

The Fifth Circuit embraced this argument on remand, endorsing UT's
claimed need to enroll minorities from ``high-performing,''
``majority-white'' high schools. . According to the Fifth Circuit, these
more privileged minorities ``bring a perspective not captured by''
students admitted under the Top Ten Percent Law, who often come ``from
highly segregated, underfunded, and underperforming schools.'' For
instance, the court determined, privileged minorities ``can enrich the
diversity of the student body in distinct ways'' because such students
have ``higher levels of preparation and better prospects for admission
to UT Austin's more demanding colleges'' than underprivileged
minorities. ; see also Fisher, 631 F d, n.~149 (concluding that the Top
Ten Percent Plan ``widens the `credentials gap' between minority and
non-minority students at the University, which risks driving away
matriculating minority students from difficult majors like business or
the sciences'').

Remarkably, UT now contends that petitioner has ``fabricat{[}ed{]}'' the
argument that it is seeking affluent minorities. Brief for Respondents
2. That claim is impossible to square with UT's prior statements to this
Court in the briefing and oral argument in Fisher I Moreover, although
UT reframes its argument, it continues to assert that it needs
affirmative action to admit privileged minorities. For instance, UT's
brief highlights its interest in admitting ``{[}t{]}he black student
with high grades from Andover.'' Brief for Respondents 33. Similarly, at
oral argument, UT claimed that its ``interests in the educational
benefits of diversity would not be met if all of {[}the{]} minority
students were \ldots{} coming from depressed socioeconomic
backgrounds.'' Tr. of Oral Arg. 53 (Dec.~9, 2015); see also 45.

Ultimately, UT's intraracial diversity rationale relies on the baseless
assumption that there is something wrong with African--American and
Hispanic students admitted through the Top Ten Percent Plan, because
they are ``from the lower-performing, racially identifiable schools.'' ;
see --43 (explaining that ``the basis'' for UT's conclusion that it was
``not getting a variety of perspectives among African--Americans or
Hispanics'' was the fact that the Top Ten Percent Plan admits
underprivileged minorities from highly segregated schools). In effect,
UT asks the Court ``to assume''---without any evidence---``that
minorities admitted under the Top Ten Percent Law \ldots{} are somehow
more homogenous, less dynamic, and more undesirably stereotypical than
those admitted under holistic review.'' (Garza, J., dissenting). And
UT's assumptions appear to be based on the pernicious stereotype that
the African--Americans and Hispanics admitted through the Top Ten
Percent Plan only got in because they did not have to compete against
very many whites and Asian--Americans. See Tr. of Oral Arg. 42--43
(Dec.~9, 2015). These are ``the very stereotypical assumptions
{[}that{]} the Equal Protection Clause forbids.'' Miller, UT cannot
satisfy its burden by attempting to ``substitute racial stereotype for
evidence, and racial prejudice for reason.'' Calhoun v. United States,
568 U.S. --------, --------, (SOTOMAYOR, J., respecting denial of
certiorari).

In addition to relying on stereotypes, UT's argument that it needs
racial preferences to admit privileged minorities turns the concept of
affirmative action on its head. When affirmative action programs were
first adopted, it was for the purpose of helping the disadvantaged. See,
e.g., Bakke--275, (opinion of Powell, J.) (explaining that the school's
affirmative action program was designed ``to increase the
representation'' of " `economically and/or educationally disadvantaged'
applicants"). Now we are told that a program that tends to admit poor
and disadvantaged minority students is inadequate because it does not
work to the advantage of those who are more fortunate. This is
affirmative action gone wild.

It is also far from clear that UT's assumptions about the socioeconomic
status of minorities admitted through the Top Ten Percent Plan are even
remotely accurate. Take, for example, parental education. In 2008, when
petitioner applied to UT, approximately 79\% of Texans aged 25 years or
older had a high school diploma, 17\% had a bachelor's degree, and 8\%
had a graduate or professional degree. Dept. of Educ., Nat. Center for
Educ. Statistics, T. Snyder \& S. Dillow, Digest of Education Statistics
2010, p.~29 (2011). In contrast, 96\% of African--Americans admitted
through the Top Ten Percent Plan had a parent with a high school
diploma, 59\% had a parent with a bachelor's degree, and 26\% had a
parent with a graduate or professional degree. See UT, Office of
Admissions, Student Profile, Admitted Freshman Class of 2008, p.~8 (rev.
Aug.~1, 2012) (2008 Student Profile), online at
https://uteas.app.box.com/s/twqozsbm2vb9lhm14o0v0czvqs1ygzqr/1/7732448553/23476747441/1.
Similarly, 83\% of Hispanics admitted through the Top Ten Percent Plan
had a parent with a high school diploma, 42\% had a parent with a
bachelor's degree, and 21\% had a parent with a graduate or professional
degree. As these statistics make plain, the minorities that UT
characterizes as ``coming from depressed socioeconomic backgrounds,''
Tr. of Oral Arg. 53 (Dec.~9, 2015), generally come from households with
education levels exceeding the norm in Texas.

Or consider income levels. In 2008, the median annual household income
in Texas was \$49,453. United States Census Bureau, A. Noss, Household
Income for States: 2008 and 2009, p.~4 (2010), online at
https://www.census.gov/prod/2010pubs/acsbr09-2.pdf. The household income
levels for Top Ten Percent African--American and Hispanic admittees were
on par: Roughly half of such admittees came from households below the
Texas median, and half came from households above the median. See 2008
Student Profile 6. And a large portion of these admittees are from
households with income levels far exceeding the Texas median.
Specifically, 25\% of African--Americans and 27\% of Hispanics admitted
through the Top Ten Percent Plan in 2008 were raised in households with
incomes exceeding \$80,000. In light of this evidence, UT's actual
argument is not that it needs affirmative action to ensure that its
minority admittees are representative of the State of Texas. Rather, UT
is asserting that it needs affirmative action to ensure that its
minority students disproportionally come from families that are
wealthier and better educated than the average Texas family.

In addition to using socioeconomic status to falsely denigrate the
minority students admitted through the Top Ten Percent Plan, UT also
argues that such students are academically inferior. See, e.g., Brief
for Respondents in No.~11--345 (``{[}T{]}he top 10\% law systematically
hinders UT's efforts to assemble a class that is \ldots{} academically
excellent''). ``On average,'' UT claims, ``African--American and
Hispanic holistic admits have higher SAT scores than their Top 10\%
counterparts.'' Brief for Respondents 43, n.~8. As a result, UT argues
that it needs race-conscious admissions to enroll academically superior
minority students with higher SAT scores. Regrettably, the majority
seems to embrace this argument as well. See (``{[}T{]}he Equal
Protection Clause does not force universities to choose between a
diverse student body and a reputation for academic excellence'').

This argument fails for a number of reasons. First, it is simply not
true that Top Ten Percent minority admittees are academically inferior
to holistic admittees. In fact, as UT's president explained in 2000,
``top 10 percent high school students make much higher grades in college
than non-top 10 percent students,'' and ``{[}s{]}trong academic
performance in high school is an even better predictor of success in
college than standardized test scores.'' App. 393a--394a; see also
Lavergne Deposition 41--42 (agreeing that ``it's generally true that
students admitted pursuant to HB 588 {[}the Top Ten Percent Law{]} have
a higher level of academic performance at the University than students
admitted outside of HB 588''). Indeed, the statistics in the record
reveal that, for each year between 2003 and 2007, African--American
in-state freshmen who were admitted under the Top Ten Percent Law earned
a higher mean grade point average than those admitted outside of the Top
Ten Percent Law. Supp. App. 164a. The same is true for Hispanic
students. a. These conclusions correspond to the results of nationwide
studies showing that high school grades are a better predictor of
success in college than SAT scores.

It is also more than a little ironic that UT uses the SAT, which has
often been accused of reflecting racial and cultural bias, as a reason
for dissatisfaction with poor and disadvantaged African--American and
Hispanic students who excel both in high school and in college. Even if
the SAT does not reflect such bias (and I am ill equipped to express a
view on that subject), SAT scores clearly correlate with wealth.

UT certainly has a compelling interest in admitting students who will
achieve academic success, but it does not follow that it has a
compelling interest in maximizing admittees' SAT scores. Approximately
850 4--year--degree institutions do not require the SAT or ACT as part
of the admissions process. See J. Soares, SAT Wars: The Case for
Test--Optional College Admissions 2 (2012). This includes many excellent
schools. To the extent that intraracial diversity refers to something
other than admitting privileged minorities and minorities with higher
SAT scores, UT has failed to define that interest with any clarity. UT
``has not provided any concrete targets for admitting more minority
students possessing {[}the{]} unique qualitative-diversity
characteristics'' it desires. (Garza, J., dissenting). Nor has UT
specified which characteristics, viewpoints, and life experiences are
supposedly lacking in the African--Americans and Hispanics admitted
through the Top Ten Percent Plan. In fact, because UT administrators
make no collective, qualitative assessment of the minorities admitted
automatically, they have no way of knowing which attributes are missing.
See (admitting that there is no way of knowing ``how students admitted
solely based on their class rank differ in their contribution to
diversity from students admitted through holistic review''); (Garza, J.,
dissenting) (``The University does not assess whether Top Ten Percent
Law admittees exhibit sufficient diversity within diversity, whether the
requisite `change agents' are among them, and whether these admittees
are able, collectively or individually, to combat pernicious
stereotypes''). Furthermore, UT has not identified ``when, if ever, its
goal (which remains undefined) for qualitative diversity will be
reached.'' UT's intraracial diversity rationale is thus too imprecise to
permit strict scrutiny analysis.

Finally, UT's shifting positions on intraracial diversity, and the fact
that intraracial diversity was not emphasized in the Proposal, suggest
that it was not ``the actual purpose underlying the discriminatory
classification.'' Mississippi Univ. for Women v. Hogan, Instead, it
appears to be a post hoc rationalization.

UT also alleges---and the majority embraces---an interest in avoiding
``feelings of loneliness and isolation'' among minority students. ; see
Brief for Respondents 7--8, 38--39. In support of this argument, they
cite only demographic data and anecdotal statements by UT officials that
some students (we are not told how many) feel ``isolated.'' This vague
interest cannot possibly satisfy strict scrutiny.If UT is seeking
demographic parity to avoid isolation, that is impermissible racial
balancing. See Part II--C--1. And linking racial loneliness and
isolation to state demographics is illogical. Imagine, for example, that
an African--American student attends a university that is 20\%
African--American. If racial isolation depends on a comparison to state
demographics, then that student is more likely to feel isolated if the
school is located in Mississippi (which is 37 \% African--American) than
if it is located in Montana (which is 0 \% African--American). See
United States Census Bureau, QuickFacts, online at
https://www.census.gov/quickfacts/table/PST045215/28,30. In reality,
however, the student may feel---if anything---less isolated in
Mississippi, where African--Americans are more prevalent in the
population at large.

If, on the other hand, state demographics are not driving UT's interest
in avoiding racial isolation, then its treatment of Asian--American
students is hard to understand. As the District Court noted, ``the gross
number of Hispanic students attending UT exceeds the gross number of
Asian--American students.'' 645 F.Supp d.~In 2008, for example, UT
enrolled 1,338 Hispanic freshmen and 1,249 Asian--American freshmen.
Supp. App. 156a. UT never explains why the Hispanic students---but not
the Asian--American students---are isolated and lonely enough to receive
an admissions boost, notwithstanding the fact that there are more
Hispanics than Asian--Americans in the student population. The anecdotal
statements from UT officials certainly do not indicate that Hispanics
are somehow lonelier than Asian--Americans.

Ultimately, UT has failed to articulate its interest in preventing
racial isolation with any clarity, and it has provided no clear
indication of how it will know when such isolation no longer exists.
Like UT's purported interests in demographic parity, classroom
diversity, and intraracial diversity, its interest in avoiding racial
isolation cannot justify the use of racial preferences.

Even assuming UT is correct that, under Grutter, it need only cite a
generic interest in the educational benefits of diversity, its plan
still fails strict scrutiny because it is not narrowly tailored. Narrow
tailoring requires ``a careful judicial inquiry into whether a
university could achieve sufficient diversity without using racial
classifications.'' Fisher I, 570 U.S., at --------, ``If a `''nonracial
approach \ldots{} could promote the substantial interest about as well
and at tolerable administrative expense," ' then the university may not
consider race." --------, 133 S.Ct. (citations omitted). Here, there is
no evidence that race-blind, holistic review would not achieve UT's
goals at least ``about as well'' as UT's race-based policy. In addition,
UT could have adopted other approaches to further its goals, such as
intensifying its outreach efforts, uncapping the Top Ten Percent Law, or
placing greater weight on socioeconomic factors.

The majority argues that none of these alternatives is ``a workable
means for the University to attain the benefits of diversity it
sought.'' Tellingly, however, the majority devotes only a single,
conclusory sentence to the most obvious race-neutral alternative:
race-blind, holistic review that considers the applicant's unique
characteristics and personal circumstances. See Under a system that
combines the Top Ten Percent Plan with race-blind, holistic review, UT
could still admit ``the star athlete or musician whose grades suffered
because of daily practices and training,'' the ``talented young
biologist who struggled to maintain above-average grades in humanities
classes,'' and the ``student whose freshman-year grades were poor
because of a family crisis but who got herself back on track in her last
three years of school.'' All of these unique circumstances can be
considered without injecting race into the process. Because UT has
failed to provide any evidence whatsoever that race-conscious holistic
review will achieve its diversity objectives more effectively than
race-blind holistic review, it cannot satisfy the heavy burden imposed
by the strict scrutiny standard.

The fact that UT's racial preferences are unnecessary to achieve its
stated goals is further demonstrated by their minimal effect on UT's
diversity. In 2004, when race was not a factor, 3 \% of non-Top Ten
Percent Texas enrollees were African--American and 11 \% were Hispanic.
See Supp. App. 157a. It would stand to reason that at least the same
percentages of African--American and Hispanic students would have been
admitted through holistic review in 2008 even if race were not a factor.
If that assumption is correct, then race was determinative for only 15
African--American students and 18 Hispanic students in 2008
(representing 0 \% and 0 \%, respectively, of the total enrolled
first-time freshmen from Texas high schools). See

The majority contends that ``{[}t{]}he fact that race consciousness
played a role in only a small portion of admissions decisions should be
a hallmark of narrow tailoring, not evidence of unconstitutionality.''
This argument directly contradicts this Court's precedent. Because
racial classifications are " `a highly suspect tool,' " Grutter, they
should be employed only ``as a last resort,'' Croson, (opinion of
KENNEDY, J.); see also Grutter, (``{[}R{]}acial classifications, however
compelling their goals, are potentially so dangerous that they may be
employed no more broadly than the interest demands''). Where, as here,
racial preferences have only a slight impact on minority enrollment, a
race-neutral alternative likely could have reached the same result. See
Parents Involved--734, (holding that the ``minimal effect'' of school
districts' racial classifications ``casts doubt on the necessity of
using {[}such{]} classifications'' and ``suggests that other means {[}of
achieving their objectives{]} would be effective''). As Justice KENNEDY
once aptly put it, ``the small number of {[}students{]} affected
suggests that the schoo{[}l{]} could have achieved {[}its{]} stated ends
through different means.'' (opinion concurring in part and concurring in
judgment). And in this case, a race-neutral alternative could accomplish
UT's objectives without gratuitously branding the covers of tens of
thousands of applications with a bare racial stamp and ``tell{[}ing{]}
each student he or she is to be defined by race.'' The majority purports
to agree with much of the above analysis. The Court acknowledges that "
`because racial characteristics so seldom provide a relevant basis for
disparate treatment,' " " `{[}r{]}ace may not be considered {[}by a
university{]} unless the admissions process can withstand strict
scrutiny.' " The Court admits that the burden of proof is on UT, and
that ``a university bears a heavy burden in showing that it had not
obtained the educational benefits of diversity before it turned to a
race-conscious plan,'' And the Court recognizes that the record here is
``almost devoid of information about the students who secured admission
to the University through the Plan,'' and that ``{[}t{]}he Court thus
cannot know how students admitted solely based on their class rank
differ in their contribution to diversity from students admitted through
holistic review.'' This should be the end of the case: Without
identifying what was missing from the African--American and Hispanic
students it was already admitting through its race-neutral process, and
without showing how the use of race-based admissions could rectify the
deficiency, UT cannot demonstrate that its procedure is narrowly
tailored.

Yet, somehow, the majority concludes that petitioner must lose as a
result of UT's failure to provide evidence justifying its decision to
employ racial discrimination. Tellingly, the Court frames its analysis
as if petitioner bears the burden of proof here. See -- 2225. But it is
not the petitioner's burden to show that the consideration of race is
unconstitutional. To the extent the record is inadequate, the
responsibility lies with UT. For ``{[}w{]}hen a court subjects
governmental action to strict scrutiny, it cannot construe ambiguities
in favor of the State,'' Parents Involved, (opinion of KENNEDY, J.),
particularly where, as here, the summary judgment posture obligates the
Court to view the facts in the light most favorable to petitioner, see
Matsushita Elec. Industrial Co.~v. Zenith Radio Corp., Given that the
University bears the burden of proof, it is not surprising that UT never
made the argument that it should win based on the lack of evidence. UT
instead asserts that ``if the Court believes there are any deficiencies
in {[}the{]} record that cast doubt on the constitutionality of UT's
policy, the answer is to order a trial, not to grant summary judgment.''
Brief for Respondents 51; see also --53 (``{[}I{]}f this Court has any
doubts about how the Top 10\% Law works, or how UT's holistic plan
offsets the tradeoffs of the Top 10\% Law, the answer is to remand for a
trial''). Nevertheless, the majority cites three reasons for breaking
from the normal strict scrutiny standard. None of these is convincing.

First, the Court states that, while ``th{[}e{]} evidentiary gap perhaps
could be filled by a remand to the district court for further
factfinding'' in ``an ordinary case,'' that will not work here because
``{[}w{]}hen petitioner's application was rejected, \ldots{} the
University's combined percentage-plan/holistic-review approach to
admission had been in effect for just three years,'' so ``further
factfinding'' ``might yield little insight.'' This reasoning is
dangerously incorrect. The Equal Protection Clause does not provide a
3--year grace period for racial discrimination. Under strict scrutiny,
UT was required to identify evidence that race-based admissions were
necessary to achieve a compelling interest before it put them in
place---not three or more years after. See (``Petitioner is correct that
a university bears a heavy burden in showing that it had not obtained
the educational benefits of diversity before it turned to a
race-conscious plan'' (emphasis added)); Fisher I, 570 U.S., at
--------, 133 S.Ct. (``{[}S{]}trict scrutiny imposes on the university
the ultimate burden of demonstrating, before turning to racial
classifications, that available, workable race-neutral alternatives do
not suffice'' (emphasis added)). UT's failure to obtain actual evidence
that racial preferences were necessary before resolving to use them only
confirms that its decision to inject race into admissions was a
reflexive response to Grutter, and that UT did not seriously consider
whether race-neutral means would serve its goals as well as a race-based
process.

Second, in an effort to excuse UT's lack of evidence, the Court argues
that because ``the University lacks any authority to alter the role of
the Top Ten Percent Plan,'' ``it similarly had no reason to keep
extensive data on the Plan or the students admitted under
it---particularly in the years before Fisher I clarified the stringency
of the strict-scrutiny burden for a school that employs race-conscious
review.'' But UT has long been aware that it bears the burden of
justifying its racial discrimination under strict scrutiny. See, e.g.,
Brief for Respondents in No.~11--345 (``It is undisputed that UT's
consideration of race in its holistic admissions process triggers strict
scrutiny,'' and ``that inquiry is undeniably rigorous''). In light of
this burden, UT had every reason to keep data on the students admitted
through the Top Ten Percent Plan. Without such data, how could UT have
possibly identified any characteristics that were lacking in Top Ten
Percent admittees and that could be obtained via race-conscious
admissions? How could UT determine that employing a race-based process
would serve its goals better than, for instance, expanding the Top Ten
Percent Plan? UT could not possibly make such determinations without
studying the students admitted under the Top Ten Percent Plan. Its
failure to do so demonstrates that UT unthinkingly employed a race-based
process without examining whether the use of race was actually
necessary. This is not---as the Court claims---a ``good-faith
effor{[}t{]} to comply with the law.'' The majority's willingness to
cite UT's ``good faith'' as the basis for excusing its failure to adduce
evidence is particularly inappropriate in light of UT's well-documented
absence of good faith. Since UT described its admissions policy to this
Court in Fisher I, it has been revealed that this description was
incomplete. As explained in an independent investigation into UT
admissions, UT maintained a clandestine admissions system that evaded
public scrutiny until a former admissions officer blew the whistle in
2014. See Kroll, Inc., University of Texas at Austin---Investigation of
Admissions Practices and Allegations of Undue Influence 4 (Feb.~6, 2015)
(Kroll Report). Under this longstanding, secret process, university
officials regularly overrode normal holistic review to allow politically
connected individuals---such as donors, alumni, legislators, members of
the Board of Regents, and UT officials and faculty---to get family
members and other friends admitted to UT, despite having grades and
standardized test scores substantially below the median for admitted
students. --14; see also Blanchard \& Hoppe, Influential Texans Helped
Underqualified Students Get Into UT, Dallas Morning News, July 20, 2015,
online at
http://www.dallasnews.com/news/education/headlines/20150720-influential-texans-helped-underqualified-students-get-into-ut.ece
(``Dozens of highly influential Texans---including lawmakers,
millionaire donors and university regents---helped underqualified
students get into the University of Texas, often by writing to UT
officials, records show'').

UT officials involved in this covert process intentionally kept few
records and destroyed those that did exist. See, e.g., Kroll Report 43
(``Efforts were made to minimize paper trails and written lists during
this end-of-cycle process. At one meeting, the administrative assistants
tried not keeping any notes, but this proved difficult, so they took
notes and later shredded them. One administrative assistant usually
brought to these meetings a stack of index cards that were subsequently
destroyed''); see also (finding that ``written records or notes'' of the
secret admissions meetings ``are not maintained and are typically
shredded''). And in the course of this litigation, UT has been less than
forthright concerning its treatment of well-connected applicants.
Compare, e.g., Tr. of Oral Arg. 51 (Dec.~9, 2015) (``University of Texas
does not do legacy, Your Honor''), and App. 281a (``{[}O{]}ur legacy
policy is such that we don't consider legacy''), with Kroll Report 29
(discussing evidence that ``alumni/legacy influence'' ``results each
year in certain applicants receiving a competitive boost or special
consideration in the admissions process,'' and noting that this is ``an
aspect of the admissions process that does not appear in the public
representations of UT--Austin's admissions process''). Despite UT's
apparent readiness to mislead the public and the Court, the majority is
``willing to be satisfied by {[}UT's{]} profession of its own good
faith.'' Grutter, (KENNEDY, J., dissenting). Notwithstanding the
majority's claims to the contrary, UT should have access to plenty of
information about ``how students admitted solely based on their class
rank differ in their contribution to diversity from students admitted
through holistic review.'' UT undoubtedly knows which students were
admitted through the Top Ten Percent Plan and which were admitted
through holistic review. See, e.g., Supp. App. 157a. And it undoubtedly
has a record of all of the classes in which these students enrolled.
See, e.g., UT, Office of the Registrar, Transcript---Official, online at
https://registrar.utexas.edu/students/transcripts-official (instructing
graduates on how to obtain a transcript listing a ``comprehensive
record'' of classes taken). UT could use this information to demonstrate
whether the Top Ten Percent minority admittees were more or less likely
than the holistic minority admittees to choose to enroll in the courses
lacking diversity.

In addition, UT assigns PAI scores to all students---including those
admitted through the Top Ten Percent Plan---for purposes of admission to
individual majors. Accordingly, all students must submit a full
application containing essays, letters of recommendation, a resume, a
list of courses taken in high school, and a description of any
extracurricular activities, leadership experience, or special
circumstances. See App. 212a--214a; 235a--236a; , n.~14 (Garza, J.,
dissenting). Unless UT has destroyed these files, it could use them to
compare the unique personal characteristics of Top Ten minority
admittees with those of holistic minority admittees, and to determine
whether the Top Ten admittees are, in fact, less desirable than the
holistic admittees. This may require UT to expend some resources, but
that is an appropriate burden in light of the strict scrutiny standard
and the fact that all of the relevant information is in UT's possession.
The cost of factfinding is a strange basis for awarding a victory to UT,
which has a huge budget, and a loss to petitioner, who does not.

Finally, while I agree with the majority and the Fifth Circuit that
Fisher I significantly changed the governing law by clarifying the
stringency of the strict scrutiny standard, that does not excuse UT from
meeting that heavy burden. In Adarand, for instance, another case in
which the Court clarified the rigor of the strict scrutiny standard, the
Court acknowledged that its decision ``alter{[}ed{]} the playing field
in some important respects.'' 515 U.S., As a result, it ``remand{[}ed{]}
the case to the lower courts for further consideration in light of the
principles {[}it had{]} announced .'' (emphasis added). In other words,
the Court made clear that---notwithstanding the shift in the law---the
government had to meet the clarified burden it was announcing. The Court
did not embrace the notion that its decision to alter the stringency of
the strict scrutiny standard somehow allowed the government to
automatically prevail.

Third, the majority notes that this litigation has persisted for many
years, that petitioner has already graduated from another college, that
UT's policy may have changed over time, and that this case may offer
little prospective guidance. At most, these considerations counsel in
favor of dismissing this case as improvidently granted. But see, e.g.,
Gratz, and n.~1, 260--262, (rejecting the dissent's argument that,
because the case had already persisted long enough for the petitioners
to graduate from other schools, the case should be dismissed); (Stevens,
J., dissenting). None of these considerations has any bearing whatsoever
on the merits of this suit. The majority cannot side with UT simply
because it is tired of this case.

It is important to understand what is and what is not at stake in this
case. What is not at stake is whether UT or any other university may
adopt an admissions plan that results in a student body with a broad
representation of students from all racial and ethnic groups. UT
previously had a race-neutral plan that it claimed had ``effectively
compensated for the loss of affirmative action,'' App. 396a, and UT
could have taken other steps that would have increased the diversity of
its admitted students without taking race or ethnic background into
account.

What is at stake is whether university administrators may justify
systematic racial discrimination simply by asserting that such
discrimination is necessary to achieve ``the educational benefits of
diversity,'' without explaining---much less proving---why the
discrimination is needed or how the discriminatory plan is well crafted
to serve its objectives. Even though UT has never provided any coherent
explanation for its asserted need to discriminate on the basis of race,
and even though UT's position relies on a series of unsupported and
noxious racial assumptions, the majority concludes that UT has met its
heavy burden. This conclusion is remarkable---and remarkably wrong.

Because UT has failed to satisfy strict scrutiny, I respectfully
dissent.

\hypertarget{gender-and-intermediate-scrutiny}{%
\subsection{Gender and Intermediate
Scrutiny}\label{gender-and-intermediate-scrutiny}}

\hypertarget{craig-v.-boren}{%
\subsubsection{Craig v. Boren}\label{craig-v.-boren}}

\textbf{Mr.~Justice BRENNAN delivered the opinion of the Court.} The
interaction of two sections of an Oklahoma statute, Okla.Stat., Tit. 37,
§§ 241 and 245 (1958 and Supp ),1 prohibits the sale of
``nonintoxicating'' 3 \% beer to males under the age of 21 and to
females under the age of 18. The question to be decided is whether such
a gender-based differential constitutes a denial to males 18-20 years of
age of the equal protection of the laws in violation of the Fourteenth
Amendment.

Before 1972, Oklahoma defined the commencement of civil majority at age
18 for females and age 21 for males. Okla.Stat., Tit. 15, § 13 (1972 and
Supp ). In contrast, females were held criminally responsible as adults
at age 18 and males at age 16. Okla.Stat., Tit. 10, § 1101(a) (Supp ).
After the Court of Appeals for the Tenth Circuit held in 1972, on the
authority of Reed v. Reedthat the age distinction was unconstitutional
for purposes of establishing criminal responsibility as adults, Lamb v.
Brown, the Oklahoma Legislature fixed age 18 as applicable to both males
and females. Okla.Stat., Tit. 10, § 1101(a) (Supp ). In 1972, 18 also
was established as the age of majority for males and females in civil
matters, Okla.Stat., Tit. 15, § 13 (1972 and Supp ), except that §§ 241
and 245 of the 3 \% beer statute were simultaneously codified to create
an exception to the gender-free rule.

Reed v. Reed has also provided the underpinning for decisions that have
invalidated statutes employing gender as an inaccurate proxy for other,
more germane bases of classification. Hence, ``archaic and overbroad''
generalizations, Schlesinger v. Ballardconcerning the financial position
of servicewomen, Frontiero v. Richardsonn. 23, and working women,
Weinberger v. Wiesenfeld, could not justify use of a gender line in
determining eligibility for certain governmental entitlements.
Similarly, increasingly outdated misconceptions concerning the role of
females in the home rather than in the ``marketplace and world of
ideas'' were rejected as loose-fitting characterizations incapable of
supporting state statutory schemes that were premised upon their
accuracy. Stanton v. Stanton; Taylor v. Louisianan. 17, In light of the
weak congruence between gender and the characteristic or trait that
gender purported to represent, it was necessary that the legislatures
choose either to realign their substantive laws in a gender-neutral
fashion, or to adopt procedures for identifying those instances where
the sex-centered generalization actually comported with fact. See, e.
g., Stanley v. Illinois S.Ct.; cf.~Cleveland Board of Education v.
LaFleur,

In this case, too, ``Reed, we feel is controlling . .,'' Stanton v.
StantonWe turn then to the question whether, under Reed, the difference
between males and females with respect to the purchase of 3 \% beer
warrants the differential in age drawn by the Oklahoma statute. We
conclude that it does not.

The District Court recognized that Reed v. Reed was controlling. In
applying the teachings of that case, the court found the requisite
important governmental objective in the traffic-safety goal proffered by
the Oklahoma Attorney General. It then concluded that the statistics
introduced by the appellees established that the gender-based
distinction was substantially related to achievement of that goal.

We accept for purposes of discussion the District Court's identification
of the objective underlying §§ 241 and 245 as the enhancement of traffic
safety. 7 Clearly, the protection of public health and safety represents
an important function of state and local governments. However,
appellees' statistics in our view cannot support the conclusion that the
gender-based distinction closely serves to achieve that objective and
therefore the distinction cannot under Reed withstand equal protection
challenge.

The appellees introduced a variety of statistical surveys. First, an
analysis of arrest statistics for 1973 demonstrated that 18-20-year-old
male arrests for ``driving under the influence'' and ``drunkenness''
substantially exceeded female arrests for that same age period
Similarly, youths aged 17-21 were found to be overrepresented among
those killed or injured in traffic accidents, with males again
numerically exceeding females in this regard Third, a random roadside
survey in Oklahoma City revealed that young males were more inclined to
drive and drink beer than were their female counterparts. 10Fourth,
Federal Bureau of Investigation nationwide statistics exhibited a
notable increase in arrests for ``driving under the influence.'' 11
Finally, statistical evidence gathered in other jurisdictions,
particularly Minnesota and Michigan, was offered to corroborate
Oklahoma's experience by indicating the pervasiveness of youthful
participation in motor vehicle accidents following the imbibing of
alcohol. Conceding that ``the case is not free from doubt,'' 399
F.Supp., the District Court nonetheless concluded that this statistical
showing substantiated ``a rational basis for the legislative judgment
underlying the challenged classification.''

Even were this statistical evidence accepted as accurate, it
nevertheless offers only a weak answer to the equal protection question
presented here. The most focused and relevant of the statistical
surveys, arrests of 18-20-year-olds for alcohol-related driving
offenses, exemplifies the ultimate unpersuasiveness of this evidentiary
record. Viewed in terms of the correlation between sex and the actual
activity that Oklahoma seeks to regulate driving while under the
influence of alcohol the statistics broadly establish that \% of females
and 2\% of males in that age group were arrested for that offense. While
such a disparity is not trivial in a statistical sense, it hardly can
form the basis for employment of a gender line as a classifying device.
Certainly if maleness is to serve as a proxy for drinking and driving, a
correlation of 2\% must be considered an unduly tenuous ``fit.'' 12
Indeed, prior cases have consistently rejected the use of sex as a
decisionmaking factor even though the statutes in question certainly
rested on far more predictive empirical relationships than this.

Moreover, the statistics exhibit a variety of other shortcomings that
seriously impugn their value to equal protection analysis. Setting aside
the obvious methodological problems,14 the surveys do not adequately
justify the salient features of Oklahoma's gender-based traffic-safety
law. None purports to measure the use and dangerousness of 3 \% beer as
opposed to alcohol generally, a detail that is of particular importance
since, in light of its low alcohol level, Oklahoma apparently considers
the 3 \% beverage to be ``nonintoxicating.'' Okla.Stat., Tit. 37, § 163
(1958); see State ex rel. Springer v. Bliss, 199 Okl. 198, 185 P d 220
(1947). Moreover, many of the studies, while graphically documenting the
unfortunate increase in driving while under the influence of alcohol,
make no effort to relate their findings to age-sex differentials as
involved here Indeed, the only survey that explicitly centered its
attention upon young drivers and their use of beer albeit apparently not
of the diluted 3 \% variety reached results that hardly can be viewed as
impressive in justifying either a gender or age classification There is
no reason to belabor this line of analysis. It is unrealistic to expect
either members of the judiciary or state officials to be well versed in
the rigors of experimental or statistical technique. But this merely
illustrates that proving broad sociological propositions by statistics
is a dubious business, and one that inevitably is in tension with the
normative philosophy that underlies the Equal Protection Clause Suffice
to say that the showing offered by the appellees does not satisfy us
that sex represents a legitimate, accurate proxy for the regulation of
drinking and driving. In fact, when it is further recognized that
Oklahoma's statute prohibits only the selling of 3 \% beer to young
males and not their drinking the beverage once acquired (even after
purchase by their 18-20-year-old female companions), the relationship
between gender and traffic safety becomes far too tenuous to satisfy
Reed's requirement that the gender-based difference be substantially
related to achievement of the statutory objective.

We hold, therefore, that under Reed, Oklahoma's 3 \% beer statute
invidiously discriminates against males 18-20 years of age.

\textbf{Mr.~Justice POWELL, concurring.}

I join the opinion of the Court as I am in general agreement with it. I
do have reservations as to some of the discussion concerning the
appropriate standard for equal protection analysis and the relevance of
the statistical evidence. Accordingly, I add this concurring statement.

With respect to the equal protection standard, I agree that Reed v.
Reedis the most relevant precedent. But I find it unnecessary, in
deciding this case, to read that decision as broadly as some of the
Court's language may imply. Reed and subsequent cases involving
gender-based classifications make clear that the Court subjects such
classifications to a more critical examination than is normally applied
when ``fundamental'' constitutional rights and ``suspect classes'' are
not present. * I view this as a relatively easy case. No one questions
the legitimacy or importance of the asserted governmental objective: the
promotion of highway safety. The decision of the case turns on whether
the state legislature, by the classification it has chosen, had adopted
a means that bears a " `fair and substantial relation' " to this
objective. quoting Royster Guano Co.~v. Virginia,

It seems to me that the statistics offered by appellees and relied upon
by the District Court do tend generally to support the view that young
men drive more, possibly are inclined to drink more, and for various
reasons are involved in more accidents than young women. Even so, I am
not persuaded that these facts and the inferences fairly drawn from them
justify this classification based on a three-year age differential
between the sexes, and especially one that it so easily circumvented as
to be virtually meaningless. Putting it differently, this gender-based
classification does not bear a fair and substantial relation to the
object of the legislation.

\textbf{Mr.~Justice STEVENS, concurring.}

There is only one Equal Protection Clause. It requires every State to
govern impartially. It does not direct the courts to apply one standard
of review in some cases and a different standard in other cases.
Whatever criticism may be leveled at a judicial opinion implying that
there are at least three such standards applies with the same force to a
double standard.

I am inclined to believe that what has become known as the two-tiered
analysis of equal protection claims does not describe a completely
logical method of deciding cases, but rather is a method the Court has
employed to explain decisions that actually apply a single standard in a
reasonably consistent fashion. I also suspect that a careful explanation
of the reasons motivating particular decisions may contribute more to an
identification of that standard than an attempt to articulate it in
all-encompassing terms. It may therefore be appropriate for me to state
the principal reasons which persuaded me to join the Court's opinion.

In this case, the classification is not as obnoxious as some the Court
has condemned,1 nor as inoffensive as some the Court has accepted. It is
objectionable because it is based on an accident of birth,2 because it
is a mere remnant of the now almost universally rejected tradition of
discriminating against males in this age bracket,3 and because, to the
extent it reflects any physical difference between males and females, it
is actually perverse. 4 The question then is whether the traffic safety
justification put forward by the State is sufficient to make an
otherwise offensive classification acceptable.

The classification is not totally irrational. For the evidence does
indicate that there are more males than females in this age bracket who
drive and also more who drink. Nevertheless, there are several reasons
why I regard the justification as unacceptable. It is difficult to
believe that the statute was actually intended to cope with the problem
of traffic safety,5 since it has only a minimal effect on access to a
not very intoxicating beverage and does not prohibit its consumption
Moreover, the empirical data submitted by the State accentuate the
unfairness of treating all 18-21-year-old males as inferior to their
female counterparts. The legislation imposes a restraint on 100\% of the
males in the class allegedly because about 2\% of them have probably
violated one or more laws relating to the consumption of alcoholic
beverages. 7 It is unlikely that this law will have a significant
deterrent effect either on that 2\% or on the law-abiding 98\%. But even
assuming some such slight benefit, it does not seem to me that an insult
to all of the young men of the State can be justified by visiting the
sins of the 2\% on the 98\%.

\textbf{Mr.~Justice REHNQUIST, dissenting.}

Most obviously unavailable to support any kind of special scrutiny in
this case, is a history or pattern of past discrimination, such as was
relied on by the plurality in Frontiero to support its invocation of
strict scrutiny. There is no suggestion in the Court's opinion that
males in this age group are in any way peculiarly disadvantaged, subject
to systematic discriminatory treatment, or otherwise in need of special
solicitude from the courts.

The Court does not discuss the nature of the right involved, and there
is no reason to believe that it sees the purchase of 3 \% beer as
implicating any important interest, let alone one that is
``fundamental'' in the constitutional sense of invoking strict scrutiny.
Indeed, the Court's accurate observation that the statute affects the
selling but not the drinking of 3 \% beer, further emphasizes the
limited effect that it has on even those persons in the age group
involved. There is, in sum, nothing about the statutory classification
involved here to suggest that it affects an interest, or works against a
group, which can claim under the Equal Protection Clause that it is
entitled to special judicial protection.

It is true that a number of our opinions contain broadly phrased dicta
implying that the same test should be applied to all classifications
based on sex, whether affecting females or males. E. g., Frontiero v.
Richardson S.Ct.; Reed v. Reed, However, before today, no decision of
this Court has applied an elevated level of scrutiny to invalidate a
statutory discrimination harmful to males, except where the statute
impaired an important personal interest protected by the Constitution
There being no such interest here, and there being no plausible argument
that this is a discrimination against females,2 the Court's reliance on
our previous sex-discrimination cases is ill-founded. It treats gender
classification as a talisman which without regard to the rights involved
or the persons affected calls into effect a heavier burden of judicial
review.

The Court's conclusion that a law which treats males less favorably than
females ``must serve important governmental objectives and must be
substantially related to achievement of those objectives'' apparently
comes out of thin air. The Equal Protection Clause contains no such
language, and none of our previous cases adopt that standard. I would
think we have had enough difficulty with the two standards of review
which our cases have recognized the norm of ``rational basis,'' and the
``compelling state interest'' required where a ``suspect
classification'' is involved so as to counsel weightily against the
insertion of still another ``standard'' between those two. How is this
Court to divine what objectives are important? How is it to determine
whether a particular law is ``substantially'' related to the achievement
of such objective, rather than related in some other way to its
achievement? Both of the phrases used are so diaphanous and elastic as
to invite subjective judicial preferences or prejudices relating to
particular types of legislation, masquerading as judgments whether such
legislation is directed at ``important'' objectives or, whether the
relationship to those objectives is ``substantial'' enough.

I would have thought that if this Court were to leave anything to
decision by the popularly elected branches of the Government, where no
constitutional claim other than that of equal protection is invoked, it
would be the decision as to what governmental objectives to be achieved
by law are ``important,'' and which are not. As for the second part of
the Court's new test, the Judicial Branch is probably in no worse
position than the Legislative or Executive Branches to determine if
there is any rational relationship between a classification and the
purpose which it might be thought to serve. But the introduction of the
adverb ``substantially'' requires courts to make subjective judgments as
to operational effects, for which neither their expertise nor their
access to data fits them. And even if we manage to avoid both confusion
and the mirroring of our own preferences in the development of this new
doctrine, the thousands of judges in other courts who must interpret the
Equal Protection Clause may not be so fortunate.

\hypertarget{michael-m.-v.-superior-court-of-sonoma-county}{%
\subsubsection{Michael M. v. Superior Court of Sonoma
County}\label{michael-m.-v.-superior-court-of-sonoma-county}}

450 U.S. 464 (1981)

\emph{Note on this case: the actual facts do not involve a mere
statutory rape, but a violent forcible rape. I have removed the
description of that crime in recognition of the fact that some students
may have experienced similar trauma, but it is worth noting the court's
utter and utterly unjustifiable failure to confront the violent nature
of the crime.}

\textbf{Justice Rehnquist announced the judgment of the Court and
delivered an opinion, in which The Chief Justice, Justice Stewart, and
Justice Powell joined.}

The question presented in this case is whether California's ``statutory
rape'' law, § 261 of the Cal. Penal Code Ann. (West Supp. 1981),
violates the Equal Protection Clause of the Fourteenth Amendment.
Section 261 defines unlawful sexual intercourse as ``an act of sexual
intercourse accomplished with a female not the wife of the perpetrator,
where the female is under the age of 18 years.'' The statute thus makes
men alone criminally liable for the act of sexual intercourse.

The Supreme Court held that ``section 261 discriminates on the basis of
sex because only females may be victims, and only males may violate the
section.'' 25 Cal. 3d 608, 611, 601 P. 2d 572, 574. The court then
subjected the classification to ``strict scrutiny,'' stating that it
must be justified by a compelling state interest. It found that the
classification was ``supported not by mere social convention but by the
immutable physiological fact that it is the female exclusively who can
become pregnant.'' Canvassing ``the tragic human costs of illegitimate
teenage pregnancies,'' including the large number of teenage abortions,
the increased medical risk associated with teenage .pregnancies, and the
social consequences of teenage childbearing, the court concluded that
the State has a compelling interest in preventing such pregnancies.
Because males alone can ``physiologically cause the result which the law
properly seeks to avoid,'' the court further held that the gender
classification was readily justified as a means of identifying offender
and victim. For the reasons stated below, we affirm the judgment of the
California Supreme Court.

As is evident from our opinions, the Court has had some difficulty in
agreeing upon the proper approach and analysis in cases involving
challenges to gender-based classifications. The issues posed by such
challenges range from issues of standing, see Orr v. Orr, to the
appropriate standard of judicial review for the substantive
classification. Unlike the California Supreme Court, we have not held
that gender-based classifications are ``inherently suspect'' and thus we
do not apply so-called ``strict scrutiny'' to those classifications. See
Stanton v. Stanton, 421 U. S. 7 (1975). Our cases have held, however,
that the traditional minimum rationality test takes on a somewhat
``sharper focus'' when gender-based classifications are challenged. See
Craig v. Borenn." (1976) (Powell, J., concurring). In Reed v. Reed, for
example, the Court stated that a gender-based classification will be
upheld if it bears a ``fair and substantial relationship'' to legitimate
state ends, while in Craig v. Boren, the Court restated the test to
require the classification to bear a ``substantial relationship'' to
``important governmental objectives.''

Underlying these decisions is the principle that a legislature may not
``make overbroad generalizations based on sex which are entirely
unrelated to any differences between men and women or which demean the
ability or social status of the affected class.'' Parham v. Hughes
(plurality opinion of Stewart, J.). But because the Equal Protection
Clause does not ``demand that a statute necessarily apply equally to all
persons'' or require `` 'things which are different in fact\ldots{} to
be treated in law as though they were the same/ '' Rinaldi v. Yeager,
quoting Tigner v. Texas, this Court has consistently upheld statutes
where the gender classification is not invidious, but rather
realistically reflects the fact that the sexes are not similarly
situated in certain circumstances. Parham v. Hughes; Califano v.
Webster; Schlesinger v. Ballard; Kahn v. Shevin. As the Court has
stated, a legislature may ``provide for the special problems of women.''
Weinberger v. Wiesenfeld.

Applying those principles to this case, the fact that the California
Legislature criminalized the act of illicit sexual intercourse with a
minor female is a sure indication of its intent or purpose to discourage
that conduct Precisely why the legislature desired that result is of
course somewhat less clear. This Court has long recognized that
``{[}¿Inquiries into congressional motives or purposes are a hazardous
matter,'' United States v. O'Brien (1968); Palmer v. Thompson, and the
search for the ``actual'' or ``primary'' purpose of a statute is likely
to be elusive. Arlington Heights v. Metropolitan Housing Dev. Corp.;
McGinnis v. Royster (1973). Here, for example, the individual
legislators may have voted for the statute for a variety of reasons.
Some legislators may have been concerned about preventing teenage
pregnancies, others about protecting young females from physical injury
or from the loss of ``chastity,'' and still others about promoting
various religious and moral attitudes towards premarital sex.

The justification for the statute offered by the State, and accepted by
the Supreme Court of California, is that the legislature sought to
prevent illegitimate teenage pregnancies. That finding, of course, is
entitled to great deference. Reitman v. Mulkey (1967). And although our
cases establish that the State's asserted reason for the enactment of a
statute may be rejected, if it ``could not have been a goal of the
legislation,'' Weinberger v. Wiesenfeld, n.~16, this is not such a case.

We are satisfied not only that the prevention of illegitimate pregnancy
is at least one of the ``purposes'' of the statute, but also that the
State has a strong interest in preventing such pregnancy. At the risk of
stating the obvious, teenage pregnancies, which have increased
dramatically over the last two decades,3 have significant social,
medical, and economic consequences for both the mother and her child,
and the State Of particular concern to the State is that approximately
half of all teenage pregnancies end in abortion And of those children
who are born, their illegitimacy makes them likely candidates to become
wards of the State.

We need not be medical doctors to discern that young men and young women
are not similarly situated with respect to the problems and the risks of
sexual intercourse. Only women may become pregnant, and they suffer
disproportionately the profound physical, emotional, and psychological
consequences of sexual activity. The statute at issue here protects
women from sexual intercourse at an age when those consequences are
particularly severe.

The question thus boils down to whether a State may attack the problem
of sexual intercourse and teenage pregnancy directly by prohibiting a
male from having sexual intercourse with a minor female We hold that
such a statute is sufficiently related to the State's objectives to pass
constitutional muster.

Because virtually all of the significant harmful and inescapably
identifiable consequences of teenage pregnancy fall on the young female,
a legislature acts well within its authority when it elects to punish
only the participant who, by nature, suffers few of the consequences of
his conduct. It is hardly unreasonable for a legislature acting to
protect minor females to exclude them from punishment. Moreover, the
risk of pregnancy itself constitutes a substantial deterrence to young
females. No similar natural sanctions deter males. A criminal sanction
imposed solely on males thus serves to roughly ``equalize'' the
deterrents on the sexes.

We are unable to accept petitioner's contention that the statute is
impermissibly underinclusive and must, in order to pass judicial
scrutiny, be broadened so as to hold the female as criminally liable as
the male. It is argued that this statute is not necessary to deter
teenage pregnancy because a gender-neutral statute, where both male and
female would be subject to prosecution, would serve that goal equally
well. The relevant inquiry, however, is not whether the statute is drawn
as precisely as it might have been, but whether the line chosen by the
California Legislature is within constitutional limitations. Kahn v.
Shevin, n.~10.

In any event, we cannot say that a gender-neutral statute would be as
effective as the statute California has chosen to enact. The State
persuasively contends that a gender-neutral statute would frustrate its
interest in effective enforcement. Its view is that a female is surely
less likely to report violations of the statute if she herself would be
subject to criminal prosecution In an area already fraught with
prose-cutorial difficulties, we decline to hold that the Equal
Protection Clause requires a legislature to enact a statute so broad
that it may well be incapable of enforcement.

We similarly reject petitioner's argument that § 261 is impermissibly
overbroad because it makes unlawful sexual intercourse with prepubescent
females, who are, by definition, incapable of becoming pregnant. Quite
apart from the fact that the statute could well be justified on the
grounds that very young females are particularly susceptible to physical
injury from sexual intercourse, see Rundlett v. Oliver, CA1 1979), it is
ludicrous to suggest that the Constitution requires the California
Legislature to limit the scope of its rape statute to older teenagers
and exclude young girls.

There remains only petitioner's contention that the statute is
unconstitutional as it is applied to him because he, like Sharon, was
under 18 at the time of sexual intercourse. Petitioner argues that the
statute is flawed because it presumes that as between two persons under
18, the male is the culpable aggressor We find petitioner's contentions
unpersuasive. Contrary to his assertions, the statute does not rest on
the assumption that males are generally the aggressors. It is instead an
attempt by a legislature to prevent illegitimate teenage pregnancy by
providing an additional deterrent for men. The age of the man is
irrelevant since young men are as capable as older men of inflicting the
harm sought to be. prevented.

In upholding the California statute we also recognize that this is not a
case where a statute is being challenged on the grounds that it
``invidiously discriminates'' against females. To the contrary, the
statute places a burden on males which is not shared by females. But we
find nothing to suggest that men, because of past discrimination or
peculiar disadvantages, are in need of the special solicitude of the
courts. Nor is this a case where the gender classification is made
``solely for . administrative convenience,'' as in Frontiero v.
Richardson (emphasis omitted), or rests on ``the baggage of sexual
stereotypes'' as in Orr v. Orr. As we have held, the statute instead
reasonably reflects the fact that the consequences of sexual intercourse
and pregnancy fall more heavily on the female than on the male.

Accordingly the judgment of the California Supreme Court is Affirmed.

\hypertarget{united-states-v.-virginia}{%
\subsubsection{United States v.
Virginia}\label{united-states-v.-virginia}}

518 U.S. 515 (1996)

\textbf{JUSTICE GINSBURG delivered the opinion of the Court.}

Virginia's public institutions of higher learning include an
incomparable military college, Virginia Military Institute (VMI). The
United States maintains that the Constitution's equal protection
guarantee precludes Virginia from reserving exclusively to men the
unique educational opportunities VMI affords. We agree.

Founded in 1839, VMI is today the sole single-sex school among
Virginia's 15 public institutions of higher learning. VMI's distinctive
mission is to produce ``citizen-soldiers,'' men prepared for leadership
in civilian life and in military service. VMI pursues this mission
through pervasive training of a kind not available anywhere else in
Virginia. Assigning prime place to character development, VMI uses an
``adversative method'' modeled on English public schools and once
characteristic of military instruction. VMI constantly endeavors to
instill physical and mental discipline in its cadets and impart to them
a strong moral code. The school's graduates leave VMI with heightened
comprehension of their capacity to deal with duress and stress, and a
large sense of accomplishment for completing the hazardous course.

VMI has notably succeeded in its mission to produce leaders; among its
alumni are military generals, Members of Congress, and business
executives. The school's alumni overwhelmingly perceive that their VMI
training helped them to realize their personal goals. VMI's endowment
reflects the loyalty of its graduates; VMI has the largest per-student
endowment of all public undergraduate institutions in the Nation.

Neither the goal of producing citizen-soldiers nor VMI's implementing
methodology is inherently unsuitable to women. And the school's
impressive record in producing leaders has made admission desirable to
some women. Nevertheless, Virginia has elected to preserve exclusively
for men the advantages and opportunities a VMI education affords.

From its establishment in 1839 as one of the Nation's first state
military colleges, see 1839 Va. Acts, ch.~20, VMI has remained
financially supported by Virginia and ``subject to the control of the
{[}Virginia{]} General Assembly,'' Va. Code Ann. § 23-92 (1993). First
southern college to teach engineering and industrial chemistry, see H.
Wise, Drawing Out the Man: The VMI Story 13 (1978) (The VMI Story), VMI
once provided teachers for the Commonwealth's schools, see 1842 Va.
Acts, ch.~24, § 2 (requiring every cadet to teach in one of the
Commonwealth's schools for a 2-year period). n1 Civil War strife
threatened the school's vitality, but a resourceful superintendent
regained legislative support by highlighting ``VMI's great
potential{[},{]} through its technical know-how,'' to advance Virginia's
postwar recovery. The VMI Story 47.

VMI today enrolls about 1,300 men as cadets. n2 Its academic offerings
in the liberal arts, sciences, and engineering are also available at
other public colleges and universities in Virginia. But VMI's mission is
special. It is the mission of the school``\,`to produce educated and
honorable men, prepared for the varied work of civil life, imbued with
love of learning, confident in the functions and attitudes of
leadership, possessing a high sense of public service, advocates of the
American democracy and free enterprise system, and ready as
citizen-soldiers to defend their country in time of national peril.'\,''
766 F. Supp. 1407, 1425 (WD Va. 1991) (quoting Mission Study Committee
of the VMI Board of Visitors, Report, May 16, 1986).In contrast to the
federal service academies, institutions maintained ``to prepare cadets
for career service in the armed forces,'' VMI's program ``is directed at
preparation for both military and civilian life''; ``only about 15\% of
VMI cadets enter career military service.'' 766 F. Supp..

VMI produces its ``citizen-soldiers'' through ``an adversative, or
doubting, model of education'' which features ``physical rigor, mental
stress, absolute equality of treatment, absence of privacy, minute
regulation of behavior, and indoctrination in desirable values.'' As one
Commandant of Cadets described it, the adversative method ``\,`dissects
the young student,'\,'' and makes him aware of his ``\,`limits and
capabilities,'\,'' so that he knows ``\,`how far he can go with his
anger, . how much he can take under stress, . exactly what he can do
when he is physically exhausted.'\,'' -1422 (quoting Col. N. Bissell).

VMI cadets live in spartan barracks where surveillance is constant and
privacy nonexistent; they wear uniforms, eat together in the mess hall,
and regularly participate in drills. 1432. Entering students are
incessantly exposed to the rat line, ``an extreme form of the
adversative model,'' comparable in intensity to Marine Corps boot camp.
Tormenting and punishing, the rat line bonds new cadets to their fellow
sufferers and, when they have completed the 7-month experience, to their
former tormentors.

VMI's ``adversative model'' is further characterized by a hierarchical
``class system'' of privileges and responsibilities, a ``dyke system''
for assigning a senior class mentor to each entering class ``rat,'' and
a stringently enforced ``honor code,'' which prescribes that a cadet
``\,`does not lie, cheat, steal nor tolerate those who do.'\,'' -1423.

VMI attracts some applicants because of its reputation as an
extraordinarily challenging military school, and ``because its alumni
are exceptionally close to the school.'' ``Women have no opportunity
anywhere to gain the benefits of {[}the system of education at VMI{]}.''

In 1990, prompted by a complaint filed with the Attorney General by a
female high-school student seeking admission to VMI, the United States
sued the Commonwealth of Virginia and VMI, alleging that VMI's
exclusively male admission policy violated the Equal Protection Clause
of the Fourteenth Amendment. n3 Trial of the action consumed six days
and involved an array of expert witnesses on each side.

In the two years preceding the lawsuit, the District Court noted, VMI
had received inquiries from 347 women, but had responded to none of
them. ``Some women, at least,'' the court said, ``would want to attend
the school if they had the opportunity.'' The court further recognized
that, with recruitment, VMI could ``achieve at least 10\% female
enrollment'' -- ``a sufficient `critical mass' to provide the female
cadets with a positive educational experience.'' -1438. And it was also
established that ``some women are capable of all of the individual
activities required of VMI cadets.'' In addition, experts agreed that if
VMI admitted women, ``the VMI ROTC experience would become a better
training program from the perspective of the armed forces, because it
would provide training in dealing with a mixed-gender army.''

The District Court ruled in favor of VMI, however, and rejected the
equal protection challenge pressed by the United States. That court
correctly recognized that Mississippi Univ. for Women v. Hogan 1090,
(1982), was the closest guide. 766 F. Supp.. There, this Court
underscored that a party seeking to uphold government action based on
sex must establish an ``exceedingly persuasive justification'' for the
classification. Mississippi Univ. for Women, 458 U.S. (internal
quotation marks omitted). To succeed, the defender of the challenged
action must show ``at least that the classification serves important
governmental objectives and that the discriminatory means employed are
substantially related to the achievement of those objectives.''
(internal quotation marks omitted).

The District Court reasoned that education in ``a single gender
environment, be it male or female,'' yields substantial benefits. 766 F.
Supp.. VMI's school for men brought diversity to an otherwise
coeducational Virginia system, and that diversity was ``enhanced by
VMI's unique method of instruction.'' If single-gender education for
males ranks as an important governmental objective, it becomes obvious,
the District Court concluded, that the only means of achieving the
objective ``is to exclude women from the all-male institution -- VMI.''

``Women are {[}indeed{]} denied a unique educational opportunity that is
available only at VMI,'' the District Court acknowledged. But
``{[}VMI's{]} single-sex status would be lost, and some aspects of the
{[}school's{]} distinctive method would be altered'' if women were
admitted, : ``Allowance for personal privacy would have to be made,'' ;
``physical education requirements would have to be altered, at least for
the women,'' ; the adversative environment could not survive unmodified,
-1413. Thus, ``sufficient constitutional justification'' had been shown,
the District Court held, ``for continuing {[}VMI's{]} single-sex
policy.''

The Court of Appeals for the Fourth Circuit disagreed and vacated the
District Court's judgment. The appellate court held: ``The Commonwealth
of Virginia has not . advanced any state policy by which it can justify
its determination, under an announced policy of diversity, to afford
VMI's unique type of program to men and not to women.''

The appeals court greeted with skepticism Virginia's assertion that it
offers single-sex education at VMI as a facet of the Commonwealth's
overarching and undisputed policy to advance ``autonomy and diversity.''
The court underscored Virginia's nondiscrimination commitment: ``\,`It
is extremely important that {[}colleges and universities{]} deal with
faculty, staff, and students without regard to sex, race, or ethnic
origin.'\,'' (quoting 1990 Report of the Virginia Commission on the
University of the 21st Century). ``That statement,'' the Court of
Appeals said, ``is the only explicit one that we have found in the
record in which the Commonwealth has expressed itself with respect to
gender distinctions.'' 976 F d.~Furthermore, the appeals court observed,
in urging ``diversity'' to justify an all-male VMI, the Commonwealth had
supplied ``no explanation for the movement away from {[}single-sex
education{]} in Virginia by public colleges and universities.'' In
short, the court concluded, ``{[}a{]} policy of diversity which aims to
provide an array of educational opportunities, including single-gender
institutions, must do more than favor one gender.''

The parties agreed that ``some women can meet the physical standards now
imposed on men,'' and the court was satisfied that ``neither the goal of
producing citizen soldiers nor VMI's implementing methodology is
inherently unsuitable to women,'' The Court of Appeals, however,
accepted the District Court's finding that ``at least these three
aspects of VMI's program -- physical training, the absence of privacy,
and the adversative approach -- would be materially affected by
coeducation.'' -897. Remanding the case, the appeals court assigned to
Virginia, in the first instance, responsibility for selecting a remedial
course. The court suggested these options for the Commonwealth: Admit
women to VMI; establish parallel institutions or programs; or abandon
state support, leaving VMI free to pursue its policies as a private
institution. In May 1993, this Court denied certiorari. See 508 U.S.
946; see also (opinion of SCALIA, J., noting the interlocutory posture
of the litigation).

In response to the Fourth Circuit's ruling, Virginia proposed a parallel
program for women: Virginia Women's Institute for Leadership (VWIL). The
4-year, state-sponsored undergraduate program would be located at Mary
Baldwin College, a private liberal arts school for women, and would be
open, initially, to about 25 to 30 students. Although VWIL would share
VMI's mission -- to produce ``citizen-soldiers'' -- the VWIL program
would differ, as does Mary Baldwin College, from VMI in academic
offerings, methods of education, and financial resources.

The average combined SAT score of entrants at Mary Baldwin is about 100
points lower than the score for VMI freshmen. See Mary Baldwin's faculty
holds ``significantly fewer Ph. D.'s than the faculty at VMI,'' and
receives significantly lower salaries, see Tr. 158 (testimony of James
Lott, Dean of Mary Baldwin College), reprinted in 2 App. in Nos. 94-1667
and 94-1717 (CA4) (hereinafter Tr.). While VMI offers degrees in liberal
arts, the sciences, and engineering, Mary Baldwin, at the time of trial,
offered only bachelor of arts degrees. See 852 F. Supp.. A VWIL student
seeking to earn an engineering degree could gain one, without public
support, by attending Washington University in St.~Louis, Missouri, for
two years, paying the required private tuition. See

Experts in educating women at the college level composed the Task Force
charged with designing the VWIL program; Task Force members were drawn
from Mary Baldwin's own faculty and staff. Training its attention on
methods of instruction appropriate for ``most women,'' the Task Force
determined that a military model would be ``wholly inappropriate'' for
VWIL. ; see CA4 1995).

VWIL students would participate in ROTC programs and a newly
established, ``largely ceremonial'' Virginia Corps of Cadets, but the
VWIL House would not have a military format, 852 F. Supp., and VWIL
would not require its students to eat meals together or to wear uniforms
during the schoolday, In lieu of VMI's adversative method, the VWIL Task
Force favored ``a cooperative method which reinforces self-esteem.'' In
addition to the standard bachelor of arts program offered at Mary
Baldwin, VWIL students would take courses in leadership, complete an
off-campus leadership externship, participate in community service
projects, and assist in arranging a speaker series. See 44 F d.

Virginia represented that it will provide equal financial support for
in-state VWIL students and VMI cadets, 852 F. Supp., and the VMI
Foundation agreed to supply a \$ 5 million endowment for the VWIL
program, Mary Baldwin's own endowment is about \$ 19 million; VMI's is
\$ 131 million. Mary Baldwin will add \$ 35 million to its endowment
based on future commitments; VMI will add \$ 220 million. The VMI Alumni
Association has developed a network of employers interested in hiring
VMI graduates. The Association has agreed to open its network to VWIL
graduates, but those graduates will not have the advantage afforded by a
VMI degree.

Virginia returned to the District Court seeking approval of its proposed
remedial plan, and the court decided the plan met the requirements of
the Equal Protection Clause. The District Court again acknowledged
evidentiary support for these determinations: ``The VMI methodology
could be used to educate women and, in fact, some women . may prefer the
VMI methodology to the VWIL methodology.'' But the ``controlling legal
principles,'' the District Court decided, ``do not require the
Commonwealth to provide a mirror image VMI for women.'' The court
anticipated that the two schools would ``achieve substantially similar
outcomes.'' It concluded: ``If VMI marches to the beat of a drum, then
Mary Baldwin marches to the melody of a fife and when the march is over,
both will have arrived at the same destination.''

A divided Court of Appeals affirmed the District Court's judgment. CA4
1995). This time, the appellate court determined to give ``greater
scrutiny to the selection of means than to the {[}Commonwealth's{]}
proffered objective.'' The official objective or purpose, the court
said, should be reviewed deferentially. Respect for the ``legislative
will,'' the court reasoned, meant that the judiciary should take a
``cautious approach,'' inquiring into the ``legitimacy'' of the
governmental objective and refusing approval for any purpose revealed to
be ``pernicious.''

``Providing the option of a single-gender college education may be
considered a legitimate and important aspect of a public system of
higher education,'' the appeals court observed, ; that objective, the
court added, is ``not pernicious,'' Moreover, the court continued, the
adversative method vital to a VMI education ``has never been tolerated
in a sexually heterogeneous environment.'' The method itself ``was not
designed to exclude women,'' the court noted, but women could not be
accommodated in the VMI program, the court believed, for female
participation in VMI's adversative training ``would destroy . any sense
of decency that still permeates the relationship between the sexes.''

Having determined, deferentially, the legitimacy of Virginia's purpose,
the court considered the question of means. Exclusion of ``men at Mary
Baldwin College and women at VMI,'' the court said, was essential to
Virginia's purpose, for without such exclusion, the Commonwealth could
not ``accomplish {[}its{]} objective of providing single-gender
education.''

The court recognized that, as it analyzed the case, means merged into
end, and the merger risked ``bypassing any equal protection scrutiny.''
The court therefore added another inquiry, a decisive test it called
``substantive comparability.'' The key question, the court said, was
whether men at VMI and women at VWIL would obtain ``substantively
comparable benefits at their institution or through other means offered
by the State.'' Although the appeals court recognized that the VWIL
degree ``lacks the historical benefit and prestige'' of a VMI degree, it
nevertheless found the educational opportunities at the two schools
``sufficiently comparable.''

Senior Circuit Judge Phillips dissented. The court, in his judgment, had
not held Virginia to the burden of showing an ``\,`exceedingly
persuasive {[}justification{]}'\,'' for the Commonwealth's action.
(quoting Mississippi Univ. for Women, 458 U.S.). In Judge Phillips'
view, the court had accepted ``rationalizations compelled by the
exigencies of this litigation,'' and had not confronted the
Commonwealth's ``actual overriding purpose.'' 44 F d.~That purpose,
Judge Phillips said, was clear from the historical record; it was ``not
to create a new type of educational opportunity for women, . nor to
further diversify the Commonwealth's higher education system{[},{]} .
but {[}was{]} simply . to allow VMI to continue to exclude women in
order to preserve its historic character and mission.''

Judge Phillips suggested that the Commonwealth would satisfy the
Constitution's equal protection requirement if it ``simultaneously
opened single-gender undergraduate institutions having substantially
comparable curricular and extra-curricular programs, funding, physical
plant, administration and support services, and faculty and library
resources.'' But he thought it evident that the proposed VWIL program,
in comparison to VMI, fell ``far short . from providing substantially
equal tangible and intangible educational benefits to men and women.''

The Fourth Circuit denied rehearing en banc. Circuit Judge Motz, joined
by Circuit Judges Hall, Murnaghan, and Michael, filed a dissenting
opinion. Judge Motz agreed with Judge Phillips that Virginia had not
shown an ``\,`exceedingly persuasive justification'\,'' for the
disparate opportunities the Commonwealth supported. (quoting Mississippi
Univ. for Women). She asked: ``{[}H{]}ow can a degree from a yet to be
implemented supplemental program at Mary Baldwin be held `substantively
comparable' to a degree from a venerable Virginia military institution
that was established more than 150 years ago?'' ``Women need not be
guaranteed equal `results,'\,'' Judge Motz said, ``but the Equal
Protection Clause does require equal opportunity {[}and{]} that
opportunity is being denied here.''

The cross-petitions in this case present two ultimate issues. First,
does Virginia's exclusion of women from the educational opportunities
provided by VMI -- extraordinary opportunities for military training and
civilian leadership development -- deny to women ``capable of all of the
individual activities required of VMI cadets,'' the equal protection of
the laws guaranteed by the Fourteenth Amendment? Second, if VMI's
``unique'' situation, -- as Virginia's sole single-sex public
institution of higher education -- offends the Constitution's equal
protection principle, what is the remedial requirement?

We note, once again, the core instruction of this Court's pathmarking
decisions in J. E. B. v. Alabama ex rel. T. B., and Mississippi Univ.
for Women: Parties who seek to defend gender-based government action
must demonstrate an ``exceedingly persuasive justification'' for that
action.

Today's skeptical scrutiny of official action denying rights or
opportunities based on sex responds to volumes of history. As a
plurality of this Court acknowledged a generation ago, ``our Nation has
had a long and unfortunate history of sex discrimination.'' Frontiero v.
Richardson, (1973). Through a century plus three decades and more of
that history, women did not count among voters composing ``We the
People''; n5 not until 1920 did women gain a constitutional right to the
franchise. And for a half century thereafter, it remained the prevailing
doctrine that government, both federal and state, could withhold from
women opportunities accorded men so long as any ``basis in reason''
could be conceived for the discrimination. See, e. g., Goesaert v.
Cleary, (1948) (rejecting challenge of female tavern owner and her
daughter to Michigan law denying bartender licenses to females -- except
for wives and daughters of male tavern owners; Court would not ``give
ear'' to the contention that ``an unchivalrous desire of male bartenders
to . monopolize the calling'' prompted the legislation).

In 1971, for the first time in our Nation's history, this Court ruled in
favor of a woman who complained that her State had denied her the equal
protection of its laws. Reed v. Reed, (holding unconstitutional Idaho
Code prescription that, among ``\,`several persons claiming and equally
entitled to administer {[}a decedent's estate{]}, males must be
preferred to females'\,''). Since Reed, the Court has repeatedly
recognized that neither federal nor state government acts compatibly
with the equal protection principle when a law or official policy denies
to women, simply because they are women, full citizenship stature --
equal opportunity to aspire, achieve, participate in and contribute to
society based on their individual talents and capacities. See, e. g.,
Kirchberg v. Feenstra, (1981) (affirming invalidity of Louisiana law
that made husband ``head and master'' of property jointly owned with his
wife, giving him unilateral right to dispose of such property without
his wife's consent); Stanton v. Stanton 688, (1975) (invalidating Utah
requirement that parents support boys until age 21, girls only until age
18).

Without equating gender classifications, for all purposes, to
classifications based on race or national origin, n6 the Court, in
post-Reed decisions, has carefully inspected official action that closes
a door or denies opportunity to women (or to men). See J. E. B.
(KENNEDY, J., concurring in judgment) (case law evolving since 1971
``reveal{[}s{]} a strong presumption that gender classifications are
invalid''). To summarize the Court's current directions for cases of
official classification based on gender: Focusing on the differential
treatment or denial of opportunity for which relief is sought, the
reviewing court must determine whether the proffered justification is
``exceedingly persuasive.'' The burden of justification is demanding and
it rests entirely on the State. See Mississippi Univ. for Women, 458
U.S.. The State must show ``at least that the {[}challenged{]}
classification serves `important governmental objectives and that the
discriminatory means employed' are `substantially related to the
achievement of those objectives.'\,'' (quoting Wengler v. Druggists Mut.
Ins. Co., (1980)). The justification must be genuine, not hypothesized
or invented post hoc in response to litigation. And it must not rely on
overbroad generalizations about the different talents, capacities, or
preferences of males and females.

The heightened review standard our precedent establishes does not make
sex a proscribed classification. Supposed ``inherent differences'' are
no longer accepted as a ground for race or national origin
classifications. See Loving v. Virginia. Physical differences between
men and women, however, are enduring: ``The two sexes are not fungible;
a community made up exclusively of one {[}sex{]} is different from a
community composed of both.'' Ballard v. United States.

``Inherent differences'' between men and women, we have come to
appreciate, remain cause for celebration, but not for denigration of the
members of either sex or for artificial constraints on an individual's
opportunity. Sex classifications may be used to compensate women ``for
particular economic disabilities {[}they have{]} suffered,'' Califano v.
Webster, (1977) (per curiam), to ``promote equal employment
opportunity,'' see California Fed. Sav. \& Loan Assn. v. Guerra, (1987),
to advance full development of the talent and capacities of our Nation's
people. n7 But such classifications may not be used, as they once were,
see Goesaert, 335 U.S., to create or perpetuate the legal, social, and
economic inferiority of women.

Measuring the record in this case against the review standard just
described, we conclude that Virginia has shown no ``exceedingly
persuasive justification'' for excluding all women from the
citizen-soldier training afforded by VMI. We therefore affirm the Fourth
Circuit's initial judgment, which held that Virginia had violated the
Fourteenth Amendment's Equal Protection Clause. Because the remedy
proffered by Virginia -- the Mary Baldwin VWIL program -- does not cure
the constitutional violation, i.e., it does not provide equal
opportunity, we reverse the Fourth Circuit's final judgment in this
case.

The Fourth Circuit initially held that Virginia had advanced no state
policy by which it could justify, under equal protection principles, its
determination ``to afford VMI's unique type of program to men and not to
women.'' 976 F d.~Virginia challenges that ``liability'' ruling and
asserts two justifications in defense of VMI's exclusion of women.
First, the Commonwealth contends, ``single-sex education provides
important educational benefits,'' Brief for Cross-Petitioners 20, and
the option of single-sex education contributes to ``diversity in
educational approaches,'' Second, the Commonwealth argues, ``the unique
VMI method of character development and leadership training,'' the
school's adversative approach, would have to be modified were VMI to
admit women. -36 (internal quotation marks omitted). We consider these
two justifications in turn.

Single-sex education affords pedagogical benefits to at least some
students, Virginia emphasizes, and that reality is uncontested in this
litigation. n8 Similarly, it is not disputed that diversity among public
educational institutions can serve the public good. But Virginia has not
shown that VMI was established, or has been maintained, with a view to
diversifying, by its categorical exclusion of women, educational
opportunities within the Commonwealth. In cases of this genre, our
precedent instructs that ``benign'' justifications proffered in defense
of categorical exclusions will not be accepted automatically; a tenable
justification must describe actual state purposes, not rationalizations
for actions in fact differently grounded. See Wiesenfeld, 420 U.S., and
n.~16 (``mere recitation of a benign {[}or{]} compensatory purpose''
does not block ``inquiry into the actual purposes'' of
government-maintained gender-based classifications); Goldfarb, 430 U.S.
(rejecting government-proffered purposes after ``inquiry into the actual
purposes'') (internal quotation marks omitted).

Mississippi Univ. for Women is immediately in point. There the State
asserted, in justification of its exclusion of men from a nursing
school, that it was engaging in ``educational affirmative action'' by
``compensating for discrimination against women.'' 458 U.S.. Undertaking
a ``searching analysis,'' the Court found no close resemblance between
``the alleged objective'' and ``the actual purpose underlying the
discriminatory classification,'' Pursuing a similar inquiry here, we
reach the same conclusion.

Neither recent nor distant history bears out Virginia's alleged pursuit
of diversity through single-sex educational options. In 1839, when the
Commonwealth established VMI, a range of educational opportunities for
men and women was scarcely contemplated. Higher education at the time
was considered dangerous for women; n9 reflecting widely held views
about women's proper place, the Nation's first universities and colleges
-- for example, Harvard in Massachusetts, William and Mary in Virginia
-- admitted only men. See E. Farello, A History of the Education of
Women in the United States 163 (1970). VMI was not at all novel in this
respect: In admitting no women, VMI followed the lead of the
Commonwealth's flagship school, the University of Virginia, founded in
1819.

``No struggle for the admission of women to a state university,'' a
historian has recounted, ``was longer drawn out, or developed more
bitterness, than that at the University of Virginia.'' 2 T. Woody, A
History of Women's Education in the United States 254 (1929) (History of
Women's Education). In 1879, the State Senate resolved to look into the
possibility of higher education for women, recognizing that Virginia
``\,`has never, at any period of her history,'\,'' provided for the
higher education of her daughters, though she ``\,`has liberally
provided for the higher education of her sons.'\,'' (quoting 10 Educ. J.
Va. 212 (1879)). Despite this recognition, no new opportunities were
instantly open to women. n

Virginia eventually provided for several women's seminaries and
colleges. Farmville Female Seminary became a public institution in 1884.
See, n.~2. Two women's schools, Mary Washington College and James
Madison University, were founded in 1908; another, Radford University,
was founded in 1910. 766 F. Supp.. By the mid-1970's, all four schools
had become coeducational.

Debate concerning women's admission as undergraduates at the main
university continued well past the century's midpoint. Familiar
arguments were rehearsed. If women were admitted, it was feared, they
``would encroach on the rights of men; there would be new problems of
government, perhaps scandals; the old honor system would have to be
changed; standards would be lowered to those of other coeducational
schools; and the glorious reputation of the university, as a school for
men, would be trailed in the dust.'' 2 History of Women's Education 255.

Ultimately, in 1970, ``the most prestigious institution of higher
education in Virginia,'' the University of Virginia, introduced
coeducation and, in 1972, began to admit women on an equal basis with
men. See Kirstein v. Rector and Visitors of Univ. of Virginia, 309 F.
Supp. 184, 186 (ED Va. 1970). A three-judge Federal District Court
confirmed: ``Virginia may not now deny to women, on the basis of sex,
educational opportunities at the Charlottesville campus that are not
afforded in other institutions operated by the State.''

Virginia describes the current absence of public single-sex higher
education for women as ``an historical anomaly.'' Brief for
Cross-Petitioners 30. But the historical record indicates action more
deliberate than anomalous: First, protection of women against higher
education; next, schools for women far from equal in resources and
stature to schools for men; finally, conversion of the separate schools
to coeducation. The state legislature, prior to the advent of this
controversy, had repealed ``all Virginia statutes requiring individual
institutions to admit only men or women.'' 766 F. Supp.. And in 1990, an
official commission, ``legislatively established to chart the future
goals of higher education in Virginia,'' reaffirmed the policy ``'of
affording broad access'' while maintaining ``autonomy and diversity.'''
976 F d (quoting Report of the Virginia Commission on the University of
the 21st Century). Significantly, the Commission reported:``\,`Because
colleges and universities provide opportunities for students to develop
values and learn from role models, it is extremely important that they
deal with faculty, staff, and students without regard to sex, race, or
ethnic origin.'\,'' (emphasis supplied by Court of Appeals deleted).This
statement, the Court of Appeals observed, ``is the only explicit one
that we have found in the record in which the Commonwealth has expressed
itself with respect to gender distinctions.''

Our 1982 decision in Mississippi Univ. for Women prompted VMI to
reexamine its male-only admission policy. See 766 F. Supp.. Virginia
relies on that reexamination as a legitimate basis for maintaining VMI's
single-sex character. See Reply Brief for Cross-Petitioners 6. A Mission
Study Committee, appointed by the VMI Board of Visitors, studied the
problem from October 1983 until May 1986, and in that month counseled
against ``change of VMI status as a single-sex college.'' See 766 F.
Supp. (internal quotation marks omitted). Whatever internal purpose the
Mission Study Committee served -- and however well meaning the framers
of the report -- we can hardly extract from that effort any commonwealth
policy evenhandedly to advance diverse educational options. As the
District Court observed, the Committee's analysis ``primarily focused on
anticipated difficulties in attracting females to VMI,'' and the report,
overall, supplied ``very little indication of how the conclusion was
reached.''

In sum, we find no persuasive evidence in this record that VMI's
male-only admission policy ``is in furtherance of a state policy of
`diversity.'\,'' See 976 F d.~No such policy, the Fourth Circuit
observed, can be discerned from the movement of all other public
colleges and universities in Virginia away from single-sex education.
See That court also questioned ``how one institution with autonomy, but
with no authority over any other state institution, can give effect to a
state policy of diversity among institutions.'' A purpose genuinely to
advance an array of educational options, as the Court of Appeals
recognized, is not served by VMI's historic and constant plan -- a plan
to ``afford a unique educational benefit only to males.'' However
``liberally'' this plan serves the Commonwealth's sons, it makes no
provision whatever for her daughters. That is not equal protection.

Virginia next argues that VMI's adversative method of training provides
educational benefits that cannot be made available, unmodified, to
women. Alterations to accommodate women would necessarily be
``radical,'' so ``drastic,'' Virginia asserts, as to transform, indeed
``destroy,'' VMI's program. See Brief for Cross-Petitioners 34-36.
Neither sex would be favored by the transformation, Virginia maintains:
Men would be deprived of the unique opportunity currently available to
them; women would not gain that opportunity because their participation
would ``eliminate the very aspects of {[}the{]} program that distinguish
{[}VMI{]} from . other institutions of higher education in Virginia.''

The District Court forecast from expert witness testimony, and the Court
of Appeals accepted, that coeducation would materially affect ``at least
these three aspects of VMI's program -- physical training, the absence
of privacy, and the adversative approach.'' 976 F d.~And it is
uncontested that women's admission would require accommodations,
primarily in arranging housing assignments and physical training
programs for female cadets. See Brief for Cross-Respondent 11, 29-30. It
is also undisputed, however, that ``the VMI methodology could be used to
educate women.'' 852 F. Supp.. The District Court even allowed that some
women may prefer it to the methodology a women's college might pursue.
See ``Some women, at least, would want to attend {[}VMI{]} if they had
the opportunity,'' the District Court recognized, 766 F. Supp., and
``some women,'' the expert testimony established, ``are capable of all
of the individual activities required of VMI cadets,'' The parties,
furthermore, agree that ``some women can meet the physical standards
{[}VMI{]} now impose{[}s{]} on men.'' 976 F d.~In sum, as the Court of
Appeals stated, ``neither the goal of producing citizen soldiers,''
VMI's raison d'etre, ``nor VMI's implementing methodology is inherently
unsuitable to women.''

In support of its initial judgment for Virginia, a judgment rejecting
all equal protection objections presented by the United States, the
District Court made ``findings'' on ``gender-based developmental
differences.'' 766 F. Supp.. These ``findings'' restate the opinions of
Virginia's expert witnesses, opinions about typically male or typically
female ``tendencies.'' For example, ``males tend to need an atmosphere
of adversativeness,'' while ``females tend to thrive in a cooperative
atmosphere.'' ``I'm not saying that some women don't do well under
{[}the{]} adversative model,'' VMI's expert on educational institutions
testified, ``undoubtedly there are some {[}women{]} who do''; but
educational experiences must be designed ``around the rule,'' this
expert maintained, and not ``around the exception.'' (internal quotation
marks omitted).

The United States does not challenge any expert witness estimation on
average capacities or preferences of men and women. Instead, the United
States emphasizes that time and again since this Court's turning point
decision in Reed v. Reed, we have cautioned reviewing courts to take a
``hard look'' at generalizations or ``tendencies'' of the kind pressed
by Virginia, and relied upon by the District Court. See O'Connor,
Portia's Progress, 66 N. Y. U. L. Rev.~1546, 1551 (1991). State actors
controlling gates to opportunity, we have instructed, may not exclude
qualified individuals based on ``fixed notions concerning the roles and
abilities of males and females.'' Mississippi Univ. for Women, 458 U.S.;
see J. E. B., 511 U.S., n.~11 (equal protection principles, as applied
to gender classifications, mean state actors may not rely on
``overbroad'' generalizations to make ``judgments about people that are
likely to . perpetuate historical patterns of discrimination'').

It may be assumed, for purposes of this decision, that most women would
not choose VMI's adversative method. As Fourth Circuit Judge Motz
observed, however, in her dissent from the Court of Appeals' denial of
rehearing en banc, it is also probable that ``many men would not want to
be educated in such an environment.'' 52 F d.~(On that point, even our
dissenting colleague might agree.) Education, to be sure, is not a ``one
size fits all'' business. The issue, however, is not whether ``women --
or men -- should be forced to attend VMI''; rather, the question is
whether the Commonwealth can constitutionally deny to women who have the
will and capacity, the training and attendant opportunities that VMI
uniquely affords.

The notion that admission of women would downgrade VMI's stature,
destroy the adversative system and, with it, even the school, n11 is a
judgment hardly proved, n12 a prediction hardly different from other
``self-fulfilling prophec{[}ies{]},'' see Mississippi Univ. for Women,
458 U.S., once routinely used to deny rights or opportunities. When
women first sought admission to the bar and access to legal education,
concerns of the same order were expressed. For example, in 1876, the
Court of Common Pleas of Hennepin County, Minnesota, explained why women
were thought ineligible for the practice of law. Women train and educate
the young, the court said, which``forbids that they shall bestow that
time (early and late) and labor, so essential in attaining to the
eminence to which the true lawyer should ever aspire. It cannot
therefore be said that the opposition of courts to the admission of
females to practice . is to any extent the outgrowth of . `old
fogyism{[}.{]}' . It arises rather from a comprehension of the magnitude
of the responsibilities connected with the successful practice of law,
and a desire to grade up the profession.'' In re Application of Martha
Angle Dorsett to Be Admitted to Practice as Attorney and Counselor at
Law (Minn. C. P. Hennepin Cty., 1876), in The Syllabi, Oct.~21, 1876,
pp.~5, 6 (emphasis added). A like fear, according to a 1925 report,
accounted for Columbia Law School's resistance to women's admission,
although``the faculty . never maintained that women could not master
legal learning . . No, its argument has been . more practical. If women
were admitted to the Columbia Law School, {[}the faculty{]} said, then
the choicer, more manly and red-blooded graduates of our great
universities would go to the Harvard Law School!'' The Nation, Feb.~18,
1925, p.~173.

Medical faculties similarly resisted men and women as partners in the
study of medicine. See R. Morantz-Sanchez, Sympathy and Science: Women
Physicians in American Medicine 51(1985); see also M. Walsh, ``Doctors
Wanted: No Women Need Apply'' 121-122 (1977) (quoting E. Clarke, Medical
Education of Women, 4 Boston Med. \& Surg. J. 345, 346 (1869) (``\,`God
forbid that I should ever see men and women aiding each other to display
with the scalpel the secrets of the reproductive system . .'\,'')); cf.,
n.~9. More recently, women seeking careers in policing encountered
resistance based on fears that their presence would ``undermine male
solidarity,'' see F. Heidensohn, Women in Control? 201 (1992); deprive
male partners of adequate assistance, see -185; and lead to sexual
misconduct, see C. Milton et al., Women in Policing 32-33 (1974). Field
studies did not confirm these fears. See Heidensohn; P. Bloch \& D.
Anderson, Policewomen on Patrol: Final Report (1974).

Women's successful entry into the federal military academies, n13 and
their participation in the Nation's military forces, n14 indicate that
Virginia's fears for the future of VMI may not be solidly grounded. n15
The Commonwealth's justification for excluding all women from
``citizen-soldier'' training for which some are qualified, in any event,
cannot rank as ``exceedingly persuasive,'' as we have explained and
applied that standard.

Virginia and VMI trained their argument on ``means'' rather than
``end,'' and thus misperceived our precedent. Single-sex education at
VMI serves an ``important governmental objective,'' they maintained, and
exclusion of women is not only ``substantially related,'' it is
essential to that objective. By this notably circular argument, the
``straightforward'' test Mississippi Univ. for Women described, see 458
U.S., was bent and bowed.

The Commonwealth's misunderstanding and, in turn, the District Court's,
is apparent from VMI's mission: to produce ``citizen-soldiers,''
individuals``\,`imbued with love of learning, confident in the functions
and attitudes of leadership, possessing a high sense of public service,
advocates of the American democracy and free enterprise system, and
ready . to defend their country in time of national peril.'\,'' 766 F.
Supp. (quoting Mission Study Committee of the VMI Board of Visitors,
Report, May 16, 1986).Surely that goal is great enough to accommodate
women, who today count as citizens in our American democracy equal in
stature to men. Just as surely, the Commonwealth's great goal is not
substantially advanced by women's categorical exclusion, in total
disregard of their individual merit, from the Commonwealth's premier
``citizen-soldier'' corps. n16 Virginia, in sum, ``has fallen far short
of establishing the `exceedingly persuasive justification,'\,''
Mississippi Univ. for Women, 458 U.S., that must be the solid base for
any gender-defined classification.

In the second phase of the litigation, Virginia presented its remedial
plan -- maintain VMI as a male-only college and create VWIL as a
separate program for women. The plan met District Court approval. The
Fourth Circuit, in turn, deferentially reviewed the Commonwealth's
proposal and decided that the two single-sex programs directly served
Virginia's reasserted purposes: single-gender education, and ``achieving
the results of an adversative method in a military environment.'' See 44
F d, 1239. Inspecting the VMI and VWIL educational programs to determine
whether they ``afforded to both genders benefits comparable in
substance, {[}if{]} not in form and detail,'' the Court of Appeals
concluded that Virginia had arranged for men and women opportunities
``sufficiently comparable'' to survive equal protection evaluation,
-1241. The United States challenges this ``remedial'' ruling as
pervasively misguided.

A remedial decree, this Court has said, must closely fit the
constitutional violation; it must be shaped to place persons
unconstitutionally denied an opportunity or advantage in ``the position
they would have occupied in the absence of {[}discrimination{]}.'' See
Milliken v. Bradley, (1977) (internal quotation marks omitted). The
constitutional violation in this case is the categorical exclusion of
women from an extraordinary educational opportunity afforded men. A
proper remedy for an unconstitutional exclusion, we have explained, aims
to ``eliminate {[}so far as possible{]} the discriminatory effects of
the past'' and to ``bar like discrimination in the future.'' Louisiana
v. United States, (1965).

Virginia chose not to eliminate, but to leave untouched, VMI's
exclusionary policy. For women only, however, Virginia proposed a
separate program, different in kind from VMI and unequal in tangible and
intangible facilities. n17 Having violated the Constitution's equal
protection requirement, Virginia was obliged to show that its remedial
proposal ``directly addressed and related to'' the violation, see
Milliken, 433 U.S., i. e., the equal protection denied to women ready,
willing, and able to benefit from educational opportunities of the kind
VMI offers. Virginia described VWIL as a ``parallel program,'' and
asserted that VWIL shares VMI's mission of producing
``citizen-soldiers'' and VMI's goals of providing ``education, military
training, mental and physical discipline, character . and leadership
development.'' Brief for Respondents 24 (internal quotation marks
omitted). If the VWIL program could not ``eliminate the discriminatory
effects of the past,'' could it at least ``bar like discrimination in
the future''? See Louisiana, 380 U.S.. A comparison of the programs said
to be ``parallel'' informs our answer. In exposing the character of, and
differences in, the VMI and VWIL programs, we recapitulate facts earlier
presented. See, 526-527.

VWIL affords women no opportunity to experience the rigorous military
training for which VMI is famed. See 766 F. Supp. (``No other school in
Virginia or in the United States, public or private, offers the same
kind of rigorous military training as is available at VMI.''); (VMI ``is
known to be the most challenging military school in the United
States''). Instead, the VWIL program ``deemphasize{[}s{]}'' military
education, 44 F d, and uses a ``cooperative method'' of education
``which reinforces self-esteem,'' 852 F. Supp..

VWIL students participate in ROTC and a ``largely ceremonial'' Virginia
Corps of Cadets, see 44 F d, but Virginia deliberately did not make VWIL
a military institute. The VWIL House is not a military-style residence
and VWIL students need not live together throughout the 4-year program,
eat meals together, or wear uniforms during the school day. See 852 F.
Supp., 495. VWIL students thus do not experience the ``barracks'' life
``crucial to the VMI experience,'' the spartan living arrangements
designed to foster an ``egalitarian ethic.'' See 766 F. Supp.. ``The
most important aspects of the VMI educational experience occur in the
barracks,'' the District Court found, yet Virginia deemed that core
experience nonessential, indeed inappropriate, for training its female
citizen-soldiers.

VWIL students receive their ``leadership training'' in seminars,
externships, and speaker series, see 852 F. Supp., episodes and
encounters lacking the ``physical rigor, mental stress, . minute
regulation of behavior, and indoctrination in desirable values'' made
hallmarks of VMI's citizen-soldier training, see 766 F. Supp.. n18 Kept
away from the pressures, hazards, and psychological bonding
characteristic of VMI's adversative training, see VWIL students will not
know the ``feeling of tremendous accomplishment'' commonly experienced
by VMI's successful cadets,

Virginia maintains that these methodological differences are ``justified
pedagogically,'' based on ``important differences between men and women
in learning and developmental needs,'' ``psychological and sociological
differences'' Virginia describes as ``real'' and ``not stereotypes.''
Brief for Respondents 28 (internal quotation marks omitted). The Task
Force charged with developing the leadership program for women, drawn
from the staff and faculty at Mary Baldwin College, ``determined that a
military model and, especially VMI's adversative method, would be wholly
inappropriate for educating and training most women.'' 852 F. Supp.
(emphasis added). See also 44 F d (noting Task Force conclusion that,
while ``some women would be suited to and interested in {[}a VMI-style
experience{]},'' VMI's adversative method ``would not be effective for
women as a group'') (emphasis added). The Commonwealth embraced the Task
Force view, as did expert witnesses who testified for Virginia. See 852
F. Supp..

As earlier stated, see, generalizations about ``the way women are,''
estimates of what is appropriate for most women, no longer justify
denying opportunity to women whose talent and capacity place them
outside the average description. Notably, Virginia never asserted that
VMI's method of education suits most men. It is also revealing that
Virginia accounted for its failure to make the VWIL experience ``the
entirely militaristic experience of VMI'' on the ground that VWIL ``is
planned for women who do not necessarily expect to pursue military
careers.'' 852 F. Supp.. By that reasoning, VMI's ``entirely
militaristic'' program would be inappropriate for men in general or as a
group, for ``only about 15\% of VMI cadets enter career military
service.'' See 766 F. Supp..

In contrast to the generalizations about women on which Virginia rests,
we note again these dispositive realities: VMI's ``implementing
methodology'' is not ``inherently unsuitable to women,'' 976 F d; ``some
women . do well under {[}the{]} adversative model,'' 766 F. Supp.
(internal quotation marks omitted); ``some women, at least, would want
to attend {[}VMI{]} if they had the opportunity,'' ; ``some women are
capable of all of the individual activities required of VMI cadets,''
and ``can meet the physical standards {[}VMI{]} now impose{[}s{]} on
men,'' 976 F d.~It is on behalf of these women that the United States
has instituted this suit, and it is for them that a remedy must be
crafted, n19 a remedy that will end their exclusion from a
state-supplied educational opportunity for which they are fit, a decree
that will ``bar like discrimination in the future.'' Louisiana, 380
U.S..

In myriad respects other than military training, VWIL does not qualify
as VMI's equal. VWIL's student body, faculty, course offerings, and
facilities hardly match VMI's. Nor can the VWIL graduate anticipate the
benefits associated with VMI's 157-year history, the school's prestige,
and its influential alumni network.

Mary Baldwin College, whose degree VWIL students will gain, enrolls
first-year women with an average combined SAT score about 100 points
lower than the average score for VMI freshmen. 852 F. Supp.. The Mary
Baldwin faculty holds ``significantly fewer Ph. D.'s,'' and receives
substantially lower salaries, see Tr. 158 (testimony of James Lott, Dean
of Mary Baldwin College), than the faculty at VMI.

Mary Baldwin does not offer a VWIL student the range of curricular
choices available to a VMI cadet. VMI awards baccalaureate degrees in
liberal arts, biology, chemistry, civil engineering, electrical and
computer engineering, and mechanical engineering. See 852 F. Supp.;
Virginia Military Institute: More than an Education 11 (Govt. exh. 75,
lodged with Clerk of this Court). VWIL students attend a school that
``does not have a math and science focus,'' 852 F. Supp.; they cannot
take at Mary Baldwin any courses in engineering or the advanced math and
physics courses VMI offers, see

For physical training, Mary Baldwin has ``two multi-purpose fields'' and
``one gymnasium.'' VMI has ``an NCAA competition level indoor track and
field facility; a number of multi-purpose fields; baseball, soccer and
lacrosse fields; an obstacle course; large boxing, wrestling and martial
arts facilities; an 11-laps-to-the-mile indoor running course; an indoor
pool; indoor and outdoor rifle ranges; and a football stadium that also
contains a practice field and outdoor track.''

Although Virginia has represented that it will provide equal financial
support for in-state VWIL students and VMI cadets, and the VMI
Foundation has agreed to endow VWIL with \$ 5 million, the difference
between the two schools' financial reserves is pronounced. Mary
Baldwin's endowment, currently about \$ 19 million, will gain an
additional \$ 35 million based on future commitments; VMI's current
endowment, \$ 131 million -- the largest public college per-student
endowment in the Nation -- will gain \$ 220 million.

The VWIL student does not graduate with the advantage of a VMI degree.
Her diploma does not unite her with the legions of VMI ``graduates
{[}who{]} have distinguished themselves'' in military and civilian life.
See 976 F d.~``{[}VMI{]} alumni are exceptionally close to the school,''
and that closeness accounts, in part, for VMI's success in attracting
applicants. See 766 F. Supp.. A VWIL graduate cannot assume that the
``network of business owners, corporations, VMI graduates and
non-graduate employers . interested in hiring VMI graduates,'' 852 F.
Supp., will be equally responsive to her search for employment, see 44 F
d (Phillips, J., dissenting) (``the powerful political and economic ties
of the VMI alumni network cannot be expected to open'' for graduates of
the fledgling VWIL program).

Virginia, in sum, while maintaining VMI for men only, has failed to
provide any ``comparable single-gender women's institution.'' Instead,
the Commonwealth has created a VWIL program fairly appraised as a ``pale
shadow'' of VMI in terms of the range of curricular choices and faculty
stature, funding, prestige, alumni support and influence. See (Phillips,
J., dissenting).

Virginia's VWIL solution is reminiscent of the remedy Texas proposed 50
years ago, in response to a state trial court's 1946 ruling that, given
the equal protection guarantee, African Americans could not be denied a
legal education at a state facility. See Sweatt v. Painter 1114, (1950).
Reluctant to admit African Americans to its flagship University of Texas
Law School, the State set up a separate school for Heman Sweatt and
other black law students. As originally opened, the new school had no
independent faculty or library, and it lacked accreditation.
Nevertheless, the state trial and appellate courts were satisfied that
the new school offered Sweatt opportunities for the study of law
``substantially equivalent to those offered by the State to white
students at the University of Texas.'' (internal quotation marks
omitted).

Before this Court considered the case, the new school had gained ``a
faculty of five full-time professors; a student body of 23; a library of
some 16,500 volumes serviced by a full-time staff; a practice court and
legal aid association; and one alumnus who had become a member of the
Texas Bar.'' This Court contrasted resources at the new school with
those at the school from which Sweatt had been excluded. The University
of Texas Law School had a full-time faculty of 16, a student body of
850, a library containing over 65,000 volumes, scholarship funds, a law
review, and moot court facilities. -633.

More important than the tangible features, the Court emphasized, are
``those qualities which are incapable of objective measurement but which
make for greatness'' in a school, including ``reputation of the faculty,
experience of the administration, position and influence of the alumni,
standing in the community, traditions and prestige.'' Facing the marked
differences reported in the Sweatt opinion, the Court unanimously ruled
that Texas had not shown ``substantial equality in the {[}separate{]}
educational opportunities'' the State offered. Accordingly, the Court
held, the Equal Protection Clause required Texas to admit African
Americans to the University of Texas Law School. In line with Sweatt, we
rule here that Virginia has not shown substantial equality in the
separate educational opportunities the Commonwealth supports at VWIL and
VMI.

When Virginia tendered its VWIL plan, the Fourth Circuit did not inquire
whether the proposed remedy, approved by the District Court, placed
women denied the VMI advantage in ``the position they would have
occupied in the absence of {[}discrimination{]}.'' Milliken, 433 U.S.
(internal quotation marks omitted). Instead, the Court of Appeals
considered whether the Commonwealth could provide, with fidelity to the
equal protection principle, separate and unequal educational programs
for men and women.

The Fourth Circuit acknowledged that ``the VWIL degree from Mary Baldwin
College lacks the historical benefit and prestige of a degree from
VMI.'' 44 F d.~The Court of Appeals further observed that VMI is ``an
ongoing and successful institution with a long history,'' and there
remains no ``comparable single-gender women's institution.''
Nevertheless, the appeals court declared the substantially different and
significantly unequal VWIL program satisfactory. The court reached that
result by revising the applicable standard of review. The Fourth Circuit
displaced the standard developed in our precedent, see, and substituted
a standard of its own invention.

We have earlier described the deferential review in which the Court of
Appeals engaged, see, a brand of review inconsistent with the more
exacting standard our precedent requires, see. Quoting in part from
Mississippi Univ. for Women, the Court of Appeals candidly described its
own analysis as one capable of checking a legislative purpose ranked as
``pernicious,'' but generally according ``deference to {[}the{]}
legislative will.'' 44 F d, 1236. Recognizing that it had extracted from
our decisions a test yielding ``little or no scrutiny of the effect of a
classification directed at {[}single-gender education{]},'' the Court of
Appeals devised another test, a ``substantive comparability'' inquiry,
and proceeded to find that new test satisfied,

The Fourth Circuit plainly erred in exposing Virginia's VWIL plan to a
deferential analysis, for ``all gender-based classifications today''
warrant ``heightened scrutiny.'' See J. E. B., 511 U.S.. Valuable as
VWIL may prove for students who seek the program offered, Virginia's
remedy affords no cure at all for the opportunities and advantages
withheld from women who want a VMI education and can make the grade.
See. n20 In sum, Virginia's remedy does not match the constitutional
violation; the Commonwealth has shown no ``exceedingly persuasive
justification'' for withholding from women qualified for the experience
premier training of the kind VMI affords.

A generation ago, ``the authorities controlling Virginia higher
education,'' despite long established tradition, agreed ``to innovate
and favorably entertained the {[}then{]} relatively new idea that there
must be no discrimination by sex in offering educational opportunity.''
Kirstein, 309 F. Supp.. Commencing in 1970, Virginia opened to women
``educational opportunities at the Charlottesville campus that
{[}were{]} not afforded in other {[}state-operated{]} institutions.'' ;
see. A federal court approved the Commonwealth's innovation, emphasizing
that the University of Virginia ``offered courses of instruction . not
available elsewhere.'' 309 F. Supp.. The court further noted: ``There
exists at Charlottesville a `prestige' factor {[}not paralleled in{]}
other Virginia educational institutions.''

VMI, too, offers an educational opportunity no other Virginia
institution provides, and the school's ``prestige'' -- associated with
its success in developing ``citizen-soldiers'' -- is unequaled. Virginia
has closed this facility to its daughters and, instead, has devised for
them a ``parallel program,'' with a faculty less impressively
credentialed and less well paid, more limited course offerings, fewer
opportunities for military training and for scientific specialization.
Cf. Sweatt, 339 U.S.. VMI, beyond question, ``possesses to a far greater
degree'' than the VWIL program ``those qualities which are incapable of
objective measurement but which make for greatness in a . school,''
including ``position and influence of the alumni, standing in the
community, traditions and prestige.'' Women seeking and fit for a
VMI-quality education cannot be offered anything less, under the
Commonwealth's obligation to afford them genuinely equal protection.

A prime part of the history of our Constitution, historian Richard
Morris recounted, is the story of the extension of constitutional rights
and protections to people once ignored or excluded. n21 VMI's story
continued as our comprehension of ``We the People'' expanded. See,
n.~16. There is no reason to believe that the admission of women capable
of all the activities required of VMI cadets would destroy the Institute
rather than enhance its capacity to serve the ``more perfect Union.''

For the reasons stated, the initial judgment of the Court of Appeals,
CA4 1992), is affirmed, the final judgment of the Court of Appeals, CA4
1995), is reversed, and the case is remanded for further proceedings
consistent with this opinion.

It is so ordered.

\textbf{CHIEF JUSTICE REHNQUIST, concurring in the judgment.}

The Court holds first that Virginia violates the Equal Protection Clause
by maintaining the Virginia Military Institute's (VMI's) all-male
admissions policy, and second that establishing the Virginia Women's
Institute for Leadership (VWIL) program does not remedy that violation.
While I agree with these conclusions, I disagree with the Court's
analysis and so I write separately.

Two decades ago in Craig v. Boren, (1976), we announced that ``to
withstand constitutional challenge, . classifications by gender must
serve important governmental objectives and must be substantially
related to achievement of those objectives.'' We have adhered to that
standard of scrutiny ever since. See Califano v. Goldfarb, (1977);
Califano v. Webster, (1977); Orr v. Orr, (1979); Caban v. Mohammed,
(1979); Davis v. Passman, 235, n.~9, (1979); Personnel Administrator of
Mass. v. Feeney, (1979); Califano v. Westcott, (1979); Wengler v.
Druggists Mut. Ins. Co., (1980); Kirchberg v. Feenstra, (1981); Michael
M. v. Superior Court, Sonoma Cty., (1981); Mississippi Univ. for Women
v. Hogan, (1982); Heckler v. Mathews, (1984); J. E. B. v. Alabama ex
rel. T. B., n.~6, (1994). While the majority adheres to this test today,
it also says that the Commonwealth must demonstrate an ``\,`exceedingly
persuasive justification'\,'' to support a gender-based classification.
See 530, 531, 533, 534, 545, 546, 556. It is unfortunate that the Court
thereby introduces an element of uncertainty respecting the appropriate
test.

While terms like ``important governmental objective'' and
``substantially related'' are hardly models of precision, they have more
content and specificity than does the phrase ``exceedingly persuasive
justification.'' That phrase is best confined, as it was first used, as
an observation on the difficulty of meeting the applicable test, not as
a formulation of the test itself. See, e. g., Feeney (``These precedents
dictate that any state law overtly or covertly designed to prefer males
over females in public employment require an exceedingly persuasive
justification''). To avoid introducing potential confusion, I would have
adhered more closely to our traditional, ``firmly established,'' Hogan;
Heckler, standard that a gender-based classification ``must bear a close
and substantial relationship to important governmental objectives.''
Feeney.

Our cases dealing with gender discrimination also require that the
proffered purpose for the challenged law be the actual purpose. See
-536. It is on this ground that the Court rejects the first of two
justifications Virginia offers for VMI's single-sex admissions policy,
namely, the goal of diversity among its public educational institutions.
While I ultimately agree that the Commonwealth has not carried the day
with this justification, I disagree with the Court's method of analyzing
the issue.

VMI was founded in 1839, and, as the Court notes, admission was limited
to men because under the then-prevailing view men, not women, were
destined for higher education. However misguided this point of view may
be by present-day standards, it surely was not unconstitutional in 1839.
The adoption of the Fourteenth Amendment, with its Equal Protection
Clause, was nearly 30 years in the future. The interpretation of the
Equal Protection Clause to require heightened scrutiny for gender
discrimination was yet another century away.

Long after the adoption of the Fourteenth Amendment, and well into this
century, legal distinctions between men and women were thought to raise
no question under the Equal Protection Clause. The Court refers to our
decision in Goesaert v. Cleary 163, (1948). Likewise representing that
now abandoned view was Hoyt v. Florida 118, (1961), where the Court
upheld a Florida system of jury selection in which men were
automatically placed on jury lists, but women were placed there only if
they expressed an affirmative desire to serve. The Court noted that
despite advances in women's opportunities, the ``woman is still regarded
as the center of home and family life.''

Then, in 1971, we decided Reed v. Reed 225, which the Court correctly
refers to as a seminal case. But its facts have nothing to do with
admissions to any sort of educational institution. An Idaho statute
governing the administration of estates and probate preferred men to
women if the other statutory qualifications were equal. The statute's
purpose, according to the Idaho Supreme Court, was to avoid hearings to
determine who was better qualified as between a man and a woman both
applying for letters of administration. This Court held that such a rule
violated the Fourteenth Amendment because ``a mandatory preference to
members of either sex over members of the other, merely to accomplish
the elimination of hearings,'' was an ``arbitrary legislative choice
forbidden by the Equal Protection Clause.'' The brief opinion in Reed
made no mention of either Goesaert or Hoyt.

Even at the time of our decision in Reed v. Reed, therefore, Virginia
and VMI were scarcely on notice that its holding would be extended
across the constitutional board. They were entitled to believe that
``one swallow doesn't make a summer'' and await further developments.
Those developments were 11 years in coming. In Mississippi Univ. for
Women v. Hogana case actually involving a single-sex admissions policy
in higher education, the Court held that the exclusion of men from a
nursing program violated the Equal Protection Clause. This holding did
place Virginia on notice that VMI's men-only admissions policy was open
to serious question.

The VMI Board of Visitors, in response, appointed a Mission Study
Committee to examine ``the legality and wisdom of VMI's single-sex
policy in light of'' Hogan. 766 F. Supp. 1407, 1427 (WD Va. 1991). But
the committee ended up cryptically recommending against changing VMI's
status as a single-sex college. After three years of study, the
committee found ``\,`no information'\,'' that would warrant a change in
VMI's status. Even the District Court, ultimately sympathetic to VMI's
position, found that ``the Report provided very little indication of how
{[}its{]} conclusion was reached'' and that ``the one and one-half pages
in the committee's final report devoted to analyzing the information it
obtained primarily focuses on anticipated difficulties in attracting
females to VMI.'' The reasons given in the report for not changing the
policy were the changes that admission of women to VMI would require,
and the likely effect of those changes on the institution. That VMI
would have to change is simply not helpful in addressing the
constitutionality of the status after Hogan.

Before this Court, Virginia has sought to justify VMI's single-sex
admissions policy primarily on the basis that diversity in education is
desirable, and that while most of the public institutions of higher
learning in the Commonwealth are coeducational, there should also be
room for single-sex institutions. I agree with the Court that there is
scant evidence in the record that this was the real reason that Virginia
decided to maintain VMI as men only. * But, unlike the majority, I would
consider only evidence that postdates our decision in Hogan, and would
draw no negative inferences from the Commonwealth's actions before that
time. I think that after Hogan, the Commonwealth was entitled to
reconsider its policy with respect to VMI, and not to have earlier
justifications, or lack thereof, held against it.

Even if diversity in educational opportunity were the Commonwealth's
actual objective, the Commonwealth's position would still be
problematic. The difficulty with its position is that the diversity
benefited only one sex; there was single-sex public education available
for men at VMI, but no corresponding single-sex public education
available for women. When Hogan placed Virginia on notice that VMI's
admissions policy possibly was unconstitutional, VMI could have dealt
with the problem by admitting women; but its governing body felt
strongly that the admission of women would have seriously harmed the
institution's educational approach. Was there something else the
Commonwealth could have done to avoid an equal protection violation?
Since the Commonwealth did nothing, we do not have to definitively
answer that question.

I do not think, however, that the Commonwealth's options were as limited
as the majority may imply. The Court cites, without expressly approving
it, a statement from the opinion of the dissenting judge in the Court of
Appeals, to the effect that the Commonwealth could have ``simultaneously
opened single-gender undergraduate institutions having substantially
comparable curricular and extra-curricular programs, funding, physical
plant, administration and support services, and faculty and library
resources.'' (internal quotation marks omitted). If this statement is
thought to exclude other possibilities, it is too stringent a
requirement. VMI had been in operation for over a century and a half,
and had an established, successful, and devoted group of alumni. No
legislative wand could instantly call into existence a similar
institution for women; and it would be a tremendous loss to scrap VMI's
history and tradition. In the words of Grover Cleveland's second
inaugural address, the Commonwealth faced a condition, not a theory. And
it was a condition that had been brought about, not through defiance of
decisions construing gender bias under the Equal Protection Clause, but,
until the decision in Hogan, a condition that had not appeared to offend
the Constitution. Had Virginia made a genuine effort to devote
comparable public resources to a facility for women, and followed
through on such a plan, it might well have avoided an equal protection
violation. I do not believe the Commonwealth was faced with the stark
choice of either admitting women to VMI, on the one hand, or abandoning
VMI and starting from scratch for both men and women, on the other.

But, as I have noted, neither the governing board of VMI nor the
Commonwealth took any action after 1982. If diversity in the form of
single-sex, as well as coeducational, institutions of higher learning
were to be available to Virginians, that diversity had to be available
to women as well as to men.

The dissent criticizes me for ``disregarding the four all-women's
private colleges in Virginia (generously assisted by public funds).''
Post. The private women's colleges are treated by the Commonwealth
exactly as all other private schools are treated, which includes the
provision of tuition-assistance grants to Virginia residents. Virginia
gives no special support to the women's single-sex education. But
obviously, the same is not true for men's education. Had the
Commonwealth provided the kind of support for the private women's
schools that it provides for VMI, this may have been a very different
case. For in so doing, the Commonwealth would have demonstrated that its
interest in providing a single-sex education for men was to some measure
matched by an interest in providing the same opportunity for women.

Virginia offers a second justification for the single-sex admissions
policy: maintenance of the adversative method. I agree with the Court
that this justification does not serve an important governmental
objective. A State does not have substantial interest in the adversative
methodology unless it is pedagogically beneficial. While considerable
evidence shows that a single-sex education is pedagogically beneficial
for some students, see 766 F. Supp., and hence a State may have a valid
interest in promoting that methodology, there is no similar evidence in
the record that an adversative method is pedagogically beneficial or is
any more likely to produce character traits than other methodologies.

The Court defines the constitutional violation in these cases as ``the
categorical exclusion of women from an extraordinary educational
opportunity afforded to men.'' By defining the violation in this way,
and by emphasizing that a remedy for a constitutional violation must
place the victims of discrimination in ``\,`the position they would have
occupied in the absence of {[}discrimination{]},'\,'' , the Court
necessarily implies that the only adequate remedy would be the admission
of women to the all-male institution. As the foregoing discussion
suggests, I would not define the violation in this way; it is not the
``exclusion of women'' that violates the Equal Protection Clause, but
the maintenance of an all-men school without providing any -- much less
a comparable -- institution for women.

Accordingly, the remedy should not necessarily require either the
admission of women to VMI or the creation of a VMI clone for women. An
adequate remedy in my opinion might be a demonstration by Virginia that
its interest in educating men in a single-sex environment is matched by
its interest in educating women in a single-sex institution. To
demonstrate such, the Commonwealth does not need to create two
institutions with the same number of faculty Ph. D.'s, similar SAT
scores, or comparable athletic fields. See Nor would it necessarily
require that the women's institution offer the same curriculum as the
men's; one could be strong in computer science, the other could be
strong in liberal arts. It would be a sufficient remedy, I think, if the
two institutions offered the same quality of education and were of the
same overall caliber.

If a State decides to create single-sex programs, the State would, I
expect, consider the public's interest and demand in designing
curricula. And rightfully so. But the State should avoid assuming demand
based on stereotypes; it must not assume a priori, without evidence,
that there would be no interest in a women's school of civil
engineering, or in a men's school of nursing.

In the end, the women's institution Virginia proposes, VWIL, fails as a
remedy, because it is distinctly inferior to the existing men's
institution and will continue to be for the foreseeable future. VWIL
simply is not, in any sense, the institution that VMI is. In particular,
VWIL is a program appended to a private college, not a self-standing
institution; and VWIL is substantially underfunded as compared to VMI. I
therefore ultimately agree with the Court that Virginia has not provided
an adequate remedy.

\textbf{JUSTICE SCALIA, dissenting.}

Today the Court shuts down an institution that has served the people of
the Commonwealth of Virginia with pride and distinction for over a
century and a half. To achieve that desired result, it rejects (contrary
to our established practice) the factual findings of two courts below,
sweeps aside the precedents of this Court, and ignores the history of
our people. As to facts: It explicitly rejects the finding that there
exist ``gender-based developmental differences'' supporting Virginia's
restriction of the ``adversative'' method to only a men's institution,
and the finding that the all-male composition of the Virginia Military
Institute (VMI) is essential to that institution's character. As to
precedent: It drastically revises our established standards for
reviewing sex-based classifications. And as to history: It counts for
nothing the long tradition, enduring down to the present, of men's
military colleges supported by both States and the Federal Government.

Much of the Court's opinion is devoted to deprecating the
closed-mindedness of our forebears with regard to women's education, and
even with regard to the treatment of women in areas that have nothing to
do with education. Closed minded they were -- as every age is, including
our own, with regard to matters it cannot guess, because it simply does
not consider them debatable. The virtue of a democratic system with a
First Amendment is that it readily enables the people, over time, to be
persuaded that what they took for granted is not so, and to change their
laws accordingly. That system is destroyed if the smug assurances of
each age are removed from the democratic process and written into the
Constitution. So to counterbalance the Court's criticism of our
ancestors, let me say a word in their praise: They left us free to
change. The same cannot be said of this most illiberal Court, which has
embarked on a course of inscribing one after another of the current
preferences of the society (and in some cases only the counter
majoritarian preferences of the society's law-trained elite) into our
Basic Law. Today it enshrines the notion that no substantial educational
value is to be served by an all-men's military academy -- so that the
decision by the people of Virginia to maintain such an institution
denies equal protection to women who cannot attend that institution but
can attend others. Since it is entirely clear that the Constitution of
the United States -- the old one -- takes no sides in this educational
debate, I dissent.

I shall devote most of my analysis to evaluating the Court's opinion on
the basis of our current equal protection jurisprudence, which regards
this Court as free to evaluate everything under the sun by applying one
of three tests: ``rational basis'' scrutiny, intermediate scrutiny, or
strict scrutiny. These tests are no more scientific than their names
suggest, and a further element of randomness is added by the fact that
it is largely up to us which test will be applied in each case. Strict
scrutiny, we have said, is reserved for state ``classifications based on
race or national origin and classifications affecting fundamental
rights,'' Clark v. Jeter, (1988) (citation omitted). It is my position
that the term ``fundamental rights'' should be limited to
``interest{[}s{]} traditionally protected by our society,'' Michael H.
v. Gerald D., (1989) (plurality opinion of SCALIA, J.); but the Court
has not accepted that view, so that strict scrutiny will be applied to
the deprivation of whatever sort of right we consider ``fundamental.''
We have no established criterion for ``intermediate scrutiny'' either,
but essentially apply it when it seems like a good idea to load the
dice. So far it has been applied to content-neutral restrictions that
place an incidental burden on speech, to disabilities attendant to
illegitimacy, and to discrimination on the basis of sex. See, e. g.,
Turner Broadcasting System, Inc.~v. FCC, (1994); Mills v. Habluetzel,
(1982); Craig v. Boren, (1976).

I have no problem with a system of abstract tests such as rational
basis, intermediate, and strict scrutiny (though I think we can do
better than applying strict scrutiny and intermediate scrutiny whenever
we feel like it). Such formulas are essential to evaluating whether the
new restrictions that a changing society constantly imposes upon private
conduct comport with that ``equal protection'' our society has always
accorded in the past. But in my view the function of this Court is to
preserve our society's values regarding (among other things) equal
protection, not to revise them; to prevent backsliding from the degree
of restriction the Constitution imposed upon democratic government, not
to prescribe, on our own authority, progressively higher degrees. For
that reason it is my view that, whatever abstract tests we may choose to
devise, they cannot supersede -- and indeed ought to be crafted so as to
reflect -- those constant and unbroken national traditions that embody
the people's understanding of ambiguous constitutional texts. More
specifically, it is my view that ``when a practice not expressly
prohibited by the text of the Bill of Rights bears the endorsement of a
long tradition of open, widespread, and unchallenged use that dates back
to the beginning of the Republic, we have no proper basis for striking
it down.'' Rutan v. Republican Party of Ill., (1990) (SCALIA, J.,
dissenting). The same applies, mutatis mutandis, to a practice asserted
to be in violation of the post-Civil War Fourteenth Amendment. See, e.
g., Burnham v. Superior Court of Cal., County of Marin 631, (1990)
(plurality opinion of SCALIA, J.) (Due Process Clause); J. E. B. v.
Alabama ex rel. T. B., (1994) (SCALIA, J., dissenting) (Equal Protection
Clause); Planned Parenthood of Southeastern Pa. v. Casey, 1000-1001,
(1992) (SCALIA, J., dissenting) (various alleged ``penumbras'').

The all-male constitution of VMI comes squarely within such a governing
tradition. Founded by the Commonwealth of Virginia in 1839 and
continuously maintained by it since, VMI has always admitted only men.
And in that regard it has not been unusual. For almost all of VMI's more
than a century and a half of existence, its single-sex status reflected
the uniform practice for government-supported military colleges. Another
famous Southern institution, The Citadel, has existed as a state-funded
school of South Carolina since 1842. And all the federal military
colleges -- West Point, the Naval Academy at Annapolis, and even the Air
Force Academy, which was not established until 1954 -- admitted only
males for most of their history. Their admission of women in 1976 (upon
which the Court today relies, see nn.~13, 15) came not by court decree,
but because the people, through their elected representatives, decreed a
change. See, e. g., § 803(a), 89 Stat. 537, note following 10 U.S.C. §
4342. In other words, the tradition of having government-funded military
schools for men is as well rooted in the traditions of this country as
the tradition of sending only men into military combat. The people may
decide to change the one tradition, like the other, through democratic
processes; but the assertion that either tradition has been
unconstitutional through the centuries is not law, but
politics-smuggled-into-law.

And the same applies, more broadly, to single-sex education in general,
which, as I shall discuss, is threatened by today's decision with the
cutoff of all state and federal support. Government-run nonmilitary
educational institutions for the two sexes have until very recently also
been part of our national tradition. ``{[}It is{]} coeducation,
historically, {[}that{]} is a novel educational theory. From grade
school through high school, college, and graduate and professional
training, much of the Nation's population during much of our history has
been educated in sexually segregated classrooms.'' Mississippi Univ. for
Women v. Hogan, (1982) (Powell, J., dissenting); see -739. These
traditions may of course be changed by the democratic decisions of the
people, as they largely have been.

Today, however, change is forced upon Virginia, and reversion to
single-sex education is prohibited nationwide, not by democratic
processes but by order of this Court. Even while bemoaning the sorry,
bygone days of ``fixed notions'' concerning women's education, see and
n.~10, 537, the Court favors current notions so fixedly that it is
willing to write them into the Constitution of the United States by
application of custom-built ``tests.'' This is not the interpretation of
a Constitution, but the creation of one.

To reject the Court's disposition today, however, it is not necessary to
accept my view that the Court's made-up tests cannot displace
longstanding national traditions as the primary determinant of what the
Constitution means. It is only necessary to apply honestly the test the
Court has been applying to sex-based classifications for the past two
decades. It is well settled, as JUSTICE O'CONNOR stated some time ago
for a unanimous Court, that we evaluate a statutory classification based
on sex under a standard that lies ``between the extremes of rational
basis review and strict scrutiny.'' Clark v. Jeter, 486 U.S.. We have
denominated this standard ``intermediate scrutiny'' and under it have
inquired whether the statutory classification is ``substantially related
to an important governmental objective.'' See, e. g., Heckler v.
Mathews, (1984); Wengler v. Druggists Mut. Ins. Co., (1980); Craig v.
Boren, 429 U.S..

Before I proceed to apply this standard to VMI, I must comment upon the
manner in which the Court avoids doing so. Notwithstanding our
above-described precedents and their ``\,`firmly established
principles,'\,'' Heckler (quoting Hogan), the United States urged us to
hold in this litigation ``that strict scrutiny is the correct
constitutional standard for evaluating classifications that deny
opportunities to individuals based on their sex.'' Brief for United
States in No.~94-2107, p.~16. (This was in flat contradiction of the
Government's position below, which was, in its own words, to ``state
unequivocally that the appropriate standard in this case is
`intermediate scrutiny.'\,'' 2 Record, Doc. No.~88, p.~3 (emphasis
added).) The Court, while making no reference to the Government's
argument, effectively accepts it.

Although the Court in two places recites the test as stated in Hogan,
see -533, which asks whether the State has demonstrated ``that the
classification serves important governmental objectives and that the
discriminatory means employed are substantially related to the
achievement of those objectives,'' 458 U.S. (internal quotation marks
omitted), the Court never answers the question presented in anything
resembling that form. When it engages in analysis, the Court instead
prefers the phrase ``exceedingly persuasive justification'' from Hogan.
The Court's nine invocations of that phrase, see 530, 531, 533, 534,
545, 546, 556, and even its fanciful description of that imponderable as
``the core instruction'' of the Court's decisions in J. E. B. v. Alabama
ex rel. T. B. 89, (1994), and Hogansee would be unobjectionable if the
Court acknowledged that whether a ``justification'' is ``exceedingly
persuasive'' must be assessed by asking ``{[}whether{]} the
classification serves important governmental objectives and
{[}whether{]} the discriminatory means employed are substantially
related to the achievement of those objectives.'' Instead, however, the
Court proceeds to interpret ``exceedingly persuasive justification'' in
a fashion that contradicts the reasoning of Hogan and our other
precedents.

That is essential to the Court's result, which can only be achieved by
establishing that intermediate scrutiny is not survived if there are
some women interested in attending VMI, capable of undertaking its
activities, and able to meet its physical demands. Thus, the Court
summarizes its holding as follows:

``In contrast to the generalizations about women on which Virginia
rests, we note again these dispositive realities: VMI's implementing
methodology is not inherently unsuitable to women; some women do well
under the adversative model; some women, at least, would want to attend
VMI if they had the opportunity; some women are capable of all of the
individual activities required of VMI cadets and can meet the physical
standards VMI now imposes on men.'' (internal quotation marks,
citations, and punctuation omitted; emphasis added). Similarly, the
Court states that ``the Commonwealth's justification for excluding all
women from `citizen-soldier' training for which some are qualified .
cannot rank as `exceedingly persuasive' . .'' n

\begin{center}\rule{0.5\linewidth}{\linethickness}\end{center}

Footnotes

n1 Accord, (``In sum . ., neither the goal of producing
citizen-soldiers, VMI's raison d'etre, nor VMI's implementing
methodology is inherently unsuitable to women'') (internal quotation
marks omitted; emphasis added); (``The question is whether the
Commonwealth can constitutionally deny to women who have the will and
capacity, the training and attendant opportunities that VMI uniquely
affords''); (the ``violation'' is that ``equal protection {[}has been{]}
denied to women ready, willing, and able to benefit from educational
opportunities of the kind VMI offers''); (``As earlier stated, see,
generalizations about `the way women are,' estimates of what is
appropriate for most women, no longer justify denying opportunity to
women whose talent and capacity place them outside the average
description'').

Only the amorphous ``exceedingly persuasive justification'' phrase, and
not the standard elaboration of intermediate scrutiny, can be made to
yield this conclusion that VMI's single-sex composition is
unconstitutional because there exist several women (or, one would have
to conclude under the Court's reasoning, a single woman) willing and
able to undertake VMI's program. Intermediate scrutiny has never
required a least-restrictive-means analysis, but only a ``substantial
relation'' between the classification and the state interests that it
serves. Thus, in Califano v. Webster 360, (1977) (per curiam), we upheld
a congressional statute that provided higher Social Security benefits
for women than for men. We reasoned that ``women . as such have been
unfairly hindered from earning as much as men,'' but we did not require
proof that each woman so benefited had suffered discrimination or that
each disadvantaged man had not; it was sufficient that even under the
former congressional scheme ``women on the average received lower
retirement benefits than men.'' and n.~5 (emphasis added). The reasoning
in our other intermediate-scrutiny cases has similarly required only a
substantial relation between end and means, not a perfect fit. In
Rostker v. Goldberg 478, (1981), we held that selective-service
registration could constitutionally exclude women, because even
``assuming that a small number of women could be drafted for noncombat
roles, Congress simply did not consider it worth the added burdens of
including women in draft and registration plans.'' In Metro
Broadcasting, Inc.~v. FCC, 582-583, (1990), overruled on other grounds,
Adarand Constructors, Inc.~v. Pena, (1995), we held that a
classification need not be accurate ``in every case'' to survive
intermediate scrutiny so long as, ``in the aggregate,'' it advances the
underlying objective. There is simply no support in our cases for the
notion that a sex-based classification is invalid unless it relates to
characteristics that hold true in every instance.

Not content to execute a de facto abandonment of the intermediate
scrutiny that has been our standard for sex-based classifications for
some two decades, the Court purports to reserve the question whether,
even in principle, a higher standard (i. e., strict scrutiny) should
apply. ``The Court has,'' it says, ``thus far reserved most stringent
judicial scrutiny for classifications based on race or national origin .
.,'' n.~6 (emphasis added); and it describes our earlier cases as having
done no more than decline to ``equate gender classifications, for all
purposes, to classifications based on race or national origin,''
(emphasis added). The wonderful thing about these statements is that
they are not actually false -- just as it would not be actually false to
say that ``our cases have thus far reserved the `beyond a reasonable
doubt' standard of proof for criminal cases,'' or that ``we have not
equated tort actions, for all purposes, to criminal prosecutions.'' But
the statements are misleading, insofar as they suggest that we have not
already categorically held strict scrutiny to be inapplicable to
sex-based classifications. See, e. g., Heckler v. Mathews (1984)
(upholding state action after applying only intermediate scrutiny);
Michael M. v. Superior Court, Sonoma Cty. (same) (plurality and both
concurring opinions); Califano v. Webster(same) (per curiam). And the
statements are irresponsible, insofar as they are calculated to
destabilize current law. Our task is to clarify the law -- not to muddy
the waters, and not to exact overcompliance by intimidation. The States
and the Federal Government are entitled to know before they act the
standard to which they will be held, rather than be compelled to guess
about the outcome of Supreme Court peek-a-boo.

The Court's intimations are particularly out of place because it is
perfectly clear that, if the question of the applicable standard of
review for sex-based classifications were to be regarded as an
appropriate subject for reconsideration, the stronger argument would be
not for elevating the standard to strict scrutiny, but for reducing it
to rational-basis review. The latter certainly has a firmer foundation
in our past jurisprudence: Whereas no majority of the Court has ever
applied strict scrutiny in a case involving sex-based classifications,
we routinely applied rational-basis review until the 1970's. And of
course normal, rational-basis review of sex-based classifications would
be much more in accord with the genesis of heightened standards of
judicial review, the famous footnote in United States v. Carolene
Products Co, which said (intimatingly) that we did not have to inquire
in the case at hand``whether prejudice against discrete and insular
minorities may be a special condition, which tends seriously to curtail
the operation of those political processes ordinarily to be relied upon
to protect minorities, and which may call for a correspondingly more
searching judicial inquiry.'' It is hard to consider women a ``discrete
and insular minority'' unable to employ the ``political processes
ordinarily to be relied upon,'' when they constitute a majority of the
electorate. And the suggestion that they are incapable of exerting that
political power smacks of the same paternalism that the Court so roundly
condemns. Moreover, a long list of legislation proves the proposition
false. See, e. g., Equal Pay Act of 1963, 29 U.S.C. § 206(d); Title VII
of the Civil Rights Act of 1964, 42 U.S.C. § 2000e-2; Title IX of the
Education Amendments of 1972, 20 U.S.C. § 1681; Women's Business
Ownership Act of 1988, Pub. L. 100Stat. 2689; Violence Against Women Act
of 1994, Pub. L. 103-322, Title IV, 108 Stat. 1902.

With this explanation of how the Court has succeeded in making its
analysis seem orthodox -- and indeed, if intimations are to be believed,
even overly generous to VMI -- I now proceed to describe how the
analysis should have been conducted. The question to be answered, I
repeat, is whether the exclusion of women from VMI is ``substantially
related to an important governmental objective.''

It is beyond question that Virginia has an important state interest in
providing effective college education for its citizens. That single-sex
instruction is an approach substantially related to that interest should
be evident enough from the long and continuing history in this country
of men's and women's colleges. But beyond that, as the Court of Appeals
here stated: ``That single-gender education at the college level is
beneficial to both sexes is a fact established in this case.'' CA4 1995)
(emphasis added).

The evidence establishing that fact was overwhelming -- indeed,
``virtually uncontradicted'' in the words of the court that received the
evidence. As an initial matter, Virginia demonstrated at trial that
``{[}a{]} substantial body of contemporary scholarship and research
supports the proposition that, although males and females have
significant areas of developmental overlap, they also have differing
developmental needs that are deep-seated.'' While no one questioned that
for many students a coeducational environment was nonetheless not
inappropriate, that could not obscure the demonstrated benefits of
single-sex colleges. For example, the District Court stated as follows:

``One empirical study in evidence, not questioned by any expert,
demonstrates that single-sex colleges provide better educational
experiences than coeducational institutions. Students of both sexes
become more academically involved, interact with faculty frequently,
show larger increases in intellectual self-esteem and are more satisfied
with practically all aspects of college experience (the sole exception
is social life) compared with their counterparts in coeducational
institutions. Attendance at an all-male college substantially increases
the likelihood that a student will carry out career plans in law,
business and college teaching, and also has a substantial positive
effect on starting salaries in business. Women's colleges increase the
chances that those who attend will obtain positions of leadership,
complete the baccalaureate degree, and aspire to higher degrees.'' .See
also -1435 (factual findings). ``In the light of this very substantial
authority favoring single-sex education,'' the District Court concluded
that ``the VMI Board's decision to maintain an all-male institution is
fully justified even without taking into consideration the other unique
features of VMI's teaching and training.'' This finding alone, which
even this Court cannot dispute, see should be sufficient to demonstrate
the constitutionality of VMI's all-male composition.

But besides its single-sex constitution, VMI is different from other
colleges in another way. It employs a ``distinctive educational
method,'' sometimes referred to as the ``adversative, or doubting, model
of education.'' 766 F. Supp., 1421. ``Physical rigor, mental stress,
absolute equality of treatment, absence of privacy, minute regulation of
behavior, and indoctrination in desirable values are the salient
attributes of the VMI educational experience.'' No one contends that
this method is appropriate for all individuals; education is not a ``one
size fits all'' business. Just as a State may wish to support junior
colleges, vocational institutes, or a law school that emphasizes case
practice instead of classroom study, so too a State's decision to
maintain within its system one school that provides the adversative
method is ``substantially related'' to its goal of good education.
Moreover, it was uncontested that ``if the state were to establish a
women's VMI-type {[}i. e., adversative{]} program, the program would
attract an insufficient number of participants to make the program
work,'' 44 F d; and it was found by the District Court that if Virginia
were to include women in VMI, the school ``would eventually find it
necessary to drop the adversative system altogether,'' 766 F. Supp..
Thus, Virginia's options were an adversative method that excludes women
or no adversative method at all.

There can be no serious dispute that, as the District Court found,
single-sex education and a distinctive educational method ``represent
legitimate contributions to diversity in the Virginia higher education
system.'' As a theoretical matter, Virginia's educational interest would
have been best served (insofar as the two factors we have mentioned are
concerned) by six different types of public colleges -- an all-men's, an
all-women's, and a coeducational college run in the ``adversative
method,'' and an all-men's, an all-women's, and a coeducational college
run in the ``traditional method.'' But as a practical matter, of course,
Virginia's financial resources, like any State's, are not limitless, and
the Commonwealth must select among the available options. Virginia thus
has decided to fund, in addition to some 14 coeducational 4-year
colleges, one college that is run as an all-male school on the
adversative model: the Virginia Military Institute.

Virginia did not make this determination regarding the make-up of its
public college system on the unrealistic assumption that no other
colleges exist. Substantial evidence in the District Court demonstrated
that the Commonwealth has long proceeded on the principle that
``\,`higher education resources should be viewed as a whole -- public
and private'\,'' -- because such an approach enhances diversity and
because ``\,`it is academic and economic waste to permit unwarranted
duplication.'\,'' (quoting 1974 Report of the General Assembly
Commission on Higher Education to the General Assembly of Virginia). It
is thus significant that, whereas there are ``four all-female private
{[}colleges{]} in Virginia,'' there is only ``one private all-male
college,'' which ``indicates that the private sector is providing for
the {[}former{]} form of education to a much greater extent that it
provides for all-male education.'' In these circumstances, Virginia's
election to fund one public all-male institution and one on the
adversative model -- and to concentrate its resources in a single entity
that serves both these interests in diversity -- is substantially
related to the Commonwealth's important educational interests.

The Court today has no adequate response to this clear demonstration of
the conclusion produced by application of intermediate scrutiny. Rather,
it relies on a series of contentions that are irrelevant or erroneous as
a matter of law, foreclosed by the record in this litigation, or both.

\begin{enumerate}
\def\labelenumi{\arabic{enumi}.}
\item
  I have already pointed out the Court's most fundamental error, which
  is its reasoning that VMI's all-male composition is unconstitutional
  because ``some women are capable of all of the individual activities
  required of VMI cadets,'' and would prefer military training on the
  adversative model. See. This unacknowledged adoption of what amounts
  to (at least) strict scrutiny is without antecedent in our
  sex-discrimination cases and by itself discredits the Court's
  decision.
\item
  The Court suggests that Virginia's claimed purpose in maintaining VMI
  as an all-male institution -- its asserted interest in promoting
  diversity of educational options -- is not ``genuine,'' but is a
  pretext for discriminating against women. ; see To support this
  charge, the Court would have to impute that base motive to VMI's
  Mission Study Committee, which conducted a 3-year study from 1983 to
  1986 and recommended to VMI's Board of Visitors that the school remain
  all male. The committee, a majority of whose members consisted of
  non-VMI graduates, ``read materials on education and on women in the
  military,'' ``made site visits to single-sex and newly coeducational
  institutions'' including West Point and the Naval Academy, and
  ``considered the reasons that other institutions had changed from
  single-sex to coeducational status''; its work was praised as
  ``thorough'' in the accreditation review of VMI conducted by the
  Southern Association of Colleges and Schools. See 766 F. Supp., 1428;
  see also -1430 (detailed findings of fact concerning the Mission Study
  Committee). The Court states that ``whatever internal purpose the
  Mission Study Committee served -- and however well meaning the framers
  of the report -- we can hardly extract from that effort any
  commonwealth policy evenhandedly to advance diverse educational
  options.'' But whether it is part of the evidence to prove that
  diversity was the Commonwealth's objective (its short report said
  nothing on that particular subject) is quite separate from whether it
  is part of the evidence to prove that antifeminism was not. The
  relevance of the Mission Study Committee is that its very creation,
  its sober 3-year study, and the analysis it produced utterly refute
  the claim that VMI has elected to maintain its all-male student-body
  composition for some misogynistic reason.
\end{enumerate}

The Court also supports its analysis of Virginia's ``actual state
purposes'' in maintaining VMI's student body as all male by stating that
there is no explicit statement in the record ``\,`in which the
Commonwealth has expressed itself'\,'' concerning those purposes. That
is wrong on numerous grounds. First and foremost, in its implication
that such an explicit statement of ``actual purposes'' is needed. The
Court adopts, in effect, the argument of the United States that since
the exclusion of women from VMI in 1839 was based on the ``assumptions''
of the time ``that men alone were fit for military and leadership
roles,'' and since ``before this litigation was initiated, Virginia
never sought to supply a valid, contemporary rationale for VMI's
exclusionary policy,'' ``that failure itself renders the VMI policy
invalid.'' Brief for United States in No.~94-2107. This is an unheard-of
doctrine. Each state decision to adopt or maintain a governmental policy
need not be accompanied -- in anticipation of litigation and on pain of
being found to lack a relevant state interest -- by a lawyer's
contemporaneous recitation of the State's purposes. The Constitution is
not some giant Administrative Procedure Act, which imposes upon the
States the obligation to set forth a ``statement of basis and purpose''
for their sovereign Acts, see 5 U.S.C. § 553(c). The situation would be
different if what the Court assumes to have been the 1839 policy had
been enshrined and remained enshrined in legislation -- a VMI charter,
perhaps, pronouncing that the institution's purpose is to keep women in
their place. But since the 1839 policy was no more explicitly recorded
than the Court contends the present one is, the mere fact that today's
Commonwealth continues to fund VMI ``is enough to answer {[}the United
States'{]} contention that the {[}classification{]} was the `accidental
by-product of a traditional way of thinking about females.'\,''

It is, moreover, not true that Virginia's contemporary reasons for
maintaining VMI are not explicitly recorded. It is hard to imagine a
more authoritative source on this subject than the 1990 Report of the
Virginia Commission on the University of the 21st Century (1990 Report).
As the parties stipulated, that report ``notes that the hallmarks of
Virginia's educational policy are `diversity and autonomy.'\,''
Stipulations of Fact 37, reprinted in Lodged Materials from the Record
64 (Lodged Materials). It said: ``The formal system of higher education
in Virginia includes a great array of institutions: state-supported and
independent, two-year and senior, research and highly specialized,
traditionally black and single-sex.'' 1990 Report, quoted in relevant
part at Lodged Materials 64-65 (emphasis added). n2 The Court's only
response to this is repeated reliance on the Court of Appeals' assertion
that ``\,`the only explicit {[}statement{]} that we have found in the
record in which the Commonwealth has expressed itself with respect to
gender distinctions'\,'' (namely, the statement in the 1990 Report that
the Commonwealth's institutions must ``deal with faculty, staff, and
students without regard to sex'') had nothing to do with the purpose of
diversity. (quoting 976 F d). This proves, I suppose, that the Court of
Appeals did not find a statement dealing with sex and diversity in the
record; but the pertinent question (accepting the need for such a
statement) is whether it was there. And the plain fact, which the Court
does not deny, is that it was.

This statement is supported by other evidence in the record
demonstrating, by reference to both public and private institutions,
that Virginia actively seeks to foster its ``\,`rich heritage of
pluralism and diversity in higher education,'\,'' 1969 Report of the
Virginia Commission on Constitutional Revision, quoted in relevant part
at Lodged Materials 53; that Virginia views ``\,`one special
characteristic of the Virginia system {[}as being{]} its diversity,'\,''
1989 Virginia Plan for Higher Education, quoted in relevant part at
Lodged Materials 64; and that in the Commonwealth's view ``higher
education resources should be viewed as a whole -- public and private''
because ``\,`Virginia needs the diversity inherent in a dual system of
higher education,'\,'' 1974 Report of the General Assembly Commission on
Higher Education to the General Assembly of Virginia, quoted in 766 F.
Supp. 1407, 1420 (WD Va. 1991). See also Budget Initiatives for
1990-1992 of State Council of Higher Education for Virginia 10 (June 21,
1989) (Budget Initiatives), quoted at n.~3, infra. It should be noted
(for this point will be crucial to my later discussion) that these
official reports quoted here, in text and footnote, regard the
Commonwealth's educational system -- public and private -- as a unitary
one.

The Court contends that ``{[}a{]} purpose genuinely to advance an array
of educational options . is not served'' by VMI. It relies on the fact
that all of Virginia's other public colleges have become coeducational.
The apparent theory of this argument is that unless Virginia pursues a
great deal of diversity, its pursuit of some diversity must be a sham.
This fails to take account of the fact that Virginia's resources cannot
support all possible permutations of schools, see, and of the fact that
Virginia coordinates its public educational offerings with the offerings
of in-state private educational institutions that the Commonwealth
provides money for its residents to attend and otherwise assists --
which include four women's colleges. n

The Commonwealth provides tuition assistance, scholarship grants,
guaranteed loans, and work-study funds for residents of Virginia who
attend private colleges in the Commonwealth. These programs involve
substantial expenditures: for example, Virginia appropriated \$
4,413,750 (not counting federal funds it also earmarked) for the College
Scholarship Assistance Program for both 1996 and 1997, and for the
Tuition Assistance Grant Program appropriated \$ 21,568,000 for 1996 and
\$ 25,842,000 for 1997.

In addition, as the parties stipulated in the District Court, the
Commonwealth provides other financial support and assistance to private
institutions -- including single-sex colleges -- through low-cost
building loans, state-funded services contracts, and other programs. The
State Council of Higher Education for Virginia, in a 1989 document not
created for purposes of this litigation but introduced into evidence,
has described these various programs as a ``means by which the
Commonwealth can provide funding to its independent institutions,
thereby helping to maintain a diverse system of higher education.''

Finally, the Court unreasonably suggests that there is some pretext in
Virginia's reliance upon decentralized decisionmaking to achieve
diversity -- its granting of substantial autonomy to each institution
with regard to student-body composition and other matters, see 766 F.
Supp.. The Court adopts the suggestion of the Court of Appeals that it
is not possible for ``one institution with autonomy, but with no
authority over any other state institution, {[}to{]} give effect to a
state policy of diversity among institutions.'' (internal quotation
marks omitted). If it were impossible for individual human beings (or
groups of human beings) to act autonomously in effective pursuit of a
common goal, the game of soccer would not exist. And where the goal is
diversity in a free market for services, that tends to be achieved even
by autonomous actors who act out of entirely selfish interests and make
no effort to cooperate. Each Virginia institution, that is to say, has a
natural incentive to make itself distinctive in order to attract a
particular segment of student applicants. And of course none of the
institutions is entirely autonomous; if and when the legislature decides
that a particular school is not well serving the interest of diversity
-- if it decides, for example, that a men's school is not much needed --
funding will cease.

The Court, unfamiliar with the Commonwealth's policy of diverse and
independent institutions, and in any event careless of state and local
traditions, must be forgiven by Virginians for quoting a reference to
``\,`the Charlottesville campus'\,'' of the University of Virginia. See
The University of Virginia, an institution even older than VMI, though
not as old as another of the Commonwealth's universities, the College of
William and Mary, occupies the portion of Charlottesville known, not as
the ``campus,'' but as ``the grounds.'' More importantly, even if it
were a ``campus,'' there would be no need to specify ``the
Charlottesville campus,'' as one might refer to the Bloomington or
Indianapolis campus of Indiana University. Unlike university systems
with which the Court is perhaps more familiar, such as those in New York
(e. g., the State University of New York at Binghamton or Buffalo),
Illinois (University of Illinois at Urbana-Champaign or at Chicago), and
California (University of California, Los Angeles, or University of
California, Berkeley), there is only one University of Virginia. It
happens (because Thomas Jefferson lived near there) to be located at
Charlottesville. To many Virginians it is known, simply, as ``the
University,'' which suffices to distinguish it from the Commonwealth's
other institutions offering 4-year college instruction, which include
Christopher Newport College, Clinch Valley College, the College of
William and Mary, George Mason University, James Madison University,
Longwood College, Mary Washington University, Norfolk State University,
Old Dominion University, Radford University, Virginia Commonwealth
University, Virginia Polytechnic Institute and State University,
Virginia State University -- and, of course, VMI.

\begin{enumerate}
\def\labelenumi{\arabic{enumi}.}
\setcounter{enumi}{2}
\tightlist
\item
  In addition to disparaging Virginia's claim that VMI's single-sex
  status serves a state interest in diversity, the Court finds fault
  with Virginia's failure to offer education based on the adversative
  training method to women. It dismisses the District Court's
  ``\,`findings' on `gender-based developmental differences'\,'' on the
  ground that ``these `findings' restate the opinions of Virginia's
  expert witnesses, opinions about typically male or typically female
  `tendencies.'\,'' (quoting 766 F. Supp.). How remarkable to criticize
  the District Court on the ground that its findings rest on the
  evidence (i. e., the testimony of Virginia's witnesses)! That is what
  findings are supposed to do. It is indefensible to tell the
  Commonwealth that ``the burden of justification is demanding and it
  rests entirely on {[}you{]},'' and then to ignore the District Court's
  findings because they rest on the evidence put forward by the
  Commonwealth -- particularly when, as the District Court said, ``the
  evidence in the case . is virtually uncontradicted.''
\end{enumerate}

Ultimately, in fact, the Court does not deny the evidence supporting
these findings. See It instead makes evident that the parties to this
litigation could have saved themselves a great deal of time, trouble,
and expense by omitting a trial. The Court simply dispenses with the
evidence submitted at trial -- it never says that a single finding of
the District Court is clearly erroneous -- in favor of the Justices' own
view of the world, which the Court proceeds to support with (1)
references to observations of someone who is not a witness, nor even an
educational expert, nor even a judge who reviewed the record or
participated in the judgment below, but rather a judge who merely
dissented from the Court of Appeals' decision not to rehear this
litigation en banc, see (2) citations of nonevidentiary materials such
as amicus curiae briefs filed in this Court, see nn.~13, 14, and (3)
various historical anecdotes designed to demonstrate that Virginia's
support for VMI as currently constituted reminds the Justices of the
``bad old days.''

It is not too much to say that this approach to the litigation has
rendered the trial a sham. But treating the evidence as irrelevant is
absolutely necessary for the Court to reach its conclusion. Not a single
witness contested, for example, Virginia's ``substantial body of
`exceedingly persuasive' evidence . that some students, both male and
female, benefit from attending a single-sex college'' and ``{[}that{]}
for those students, the opportunity to attend a single-sex college is a
valuable one, likely to lead to better academic and professional
achievement.'' 766 F. Supp.. Even the United States' expert witness
``called himself a `believer in single-sex education,'\,'' although it
was his ``personal, philosophical preference,'' not one ``born of
educational-benefit considerations,'' ``that single-sex education should
be provided only by the private sector.''

\begin{enumerate}
\def\labelenumi{\arabic{enumi}.}
\setcounter{enumi}{3}
\tightlist
\item
  The Court contends that Virginia, and the District Court, erred, and
  ``misperceived our precedent,'' by ``training their argument on
  `means' rather than `end,'\,'' The Court focuses on ``VMI's mission,''
  which is to produce individuals ``imbued with love of learning,
  confident in the functions and attitudes of leadership, possessing a
  high sense of public service, advocates of the American democracy and
  free enterprise system, and ready . to defend their country in time of
  national peril.'' 766 F. Supp. (quoting Mission Study Committee of the
  VMI Board of Visitors, Report, May 16, 1986). ``Surely,'' the Court
  says, ``that goal is great enough to accommodate women.''
\end{enumerate}

This is lawmaking by indirection. What the Court describes as ``VMI's
mission'' is no less the mission of all Virginia colleges. Which of them
would the Old Dominion continue to fund if they did not aim to create
individuals ``imbued with love of learning, etc.,'' right down to being
ready ``to defend their country in time of national peril''? It can be
summed up as ``learning, leadership, and patriotism.'' To be sure, those
general educational values are described in a particularly martial
fashion in VMI's mission statement, in accordance with the military,
adversative, and all-male character of the institution. But imparting
those values in that fashion -- i. e., in a military, adversative,
all-male environment -- is the distinctive mission of VMI. And as I have
discussed (and both courts below found), that mission is not ``great
enough to accommodate women.''

The Court's analysis at least has the benefit of producing foreseeable
results. Applied generally, it means that whenever a State's ultimate
objective is ``great enough to accommodate women'' (as it always will
be), then the State will be held to have violated the Equal Protection
Clause if it restricts to men even one means by which it pursues that
objective -- no matter how few women are interested in pursuing the
objective by that means, no matter how much the single-sex program will
have to be changed if both sexes are admitted, and no matter how
beneficial that program has theretofore been to its participants.

\begin{enumerate}
\def\labelenumi{\arabic{enumi}.}
\setcounter{enumi}{4}
\tightlist
\item
  The Court argues that VMI would not have to change very much if it
  were to admit women. See, e. g., The principal response to that
  argument is that it is irrelevant: If VMI's single-sex status is
  substantially related to the government's important educational
  objectives, as I have demonstrated above and as the Court refuses to
  discuss, that concludes the inquiry. There should be no debate in the
  federal judiciary over ``how much'' VMI would be required to change if
  it admitted women and whether that would constitute ``too much''
  change.
\end{enumerate}

But if such a debate were relevant, the Court would certainly be on the
losing side. The District Court found as follows: ``The evidence
establishes that key elements of the adversative VMI educational system,
with its focus on barracks life, would be fundamentally altered, and the
distinctive ends of the system would be thwarted, if VMI were forced to
admit females and to make changes necessary to accommodate their needs
and interests.'' 766 F. Supp.. Changes that the District Court's
detailed analysis found would be required include new allowances for
personal privacy in the barracks, such as locked doors and coverings on
windows, which would detract from VMI's approach of regulating minute
details of student behavior, ``contradict the principle that everyone is
constantly subject to scrutiny by everyone else,'' and impair VMI's
``total egalitarian approach'' under which every student must be
``treated alike''; changes in the physical training program, which would
reduce ``the intensity and aggressiveness of the current program''; and
various modifications in other respects of the adversative training
program that permeates student life. See As the Court of Appeals
summarized it, ``the record supports the district court's findings that
at least these three aspects of VMI's program -- physical training, the
absence of privacy, and the adversative approach -- would be materially
affected by coeducation, leading to a substantial change in the
egalitarian ethos that is a critical aspect of VMI's training.''

In the face of these findings by two courts below, amply supported by
the evidence, and resulting in the conclusion that VMI would be
fundamentally altered if it admitted women, this Court simply pronounces
that ``the notion that admission of women would downgrade VMI's stature,
destroy the adversative system and, with it, even the school, is a
judgment hardly proved.'' (footnote omitted). The point about
``downgrading VMI's stature'' is a straw man; no one has made any such
claim. The point about ``destroying the adversative system'' is simply
false; the District Court not only stated that ``evidence supports this
theory,'' but specifically concluded that while ``without a doubt'' VMI
could assimilate women, ``it is equally without a doubt that VMI's
present methods of training and education would have to be changed'' by
a ``move away from its adversative new cadet system.'' And the point
about ``destroying the school,'' depending upon what that ambiguous
phrase is intended to mean, is either false or else sets a standard much
higher than VMI had to meet. It sufficed to establish, as the District
Court stated, that VMI would be ``significantly different'' upon the
admission of women, and ``would eventually find it necessary to drop the
adversative system altogether.''

The Court's do-it-yourself approach to factfinding, which throughout is
contrary to our well-settled rule that we will not ``undertake to review
concurrent findings of fact by two courts below in the absence of a very
obvious and exceptional showing of error,'' is exemplified by its
invocation of the experience of the federal military academies to prove
that not much change would occur. In fact, the District Court noted that
``the West Point experience'' supported the theory that a coeducational
VMI would have to ``adopt a {[}different{]} system,'' for West Point
found it necessary upon becoming coeducational to ``move away'' from its
adversative system. ``Without a doubt, VMI's present methods of training
and education would have to be changed as West Point's were.''

\begin{enumerate}
\def\labelenumi{\arabic{enumi}.}
\setcounter{enumi}{5}
\tightlist
\item
  Finally, the absence of a precise ``all-women's analogue'' to VMI is
  irrelevant. In Mississippi Univ. for Women v. Hogan, we attached no
  constitutional significance to the absence of an all-male nursing
  school. As Virginia notes, if a program restricted to one sex is
  necessarily unconstitutional unless there is a parallel program
  restricted to the other sex, ``the opinion in Hogan could have ended
  with its first footnote, which observed that `Mississippi maintains no
  other single-sex public university or college.'\,''
\end{enumerate}

Although there is no precise female-only analogue to VMI, Virginia has
created during this litigation the Virginia Women's Institute for
Leadership (VWIL), a state-funded all-women's program run by Mary
Baldwin College. I have thus far said nothing about VWIL because it is,
under our established test, irrelevant, so long as VMI's all-male
character is ``substantially related'' to an important state goal. But
VWIL now exists, and the Court's treatment of it shows how far reaching
today's decision is.

VWIL was carefully designed by professional educators who have long
experience in educating young women. The program rejects the proposition
that there is a ``difference in the respective spheres and destinies of
man and woman,'' Bradwell v. State, and is designed to ``provide an
all-female program that will achieve substantially similar outcomes
{[}to VMI's{]} in an all-female environment.'' After holding a trial
where voluminous evidence was submitted and making detailed findings of
fact, the District Court concluded that ``there is a legitimate
pedagogical basis for the different means employed {[}by VMI and VWIL{]}
to achieve the substantially similar ends.'' The Court of Appeals
undertook a detailed review of the record and affirmed. But it is Mary
Baldwin College, which runs VWIL, that has made the point most
succinctly:

``It would have been possible to develop the VWIL program to more
closely resemble VMI, with adversative techniques associated with the
rat line and barracks-like living quarters. Simply replicating an
existing program would have required far less thought, research, and
educational expertise. But such a facile approach would have produced a
paper program with no real prospect of successful implementation.''
Brief for Mary Baldwin College as Amicus Curiae 5.It is worth noting
that none of the United States' own experts in the remedial phase of
this litigation was willing to testify that VMI's adversative method was
an appropriate methodology for educating women. This Court, however,
does not care. Even though VWIL was carefully designed by professional
educators who have tremendous experience in the area, and survived the
test of adversarial litigation, the Court simply declares, with no basis
in the evidence, that these professionals acted on ``\,`overbroad'
generalizations.''

The Court is incorrect in suggesting that the Court of Appeals applied a
``deferential'' ``brand of review inconsistent with the more exacting
standard our precedent requires.'' That court ``inquired (1) whether the
state's objective is `legitimate and important,' and (2) whether `the
requisite direct, substantial relationship between objective and means
is present.'\,'' To be sure, such review is ``deferential'' to a degree
that the Court's new standard is not, for it is intermediate scrutiny.
(The Court cannot evade this point or prove the Court of Appeals too
deferential by stating that that court ``devised another test, a
`substantive comparability' inquiry,''' for as that court explained, its
``substantive comparability'' inquiry was an ``additional step'' that it
engrafted on ``the traditional test'' of intermediate scrutiny.

A few words are appropriate in response to the concurrence, which finds
VMI unconstitutional on a basis that is more moderate than the Court's
but only at the expense of being even more implausible. The concurrence
offers three reasons: First, that there is ``scant evidence in the
record,'' that diversity of educational offering was the real reason for
Virginia's maintaining VMI. ``Scant'' has the advantage of being an
imprecise term. I have cited the clearest statements of diversity as a
goal for higher education in the 1990 Report, the 1989 Virginia Plan for
Higher Education, the Budget Initiatives prepared in 1989 by the State
Council of Higher Education for Virginia, the 1974 Report of the General
Assembly Commission on Higher Education to the General Assembly of
Virginia, and the 1969 Report of the Virginia Commission on
Constitutional Revision. See, 581-582, and n.~2, 583, n.~3. There is no
evidence to the contrary, once one rejects (as the concurrence rightly
does) the relevance of VMI's founding in days when attitudes toward the
education of women were different. Is this conceivably not enough to
foreclose rejecting as clearly erroneous the District Court's
determination regarding ``the Commonwealth's objective of educational
diversity''? 766 F. Supp.. Especially since it is absurd on its face
even to demand ``evidence'' to prove that the Commonwealth's reason for
maintaining a men's military academy is that a men's military academy
provides a distinctive type of educational experience (i. e., fosters
diversity). What other purpose would the Commonwealth have? One may
argue, as the Court does, that this type of diversity is designed only
to indulge hostility toward women -- but that is a separate point,
explicitly rejected by the concurrence, and amply refuted by the
evidence I have mentioned in discussing the Court's opinion. What is now
under discussion -- the concurrence's making central to the disposition
of this litigation the supposedly ``scant'' evidence that Virginia
maintained VMI in order to offer a diverse educational experience -- is
rather like making crucial to the lawfulness of the United States Army
record ``evidence'' that its purpose is to do battle. A legal culture
that has forgotten the concept of res ipsa loquitur deserves the fate
that it today decrees for VMI.

The concurrence states that it ``read{[}s{]} the Court'' not ``as saying
that the diversity rationale is a pretext'' for discriminating against
women, but as saying merely that the diversity rationale is not genuine.
The Court itself makes no such disclaimer, which would be difficult to
credit inasmuch as the foundation for its conclusion that the diversity
rationale is not ``genuine,'' is its antecedent discussion of Virginia's
``deliberate'' actions over the past century and a half, based on
``familiar arguments,'' that sought to enforce once ``widely held views
about women's proper place,''

Second, the concurrence dismisses out of hand what it calls Virginia's
``second justification for the single-sex admissions policy: maintenance
of the adversative method.'' The concurrence reasons that ``this
justification does not serve an important governmental objective''
because, whatever the record may show about the pedagogical benefits of
single-sex education, ``there is no similar evidence in the record that
an adversative method is pedagogically beneficial or is any more likely
to produce character traits than other methodologies.'' That is simply
wrong. See, e. g., 766 F. Supp. (factual findings concerning character
traits produced by VMI's adversative methodology); (factual findings
concerning benefits for many college-age men of an adversative approach
in general). In reality, the pedagogical benefits of VMI's adversative
approach were not only proved, but were a given in this litigation. The
reason the woman applicant who prompted this suit wanted to enter VMI
was assuredly not that she wanted to go to an all-male school; it would
cease being all-male as soon as she entered. She wanted the distinctive
adversative education that VMI provided, and the battle was joined (in
the main) over whether VMI had a basis for excluding women from that
approach. The Court's opinion recognizes this, and devotes much of its
opinion to demonstrating that ``\,`some women . do well under {[}the{]}
adversative model'\,'' and that ``it is on behalf of these women that
the United States has instituted this suit.'' (quoting 766 F. Supp.). Of
course, in the last analysis it does not matter whether there are any
benefits to the adversative method. The concurrence does not contest
that there are benefits to single-sex education, and that alone suffices
to make Virginia's case, since admission of a woman will even more
surely put an end to VMI's single-sex education than it will to VMI's
adversative methodology.

A third reason the concurrence offers in support of the judgment is that
the Commonwealth and VMI were not quick enough to react to the ``further
developments'' in this Court's evolving jurisprudence. Specifically, the
concurrence believes it should have been clear after Hogan that ``the
difficulty with {[}Virginia's{]} position is that the diversity
benefited only one sex; there was single-sex public education available
for men at VMI, but no corresponding single-sex public education
available for women.'' If only, the concurrence asserts, Virginia had
``made a genuine effort to devote comparable public resources to a
facility for women, and followed through on such a plan, it might well
have avoided an equal protection violation.'' That is to say, the
concurrence believes that after our decision in Hogan (which held a
program of the Mississippi University for Women to be unconstitutional
-- without any reliance on the fact that there was no corresponding
Mississippi all-men's program), the Commonwealth should have known that
what this Court expected of it was . yes!, the creation of a state
all-women's program. Any lawyer who gave that advice to the Commonwealth
ought to have been either disbarred or committed. (The proof of that
pudding is today's 6-Justice majority opinion.) And any Virginia
politician who proposed such a step when there were already four 4-year
women's colleges in Virginia (assisted by state support that may well
exceed, in the aggregate, what VMI costs, see n.~3) ought to have been
recalled.

In any event, ``diversity in the form of single-sex, as well as
coeducational, institutions of higher learning'' is ``available to women
as well as to men'' in Virginia. The concurrence is able to assert the
contrary only by disregarding the four all-women's private colleges in
Virginia (generously assisted by public funds) and the Commonwealth's
longstanding policy of coordinating public with private educational
offerings, see, 581-582, and n.~2, 583-584, and n.~3. According to the
concurrence, the reason Virginia's assistance to its four all-women's
private colleges does not count is that ``the private women's colleges
are treated by the State exactly as all other private schools are
treated.'' But if Virginia cannot get credit for assisting women's
education if it only treats women's private schools as it does all other
private schools, then why should it get blame for assisting men's
education if it only treats VMI as it does all other public schools?
This is a great puzzlement.

As is frequently true, the Court's decision today will have consequences
that extend far beyond the parties to the litigation. What I take to be
the Court's unease with these consequences, and its resulting
unwillingness to acknowledge them, cannot alter the reality.

Under the constitutional principles announced and applied today,
single-sex public education is unconstitutional. By going through the
motions of applying a balancing test -- asking whether the State has
adduced an ``exceedingly persuasive justification'' for its sex-based
classification -- the Court creates the illusion that government
officials in some future case will have a clear shot at justifying some
sort of single-sex public education. Indeed, the Court seeks to create
even a greater illusion than that: It purports to have said nothing of
relevance to other public schools at all. ``We address specifically and
only an educational opportunity recognized . as `unique.'\,''

The Supreme Court of the United States does not sit to announce
``unique'' dispositions. Its principal function is to establish
precedent -- that is, to set forth principles of law that every court in
America must follow. As we said only this Term, we expect both ourselves
and lower courts to adhere to the ``rationale upon which the Court based
the results of its earlier decisions.'' Seminole Tribe of Fla. v.
Florida, (1996) (emphasis added). That is the principal reason we
publish our opinions.

And the rationale of today's decision is sweeping: for sex-based
classifications, a redefinition of intermediate scrutiny that makes it
indistinguishable from strict scrutiny. See. Indeed, the Court indicates
that if any program restricted to one sex is ``unique,'' it must be
opened to members of the opposite sex ``who have the will and capacity''
to participate in it. I suggest that the single-sex program that will
not be capable of being characterized as ``unique'' is not only unique
but nonexistent.

In this regard, I note that the Court -- which I concede is under no
obligation to do so -- provides no example of a program that would pass
muster under its reasoning today: not even, for example, a football or
wrestling program. On the Court's theory, any woman ready, willing, and
physically able to participate in such a program would, as a
constitutional matter, be entitled to do so.

In any event, regardless of whether the Court's rationale leaves some
small amount of room for lawyers to argue, it ensures that single-sex
public education is functionally dead. The costs of litigating the
constitutionality of a single-sex education program, and the risks of
ultimately losing that litigation, are simply too high to be embraced by
public officials. Any person with standing to challenge any sex-based
classification can haul the State into federal court and compel it to
establish by evidence (presumably in the form of expert testimony) that
there is an ``exceedingly persuasive justification'' for the
classification. Should the courts happen to interpret that vacuous
phrase as establishing a standard that is not utterly impossible of
achievement, there is considerable risk that whether the standard has
been met will not be determined on the basis of the record evidence --
indeed, that will necessarily be the approach of any court that seeks to
walk the path the Court has trod today. No state official in his right
mind will buy such a high-cost, high-risk lawsuit by commencing a
single-sex program. The enemies of single-sex education have won; by
persuading only seven Justices (five would have been enough) that their
view of the world is enshrined in the Constitution, they have
effectively imposed that view on all 50 States.

This is especially regrettable because, as the District Court here
determined, educational experts in recent years have increasingly come
to ``support {[}the{]} view that substantial educational benefits flow
from a single-gender environment, be it male or female, that cannot be
replicated in a coeducational setting.'' 766 F. Supp. (emphasis added).
``The evidence in this case,'' for example, ``is virtually
uncontradicted'' to that effect. Until quite recently, some public
officials have attempted to institute new single-sex programs, at least
as experiments. In 1991, for example, the Detroit Board of Education
announced a program to establish three boys-only schools for inner-city
youth; it was met with a lawsuit, a preliminary injunction was swiftly
entered by a District Court that purported to rely on Hogan, and the
Detroit Board of Education voted to abandon the litigation and thus
abandon the plan, see Detroit Plan to Aid Blacks with All-Boy Schools
Abandoned, Los Angeles Times, Nov.~8, 1991, p.~A4, col.~1. Today's
opinion assures that no such experiment will be tried again.

There are few extant single-sex public educational programs. The
potential of today's decision for widespread disruption of existing
institutions lies in its application to private single-sex education.
Government support is immensely important to private educational
institutions. Mary Baldwin College -- which designed and runs VWIL --
notes that private institutions of higher education in the 1990-1991
school year derived approximately 19 percent of their budgets from
federal, state, and local government funds, not including financial aid
to students. See Brief for Mary Baldwin College as Amicus Curiae 22,
n.~13 (citing U.S. Dept. of Education, National Center for Education
Statistics, Digest of Education Statistics, p.~38 and Note (1993)).
Charitable status under the tax laws is also highly significant for
private educational institutions, and it is certainly not beyond the
Court that rendered today's decision to hold that a donation to a
single-sex college should be deemed contrary to public policy and
therefore not deductible if the college discriminates on the basis of
sex.

The Court adverts to private single-sex education only briefly, and only
to make the assertion (mentioned above) that ``we address specifically
and only an educational opportunity recognized by the District Court and
the Court of Appeals as `unique.'\,'' n.~7. As I have already remarked,
see, that assurance assures nothing, unless it is to be taken as a
promise that in the future the Court will disclaim the reasoning it has
used today to destroy VMI. The Government, in its briefs to this Court,
at least purports to address the consequences of its attack on VMI for
public support of private single-sex education. It contends that private
colleges that are the direct or indirect beneficiaries of government
funding are not thereby necessarily converted into state actors to which
the Equal Protection Clause is then applicable. That is true. It is also
virtually meaningless.

The issue will be not whether government assistance turns private
colleges into state actors, but whether the government itself would be
violating the Constitution by providing state support to single-sex
colleges. For example, in Norwood v. Harrison, we saw no room to
distinguish between state operation of racially segregated schools and
state support of privately run segregated schools. ``Racial
discrimination in state-operated schools is barred by the Constitution
and `it is also axiomatic that a state may not induce, encourage or
promote private persons to accomplish what it is constitutionally
forbidden to accomplish.'\,'' ; see also Cooper v. Aaron, (1958)
(``State support of segregated schools through any arrangement,
management, funds, or property cannot be squared with the
{[}Fourteenth{]} Amendment's command that no State shall deny to any
person within its jurisdiction the equal protection of the laws'');
Grove City College v. Bell, (1984) (case arising under Title IX of the
Education Amendments of 1972 and stating that ``the economic effect of
direct and indirect assistance often is indistinguishable''). When the
Government was pressed at oral argument concerning the implications of
these cases for private single-sex education if government-provided
single-sex education is unconstitutional, it stated that the
implications will not be so disastrous, since States can provide funding
to racially segregated private schools, ``depending on the
circumstances,'' Tr. of Oral Arg. 56. I cannot imagine what those
``circumstances'' might be, and it would be as foolish for
private-school administrators to think that that assurance from the
Justice Department will outlive the day it was made, as it was for VMI
to think that the Justice Department's ``unequivocal'' support for an
intermediate-scrutiny standard in this litigation would survive the
Government's loss in the courts below.

The only hope for state-assisted single-sex private schools is that the
Court will not apply in the future the principles of law it has applied
today. That is a substantial hope, I am happy and ashamed to say. After
all, did not the Court today abandon the principles of law it has
applied in our earlier sex-classification cases? And does not the Court
positively invite private colleges to rely upon our ad-hocery by
assuring them this litigation is ``unique''? I would not advise the
foundation of any new single-sex college (especially an all-male one)
with the expectation of being allowed to receive any government support;
but it is too soon to abandon in despair those single-sex colleges
already in existence. It will certainly be possible for this Court to
write a future opinion that ignores the broad principles of law set
forth today, and that characterizes as utterly dispositive the opinion's
perceptions that VMI was a uniquely prestigious all-male institution,
conceived in chauvinism, etc., etc. I will not join that opinion.

Justice Brandeis said it is ``one of the happy incidents of the federal
system that a single courageous State may, if its citizens choose, serve
as a laboratory; and try novel social and economic experiments without
risk to the rest of the country.'' New State Ice Co.~v. Liebmann, (1932)
(dissenting opinion). But it is one of the unhappy incidents of the
federal system that a self-righteous Supreme Court, acting on its
Members' personal view of what would make a ``\,`more perfect
Union,'\,'' (a criterion only slightly more restrictive than a ``more
perfect world''), can impose its own favored social and economic
dispositions nationwide. As today's disposition, and others this single
Term, show, this places it beyond the power of a ``single courageous
State,'' not only to introduce novel dispositions that the Court frowns
upon, but to reintroduce, or indeed even adhere to, disfavored
dispositions that are centuries old. The sphere of self-government
reserved to the people of the Republic is progressively narrowed.

In the course of this dissent, I have referred approvingly to the
opinion of my former colleague, Justice Powell, in Mississippi Univ. for
Women v. Hogan. Many of the points made in his dissent apply with equal
force here -- in particular, the criticism of judicial opinions that
purport to be ``narrow'' but whose ``logic'' is ``sweeping.'' But there
is one statement with which I cannot agree. Justice Powell observed that
the Court's decision in Hogan, which struck down a single-sex program
offered by the Mississippi University for Women, had thereby ``left
without honor . an element of diversity that has characterized much of
American education and enriched much of American life.'' Today's
decision does not leave VMI without honor; no court opinion can do that.

In an odd sort of way, it is precisely VMI's attachment to such
old-fashioned concepts as manly ``honor'' that has made it, and the
system it represents, the target of those who today succeed in
abolishing public single-sex education. The record contains a booklet
that all first-year VMI students (the so-called ``rats'') were required
to keep in their possession at all times. Near the end there appears the
following period piece, entitled ``The Code of a Gentleman'':

"Without a strict observance of the fundamental Code of Honor, no man,
no matter how `polished,' can be considered a gentleman. The honor of a
gentleman demands the inviolability of his word, and the
incorruptibility of his principles. He is the descendant of the knight,
the crusader; he is the defender of the defense-less and the champion of
justice . or he is not a Gentleman.

"A Gentleman .

"Does not discuss his family affairs in public or with acquaintances.

"Does not speak more than casually about his girl friend.

"Does not go to a lady's house if he is affected by alcohol. He is
temperate in the use of alcohol.

"Does not lose his temper; nor exhibit anger, fear, hate, embarrassment,
ardor or hilarity in public.

"Does not hail a lady from a club window.

"A gentleman never discusses the merits or demerits of a lady.

"Does not mention names exactly as he avoids the mention of what things
cost.

"Does not borrow money from a friend, except in dire need. Money
borrowed is a debt of honor, and must be repaid as promptly as possible.
Debts incurred by a deceased parent, brother, sister or grown child are
assumed by honorable men as a debt of honor.

" Does not display his wealth, money or possessions.

"Does not put his manners on and off, whether in the club or in a
ballroom. He treats people with courtesy, no matter what their social
position may be.

"Does not slap strangers on the back nor so much as lay a finger on a
lady.

"Does not `lick the boots of those above' nor `kick the face of those
below him on the social ladder.'

"Does not take advantage of another's helplessness or ignorance and
assumes that no gentleman will take advantage of him.

"A Gentleman respects the reserves of others, but demands that others
respect those which are his.

``A Gentleman can become what he wills to be. . .''I do not know whether
the men of VMI lived by this code; perhaps not. But it is powerfully
impressive that a public institution of higher education still in
existence sought to have them do so. I do not think any of us, women
included, will be better off for its destruction.

\hypertarget{gay-rights-at-the-intersection-of-substantive-due-process-and-equal-protection}{%
\section{Gay Rights at the Intersection of Substantive Due Process and
Equal
Protection}\label{gay-rights-at-the-intersection-of-substantive-due-process-and-equal-protection}}

\hypertarget{bowers-v.-hardwick}{%
\subsubsection{Bowers v. Hardwick}\label{bowers-v.-hardwick}}

478 U.S. 186 (1986)

\textbf{JUSTICE WHITE delivered the opinion of the Court.}

This case does not require a judgment on whether laws against sodomy
between consenting adults in general, or between homosexuals in
particular, are wise or desirable. It raises no question about the right
or propriety of state legislative decisions to repeal their laws that
criminalize homosexual sodomy, or of state-court decisions invalidating
those laws on state constitutional grounds. The issue presented is
whether the Federal Constitution confers a fundamental right upon
homosexuals to engage in sodomy and hence invalidates the laws of the
many States that still make such conduct illegal and have done so for a
very long time. The case also calls for some judgment about the limits
of the Court's role in carrying out its constitutional mandate.

We first register our disagreement with the Court of Appeals and with
respondent that the Court's prior cases have construed the Constitution
to confer a right of privacy that extends to homosexual sodomy and for
all intents and purposes have decided this case. The reach of this line
of cases was sketched in Carey v. Population Services International.
Pierce v. Society of Sisters, and Meyer v. Nebraska, were described as
dealing with child rearing and education; Prince v. Massachusetts, with
family relationships; Skinner v. Oklahoma ex rel. Williamson, with
procreation; Loving v. Virginia, 388 U. S. 1 (1967), with marriage;
Griswold v. Connecticutand Eisenstadt v. Bairdwith contraception; and
Roe v. Wade, with abortion. The latter three cases were interpreted as
construing the Due Process Clause of the Fourteenth Amendment to confer
a fundamental individual right to decide whether or not to beget or bear
a child. Carey v. Population Services International.

Accepting the decisions in these cases and the above description of
them, we think it evident that none of the rights announced in those
cases bears any resemblance to the claimed constitutional right of
homosexuals to engage in acts of sodomy that is asserted in this case.
No connection between family, marriage, or procreation on the one hand
and homosexual activity on the other has been demonstrated, either by
the Court of Appeals or by respondent. Moreover, any claim that these
cases nevertheless stand for the proposition that any kind of private
sexual conduct between consenting adults is constitutionally insulated
from state proscription is unsupportable. Indeed, the Court's opinion in
Carey twice asserted that the privacy right, which the Griswold line of
cases found to be one of the protections provided by the Due Process
Clause, did not reach so far. 431 U. S., n.~5, 694, n.~17.

Precedent aside, however, respondent would have us announce, as the
Court of Appeals did, a fundamental right to engage in homosexual
sodomy. This we are quite unwilling to do. It is true that despite the
language of the Due Process Clauses of the Fifth and Fourteenth
Amendments, which appears to focus only on the processes by which life,
liberty, or property is taken, the cases are legion in which those
Clauses have been interpreted to have substantive content, subsuming
rights that to a great extent are immune from federal or state
regulation or proscription. Among such cases are those recognizing
rights that have little or no textual support in the constitutional
language. Meyer, Prince, and Pierce fall in this category, as do the
privacy cases from Griswold to Carey.

Striving to assure itself and the public that announcing rights not
readily identifiable in the Constitution's text involves much more than
the imposition of the Justices' own choice of values on the States and
the Federal Government, the Court has sought to identify the nature of
the rights qualifying for heightened judicial protection. In Palko v.
Connecticut, 326 (1937), it was said that this category includes those
fundamental liberties that are ``implicit in the concept of ordered
liberty,'' such that ``neither liberty nor justice would exist if
{[}they{]} were sacrificed.'' A different description of fundamental
liberties appeared in Moore v. East Cleveland (opinion of POWELL, J.),
where they are characterized as those liberties that are ``deeply rooted
in this Nation's history and tradition.'' (POWELL, J.). See also
Griswold v. Connecticut.

It is obvious to us that neither of these formulations would extend a
fundamental right to homosexuals to engage in acts of consensual sodomy.
Proscriptions against that conduct have ancient roots. See generally
Survey on the Constitutional Right to Privacy in the Context of
Homosexual Activity, 40 U. Miami L. Rev.~521, 525 (1986). Sodomy was a
criminal offense at common law and was forbidden by the laws of the
original 13 States when they ratified the Bill of Rights In 1868, when
the Fourteenth Amendment was ratified, all but 5 of the 37 States in the
Union had criminal sodomy laws In fact, until 1961,7 all 50 States
outlawed sodomy, and today, 24 States and the District of Columbia
continue to provide criminal penalties for sodomy performed in private
and between consenting adults. See Survey, U. Miami L. Rev., n.~9.
Against this background, to claim that a right to engage in such conduct
is ``deeply rooted in this Nation's history and tradition'' or
``implicit in the concept of ordered liberty'' is, at best, facetious.

Nor are we inclined to take a more expansive view of our authority to
discover new fundamental rights imbedded in the Due Process Clause. The
Court is most vulnerable and comes nearest to illegitimacy when it deals
with judge-made constitutional law having little or no cognizable roots
in the language or design of the Constitution. That this is so was
painfully demonstrated by the face-off between the Executive and the
Court in the 1930's, which resulted in the repudiation of much of the
substantive gloss that the Court had placed on the Due Process Clauses
of the Fifth and Fourteenth Amendments. There should be, therefore,
great resistance to expand the substantive reach of those Clauses,
particularly if it requires redefining the category of rights deemed to
be fundamental. Otherwise, the Judiciary necessarily takes to itself
further authority to govern the country without express constitutional
authority. The claimed right pressed on us today falls for short of
overcoming this resistance.

Even if the conduct at issue here is not a fundamental right, respondent
asserts that there must be a rational basis for the law and that there
is none in this case other than the presumed belief of a majority of the
electorate in Georgia that homosexual sodomy is immoral and
unacceptable. This is said to be an inadequate rationale to support the
law. The law, however, is constantly based on notions of morality, and
if all laws representing essentially moral choices are to be invalidated
under the Due Process Clause, the courts will be very busy indeed. Even
respondent makes no such claim, but insists that majority sentiments
about the morality of homosexuality should be declared inadequate. We do
not agree, and are unpersuaded that the sodomy laws of some 25 States
should be invalidated on this basis.

\textbf{JUSTICE BLACKMUN, with whom JUSTICE BRENNAN, JUSTICE MARSHALL,
and JUSTICE STEVENS join, dissenting.}

This case is no more about ``a fundamental right to engage in homosexual
sodomy,'' as the Court purports to declare, than Stanley v. Georgia, was
about a fundamental right to watch obscene movies, or Katz v. United
States, was about a fundamental right to place interstate bets from a
telephone booth. Rather, this case is about ``the most comprehensive of
rights and the right most valued by civilized men,'' namely, ``the right
to be let alone.'' Olmstead v. United States (Brandeis, J., dissenting).

The statute at issue, Ga. Code Ann. § 16-6-2 (1984), denies individuals
the right to decide for themselves whether to engage in particular forms
of private, consensual sexual activity. The Court concludes that §
16-6-2 is valid essentially because ``the laws of . many States . still
make such conduct illegal and have done so for a very long time.'' But
the fact that the moral judgments expressed by statutes like § 16-6-2
may be ''`natural and familiar . ought not to conclude our judgment upon
the question whether statutes embodying them conflict with the
Constitution of the United States.' '' Roe v. Wade, quoting Lochner v.
New York (Holmes, J., dissenting). Like Justice Holmes, I believe that
''{[}i{]}t is revolting to have no better reason for a rule of law than
that so it was laid down in the time of Henry IV. It is still more
revolting if the grounds upon which it was laid down have vanished long
since, and the rule simply persists from blind imitation of the past.''
Holmes, The Path of the Law, 10 Harv. L. Rev.~457, 469 (1897). I believe
we must analyze respondent Hardwick's claim in the light of the values
that underlie the constitutional right to privacy. If that right means
anything, it means that, before Georgia can prosecute its citizens for
making choices about the most intimate aspects of their lives, it must
do more than assert that the choice they have made is an ''`abominable
crime not fit to be named among Christians.' '' Herring v. State, 119
Ga. 709, 721, 46 S. E. 876, 882 (1904).

In its haste to reverse the Court of Appeals and hold that the
Constitution does not ``confe® a fundamental right upon homosexuals to
engage in sodomy,'' the Court relegates the actual statute being
challenged to a footnote and ignores the procedural posture of the case
before it. A fair reading of the statute and of the complaint clearly
reveals that the majority has distorted the question this case presents.

First, the Court's almost obsessive focus on homosexual activity is
particularly hard to justify in light of the broad language Georgia has
used. Unlike the Court, the Georgia Legislature has not proceeded on the
assumption that homosexuals are so different from other citizens that
their lives may be controlled in a way that would not be tolerated if it
limited the choices of those other citizens. Cf. n.~2. Rather, Georgia
has provided that ''{[}a{]} person commits the offense of sodomy when he
performs or submits to any sexual act involving the sex organs of one
person and the mouth or anus of another.'' Ga. Code Ann. § 16-6-2(a)
(1984). The sex or status of the persons who engage in the act is
irrelevant as a matter of state law. In fact, to the extent I can
discern a legislative purpose for Georgia's 1968 enactment of § 16-6-2,
that purpose seems to have been to broaden the coverage of the law to
reach heterosexual as well as homosexual activity I therefore see no
basis for the Court's decision to treat this case as an ``as applied''
challenge to § 16-6-2, see n.~2, or for Georgia's attempt, both in its
brief and at oral argument, to defend § 16-6-2 solely on the grounds
that it prohibits homosexual activity. Michael Hardwick's standing may
rest in significant part on Georgia's apparent willingness to enforce
against homosexuals a law it seems not to have any desire to enforce
against heterosexuals. See Tr. of Oral Arg. 4-5; cf.~-1206 (CA11 1985).
But his claim that § 16-6-2 involves an unconstitutional intrusion into
his privacy and his right of intimate association does not depend in any
way on his sexual orientation.

Second, I disagree with the Court's refusal to consider whether § 16-6-2
runs afoul of the Eighth or Ninth Amendments or the Equal Protection
Clause of the Fourteenth Amendment. n.~8. Respondent's complaint
expressly invoked the Ninth Amendment, see App. 6, and he relied heavily
before this Court on Griswold v. Connecticut, which identifies that
Amendment as one of the specific constitutional provisions giving ``life
and substance'' to our understanding of privacy. See Brief for
Respondent Hardwick 10-12; Tr. of Oral Arg. 33. More importantly, the
procedural posture of the case requires that we affirm the Court of
Appeals' judgment if there is any ground on which respondent may be
entitled to relief. This case is before us on petitioner's motion to
dismiss for failure to state a claim, Fed. Rule Civ. Proc. 12(b)(6). See
App. 17. It is a well-settled principle of law that ``a complaint should
not be dismissed merely because a plaintiff's allegations do not support
the particular legal theory he advances, for the court is under a duty
to examine the complaint to determine if the allegations provide for
relief on any possible theory.'' Bramlet v. Wilson, CA8 1974); see Parr
v. Great Lakes Express Co., CA7 1973); Due v. Tallahassee Theaters,
Inc., CA5 1964); United States v. Howell, CA9 1963); 5 C. Wright \& A.
Miller, Federal Practice and Procedure § 1357, pp.~601-602 (1969); see
also Conley v. Gibson (1957). Thus, even if respondent did not advance
claims based on the Eighth or Ninth Amendments, or on the Equal
Protection Clause, his complaint should not be dismissed if any of those
provisions could entitle him to relief. I need not reach either the
Eighth Amendment or the Equal Protection Clause issues because I believe
that Hardwick has stated a cognizable claim that § 16-6-2 interferes
with constitutionally protected interests in privacy and freedom of
intimate association. But neither the Eighth Amendment nor the Equal
Protection Clause is so clearly irrelevant that a claim resting on
either provision should be peremptorily dismissed The Court's cramped
reading of the issue before it makes for a short opinion, but it does
little to make for a persuasive one.

``Our cases long have recognized that the Constitution embodies a
promise that a certain private sphere of individual liberty will be kept
largely beyond the reach of government.'' Thornburgh v. American College
of Obstetricians \& Gynecologists. In construing the right to privacy,
the Court has proceeded along two somewhat distinct, albeit
complementary, lines. First, it has recognized a privacy interest with
reference to certain decisions that are properly for the individual to
make. E. g., Roe v. Wade; Pierce v. Society of Sisters. Second, it has
recognized a privacy interest with reference to certain places without
regard for the particular activities in which the individuals who occupy
them are engaged. E. g., United States v. Karo; Payton v. New York; Rios
v. United States. The case before us implicates both the decisional and
the spatial aspects of the right to privacy.

The Court concludes today that none of our prior cases dealing with
various decisions that individuals are entitled to make free of
governmental interference ``bears any resemblance to the claimed
constitutional right of homosexuals to engage in acts of sodomy that is
asserted in this case.'' While it is true that these cases may be
characterized by their connection to protection of the family, see
Roberts v. United States Jaycees, the Court's conclusion that they
extend no further than this boundary ignores the warning in Moore v.
East Cleveland (plurality opinion), against ``clos{[}ing{]} our eyes to
the basic reasons why certain rights associated with the family have
been accorded shelter under the Fourteenth Amendment's Due Process
Clause.'' We protect those rights not because they contribute, in some
direct and material way, to the general public welfare, but because they
form so central a part of an individual's life. ''{[}T{]}he concept of
privacy embodies the'moral fact that a person belongs to himself and not
others nor to society as a whole.' '' Thornburgh v. American College of
Obstetricians \& Gynecologists, n.~5 (STEVENS, J., concurring), quoting
Fried, Correspondence, 6 Phil. \& Pub. Affairs 288-289 (1977). And so we
protect the decision whether to marry precisely because marriage ``is an
association that promotes a way of life, not causes; a harmony in
living, not political faiths; a bilateral loyalty, not commercial or
social projects.'' Griswold v. Connecticut. We protect the decision
whether to have a child because parenthood alters so dramatically an
individual's self-definition, not because of demographic considerations
or the Bible's command to be fruitful and multiply. Cf. Thornburgh v.
American College of Obstetricians \& Gynecologists, n.~6 (STEVENS, J.,
concurring). And we protect the family because it contributes so
powerfully to the happiness of individuals, not because of a preference
for stereotypical households. Cf. Moore v. East Cleveland (plurality
opinion). The Court recognized in Roberts, that the ``ability
independently to define one's identity that is central to any concept of
liberty'' cannot truly be exercised in a vacuum; we all depend on the
``emotional enrichment from close ties with others.''

Only the most willful blindness could obscure the fact that sexual
intimacy is ``a sensitive, key relationship of human existence, central
to family life, community welfare, and the development of human
personality,'' Paris Adult Theatre I v. Slaton; see also Carey v.
Population Services International. The fact that individuals define
themselves in a significant way through their intimate sexual
relationships with others suggests, in a Nation as diverse as ours, that
there may be many ``right'' ways of conducting those relationships, and
that much of the richness of a relationship will come from the freedom
an individual has to choose the form and nature of these intensely
personal bonds. See Karst, The Freedom of Intimate Association, 89 Yale
L. J. 624, 637 (1980); cf.~Eisenstadt v. Baird; Roe v. Wade.

In a variety of circumstances we have recognized that a necessary
corollary of giving individuals freedom to choose how to conduct their
lives is acceptance of the fact that different individuals will make
different choices. For example, in holding that the clearly important
state interest in public education should give way to a competing claim
by the Amish to the effect that extended formal schooling threatened
their way of life, the Court declared: ``There can be no assumption that
today's majority is'right' and the Amish and others like them
are'wrong.' A way of life that is odd or even erratic but interferes
with no rights or interests of others is not to be condemned because it
is different.'' Wisconsin v. Yoder (1972). The Court claims that its
decision today merely refuses to recognize a fundamental right to engage
in homosexual sodomy; what the Court really has refused to recognize is
the fundamental interest all individuals have in controlling the nature
of their intimate associations with others.

The behavior for which Hardwick faces prosecution occurred in his own
home, a place to which the Fourth Amendment attaches special
significance. The Court's treatment of this aspect of the case is
symptomatic of its overall refusal to consider the broad principles that
have informed our treatment of privacy in specific cases. Just as the
right to privacy is more than the mere aggregation of a number of
entitlements to engage in specific behavior, so too, protecting the
physical integrity of the home is more than merely a means of protecting
specific activities that often take place there. Even when our
understanding of the contours of the right to privacy depends on
``reference to a'place,''' Katz v. United States (Harlan, J.,
concurring), ``the essence of a Fourth Amendment violation is'not the
breaking of {[}a person's{]} doors, and the rummaging of his drawers,'
but rather is'the invasion of his indefensible right of personal
security, personal liberty and private property.''' California v.
Ciraolo (POWELL, J., dissenting).

The Court's interpretation of the pivotal case of Stanley v. Georgia, is
entirely unconvincing. Stanley held that Georgia's undoubted power to
punish the public distribution of constitutionally unprotected, obscene
material did not permit the State to punish the private possession of
such material. According to the majority here, Stanley relied entirely
on the First Amendment, and thus, it is claimed, sheds no light on cases
not involving printed materials. But that is not what Stanley said.
Rather, the Stanley Court anchored its holding in the Fourth Amendment's
special protection for the individual in his home:

''`The makers of our Constitution undertook to secure conditions
favorable to the pursuit of happiness. They recognized the significance
of man's spiritual nature, of his feelings and of his intellect. They
knew that only a part of the pain, pleasure and satisfactions of life
are to be found in material things. They sought to protect Americans in
their beliefs, their thoughts, their emotions and their sensations.'
\ldots.. ``These are the rights that appellant is asserting in the case
before us. He is asserting the right to read or observe what he pleases
--- the right to satisfy his intellectual and emotional needs in the
privacy of his own home.''

The central place that Stanley gives Justice Brandeis' dissent in
Olmstead, a case raising no First Amendment claim, shows that Stanley
rested as much on the Court's understanding of the Fourth Amendment as
it did on the First. Indeed, in Paris Adult Theatre I v. Slaton, the
Court suggested that reliance on the Fourth Amendment not only supported
the Court's outcome in Stanley but actually was necessary to it: ``If
obscene material unprotected by the First Amendment in itself carried
with it a'penumbra' of constitutionally protected privacy, this Court
would not have found it necessary to decide Stanley on the narrow basis
of the'privacy of the home,' which was hardly more than a reaffirmation
that'a man's home is his castle.''' 413 U. S.. ``The right of the people
to be secure in their . houses,'' expressly guaranteed by the Fourth
Amendment, is perhaps the most ``textual'' of the various constitutional
provisions that inform our understanding of the right to privacy, and
thus I cannot agree with the Court's statement that ''{[}t{]}he right
pressed upon us here has no . support in the text of the Constitution,''
Indeed, the right of an individual to conduct intimate relationships in
the intimacy of his or her own home seems to me to be the heart of the
Constitution's protection of privacy.

The core of petitioner's defense of § 16-6-2, however, is that
respondent and others who engage in the conduct prohibited by § 16-6-2
interfere with Georgia's exercise of the ''`right of the Nation and of
the States to maintain a decent society,' '' Paris Adult Theater I v.
Slaton, quoting Jacobellis v. Ohio (Warren, C. J., dissenting).
Essentially, petitioner argues, and the Court agrees, that the fact that
the acts described in § 16-6-2 ``for hundreds of years, if not
thousands, have been uniformly condemned as immoral'' is a sufficient
reason to permit a State to ban them today. Brief for Petitioner 19.

I cannot agree that either the length of time a majority has held its
convictions or the passions with which it defends them can withdraw
legislation from this Court's security. See, e. g., Roe v. Wade; Loving
v. Virginia, 388 U. S. 1 (1967); Brown v. Board of Education As Justice
Jackson wrote so eloquently for the Court in West Virginia Board of
Education v. Barnette (1943), ``we apply the limitations of the
Constitution with no fear that freedom to be intellectually and
spiritually diverse or even contrary will disintegrate the social
organization. . {[}F{]}reedom to differ is not limited to things that do
not matter much. That would be a mere shadow of freedom. The test of its
substance is the right to differ as to things that touch the heart of
the existing order.'' See also Karst, 89 Yale L. J.. It is precisely
because the issue raised by this case touches the heart of what makes
individuals what they are that we should be especially sensitive to the
rights of those whose choices upset the majority.

The assertion that ``traditional Judeo-Christian values proscribe'' the
conduct involved, Brief for Petitioner 20, cannot provide an adequate
justification for § 16-6-2. That certain, but by no means all, religious
groups condemn the behavior at issue gives the State no license to
impose their judgments on the entire citizenry. The legitimacy of
secular legislation depends instead on whether the State can advance
some justification for its law beyond its conformity to religious
doctrine. See, e. g., McGowan v. Maryland (1961); Stone v. Graham. Thus,
far from buttressing his case, petitioner's invocation of Leviticus,
Romans, St.~Thomas Aquinas, and sodomy's heretical status during the
Middle Ages undermines his suggestion that § 16-6-2 represents a
legitimate use of secular coercive power A State can no more punish
private behavior because of religious intolerance than it can punish
such behavior because of racial animus. ``The Constitution cannot
control such prejudices, but neither can it tolerate them. Private
biases may be outside the reach of the law, but the law cannot, directly
or indirectly, give them effect.'' Palmore v. Sidoti. No matter how
uncomfortable a certain group may make the majority of this Court, we
have held that ''{[}m{]}ere public intolerance or animosity cannot
constitutionally justify the deprivation of a person's physical
liberty.'' O'Connor v. Donaldson. See also Cleburne v. Cleburne Living
Center, Inc.; United States Dept. of Agriculture v. Moreno.

Nor can § 16-6-2 be justified as a ``morally neutral'' exercise of
Georgia's power to ``protect the public environment,'' Paris Adult
Theatre I. Certainly, some private behavior can affect the fabric of
society as a whole. Reasonable people may differ about whether
particular sexual acts are moral or immoral, but ``we have ample
evidence for believing that people will not abandon morality, will not
think any better of murder, cruelty and dishonesty, merely because some
private sexual practice which they abominate is not punished by the
law.'' H. L. A. Hart, Immorality and Treason, reprinted in The Law as
Literature 220, 225 (L. Blom-Cooper ed.~1961). Petitioner and the Court
fail to see the difference between laws that protect public
sensibilities and those that enforce private morality. Statutes banning
public sexual activity are entirely consistent with protecting the
individual's liberty interest in decisions concerning sexual relations:
the same recognition that those decisions are intensely private which
justifies protecting them from governmental interference can justify
protecting individuals from unwilling exposure to the sexual activities
of others. But the mere fact that intimate behavior may be punished when
it takes place in public cannot dictate how States can regulate intimate
behavior that occurs in intimate places. See Paris Adult Theatre I,
n.~13 (``marital intercourse on a street corner or a theater stage'' can
be forbidden despite the constitutional protection identified in
Griswold v. Connecticut).

This case involves no real interference with the rights of others, for
the mere knowledge that other individuals do not adhere to one's value
system cannot be a legally cognizable interest, cf.~Diamond v. Charles
(1986), let alone an interest that can justify invading the houses,
hearts, and minds of citizens who choose to live their lives
differently.

With respect to the Equal Protection Clause's applicability to § 16-6-2,
I note that Georgia's exclusive stress before this Court on its interest
in prosecuting homosexual activity despite the gender-neutral terms of
the statute may raise serious questions of discriminatory enforcement,
questions that cannot be disposed of before this Court on a motion to
dismiss. See Yick Wo v. Hopkins (1886). The legislature having decided
that the sex of the participants is irrelevant to the legality of the
acts, I do not see why the State can defend § 16-6-2 on the ground that
individuals singled out for prosecution are of the same sex as their
partners. Thus, under the circumstances of this case, a claim under the
Equal Protection Clause may well be available without having to reach
the more controversial question whether homosexuals are a suspect class.

Although I do not think it necessary to decide today issues that are not
even remotely before us, it does seem to me that a court could find
simple, analytically sound distinctions between certain private,
consensual sexual conduct, on the one hand, and adultery and incest (the
only two vaguely specific ``sexual crimes'' to which the majority
points, ), on the other. For example, marriage, in addition to its
spiritual aspects, is a civil contract that entitles the contracting
parties to a variety of governmentally provided benefits. A State might
define the contractual commitment necessary to become eligible for these
benefits to include a commitment of fidelity and then punish individuals
for breaching that contract. Moreover, a State might conclude that
adultery is likely to injure third persons, in particular, spouses and
children of persons who engage in extramarital affairs. With respect to
incest, a court might well agree with respondent that the nature of
familial relationships renders true consent to incestuous activity
sufficiently problematical that a blanket prohibition of such activity
is warranted. See Tr. of Oral Arg. 21-22. Notably, the Court makes no
effort to explain why it has chosen to group private, consensual
homosexual activity with adultery and incest rather than with private,
consensual heterosexual activity by unmarried persons or, indeed, with
oral or anal sex within marriage.

The parallel between Loving and this case is almost uncanny. There, too,
the State relied on a religious justification for its law. Compare 388
U. S. (quoting trial court's statement that ``Almighty God created the
races white, black, yellow, malay and red, and he placed them on
separate continents. . The fact that he separated the races shows that
he did not intend for the races to mix''), with Brief for Petitioner
20-21 (relying on the Old and New Testaments and the writings of
St.~Thomas Aquinas to show that ``traditional Judeo-Christian values
proscribe such conduct''). There, too, defenders of the challenged
statute relied heavily on the fact that when the Fourteenth Amendment
was ratified, most of the States had similar prohibitions. Compare Brief
for Appellee in Loving v. Virginia, O. T. 1966, No.~395, pp.~28-29, with
and n.~6. There, too, at the time the case came before the Court, many
of the States still had criminal statutes concerning the conduct at
issue. Compare 388 U. S., n.~5 (noting that 16 States still outlawed
interracial marriage), with (noting that 24 States and the District of
Columbia have sodomy statutes). Yet the Court held, not only that the
invidious racism of Virginia's law violated the Equal Protection Clause,
see 388 U. S., but also that the law deprived the Lovings of due process
by denying them the ``freedom of choice to marry'' that had ``long been
recognized as one of the vital personal rights essential to the orderly
pursuit of happiness by free men.''

The theological nature of the origin of Anglo-American antisodomy
statutes is patent. It was not until 1533 that sodomy was made a secular
offense in England. 25 Hen. VIII, ch.~6. Until that time, the offense
was, in Sir James Stephen's words, ``merely ecclesiastical.'' 2J.
Stephen, A History of the Criminal Law of England 429-430 (1883).
Pollock and Maitland similarly observed that ''{[}t{]}he crime against
nature . was so closely connected with heresy that the vulgar had but
one name for both.'' 2 F. Pollock \& F. Maitland, The History of English
Law 554 (1895). The transfer of jurisdiction over prosecutions for
sodomy to the secular courts seems primarily due to the alteration of
ecclesiastical jurisdiction attendant on England's break with the Roman
Catholic Church, rather than to any new understanding of the sovereign's
interest in preventing or punishing the behavior involved. Cf. 6 E.
Coke, Institutes, ch.~10 (4th ed.~1797).

At oral argument a suggestion appeared that, while the Fourth
Amendment's special protection of the home might prevent the State from
enforcing § 16-6-2 against individuals who engage in consensual sexual
activity there, that protection would not make the statute invalid. See
Tr. of Oral Arg. 10-11. The suggestion misses the point entirely. If the
law is not invalid, then the police can invade the home to enforce it,
provided, of course, that they obtain a determination of probable cause
from a neutral magistrate. One of the reasons for the Court's holding in
Griswold v. Connecticut, was precisely the possibility, and repugnance,
of permitting searches to obtain evidence regarding the use of
contraceptives. -486. Permitting the kinds of searches that might be
necessary to obtain evidence of the sexual activity banned by § 16-6-2
seems no less intrusive, or repugnant. Cf. Winston v. Lee; Mary Beth G.
v. City of Chicago, CA7 1983).

\textbf{JUSTICE STEVENS, with whom JUSTICE BRENNAN and JUSTICE MARSHALL
join, dissenting.} Like the statute that is challenged in this case, the
rationale of the Court's opinion applies equally to the prohibited
conduct regardless of whether the parties who engage in it are married
or unmarried, or are of the same or different sexes Sodomy was condemned
as an odious and sinful type of behavior during the formative period of
the common law That condemnation was equally damning for heterosexual
and homosexual sodomy Moreover, it provided no special exemption for
married couples The license to cohabit and to produce legitimate
offspring simply did not include any permission to engage in sexual
conduct that was considered a ``crime against nature.''

The history of the Georgia statute before us clearly reveals this
traditional prohibition of heterosexual, as well as homosexual, sodomy
Indeed, at one point in the 20th century, Georgia's law was construed to
permit certain sexual conduct between homosexual women even though such
conduct was prohibited between heterosexuals The history of the statutes
cited by the majority as proof for the proposition that sodomy is not
constitutionally protected, and nn.~5 and 6, similarly reveals a
prohibition on heterosexual, as well as homosexual, sodomy.

Because the Georgia statute expresses the traditional view that sodomy
is an immoral kind of conduct regardless of the identity of the persons
who engage in it, I believe that a proper analysis of its
constitutionality requires consideration of two questions: First, may a
State totally prohibit the described conduct by means of a neutral law
applying without exception to all persons subject to its jurisdiction?
If not, may the State save the statute by announcing that it will only
enforce the law against homosexuals? The two questions merit separate
discussion.

Our prior cases make two propositions abundantly clear. First, the fact
that the governing majority in a State has traditionally viewed a
particular practice as immoral is not a sufficient reason for upholding
a law prohibiting the practice; neither history nor tradition could save
a law prohibiting miscegenation from constitutional attack Second,
individual decisions by married persons, concerning the intimacies of
their physical relationship, even when not intended to produce
offspring, are a form of ``liberty'' protected by the Due Process Clause
of the Fourteenth Amendment. Griswold v. Connecticut. Moreover, this
protection extends to intimate choices by unmarried as well as married
persons. Carey v. Population Services International; Eisenstadt v.
Baird.

In consideration of claims of this kind, the Court has emphasized the
individual interest in privacy, but its decisions have actually been
animated by an even more fundamental concern. As I wrote some years ago:

``These cases do not deal with the individual's interest in protection
from unwarranted public attention, comment, or exploitation. They deal,
rather, with the individual's right to make certain unusually important
decisions that will affect his own, or his family's destiny. The Court
has referred to such decisions as implicating'basic values,' as
being'fundamental,' and as being dignified by history and tradition. The
character of the Court's language in these cases brings to mind the
origins of the American heritage of freedom --- the abiding interest in
individual liberty that makes certain state intrusions on the citizen's
right to decide how he will live his own life intolerable. Guided by
history, our tradition of respect for the dignity of individual choice
in matters of conscience and the restraints implicit in the federal
system, federal judges have accepted the responsibility for recognition
and protection of these rights in appropriate cases.'' Fitzgerald v.
Porter Memorial Hospital, -720 (CA7 1975) (footnotes omitted), cert.
denied.

Society has every right to encourage its individual members to follow
particular traditions in expressing affection for one another and in
gratifying their personal desires. It, of course, may prohibit an
individual from imposing his will on another to satisfy his own selfish
interests. It also may prevent an individual from interfering with, or
violating, a legally sanctioned and protected relationship, such as
marriage. And it may explain the relative advantages and disadvantages
of different forms of intimate expression. But when individual married
couples are isolated from observation by others, the way in which they
voluntarily choose to conduct their intimate relations is a matter for
them --- not the State --- to decide The essential ``liberty'' that
animated the development of the law in cases like Griswold, Eisenstadt,
and Carey surely embraces the right to engage in nonreproductive, sexual
conduct that others may consider offensive or immoral.

Paradoxical as it may seem, our prior cases thus establish that a State
may not prohibit sodomy within ``the sacred precincts of marital
bedrooms,'' Griswold, or, indeed, between unmarried heterosexual adults.
Eisenstadt. In all events, it is perfectly clear that the State of
Georgia may not totally prohibit the conduct proscribed by § 16-6-2 of
the Georgia Criminal Code.

If the Georgia statute cannot be enforced as it is written --- if the
conduct it seeks to prohibit is a protected form of liberty for the vast
majority of Georgia's citizens --- the State must assume the burden of
justifying a selective application of its law. Either the persons to
whom Georgia seeks to apply its statute do not have the same interest in
``liberty'' that others have, or there must be a reason why the State
may be permitted to apply a generally applicable law to certain persons
that it does not apply to others.

The first possibility is plainly unacceptable. Although the meaning of
the principle that ``all men are created equal'' is not always clear, it
surely must mean that every free citizen has the same interest in
``liberty'' that the members of the majority share. From the standpoint
of the individual, the homosexual and the heterosexual have the same
interest in deciding how he will live his own life, and, more narrowly,
how he will conduct himself in his personal and voluntary associations
with his companions. State intrusion into the private conduct of either
is equally burdensome.

The second possibility is similarly unacceptable. A policy of selective
application must be supported by a neutral and legitimate interest ---
something more substantial than a habitual dislike for, or ignorance
about, the disfavored group. Neither the State nor the Court has
identified any such interest in this case. The Court has posited as a
justification for the Georgia statute ``the presumed belief of a
majority of the electorate in Georgia that homosexual sodomy is immoral
and unacceptable.'' But the Georgia electorate has expressed no such
belief --- instead, its representatives enacted a law that presumably
reflects the belief that all sodomy is immoral and unacceptable. Unless
the Court is prepared to conclude that such a law is constitutional, it
may not rely on the work product of the Georgia Legislature to support
its holding. For the Georgia statute does not single out homosexuals as
a separate class meriting special disfavored treatment.

Nor, indeed, does not Georgia prosecutor even believe that all
homosexuals who violate this statute should be punished. This conclusion
is evident from the fact that the respondent in this very case has
formally acknowledged in his complaint and in court that he has engaged,
and intends to continue to engage, in the prohibited conduct, yet the
State has elected not to process criminal charges against him. As
JUSTICE POWELL points out, moreover, Georgia's prohibition on private,
consensual sodomy has not been enforced for decades The record of
nonenforcement, in this case and in the last several decades, belies the
Attorney General's representations about the importance of the State's
selective application of its generally applicable law.

Both the Georgia statute and the Georgia prosecutor thus completely fail
to provide the Court with any support for the conclusion that homosexual
sodomy, simpliciter, is considered unacceptable conduct in that State,
and that the burden of justifying a selective application of the
generally applicable law has been met.

The Court orders the dismissal of respondent's complaint even though the
State's statute prohibits all sodomy; even though that prohibition is
concededly unconstitutional with respect to heterosexuals; and even
though the State's post hoc explanations for selective application are
belied by the State's own actions. At the very least, I think it clear
at this early stage of the litigation that respondent has alleged a
constitutional claim sufficient to withstand a motion to dismiss.

I respectfully dissent.

\hypertarget{romer-v.-evans}{%
\subsubsection{Romer v. Evans}\label{romer-v.-evans}}

\textbf{Justice Kennedy delivered the opinion of the Court.} One century
ago, the first Justice Harlan admonished this Court that the
Constitution ``neither knows nor tolerates classes among citizens.''
Plessy v. Ferguson (dissenting opinion). Unheeded then, those words now
are understood to state a commitment to the law's neutrality where the
rights of persons are at stake. The Equal Protection Clause enforces
this principle and today requires us to hold invalid a provision of
Colorado's Constitution.

The enactment challenged in this case is an amendment to the
Constitution of the State of Colorado, adopted in a 1992 statewide
referendum. The parties and the state courts refer to it as ``Amendment
2,'' its designation when submitted to the voters. The impetus for the
amendment and the contentious campaign that preceded its adoption came
in large part from ordinances that had been passed in various Colorado
municipalities. For example, the cities of Aspen and Boulder and the
city and County of Denver each had enacted ordinances which banned
discrimination in many transactions and activities, including housing,
employment, education, public accommodations, and health and welfare
services. Denver Rev.~Municipal Code, Art. IV, §§ 28-91 to 28-116
(1991); Aspen Municipal Code § 13-98 (1977); Boulder Rev.~Code §§
12-1---1 to 12-1---11 (1987). What gave rise to the statewide
controversy was the protection the ordinances afforded to persons
discriminated against by reason of their sexual orientation. See Boulder
Rev.~Code § 12-1---1 (defining ``sexual orientation'' as ``the choice of
sexual partners, i. e., bisexual, homosexual or heterosexual''); Denver
Rev.~Municipal Code, Art. IV, § 28-92 (defining ``sexual orientation''
as ``{[}t{]}he status of an individual as to his or her heterosexuality,
homosexuality or bisexuality''). Amendment 2 repeals these ordinances to
the extent they prohibit discrimination on the basis of ``homosexual,
lesbian or bisexual orientation, conduct, practices or relationships.''
Colo. Const., Art. II, § 30b.

Yet Amendment 2, in explicit terms, does more than repeal or rescind
these provisions. It prohibits all legislative, executive or judicial
action at any level of state or local government designed to protect the
named class, a class we shall refer to as homosexual persons or gays and
lesbians. The amendment reads:

``No Protected Status Based on Homosexual, Lesbian or Bisexual
Orientation. Neither the State of Colorado, through any of its branches
or departments, nor any of its agencies, political subdivisions,
municipalities or school districts, shall enact, adopt or enforce any
statute, regulation, ordinance or policy whereby homosexual, lesbian or
bisexual orientation, conduct, practices or relationships shall
constitute or otherwise be the basis of or entitle any person or class
of persons to have or claim any minority status, quota preferences,
protected status or claim of discrimination. This Section of the
Constitution shall be in all respects self-executing.''

Soon after Amendment 2 was adopted, this litigation to declare its
invalidity and enjoin its enforcement was commenced in the District
Court for the City and County of Denver. Among the plaintiffs
(respondents here) were homosexual persons, some of them government
employees. They alleged that enforcement of Amendment 2 would subject
them to immediate and substantial risk of discrimination on the basis of
their sexual orientation. Other plaintiffs (also respondents here)
included the three municipalities whose ordinances we have cited and
certain other governmental entities which had acted earlier to protect
homosexuals from discrimination but would be prevented by Amendment 2
from continuing to do so. Although Governor Romer had been on record
opposing the adoption of Amendment 2, he was named in his official
capacity as a defendant, together with the Colorado Attorney General and
the State of Colorado.

The trial court granted a preliminary injunction to stay enforcement of
Amendment 2, and an appeal was taken to the Supreme Court of Colorado.
Sustaining the interim injunction and remanding the case for further
proceedings, the State Supreme Court held that Amendment 2 was subject
to strict scrutiny under the Fourteenth Amendment because it infringed
the fundamental right of gays and lesbians to participate in the
political process. Evans v. Romer, 854 P. 2d 1270 (Colo. 1993) (Evans
I). To reach this conclusion, the state court relied on our voting
rights cases, e. g., Reynolds v. Sims; Carrington v. Rash; Harper v.
Virginia Bd. of Elections; Williams v. Rhodes, and on our precedents
involving discriminatory restructuring of governmental decisionmaking,
see, e. g., Hunter v. Erickson; Reitman v. Mulkey; Washington v. Seattle
School Dist. No.~1; Gordon v. Lance,403 U. S. 1 (1971). On remand, the
State advanced various arguments in an effort to show that Amendment 2
was narrowly tailored to serve compelling interests, but the trial court
found none sufficient. It enjoined enforcement of Amendment 2, and the
Supreme Court of Colorado, in a second opinion, affirmed the ruling. 882
P. 2d 1335 (1994) (Evans II). We granted certiorari, and now affirm the
judgment, but on a rationale different from that adopted by the State
Supreme Court.

The State's principal argument in defense of Amendment 2 is that it puts
gays and lesbians in the same position as all other persons. So, the
State says, the measure does no more than deny homosexuals special
rights. This reading of the amendment's language is implausible. We rely
not upon our own interpretation of the amendment but upon the
authoritative construction of Colorado's Supreme Court. The state court,
deeming it unnecessary to determine the full extent of the amendment's
reach, found it invalid even on a modest reading of its implications.
The critical discussion of the amendment, set out in Evans I, is as
follows:

"The immediate objective of Amendment 2 is, at a minimum, to repeal
existing statutes, regulations, ordinances, and policies of state and
local entities that barred discrimination based on sexual orientation.
See Aspen, Colo., Mun. Code § 13-98 (1977) (prohibiting discrimination
in employment, housing and public accommodations on the basis of sexual
orientation); Boulder, Colo., Rev.~Code §§ 12-1---2 to ---4 (1987)
(same); Denver, Colo., Rev.~Mun. Code art. IV, §§ 28-91 to ---116 (1991)
(same); Executive Order No.~D0035 (December 10, 1990) (prohibiting
employment discrimination for'all state employees, classified and
exempt' on the basis of sexual orientation); Colorado Insurance Code, §
10-3A C. R. S. (1992 Supp.) (forbidding health insurance providers from
determining insurability and premiums based on an applicant's, a
beneficiary's, or an insured's sexual orientation); and various
provisions prohibiting discrimination based on sexual orientation at
state colleges.

Sweeping and comprehensive is the change in legal status effected by
this law. So much is evident from the ordinances the Colorado Supreme
Court declared would be void by operation of Amendment 2. Homosexuals,
by state decree, are put in a solitary class with respect to
transactions and relations in both the private and governmental spheres.
The amendment withdraws from homosexuals, but no others, specific legal
protection from the injuries caused by discrimination, and it forbids
reinstatement of these laws and policies.

The change Amendment 2 works in the legal status of gays and lesbians in
the private sphere is far reaching, both on its own terms and when
considered in light of the structure and operation of modern
antidiscrimination laws. That structure is well illustrated by
contemporary statutes and ordinances prohibiting discrimination by
providers of public accommodations. ``At common law, innkeepers, smiths,
and others who'made profession of a public employment,' were prohibited
from refusing, without good reason, to serve a customer.'' Hurley v.
Irish-American Gay, Lesbian and Bisexual Group of Boston, Inc.. The duty
was a general one and did not specify protection for particular groups.
The common-law rules, however, proved insufficient in many instances,
and it was settled early that the Fourteenth Amendment did not give
Congress a general power to prohibit discrimination in public
accommodations, Civil Rights Cases. In consequence, most States have
chosen to counter discrimination by enacting detailed statutory schemes.
See, e. g., S. D. Codified Laws §§ 20-13-23 (1995); Iowa Code §§ 216
-216 (1994); Okla. Stat., Tit. 25, §§ 1302, 1402 (1987); 43 Pa. Cons.
Stat. §§ 953, 955 (Supp. 1995); N. J. Stat. Ann. §§ 10:5:5-4 (West Supp.
1995); N. H. Rev.~Stat. Ann. §§ 354---A:7, 354---A:10, 354---A:17
(1995); Minn. Stat. § 363 (1991 and Supp. 1995).

Colorado's state and municipal laws typify this emerging tradition of
statutory protection and follow a consistent pattern. The laws first
enumerate the persons or entities subject to a duty not to discriminate.
The list goes well beyond the entities covered by the common law. The
Boulder ordinance, for example, has a comprehensive definition of
entities deemed places of ``public accommodation.'' They include ``any
place of business engaged in any sales to the general public and any
place that offers services, facilities, privileges, or advantages to the
general public or that receives financial support through solicitation
of the general public or through governmental subsidy of any kind.''
Boulder Rev.~Code § 12-1---1(j) (1987). The Denver ordinance is of
similar breadth, applying, for example, to hotels, restaurants,
hospitals, dental clinics, theaters, banks, common carriers, travel and
insurance agencies, and ``shops and stores dealing with goods or
services of any kind,'' Denver Rev.~Municipal Code, Art. IV, § 28-92
(1991).

These statutes and ordinances also depart from the common law by
enumerating the groups or persons within their ambit of protection.
Enumeration is the essential device used to make the duty not to
discriminate concrete and to provide guidance for those who must comply.
In following this approach, Colorado's state and local governments have
not limited antidiscrimination laws to groups that have so far been
given the protection of heightened equal protection scrutiny under our
cases. See, e. g., J. E. B. v. Alabama ex rel. T. B. (sex); Lalli v.
Lalli (illegitimacy); McLaughlin v. Florida (1964) (race); Oyama v.
California(ancestry). Rather, they set forth an extensive catalog of
traits which cannot be the basis for discrimination, including age,
military status, marital status, pregnancy, parenthood, custody of a
minor child, political affiliation, physical or mental disability of an
individual or of his or her associates----and, in recent times, sexual
orientation. Aspen Municipal Code § 13-98(a)(1) (1977); Boulder
Rev.~Code §§ 12-1---1 to 12-1---4 (1987); Denver Rev.~Municipal Code,
Art. IV, §§ 28-92 to 28-119 (1991); Colo. Rev.~Stat. §§ 24-34-401 to
24-34-707 (1988 and Supp. 1995).

Amendment 2 bars homosexuals from securing protection against the
injuries that these public-accommodations laws address. That in itself
is a severe consequence, but there is more. Amendment 2, in addition,
nullifies specific legal protections for this targeted class in all
transactions in housing, sale of real estate, insurance, health and
welfare services, private education, and employment. See, e. g., Aspen
Municipal Code §§ 13-98(b), (c) (1977); Boulder Rev.~Code §§ 12-1---3
(1987); Denver Rev.~Municipal Code, Art. IV, §§ 28-93 to 28 (1991).

Not confined to the private sphere, Amendment 2 also operates to repeal
and forbid all laws or policies providing specific protection for gays
or lesbians from discrimination by every level of Colorado government.
The State Supreme Court cited two examples of protections in the
governmental sphere that are now rescinded and may not be reintroduced.
The first is Colorado Executive Order D0035 (1990), which forbids
employment discrimination against ```all state employees, classified and
exempt' on the basis of sexual orientation.'' 854 P. 2d. Also repealed,
and now forbidden, are ``various provisions prohibiting discrimination
based on sexual orientation at state colleges.'' 1285. The repeal of
these measures and the prohibition against their future reenactment
demonstrate that Amendment 2 has the same force and effect in Colorado's
governmental sector as it does elsewhere and that it applies to policies
as well as ordinary legislation.

Amendment 2's reach may not be limited to specific laws passed for the
benefit of gays and lesbians. It is a fair, if not necessary, inference
from the broad language of the amendment that it deprives gays and
lesbians even of the protection of general laws and policies that
prohibit arbitrary discrimination in governmental and private settings.
See, e. g., Colo. Rev.~Stat. § 24-4---106(7) (1988) (agency action
subject to judicial review under arbitrary and capricious standard); §
18-8---405 (making it a criminal offense for a public servant knowingly,
arbitrarily, or capriciously to refrain from performing a duty imposed
on him by law); § 10-3---1104(1)(f) (prohibiting ``unfair
discrimination'' in insurance); 4 Colo. Code of Regulations 801-1,
Policy 11-1 (1983) (prohibiting discrimination in state employment on
grounds of specified traits or ``other non-merit factor''). At some
point in the systematic administration of these laws, an official must
determine whether homosexuality is an arbitrary and, thus, forbidden
basis for decision. Yet a decision to that effect would itself amount to
a policy prohibiting discrimination on the basis of homosexuality, and
so would appear to be no more valid under Amendment 2 than the specific
prohibitions against discrimination the state court held invalid.

If this consequence follows from Amendment 2, as its broad language
suggests, it would compound the constitutional difficulties the law
creates. The state court did not decide whether the amendment has this
effect, however, and neither need we. In the course of rejecting the
argument that Amendment 2 is intended to conserve resources to fight
discrimination against suspect classes, the Colorado Supreme Court made
the limited observation that the amendment is not intended to affect
many antidiscrimination laws protecting nonsuspect classes, Romer II,
882 P. 2d, n.~9. In our view that does not resolve the issue. In any
event, even if, as we doubt, homosexuals could find some safe harbor in
laws of general application, we cannot accept the view that Amendment
2's prohibition on specific legal protections does no more than deprive
homosexuals of special rights. To the contrary, the amendment imposes a
special disability upon those persons alone. Homosexuals are forbidden
the safeguards that others enjoy or may seek without constraint. They
can obtain specific protection against discrimination only by enlisting
the citizenry of Colorado to amend the State Constitution or perhaps, on
the State's view, by trying to pass helpful laws of general
applicability. This is so no matter how local or discrete the harm, no
matter how public and widespread the injury. We find nothing special in
the protections Amendment 2 withholds. These are protections taken for
granted by most people either because they already have them or do not
need them; these are protections against exclusion from an almost
limitless number of transactions and endeavors that constitute ordinary
civic life in a free society.

The Fourteenth Amendment's promise that no person shall be denied the
equal protection of the laws must coexist with the practical necessity
that most legislation classifies for one purpose or another, with
resulting disadvantage to various groups or persons. Personnel
Administrator of Mass. v. Feeney (1979); F. S. Royster Guano Co.~v.
Virginia. We have attempted to reconcile the principle with the reality
by stating that, if a law neither burdens a fundamental right nor
targets a suspect class, we will uphold the legislative classification
so long as it bears a rational relation to some legitimate end. See, e.
g., Heller v. Doe (1993).

Amendment 2 fails, indeed defies, even this conventional inquiry. First,
the amendment has the peculiar property of imposing a broad and
undifferentiated disability on a single named group, an exceptional and,
as we shall explain, invalid form of legislation. Second, its sheer
breadth is so discontinuous with the reasons offered for it that the
amendment seems inexplicable by anything but animus toward the class it
affects; it lacks a rational relationship to legitimate state interests.

Taking the first point, even in the ordinary equal protection case
calling for the most deferential of standards, we insist on knowing the
relation between the classification adopted and the object to be
attained. The search for the link between classification and objective
gives substance to the Equal Protection Clause; it provides guidance and
discipline for the legislature, which is entitled to know what sorts of
laws it can pass; and it marks the limits of our own authority. In the
ordinary case, a law will be sustained if it can be said to advance a
legitimate government interest, even if the law seems unwise or works to
the disadvantage of a particular group, or if the rationale for it seems
tenuous. See New Orleans v. Dukes (tourism benefits justified
classification favoring pushcart vendors of certain longevity);
Williamson v. Lee Optical of Okla., Inc.~(assumed health concerns
justified law favoring optometrists over opticians); Railway Express
Agency, Inc.~v. New York(potential traffic hazards justified exemption
of vehicles advertising the owner's products from general advertising
ban); Kotch v. Board of River Port Pilot Comm'rs for Port of New Orleans
(licensing scheme that disfavored persons unrelated to current river
boat pilots justified by possible efficiency and safety benefits of a
closely knit pilotage system). The laws challenged in the cases just
cited were narrow enough in scope and grounded in a sufficient factual
context for us to ascertain some relation between the classification and
the purpose it served. By requiring that the classification bear a
rational relationship to an independent and legitimate legislative end,
we ensure that classifications are not drawn for the purpose of
disadvantaging the group burdened by the law. See Railroad Retirement
Bd. v. Fritz (Stevens, J., concurring) (``If the adverse impact on the
disfavored class is an apparent aim of the legislature, its impartiality
would be suspect'').

Amendment 2 confounds this normal process of judicial review. It is at
once too narrow and too broad. It identifies persons by a single trait
and then denies them protection across the board. The resulting
disqualification of a class of persons from the right to seek specific
protection from the law is unprecedented in our jurisprudence. The
absence of precedent for Amendment 2 is itself instructive;
``{[}d{]}iscriminations of an unusual character especially suggest
careful consideration to determine whether they are obnoxious to the
constitutional provision.'' Louisville Gas \& Elec. Co.~v. Coleman
(1928).

It is not within our constitutional tradition to enact laws of this
sort. Central both to the idea of the rule of law and to our own
Constitution's guarantee of equal protection is the principle that
government and each of its parts remain open on impartial terms to all
who seek its assistance. ```Equal protection of the laws is not achieved
through indiscriminate imposition of inequalities.''' Sweatt v. Painter
(quoting Shelley v. Kraemer). Respect for this principle explains why
laws singling out a certain class of citizens for disfavored legal
status or general hardships are rare. A law declaring that in general it
shall be more difficult for one group of citizens than for all others to
seek aid from the government is itself a denial of equal protection of
the laws in the most literal sense. ``The guaranty of'equal protection
of the laws is a pledge of the protection of equal laws.''' Skinner v.
Oklahoma ex rel. Williamson (quoting Yick Wo v. Hopkins).

Davis v. Beason, not cited by the parties but relied upon by the
dissent, is not evidence that Amendment 2 is within our constitutional
tradition, and any reliance upon it as authority for sustaining the
amendment is misplaced. In Davis, the Court approved an Idaho
territorial statute denying Mormons, polygamists, and advocates of
polygamy the right to vote and to hold office because, as the Court
construed the statute, it ``simply excludes from the privilege of
voting, or of holding any office of honor, trust or profit, those who
have been convicted of certain offences, and those who advocate a
practical resistance to the laws of the Territory and justify and
approve the commission of crimes forbidden by it.'' To the extent Davis
held that persons advocating a certain practice may be denied the right
to vote, it is no longer good law. Brandenburg v. Ohio (per curiam). To
the extent it held that the groups designated in the statute may be
deprived of the right to vote because of their status, its ruling could
not stand without surviving strict scrutiny, a most doubtful outcome.
Dunn v. Blumstein; cf.~United States v. Brown; United States v. Robel.
To the extent Davis held that a convicted felon may be denied the right
to vote, its holding is not implicated by our decision and is
unexceptionable. See Richardson v. Ramirez.

A second and related point is that laws of the kind now before us raise
the inevitable inference that the disadvantage imposed is born of
animosity toward the class of persons affected. ``{[}I{]}f the
constitutional conception of'equal protection of the laws' means
anything, it must at the very least mean that a bare . desire to harm a
politically unpopular group cannot constitute a legitimate governmental
interest.'' Department of Agriculture v. Moreno. Even laws enacted for
broad and ambitious purposes often can be explained by reference to
legitimate public policies which justify the incidental disadvantages
they impose on certain persons. Amendment 2, however, in making a
general announcement that gays and lesbians shall not have any
particular protections from the law, inflicts on them immediate,
continuing, and real injuries that outrun and belie any legitimate
justifications that may be claimed for it. We conclude that, in addition
to the far-reaching deficiencies of Amendment 2 that we have noted, the
principles it offends, in another sense, are conventional and venerable;
a law must bear a rational relationship to a legitimate governmental
purpose, Kadrmas v. Dickinson Public Schools, and Amendment 2 does not.

The primary rationale the State offers for Amendment 2 is respect for
other citizens' freedom of association, and in particular the liberties
of landlords or employers who have personal or religious objections to
homosexuality. Colorado also cites its interest in conserving resources
to fight discrimination against other groups. The breadth of the
amendment is so far removed from these particular justifications that we
find it impossible to credit them. We cannot say that Amendment 2 is
directed to any identifiable legitimate purpose or discrete objective.
It is a status-based enactment divorced from any factual context from
which we could discern a relationship to legitimate state interests; it
is a classification of persons undertaken for its own sake, something
the Equal Protection Clause does not permit. ``{[}C{]}lass legislation .
{[}is{]} obnoxious to the prohibitions of the Fourteenth Amendment . .''
Civil Rights Cases.

We must conclude that Amendment 2 classifies homosexuals not to further
a proper legislative end but to make them unequal to everyone else. This
Colorado cannot do. A State cannot so deem a class of persons a stranger
to its laws. Amendment 2 violates the Equal Protection Clause, and the
judgment of the Supreme Court of Colorado is affirmed.

\textbf{Justice Scalia, with whom The Chief Justice and Justice Thomas
join, dissenting.} The Court has mistaken a Kulturkampf for a fit of
spite. The constitutional amendment before us here is not the
manifestation of a ```bare . desire to harm''' homosexuals, but is
rather a modest attempt by seemingly tolerant Coloradans to preserve
traditional sexual mores against the efforts of a politically powerful
minority to revise those mores through use of the laws. That objective,
and the means chosen to achieve it, are not only unimpeachable under any
constitutional doctrine hitherto pronounced (hence the opinion's heavy
reliance upon principles of righteousness rather than judicial
holdings); they have been specifically approved by the Congress of the
United States and by this Court.

In holding that homosexuality cannot be singled out for disfavorable
treatment, the Court contradicts a decision, unchallenged here,
pronounced only 10 years ago, see Bowers v. Hardwick, and places the
prestige of this institution behind the proposition that opposition to
homosexuality is as reprehensible as racial or religious bias. Whether
it is or not is precisely the cultural debate that gave rise to the
Colorado constitutional amendment (and to the preferential laws against
which the amendment was directed). Since the Constitution of the United
States says nothing about this subject, it is left to be resolved by
normal democratic means, including the democratic adoption of provisions
in state constitutions. This Court has no business imposing upon all
Americans the resolution favored by the elite class from which the
Members of this institution are selected, pronouncing that ``animosity''
toward homosexuality, is evil. I vigorously dissent.

The only denial of equal treatment it contends homosexuals have suffered
is this: They may not obtain preferential treatment without amending the
State Constitution. That is to say, the principle underlying the Court's
opinion is that one who is accorded equal treatment under the laws, but
cannot as readily as others obtain preferential treatment under the
laws, has been denied equal protection of the laws. If merely stating
this alleged ``equal protection'' violation does not suffice to refute
it, our constitutional jurisprudence has achieved terminal silliness.

The central thesis of the Court's reasoning is that any group is denied
equal protection when, to obtain advantage (or, presumably, to avoid
disadvantage), it must have recourse to a more general and hence more
difficult level of political decisionmaking than others. The world has
never heard of such a principle, which is why the Court's opinion is so
long on emotive utterance and so short on relevant legal citation. And
it seems to me most unlikely that any multilevel democracy can function
under such a principle. For whenever a disadvantage is imposed, or
conferral of a benefit is prohibited, at one of the higher levels of
democratic decisionmaking (i. e., by the state legislature rather than
local government, or by the people at large in the state constitution
rather than the legislature), the affected group has (under this theory)
been denied equal protection. To take the simplest of examples, consider
a state law prohibiting the award of municipal contracts to relatives of
mayors or city councilmen. Once such a law is passed, the group composed
of such relatives must, in order to get the benefit of city contracts,
persuade the state legislature---unlike all other citizens, who need
only persuade the municipality. It is ridiculous to consider this a
denial of equal protection, which is why the Court's theory is unheard
of.

The Court might reply that the example I have given is not a denial of
equal protection only because the same ``rational basis'' (avoidance of
corruption) which renders constitutional the substantive discrimination
against relatives (i. e., the fact that they alone cannot obtain city
contracts) also automatically suffices to sustain what might be called
the electoral-procedural discrimination against them (i. e., the fact
that they must go to the state level to get this changed). This is of
course a perfectly reasonable response, and would explain why
``electoral-procedural discrimination'' has not hitherto been heard of:
A law that is valid in its substance is automatically valid in its level
of enactment. But the Court cannot afford to make this argument, for as
I shall discuss next, there is no doubt of a rational basis for the
substance of the prohibition at issue here. The Court's entire novel
theory rests upon the proposition that there is something special
---something that cannot be justified by normal ``rational basis''
analysis---in making a disadvantaged group (or a nonpreferred group)
resort to a higher decisionmaking level. That proposition finds no
support in law or logic.

I turn next to whether there was a legitimate rational basis for the
substance of the constitutional amendment---for the prohibition of
special protection for homosexuals It is unsurprising that the Court
avoids discussion of this question, since the answer is so obviously
yes. The case most relevant to the issue before us today is not even
mentioned in the Court's opinion: In Bowers v. Hardwick, we held that
the Constitution does not prohibit what virtually all States had done
from the founding of the Republic until very recent years---making
homosexual conduct a crime. That holding is unassailable, except by
those who think that the Constitution changes to suit current fashions.
But in any event it is a given in the present case: Respondents' briefs
did not urge overruling Bowers,and at oral argument respondents' counsel
expressly disavowed any intent to seek such overruling, Tr. of Oral Arg.
53. If it is constitutionally permissible for a State to make homosexual
conduct criminal, surely it is constitutionally permissible for a State
to enact other laws merely disfavoring homosexual conduct. (As the Court
of Appeals for the District of Columbia Circuit has aptly put it: ``If
the Court {[}in Bowers{]} was unwilling to object to state laws that
criminalize the behavior that defines the class, it is hardly open . to
conclude that state sponsored discrimination against the class is
invidious. After all, there can hardly be more palpable discrimination
against a class than making the conduct that defines the class
criminal.'' Padula v. Webster, .) And a fortiori it is constitutionally
permissible for a State to adopt a provision not even disfavoring
homosexual conduct, but merely prohibiting all levels of state
government from bestowing special protections upon homosexual conduct.
Respondents (who, unlike the Court, cannot afford the luxury of ignoring
inconvenient precedent) counter Bowers with the argument that a
greater-includes-the-lesser rationale cannot justify Amendment 2's
application to individuals who do not engage in homosexual acts, but are
merely of homosexual ``orientation.'' Some Courts of Appeals have
concluded that, with respect to laws of this sort at least, that is a
distinction without a difference. See Equality Foundation of Greater
Cincinnati, Inc.~v. Cincinnati, CA6 1995) (``{[}F{]}or purposes of these
proceedings, it is virtually impossible to distinguish or separate
individuals of a particular orientation which predisposes them toward a
particular sexual conduct from those who actually engage in that
particular type of sexual conduct''); Steffan v. Perry, -690 (CADC
1994). The Supreme Court of Colorado itself appears to be of this view.
See 882 P. 2d (``Amendment 2 targets this class of persons based on four
characteristics: sexual orientation; conduct; practices, and
relationships. Each characteristic provides a potentially different way
of identifying that class of persons who are gay, lesbian, or bisexual.
These four characteristics are not truly severable from one another
because each provides nothing more than a different way of identifying
the same class of persons'') (emphasis added).

The foregoing suffices to establish what the Court's failure to cite any
case remotely in point would lead one to suspect: No principle set forth
in the Constitution, nor even any imagined by this Court in the past 200
years, prohibits what Colorado has done here. But the case for Colorado
is much stronger than that. What it has done is not only unprohibited,
but eminently reasonable, with close, congressionally approved precedent
in earlier constitutional practice.

First, as to its eminent reasonableness. The Court's opinion contains
grim, disapproving hints that Coloradans have been guilty of ``animus''
or ``animosity'' toward homosexuality, as though that has been
established as un-American. Of course it is our moral heritage that one
should not hate any human being or class of human beings. But I had
thought that one could consider certain conduct reprehensible---murder,
for example, or polygamy, or cruelty to animals---and could exhibit even
``animus'' toward such conduct. Surely that is the only sort of
``animus'' at issue here: moral disapproval of homosexual conduct, the
same sort of moral disapproval that produced the centuries-old criminal
laws that we held constitutional in Bowers. The Colorado amendment does
not, to speak entirely precisely, prohibit giving favored status to
people who are homosexuals; they can be favored for many reasons---for
example, because they are senior citizens or members of racial
minorities. But it prohibits giving them favored status because of their
homosexual conduct ---that is, it prohibits favored status for
homosexuality.

But though Coloradans are, as I say, entitled to be hostile toward
homosexual conduct, the fact is that the degree of hostility reflected
by Amendment 2 is the smallest conceivable. The Court's portrayal of
Coloradans as a society fallen victim to pointless, hate-filled
``gay-bashing'' is so false as to be comical. Colorado not only is one
of the 25 States that have repealed their antisodomy laws, but was among
the first to do so. See 1971 Colo. Sess. Laws, ch.~121, § 1. But the
society that eliminates criminal punishment for homosexual acts does not
necessarily abandon the view that homosexuality is morally wrong and
socially harmful; often, abolition simply reflects the view that
enforcement of such criminal laws involves unseemly intrusion into the
intimate lives of citizens. Cf. Brief for Lambda Legal Defense and
Education Fund, Inc., et al.~as Amici Curiae in Bowers v. Hardwick,O. T.
1985, No.~85-140, p.~25, n.~21 (antisodomy statutes are ``unenforceable
by any but the most offensive snooping and wasteful allocation of law
enforcement resources''); Kadish, The Crisis of Overcriminalization, 374
The Annals of the American Academy of Political and Social Science 157,
161 (1967) (``To obtain evidence {[}in sodomy cases{]}, police are
obliged to resort to behavior which tends to degrade and demean both
themselves personally and law enforcement as an institution'').

By the time Coloradans were asked to vote on Amendment 2, their exposure
to homosexuals' quest for social endorsement was not limited to
newspaper accounts of happenings in places such as New York, Los
Angeles, San Francisco, and Key West. Three Colorado cities---Aspen,
Boulder, and Denver---had enacted ordinances that listed ``sexual
orientation'' as an impermissible ground for discrimination, equating
the moral disapproval of homosexual conduct with racial and religious
bigotry. See Aspen Municipal Code § 13-98 (1977); Boulder Rev.~Municipal
Code §§ 12-1---1 to 12-1---11 (1987); Denver Rev.~Municipal Code, Art.
IV, §§ 28-91 to 28-116 (1991). The phenomenon had even appeared
statewide: The Governor of Colorado had signed an executive order
pronouncing that ``in the State of Colorado we recognize the diversity
in our pluralistic society and strive to bring an end to discrimination
in any form,'' and directing state agency-heads to ``ensure
non-discrimination'' in hiring and promotion based on, among other
things, ``sexual orientation.'' Executive Order No.~D0035 (Dec.~10,
1990). I do not mean to be critical of these legislative successes;
homosexuals are as entitled to use the legal system for reinforcement of
their moral sentiments as is the rest of society. But they are subject
to being countered by lawful, democratic countermeasures as well.

That is where Amendment 2 came in. It sought to counter both the
geographic concentration and the disproportionate political power of
homosexuals by (1) resolving the controversy at the statewide level, and
(2) making the election a single-issue contest for both sides. It put
directly, to all the citizens of the State, the question: Should
homosexuality be given special protection? They answered no. The Court
today asserts that this most democratic of procedures is
unconstitutional. Lacking any cases to establish that facially absurd
proposition, it simply asserts that it must be unconstitutional, because
it has never happened before.

The Court today, announcing that Amendment 2 ``defies . conventional
{[}constitutional{]} inquiry,'' and ``confounds {[}the{]} normal process
of judicial review,'' employs a constitutional theory heretofore unknown
to frustrate Colorado's reasonable effort to preserve traditional
American moral values. The Court's stern disapproval of ``animosity''
towards homosexuality might be compared with what an earlier Court
(including the revered Justices Harlan and Bradley) said in Murphy v.
Ramsey, rejecting a constitutional challenge to a United States statute
that denied the franchise in federal territories to those who engaged in
polygamous cohabitation:

``{[}C{]}ertainly no legislation can be supposed more wholesome and
necessary in the founding of a free, self-governing commonwealth, fit to
take rank as one of the co-ordinate States of the Union, than that which
seeks to establish it on the basis of the idea of the family, as
consisting in and springing from the union for life of one man and one
woman in the holy estate of matrimony; the sure foundation of all that
is stable and noble in our civilization; the best guaranty of that
reverent morality which is the source of all beneficent progress in
social and political improvement.''

I would not myself indulge in such official praise for heterosexual
monogamy, because I think it no business of the courts (as opposed to
the political branches) to take sides in this culture war.

But the Court today has done so, not only by inventing a novel and
extravagant constitutional doctrine to take the victory away from
traditional forces, but even by verbally disparaging as bigotry
adherence to traditional attitudes. To suggest, for example, that this
constitutional amendment springs from nothing more than ```a bare .
desire to harm a politically unpopular group,''' quoting Department of
Agriculture v. Moreno, is nothing short of insulting. (It is also
nothing short of preposterous to call ``politically unpopular'' a group
which enjoys enormous influence in American media and politics, and
which, as the trial court here noted, though composing no more than 4\%
of the population had the support of 46\% of the voters on Amendment 2,
see App. to Pet. for Cert. C-18.)

When the Court takes sides in the culture wars, it tends to be with the
knights rather than the villeins---and more specifically with the
Templars, reflecting the views and values of the lawyer class from which
the Court's Members are drawn. How that class feels about homosexuality
will be evident to anyone who wishes to interview job applicants at
virtually any of the Nation's law schools. The interviewer may refuse to
offer a job because the applicant is a Republican; because he is an
adulterer; because he went to the wrong prep school or belongs to the
wrong country club; because he eats snails; because he is a womanizer;
because she wears real-animal fur; or even because he hates the Chicago
Cubs. But if the interviewer should wish not to be an associate or
partner of an applicant because he disapproves of the applicant's
homosexuality, then he will have violated the pledge which the
Association of American Law Schools requires all its member schools to
exact from job interviewers: ``assurance of the employer's willingness''
to hire homosexuals. Bylaws of the Association of American Law Schools,
Inc.~§ 6-4(b); Executive Committee Regulations of the Association of
American Law Schools § 6 in 1995 Handbook, Association of American Law
Schools. This law-school view of what ``prejudices'' must be stamped out
may be contrasted with the more plebeian attitudes that apparently still
prevail in the United States Congress, which has been unresponsive to
repeated attempts to extend to homosexuals the protections of federal
civil rights laws, see, e. g., Employment NonDiscrimination Act of 1994,
S. 2238, 103d Cong., 2d Sess. (1994); Civil Rights Amendments of 1975,
H. R. 5452, 94th Cong., 1st Sess. (1975), and which took the pains to
exclude them specifically from the Americans with Disabilities Act of
1990, see 42 U. S. C. § 12211(a) (1988 ed., Supp. V).

\hypertarget{lawrence-v.-texas}{%
\subsubsection{Lawrence v. Texas}\label{lawrence-v.-texas}}

\textbf{JUSTICE KENNEDY delivered the opinion of the Court.} Liberty
protects the person from unwarranted government intrusions into a
dwelling or other private places. In our tradition the State is not
omnipresent in the home. And there are other spheres of our lives and
existence, outside the home, where the State should not be a dominant
presence. Freedom extends beyond spatial bounds. Liberty presumes an
autonomy of self that includes freedom of thought, belief, expression,
and certain intimate conduct. The instant case involves liberty of the
person both in its spatial and in its more transcendent dimensions.

The question before the Court is the validity of a Texas statute making
it a crime for two persons of the same sex to engage in certain intimate
sexual conduct.

In Houston, Texas, officers of the Harris County Police Department were
dispatched to a private residence in response to a reported weapons
disturbance. They entered an apartment where one of the petitioners,
John Geddes Lawrence, resided. The right of the police to enter does not
seem to have been questioned. The officers observed Lawrence and another
man, Tyron Garner, engaging in a sexual act. The two petitioners were
arrested, held in custody overnight, and charged and convicted before a
Justice of the Peace.

The complaints described their crime as ``deviate sexual intercourse,
namely anal sex, with a member of the same sex (man).'' App. to Pet. for
Cert. 127a, 139a. The applicable state law is Tex. Penal Code Ann. § 21
(a) (2003). It provides: ``A person commits an offense if he engages in
deviate sexual intercourse with another individual of the same sex.''
The statute defines ``{[}d{]}eviate sexual intercourse'' as follows:

``(A) any contact between any part of the genitals of one person and the
mouth or anus of another person; or''(B) the penetration of the genitals
or the anus of another person with an object." § 21 (1).

We conclude the case should be resolved by determining whether the
petitioners were free as adults to engage in the private conduct in the
exercise of their liberty under the Due Process Clause of the Fourteenth
Amendment to the Constitution. For this inquiry we deem it necessary to
reconsider the Court's holding in Bowers.

The historical grounds relied upon in Bowers are more complex than the
majority opinion and the concurring opinion by Chief Justice Burger
indicate. Their historical premises are not without doubt and, at the
very least, are overstated.

This emerging recognition should have been apparent when Bowers was
decided. In 1955 the American Law Institute promulgated the Model Penal
Code and made clear that it did not recommend or provide for ``criminal
penalties for consensual sexual relations conducted in private.'' ALI,
Model Penal Code § 213 Comment 2, p.~372 (1980). It justified its
decision on three grounds: (1) The prohibitions undermined respect for
the law by penalizing conduct many people engaged in; (2) the statutes
regulated private conduct not harmful to others; and (3) the laws were
arbitrarily enforced and thus invited the danger of blackmail. ALI,
Model Penal Code, Commentary 277-280 (Tent. Draft No.~4, 1955). In 1961
Illinois changed its laws to conform to the Model Penal Code. Other
States soon followed. Brief for Cato Institute as Amicus Curiae 15-16.

In Bowers the Court referred to the fact that before 1961 all 50 States
had outlawed sodomy, and that at the time of the Court's decision 24
States and the District of Columbia had sodomy laws. 478 U. S.. Justice
Powell pointed out that these prohibitions often were being ignored,
however. Georgia, for instance, had not sought to enforce its law for
decades. -198, n.~2 (``The history of nonenforcement suggests the
moribund character today of laws criminalizing this type of private,
consensual conduct'').

The sweeping references by Chief Justice Burger to the history of
Western civilization and to Judeo-Christian moral and ethical standards
did not take account of other authorities pointing in an opposite
direction. A committee advising the British Parliament recommended in
1957 repeal of laws punishing homosexual conduct. The Wolfenden Report:
Report of the Committee on Homosexual Offenses and Prostitution (1963).
Parliament enacted the substance of those recommendations 10 years
later. Sexual Offences Act 1967, § 1.

Of even more importance, almost five years before Bowers was decided the
European Court of Human Rights considered a case with parallels to
Bowers and to today's case. An adult male resident in Northern Ireland
alleged he was a practicing homosexual who desired to engage in
consensual homosexual conduct. The laws of Northern Ireland forbade him
that right. He alleged that he had been questioned, his home had been
searched, and he feared criminal prosecution. The court held that the
laws proscribing the conduct were invalid under the European Convention
on Human Rights. Dudgeon v. United Kingdom, 45 Eur. Ct. H. R. (1981) ¶
52. Authoritative in all countries that are members of the Council of
Europe (21 nations then, 45 nations now), the decision is at odds with
the premise in Bowers that the claim put forward was insubstantial in
our Western civilization.

In our own constitutional system the deficiencies in Bowers became even
more apparent in the years following its announcement. The 25 States
with laws prohibiting the relevant conduct referenced in the Bowers
decision are reduced now to 13, of which 4 enforce their laws only
against homosexual conduct. In those States where sodomy is still
proscribed, whether for same-sex or heterosexual conduct, there is a
pattern of nonenforcement with respect to consenting adults acting in
private. The State of Texas admitted in 1994 that as of that date it had
not prosecuted anyone under those circumstances. State v. Morales, 869
S. W. 2d 941, 943.

Two principal cases decided after Bowers cast its holding into even more
doubt. In Planned Parenthood of Southeastern Pa. v. Casey, the Court
reaffirmed the substantive force of the liberty protected by the Due
Process Clause. The Casey decision again confirmed that our laws and
tradition afford constitutional protection to personal decisions
relating to marriage, procreation, contraception, family relationships,
child rearing, and education. In explaining the respect the Constitution
demands for the autonomy of the person in making these choices, we
stated as follows:

``These matters, involving the most intimate and personal choices a
person may make in a lifetime, choices central to personal dignity and
autonomy, are central to the liberty protected by the Fourteenth
Amendment. At the heart of liberty is the right to define one's own
concept of existence, of meaning, of the universe, and of the mystery of
human life. Beliefs about these matters could not define the attributes
of personhood were they formed under compulsion of the State.''

Persons in a homosexual relationship may seek autonomy for these
purposes, just as heterosexual persons do. The decision in Bowers would
deny them this right.

The second post-Bowers case of principal relevance is Romer v. Evans.
There the Court struck down class-based legislation directed at
homosexuals as a violation of the Equal Protection Clause. Romer
invalidated an amendment to Colorado's Constitution which named as a
solitary class persons who were homosexuals, lesbians, or bisexual
either by ``orientation, conduct, practices or relationships,''
(internal quotation marks omitted), and deprived them of protection
under state antidiscrimination laws. We concluded that the provision was
``born of animosity toward the class of persons affected'' and further
that it had no rational relation to a legitimate governmental purpose.

As an alternative argument in this case, counsel for the petitioners and
some amici contend that Romer provides the basis for declaring the Texas
statute invalid under the Equal Protection Clause. That is a tenable
argument, but we conclude the instant case requires us to address
whether Bowers itself has continuing validity. Were we to hold the
statute invalid under the Equal Protection Clause some might question
whether a prohibition would be valid if drawn differently, say, to
prohibit the conduct both between same-sex and different-sex
participants.

Equality of treatment and the due process right to demand respect for
conduct protected by the substantive guarantee of liberty are linked in
important respects, and a decision on the latter point advances both
interests. If protected conduct is made criminal and the law which does
so remains unexamined for its substantive validity, its stigma might
remain even if it were not enforceable as drawn for equal protection
reasons. When homosexual conduct is made criminal by the law of the
State, that declaration in and of itself is an invitation to subject
homosexual persons to discrimination both in the public and in the
private spheres. The central holding of Bowers has been brought in
question by this case, and it should be addressed. Its continuance as
precedent demeans the lives of homosexual persons.

Bowers was not correct when it was decided, and it is not correct today.
It ought not to remain binding precedent. Bowers v. Hardwick should be
and now is overruled.

The present case does not involve minors. It does not involve persons
who might be injured or coerced or who are situated in relationships
where consent might not easily be refused. It does not involve public
conduct or prostitution. It does not involve whether the government must
give formal recognition to any relationship that homosexual persons seek
to enter. The case does involve two adults who, with full and mutual
consent from each other, engaged in sexual practices common to a
homosexual lifestyle. The petitioners are entitled to respect for their
private lives. The State cannot demean their existence or control their
destiny by making their private sexual conduct a crime. Their right to
liberty under the Due Process Clause gives them the full right to engage
in their conduct without intervention of the government. ``It is a
promise of the Constitution that there is a realm of personal liberty
which the government may not enter.'' Casey. The Texas statute furthers
no legitimate state interest which can justify its intrusion into the
personal and private life of the individual.

Had those who drew and ratified the Due Process Clauses of the Fifth
Amendment or the Fourteenth Amendment known the components of liberty in
its manifold possibilities, they might have been more specific. They did
not presume to have this insight. They knew times can blind us to
certain truths and later generations can see that laws once thought
necessary and proper in fact serve only to oppress. As the Constitution
endures, persons in every generation can invoke its principles in their
own search for greater freedom.

\hypertarget{obergefell-v.-hodges}{%
\subsubsection{Obergefell v. Hodges}\label{obergefell-v.-hodges}}

576 U.S. \_\_\_ (2015)

\textbf{JUSTICE KENNEDY, delivered the opinion of the Court.} The
Constitution promises liberty to all within its reach, a liberty that
includes certain specific rights that allow persons, within a lawful
realm, to define and express their identity. The petitioners in these
cases seek to find that liberty by marrying someone of the same sex and
having their marriages deemed lawful on the same terms and conditions as
marriages between persons of the opposite sex.

These cases come from Michigan, Kentucky, Ohio, and Tennessee, States
that define marriage as a union between one man and one woman. See,
e.g., Mich. Const., Art. I, §25; Ky. Const. §233A; Ohio Rev.~Code Ann.
§3101 (Lexis 2008); Tenn. Const., Art. XI, §18. The petitioners are 14
same-sex couples and two men whose same-sex partners are deceased. The
respondents are state officials responsible for enforcing the laws in
question. The petitioners claim the respondents violate the Fourteenth
Amendment by denying them the right to marry or to have their marriages,
lawfully performed in another State, given full recognition.

Petitioners filed these suits in United States District Courts in their
home States. Each District Court ruled in their favor. Citations to
those cases are in Appendix A, infra. The respondents appealed the
decisions against them to the United States Court of Appeals for the
Sixth Circuit. It consolidated the cases and reversed the judgments of
the District Courts. DeBoer v. Snyder, The Court of Appeals held that a
State has no constitutional obligation to license same-sex marriages or
to recognize same-sex marriages performed out of State.

The petitioners sought certiorari. This Court granted review, limited to
two questions. 574 U. S. \_\_\_ (2015). The first, presented by the
cases from Michigan and Kentucky, is whether the Fourteenth Amendment
requires a State to license a marriage between two people of the same
sex. The second, presented by the cases from Ohio, Tennessee, and,
again, Kentucky, is whether the Fourteenth Amendment requires a State to
recognize a same-sex marriage licensed and performed in a State which
does grant that right.

Before addressing the principles and precedents that govern these cases,
it is appropriate to note the history of the subject now before the
Court.

From their beginning to their most recent page, the annals of human
history reveal the transcendent importance of marriage. The lifelong
union of a man and a woman always has promised nobility and dignity to
all persons, without regard to their station in life. Marriage is sacred
to those who live by their religions and offers unique fulfillment to
those who find meaning in the secular realm. Its dynamic allows two
people to find a life that could not be found alone, for a marriage
becomes greater than just the two persons. Rising from the most basic
human needs, marriage is essential to our most profound hopes and
aspirations.

The centrality of marriage to the human condition makes it unsurprising
that the institution has existed for millennia and across civilizations.
Since the dawn of history, marriage has transformed strangers into
relatives, binding families and societies together. Confucius taught
that marriage lies at the foundation of government. 2 Li Chi: Book of
Rites 266 (C. Chai \& W. Chai eds., J. Legge transl. 1967). This wisdom
was echoed centuries later and half a world away by Cicero, who wrote,
``The first bond of society is marriage; next, children; and then the
family.'' See De Officiis 57 (W. Miller transl. 1913). There are untold
references to the beauty of marriage in religious and philosophical
texts spanning time, cultures, and faiths, as well as in art and
literature in all their forms. It is fair and necessary to say these
references were based on the understanding that marriage is a union
between two persons of the opposite sex.

That history is the beginning of these cases. The respondents say it
should be the end as well. To them, it would demean a timeless
institution if the concept and lawful status of marriage were extended
to two persons of the same sex. Marriage, in their view, is by its
nature a gender-differentiated union of man and woman. This view long
has been held---and continues to be held---in good faith by reasonable
and sincere people here and throughout the world.

The petitioners acknowledge this history but contend that these cases
cannot end there. Were their intent to demean the revered idea and
reality of marriage, the petitioners' claims would be of a different
order. But that is neither their purpose nor their submission. To the
contrary, it is the enduring importance of marriage that underlies the
petitioners' contentions. This, they say, is their whole point. Far from
seeking to devalue marriage, the petitioners seek it for themselves
because of their respect---and need---for its privileges and
responsibilities. And their immutable nature dictates that same-sex
marriage is their only real path to this profound commitment.

Recounting the circumstances of three of these cases illustrates the
urgency of the petitioners' cause from their perspective. Petitioner
James Obergefell, a plaintiff in the Ohio case, met John Arthur over two
decades ago. They fell in love and started a life together, establishing
a lasting, committed relation. In 2011, however, Arthur was diagnosed
with amyotrophic lateral sclerosis, or ALS. This debilitating disease is
progressive, with no known cure. Two years ago, Obergefell and Arthur
decided to commit to one another, resolving to marry before Arthur died.
To fulfill their mutual promise, they traveled from Ohio to Maryland,
where same-sex marriage was legal. It was difficult for Arthur to move,
and so the couple were wed inside a medical transport plane as it
remained on the tarmac in Baltimore. Three months later, Arthur died.
Ohio law does not permit Obergefell to be listed as the surviving spouse
on Arthur's death certificate. By statute, they must remain strangers
even in death, a state-imposed separation Obergefell deems ``hurtful for
the rest of time.'' App. in No.~14-556 etc., p.~38. He brought suit to
be shown as the surviving spouse on Arthur's death certificate.

April DeBoer and Jayne Rowse are co-plaintiffs in the case from
Michigan. They celebrated a commitment ceremony to honor their permanent
relation in 2007. They both work as nurses, DeBoer in a neonatal unit
and Rowse in an emergency unit. In 2009, DeBoer and Rowse fostered and
then adopted a baby boy. Later that same year, they welcomed another son
into their family. The new baby, born prematurely and abandoned by his
biological mother, required around-the-clock care. The next year, a baby
girl with special needs joined their family. Michigan, however, permits
only opposite-sex married couples or single individuals to adopt, so
each child can have only one woman as his or her legal parent. If an
emergency were to arise, schools and hospitals may treat the three
children as if they had only one parent. And, were tragedy to befall
either DeBoer or Rowse, the other would have no legal rights over the
children she had not been permitted to adopt. This couple seeks relief
from the continuing uncertainty their unmarried status creates in their
lives.

Army Reserve Sergeant First Class Ijpe DeKoe and his partner Thomas
Kostura, co-plaintiffs in the Tennessee case, fell in love. In 2011,
DeKoe received orders to deploy to Afghanistan. Before leaving, he and
Kostura married in New York. A week later, DeKoe began his deployment,
which lasted for almost a year. When he returned, the two settled in
Tennessee, where DeKoe works full-time for the Army Reserve. Their
lawful marriage is stripped from them whenever they reside in Tennessee,
returning and disappearing as they travel across state lines. DeKoe, who
served this Nation to preserve the freedom the Constitution protects,
must endure a substantial burden.

The cases now before the Court involve other petitioners as well, each
with their own experiences. Their stories reveal that they seek not to
denigrate marriage but rather to live their lives, or honor their
spouses' memory, joined by its bond.

The ancient origins of marriage confirm its centrality, but it has not
stood in isolation from developments in law and society. The history of
marriage is one of both continuity and change. That institution---even
as confined to opposite-sex relations---has evolved over time.

For example, marriage was once viewed as an arrangement by the couple's
parents based on political, religious, and financial concerns; but by
the time of the Nation's founding it was understood to be a voluntary
contract between a man and a woman. See N. Cott, Public Vows: A History
of Marriage and the Nation 9-17 (2000); S. Coontz, Marriage, A History
15-16 (2005). As the role and status of women changed, the institution
further evolved. Under the centuries-old doctrine of coverture, a
married man and woman were treated by the State as a single,
male-dominated legal entity. See 1 W. Blackstone, Commentaries on the
Laws of England 430 (1765). As women gained legal, political, and
property rights, and as society began to understand that women have
their own equal dignity, the law of coverture was abandoned. See Brief
for Historians of Marriage et al.~as Amici Curiae 16-19. These and other
developments in the institution of marriage over the past centuries were
not mere superficial changes. Rather, they worked deep transformations
in its structure, affecting aspects of marriage long viewed by many as
essential. See generally N. Cott, Public Vows; S. Coontz, Marriage; H.
Hartog, Man \& Wife in America: A History (2000).

These new insights have strengthened, not weakened, the institution of
marriage. Indeed, changed understandings of marriage are characteristic
of a Nation where new dimensions of freedom become apparent to new
generations, often through perspectives that begin in pleas or protests
and then are considered in the political sphere and the judicial
process.

This dynamic can be seen in the Nation's experiences with the rights of
gays and lesbians. Until the mid-20th century, same-sex intimacy long
had been condemned as immoral by the state itself in most Western
nations, a belief often embodied in the criminal law. For this reason,
among others, many persons did not deem homosexuals to have dignity in
their own distinct identity. A truthful declaration by same-sex couples
of what was in their hearts had to remain unspoken. Even when a greater
awareness of the humanity and integrity of homosexual persons came in
the period after World War II, the argument that gays and lesbians had a
just claim to dignity was in conflict with both law and widespread
social conventions. Same-sex intimacy remained a crime in many States.
Gays and lesbians were prohibited from most government employment,
barred from military service, excluded under immigration laws, targeted
by police, and burdened in their rights to associate. See Brief for
Organization of American Historians as Amicus Curiae 5-28.

For much of the 20th century, moreover, homosexuality was treated as an
illness. When the American Psychiatric Association published the first
Diagnostic and Statistical Manual of Mental Disorders in 1952,
homosexuality was classified as a mental disorder, a position adhered to
until 1973. See Position Statement on Homosexuality and Civil Rights,
1973, in 131 Am. J. Psychiatry 497 (1974). Only in more recent years
have psychiatrists and others recognized that sexual orientation is both
a normal expression of human sexuality and immutable. See Brief for
American Psychological Association et al.~as Amici Curiae 7-17.

In the late 20th century, following substantial cultural and political
developments, same-sex couples began to lead more open and public lives
and to establish families. This development was followed by a quite
extensive discussion of the issue in both governmental and private
sectors and by a shift in public attitudes toward greater tolerance. As
a result, questions about the rights of gays and lesbians soon reached
the courts, where the issue could be discussed in the formal discourse
of the law.

This Court first gave detailed consideration to the legal status of
homosexuals in Bowers v. Hardwick. There it upheld the constitutionality
of a Georgia law deemed to criminalize certain homosexual acts. Ten
years later, in Romer v. Evans, the Court invalidated an amendment to
Colorado's Constitution that sought to foreclose any branch or political
subdivision of the State from protecting persons against discrimination
based on sexual orientation. Then, in 2003, the Court overruled Bowers,
holding that laws making same-sex intimacy a crime ``demea{[}n{]} the
lives of homosexual persons.'' Lawrence v. Texas.

Against this background, the legal question of same-sex marriage arose.
In 1993, the Hawaii Supreme Court held Hawaii's law restricting marriage
to opposite-sex couples constituted a classification on the basis of sex
and was therefore subject to strict scrutiny under the Hawaii
Constitution. Baehr v. Lewin, 74 Haw. 530, 852 P. 2d 44. Although this
decision did not mandate that same-sex marriage be allowed, some States
were concerned by its implications and reaffirmed in their laws that
marriage is defined as a union between opposite-sex partners. So too in
1996, Congress passed the Defense of Marriage Act (DOMA), 110 Stat.
2419, defining marriage for all federal-law purposes as ``only a legal
union between one man and one woman as husband and wife.'' 1 U. S. C.
§7.

The new and widespread discussion of the subject led other States to a
different conclusion. In 2003, the Supreme Judicial Court of
Massachusetts held the State's Constitution guaranteed same-sex couples
the right to marry. See Goodridge v. Department of Public Health, 440
Mass. 309, 798 N. E. 2d 941 (2003). After that ruling, some additional
States granted marriage rights to same-sex couples, either through
judicial or legislative processes. These decisions and statutes are
cited in Appendix B, infra. Two Terms ago, in United States v. Windsor,
570 U. S. \_\_\_ (2013), this Court invalidated DOMA to the extent it
barred the Federal Government from treating same-sex marriages as valid
even when they were lawful in the State where they were licensed. DOMA,
the Court held, impermissibly disparaged those same-sex couples ``who
wanted to affirm their commitment to one another before their children,
their family, their friends, and their community.'' \_\_\_ (slip op.).

Numerous cases about same-sex marriage have reached the United States
Courts of Appeals in recent years. In accordance with the judicial duty
to base their decisions on principled reasons and neutral discussions,
without scornful or disparaging commentary, courts have written a
substantial body of law considering all sides of these issues. That case
law helps to explain and formulate the underlying principles this Court
now must consider. With the exception of the opinion here under review
and one other, see Citizens for Equal Protection v. Bruning, -868 (CA8
2006), the Courts of Appeals have held that excluding same-sex couples
from marriage violates the Constitution. There also have been many
thoughtful District Court decisions addressing same-sex marriage---and
most of them, too, have concluded same-sex couples must be allowed to
marry. In addition the highest courts of many States have contributed to
this ongoing dialogue in decisions interpreting their own State
Constitutions. These state and federal judicial opinions are cited in
Appendix A, infra.

After years of litigation, legislation, referenda, and the discussions
that attended these public acts, the States are now divided on the issue
of same-sex marriage. See Office of the Atty. Gen.~of Maryland, The
State of Marriage Equality in America, State-by-State Supp. (2015).

Under the Due Process Clause of the Fourteenth Amendment, no State shall
``deprive any person of life, liberty, or property, without due process
of law.'' The fundamental liberties protected by this Clause include
most of the rights enumerated in the Bill of Rights. See Duncan v.
Louisiana (1968). In addition these liberties extend to certain personal
choices central to individual dignity and autonomy, including intimate
choices that define personal identity and beliefs. See, e.g., Eisenstadt
v. Baird; Griswold v. Connecticut (1965).

The identification and protection of fundamental rights is an enduring
part of the judicial duty to interpret the Constitution. That
responsibility, however, ``has not been reduced to any formula.'' Poe v.
Ullman (Harlan, J., dissenting). Rather, it requires courts to exercise
reasoned judgment in identifying interests of the person so fundamental
that the State must accord them its respect. See That process is guided
by many of the same considerations relevant to analysis of other
constitutional provisions that set forth broad principles rather than
specific requirements. History and tradition guide and discipline this
inquiry but do not set its outer boundaries. See Lawrence. That method
respects our history and learns from it without allowing the past alone
to rule the present.

The nature of injustice is that we may not always see it in our own
times. The generations that wrote and ratified the Bill of Rights and
the Fourteenth Amendment did not presume to know the extent of freedom
in all of its dimensions, and so they entrusted to future generations a
charter protecting the right of all persons to enjoy liberty as we learn
its meaning. When new insight reveals discord between the Constitution's
central protections and a received legal stricture, a claim to liberty
must be addressed.

Applying these established tenets, the Court has long held the right to
marry is protected by the Constitution. In Loving v. Virginia, which
invalidated bans on interracial unions, a unanimous Court held marriage
is ``one of the vital personal rights essential to the orderly pursuit
of happiness by free men.'' The Court reaffirmed that holding in
Zablocki v. Redhail, which held the right to marry was burdened by a law
prohibiting fathers who were behind on child support from marrying. The
Court again applied this principle in Turner v. Safley, which held the
right to marry was abridged by regulations limiting the privilege of
prison inmates to marry. Over time and in other contexts, the Court has
reiterated that the right to marry is fundamental under the Due Process
Clause. See, e.g., M. L. B. v. S. L. J.; Cleveland Bd. of Ed. v. LaFleur
(1974); Griswold; Skinner v. Oklahoma ex rel. Williamson; Meyer v.
Nebraska.

It cannot be denied that this Court's cases describing the right to
marry presumed a relationship involving opposite-sex partners. The
Court, like many institutions, has made assumptions defined by the world
and time of which it is a part. This was evident in Baker v. Nelson, a
one-line summary decision issued in 1972, holding the exclusion of
same-sex couples from marriage did not present a substantial federal
question.

Still, there are other, more instructive precedents. This Court's cases
have expressed constitutional principles of broader reach. In defining
the right to marry these cases have identified essential attributes of
that right based in history, tradition, and other constitutional
liberties inherent in this intimate bond. See, e.g., Lawrence; Turner;
Zablocki; Loving; Griswold. And in assessing whether the force and
rationale of its cases apply to same-sex couples, the Court must respect
the basic reasons why the right to marry has been long protected.

This analysis compels the conclusion that same-sex couples may exercise
the right to marry. The four principles and traditions to be discussed
demonstrate that the reasons marriage is fundamental under the
Constitution apply with equal force to same-sex couples.

A first premise of the Court's relevant precedents is that the right to
personal choice regarding marriage is inherent in the concept of
individual autonomy. This abiding connection between marriage and
liberty is why Loving invalidated interracial marriage bans under the
Due Process Clause. See 388 U. S.; see also Zablocki (observing Loving
held ``the right to marry is of fundamental importance for all
individuals''). Like choices concerning contraception, family
relationships, procreation, and childrearing, all of which are protected
by the Constitution, decisions concerning marriage are among the most
intimate that an individual can make. See Lawrence. Indeed, the Court
has noted it would be contradictory ``to recognize a right of privacy
with respect to other matters of family life and not with respect to the
decision to enter the relationship that is the foundation of the family
in our society.'' Zablocki.

Choices about marriage shape an individual's destiny. As the Supreme
Judicial Court of Massachusetts has explained, because ``it fulfills
yearnings for security, safe haven, and connection that express our
common humanity, civil marriage is an esteemed institution, and the
decision whether and whom to marry is among life's momentous acts of
self-definition.'' Goodridge, 440 Mass., 798 N. E. 2d.

The nature of marriage is that, through its enduring bond, two persons
together can find other freedoms, such as expression, intimacy, and
spirituality. This is true for all persons, whatever their sexual
orientation. See Windsor, -\_\_\_ (slip op.). There is dignity in the
bond between two men or two women who seek to marry and in their
autonomy to make such profound choices. Cf. Loving (``{[}T{]}he freedom
to marry, or not marry, a person of another race resides with the
individual and cannot be infringed by the State'').

A second principle in this Court's jurisprudence is that the right to
marry is fundamental because it supports a two-person union unlike any
other in its importance to the committed individuals. This point was
central to Griswold v. Connecticut, which held the Constitution protects
the right of married couples to use contraception. 381 U. S.. Suggesting
that marriage is a right ``older than the Bill of Rights,'' Griswold
described marriage this way:

``Marriage is a coming together for better or for worse, hopefully
enduring, and intimate to the degree of being sacred. It is an
association that promotes a way of life, not causes; a harmony in
living, not political faiths; a bilateral loyalty, not commercial or
social projects. Yet it is an association for as noble a purpose as any
involved in our prior decisions.''

And in Turner, the Court again acknowledged the intimate association
protected by this right, holding prisoners could not be denied the right
to marry because their committed relationships satisfied the basic
reasons why marriage is a fundamental right. See 482 U. S.. The right to
marry thus dignifies couples who ``wish to define themselves by their
commitment to each other.'' Windsor\_\_\_ (slip op.). Marriage responds
to the universal fear that a lonely person might call out only to find
no one there. It offers the hope of companionship and understanding and
assurance that while both still live there will be someone to care for
the other.

As this Court held in Lawrence, same-sex couples have the same right as
opposite-sex couples to enjoy intimate association. Lawrence invalidated
laws that made same-sex intimacy a criminal act. And it acknowledged
that ``{[}w{]}hen sexuality finds overt expression in intimate conduct
with another person, the conduct can be but one element in a personal
bond that is more enduring.'' 539 U. S.. But while Lawrence confirmed a
dimension of freedom that allows individuals to engage in intimate
association without criminal liability, it does not follow that freedom
stops there. Outlaw to outcast may be a step forward, but it does not
achieve the full promise of liberty.

A third basis for protecting the right to marry is that it safeguards
children and families and thus draws meaning from related rights of
childrearing, procreation, and education. See Pierce v. Society of
Sisters; Meyer. The Court has recognized these connections by describing
the varied rights as a unified whole: ``{[}T{]}he right to'marry,
establish a home and bring up children' is a central part of the liberty
protected by the Due Process Clause.'' Zablocki (quoting Meyer). Under
the laws of the several States, some of marriage's protections for
children and families are material. But marriage also confers more
profound benefits. By giving recognition and legal structure to their
parents' relationship, marriage allows children ``to understand the
integrity and closeness of their own family and its concord with other
families in their community and in their daily lives.'' Windsor\_\_\_
(slip op.). Marriage also affords the permanency and stability important
to children's best interests. See Brief for Scholars of the
Constitutional Rights of Children as Amici Curiae 22-27.

As all parties agree, many same-sex couples provide loving and nurturing
homes to their children, whether biological or adopted. And hundreds of
thousands of children are presently being raised by such couples. See
Brief for Gary J. Gates as Amicus Curiae 4. Most States have allowed
gays and lesbians to adopt, either as individuals or as couples, and
many adopted and foster children have same-sex parents, see This
provides powerful confirmation from the law itself that gays and
lesbians can create loving, supportive families.

Excluding same-sex couples from marriage thus conflicts with a central
premise of the right to marry. Without the recognition, stability, and
predictability marriage offers, their children suffer the stigma of
knowing their families are somehow lesser. They also suffer the
significant material costs of being raised by unmarried parents,
relegated through no fault of their own to a more difficult and
uncertain family life. The marriage laws at issue here thus harm and
humiliate the children of same-sex couples. See Windsor\_\_\_ (slip
op.).

That is not to say the right to marry is less meaningful for those who
do not or cannot have children. An ability, desire, or promise to
procreate is not and has not been a prerequisite for a valid marriage in
any State. In light of precedent protecting the right of a married
couple not to procreate, it cannot be said the Court or the States have
conditioned the right to marry on the capacity or commitment to
procreate. The constitutional marriage right has many aspects, of which
childbearing is only one.

Fourth and finally, this Court's cases and the Nation's traditions make
clear that marriage is a keystone of our social order. Alexis de
Tocqueville recognized this truth on his travels through the United
States almost two centuries ago:

``There is certainly no country in the world where the tie of marriage
is so much respected as in America . {[}W{]}hen the American retires
from the turmoil of public life to the bosom of his family, he finds in
it the image of order and of peace. . {[}H{]}e afterwards carries
{[}that image{]} with him into public affairs.'' 1 Democracy in America
309 (H. Reeve transl., rev. ed.~1990).

In Maynard v. Hill, the Court echoed de Tocqueville, explaining that
marriage is ``the foundation of the family and of society, without which
there would be neither civilization nor progress.'' Marriage, the
MaynardCourt said, has long been ```a great public institution, giving
character to our whole civil polity.''' This idea has been reiterated
even as the institution has evolved in substantial ways over time,
superseding rules related to parental consent, gender, and race once
thought by many to be essential. See generally N. Cott, Public Vows.
Marriage remains a building block of our national community.

For that reason, just as a couple vows to support each other, so does
society pledge to support the couple, offering symbolic recognition and
material benefits to protect and nourish the union. Indeed, while the
States are in general free to vary the benefits they confer on all
married couples, they have throughout our history made marriage the
basis for an expanding list of governmental rights, benefits, and
responsibilities. These aspects of marital status include: taxation;
inheritance and property rights; rules of intestate succession; spousal
privilege in the law of evidence; hospital access; medical
decisionmaking authority; adoption rights; the rights and benefits of
survivors; birth and death certificates; professional ethics rules;
campaign finance restrictions; workers' compensation benefits; health
insurance; and child custody, support, and visitation rules. See Brief
for United States as Amicus Curiae 6-9; Brief for American Bar
Association as Amicus Curiae 8-29. Valid marriage under state law is
also a significant status for over a thousand provisions of federal law.
See Windsor, -\_\_\_ (slip op.). The States have contributed to the
fundamental character of the marriage right by placing that institution
at the center of so many facets of the legal and social order.

There is no difference between same- and opposite-sex couples with
respect to this principle. Yet by virtue of their exclusion from that
institution, same-sex couples are denied the constellation of benefits
that the States have linked to marriage. This harm results in more than
just material burdens. Same-sex couples are consigned to an instability
many opposite-sex couples would deem intolerable in their own lives. As
the State itself makes marriage all the more precious by the
significance it attaches to it, exclusion from that status has the
effect of teaching that gays and lesbians are unequal in important
respects. It demeans gays and lesbians for the State to lock them out of
a central institution of the Nation's society. Same-sex couples, too,
may aspire to the transcendent purposes of marriage and seek fulfillment
in its highest meaning.

The limitation of marriage to opposite-sex couples may long have seemed
natural and just, but its inconsistency with the central meaning of the
fundamental right to marry is now manifest. With that knowledge must
come the recognition that laws excluding same-sex couples from the
marriage right impose stigma and injury of the kind prohibited by our
basic charter.

Objecting that this does not reflect an appropriate framing of the
issue, the respondents refer to Washington v. Glucksberg, which called
for a ```careful description''' of fundamental rights. They assert the
petitioners do not seek to exercise the right to marry but rather a new
and nonexistent ``right to same-sex marriage.'' Brief for Respondent in
No.~14-556, p.~8. Glucksberg did insist that liberty under the Due
Process Clause must be defined in a most circumscribed manner, with
central reference to specific historical practices. Yet while that
approach may have been appropriate for the asserted right there involved
(physician-assisted suicide), it is inconsistent with the approach this
Court has used in discussing other fundamental rights, including
marriage and intimacy. Loving did not ask about a ``right to interracial
marriage''; Turner did not ask about a ``right of inmates to marry'';
and Zablocki did not ask about a ``right of fathers with unpaid child
support duties to marry.'' Rather, each case inquired about the right to
marry in its comprehensive sense, asking if there was a sufficient
justification for excluding the relevant class from the right. See also
Glucksberg (Souter, J., concurring in judgment); -792 (BREYER, J.,
concurring in judgments).

That principle applies here. If rights were defined by who exercised
them in the past, then received practices could serve as their own
continued justification and new groups could not invoke rights once
denied. This Court has rejected that approach, both with respect to the
right to marry and the rights of gays and lesbians. See Loving 388 U.
S.; Lawrence.

The right to marry is fundamental as a matter of history and tradition,
but rights come not from ancient sources alone. They rise, too, from a
better informed understanding of how constitutional imperatives define a
liberty that remains urgent in our own era. Many who deem same-sex
marriage to be wrong reach that conclusion based on decent and honorable
religious or philosophical premises, and neither they nor their beliefs
are disparaged here. But when that sincere, personal opposition becomes
enacted law and public policy, the necessary consequence is to put the
imprimatur of the State itself on an exclusion that soon demeans or
stigmatizes those whose own liberty is then denied. Under the
Constitution, same-sex couples seek in marriage the same legal treatment
as opposite-sex couples, and it would disparage their choices and
diminish their personhood to deny them this right.

The right of same-sex couples to marry that is part of the liberty
promised by the Fourteenth Amendment is derived, too, from that
Amendment's guarantee of the equal protection of the laws. The Due
Process Clause and the Equal Protection Clause are connected in a
profound way, though they set forth independent principles. Rights
implicit in liberty and rights secured by equal protection may rest on
different precepts and are not always coextensive, yet in some instances
each may be instructive as to the meaning and reach of the other. In any
particular case one Clause may be thought to capture the essence of the
right in a more accurate and comprehensive way, even as the two Clauses
may converge in the identification and definition of the right. See M.
L. B.; -129 (KENNEDY, J., concurring in judgment); Bearden v. Georgia.
This interrelation of the two principles furthers our understanding of
what freedom is and must become.

The Court's cases touching upon the right to marry reflect this dynamic.
In Lovingthe Court invalidated a prohibition on interracial marriage
under both the Equal Protection Clause and the Due Process Clause. The
Court first declared the prohibition invalid because of its unequal
treatment of interracial couples. It stated: ``There can be no doubt
that restricting the freedom to marry solely because of racial
classifications violates the central meaning of the Equal Protection
Clause.'' 388 U. S.. With this link to equal protection the Court
proceeded to hold the prohibition offended central precepts of liberty:
``To deny this fundamental freedom on so unsupportable a basis as the
racial classifications embodied in these statutes, classifications so
directly subversive of the principle of equality at the heart of the
Fourteenth Amendment, is surely to deprive all the State's citizens of
liberty without due process of law.'' The reasons why marriage is a
fundamental right became more clear and compelling from a full awareness
and understanding of the hurt that resulted from laws barring
interracial unions.

The synergy between the two protections is illustrated further in
Zablocki. There the Court invoked the Equal Protection Clause as its
basis for invalidating the challenged law, which, as already noted,
barred fathers who were behind on child-support payments from marrying
without judicial approval. The equal protection analysis depended in
central part on the Court's holding that the law burdened a right ``of
fundamental importance.'' 434 U. S.. It was the essential nature of the
marriage right, discussed at length in Zablocki, see -387, that made
apparent the law's incompatibility with requirements of equality. Each
concept---liberty and equal protection---leads to a stronger
understanding of the other.

Indeed, in interpreting the Equal Protection Clause, the Court has
recognized that new insights and societal understandings can reveal
unjustified inequality within our most fundamental institutions that
once passed unnoticed and unchallenged. To take but one period, this
occurred with respect to marriage in the 1970's and 1980's.
Notwithstanding the gradual erosion of the doctrine of coverture, see,
invidious sex-based classifications in marriage remained common through
the mid-20th century. See App. to Brief for Appellant in Reed v. Reed,
O. T. 1971, No.~70-4, pp.~69-88 (an extensive reference to laws extant
as of 1971 treating women as unequal to men in marriage). These
classifications denied the equal dignity of men and women. One State's
law, for example, provided in 1971 that ``the husband is the head of the
family and the wife is subject to him; her legal civil existence is
merged in the husband, except so far as the law recognizes her
separately, either for her own protection, or for her benefit.'' Ga.
Code Ann. §53-501 (1935). Responding to a new awareness, the Court
invoked equal protection principles to invalidate laws imposing
sex-based inequality on marriage. See, e.g., Kirchberg v. Feenstra;
Wengler v. Druggists Mut. Ins. Co.,446 U. S. 142 (1980); Califano v.
Westcott; Orr v. Orr; Califano v. Goldfarb (plurality opinion);
Weinberger v. Wiesenfeld; Frontiero v. Richardson. Like Loving and
Zablocki, these precedents show the Equal Protection Clause can help to
identify and correct inequalities in the institution of marriage,
vindicating precepts of liberty and equality under the Constitution.

Other cases confirm this relation between liberty and equality. In M. L.
B. v. S. L. J., the Court invalidated under due process and equal
protection principles a statute requiring indigent mothers to pay a fee
in order to appeal the termination of their parental rights. See 519 U.
S.. In Eisenstadt v. Baird, the Court invoked both principles to
invalidate a prohibition on the distribution of contraceptives to
unmarried persons but not married persons. See 405 U. S.. And in Skinner
v. Oklahoma ex rel. Williamson, the Court invalidated under both
principles a law that allowed sterilization of habitual criminals. See
316 U. S..

In Lawrence the Court acknowledged the interlocking nature of these
constitutional safeguards in the context of the legal treatment of gays
and lesbians. See 539 U. S.. Although Lawrence elaborated its holding
under the Due Process Clause, it acknowledged, and sought to remedy, the
continuing inequality that resulted from laws making intimacy in the
lives of gays and lesbians a crime against the State. See Lawrence
therefore drew upon principles of liberty and equality to define and
protect the rights of gays and lesbians, holding the State ``cannot
demean their existence or control their destiny by making their private
sexual conduct a crime.''

This dynamic also applies to same-sex marriage. It is now clear that the
challenged laws burden the liberty of same-sex couples, and it must be
further acknowledged that they abridge central precepts of equality.
Here the marriage laws enforced by the respondents are in essence
unequal: same-sex couples are denied all the benefits afforded to
opposite-sex couples and are barred from exercising a fundamental right.
Especially against a long history of disapproval of their relationships,
this denial to same-sex couples of the right to marry works a grave and
continuing harm. The imposition of this disability on gays and lesbians
serves to disrespect and subordinate them. And the Equal Protection
Clause, like the Due Process Clause, prohibits this unjustified
infringement of the fundamental right to marry. See, e.g., Zablocki;
Skinner.

These considerations lead to the conclusion that the right to marry is a
fundamental right inherent in the liberty of the person, and under the
Due Process and Equal Protection Clauses of the Fourteenth Amendment
couples of the same-sex may not be deprived of that right and that
liberty. The Court now holds that same-sex couples may exercise the
fundamental right to marry. No longer may this liberty be denied to
them. Baker v. Nelson must be and now is overruled, and the State laws
challenged by Petitioners in these cases are now held invalid to the
extent they exclude same-sex couples from civil marriage on the same
terms and conditions as opposite-sex couples.

There may be an initial inclination in these cases to proceed with
caution---to await further legislation, litigation, and debate. The
respondents warn there has been insufficient democratic discourse before
deciding an issue so basic as the definition of marriage. In its ruling
on the cases now before this Court, the majority opinion for the Court
of Appeals made a cogent argument that it would be appropriate for the
respondents' States to await further public discussion and political
measures before licensing same-sex marriages. See DeBoer, 772 F. 3d.

Yet there has been far more deliberation than this argument
acknowledges. There have been referenda, legislative debates, and
grassroots campaigns, as well as countless studies, papers, books, and
other popular and scholarly writings. There has been extensive
litigation in state and federal courts. See Appendix A, infra.Judicial
opinions addressing the issue have been informed by the contentions of
parties and counsel, which, in turn, reflect the more general, societal
discussion of same-sex marriage and its meaning that has occurred over
the past decades. As more than 100 amici make clear in their filings,
many of the central institutions in American life---state and local
governments, the military, large and small businesses, labor unions,
religious organizations, law enforcement, civic groups, professional
organizations, and universities--- have devoted substantial attention to
the question. This has led to an enhanced understanding of the
issue---an understanding reflected in the arguments now presented for
resolution as a matter of constitutional law.

Of course, the Constitution contemplates that democracy is the
appropriate process for change, so long as that process does not abridge
fundamental rights. Last Term, a plurality of this Court reaffirmed the
importance of the democratic principle in Schuette v. BAMN, 572 U. S.
\_\_\_ (2014), noting the ``right of citizens to debate so they can
learn and decide and then, through the political process, act in concert
to try to shape the course of their own times.'' \textbf{\emph{-}} (slip
op.). Indeed, it is most often through democracy that liberty is
preserved and protected in our lives. But as Schuette also said,
``{[}t{]}he freedom secured by the Constitution consists, in one of its
essential dimensions, of the right of the individual not to be injured
by the unlawful exercise of governmental power.'' \_\_\_ (slip op.).
Thus, when the rights of persons are violated, ``the Constitution
requires redress by the courts,'' notwithstanding the more general value
of democratic decisionmaking. \_\_\_ (slip op.). This holds true even
when protecting individual rights affects issues of the utmost
importance and sensitivity.

The dynamic of our constitutional system is that individuals need not
await legislative action before asserting a fundamental right. The
Nation's courts are open to injured individuals who come to them to
vindicate their own direct, personal stake in our basic charter. An
individual can invoke a right to constitutional protection when he or
she is harmed, even if the broader public disagrees and even if the
legislature refuses to act. The idea of the Constitution ``was to
withdraw certain subjects from the vicissitudes of political
controversy, to place them beyond the reach of majorities and officials
and to establish them as legal principles to be applied by the courts.''
West Virginia Bd. of Ed. v. Barnette,319 U. S. 624, 638 (1943). This is
why ``fundamental rights may not be submitted to a vote; they depend on
the outcome of no elections.'' It is of no moment whether advocates of
same-sex marriage now enjoy or lack momentum in the democratic process.
The issue before the Court here is the legal question whether the
Constitution protects the right of same-sex couples to marry.

This is not the first time the Court has been asked to adopt a cautious
approach to recognizing and protecting fundamental rights. In Bowers, a
bare majority upheld a law criminalizing same-sex intimacy. See 478 U.
S., 190-195. That approach might have been viewed as a cautious
endorsement of the democratic process, which had only just begun to
consider the rights of gays and lesbians. Yet, in effect, Bowers upheld
state action that denied gays and lesbians a fundamental right and
caused them pain and humiliation. As evidenced by the dissents in that
case, the facts and principles necessary to a correct holding were known
to the Bowers Court. That is why Lawrence held Bowers was ``not correct
when it was decided.'' Although Bowers was eventually repudiated in
Lawrence, men and women were harmed in the interim, and the substantial
effects of these injuries no doubt lingered long after Bowers was
overruled. Dignitary wounds cannot always be healed with the stroke of a
pen.

A ruling against same-sex couples would have the same effect---and, like
Bowers,would be unjustified under the Fourteenth Amendment. The
petitioners' stories make clear the urgency of the issue they present to
the Court. James Obergefell now asks whether Ohio can erase his marriage
to John Arthur for all time. April DeBoer and Jayne Rowse now ask
whether Michigan may continue to deny them the certainty and stability
all mothers desire to protect their children, and for them and their
children the childhood years will pass all too soon. Ijpe DeKoe and
Thomas Kostura now ask whether Tennessee can deny to one who has served
this Nation the basic dignity of recognizing his New York marriage.
Properly presented with the petitioners' cases, the Court has a duty to
address these claims and answer these questions.

Indeed, faced with a disagreement among the Courts of Appeals---a
disagreement that caused impermissible geographic variation in the
meaning of federal law---the Court granted review to determine whether
same-sex couples may exercise the right to marry. Were the Court to
uphold the challenged laws as constitutional, it would teach the Nation
that these laws are in accord with our society's most basic compact.
Were the Court to stay its hand to allow slower, case-by-case
determination of the required availability of specific public benefits
to same-sex couples, it still would deny gays and lesbians many rights
and responsibilities intertwined with marriage.

The respondents also argue allowing same-sex couples to wed will harm
marriage as an institution by leading to fewer opposite-sex marriages.
This may occur, the respondents contend, because licensing same-sex
marriage severs the connection between natural procreation and marriage.
That argument, however, rests on a counterintuitive view of opposite-sex
couple's decisionmaking processes regarding marriage and parenthood.
Decisions about whether to marry and raise children are based on many
personal, romantic, and practical considerations; and it is unrealistic
to conclude that an opposite-sex couple would choose not to marry simply
because same-sex couples may do so. See Kitchen v. Herbert, CA10 2014)
(``{[}I{]}t is wholly illogical to believe that state recognition of the
love and commitment between same-sex couples will alter the most
intimate and personal decisions of opposite-sex couples''). The
respondents have not shown a foundation for the conclusion that allowing
same-sex marriage will cause the harmful outcomes they describe. Indeed,
with respect to this asserted basis for excluding same-sex couples from
the right to marry, it is appropriate to observe these cases involve
only the rights of two consenting adults whose marriages would pose no
risk of harm to themselves or third parties.

Finally, it must be emphasized that religions, and those who adhere to
religious doctrines, may continue to advocate with utmost, sincere
conviction that, by divine precepts, same-sex marriage should not be
condoned. The First Amendment ensures that religious organizations and
persons are given proper protection as they seek to teach the principles
that are so fulfilling and so central to their lives and faiths, and to
their own deep aspirations to continue the family structure they have
long revered. The same is true of those who oppose same-sex marriage for
other reasons. In turn, those who believe allowing same-sex marriage is
proper or indeed essential, whether as a matter of religious conviction
or secular belief, may engage those who disagree with their view in an
open and searching debate. The Constitution, however, does not permit
the State to bar same-sex couples from marriage on the same terms as
accorded to couples of the opposite sex.

These cases also present the question whether the Constitution requires
States to recognize same-sex marriages validly performed out of State.
As made clear by the case of Obergefell and Arthur, and by that of DeKoe
and Kostura, the recognition bans inflict substantial and continuing
harm on same-sex couples.

Being married in one State but having that valid marriage denied in
another is one of ``the most perplexing and distressing
complication{[}s{]}'' in the law of domestic relations. Williams v.
North Carolina (internal quotation marks omitted). Leaving the current
state of affairs in place would maintain and promote instability and
uncertainty. For some couples, even an ordinary drive into a neighboring
State to visit family or friends risks causing severe hardship in the
event of a spouse's hospitalization while across state lines. In light
of the fact that many States already allow same-sex marriage---and
hundreds of thousands of these marriages already have occurred---the
disruption caused by the recognition bans is significant and
ever-growing.

As counsel for the respondents acknowledged at argument, if States are
required by the Constitution to issue marriage licenses to same-sex
couples, the justifications for refusing to recognize those marriages
performed elsewhere are undermined. See Tr. of Oral Arg. on Question 2,
p.~44. The Court, in this decision, holds same-sex couples may exercise
the fundamental right to marry in all States. It follows that the Court
also must hold---and it now does hold---that there is no lawful basis
for a State to refuse to recognize a lawful same-sex marriage performed
in another State on the ground of its same-sex character.

No union is more profound than marriage, for it embodies the highest
ideals of love, fidelity, devotion, sacrifice, and family. In forming a
marital union, two people become something greater than once they were.
As some of the petitioners in these cases demonstrate, marriage embodies
a love that may endure even past death. It would misunderstand these men
and women to say they disrespect the idea of marriage. Their plea is
that they do respect it, respect it so deeply that they seek to find its
fulfillment for themselves. Their hope is not to be condemned to live in
loneliness, excluded from one of civilization's oldest institutions.
They ask for equal dignity in the eyes of the law. The Constitution
grants them that right.

The judgment of the Court of Appeals for the Sixth Circuit is reversed.

\textbf{CHIEF JUSTICE ROBERTS, with whom JUSTICE SCALIA and JUSTICE
THOMAS join, dissenting.} Petitioners make strong arguments rooted in
social policy and considerations of fairness. They contend that same-sex
couples should be allowed to affirm their love and commitment through
marriage, just like opposite-sex couples. That position has undeniable
appeal; over the past six years, voters and legislators in eleven States
and the District of Columbia have revised their laws to allow marriage
between two people of the same sex.

But this Court is not a legislature. Whether same-sex marriage is a good
idea should be of no concern to us. Under the Constitution, judges have
power to say what the law is, not what it should be. The people who
ratified the Constitution authorized courts to exercise ``neither force
nor will but merely judgment.'' The Federalist No.~78, p.~465 (C.
Rossiter ed.~1961) (A. Hamilton) (capitalization altered).

Although the policy arguments for extending marriage to same-sex couples
may be compelling, the legal arguments for requiring such an extension
are not. The fundamental right to marry does not include a right to make
a State change its definition of marriage. And a State's decision to
maintain the meaning of marriage that has persisted in every culture
throughout human history can hardly be called irrational. In short, our
Constitution does not enact any one theory of marriage. The people of a
State are free to expand marriage to include same-sex couples, or to
retain the historic definition.

Today, however, the Court takes the extraordinary step of ordering every
State to license and recognize same-sex marriage. Many people will
rejoice at this decision, and I begrudge none their celebration. But for
those who believe in a government of laws, not of men, the majority's
approach is deeply disheartening. Supporters of same-sex marriage have
achieved considerable success persuading their fellow citizens---through
the democratic process---to adopt their view. That ends today. Five
lawyers have closed the debate and enacted their own vision of marriage
as a matter of constitutional law. Stealing this issue from the people
will for many cast a cloud over same-sex marriage, making a dramatic
social change that much more difficult to accept.

The majority's decision is an act of will, not legal judgment. The right
it announces has no basis in the Constitution or this Court's precedent.
The majority expressly disclaims judicial ``caution'' and omits even a
pretense of humility, openly relying on its desire to remake society
according to its own ``new insight'' into the ``nature of injustice.''
As a result, the Court invalidates the marriage laws of more than half
the States and orders the transformation of a social institution that
has formed the basis of human society for millennia, for the Kalahari
Bushmen and the Han Chinese, the Carthaginians and the Aztecs. Just who
do we think we are?

It can be tempting for judges to confuse our own preferences with the
requirements of the law. But as this Court has been reminded throughout
our history, the Constitution ``is made for people of fundamentally
differing views.'' Lochner v. New York (Holmes, J., dissenting).
Accordingly, ``courts are not concerned with the wisdom or policy of
legislation.'' Id.,at 69 (Harlan, J., dissenting). The majority today
neglects that restrained conception of the judicial role. It seizes for
itself a question the Constitution leaves to the people, at a time when
the people are engaged in a vibrant debate on that question. And it
answers that question based not on neutral principles of constitutional
law, but on its own ``understanding of what freedom is and must
become.'' I have no choice but to dissent.

Understand well what this dissent is about: It is not about whether, in
my judgment, the institution of marriage should be changed to include
same-sex couples. It is instead about whether, in our democratic
republic, that decision should rest with the people acting through their
elected representatives, or with five lawyers who happen to hold
commissions authorizing them to resolve legal disputes according to law.
The Constitution leaves no doubt about the answer.

Petitioners and their amici base their arguments on the ``right to
marry'' and the imperative of ``marriage equality.'' There is no serious
dispute that, under our precedents, the Constitution protects a right to
marry and requires States to apply their marriage laws equally. The real
question in these cases is what constitutes ``marriage,'' or---more
precisely---who decides what constitutes ``marriage''?

The majority largely ignores these questions, relegating ages of human
experience with marriage to a paragraph or two. Even if history and
precedent are not ``the end'' of these cases, I would not ``sweep away
what has so long been settled'' without showing greater respect for all
that preceded us. Town of Greecev. Galloway, 572 U. S. \textbf{\emph{,
}} (2014) (slip op.).

As the majority acknowledges, marriage ``has existed for millennia and
across civilizations.'' For all those millennia, across all those
civilizations, ``marriage'' referred to only one relationship: the union
of a man and a woman. See ; Tr. of Oral Arg. on Question 1, p.~12
(petitioners conceding that they are not aware of any society that
permitted same-sex marriage before 2001). As the Court explained two
Terms ago, ``until recent years, . marriage between a man and a woman no
doubt had been thought of by most people as essential to the very
definition of that term and to its role and function throughout the
history of civilization.'' United States v. Windsor, 570 U. S.
\textbf{\emph{, }} (2013) (slip op.).

This universal definition of marriage as the union of a man and a woman
is no historical coincidence. Marriage did not come about as a result of
a political movement, discovery, disease, war, religious doctrine, or
any other moving force of world history---and certainly not as a result
of a prehistoric decision to exclude gays and lesbians. It arose in the
nature of things to meet a vital need: ensuring that children are
conceived by a mother and father committed to raising them in the stable
conditions of a lifelong relationship. See G. Quale, A History of
Marriage Systems 2 (1988); cf.~M. Cicero, De Officiis 57 (W. Miller
transl. 1913) (``For since the reproductive instinct is by nature's gift
the common possession of all living creatures, the first bond of union
is that between husband and wife; the next, that between parents and
children; then we find one home, with everything in common.'').

The premises supporting this concept of marriage are so fundamental that
they rarely require articulation. The human race must procreate to
survive. Procreation occurs through sexual relations between a man and a
woman. When sexual relations result in the conception of a child, that
child's prospects are generally better if the mother and father stay
together rather than going their separate ways. Therefore, for the good
of children and society, sexual relations that can lead to procreation
should occur only between a man and a woman committed to a lasting bond.

Society has recognized that bond as marriage. And by bestowing a
respected status and material benefits on married couples, society
encourages men and women to conduct sexual relations within marriage
rather than without. As one prominent scholar put it, ``Marriage is a
socially arranged solution for the problem of getting people to stay
together and care for children that the mere desire for children, and
the sex that makes children possible, does not solve.'' J. Q. Wilson,
The Marriage Problem 41 (2002).

This singular understanding of marriage has prevailed in the United
States throughout our history. The majority accepts that at ``the time
of the Nation's founding {[}marriage{]} was understood to be a voluntary
contract between a man and a woman.'' Early Americans drew heavily on
legal scholars like William Blackstone, who regarded marriage between
``husband and wife'' as one of the ``great relations in private life,''
and philosophers like John Locke, who described marriage as ``a
voluntary compact between man and woman'' centered on ``its chief end,
procreation'' and the ``nourishment and support'' of children. 1 W.
Blackstone, Commentaries ; J. Locke, Second Treatise of Civil Government
§§78-79, p.~39 (J. Gough ed.~1947). To those who drafted and ratified
the Constitution, this conception of marriage and family ``was a given:
its structure, its stability, roles, and values accepted by all.''
Forte, The Framers' Idea of Marriage and Family, in The Meaning of
Marriage 100, 102 (R. George \& J. Elshtain eds.~2006).

The Constitution itself says nothing about marriage, and the Framers
thereby entrusted the States with ``{[}t{]}he whole subject of the
domestic relations of husband and wife.'' Windsor, (slip op.) (quoting
In re Burrus (1890)). There is no dispute that every State at the
founding---and every State throughout our history until a dozen years
ago---defined marriage in the traditional, biologically rooted way. The
four States in these cases are typical. Their laws, before and after
statehood, have treated marriage as the union of a man and a woman. See
DeBoer v. Snyder, -399 (CA6 2014). Even when state laws did not specify
this definition expressly, no one doubted what they meant. See Jones v.
Hallahan, 501 S. W. 2d 588, 589 (Ky. App. 1973). The meaning of
``marriage'' went without saying.

Of course, many did say it. In his first American dictionary, Noah
Webster defined marriage as ``the legal union of a man and woman for
life,'' which served the purposes of ``preventing the promiscuous
intercourse of the sexes, . promoting domestic felicity, and . securing
the maintenance and education of children.'' 1 An American Dictionary of
the English Language (1828). An influential 19th-century treatise
defined marriage as ``a civil status, existing in one man and one woman
legally united for life for those civil and social purposes which are
based in the distinction of sex.'' J. Bishop, Commentaries on the Law of
Marriage and Divorce 25 (1852). The first edition of Black's Law
Dictionary defined marriage as ``the civil status of one man and one
woman united in law for life.'' Black's Law Dictionary 756 (1891)
(emphasis deleted). The dictionary maintained essentially that same
definition for the next century.

This Court's precedents have repeatedly described marriage in ways that
are consistent only with its traditional meaning. Early cases on the
subject referred to marriage as ``the union for life of one man and one
woman,'' Murphy v. Ramsey,114 U. S. 15, 45 (1885), which forms ``the
foundation of the family and of society, without which there would be
neither civilization nor progress,'' Maynard v. Hill. We later described
marriage as ``fundamental to our very existence and survival,'' an
understanding that necessarily implies a procreative component. Loving
v. Virginia; see Skinner v. Oklahoma ex rel. Williamson. More recent
cases have directly connected the right to marry with the ``right to
procreate.'' Zablocki v. Redhail.

As the majority notes, some aspects of marriage have changed over time.
Arranged marriages have largely given way to pairings based on romantic
love. States have replaced coverture, the doctrine by which a married
man and woman became a single legal entity, with laws that respect each
participant's separate status. Racial restrictions on marriage, which
``arose as an incident to slavery'' to promote ``White Supremacy,'' were
repealed by many States and ultimately struck down by this Court.
Loving.

The majority observes that these developments ``were not mere
superficial changes'' in marriage, but rather ``worked deep
transformations in its structure.'' They did not, however, work any
transformation in the core structure of marriage as the union between a
man and a woman. If you had asked a person on the street how marriage
was defined, no one would ever have said, ``Marriage is the union of a
man and a woman, where the woman is subject to coverture.'' The majority
may be right that the ``history of marriage is one of both continuity
and change,'' but the core meaning of marriage has endured. Shortly
after this Court struck down racial restrictions on marriage in Loving,
a gay couple in Minnesota sought a marriage license. They argued that
the Constitution required States to allow marriage between people of the
same sex for the same reasons that it requires States to allow marriage
between people of different races. The Minnesota Supreme Court rejected
their analogy to Loving, and this Court summarily dismissed an appeal.
Baker v. Nelson.

In the decades after Baker, greater numbers of gays and lesbians began
living openly, and many expressed a desire to have their relationships
recognized as marriages. Over time, more people came to see marriage in
a way that could be extended to such couples. Until recently, this new
view of marriage remained a minority position. After the Massachusetts
Supreme Judicial Court in 2003 interpreted its State Constitution to
require recognition of same-sex marriage, many States---including the
four at issue here---enacted constitutional amendments formally adopting
the longstanding definition of marriage.

Over the last few years, public opinion on marriage has shifted rapidly.
In 2009, the legislatures of Vermont, New Hampshire, and the District of
Columbia became the first in the Nation to enact laws that revised the
definition of marriage to include same-sex couples, while also providing
accommodations for religious believers. In 2011, the New York
Legislature enacted a similar law. In 2012, voters in Maine did the
same, reversing the result of a referendum just three years earlier in
which they had upheld the traditional definition of marriage.

In all, voters and legislators in eleven States and the District of
Columbia have changed their definitions of marriage to include same-sex
couples. The highest courts of five States have decreed that same result
under their own Constitutions. The remainder of the States retain the
traditional definition of marriage.

Petitioners brought lawsuits contending that the Due Process and Equal
Protection Clauses of the Fourteenth Amendment compel their States to
license and recognize marriages between same-sex couples. In a carefully
reasoned decision, the Court of Appeals acknowledged the democratic
``momentum'' in favor of ``expand{[}ing{]} the definition of marriage to
include gay couples,'' but concluded that petitioners had not made ``the
case for constitutionalizing the definition of marriage and for removing
the issue from the place it has been since the founding: in the hands of
state voters.''. That decision interpreted the Constitution correctly,
and I would affirm.

Petitioners first contend that the marriage laws of their States violate
the Due Process Clause. The Solicitor General of the United States,
appearing in support of petitioners, expressly disowned that position
before this Court. See Tr. of Oral Arg. on Question 1. The majority
nevertheless resolves these cases for petitioners based almost entirely
on the Due Process Clause.

The majority purports to identify four ``principles and traditions'' in
this Court's due process precedents that support a fundamental right for
same-sex couples to marry. In reality, however, the majority's approach
has no basis in principle or tradition, except for the unprincipled
tradition of judicial policymaking that characterized discredited
decisions such as Lochner v. New York. Stripped of its shiny rhetorical
gloss, the majority's argument is that the Due Process Clause gives
same-sex couples a fundamental right to marry because it will be good
for them and for society. If I were a legislator, I would certainly
consider that view as a matter of social policy. But as a judge, I find
the majority's position indefensible as a matter of constitutional law.

Petitioners'``fundamental right'' claim falls into the most sensitive
category of constitutional adjudication. Petitioners do not contend that
their States' marriage laws violate an enumerated constitutional right,
such as the freedom of speech protected by the First Amendment. There
is, after all, no ``Companionship and Understanding'' or ``Nobility and
Dignity'' Clause in the Constitution. See They argue instead that the
laws violate a right implied by the Fourteenth Amendment's requirement
that ``liberty'' may not be deprived without ``due process of law.''

This Court has interpreted the Due Process Clause to include a
``substantive'' component that protects certain liberty interests
against state deprivation ``no matter what process is provided.'' Reno
v. Flores. The theory is that some liberties are ``so rooted in the
traditions and conscience of our people as to be ranked as
fundamental,'' and therefore cannot be deprived without compelling
justification. Snyder v. Massachusetts.

Allowing unelected federal judges to select which unenumerated rights
rank as ``fundamental''---and to strike down state laws on the basis of
that determination---raises obvious concerns about the judicial role.
Our precedents have accordingly insisted that judges ``exercise the
utmost care'' in identifying implied fundamental rights, ``lest the
liberty protected by the Due Process Clause be subtly transformed into
the policy preferences of the Members of this Court.'' Washington v.
Glucksberg (internal quotation marks omitted); see Kennedy, Unenumerated
Rights and the Dictates of Judicial Restraint 13 (1986) (Address at
Stanford) (``One can conclude that certain essential, or fundamental,
rights should exist in any just society. It does not follow that each of
those essential rights is one that we as judges can enforce under the
written Constitution. The Due Process Clause is not a guarantee of every
right that should inhere in an ideal system.'').

The need for restraint in administering the strong medicine of
substantive due process is a lesson this Court has learned the hard way.
The Court first applied substantive due process to strike down a statute
in Dred Scott v. Sandford, 19 How. 393 (1857). There the Court
invalidated the Missouri Compromise on the ground that legislation
restricting the institution of slavery violated the implied rights of
slaveholders. The Court relied on its own conception of liberty and
property in doing so. It asserted that ``an act of Congress which
deprives a citizen of the United States of his liberty or property,
merely because he came himself or brought his property into a particular
Territory of the United States . could hardly be dignified with the name
of due process of law.'' In a dissent that has outlasted the majority
opinion, Justice Curtis explained that when the ``fixed rules which
govern the interpretation of laws {[}are{]} abandoned, and the
theoretical opinions of individuals are allowed to control'' the
Constitution's meaning, ``we have no longer a Constitution; we are under
the government of individual men, who for the time being have power to
declare what the Constitution is, according to their own views of what
it ought to mean.''

Dred Scott's holding was overruled on the battlefields of the Civil War
and by constitutional amendment after Appomattox, but its approach to
the Due Process Clause reappeared. In a series of early 20th-century
cases, most prominently Lochner v. New York, this Court invalidated
state statutes that presented ``meddlesome interferences with the rights
of the individual,'' and ``undue interference with liberty of person and
freedom of contract.'' 198 U. S., 61. In Lochner itself, the Court
struck down a New York law setting maximum hours for bakery employees,
because there was ``in our judgment, no reasonable foundation for
holding this to be necessary or appropriate as a health law.''

The dissenting Justices in Lochner explained that the New York law could
be viewed as a reasonable response to legislative concern about the
health of bakery employees, an issue on which there was at least ``room
for debate and for an honest difference of opinion.'' (opinion of
Harlan, J.). The majority's contrary conclusion required adopting as
constitutional law ``an economic theory which a large part of the
country does not entertain.'' (opinion of Holmes, J.). As Justice Holmes
memorably put it, ``The Fourteenth Amendment does not enact Mr.~Herbert
Spencer's Social Statics,'' a leading work on the philosophy of Social
Darwinism. The Constitution ``is not intended to embody a particular
economic theory. . It is made for people of fundamentally differing
views, and the accident of our finding certain opinions natural and
familiar or novel and even shocking ought not to conclude our judgment
upon the question whether statutes embodying them conflict with the
Constitution.'' -76.

In the decades after Lochner, the Court struck down nearly 200 laws as
violations of individual liberty, often over strong dissents contending
that ``{[}t{]}he criterion of constitutionality is not whether we
believe the law to be for the public good.'' Adkins v. Children's
Hospital of D. C. (opinion of Holmes, J.). By empowering judges to
elevate their own policy judgments to the status of constitutionally
protected ``liberty,'' the Lochner line of cases left ``no alternative
to regarding the court as a . legislative chamber.'' L. Hand, The Bill
of Rights 42 (1958).

Eventually, the Court recognized its error and vowed not to repeat it.
``The doctrine that . due process authorizes courts to hold laws
unconstitutional when they believe the legislature has acted unwisely,''
we later explained, ``has long since been discarded. We have returned to
the original constitutional proposition that courts do not substitute
their social and economic beliefs for the judgment of legislative
bodies, who are elected to pass laws.'' Ferguson v. Skrupa; see
Day-Brite Lighting, Inc.~v. Missouri (``we do not sit as a
super-legislature to weigh the wisdom of legislation''). Thus, it has
become an accepted rule that the Court will not hold laws
unconstitutional simply because we find them ``unwise, improvident, or
out of harmony with a particular school of thought.'' Williamson v. Lee
Optical of Okla., Inc..

Rejecting Lochner does not require disavowing the doctrine of implied
fundamental rights, and this Court has not done so. But to avoid
repeating Lochner's error of converting personal preferences into
constitutional mandates, our modern substantive due process cases have
stressed the need for ``judicial self-restraint.'' Collins v. Harker
Heights. Our precedents have required that implied fundamental rights be
``objectively, deeply rooted in this Nation's history and tradition,''
and ``implicit in the concept of ordered liberty, such that neither
liberty nor justice would exist if they were sacrificed.'' Glucksberg
(internal quotation marks omitted).

Although the Court articulated the importance of history and tradition
to the fundamental rights inquiry most precisely in Glucksberg, many
other cases both before and after have adopted the same approach. See,
e.g., Moore v. East Cleveland (White, J., dissenting) (``The Judiciary,
including this Court, is the most vulnerable and comes nearest to
illegitimacy when it deals with judge-made constitutional law having
little or no cognizable roots in the language or even the design of the
Constitution.''); Troxel v. Granville (2000) (KENNEDY, J., dissenting)
(consulting ```{[}o{]}ur Nation's history, legal traditions, and
practices''' and concluding that ``{[}w{]}e owe it to the Nation's
domestic relations legal structure to proceed with caution'' (quoting
Glucksberg)).

Proper reliance on history and tradition of course requires looking
beyond the individual law being challenged, so that every restriction on
liberty does not supply its own constitutional justification. The Court
is right about that. But given the few ``guideposts for responsible
decisionmaking in this unchartered area,'' Collins, ``an approach
grounded in history imposes limits on the judiciary that are more
meaningful than any based on {[}an{]} abstract formula,'' Moore, n.~12
(plurality opinion). Expanding a right suddenly and dramatically is
likely to require tearing it up from its roots. Even a sincere
profession of ``discipline'' in identifying fundamental rights, does not
provide a meaningful constraint on a judge, for ``what he is really
likely to be'discovering,' whether or not he is fully aware of it, are
his own values,'' J. Ely, Democracy and Distrust 44 (1980). The only way
to ensure restraint in this delicate enterprise is ``continual
insistence upon respect for the teachings of history, solid recognition
of the basic values that underlie our society, and wise appreciation of
the great roles {[}of{]} the doctrines of federalism and separation of
powers.'' Griswoldv. Connecticut (Harlan, J., concurring in judgment).

The majority acknowledges none of this doctrinal background, and it is
easy to see why: Its aggressive application of substantive due process
breaks sharply with decades of precedent and returns the Court to the
unprincipled approach of Lochner.

The majority's driving themes are that marriage is desirable and
petitioners desire it. The opinion describes the ``transcendent
importance'' of marriage and repeatedly insists that petitioners do not
seek to ``demean,'' ``devalue,'' ``denigrate,'' or ``disrespect'' the
institution. 6, 28. Nobody disputes those points. Indeed, the compelling
personal accounts of petitioners and others like them are likely a
primary reason why many Americans have changed their minds about whether
same-sex couples should be allowed to marry. As a matter of
constitutional law, however, the sincerity of petitioners' wishes is not
relevant.

When the majority turns to the law, it relies primarily on precedents
discussing the fundamental ``right to marry.'' Turner v. Safley;
Zablocki,434 U. S.; see Loving. These cases do not hold, of course, that
anyone who wants to get married has a constitutional right to do so.
They instead require a State to justify barriers to marriage as that
institution has always been understood. In Loving, the Court held that
racial restrictions on the right to marry lacked a compelling
justification. In Zablocki, restrictions based on child support debts
did not suffice. In Turner, restrictions based on status as a prisoner
were deemed impermissible.

None of the laws at issue in those cases purported to change the core
definition of marriage as the union of a man and a woman. The laws
challenged in Zablocki and Turner did not define marriage as ``the union
of a man and a woman, where neither party owes child support or is in
prison.'' Nor did the interracial marriage ban at issue in Loving define
marriage as ``the union of a man and a woman of the same race.'' See
Tragen, Comment, Statutory Prohibitions Against Interracial Marriage, 32
Cal. L. Rev.~269 (1944) (``at common law there was no ban on interracial
marriage''). Removing racial barriers to marriage therefore did not
change what a marriage was any more than integrating schools changed
what a school was. As the majority admits, the institution of
``marriage'' discussed in every one of these cases ``presumed a
relationship involving opposite-sex partners.''

In short, the ``right to marry'' cases stand for the important but
limited proposition that particular restrictions on access to marriage
as traditionally defined violate due process. These precedents say
nothing at all about a right to make a State change its definition of
marriage, which is the right petitioners actually seek here. See
Windsor, (ALITO, J., dissenting) (slip op.) (``What Windsor and the
United States seek is not the protection of a deeply rooted right but
the recognition of a very new right.''). Neither petitioners nor the
majority cites a single case or other legal source providing any basis
for such a constitutional right. None exists, and that is enough to
foreclose their claim.

The majority suggests that ``there are other, more instructive
precedents'' informing the right to marry. Although not entirely clear,
this reference seems to correspond to a line of cases discussing an
implied fundamental ``right of privacy.'' Griswold. In the first of
those cases, the Court invalidated a criminal law that banned the use of
contraceptives. -486. The Court stressed the invasive nature of the ban,
which threatened the intrusion of ``the police to search the sacred
precincts of marital bedrooms.'' In the Court's view, such laws
infringed the right to privacy in its most basic sense: the ``right to
be let alone.''

The Court also invoked the right to privacy in Lawrence v. Texas, which
struck down a Texas statute criminalizing homosexual sodomy. Lawrence
relied on the position that criminal sodomy laws, like bans on
contraceptives, invaded privacy by inviting ``unwarranted government
intrusions'' that ``touc{[}h{]} upon the most private human conduct,
sexual behavior . in the most private of places, the home.''

Neither Lawrence nor any other precedent in the privacy line of cases
supports the right that petitioners assert here. Unlike criminal laws
banning contraceptives and sodomy, the marriage laws at issue here
involve no government intrusion. They create no crime and impose no
punishment. Same-sex couples remain free to live together, to engage in
intimate conduct, and to raise their families as they see fit. No one is
``condemned to live in loneliness'' by the laws challenged in these
cases---no one. At the same time, the laws in no way interfere with the
``right to be let alone.''

The majority also relies on Justice Harlan's influential dissenting
opinion in Poe v. Ullman. As the majority recounts, that opinion states
that ``{[}d{]}ue process has not been reduced to any formula.'' But far
from conferring the broad interpretive discretion that the majority
discerns, Justice Harlan's opinion makes clear that courts implying
fundamental rights are not ``free to roam where unguided speculation
might take them.'' They must instead have ``regard to what history
teaches'' and exercise not only ``judgment'' but ``restraint.'' Of
particular relevance, Justice Harlan explained that ``laws regarding
marriage which provide both when the sexual powers may be used and the
legal and societal context in which children are born and brought up .
form a pattern so deeply pressed into the substance of our social life
that any Constitutional doctrine in this area must build upon that
basis.''

In sum, the privacy cases provide no support for the majority's
position, because petitioners do not seek privacy. Quite the opposite,
they seek public recognition of their relationships, along with
corresponding government benefits. Our cases have consistently refused
to allow litigants to convert the shield provided by constitutional
liberties into a sword to demand positive entitlements from the State.
Thus, although the right to privacy recognized by our precedents
certainly plays a role in protecting the intimate conduct of same-sex
couples, it provides no affirmative right to redefine marriage and no
basis for striking down the laws at issue here.

Perhaps recognizing how little support it can derive from precedent, the
majority goes out of its way to jettison the ``careful'' approach to
implied fundamental rights taken by this Court in Glucksberg. It is
revealing that the majority's position requires it to effectively
overrule Glucksberg,the leading modern case setting the bounds of
substantive due process. At least this part of the majority opinion has
the virtue of candor. Nobody could rightly accuse the majority of taking
a careful approach.

Ultimately, only one precedent offers any support for the majority's
methodology: Lochner v. New York. The majority opens its opinion by
announcing petitioners' right to ``define and express their identity.''
The majority later explains that ``the right to personal choice
regarding marriage is inherent in the concept of individual autonomy.''
This freewheeling notion of individual autonomy echoes nothing so much
as ``the general right of an individual to be free in his person and in
his power to contract in relation to his own labor.'' Lochner (emphasis
added).

To be fair, the majority does not suggest that its individual autonomy
right is entirely unconstrained. The constraints it sets are precisely
those that accord with its own ``reasoned judgment,'' informed by its
``new insight'' into the ``nature of injustice,'' which was invisible to
all who came before but has become clear ``as we learn {[}the{]}
meaning'' of liberty. The truth is that today's decision rests on
nothing more than the majority's own conviction that same-sex couples
should be allowed to marry because they want to, and that ``it would
disparage their choices and diminish their personhood to deny them this
right.'' Whatever force that belief may have as a matter of moral
philosophy, it has no more basis in the Constitution than did the naked
policy preferences adopted in Lochner. See 198 U. S. (``We do not
believe in the soundness of the views which uphold this law,'' which
``is an illegal interference with the rights of individuals . to make
contracts regarding labor upon such terms as they may think best'').

The majority recognizes that today's cases do not mark ``the first time
the Court has been asked to adopt a cautious approach to recognizing and
protecting fundamental rights.'' On that much, we agree. The Court was
``asked''---and it agreed---to ``adopt a cautious approach'' to implying
fundamental rights after the debacle of the Lochner era. Today, the
majority casts caution aside and revives the grave errors of that
period.

One immediate question invited by the majority's position is whether
States may retain the definition of marriage as a union of two people.
Although the majority randomly inserts the adjective ``two'' in various
places, it offers no reason at all why the two-person element of the
core definition of marriage may be preserved while the man-woman element
may not. Indeed, from the standpoint of history and tradition, a leap
from opposite-sex marriage to same-sex marriage is much greater than one
from a two-person union to plural unions, which have deep roots in some
cultures around the world. If the majority is willing to take the big
leap, it is hard to see how it can say no to the shorter one.

It is striking how much of the majority's reasoning would apply with
equal force to the claim of a fundamental right to plural marriage. If
``{[}t{]}here is dignity in the bond between two men or two women who
seek to marry and in their autonomy to make such profound choices,'' why
would there be any less dignity in the bond between three people who, in
exercising their autonomy, seek to make the profound choice to marry? If
a same-sex couple has the constitutional right to marry because their
children would otherwise ``suffer the stigma of knowing their families
are somehow lesser,'' why wouldn't the same reasoning apply to a family
of three or more persons raising children? If not having the opportunity
to marry ``serves to disrespect and subordinate'' gay and lesbian
couples, why wouldn't the same ``imposition of this disability,'' serve
to disrespect and subordinate people who find fulfillment in polyamorous
relationships?

I do not mean to equate marriage between same-sex couples with plural
marriages in all respects. There may well be relevant differences that
compel different legal analysis. But if there are, petitioners have not
pointed to any. When asked about a plural marital union at oral
argument, petitioners asserted that a State ``doesn't have such an
institution.'' But that is exactly the point: the States at issue here
do not have an institution of same-sex marriage, either.

Near the end of its opinion, the majority offers perhaps the clearest
insight into its decision. Expanding marriage to include same-sex
couples, the majority insists, would ``pose no risk of harm to
themselves or third parties.'' This argument again echoes Lochner, which
relied on its assessment that ``we think that a law like the one before
us involves neither the safety, the morals nor the welfare of the
public, and that the interest of the public is not in the slightest
degree affected by such an act.''

Then and now, this assertion of the ``harm principle'' sounds more in
philosophy than law. The elevation of the fullest individual
self-realization over the constraints that society has expressed in law
may or may not be attractive moral philosophy. But a Justice's
commission does not confer any special moral, philosophical, or social
insight sufficient to justify imposing those perceptions on fellow
citizens under the pretense of ``due process.'' There is indeed a
process due the people on issues of this sort---the democratic process.
Respecting that understanding requires the Court to be guided by law,
not any particular school of social thought. As Judge Henry Friendly
once put it, echoing Justice Holmes's dissent in Lochner, the Fourteenth
Amendment does not enact John Stuart Mill's On Liberty any more than it
enacts Herbert Spencer's Social Statics. And it certainly does not enact
any one concept of marriage.

The majority's understanding of due process lays out a tantalizing
vision of the future for Members of this Court: If an unvarying social
institution enduring over all of recorded history cannot inhibit
judicial policymaking, what can? But this approach is dangerous for the
rule of law. The purpose of insisting that implied fundamental rights
have roots in the history and tradition of our people is to ensure that
when unelected judges strike down democratically enacted laws, they do
so based on something more than their own beliefs. The Court today not
only overlooks our country's entire history and tradition but actively
repudiates it, preferring to live only in the heady days of the here and
now. I agree with the majority that the ``nature of injustice is that we
may not always see it in our own times.'' As petitioners put it, ``times
can blind.'' But to blind yourself to history is both prideful and
unwise. ``The past is never dead. It's not even past.'' W. Faulkner,
Requiem for a Nun 92 (1951).

In addition to their due process argument, petitioners contend that the
Equal Protection Clause requires their States to license and recognize
same-sex marriages. The majority does not seriously engage with this
claim. Its discussion is, quite frankly, difficult to follow. The
central point seems to be that there is a ``synergy between'' the Equal
Protection Clause and the Due Process Clause, and that some precedents
relying on one Clause have also relied on the other. Absent from this
portion of the opinion, however, is anything resembling our usual
framework for deciding equal protection cases. It is casebook doctrine
that the ``modern Supreme Court's treatment of equal protection claims
has used a means-ends methodology in which judges ask whether the
classification the government is using is sufficiently related to the
goals it is pursuing.'' G. Stone, L. Seidman, C. Sunstein, M. Tushnet,
\& P. Karlan, Constitutional Law 453 (7th ed.~2013). The majority's
approach today is different:

``Rights implicit in liberty and rights secured by equal protection may
rest on different precepts and are not always co-extensive, yet in some
instances each may be instructive as to the meaning and reach of the
other. In any particular case one Clause may be thought to capture the
essence of the right in a more accurate and comprehensive way, even as
the two Clauses may converge in the identification and definition of the
right.'' The majority goes on to assert in conclusory fashion that the
Equal Protection Clause provides an alternative basis for its holding.
Yet the majority fails to provide even a single sentence explaining how
the Equal Protection Clause supplies independent weight for its
position, nor does it attempt to justify its gratuitous violation of the
canon against unnecessarily resolving constitutional questions. See
Northwest Austin Municipal Util. Dist. No.~One v. Holder. In any event,
the marriage laws at issue here do not violate the Equal Protection
Clause, because distinguishing between opposite-sex and same-sex couples
is rationally related to the States'``legitimate state interest'' in
``preserving the traditional institution of marriage.'' Lawrence
(O'Connor, J., concurring in judgment).

It is important to note with precision which laws petitioners have
challenged. Although they discuss some of the ancillary legal benefits
that accompany marriage, such as hospital visitation rights and
recognition of spousal status on official documents, petitioners'
lawsuits target the laws defining marriage generally rather than those
allocating benefits specifically. The equal protection analysis might be
different, in my view, if we were confronted with a more focused
challenge to the denial of certain tangible benefits. Of course, those
more selective claims will not arise now that the Court has taken the
drastic step of requiring every State to license and recognize marriages
between same-sex couples.

The legitimacy of this Court ultimately rests ``upon the respect
accorded to its judgments.'' Republican Party of Minn. v. White
(KENNEDY, J., concurring). That respect flows from the perception---and
reality---that we exercise humility and restraint in deciding cases
according to the Constitution and law. The role of the Court envisioned
by the majority today, however, is anything but humble or restrained.
Over and over, the majority exalts the role of the judiciary in
delivering social change. In the majority's telling, it is the courts,
not the people, who are responsible for making ``new dimensions of
freedom . apparent to new generations,'' for providing ``formal
discourse'' on social issues, and for ensuring ``neutral discussions,
without scornful or disparaging commentary.''

Nowhere is the majority's extravagant conception of judicial supremacy
more evident than in its description--- and dismissal---of the public
debate regarding same-sex marriage. Yes, the majority concedes, on one
side are thousands of years of human history in every society known to
have populated the planet. But on the other side, there has been
``extensive litigation,'' ``many thoughtful District Court decisions,''
``countless studies, papers, books, and other popular and scholarly
writings,'' and ``more than 100'' amicus briefs in these cases alone.
What would be the point of allowing the democratic process to go on? It
is high time for the Court to decide the meaning of marriage, based on
five lawyers' ``better informed understanding'' of ``a liberty that
remains urgent in our own era.'' The answer is surely there in one of
those amicus briefs or studies.

Those who founded our country would not recognize the majority's
conception of the judicial role. They after all risked their lives and
fortunes for the precious right to govern themselves. They would never
have imagined yielding that right on a question of social policy to
unaccountable and unelected judges. And they certainly would not have
been satisfied by a system empowering judges to override policy
judgments so long as they do so after ``a quite extensive discussion.''
In our democracy, debate about the content of the law is not an
exhaustion requirement to be checked off before courts can impose their
will. ``Surely the Constitution does not put either the legislative
branch or the executive branch in the position of a television quiz show
contestant so that when a given period of time has elapsed and a problem
remains unresolved by them, the federal judiciary may press a buzzer and
take its turn at fashioning a solution.'' Rehnquist, The Notion of a
Living Constitution, 54 Texas L. Rev.~693, 700 (1976). As a plurality of
this Court explained just last year, ``It is demeaning to the democratic
process to presume that voters are not capable of deciding an issue of
this sensitivity on decent and rational grounds.'' Schuette v. BAMN.

The Court's accumulation of power does not occur in a vacuum. It comes
at the expense of the people. And they know it. Here and abroad, people
are in the midst of a serious and thoughtful public debate on the issue
of same-sex marriage. They see voters carefully considering same-sex
marriage, casting ballots in favor or opposed, and sometimes changing
their minds. They see political leaders similarly reexamining their
positions, and either reversing course or explaining adherence to old
convictions confirmed anew. They see governments and businesses
modifying policies and practices with respect to same-sex couples, and
participating actively in the civic discourse. They see countries
overseas democratically accepting profound social change, or declining
to do so. This deliberative process is making people take seriously
questions that they may not have even regarded as questions before.

When decisions are reached through democratic means, some people will
inevitably be disappointed with the results. But those whose views do
not prevail at least know that they have had their say, and accordingly
are---in the tradition of our political culture---reconciled to the
result of a fair and honest debate. In addition, they can gear up to
raise the issue later, hoping to persuade enough on the winning side to
think again. ``That is exactly how our system of government is supposed
to work.'' Post (SCALIA, J., dissenting).

But today the Court puts a stop to all that. By deciding this question
under the Constitution, the Court removes it from the realm of
democratic decision. There will be consequences to shutting down the
political process on an issue of such profound public significance.
Closing debate tends to close minds. People denied a voice are less
likely to accept the ruling of a court on an issue that does not seem to
be the sort of thing courts usually decide. As a thoughtful commentator
observed about another issue, ``The political process was moving . .,
not swiftly enough for advocates of quick, complete change, but
majoritarian institutions were listening and acting. Heavy-handed
judicial intervention was difficult to justify and appears to have
provoked, not resolved, conflict.'' Ginsburg, Some Thoughts on Autonomy
and Equality in Relation to Roe v. Wade, 63 N. C. L. Rev.~375, 385-386
(1985) (footnote omitted). Indeed, however heartened the proponents of
same-sex marriage might be on this day, it is worth acknowledging what
they have lost, and lost forever: the opportunity to win the true
acceptance that comes from persuading their fellow citizens of the
justice of their cause. And they lose this just when the winds of change
were freshening at their backs.

Federal courts are blunt instruments when it comes to creating rights.
They have constitutional power only to resolve concrete cases or
controversies; they do not have the flexibility of legislatures to
address concerns of parties not before the court or to anticipate
problems that may arise from the exercise of a new right. Today's
decision, for example, creates serious questions about religious
liberty. Many good and decent people oppose same-sex marriage as a tenet
of faith, and their freedom to exercise religion is---unlike the right
imagined by the majority--- actually spelled out in the Constitution.

Respect for sincere religious conviction has led voters and legislators
in every State that has adopted same-sex marriage democratically to
include accommodations for religious practice. The majority's decision
imposing same-sex marriage cannot, of course, create any such
accommodations. The majority graciously suggests that religious
believers may continue to ``advocate'' and ``teach'' their views of
marriage. The First Amendment guarantees, however, the freedom to
``exercise'' religion. Ominously, that is not a word the majority uses.

Hard questions arise when people of faith exercise religion in ways that
may be seen to conflict with the new right to same-sex marriage---when,
for example, a religious college provides married student housing only
to opposite-sex married couples, or a religious adoption agency declines
to place children with same-sex married couples. Indeed, the Solicitor
General candidly acknowledged that the tax exemptions of some religious
institutions would be in question if they opposed same-sex marriage. See
Tr. of Oral Arg. on Question 1. There is little doubt that these and
similar questions will soon be before this Court. Unfortunately, people
of faith can take no comfort in the treatment they receive from the
majority today.

Perhaps the most discouraging aspect of today's decision is the extent
to which the majority feels compelled to sully those on the other side
of the debate. The majority offers a cursory assurance that it does not
intend to disparage people who, as a matter of conscience, cannot accept
same-sex marriage. That disclaimer is hard to square with the very next
sentence, in which the majority explains that ``the necessary
consequence'' of laws codifying the traditional definition of marriage
is to ``demea{[}n{]} or stigmatiz{[}e{]}'' same-sex couples. The
majority reiterates such characterizations over and over. By the
majority's account, Americans who did nothing more than follow the
understanding of marriage that has existed for our entire history---in
particular, the tens of millions of people who voted to reaffirm their
States' enduring definition of marriage---have acted to ``lock . out,''
``disparage,'' ``disrespect and subordinate,'' and inflict
``{[}d{]}ignitary wounds'' upon their gay and lesbian neighbors. 22, 25.
These apparent assaults on the character of fairminded people will have
an effect, in society and in court. See post (ALITO, J., dissenting).
Moreover, they are entirely gratuitous. It is one thing for the majority
to conclude that the Constitution protects a right to same-sex marriage;
it is something else to portray everyone who does not share the
majority's ``better informed understanding'' as bigoted.

In the face of all this, a much different view of the Court's role is
possible. That view is more modest and restrained. It is more skeptical
that the legal abilities of judges also reflect insight into moral and
philosophical issues. It is more sensitive to the fact that judges are
unelected and unaccountable, and that the legitimacy of their power
depends on confining it to the exercise of legal judgment. It is more
attuned to the lessons of history, and what it has meant for the country
and Court when Justices have exceeded their proper bounds. And it is
less pretentious than to suppose that while people around the world have
viewed an institution in a particular way for thousands of years, the
present generation and the present Court are the ones chosen to burst
the bonds of that history and tradition.

If you are among the many Americans---of whatever sexual
orientation---who favor expanding same-sex marriage, by all means
celebrate today's decision. Celebrate the achievement of a desired goal.
Celebrate the opportunity for a new expression of commitment to a
partner. Celebrate the availability of new benefits. But do not
celebrate the Constitution. It had nothing to do with it.

I respectfully dissent.

\textbf{JUSTICE SCALIA, with whom JUSTICE THOMAS joins, dissenting.} I
join THE CHIEF JUSTICE's opinion in full. I write separately to call
attention to this Court's threat to American democracy.

The substance of today's decree is not of immense personal importance to
me. The law can recognize as marriage whatever sexual attachments and
living arrangements it wishes, and can accord them favorable civil
consequences, from tax treatment to rights of inheritance. Those civil
consequences---and the public approval that conferring the name of
marriage evidences---can perhaps have adverse social effects, but no
more adverse than the effects of many other controversial laws. So it is
not of special importance to me what the law says about marriage. It is
of overwhelming importance, however, who it is that rules me. Today's
decree says that my Ruler, and the Ruler of 320 million Americans
coast-to-coast, is a majority of the nine lawyers on the Supreme Court.
The opinion in these cases is the furthest extension in fact--- and the
furthest extension one can even imagine---of the Court's claimed power
to create ``liberties'' that the Constitution and its Amendments neglect
to mention. This practice of constitutional revision by an unelected
committee of nine, always accompanied (as it is today) by extravagant
praise of liberty, robs the People of the most important liberty they
asserted in the Declaration of Independence and won in the Revolution of
1776: the freedom to govern themselves.

Until the courts put a stop to it, public debate over same-sex marriage
displayed American democracy at its best. Individuals on both sides of
the issue passionately, but respectfully, attempted to persuade their
fellow citizens to accept their views. Americans considered the
arguments and put the question to a vote. The electorates of 11 States,
either directly or through their representatives, chose to expand the
traditional definition of marriage. Many more decided not to Win or
lose, advocates for both sides continued pressing their cases, secure in
the knowledge that an electoral loss can be negated by a later electoral
win. That is exactly how our system of government is supposed to work.

The Constitution places some constraints on self-rule--- constraints
adopted by the People themselves when they ratified the Constitution and
its Amendments. Forbidden are laws ``impairing the Obligation of
Contracts,'' denying ``Full Faith and Credit'' to the ``public Acts'' of
other States,4 prohibiting the free exercise of religion,5 abridging the
freedom of speech,6 infringing the right to keep and bear arms,7
authorizing unreasonable searches and seizures,8 and so forth. Aside
from these limitations, those powers ``reserved to the States
respectively, or to the people'' can be exercised as the States or the
People desire. These cases ask us to decide whether the Fourteenth
Amendment contains a limitation that requires the States to license and
recognize marriages between two people of the same sex. Does it remove
that issue from the political process?

Of course not. It would be surprising to find a prescription regarding
marriage in the Federal Constitution since, as the author of today's
opinion reminded us only two years ago (in an opinion joined by the same
Justices who join him today):

``{[}R{]}egulation of domestic relations is an area that has long been
regarded as a virtually exclusive province of the States.''

``{[}T{]}he Federal Government, through our history, has deferred to
state-law policy decisions with respect to domestic relations.''

But we need not speculate. When the Fourteenth Amendment was ratified in
1868, every State limited marriage to one man and one woman, and no one
doubted the constitutionality of doing so. That resolves these cases.
When it comes to determining the meaning of a vague constitutional
provision---such as ``due process of law'' or ``equal protection of the
laws''---it is unquestionable that the People who ratified that
provision did not understand it to prohibit a practice that remained
both universal and uncontroversial in the years after ratification We
have no basis for striking down a practice that is not expressly
prohibited by the Fourteenth Amendment's text, and that bears the
endorsement of a long tradition of open, widespread, and unchallenged
use dating back to the Amendment's ratification. Since there is no doubt
whatever that the People never decided to prohibit the limitation of
marriage to opposite-sex couples, the public debate over same-sex
marriage must be allowed to continue.

But the Court ends this debate, in an opinion lacking even a thin veneer
of law. Buried beneath the mummeries and straining-to-be-memorable
passages of the opinion is a candid and startling assertion: No matter
what it was the People ratified, the Fourteenth Amendment protects those
rights that the Judiciary, in its ``reasoned judgment,'' thinks the
Fourteenth Amendment ought to protect That is so because ``{[}t{]}he
generations that wrote and ratified the Bill of Rights and the
Fourteenth Amendment did not presume to know the extent of freedom in
all of its dimensions. .''14 One would think that sentence would
continue: ``. . and therefore they provided for a means by which the
People could amend the Constitution,'' or perhaps ``. . and therefore
they left the creation of additional liberties, such as the freedom to
marry someone of the same sex, to the People, through the never-ending
process of legislation.'' But no. What logically follows, in the
majority's judge-empowering estimation, is: ``and so they entrusted to
future generations a charter protecting the right of all persons to
enjoy liberty as we learn its meaning.''15 The ``we,'' needless to say,
is the nine of us. ``History and tradition guide and discipline
{[}our{]} inquiry but do not set its outer boundaries.''16 Thus, rather
than focusing on the People's understanding of ``liberty''---at the time
of ratification or even today---the majority focuses on four
``principles and traditions'' that, in the majority's view, prohibit
States from defining marriage as an institution consisting of one man
and one woman.

This is a naked judicial claim to legislative---indeed,
super-legislative---power; a claim fundamentally at odds with our system
of government. Except as limited by a constitutional prohibition agreed
to by the People, the States are free to adopt whatever laws they like,
even those that offend the esteemed Justices'``reasoned judgment.'' A
system of government that makes the People subordinate to a committee of
nine unelected lawyers does not deserve to be called a democracy.

Judges are selected precisely for their skill as lawyers; whether they
reflect the policy views of a particular constituency is not (or should
not be) relevant. Not surprisingly then, the Federal Judiciary is hardly
a cross-section of America. Take, for example, this Court, which
consists of only nine men and women, all of them successful lawyers18
who studied at Harvard or Yale Law School. Four of the nine are natives
of New York City. Eight of them grew up in east- and west-coast States.
Only one hails from the vast expanse in-between. Not a single
Southwesterner or even, to tell the truth, a genuine Westerner
(California does not count). Not a single evangelical Christian (a group
that comprises about one quarter of Americans19), or even a Protestant
of any denomination. The strikingly unrepresentative character of the
body voting on today's social upheaval would be irrelevant if they were
functioning as judges, answering the legal question whether the American
people had ever ratified a constitutional provision that was understood
to proscribe the traditional definition of marriage. But of course the
Justices in today's majority are not voting on that basis; they say they
are not. And to allow the policy question of same-sex marriage to be
considered and resolved by a select, patrician, highly unrepresentative
panel of nine is to violate a principle even more fundamental than no
taxation without representation: no social transformation without
representation.

But what really astounds is the hubris reflected in today's judicial
Putsch. The five Justices who compose today's majority are entirely
comfortable concluding that every State violated the Constitution for
all of the 135 years between the Fourteenth Amendment's ratification and
Massachusetts' permitting of same-sex marriages in 2003 They have
discovered in the Fourteenth Amendment a ``fundamental right''
overlooked by every person alive at the time of ratification, and almost
everyone else in the time since. They see what lesser legal minds---
minds like Thomas Cooley, John Marshall Harlan, Oliver Wendell Holmes,
Jr., Learned Hand, Louis Brandeis, William Howard Taft, Benjamin
Cardozo, Hugo Black, Felix Frankfurter, Robert Jackson, and Henry
Friendly--- could not. They are certain that the People ratified the
Fourteenth Amendment to bestow on them the power to remove questions
from the democratic process when that is called for by their ``reasoned
judgment.'' These Justices know that limiting marriage to one man and
one woman is contrary to reason; they know that an institution as old as
government itself, and accepted by every nation in history until 15
years ago,21cannot possibly be supported by anything other than
ignorance or bigotry. And they are willing to say that any citizen who
does not agree with that, who adheres to what was, until 15 years ago,
the unanimous judgment of all generations and all societies, stands
against the Constitution.

The opinion is couched in a style that is as pretentious as its content
is egotistic. It is one thing for separate concurring or dissenting
opinions to contain extravagances, even silly extravagances, of thought
and expression; it is something else for the official opinion of the
Court to do so Of course the opinion's showy profundities are often
profoundly incoherent. ``The nature of marriage is that, through its
enduring bond, two persons together can find other freedoms, such as
expression, intimacy, and spirituality.''23 (Really? Who ever thought
that intimacy and spirituality {[}whatever that means{]} were freedoms?
And if intimacy is, one would think Freedom of Intimacy is abridged
rather than expanded by marriage. Ask the nearest hippie. Expression,
sure enough, is a freedom, but anyone in a long-lasting marriage will
attest that that happy state constricts, rather than expands, what one
can prudently say.) Rights, we are told, can ``rise . from a better
informed understanding of how constitutional imperatives define a
liberty that remains urgent in our own era.''24 (Huh? How can a better
informed understanding of how constitutional imperatives {[}whatever
that means{]} define {[}whatever that means{]} an urgent liberty
{[}never mind{]}, give birth to a right?) And we are told that,
``{[}i{]}n any particular case,'' either the Equal Protection or Due
Process Clause ``may be thought to capture the essence of {[}a{]} right
in a more accurate and comprehensive way,'' than the other, ``even as
the two Clauses may converge in the identification and definition of the
right.'' (What say? What possible ``essence'' does substantive due
process ``capture'' in an ``accurate and comprehensive way''? It stands
for nothing whatever, except those freedoms and entitlements that this
Court really likes. And the Equal Protection Clause, as employed today,
identifies nothing except a difference in treatment that this Court
really dislikes. Hardly a distillation of essence. If the opinion is
correct that the two clauses ``converge in the identification and
definition of {[}a{]} right,'' that is only because the majority's likes
and dislikes are predictably compatible.) I could go on. The world does
not expect logic and precision in poetry or inspirational pop
philosophy; it demands them in the law. The stuff contained in today's
opinion has to diminish this Court's reputation for clear thinking and
sober analysis.

Hubris is sometimes defined as o'erweening pride; and pride, we know,
goeth before a fall. The Judiciary is the ``least dangerous'' of the
federal branches because it has ``neither Force nor Will, but merely
judgment; and must ultimately depend upon the aid of the executive arm''
and the States, ``even for the efficacy of its judgments.''26 With each
decision of ours that takes from the People a question properly left to
them---with each decision that is unabashedly based not on law, but on
the ``reasoned judgment'' of a bare majority of this Court---we move one
step closer to being reminded of our impotence.

\textbf{JUSTICE THOMAS, with whom JUSTICE SCALIA joins, dissenting.} The
Court's decision today is at odds not only with the Constitution, but
with the principles upon which our Nation was built. Since well before
1787, liberty has been understood as freedom from government action, not
entitlement to government benefits. The Framers created our Constitution
to preserve that understanding of liberty. Yet the majority invokes our
Constitution in the name of a ``liberty'' that the Framers would not
have recognized, to the detriment of the liberty they sought to protect.
Along the way, it rejects the idea---captured in our Declaration of
Independence---that human dignity is innate and suggests instead that it
comes from the Government. This distortion of our Constitution not only
ignores the text, it inverts the relationship between the individual and
the state in our Republic. I cannot agree with it.

Petitioners cannot claim, under the most plausible definition of
``liberty,'' that they have been imprisoned or physically restrained by
the States for participating in same-sex relationships. To the contrary,
they have been able to cohabitate and raise their children in peace.
They have been able to hold civil marriage ceremonies in States that
recognize same-sex marriages and private religious ceremonies in all
States. They have been able to travel freely around the country, making
their homes where they please. Far from being incarcerated or physically
restrained, petitioners have been left alone to order their lives as
they see fit.

Nor, under the broader definition, can they claim that the States have
restricted their ability to go about their daily lives as they would be
able to absent governmental restrictions. Petitioners do not ask this
Court to order the States to stop restricting their ability to enter
same-sex relationships, to engage in intimate behavior, to make vows to
their partners in public ceremonies, to engage in religious wedding
ceremonies, to hold themselves out as married, or to raise children. The
States have imposed no such restrictions. Nor have the States prevented
petitioners from approximating a number of incidents of marriage through
private legal means, such as wills, trusts, and powers of attorney.

Instead, the States have refused to grant them governmental
entitlements. Petitioners claim that as a matter of ``liberty,'' they
are entitled to access privileges and benefits that exist solely because
of the government. They want, for example, to receive the State's
imprimatur on their marriages---on state issued marriage licenses, death
certificates, or other official forms. And they want to receive various
monetary benefits, including reduced inheritance taxes upon the death of
a spouse, compensation if a spouse dies as a result of a work-related
injury, or loss of consortium damages in tort suits. But receiving
governmental recognition and benefits has nothing to do with any
understanding of ``liberty'' that the Framers would have recognized.

The suggestion of petitioners and their amici that antimiscegenation
laws are akin to laws defining marriage as between one man and one woman
is both offensive and inaccurate. ``America's earliest laws against
interracial sex and marriage were spawned by slavery.'' P. Pascoe, What
Comes Naturally: Miscegenation Law and the Making of Race in America 19
(2009). For instance, Maryland's 1664 law prohibiting marriages between
``\texttt{freeborne\ English\ women\textquotesingle{}"\ and\ "}Negro
Sla{[}v{]}es''' was passed as part of the very act that authorized
lifelong slavery in the colony. -20. Virginia's antimiscegenation laws
likewise were passed in a 1691 resolution entitled ``An act for
suppressing outlying Slaves.'' Act of Apr.~1691, Ch. XVI, 3 Va. Stat. 86
(W. Hening ed.~1823) (reprint 1969) (italics deleted). ``It was not
until the Civil War threw the future of slavery into doubt that lawyers,
legislators, and judges began to develop the elaborate justifications
that signified the emergence of miscegenation law and made restrictions
on interracial marriage the foundation of post-Civil War white
supremacy.'' Pascoe.

\hypertarget{congresss-enforcement-power}{%
\section{Congress's Enforcement
Power}\label{congresss-enforcement-power}}

\hypertarget{city-of-boerne-v.-flores}{%
\subsubsection{City of Boerne v.
Flores}\label{city-of-boerne-v.-flores}}

\textbf{Justice Kennedy delivered the opinion of the Court.} A decision
by local zoning authorities to deny a church a building permit was
challenged under the Religious Freedom Restoration Act of 1993 (RFRA or
Act), 107 Stat. 1488, 42 U. S. C. § 2000bb et seq. The case calls into
question the authority of Congress to enact RFRA. We conclude the
statute exceeds Congress' power.

Situated on a hill in the city of Boerne, Texas, some 28 miles northwest
of San Antonio, is St.~Peter Catholic Church. Built in 1923, the
church's structure replicates the mission style of the region's earlier
history. The church seats about 230 worshippers, a number too small for
its growing parish. Some 40 to 60 parishioners cannot be accommodated at
some Sunday masses. In order to meet the needs of the congregation the
Archbishop of San Antonio gave permission to the parish to plan
alterations to enlarge the building.

A few months later, the Boerne City Council passed an ordinance
authorizing the city's Historic Landmark Commission to prepare a
preservation plan with proposed historic landmarks and districts. Under
the ordinance, the commission must preapprove construction affecting
historic landmarks or buildings in a historic district.

Soon afterwards, the Archbishop applied for a building permit so
construction to enlarge the church could proceed. City authorities,
relying on the ordinance and the designation of a historic district
(which, they argued, included the church), denied the application. The
Archbishop brought this suit challenging the permit denial in the United
States District Court for the Western District of Texas.

The complaint contained various claims, but to this point the litigation
has centered on RFRA and the question of its constitutionality. The
Archbishop relied upon RFRA as one basis for relief from the refusal to
issue the permit. The District Court concluded that by enacting RFRA
Congress exceeded the scope of its enforcement power under § 5 of the
Fourteenth Amendment. The court certified its order for interlocutory
appeal and the Fifth Circuit reversed, finding RFRA to be
constitutional. We granted certiorari, and now reverse.

Smith held that neutral, generally applicable laws may be applied to
religious practices even when not supported by a compelling governmental
interest. The parties disagree over whether RFRA is a proper exercise of
Congress' § 5 power ``to enforce'' by ``appropriate legislation'' the
constitutional guarantee that no State shall deprive any person of
``life, liberty, or property, without due process of law,'' nor deny any
person ``equal protection of the laws.''

In defense of the Act, respondent the Archbishop contends, with support
from the United States, that RFRA is permissible enforcement
legislation. Congress, it is said, is only protecting by legislation one
of the liberties guaranteed by the Fourteenth Amendment's Due Process
Clause, the free exercise of religion, beyond what is necessary under
Smith It is said the congressional decision to dispense with proof of
deliberate or overt discrimination and instead concentrate on a law's
effects accords with the settled understanding that § 5 includes the
power to enact legislation designed to prevent, as well as remedy,
constitutional violations. It is further contended that Congress' § 5
power is not limited to remedial or preventive legislation.

All must acknowledge that § 5 is ``a positive grant of legislative
power'' to Congress, Katzenbach v. Morgan. In Ex parte Virginia (1880),
we explained the scope of Congress' § 5 power in the following broad
terms:

``Whatever legislation is appropriate, that is, adapted to carry out the
objects the amendments have in view, whatever tends to enforce
submission to the prohibitions they contain, and to secure to all
persons the enjoyment of perfect equality of civil rights and the equal
protection of the laws against State denial or invasion, if not
prohibited, is brought within the domain of congressional power.''

Legislation which deters or remedies constitutional violations can fall
within the sweep of Congress' enforcement power even if in the process
it prohibits conduct which is not itself unconstitutional and intrudes
into ``legislative spheres of autonomy previously reserved to the
States.'' Fitzpatrick v. Bitzer. For example, the Court upheld a
suspension of literacy tests and similar voting requirements under
Congress' parallel power to enforce the provisions of the Fifteenth
Amendment, see U. S. Const., Amdt. 15, § 2, as a measure to combat
racial discrimination in voting, South Carolina v. Katzenbach (1966),
despite the facial constitutionality of the tests under Lassiter v.
Northampton County Bd. of Elections. We have also concluded that other
measures protecting voting rights are within Congress' power to enforce
the Fourteenth and Fifteenth Amendments, despite the burdens those
measures placed on the States. South Carolina v. Katzenbach (upholding
several provisions of the Voting Rights Act of 1965); Katzenbach v.
Morgan (upholding ban on literacy tests that prohibited certain people
schooled in Puerto Rico from voting); Oregon v. Mitchell (upholding
5-year nationwide ban on literacy tests and similar voting requirements
for registering to vote); City of Rome v. United States (upholding
7-year extension of the Voting Rights Act's requirement that certain
jurisdictions preclear any change to a ```standard, practice, or
procedure with respect to voting'''); see also James Everard's Breweries
v. Day (upholding ban on medical prescription of intoxicating malt
liquors as appropriate to enforce Eighteenth Amendment ban on
manufacture, sale, or transportation of intoxicating liquors for
beverage purposes).

It is also true, however, that ``{[}a{]}s broad as the congressional
enforcement power is, it is not unlimited.'' Oregon v. Mitchell (opinion
of Black, J.). In assessing the breadth of § 5's enforcement power, we
begin with its text. Congress has been given the power ``to enforce''
the ``provisions of this article.'' We agree with respondent, of course,
that Congress can enact legislation under § 5 enforcing the
constitutional right to the free exercise of religion. The ``provisions
of this article,'' to which § 5 refers, include the Due Process Clause
of the Fourteenth Amendment. Congress' power to enforce the Free
Exercise Clause follows from our holding in Cantwell v. Connecticut,
that the ``fundamental concept of liberty embodied in {[}the Fourteenth
Amendment's Due Process Clause{]} embraces the liberties guaranteed by
the First Amendment.'' See also United States v. Price (there is ``no
doubt of the power of Congress to enforce by appropriate criminal
sanction every right guaranteed by the Due Process Clause of the
Fourteenth Amendment'' (internal quotation marks and citation omitted)).

Congress' power under § 5, however, extends only to ``enforc{[}ing{]}''
the provisions of the Fourteenth Amendment. The Court has described this
power as ``remedial,'' South Carolina v. Katzenbach. The design of the
Amendment and the text of § 5 are inconsistent with the suggestion that
Congress has the power to decree the substance of the Fourteenth
Amendment's restrictions on the States. Legislation which alters the
meaning of the Free Exercise Clause cannot be said to be enforcing the
Clause. Congress does not enforce a constitutional right by changing
what the right is. It has been given the power ``to enforce,'' not the
power to determine what constitutes a constitutional violation. Were it
not so, what Congress would be enforcing would no longer be, in any
meaningful sense, the ``provisions of {[}the Fourteenth Amendment{]}.''

While the line between measures that remedy or prevent unconstitutional
actions and measures that make a substantive change in the governing law
is not easy to discern, and Congress must have wide latitude in
determining where it lies, the distinction exists and must be observed.
There must be a congruence and proportionality between the injury to be
prevented or remedied and the means adopted to that end. Lacking such a
connection, legislation may become substantive in operation and effect.
History and our case law support drawing the distinction, one apparent
from the text of the Amendment.

The Fourteenth Amendment's history confirms the remedial, rather than
substantive, nature of the Enforcement Clause. The Joint Committee on
Reconstruction of the 39th Congress began drafting what would become the
Fourteenth Amendment in January 1866. The objections to the Committee's
first draft of the Amendment, and the rejection of the draft, have a
direct bearing on the central issue of defining Congress' enforcement
power. In February, Republican Representative John Bingham of Ohio
reported the following draft Amendment to the House of Representatives
on behalf of the Joint Committee:

``The Congress shall have power to make all laws which shall be
necessary and proper to secure to the citizens of each State all
privileges and immunities of citizens in the several States, and to all
persons in the several States equal protection in the rights of life,
liberty, and property.''

The remedial and preventive nature of Congress' enforcement power, and
the limitation inherent in the power, were confirmed in our earliest
cases on the Fourteenth Amendment. In the Civil Rights Cases, 109 U. S.
3 (1883), the Court invalidated sections of the Civil Rights Act of 1875
which prescribed criminal penalties for denying to any person ``the full
enjoyment of'' public accommodations and conveyances, on the grounds
that it exceeded Congress' power by seeking to regulate private conduct.
The Enforcement Clause, the Court said, did not authorize Congress to
pass ``general legislation upon the rights of the citizen, but
corrective legislation, that is, such as may be necessary and proper for
counteracting such laws as the States may adopt or enforce, and which,
by the amendment, they are prohibited from making or enforcing . .''
-14. The power to ``legislate generally upon'' life, liberty, and
property, as opposed to the ``power to provide modes of redress''
against offensive state action, was ``repugnant'' to the Constitution.
See also United States v. Reese; United States v. Harris; James v.
Bowman. Although the specific holdings of these early cases might have
been superseded or modified, see, e. g., Heart of Atlanta Motel, Inc.~v.
United States; United States v. Guest, their treatment of Congress' § 5
power as corrective or preventive, not definitional, has not been
questioned.

Recent cases have continued to revolve around the question whether § 5
legislation can be considered remedial. In South Carolina v.
Katzenbachwe emphasized that ``{[}t{]}he constitutional propriety of
{[}legislation adopted under the Enforcement Clause{]} must be judged
with reference to the historical experience . it reflects.'' There we
upheld various provisions of the Voting Rights Act of 1965, finding them
to be ``remedies aimed at areas where voting discrimination has been
most flagrant,'' and necessary to ``banish the blight of racial
discrimination in voting, which has infected the electoral process in
parts of our country for nearly a century,'' We noted evidence in the
record reflecting the subsisting and pervasive discriminatory---and
therefore unconstitutional---use of literacy tests. The Act's new
remedies, which used the administrative resources of the Federal
Government, included the suspension of both literacy tests and, pending
federal review, all new voting regulations in covered jurisdictions, as
well as the assignment of federal examiners to list qualified applicants
enabling those listed to vote. The new, unprecedented remedies were
deemed necessary given the ineffectiveness of the existing voting rights
laws, and the slow, costly character of case-by-case litigation.

We now turn to consider whether RFRA can be considered enforcement
legislation under § 5 of the Fourteenth Amendment.

Respondent contends that RFRA is a proper exercise of Congress' remedial
or preventive power. The Act, it is said, is a reasonable means of
protecting the free exercise of religion as defined by Smith. It
prevents and remedies laws which are enacted with the unconstitutional
object of targeting religious beliefs and practices. See Church of
Lukumi Babalu Aye, Inc.~v. Hialeah (``{[}A{]} law targeting religious
beliefs as such is never permissible''). To avoid the difficulty of
proving such violations, it is said, Congress can simply invalidate any
law which imposes a substantial burden on a religious practice unless it
is justified by a compelling interest and is the least restrictive means
of accomplishing that interest. If Congress can prohibit laws with
discriminatory effects in order to prevent racial discrimination in
violation of the Equal Protection Clause, then it can do the same,
respondent argues, to promote religious liberty.

While preventive rules are sometimes appropriate remedial measures,
there must be a congruence between the means used and the ends to be
achieved. The appropriateness of remedial measures must be considered in
light of the evil presented. Strong measures appropriate to address one
harm may be an unwarranted response to another, lesser one.

Regardless of the state of the legislative record, RFRA cannot be
considered remedial, preventive legislation, if those terms are to have
any meaning. RFRA is so out of proportion to a supposed remedial or
preventive object that it cannot be understood as responsive to, or
designed to prevent, unconstitutional behavior. It appears, instead, to
attempt a substantive change in constitutional protections. Preventive
measures prohibiting certain types of laws may be appropriate when there
is reason to believe that many of the laws affected by the congressional
enactment have a significant likelihood of being unconstitutional. See
City of Rome (since ``jurisdictions with a demonstrable history of
intentional racial discrimination . create the risk of purposeful
discrimination,'' Congress could ``prohibit changes that have a
discriminatory impact'' in those jurisdictions). Remedial legislation
under § 5 ``should be adapted to the mischief and wrong which the
{[}Fourteenth{]} {[}A{]}mendment was intended to provide against.''
Civil Rights Cases.

RFRA is not so confined. Sweeping coverage ensures its intrusion at
every level of government, displacing laws and prohibiting official
actions of almost every description and regardless of subject matter.
RFRA's restrictions apply to every agency and official of the Federal,
State, and local Governments. RFRA applies to all federal and state law,
statutory or otherwise, whether adopted before or after its enactment.
RFRA has no termination date or termination mechanism. Any law is
subject to challenge at any time by any individual who alleges a
substantial burden on his or her free exercise of religion.

The stringent test RFRA demands of state laws reflects a lack of
proportionality or congruence between the means adopted and the
legitimate end to be achieved. If an objector can show a substantial
burden on his free exercise, the State must demonstrate a compelling
governmental interest and show that the law is the least restrictive
means of furthering its interest. Claims that a law substantially
burdens someone's exercise of religion will often be difficult to
contest. See Smith (``What principle of law or logic can be brought to
bear to contradict a believer's assertion that a particular act
is'central' to his personal faith?''); (``The distinction between
questions of centrality and questions of sincerity and burden is
admittedly fine . .'') (O'Connor, J., concurring in judgment). Requiring
a State to demonstrate a compelling interest and show that it has
adopted the least restrictive means of achieving that interest is the
most demanding test known to constitutional law. If ```compelling
interest' really means what it says . many laws will not meet the test.
. {[}The test{]} would open the prospect of constitutionally required
religious exemptions from civic obligations of almost every conceivable
kind.'' Laws valid under Smith would fall under RFRA without regard to
whether they had the object of stifling or punishing free exercise. We
make these observations not to reargue the position of the majority in
Smith but to illustrate the substantive alteration of its holding
attempted by RFRA. Even assuming RFRA would be interpreted in effect to
mandate some lesser test, say, one equivalent to intermediate scrutiny,
the statute nevertheless would require searching judicial scrutiny of
state law with the attendant likelihood of invalidation. This is a
considerable congressional intrusion into the States' traditional
prerogatives and general authority to regulate for the health and
welfare of their citizens.

The substantial costs RFRA exacts, both in practical terms of imposing a
heavy litigation burden on the States and in terms of curtailing their
traditional general regulatory power, far exceed any pattern or practice
of unconstitutional conduct under the Free Exercise Clause as
interpreted in Smith. Simply put, RFRA is not designed to identify and
counteract state laws likely to be unconstitutional because of their
treatment of religion. In most cases, the state laws to which RFRA
applies are not ones which will have been motivated by religious
bigotry. If a state law disproportionately burdened a particular class
of religious observers, this circumstance might be evidence of an
impermissible legislative motive. Cf. Washington v. Davis. RFRA's
substantial-burden test, however, is not even a discriminatory-effects
or disparate-impact test. It is a reality of the modern regulatory state
that numerous state laws, such as the zoning regulations at issue here,
impose a substantial burden on a large class of individuals. When the
exercise of religion has been burdened in an incidental way by a law of
general application, it does not follow that the persons affected have
been burdened any more than other citizens, let alone burdened because
of their religious beliefs. In addition, the Act imposes in every case a
least restrictive means requirement---a requirement that was not used in
the pre-Smith jurisprudence RFRA purported to codify---which also
indicates that the legislation is broader than is appropriate if the
goal is to prevent and remedy constitutional violations.

Broad as the power of Congress is under the Enforcement Clause of the
Fourteenth Amendment, RFRA contradicts vital principles necessary to
maintain separation of powers and the federal balance. The judgment of
the Court of Appeals sustaining the Act's constitutionality is reversed.

It is so ordered.

\hypertarget{united-states-v.-morrison}{%
\subsubsection{United States v.
Morrison}\label{united-states-v.-morrison}}

529 U.S. 598 (2000)

\textbf{Chief Justice Rehnquist delivered the opinion of the Court.}

Because we conclude that the Commerce Clause does not provide Congress
with authority to enact §13981, we address petitioners' alternative
argument that the section's civil remedy should be upheld as an exercise
of Congress' remedial power under §5 of the Fourteenth Amendment. As
noted above, Congress expressly invoked the Fourteenth Amendment as a
source of authority to enact §13981.

The principles governing an analysis of congressional legislation under
§5 are well settled. Section 5 states that Congress may ``\,`enforce,'
by'appropriate legislation' the constitutional guarantee that no State
shall deprive any person of'life, liberty or property, without due
process of law,' nor deny any person'equal protection of the laws.'''
City of Boerne v. Flores. Section 5 is ``a positive grant of legislative
power,'' Katzenbach v. Morgan, that includes authority to ``prohibit
conduct which is not itself unconstitutional and {[}to{]} intrud{[}e{]}
into'legislative spheres of autonomy previously reserved to the
States.''' Flores (quoting Fitzpatrick v. Bitzer). However, ``{[}a{]}s
broad as the congressional enforcement power is, it is not unlimited.''
Oregon v. Mitchell. In fact, as we discuss in detail below, several
limitations inherent in §5's text and constitutional context have been
recognized since the Fourteenth Amendment was adopted.

Petitioners' §5 argument is founded on an assertion that there is
pervasive bias in various state justice systems against victims of
gender-motivated violence. This assertion is supported by a voluminous
congressional record. Specifically, Congress received evidence that many
participants in state justice systems are perpetuating an array of
erroneous stereotypes and assumptions. Congress concluded that these
discriminatory stereotypes often result in insufficient investigation
and prosecution of gender-motivated crime, inappropriate focus on the
behavior and credibility of the victims of that crime, and unacceptably
lenient punishments for those who are actually convicted of
gender-motivated violence. Petitioners contend that this bias denies
victims of gender-motivated violence the equal protection of the laws
and that Congress therefore acted appropriately in enacting a private
civil remedy against the perpetrators of gender-motivated violence to
both remedy the States' bias and deter future instances of
discrimination in the state courts.

As our cases have established, state-sponsored gender discrimination
violates equal protection unless it serves ``important governmental
objectives and \_ the discriminatory means employed'' are
``substantially related to the achievement of those objectives.''
However, the language and purpose of the Fourteenth Amendment place
certain limitations on the manner in which Congress may attack
discriminatory conduct. These limitations are necessary to prevent the
Fourteenth Amendment from obliterating the Framers' carefully crafted
balance of power between the States and the National Government. See
Flores (reviewing the history of the Fourteenth Amendment's enactment
and discussing the contemporary belief that the Amendment ``does not
concentrate power in the general government for any purpose of police
government within the States''). Foremost among these limitations is the
time-honored principle that the Fourteenth Amendment, by its very terms,
prohibits only state action. ``{[}T{]}he principle has become firmly
embedded in our constitutional law that the action inhibited by the
first section of the Fourteenth Amendment is only such action as may
fairly be said to be that of the States. That Amendment erects no shield
against merely private conduct, however discriminatory or wrongful.''
Shelley v. Kraemer (1948).

Shortly after the Fourteenth Amendment was adopted, we decided two cases
interpreting the Amendment's provisions, United States v. Harris, and
the Civil Rights Cases, 109 U.S. 3 (1883). In Harris, the Court
considered a challenge to §2 of the Civil Rights Act of 1871. That
section sought to punish ``private persons'' for ``conspiring to deprive
any one of the equal protection of the laws enacted by the State.'' We
concluded that this law exceeded Congress' §5 power because the law was
``directed exclusively against the action of private persons, without
reference to the laws of the State, or their administration by her
officers.'' In so doing, we reemphasized our statement from Virginia v.
Rives, that ``these provisions of the fourteenth amendment have
reference to State action exclusively, and not to any action of private
individuals.''

We reached a similar conclusion in the Civil Rights Cases. In those
consolidated cases, we held that the public accommodation provisions of
the Civil Rights Act of 1875, which applied to purely private conduct,
were beyond the scope of the §5 enforcement power. (``Individual
invasion of individual rights is not the subject-matter of the
{[}Fourteenth{]} {[}A{]}mendment''). See also, e.g., Romer v. Evans
(``{[}I{]}t was settled early that the Fourteenth Amendment did not give
Congress a general power to prohibit discrimination in public
accommodations''); Lugar v. Edmondson Oil Co.~(``Careful adherence to
the'state action' requirement preserves an area of individual freedom by
limiting the reach of federal law and federal judicial power''); United
States v. Cruikshank (``The fourteenth amendment prohibits a state from
depriving any person of life, liberty, or property, without due process
of law; but this adds nothing to the rights of one citizen as against
another. It simply furnishes an additional guaranty against any
encroachment by the States upon the fundamental rights which belong to
every citizen as a member of society'').

The force of the doctrine of stare decisis behind these decisions stems
not only from the length of time they have been on the books, but also
from the insight attributable to the Members of the Court at that time.
Every Member had been appointed by President Lincoln, Grant, Hayes,
Garfield, or Arthur-and each of their judicial appointees obviously had
intimate knowledge and familiarity with the events surrounding the
adoption of the Fourteenth Amendment.

Petitioners contend that two more recent decisions have in effect
overruled this longstanding limitation on Congress' §5 authority. They
rely on United States v. Guest for the proposition that the rule laid
down in the Civil Rights Cases is no longer good law. In Guest, the
Court reversed the construction of an indictment under 18 U.S.C. § 241
saying in the course of its opinion that ``we deal here with issues of
statutory construction, not with issues of constitutional power.'' Three
Members of the Court, in a separate opinion by Justice Brennan,
expressed the view that the Civil Rights Cases were wrongly decided, and
that Congress could under §5 prohibit actions by private individuals.
Three other Members of the Court, who joined the opinion of the Court,
joined a separate opinion by Justice Clark which in two or three
sentences stated the conclusion that Congress could ``punis{[}h{]} all
conspiracies-with or without state action-that interfere with Fourteenth
Amendment rights.'' Justice Harlan, in another separate opinion,
commented with respect to the statement by these Justices:

``The action of three of the Justices who joined the Court's opinion in
nonetheless cursorily pronouncing themselves on the far-reaching
constitutional questions deliberately not reached in Part I seems to me,
to say the very least, extraordinary.''

Though these three Justices saw fit to opine on matters not before the
Court in Guest, the Court had no occasion to revisit the Civil Rights
Cases and Harris, having determined ``the indictment {[}charging private
individuals with conspiring to deprive blacks of equal access to state
facilities{]} in fact contain{[}ed{]} an express allegation of state
involvement.'' The Court concluded that the implicit allegation of
``active connivance by agents of the State'' eliminated any need to
decide ``the threshold level that state action must attain in order to
create rights under the Equal Protection Clause.'' All of this Justice
Clark explicitly acknowledged.

To accept petitioners' argument, moreover, one must add to the three
Justices joining Justice Brennan's reasoned explanation for his belief
that the Civil Rights Cases were wrongly decided, the three Justices
joining Justice Clark's opinion who gave no explanation whatever for
their similar view. This is simply not the way that reasoned
constitutional adjudication proceeds. We accordingly have no hesitation
in saying that it would take more than the naked dicta contained in
Justice Clark's opinion, when added to Justice Brennan's opinion, to
cast any doubt upon the enduring vitality of the Civil Rights Cases and
Harris.

Petitioners also rely on District of Columbia v. Carter. Carter was a
case addressing the question whether the District of Columbia was a
``State'' within the meaning of Rev.~Stat. §1979, 42 U.S.C. § 1983-a
section which by its terms requires state action before it may be
employed. A footnote in that opinion recites the same litany respecting
Guest that petitioners rely on. This litany is of course entirely dicta,
and in any event cannot rise above its source. We believe that the
description of the §5 power contained in the Civil Rights Cases is
correct:

``But where a subject has not submitted to the general legislative power
of Congress, but is only submitted thereto for the purpose of rendering
effective some prohibition against particular {[}s{]}tate legislation or
{[}s{]}tate action in reference to that subject, the power given is
limited by its object, any legislation by Congress in the matter must
necessarily be corrective in its character, adapted to counteract and
redress the operation of such prohibited state laws or proceedings of
{[}s{]}tate officers.''

Petitioners alternatively argue that, unlike the situation in the Civil
Rights Cases, here there has been gender-based disparate treatment by
state authorities, whereas in those cases there was no indication of
such state action. There is abundant evidence, however, to show that the
Congresses that enacted the Civil Rights Acts of 1871 and 1875 had a
purpose similar to that of Congress in enacting §13981: There were state
laws on the books bespeaking equality of treatment, but in the
administration of these laws there was discrimination against newly
freed slaves. The statement of Representative Garfield in the House and
that of Senator Sumner in the Senate are representative:

``{[}T{]}he chief complaint is not that the laws of the State are
unequal, but that even where the laws are just and equal on their face,
yet, by a systematic maladministration of them, or a neglect or refusal
to enforce their provisions, a portion of the people are denied equal
protection under them.'' (statement of Rep.~Garfield).

``The Legislature of South Carolina has passed a law giving precisely
the rights contained in your'supplementary civil rights bill.' But such
a law remains a dead letter on her statute-books, because the State
courts, comprised largely of those whom the Senator wishes to obtain
amnesty for, refuse to enforce it.'' (statement of Sen.~Sumner).

But even if that distinction were valid, we do not believe it would save
§13981's civil remedy. For the remedy is simply not ``corrective in its
character, adapted to counteract and redress the operation of such
prohibited {[}s{]}tate laws or proceedings of {[}s{]}tate officers.''
Civil Rights Cases. Or, as we have phrased it in more recent cases,
prophylactic legislation under §5 must have a ``'congruence and
proportionality between the injury to be prevented or remedied and the
means adopted to that end.'' Florida Prepaid Postsecondary Ed. Expense
Bd. v. College Savings Bank; Flores. Section 13981 is not aimed at
proscribing discrimination by officials which the Fourteenth Amendment
might not itself proscribe; it is directed not at any State or state
actor, but at individuals who have committed criminal acts motivated by
gender bias.

In the present cases, for example, §13981 visits no consequence whatever
on any Virginia public official involved in investigating or prosecuting
Brzonkala's assault. The section is, therefore, unlike any of the §5
remedies that we have previously upheld. For example, in Katzenbach v.
Morgan, Congress prohibited New York from imposing literacy tests as a
prerequisite for voting because it found that such a requirement
disenfranchised thousands of Puerto Rican immigrants who had been
educated in the Spanish language of their home territory. That law,
which we upheld, was directed at New York officials who administered the
State's election law and prohibited them from using a provision of that
law. In South Carolina v. Katzenbach, Congress imposed voting rights
requirements on States that, Congress found, had a history of
discriminating against blacks in voting. The remedy was also directed at
state officials in those States. Similarly, in Ex parte Virginia,
Congress criminally punished state officials who intentionally
discriminated in jury selection; again, the remedy was directed to the
culpable state official.

Section 13981 is also different from these previously upheld remedies in
that it applies uniformly throughout the Nation. Congress' findings
indicate that the problem of discrimination against the victims of
gender-motivated crimes does not exist in all States, or even most
States. By contrast, the §5 remedy upheld in Katzenbach v. Morganwas
directed only to the State where the evil found by Congress existed, and
in South Carolina v. Katzenbach the remedy was directed only to those
States in which Congress found that there had been discrimination.

\hypertarget{note-on-other-sources-of-congressional-authority-to-prohibit-discrimination}{%
\subsubsection{Note on other sources of Congressional authority to
prohibit
discrimination}\label{note-on-other-sources-of-congressional-authority-to-prohibit-discrimination}}

Jones v. Alfred H. Mayer Co.~held that Congress has authority to
prohibit the ``badges and incidents of slavery'' under the 13th
Amendment, including the power to regulate private discrimination.
Congress also has, of course, broad Commerce Clause powers to regulate
economic activity, and a number of civil rights laws have been upheld on
that basis. The Voting Rights Act has been upheld and then partially
struck down under Congress's enforcement power under the 15th Amendment.
See Shelby County v. Holder

\end{document}
